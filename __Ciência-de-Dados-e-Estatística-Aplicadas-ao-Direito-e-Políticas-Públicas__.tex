% Options for packages loaded elsewhere
% Options for packages loaded elsewhere
\PassOptionsToPackage{unicode}{hyperref}
\PassOptionsToPackage{hyphens}{url}
\PassOptionsToPackage{dvipsnames,svgnames,x11names}{xcolor}
%
\documentclass[
  12pt,
  a4paper,
  DIV=11,
  numbers=noendperiod]{scrreprt}
\usepackage{xcolor}
\usepackage[top=30mm,left=30mm,bottom=20mm,right=20mm,heightrounded]{geometry}
\usepackage{amsmath,amssymb}
\setcounter{secnumdepth}{5}
\usepackage{iftex}
\ifPDFTeX
  \usepackage[T1]{fontenc}
  \usepackage[utf8]{inputenc}
  \usepackage{textcomp} % provide euro and other symbols
\else % if luatex or xetex
  \usepackage{unicode-math} % this also loads fontspec
  \defaultfontfeatures{Scale=MatchLowercase}
  \defaultfontfeatures[\rmfamily]{Ligatures=TeX,Scale=1}
\fi
\usepackage{lmodern}
\ifPDFTeX\else
  % xetex/luatex font selection
\fi
% Use upquote if available, for straight quotes in verbatim environments
\IfFileExists{upquote.sty}{\usepackage{upquote}}{}
\IfFileExists{microtype.sty}{% use microtype if available
  \usepackage[]{microtype}
  \UseMicrotypeSet[protrusion]{basicmath} % disable protrusion for tt fonts
}{}
\makeatletter
\@ifundefined{KOMAClassName}{% if non-KOMA class
  \IfFileExists{parskip.sty}{%
    \usepackage{parskip}
  }{% else
    \setlength{\parindent}{0pt}
    \setlength{\parskip}{6pt plus 2pt minus 1pt}}
}{% if KOMA class
  \KOMAoptions{parskip=half}}
\makeatother
% Make \paragraph and \subparagraph free-standing
\makeatletter
\ifx\paragraph\undefined\else
  \let\oldparagraph\paragraph
  \renewcommand{\paragraph}{
    \@ifstar
      \xxxParagraphStar
      \xxxParagraphNoStar
  }
  \newcommand{\xxxParagraphStar}[1]{\oldparagraph*{#1}\mbox{}}
  \newcommand{\xxxParagraphNoStar}[1]{\oldparagraph{#1}\mbox{}}
\fi
\ifx\subparagraph\undefined\else
  \let\oldsubparagraph\subparagraph
  \renewcommand{\subparagraph}{
    \@ifstar
      \xxxSubParagraphStar
      \xxxSubParagraphNoStar
  }
  \newcommand{\xxxSubParagraphStar}[1]{\oldsubparagraph*{#1}\mbox{}}
  \newcommand{\xxxSubParagraphNoStar}[1]{\oldsubparagraph{#1}\mbox{}}
\fi
\makeatother

\usepackage{color}
\usepackage{fancyvrb}
\newcommand{\VerbBar}{|}
\newcommand{\VERB}{\Verb[commandchars=\\\{\}]}
\DefineVerbatimEnvironment{Highlighting}{Verbatim}{commandchars=\\\{\}}
% Add ',fontsize=\small' for more characters per line
\usepackage{framed}
\definecolor{shadecolor}{RGB}{241,243,245}
\newenvironment{Shaded}{\begin{snugshade}}{\end{snugshade}}
\newcommand{\AlertTok}[1]{\textcolor[rgb]{0.68,0.00,0.00}{#1}}
\newcommand{\AnnotationTok}[1]{\textcolor[rgb]{0.37,0.37,0.37}{#1}}
\newcommand{\AttributeTok}[1]{\textcolor[rgb]{0.40,0.45,0.13}{#1}}
\newcommand{\BaseNTok}[1]{\textcolor[rgb]{0.68,0.00,0.00}{#1}}
\newcommand{\BuiltInTok}[1]{\textcolor[rgb]{0.00,0.23,0.31}{#1}}
\newcommand{\CharTok}[1]{\textcolor[rgb]{0.13,0.47,0.30}{#1}}
\newcommand{\CommentTok}[1]{\textcolor[rgb]{0.37,0.37,0.37}{#1}}
\newcommand{\CommentVarTok}[1]{\textcolor[rgb]{0.37,0.37,0.37}{\textit{#1}}}
\newcommand{\ConstantTok}[1]{\textcolor[rgb]{0.56,0.35,0.01}{#1}}
\newcommand{\ControlFlowTok}[1]{\textcolor[rgb]{0.00,0.23,0.31}{\textbf{#1}}}
\newcommand{\DataTypeTok}[1]{\textcolor[rgb]{0.68,0.00,0.00}{#1}}
\newcommand{\DecValTok}[1]{\textcolor[rgb]{0.68,0.00,0.00}{#1}}
\newcommand{\DocumentationTok}[1]{\textcolor[rgb]{0.37,0.37,0.37}{\textit{#1}}}
\newcommand{\ErrorTok}[1]{\textcolor[rgb]{0.68,0.00,0.00}{#1}}
\newcommand{\ExtensionTok}[1]{\textcolor[rgb]{0.00,0.23,0.31}{#1}}
\newcommand{\FloatTok}[1]{\textcolor[rgb]{0.68,0.00,0.00}{#1}}
\newcommand{\FunctionTok}[1]{\textcolor[rgb]{0.28,0.35,0.67}{#1}}
\newcommand{\ImportTok}[1]{\textcolor[rgb]{0.00,0.46,0.62}{#1}}
\newcommand{\InformationTok}[1]{\textcolor[rgb]{0.37,0.37,0.37}{#1}}
\newcommand{\KeywordTok}[1]{\textcolor[rgb]{0.00,0.23,0.31}{\textbf{#1}}}
\newcommand{\NormalTok}[1]{\textcolor[rgb]{0.00,0.23,0.31}{#1}}
\newcommand{\OperatorTok}[1]{\textcolor[rgb]{0.37,0.37,0.37}{#1}}
\newcommand{\OtherTok}[1]{\textcolor[rgb]{0.00,0.23,0.31}{#1}}
\newcommand{\PreprocessorTok}[1]{\textcolor[rgb]{0.68,0.00,0.00}{#1}}
\newcommand{\RegionMarkerTok}[1]{\textcolor[rgb]{0.00,0.23,0.31}{#1}}
\newcommand{\SpecialCharTok}[1]{\textcolor[rgb]{0.37,0.37,0.37}{#1}}
\newcommand{\SpecialStringTok}[1]{\textcolor[rgb]{0.13,0.47,0.30}{#1}}
\newcommand{\StringTok}[1]{\textcolor[rgb]{0.13,0.47,0.30}{#1}}
\newcommand{\VariableTok}[1]{\textcolor[rgb]{0.07,0.07,0.07}{#1}}
\newcommand{\VerbatimStringTok}[1]{\textcolor[rgb]{0.13,0.47,0.30}{#1}}
\newcommand{\WarningTok}[1]{\textcolor[rgb]{0.37,0.37,0.37}{\textit{#1}}}

\usepackage{longtable,booktabs,array}
\usepackage{calc} % for calculating minipage widths
% Correct order of tables after \paragraph or \subparagraph
\usepackage{etoolbox}
\makeatletter
\patchcmd\longtable{\par}{\if@noskipsec\mbox{}\fi\par}{}{}
\makeatother
% Allow footnotes in longtable head/foot
\IfFileExists{footnotehyper.sty}{\usepackage{footnotehyper}}{\usepackage{footnote}}
\makesavenoteenv{longtable}
\usepackage{graphicx}
\makeatletter
\newsavebox\pandoc@box
\newcommand*\pandocbounded[1]{% scales image to fit in text height/width
  \sbox\pandoc@box{#1}%
  \Gscale@div\@tempa{\textheight}{\dimexpr\ht\pandoc@box+\dp\pandoc@box\relax}%
  \Gscale@div\@tempb{\linewidth}{\wd\pandoc@box}%
  \ifdim\@tempb\p@<\@tempa\p@\let\@tempa\@tempb\fi% select the smaller of both
  \ifdim\@tempa\p@<\p@\scalebox{\@tempa}{\usebox\pandoc@box}%
  \else\usebox{\pandoc@box}%
  \fi%
}
% Set default figure placement to htbp
\def\fps@figure{htbp}
\makeatother

\ifLuaTeX
  \usepackage{luacolor}
  \usepackage[soul]{lua-ul}
\else
  \usepackage{soul}
\fi

% definitions for citeproc citations
\NewDocumentCommand\citeproctext{}{}
\NewDocumentCommand\citeproc{mm}{%
  \begingroup\def\citeproctext{#2}\cite{#1}\endgroup}
\makeatletter
 % allow citations to break across lines
 \let\@cite@ofmt\@firstofone
 % avoid brackets around text for \cite:
 \def\@biblabel#1{}
 \def\@cite#1#2{{#1\if@tempswa , #2\fi}}
\makeatother
\newlength{\cslhangindent}
\setlength{\cslhangindent}{1.5em}
\newlength{\csllabelwidth}
\setlength{\csllabelwidth}{3em}
\newenvironment{CSLReferences}[2] % #1 hanging-indent, #2 entry-spacing
 {\begin{list}{}{%
  \setlength{\itemindent}{0pt}
  \setlength{\leftmargin}{0pt}
  \setlength{\parsep}{0pt}
  % turn on hanging indent if param 1 is 1
  \ifodd #1
   \setlength{\leftmargin}{\cslhangindent}
   \setlength{\itemindent}{-1\cslhangindent}
  \fi
  % set entry spacing
  \setlength{\itemsep}{#2\baselineskip}}}
 {\end{list}}
\usepackage{calc}
\newcommand{\CSLBlock}[1]{\hfill\break\parbox[t]{\linewidth}{\strut\ignorespaces#1\strut}}
\newcommand{\CSLLeftMargin}[1]{\parbox[t]{\csllabelwidth}{\strut#1\strut}}
\newcommand{\CSLRightInline}[1]{\parbox[t]{\linewidth - \csllabelwidth}{\strut#1\strut}}
\newcommand{\CSLIndent}[1]{\hspace{\cslhangindent}#1}



\setlength{\emergencystretch}{3em} % prevent overfull lines

\providecommand{\tightlist}{%
  \setlength{\itemsep}{0pt}\setlength{\parskip}{0pt}}



 


\usepackage{booktabs}
\usepackage{caption}
\usepackage{longtable}
\usepackage{colortbl}
\usepackage{array}
\usepackage{anyfontsize}
\usepackage{multirow}
\usepackage{wrapfig}
\usepackage{float}
\usepackage{pdflscape}
\usepackage{tabu}
\usepackage{threeparttable}
\usepackage{threeparttablex}
\usepackage[normalem]{ulem}
\usepackage{makecell}
\usepackage{xcolor}
\usepackage{fontspec}
\usepackage{polyglossia}
\setmainlanguage{portuguese}
\setmainfont{Liberation Serif}
\usepackage{grffile}
\usepackage{fvextra}
\DefineVerbatimEnvironment{Highlighting}{Verbatim}{breaklines,commandchars=\\\{\}}
\DefineVerbatimEnvironment{OutputCode}{Verbatim}{breaklines,commandchars=\\\{\}}
\usepackage{fancyhdr}
\pagestyle{fancy}
\fancyhf{}
\fancyhead[LE]{\nouppercase{\rightmark\hfill\leftmark}}
\fancyhead[RO]{\nouppercase{\leftmark\hfill\rightmark}}
\fancyfoot[LE,RO]{\hfill\thepage\hfill}
\KOMAoption{captions}{tableheading}
\titlehead{\includegraphics[width=6.25in]{cover.pdf}}
\makeatletter
\@ifpackageloaded{tcolorbox}{}{\usepackage[skins,breakable]{tcolorbox}}
\@ifpackageloaded{fontawesome5}{}{\usepackage{fontawesome5}}
\definecolor{quarto-callout-color}{HTML}{909090}
\definecolor{quarto-callout-note-color}{HTML}{0758E5}
\definecolor{quarto-callout-important-color}{HTML}{CC1914}
\definecolor{quarto-callout-warning-color}{HTML}{EB9113}
\definecolor{quarto-callout-tip-color}{HTML}{00A047}
\definecolor{quarto-callout-caution-color}{HTML}{FC5300}
\definecolor{quarto-callout-color-frame}{HTML}{acacac}
\definecolor{quarto-callout-note-color-frame}{HTML}{4582ec}
\definecolor{quarto-callout-important-color-frame}{HTML}{d9534f}
\definecolor{quarto-callout-warning-color-frame}{HTML}{f0ad4e}
\definecolor{quarto-callout-tip-color-frame}{HTML}{02b875}
\definecolor{quarto-callout-caution-color-frame}{HTML}{fd7e14}
\makeatother
\makeatletter
\@ifpackageloaded{bookmark}{}{\usepackage{bookmark}}
\makeatother
\makeatletter
\@ifpackageloaded{caption}{}{\usepackage{caption}}
\AtBeginDocument{%
\ifdefined\contentsname
  \renewcommand*\contentsname{Table of contents}
\else
  \newcommand\contentsname{Table of contents}
\fi
\ifdefined\listfigurename
  \renewcommand*\listfigurename{Lista de Figuras}
\else
  \newcommand\listfigurename{Lista de Figuras}
\fi
\ifdefined\listtablename
  \renewcommand*\listtablename{Lista de Tabelas}
\else
  \newcommand\listtablename{Lista de Tabelas}
\fi
\ifdefined\figurename
  \renewcommand*\figurename{Fig.}
\else
  \newcommand\figurename{Fig.}
\fi
\ifdefined\tablename
  \renewcommand*\tablename{Tab.}
\else
  \newcommand\tablename{Tab.}
\fi
}
\@ifpackageloaded{float}{}{\usepackage{float}}
\floatstyle{ruled}
\@ifundefined{c@chapter}{\newfloat{codelisting}{h}{lop}}{\newfloat{codelisting}{h}{lop}[chapter]}
\floatname{codelisting}{Listing}
\newcommand*\listoflistings{\listof{codelisting}{Lista de Definições}}
\captionsetup{labelsep=none}
\makeatother
\makeatletter
\makeatother
\makeatletter
\@ifpackageloaded{caption}{}{\usepackage{caption}}
\@ifpackageloaded{subcaption}{}{\usepackage{subcaption}}
\makeatother
\makeatletter
\@ifpackageloaded{tcolorbox}{}{\usepackage[skins,breakable]{tcolorbox}}
\makeatother
\makeatletter
\@ifundefined{shadecolor}{\definecolor{shadecolor}{rgb}{.97, .97, .97}}{}
\makeatother
\makeatletter
\makeatother
\makeatletter
\ifdefined\Shaded\renewenvironment{Shaded}{\begin{tcolorbox}[frame hidden, enhanced, interior hidden, sharp corners, breakable, boxrule=0pt]}{\end{tcolorbox}}\fi
\makeatother
\usepackage{bookmark}
\IfFileExists{xurl.sty}{\usepackage{xurl}}{} % add URL line breaks if available
\urlstyle{same}
\hypersetup{
  pdftitle={Ciência de Dados e Estatística Aplicadas ao Direito e Políticas Públicas},
  pdfauthor={Prof.~Dr.~Cleuler Barbosa das Neves},
  colorlinks=true,
  linkcolor={blue},
  filecolor={Maroon},
  citecolor={Blue},
  urlcolor={Blue},
  pdfcreator={LaTeX via pandoc}}


\title{\textbf{Ciência de Dados e Estatística Aplicadas ao Direito e
Políticas Públicas}}
\usepackage{etoolbox}
\makeatletter
\providecommand{\subtitle}[1]{% add subtitle to \maketitle
  \apptocmd{\@title}{\par {\large #1 \par}}{}{}
}
\makeatother
\subtitle{\href{https://www.youtube.com/watch?v=4VASegKKzpY}{\includegraphics[width=5.7in,height=\textheight,keepaspectratio]{logoUfg.png}}\\
\textbf{Ciência de Dados e Estatística com
\href{https://www.r-project.org/}{\includegraphics[width=0.35in,height=\textheight,keepaspectratio]{R-logo.png}}\\
Aplicada ao
Di\href{https://posit.co/download/rstudio-desktop/}{\includegraphics[width=0.27in,height=\textheight,keepaspectratio]{Rstudio.jpg}}eito
e Políticas Públicas.\\
74: 3 Axiomas, 16 Teoremas e 55 Conceitos essenciais\\
para 1 Estatística prática e CD aplicada ao Direito e Políticas
Públicas}\\
\strut \\
\textbf{RELATÓRIO DE PRODUTO TÉCNICO}\\
Desenvolvimento de softwares em Linguagem R na disciplina CDEaDPP, 2.
sem. 2025\\
Ciência de Dados e Estatística com R Aplicada ao Direito e Políticas
Públicas\\
(Disciplina DPP0053 junto ao PPGDP da UFG)\\}
\author{Prof.~Dr.~Cleuler Barbosa das Neves}
\date{Nov 12, 2025}
\begin{document}
\maketitle
\begin{abstract}
Ficha catalográfica

\_\_\_\_\_\_\_\_\_\_\_\_\_\_\_\_\_\_\_\_\_\_\_\_\_\_\_\_\_\_\_\_\_\_\_\_\_\_\_\_\_\_

\begin{verbatim}
Neves, Cleuler Barbosa das
   Relatório de Produto Técnico consistente na escrita e
desenvolvimento de softwares em Linguagem R de apoio à
resolução de problemas na disciplina concebida para o
doutorado do PPGDP - Programa de Pós-Graduação em Direito
e Políticas Públicas que funciona junto à Faculdade de
Direito da UFG - Universidade Federal de Goiás, ministrada
no 2. sem. 2025 e intitulada Ciência de Dados e Estatística
Aplicadas ao Direito e Políticas Públicas - DPP0053
(ago. 2025/dez. 2025). Goiânia: FD/UFG, 2025. Geração de um
e-book com esse conjunto de scripts em R hospedado em pasta
do PPGDP a fim de apoiar os estudantes de mestrado/doutorado,
pós-graduação estrito senso, na compreensão do conteúdo
transmitido e também no auxílio à resolução de exercícios
em sala de aula e também dos problemas de pesquisa desses
mestrandos e doutorandos. Geração de simulações computacionais
de problemas de pesquisa em Direito e Políticas Públicas.
Modelo para aqueles que optarem por apresentar seus relatórios
de pesquisa em formato de e-book.
200 p. : il., col., gráf., tab.
Inclui bibliografia
ISBN 978-85-328-0207-7
1. Direito. 2. Políticas Públicas. 3. Estatística Descritiva
e Inferencial. 4. Linguagem R 5. Ciência de Dados
6. Simulações
I. Título
CDD: 519.207.                                  CDU: 519.2:7
\end{verbatim}

\_\_\_\_\_\_\_\_\_\_\_\_\_\_\_\_\_\_\_\_\_\_\_\_\_\_\_\_\_\_\_\_\_\_\_\_\_\_\_\_\_\_
\end{abstract}

\renewcommand*\contentsname{Sumário}
{
\hypersetup{linkcolor=}
\setcounter{tocdepth}{2}
\tableofcontents
}
\listoffigures
\listoftables

\bookmarksetup{startatroot}

\chapter*{Pretexto}\label{sec-pretexto}
\addcontentsline{toc}{chapter}{Pretexto}

\markboth{Pretexto}{Pretexto}

Estado Moderno e o problema do poder de controle público, sob a pele
privada, de seus cidadãos: competências criminal e cível discricionárias
em ponderação com o princípio do \emph{devido processo legal procedural
e substantivo}.

Num tal Moderno Estado Democrático de Direito tornam-se importantes as
Análises de Processo e de Resultados do (des)controle do processo penal
acusatório e da administração da execução penal.

Bem como do processo cível, notadamente das ações coletivas.

Na interface do Direito e Políticas Públicas busca-se compreender o
fenômeno da \textbf{\emph{baixa efetividade}} de inúmeras normas
constitucionais, notadamente aquelas insertas na CF/1988 na condição de
direitos fundamentais.

Esta disciplina, CDE-a-DPP: \emph{Ciência de Dados e} \emph{Estatística
(Básica e Inferecnial, com R) aplicada ao Direito e Políticas Públicas},
foi concebida absolutamente apoiada no \emph{dogma} de que os fenômenos
jurídicos e, mais ainda, sua conjunção com as Políticas Públicas
(pródiga no uso de \emph{indicadores} e de rakeamentos em suas
abordagens para análise de processos e de resultados; ex.: AIR - Análise
de Impacto Regulatório), \textbf{\emph{apresentam uma componente
estocástica}} {[}BECKER (2015){]}(DIETZ; KALOF, Linda, 2015) e podem ser
\textbf{\emph{objeto de contagem ou de medição}}, dado que a
\textbf{\emph{mente humana é um instrumento de medição}} (KAHNEMAN,
Daniel; SIBONY, Oliver; SUNSTEIN, Cass R., 2021) \emph{quando opera
julgamentos sob incerteza} (KAHNEMAN, Daniel; SLOVIC, Paul; TVERSKY,
Amos, 1982).

\section*{Load packages}\label{sec-load-packages}
\addcontentsline{toc}{section}{Load packages}

\markright{Load packages}

\begin{Shaded}
\begin{Highlighting}[numbers=left,,]
\InformationTok{\textasciigrave{}\textasciigrave{}\textasciigrave{}\{r\}}
\CommentTok{\#| label: init}
\CommentTok{\#| warning: false}

\CommentTok{\# load packages}
\CommentTok{\# carregar os pacotes}
\CommentTok{\# necessários para rodar os scripts em R}
\FunctionTok{library}\NormalTok{(tidyverse)}
\FunctionTok{library}\NormalTok{(downlit)}
\FunctionTok{library}\NormalTok{(reticulate)}
\FunctionTok{library}\NormalTok{(quarto)}
\FunctionTok{library}\NormalTok{(qreport)}
\CommentTok{\# library(kableExtra)}
\FunctionTok{library}\NormalTok{(gt)}
\FunctionTok{library}\NormalTok{(latex2exp)}
\FunctionTok{library}\NormalTok{(ggplot2)}
\FunctionTok{library}\NormalTok{(dplyr)}
\FunctionTok{library}\NormalTok{(tibble)}
\FunctionTok{library}\NormalTok{(lubridate)}
\FunctionTok{library}\NormalTok{(mixtools)}
\FunctionTok{library}\NormalTok{(descr)}
\FunctionTok{library}\NormalTok{(magrittr)}
\InformationTok{\textasciigrave{}\textasciigrave{}\textasciigrave{}}
\end{Highlighting}
\end{Shaded}

\section*{Linguagem R - pacote
Quarto}\label{sec-linguagem-r---pacote-quarto}
\addcontentsline{toc}{section}{Linguagem R - pacote Quarto}

\markright{Linguagem R - pacote Quarto}

\textbf{\emph{Desenvolvimento de softwares em linguagem R elaborados
como apoio para resolução de problemas de pesquisa e para simulações}}
na disciplina de doutorado/mestrado do PPGDP - Programa de
Pós-Graduaçãoem Direito e Políticas Públicas sediado na Faculdade de
Direito da UFG - Universidade Federal de Goiás, ministrada no 2. sem.
2024 e intitulada \textbf{Ciência de Dados e Estatística aplicada ao
Direito e Políticas Públicas -- CDEaDPP} - código DPP0053 (ago.
2024/dez. 2024). Goiânia: FD/UFG, 2024.

Pretende-se gerar um \emph{e-book} com os diversos \emph{scripts} desses
\textbf{\emph{softwares em linguagem R}} a fim apoiar os estudantes na
compreensão do conteúdo transmitido e também no auxílio à resolução de
problemas de pesquisa e de geração de simulações, de modo a aplicar as
\textbf{\emph{técnicas}} \textbf{de \emph{pesquisa empírica
predominantemente quantitativas}} que podem ser consideradas
\textbf{adequadas} e que ainda \ul{\textbf{\emph{não}}} foram
\textbf{\emph{aplicadas}} ou foram \textbf{\emph{incompletamente
aplicadas}} em \emph{pesquisas anteriores do PPGDP}, a fim de
\textbf{\emph{avançar}} na \textbf{\emph{fronteira}} \textbf{do
\emph{conhecimento}} do \textbf{\emph{estado da arte}} e da
\textbf{\emph{pesquisa empírica}} (levantar dados
1\textsuperscript{\ul{os}} e 2\textsuperscript{\ul{os}}) até aqui
\textbf{\emph{empreendida neste programa}}, como, por exemplo:
\textbf{\emph{testes de Hipótese}} em geral e da \textbf{\emph{qualidade
do ajuste de Modelos}} (seleção), modelo de regressão logística, modelos
lineares generalizados (glm), análise de sobrevivência, análise de
fatores de risco, análise de relações (DAG's e inferência bayesiana),
análise estatística da decisão por meio de árvore de decisão,
aprendizado de máquina (supervisionado e não supervisionado), modelos de
reologia e escoamento para análise de fluxos processuais no sistema de
justiça criminal e cível etc.

Também servirá de modelo para aqueles que optarem por apresentar seus
Relatórios de Pesquisa ou mesmo dissertações, relatórios de produção
técnica, tese etc. em formato de \emph{e-book}, que pode ser hospedado
em um domínio da web.

Produzido em \texttt{Quarto}.\\
Para saber mais sobre \texttt{Quarto} cf.: \url{https://quarto.org}.

Esta é uma obra científica produzida no \texttt{R\ Quarto}, numa
proposta de \emph{pesquisa empírica com produção técnica relevante}
junto ao PPGDP/UFG, Progama de Pós-Graduação Profissional assim
caracterizado:

\begin{itemize}
\item
  \phantomsection\label{AreaConc}
  Área de concentração: \emph{Direito da Administração e Políticas
  Públicas};
\item
  \phantomsection\label{LinhasPesq}
  Linhas de pesquisa:

  \begin{enumerate}
  \def\labelenumi{\Roman{enumi}.}
  \tightlist
  \item
    \emph{Regulação}, \emph{Efetividade} e \emph{Controle}
    \emph{Constitucional} das Políticas Públicas;
  \item
    Políticas Públicas \emph{de Segurança} e \emph{de Enfrentamento à
    Desigualdade Estrutural}, e
  \item
    \emph{Novas Tecnologias} e \emph{Novas Práticas} em \emph{Políticas
    Públicas}: Soluções Jurídicas
  \end{enumerate}
\item
  Conduzida por: Cleuler Barbosa das NEVES. (\hyperref[0]{Lattes})
\item
  e: Gaspar Alexandre Machado de SOUSA.
  (\href{http://lattes.cnpq.br/6135605692550160}{Lattes})
\end{itemize}

\emph{Recommended Citation}:

NEVES, Cleuler Barbosa das. \textbf{Ciência de Dados e Estatística
aplicadas ao Direito e Políticas Públicas -- CDEaDPP}: desenvolvimento
de softwares em linguagem R elaborados como apoio para resolução de
problemas de pesquisa e para simulações de modo a aplicar as
\textbf{\emph{técnicas}} \textbf{de \emph{pesquisa empírica
predominantemente quantitativas}} que podem ser consideradas
\textbf{adequadas} e que ainda \ul{\textbf{\emph{não}}} foram
\textbf{\emph{aplicadas}} ou foram \textbf{\emph{incompletamente
aplicadas}} em \emph{pesquisas anteriores do PPGDP}, a fim de
\textbf{\emph{avançar}} na \textbf{\emph{fronteira}} \textbf{do
\emph{conhecimento}} do \textbf{\emph{estado da arte}} e da
\textbf{\emph{pesquisa empírica}} (levantar dados
1\textsuperscript{\ul{os}} e 2\textsuperscript{\ul{os}}) até aqui
\textbf{\emph{empreendida neste programa}}. Disciplina ministrada pela
primeira vez no 2. sem. 2024 - código DPP0053 (ago. 2024/dez. 2024).
Goiânia: FD/UFG, 2024.

\section*{Sobre o autor}\label{sec-sobre-Autor}
\addcontentsline{toc}{section}{Sobre o autor}

\markright{Sobre o autor}

Cleuler Barbosa DAS NEVES
(\href{http://lattes.cnpq.br/3567330317986829}{Lattes}) tem graduação em
Direito pela Universidade Federal de Goiás (1997), especialização e
mestrado em Direito Agrário pela Universidade Federal de Goiás (2001) e
doutorado em Ciências Ambientais pela Universidade Federal de Goiás
(2006). Atualmente é
\href{https://www.youtube.com/watch?v=K6PwUG283DU}{professor}
\emph{titular} integrante do quadro permanente do Programa de Mestrado
Profissional em Direito e Políticas Públicas (PPGDP) da UFG e também
junto à graduação em Direito da UFG. Procurador do Estado de Goiás
(5/2/1999). Tem experiência na área de Direito, com ênfase em Direito
Público, tendo atuado em Direito Agrário e Direito Ambiental, com foco
de pesquisas em Áreas de Preservação Permanente - APP's, em recursos
hídricos e em Processo Civil e Administrativo. Atualmente atuando em
Direito Administrativo, com foco de pesquisas em: i)
\textsc{Conflituosidade}, \textsc{Consensualidade e Políticas Públicas}:
\textsc{Mediação, Conciliação e Arbitragem e outros mecanismos
consensuais na Administração Pública} (Linha 3 do PPGDP) e em ii) Defesa
Social e Segurança Pública: desafios para a implantação de políticas
públicas de segurança no Brasil (Linha 2 do PPGDP).

Também é graduado em Engenharia Elétrica pela Universidade Federal de
Goiás (1986), com especialização em Engenharia de
\href{https://www.youtube.com/watch?v=i9L6ypmosdE}{Petróleo} pela
Universidade Federal da Bahia em convênio com Centro de Treinamento do
Nordeste da Petrobrás (1990).

Ministra disciplinas relacionadas à Metodologia da Pesquisa Científica e
à Epistemologia na graduação e na pós-graduação estrito senso da
Faculdade de Direito da UFG (PPGDA e PPGDP) há mais de quinze anos.

Tem dedicado-se ao estudo da Teoria da Probabilidade e Estatística
aplicadas à pesquisa empírica, especialmente à
\emph{quali}-\emph{quantitativa}, no campo do Direito em sua interface
com as Políticas Públicas há mais de sete anos. Cursa atualmente a
especialização \emph{lato senso} em: \emph{Data Science} e Estatística
Aplicada, oferecida pela FEN - Faculdade de Enfermagem e pelo IME -
Instituto de Matemática e Estatística da UFG.

Num breviário de sua trajetória acadêmica, enumera suas seguintes
produções (acadêmicas e técnicas) que, de um modo ou de outro,
conduziram-lhe até a concepação das presentes notas de aula:

\begin{enumerate}
\def\labelenumi{\arabic{enumi}.}
\item
  Dissertação (2001) (DAS NEVES, 2001): suas primeiras reflexões sobre a
  \emph{questão do método} (cap. 2, p.~28-79), ainda sob influxos
  preponderantemente racionalista. Depois (2011) publicada como livro
  (DAS NEVES, 2011).
\item
  Tese (2006) (DAS NEVES, 2006): trata dos \emph{fenômenos} da
  \emph{nomogênese}, da \emph{nomoconcreção} e da
  \emph{nomointernalização}/\emph{nomosocialização} (itens 1.3, 1.4 e
  1.5, cap. 1, f.~45-73) abordados à luz da \emph{Teoria Tridimensional}
  de Reale (REALE, 2017), situando-a \emph{entre razões e sentimentos},
  conforme as \emph{Lições sobre Ética}, de Ernst Tugendhat (TUGENDHAT,
  2003), sob a orientação do Prof.~Dr.~Nivado dos Santos. Disponível em:
  \url{https://files.cercomp.ufg.br/weby/up/104/o/Cleuler_Neves.pdf};
\item
  artigo (2011): \emph{Delimitação de Áreas de Preservação Permanente
  determinadas pelo relevo}: aplicação da legislação ambiental em duas
  Microbacias Hidrográficas no Estado de Goiás (BORGES; NEVES; CASTRO,
  2011), dado que o \emph{principal produto} de sua tese de doutoramento
  foi estabeler \emph{critérios para delimitação de APPs pelo Relevo}.
  Disponível em:
  \url{https://rbgeomorfologia.org.br/rbg/article/view/263};
\item
  Livro (2018): \emph{Trabalhos Científicos em Direito}: da elaboração à
  defesa, no campo acadêmico e profissional. Rio de Janeiro: Lumen
  Juris, há mais de sete anos no prelo (DAS NEVES, "ainda no prelo");
\item
  artigo (2018): Dever de consensualidade na atuação administrativa.
  \emph{Revista de Informação Legislativa}: RIL (DAS NEVES; FERREIRA
  FILHO, 2018a), v. 55, n.~218, p.~63-84, abr./jun. 2018. (qualis A2).
  Disponível em:
  \url{https://www12.senado.leg.br/ril/edicoes/55/218/ril_v55_n218_p63.pdf}
\item
  artigo (2018): Escolha do árbitro na terminação de conflitos
  administrativos: limites e possibilidades da atuação de um advogado
  público. \emph{A\&C -- Revista de Direito Administrativo \&
  Constitucional} (DAS NEVES; FERREIRA FILHO, 2018b), Belo Horizonte,
  ano 18, n.~71, p.~167-195, jan./mar. 2018. qualis A2. DOI:
  10.21056/aec.v18i71.587. Disponível em:
  \url{http://www.revistaaec.com/index.php/revistaaec/article/view/587}
\item
  artigo (2019): Previsão orçamentária de custo para alimentação em
  sessões de conciliação do Tribunal de Justiça de Goiás, com fundamento
  em pesquisa empírica. Artigo publicado na \emph{Revista Fórum
  Administrativo}: Direito Público (DAS NEVES; TOMÁS, 2019), ano 19,
  n.~19, p.~18-25, maio 2019 (qualis A4);
\item
  artigo (2019): Controle concomitante de editais de licitação de obras
  como política pública de prevenção à corrupção. \emph{Revista Fórum
  Administrativo}: Direito Público (DAS NEVES; NAVES, 2019). Belo
  Horizonte, v. 19, n.~220, p.~20-32, jun. 2019. Disponível em:
  \href{https://d1wqtxts1xzle7.cloudfront.net/60647461/Controle_concomitante_de_editais_de_licitacao_de_obras_como_politica_publica_de_prevencao_a_corrupcao20190919-80536-1cz6fg6-libre.pdf?1568923862=&response-content-disposition=attachment\%3B+filename\%3DControle_concomitante_de_editais_de_lici.pdf&Expires=1697602962&Signature=NdqS1iqtzLdSPNNB6-0g1qtabglXj8O5D4VBEI3zrMfoKB41LRPTucEP2LWCGualq~D8jmk4fNP4gXcowbsbPk3wYCeHuSCTolKftuJzSGeEAT1so70whKXqz69iV8lIp~91XbwfHAIxvVej1DKkbb4O3a~IdK8j5phOuRFaUZlS9V7r-5wy2U06EZG9v3R-EKzD4ActlA2z7O6SuvL-F1m8eE9BetdoFxfUYPF7que9Jytcc8pWEZJWDXUeiKxWBWXHuGj0VACeUQ8fvuMjFpNfKbkLhpeoy1WeK9UDXztZfVY6L41Jiz1CaR1a3UAdHVWz3GyIaUsHnssLnCPKVA__&Key-Pair-Id=APKAJLOHF5GGSLRBV4ZA}{clique
  aqui}.
\item
  artigo (2020): Avaliação de Políticas Públicas: uma abordagem DPP
  aplicada ao programa de incentivo fiscal ``PRODUZIR'' no Estado de
  Goiás (2000-2017). \emph{Revista do Direito} (Santa Cruz do Sul
  \emph{on line}), v. 3, p.~104-123, 2020. (DAS NEVES; SILVA, 2020).
  DOI: 10.17058/rdunisc.v3i50.14648. Disponível em:
  \href{https://online.unisc.br/seer/index.php/direito/article/download/14648/8938}{clique
  aqui}
\item
  artigo (2020): A Glicobiologia e a Psicologia Comportamental como
  Elementos Exógenos Estimuladores da Conciliação Judicial. Artigo
  publicado na \emph{Revista Fórum Administrativo}: Direito Público (DAS
  NEVES; BEZERRA, 2020), ano 20, n.~229, p.~26-34, mar. 2020 (qualis
  A4);
\item
  artigo (2021): Avaliação de Políticas Públicas: análises de quebras
  estruturais em séries temporais de indicadores para aferir os
  resultados do programa de incentivo fiscal ``Produzir'' no Estado de
  Goiás (2000 - 2017). \emph{Revista de Estudos Empíricos em Direito}
  {[}\emph{Brazilian Journal of Empirical Legal Studies}{]}, v. 8,
  p.~1-51, 2021, \url{https://doi.org/10.19092/reed.v8i.506} (qualis
  B1). (SILVA; DAS NEVES, 2021);
\item
  artigo (2021): \emph{Uma hermenêutica para antinomias de princípios}:
  limites para seu controle constitucional e políticas públicas (DAS
  NEVES; Rocha Lima, 2021), que aborda o fenômeno da nomoconcreção pela
  ponderação de princípios à luz de uma Teoria racional dos Direitos
  Fundamentais (ALEXY, 2008), tentando afastar seus resíduos metafísicos
  por meio de uma Teoria Retórica do Direito, como preconizada por João
  Maurício Adeodato (ADEODATO, 2014) e sob o ponto de vista da Ciência
  do Direito como \emph{Teoria da Decisão} (FERRAZ JÚNIOR, 1980). Qualis
  A2. Disponível em: \url{https://doi.org/10.21056/aec.v21i84.1210};
\item
  artigo (2021): Lei Anticrime e Colaboração Premiada: os limites da
  sanção premial. \emph{Humanidades \& Inovação}: Novas teses jurídicas
  (DAS NEVES; FIRMINO, 2021). v. 8, n.~51, p.~147-160, jul. 2021. Qualis
  B1. Disponível em:
  \url{https://revista.unitins.br/index.php/humanidadeseinovacao/article/download/5115/3204}
\item
  artigo (2022): \emph{Variância da cor da morte violenta no seio da
  República Federativa do Brasil}: um estudo da variabilidade do perfil
  da morte violenta por sexo e por cor nas 27 Unidades Federativas
  (1997-2019) (DAS NEVES; MATOS, 2022a), que faz uma análise das
  \emph{séries históricas} da taxa de homicídios/100.000hab./ano do país
  e de seus 26 Estados mais o DF nos últimos 40 anos por sexo e por cor
  (ainda não publicado, aguardando resposta da revista
  \href{https://online.unisc.br/seer/index.php/barbaroi}{Barbarói} da
  Unisc - B1);
\item
  artigo (2022): \emph{Avaliação do Sistema Único de Segurança Pública
  --- SUSP (PNSPDS 2018-2028)}: um modelo para capturar os níveis, a
  tendência e a variabilidade da taxa de homicídios em cada um dos 26
  Estados e no DF (Ipea/IBGE 1998-2019 e MJSP jan. 2018-abr. 2021) (DAS
  NEVES; MATOS, 2022b), que faz uma análise do SUSP - Sistema Único de
  Segurança Pública e do 1º PNSPDS - Plano Nacional de Segurança Pública
  e Defesa Social com base nas \emph{séries históricas} da taxa de
  homicídios/100.000hab./ano do país e de seus 26 Estados mais o DF nos
  últimos 40 anos (ainda não publicado, aguardando resposta do Dossiê 3º
  milênio - B1);
\item
  artigo (2022): \emph{Abordagem Policial --- PMGO (2016-2018)}: Sexo,
  Idade e Luz do Dia num Baculejo à Cor da Pele. Apresentado no XI
  ENCONTRO DE PESQUISA EMPÍRICA EM DIREITO, GT n.~11 - Aspectos Teóricos
  e Metodológicos e Proposições Normativas baseadas em Evidências
  Empíricas, Curitiba, 22-26 Ago. 2022. Que se vale de \emph{DAGs}
  (\emph{Direct Acyclic Graphs}) \& Redes Bayesianas numa ousada e
  frustrada tentativa de \emph{modelagem causal} no campo das Ciências
  Sociais aplicadas (ainda não publicado, \emph{não submetido}) (DAS
  NEVES, 2022a);
\item
  Livro em co-autoria (2022): \emph{Inteligência Artificial e
  Jurisprudência} - delimitação jurisprudencial nas decisões do TCU do
  conceito aberto de cláusula restritiva ao caráter competitivo em
  editais de licitação (SILVA; DAS NEVES, 2022).
\item
  Livro em co-autoria (2022): \emph{Política pública de fomento às micro
  e pequenas empresas pelo poder das compras públicas no estado de
  Goiás} - controle externo pelo TCE/GO (2006-2019) (BARZELLAY; DAS
  NEVES, 2022).
\item
  artigo (2022): \emph{Para uma Teoria Geral do Processo Penal Byroniana
  -- TGPP-By}. Artigo escrito para coletânea em homenagem ao Professor
  Byron Seabra Guimarães,† aposentado na cadeira de Direito Processual
  Penal da Faculdade de Direito da UFG. Promoção da EJUG, edital
  n.~05/2021, projeto ``Coleção Bico de Pena''. (DAS NEVES, 2022b)
  (aguardando publicação).
\item
  artigo (2023): \emph{E-commerce dos dados pessoais e a LGPD}:
  abordagem de uma lacuna à luz da Teoria do Ordenamento Jurídico de
  Bobbio (DAS NEVES; MATOS, 2022c); que considera a possibilidade da
  \emph{ponderação de princípios aplicada a um caso de integração de
  lacuna} na LGPD como um ensaio de expansão analógica da tensão entre a
  \emph{norma geral exclusiva} e a \emph{norma geral inclusiva}, como
  abordadas na \emph{Teoria do Ordenamento Jurídico} de Norberto Bobbio
  (BOBBIO, 1999). Publicado na \emph{Revista Constituição, Economia e
  Desenvolvimento}: Revista Eletrônica da Academia Brasileira de Direito
  Constitucional (ABDConst). Curitiba, v. 15, n.~28, p.~201-262,
  jan.-jun. 2023. ISSN 2177-8256. Disponível em:
  \href{https://www.abdconstojs.com.br/index.php/revista/article/view/461/311}{\ul{https://www.abdconstojs.com.br/index.php/revista/article/view/461/311}};
\item
  Tese (2023): NEVES, Cleuler Barbosa das. \textbf{Ruído num Espaço de
  Probabilidades Democrático de Direito}: análise do Furto Simples na
  perspectiva Estatística aplicada ao Direito. Tese/cátedra inédita para
  fins de avaliação de progressão para Professor Titular junto à
  Faculdade de Direito da UFG. 1. out. 2023, Goiânia-GO. Disponível em:
  \href{https://cleuler.com/}{cleuler.com}.
\item
  artigo (2024): NEVES, Cleuler Barbosa das. MATOS, Gisele Gomes. A cor
  nas abordagens policiais no estado de Goiás: 2016-2018.
  \emph{Contribuciones a Las Ciencias Sociales}, São José dos Pinhais,
  v. 17, n.~2, p.~01-41, e4146, fev. 2024.ISSN 1988-7833. DOI:
  \href{https://doi.org/10.55905/revconv.17n.2-045}{\ul{https://doi.org/10.55905/revconv.17n.2-045}}
  . Disponível em:
  \href{https://ojs.revistacontribuciones.com/ojs/index.php/clcs/article/view/4146}{\ul{https://ojs.revistacontribuciones.com/ojs/index.php/clcs/article/view/4146}}
\item
  artigo (2024): NEVES, Cleuler Barbosa das; MATOS, Gisele Gomes.
  Avaliação do Sistema Único de Segurança Pública -- SUSP (PNSPDS
  2018-2028): um modelo para capturar os Níveis, a Tendência e a
  Variabilidade da Taxa de Homicídios em cada um dos 26 Estados e no DF
  (Ipea/IBGE 1998-2019 e MJSP jan. 2018-Abr. 2021), \emph{Revista
  Observatorio de la Economia Latino Americana}, Curitiba, v. 22, n.~4,
  p.~01-55, e4204, abr. 2024. ISSN 1696-8352. DOI:
  \url{https://doi.org/10.55905/oelv22n4-110} .
  \href{https://ojs.observatoriolatinoamericano.com/ojs/index.php/olel/article/view/4204}{\ul{https://ojs.observatoriolatinoamericano.com/ojs/index.php/olel/article/view/4204}}
\item
  Artigo (2024): MATOS, Gisele Gomes; NEVES, Cleuler Barbosa das. Da
  formação do mito da democracia racial à era do constitucionalismo
  contemporâneo: a premente necessidade de superação da mera igualdade
  formal da CF/88, \emph{Revista Goyazes / Trbunal de Justiça do Estado
  de Goiás}, Goiânia, v. 2, n.~1, p.~11-32, 2. sem. 2024. ISSN
  2965-8039. DOI: 10.5281/zenodo.13919778. Disponível em:
  \url{https://revistagoyazes.tjgo.jus.br/goyazes/article/view/20/79.}
\item
  Artigo A2 (2025): NEVES, Cleuler Barbosa das; SANTOS, Maysa Teixeira.
  Política pública judiciária: análise do conceito de nudge na atuação
  do poder judiciário pela efetividade incremental das reurb-s de 56
  municípios goianos (2023-2024), \emph{Revista Aracê -- Direitos
  Humanos em Revista}, v. 7, n.~2, 2025, p.~4863-4882. ISSN 2358-2472.
  DOI: \url{10.56238/arev7n2-021}. Disponível em:
  \url{https://periodicos.newsciencepubl.com/arace/article/view/3126.}
  Acesso em: 14 fev. 2025.
\item
  Artigo A2 (2025): NEVES, Cleuler Barbosa das; MATOS, Gisele Gomes.
  Microssistema legal brasileiro da ``proteção'' dos Dados Pessoais: uma
  suposta efetiva garantia da Titularidade Privada dos Próprios Dados
  Pessoais. \emph{Revista Aracê -- Direitos Humanos em Revista}, {[}S.
  l.{]}, v. 7, n.~3, p.~10805-10867, 2025. ISSN: 2358-2472. DOI:
  \url{10.56238/arev7n3-045}. Disponível em:
  \url{https://periodicos.newsciencepubl.com/arace/article/view/3690.}
  Acesso em: 9 mar. 2025. Bolsita de produtividade PPGDP/UFG. Brazilian
  Legal Microsystem for the `Protection' of Personal Data: a Supposed
  Effective Guarantee of Private Ownership of One's Own Personal Data.
\item
  Artigo A2 (2025): NEVES, Cleuler Barbosa das; MATOS, Gisele Gomes.
  Variância da cor da morte violenta no seio da República Federativa do
  Brasil: um estudo da variabilidade do perfil da morte violenta por
  sexo e por cor nas 27 Unidades Federativas (1997-2019). Revista
  Caderno Pedagógico (Lajeado.online), v. 22, n.~5, e14703, p.~1- 56,
  2025. ISSN:1983-0882. DOI: \url{10.56238/arev7n2-021}. Disponível em:
  \url{https://doi.org/10.54033/cadpedv22n5-072.} Acesso em: 12 mar.
  2025. Bolsista de produtividade do Programa de Pós-Graduação em
  Direito e Políticas Públicas da Universidade Federal de Goiás
  (PPGDP/UFG).
\item
  Artigo A2 (2025): NEVES, Cleuler Barbosa das; TOZETTO, Nathália Suzana
  Costa Silva. Diagnóstico do arranjo institucional e orçamentário de
  Goiânia (2011/21): existe uma política pública de meio ambiente?
  \emph{Revista Caderno Pedagógico} (Lajeado.online), v. 22, n.~7,
  e16498, p.~1-35, 2025. ISSN:1983-0882. DOI:
  \url{10.54033/cadpedv22n7-219}. Disponível em:
  \url{https://ojs.studiespublicacoes.com.br/ojs/index.php/cadped/article/view/16498.}
  Acesso em: 03 ago. 2025. {[}sem a observação de bolsista produtividade
  do PPGDP{]}
\item
  Artigo A2: NEVES, Cleuler Barbosa das; TOZETTO, Nathália Suzana Costa
  Silva. Inadequação das astreintes contra o Estado: da coerção genérica
  ao controle estruturado das políticas públicas. \emph{Revista de
  Geopolítica -- ReGeo}, v. 16, n.~4, e710, p.~1-18, 2025.
  ISSN:2177-3246. DOI:
  \href{https://mail.revistageo.com.br/revista/article/view/710}{10.56238/revgeov16n4-065}.
  Disponível em:
  \url{https://mail.revistageo.com.br/revista/article/view/710/524}.
  Acesso em: 24 set. 2025.
\item
  Artigo A2: NEVES, Cleuler Barbosa das; GUIMARÃES; Tarsila Costa.
  Validade da colaboração premiada de advogado integrante de organização
  criminosa no ordenamento jurídico brasileiro. Aceito pela
  \emph{Revista Aracê}, {[}S. l.{]}. ISSN: 2358-2472.
\end{enumerate}

\newpage{}

\section*{RESUMO}\label{sec-resumo}
\addcontentsline{toc}{section}{RESUMO}

\markright{RESUMO}

Disciplina DPP0053 -Ciência de Dados e Estatística aplicada ao Direito e
Políticas Públicas -- CDEaDPP do Programa de Pós-Graduação em Direito e
Políticas Públicas -- PPGDP da Faculdade de Direito da Universidade
Federal de Goiás -- UFG, sob a responsabilidade do Prof.~Dr.~Cleuler
Barbosa das Neves e do Prof.~Dr.~Gaspar Alexandre Machado de Sousa,
ministrada no 2. sem. de 2025, 9ª turma Mestrado e 1ª Doutorado
selecionadas, na Sala de Aula do PPGDP, com carga horária de 64 h-a (16
encontros de 4 h-a) e mais 32 h-a de atendimento para customizar o
\emph{desenho da pesquisa empírica}, lecionada no seguinte horário:
quintas-feiras das 14:00 às 18:00 h, com intervalo: 15:50 às 16:10 h.

\textbf{Palavras-chave}: 1. Direito. 2. Políticas Públicas. 3.
Estatística Básica. 4. Axiomas, Teoremas e Conceitos essenciais. 5.
Pesquisa empírica. 6. Análises de processo e de impacto.

\newpage{}

\section*{\texorpdfstring{\emph{ABSTRACT}}{ABSTRACT}}\label{sec-abstract}
\addcontentsline{toc}{section}{\emph{ABSTRACT}}

\markright{\emph{ABSTRACT}}

\emph{Discipline DPP0053 - Data Science and Statistics applied to Law
and Public Policy - CDEaDPP of the Postgraduate Program in Law and
Public Policy - PPGDP of the Faculty of Law of the Federal University of
Goiás - UFG, under the responsibility of Prof.~Dr.~Cleuler Barbosa das
Neves and Prof.~Dr.~Gaspar Alexandre Machado de Sousa, taught in the 2nd
semester of 2025, 9th Master's and 1st Doctorate class selected, in the
PPGDP Classroom, with a workload of 64 hours (16 meetings of 4 hours)
and another 32 hours of assistance to customize the design of the
empirical research, taught at the following times: Thursdays from 2:00
pm to 6:00 pm, with a break: 3:50 pm to 4:10 pm.}

\emph{\textbf{Keywords}: 1. Law. 2. Public Policies. 3. Basic
Statistical. 4. Essential Axioms, Theorems and Concepts. 5. Empirical
research 6. Process and Impact's analysis.}

\newpage{}

\section*{Epígrafe}\label{sec-Epigrafe}
\addcontentsline{toc}{section}{Epígrafe}

\markright{Epígrafe}

\begin{quote}
\begin{tcolorbox}[enhanced jigsaw, arc=.35mm, opacitybacktitle=0.6, colframe=quarto-callout-tip-color-frame, titlerule=0mm, leftrule=.75mm, left=2mm, colbacktitle=quarto-callout-tip-color!10!white, breakable, toprule=.15mm, bottomtitle=1mm, opacityback=0, coltitle=black, title=\textcolor{quarto-callout-tip-color}{\faLightbulb}\hspace{0.5em}{Tip}, rightrule=.15mm, bottomrule=.15mm, toptitle=1mm, colback=white]

\begin{quote}
\section*{Procura da Pesquisa}\label{procura-da-pesquisa}
\addcontentsline{toc}{section}{Procura da Pesquisa}

\markright{Procura da Pesquisa}

\textbf{Entre Letras e Números: empoderando dados}

Penetra surdamente no reino das \emph{letras e números}.

Lá estão os \(\mu\)\emph{odelos} que esperam ser \emph{representados}.

Estão paralisados, mas não há desespero,

há calma e frescura na superfície intata.

Ei-los sós e mudos, em estado de \emph{símbolos}.

Convive com teus \(\mu\)\emph{odelos}, antes de \emph{descrevê}-los.

Tem paciência se obscuros. Calma, se te provocam.

Espera que cada um se realize e consume

com seu poder \emph{simbólico}

e seu poder de silêncio.

Não forces o \(\mu\)\emph{odelo} a desprender-se do limbo.

Não colhas no chão o \(\mu\)\emph{odelo} que se perdeu.

Não \emph{envieses} o \(\mu\)\emph{odelo}. Aceita-o

como ele aceitará sua forma definitiva e concentrada

no \emph{tempo}-espaço.

Chega mais perto e contempla \emph{os symbols}.

Cada \emph{um}

tem mil faces secretas sob a face neutra

e te pergunta, sem interesse pela resposta,

pobre ou terrível, que lhe deres:

Trouxeste \emph{la chiave}
{[}\(\Theta \varepsilon \omega \rho \iota \alpha\){]}?

(parafraseando
\href{https://www.letras.mus.br/carlos-drummond-de-andrade/460651/}{Carlos
Drummond de Andrade})
\end{quote}

\end{tcolorbox}
\end{quote}

\newpage{}

\section*{Dedicatória}\label{sec-dedicatoria}
\addcontentsline{toc}{section}{Dedicatória}

\markright{Dedicatória}

\begin{tcolorbox}[enhanced jigsaw, arc=.35mm, opacitybacktitle=0.6, colframe=quarto-callout-important-color-frame, titlerule=0mm, leftrule=.75mm, left=2mm, colbacktitle=quarto-callout-important-color!10!white, breakable, toprule=.15mm, bottomtitle=1mm, opacityback=0, coltitle=black, title=\textcolor{quarto-callout-important-color}{\faExclamation}\hspace{0.5em}{Tributos}, rightrule=.15mm, bottomrule=.15mm, toptitle=1mm, colback=white]

À \href{https://www.youtube.com/watch?v=_x9mNq-eB6w}{minha mãe} Abigair
de Souza Neves.

A \hyperref[0]{meu pai}: João Barbosa das
Neves.\href{https://www.goianatv.com/2014/07/ariano-suassuna-1927-2014-cultura-esta.html}{†}

A ambos, por tudo: e pelos sacrifícios que não tive que fazer.

Aos meus irmãos Tuller,
Euler\href{https://www.goianatv.com/2014/07/ariano-suassuna-1927-2014-cultura-esta.html}{†}
e Miler. Meu grupo manos, para o que der e vier.

À \href{https://www.youtube.com/watch?v=UsXa9v-UXgM}{Gisele},
\href{https://www.youtube.com/watch?v=1tO_6H0fVcs}{rima de ventos e
vela} que me leva e me trás sob o faro do amor, em rajadas que me tragam
e que trago. Eu já não me pergunto incansavelmente por que isso se deu:
não sei!
\href{https://twitter.com/em_com/status/492057436183089154/photo/1}{Só
sei que foi assim}.

Aos filhos, Ana Clara, Carlos André, Ana Luísa e Isadora, oro em
\href{https://www.youtube.com/watch?app=desktop&v=Hk3GCxivgZE}{Deus} que
lhes ponha saúde e virtude.

\end{tcolorbox}

\section*{Agradecimentos}\label{agradecimentos}
\addcontentsline{toc}{section}{Agradecimentos}

\markright{Agradecimentos}

À Universidade Federal de Goiás de Goiás, que há mais de 40 anos
abriga-me, entre o Samabaia e a Praça \emph{Universitates}; \emph{locus}
em que apreendi um \emph{corpus} de letras e números e em que
multipliquei \emph{tempus} do que dividi.

Aos Professores que me ensinaram aprender a aprender: a todos eles.

Muito obrigado a todos os que me impulsionaram por nossos caminhos
cruzados.

\bookmarksetup{startatroot}

\chapter{Introdução}\label{sec-introducao}

Segundo (HARARI, 2018 , p.~107-111) , dentro do horizonte de tempo deste
Século XXI, a conexão dos indivíduos à \emph{web} será vital e ``À
medida que, através de sensores biométricos, cada vez mais dados fluírem
de seu corpo e seu cérebro para máquinas inteligentes, será fácil para
corporações e agências do governo \emph{conhecer} você, \emph{manipular}
você e \emph{tomar decisões} por você'' (destacou-se).

Não se pode imaginar risco maior para os modernos Estados Democráticos
de Direito, pois nessa era digital em que já se vive revelou-se o poder
daquelas instituições públicas ou privadas que detêm o poder dos dados
(HARARI, 2020, p.~1) (LOCK et al., 2017).

Na interface do Direito e Políticas Públicas busca-se compreender o
fenômeno da \textbf{\emph{baixa efetividade}} de inúmeras normas
constitucionais, notadamente aquelas insertas na CF/1988 na condição de
direitos fundamentais.

Esta disciplina, CDE-a-DPP: \emph{Ciência de Dados e} \emph{Estatística
Básica} {[}\emph{com R}{]} \emph{Aplicada ao Direito e Políticas
Públicas}, foi concebida absolutamente apoiada no \emph{dogma} de que os
\textbf{\emph{fenômenos jurídicos}} e, mais ainda, sua conjunção com as
Políticas Públicas (prodiga no uso de \emph{indicadores}, de
\emph{índices} e de \emph{rakeamentos} em suas abordagens para análise
de processos e de resultados; ex.: AIR - Análise de Impacto
Regulatório), \textbf{\emph{apresentam uma componente estocástica}}
{[}BECKER (2015){]}(DIETZ; KALOF, Linda, 2015) e podem ser
\textbf{\emph{objeto de medição}}, dado que a \textbf{\emph{mente humana
é um instrumento de medição}} (KAHNEMAN, Daniel; SIBONY, Oliver;
SUNSTEIN, Cass R., 2021) quando opera julgamentos sob incerteza
(KAHNEMAN, Daniel; SLOVIC, Paul; TVERSKY, Amos, 1982).

Parte-se do \emph{pressuposto} de que Direito é \emph{uma Ciência Social
Aplicada} e, assim como a Demografia, a Biometria, a Psicologia, a
Economia, a Ciência Política, a Epidemiologia, a Ciência Forense etc.,
reclamam cada vez mais uma maior e melhor \emph{compreensão} do seu
\emph{sujeito}-\emph{objeto} por meio de \emph{pesquisas empíricas}.

O que impele seus atuais pesquisadores, quer em programas de
pós-graduação acadêmicos ou profissionais, quer desde a graduação
(iniciação científica, elaboração de TCCs e Núcleo Livre), a
\emph{levantar dados} (1º\textsuperscript{s} e 2º\textsuperscript{s}),
no curso de pesquisas \emph{quali-quantis}, para deles extrair
\emph{informações} e \emph{padrões} perceptíveis, momento em que a
\textbf{\emph{Estatística Descritiva}}, com suas \emph{consagradas
técnicas} para \emph{coletar}, \emph{ordenar}, \emph{resumir} e
\emph{apresentar} dados e informações torna-se imprescindível.

A rigor, há de \emph{4 a 7 momentos} de \emph{síntese indutiva} em que
um pesquisador empírico que colete dados na interface do Direito com as
Políticas Públicas depara-se no curso de sua pesquisa:

\begin{enumerate}
\def\labelenumi{\roman{enumi}.}
\item
  na apropriação e \emph{operacionalização de conceitos} e de
  indicadores necessários e suficientes para abordar sua
  \emph{situção}-\emph{problema} de pesquisa;
\item
  na escolha e coleta dos dados de campo, notadamente quando, ao invés
  de censo, optar-se por colher \textbf{\emph{1 amostra}} (parte) de
  tamanho adequado da população alvo, ou melhor, da população disponível
  (todo) de tamanho finito; pois, não raro, pretende valer-se de algum
  \emph{poder inferencial} para, \emph{a partir de uma parte, extrair
  conclusões para o todo};
\item
  na distribuição aleatorizada dos elementos entre um Grupo de
  Tratamento e um Grupo de Controle (que recebe um \emph{placebo} sob
  \emph{duplo cego}), nos experimentos controlados; para \emph{extração
  inferencial de conclusões causais} a respeito dos efeitos esperados do
  tratamento (T) sobre toda uma \emph{População disponível} pesquisada;
\item
  na construção de \emph{contrafactuais} (\emph{quasi-experimentos}),
  quando, no caso das Ciências Sociais Aplicadas, não se pode dispor,
  por razões éticas, de um Grupo de Controle para se pesquisar sobre
  \emph{efeitos causais} de \emph{um Tratamento social} sobre toda uma
  População disponível pesquisada;
\item
  na tomada de \emph{decisões político-administrativas baseadas em
  evidências} (dados e informações);
\item
  na escolha das ações para implementação e monitoramento (diagnóstico,
  avaliação e correção/controle) de indicadores e de metas assim
  colocados;
\item
  na \emph{prescrição} normativa de \emph{novas intervenções} para
  melhoria na eficiência (orçamentária), na eficácia (jurídica) e na
  \textbf{\emph{efetividade}} (social) dessas implementações metódicas,
  com foco na \emph{melhoria da efetividade dos direitos fundamentais
  prescritos na CF/1988 no decorrer do tempo}.
\end{enumerate}

Nessa ótica, é necessário e preciso \textbf{\emph{superar}} o
\emph{preconceito dos juristas} com as Ciências Exatas, em especial com
a Probabilidade e Estatística. Além do dogma da onipotência dos
juristas, que realizam a defesa de suas hipóteses como se fossem
permeados de uma certeza determinística. É preciso reposicionar a
Doutrina, que não é mais fonte do Direito desde a Modernidade e que
repousa exclusivamente sobre argumento de autoridade e que não desvela
os interesses subalternados de seus autores (ausência de
\emph{disclaimer}).

A \textbf{\emph{Estatística Indutiva}} ou \textbf{\emph{Inferencial}}
--- com sua \emph{Teoria da Amostragem}, desenvolvida a partir doe
descobertas iniciadas no Séc. XVII:

\begin{itemize}
\item
  1654 com as famosas cartas trocadas por Pascal e Fermat, este um
  \emph{jurista e matemático}, aquele um \emph{filósofo e matemático},
  em que se \emph{resolve} o \emph{problema da repartição do cacife de
  um jogo de pontos interrompido};
\item
  1713 publica-se, postumamente, a obra \emph{Ars Conjecturandi}, de
  Jaques Bernoulli, um \emph{jurista e matemático}) --- que possibilita
  o emprego de \emph{desenhos de pesquisa} (\emph{D.o.E} - \emph{Design
  of Experiment}) em que se pode \emph{validamente} operar essas
  \emph{inferências indutivas} por meio do estudo, da descrição e da
  comparação com os \emph{padrões} encontrados nos \emph{fenômenos
  aleatórios} (\emph{alea} e Teoria das Medições, pelo estudo dos
  \emph{viéses} e \emph{ruídos}). É quem enuncia e prova pela primeira
  vez a Lei dos Grandes Números.
\end{itemize}

É, pois, preciso dar-se ao trabalho de \emph{refletir} sobre as
possibilidades dos \emph{desenhos de pesquisa empírica}, explicitando-os
em nossos \emph{projetos} e \emph{protocolos de pesquisa empírica}, a
fim de conferir \textbf{\emph{replicabilidade}},
\textbf{\emph{validade}} e \textbf{\emph{confiabilidade}} aos diversos
\textbf{\emph{instrumentos de medição}} aplicáveis na interface do
Direito e das Políticas Públicas, com uma \emph{DPP-metria} ainda por
ser concebida, construída, testada e validada.

Como \emph{um resultado esperado} a partir dessa abordagem de pesquisa,
aponta-se para \emph{um provável aumento da confiança no moderno Estado
Democrático de Direito}, pela busca --- por meio de \emph{diagnósticos}
e \emph{decisões} \emph{basedos em evidências empíricas} --- da
\emph{redução} de \emph{viéses} e de \emph{ruídos} observáveis na
atuação das suas instituições (KAHNEMAN, Daniel; SIBONY, Oliver;
SUNSTEIN, Cass R., 2021) no trato com seus concidadãos (ex. no Sistema
de Justiça Criminal e Cível brasileiro).

Ou seja, o grande desafio a ser enfrentado é construir uma Metodologia
(\emph{D.o.E.}) nos protocolos de pesquisa empírica do Direito e das
Políticas como um caminho \emph{ad-versus} ao da propalada \emph{defesa
de hipóteses}, tão encontradiço na doutrina jurídica nacional e, não
raro, mesmo em variegados programas de pós-graduação meramente acadêmica
do Direito.

Trata-se da expectativa de qualificar a pesquisa acadêmica brasileira ao
ponto de romper e superar a \emph{dicotomia} entre \emph{pesquisa pura}
ou meramente acadêmica e \emph{pesquisa aplicada} ou profissional no
campo do Direito pela intrusão de uma ``impureza'' no seu susbtrato: o
reconhecimento da importância da \textbf{\emph{modelagem}} da
\emph{alea} em \emph{pesquisas empíricas} com emprego de \emph{modelos
informacionais} (que apresentam uma parcela estocástica: sinal e ruído)
ao tomar a \emph{mente humana no seu processo decisório como um
instrumento de medição que opera sob incertezas}.

A proposta desta disciplina (CDE-a-DPP) é pretenciosa no sentido de dar
um passo inicial junto à pós-graduação estrito senso exatamente nessa
direção reflexiva, conjugando a compreensão e a sedimentação de uma
Teoria Estatística Descritiva/Inferencial aprofundada combinada com
Ciência de Dados voltadas para o emprego de \emph{ferramentas
tecnológicas disponíveis gratuitamente na web} (ex.
\href{https://www.lock5stat.com/StatKey/index.html}{StatKey},
\href{https://www.statdisk.com/}{statdisk} etc.), notadamente pelo
domínio, desenvolvimento e aplicção do básico em
\href{https://www.r-project.org/}{Linguagem R}, por meio da
\href{https://posit.co/download/rstudio-desktop/}{IDE R-Studio}; com o
que se almeja \emph{uma ainda mais desafiadora pretensão} de iniciar
\emph{uma formação interdisciplinar} de \emph{Cientistas de Dados} no
seio de um curso de pós-graduação estrito senso em Direito e Políticas
Públicas.

\section{Objetivo Geral}\label{objetivo-geral}

Propiciar aos pós-graduandos ingressos no PPGDP que já cursaram a
disciplina DPP0009 - SEMINÁRIOS INTEGRADOS DE PESQUISA EM ARTICULAÇÃO À
\emph{PRAXIS} PROFISSIONAL -- SIP-APP, um \emph{letramento} em
\emph{Ciência de Dados}, \emph{Estatística} e na abordagem
\emph{Direito} e \emph{Políticas Públicas} (\emph{Múltiplas Linguagens})
e o acesso a um ferramental \emph{necessário} para a \emph{produção
científica} com emprego de técnicas \emph{predominantemente}
\emph{quantitativas}, notadamente quanto ao conhecimento daquelas
\emph{necessárias} para a problematização (definição e delimitação da
situação/problema com um adequado desenho do experimento e esquema/plano
amostral a ser executado), a seleção de fontes, \ul{\textbf{\emph{uma}}}
\textbf{\emph{coleta}} \textbf{\emph{válida}} e
\textbf{\emph{fidedigna}} de \textbf{\emph{dados}}, uma revisão de
literatura \textbf{\emph{sistemática}}, \emph{análise} e
\emph{interpretação} de \textbf{\emph{dados}} e
\textbf{\emph{informações}}, uma \emph{discussão cientificamente
organizada dos \textbf{resultados}} e a \emph{publicação} de um
\emph{trabalho científico}, com foco na revisão e aprimoramento de um
projeto de pesquisa (com um \emph{plano de trabalho}) através do
\textbf{\emph{delineamento}} de um \textbf{\emph{protocolo de pesquisa}}
(Desenho de Pesquisa com Plano/Esquema Amostral) que possam articular a
teoria com a praxis profissional em investigações jurídicas aplicadas,
em \textbf{pesquisa jurídica empírica} e em projetos de pesquisa-ação na
interface do Direito com as Políticas Públicas.

\textbf{\emph{Objetivo geral é focar}} nas \textbf{\emph{técnicas}}
\textbf{de \emph{pesquisa empírica predominantemente quantitativas}} que
podem ser consideradas \textbf{adequadas} e que ainda
\ul{\textbf{\emph{não}}} foram \textbf{\emph{aplicadas}} ou foram
\textbf{\emph{incompletamente aplicadas}} em \emph{pesquisas anteriores
do PPGDP}, a fim de \textbf{\emph{avançar}} na \textbf{\emph{fronteira}}
\textbf{do \emph{conhecimento}} do \textbf{\emph{estado da arte}} e da
\textbf{\emph{pesquisa empírica}} (levantar dados
1\textsuperscript{\ul{os}} e 2\textsuperscript{\ul{os}}) até aqui
\textbf{\emph{empreendida neste programa}}, como, por exemplo:
\textbf{\emph{testes de Hipótese}} em geral e da \textbf{\emph{qualidade
do ajuste de Modelos}} (seleção), modelo de regressão logística, modelos
lineares generalizados (glm), análise de sobrevivência, análise de
fatores de risco, análise de relações (DAG's e inferência bayesiana),
análise estatística da decisão por meio de árvore de decisão,
aprendizado de máquina (supervisionado e não supervisionado), modelos de
reologia e escoamento para análise de fluxos processuais no sistema de
justiça criminal e cível etc.

\section{Objetivos Específicos}\label{objetivos-especuxedficos}

\textbf{\emph{Capacitar}} pós-graduandos (mestrado e doutorado) e
egressos (laboratórios e observatórios) para o desenvolvimento das
diversas fases de uma pesquisa científica -- aprimorando a reelaboração
de um projeto (NBR 15.287/2011), da sua redação monográfica (NBR
14.724/2011) e de relatórios técnicos (NBR 10.719/2015) sobre produtos
alcançados na pesquisa empírica (pura ou aplicada) até sua
\emph{comunicação} (Relatório preliminar e definitivo de pesquisa),
tendo em vista atender ao art. 42, RPPGDP.\footnote{``Art. 42. O produto
  final da pesquisa do estudante no Programa pode ter a forma de:

  I- Dissertação;

  II- Estudo de Caso;

  III- Projeto Regulatório;

  IV- Desenvolvimento de processos e técnicas.''} Isso, com foco menos
no aprendizado das regras de notação (o que se compreende como uma
habilidade que já se deve dominar) e mais no \emph{aprendizado} dos
\textbf{\emph{conceitos}} da \textbf{\emph{Metodologia Científica}} e de
\emph{técnicas de \textbf{pesquisa}} \textbf{empírica}
\textbf{\emph{predominantemente quantitativas}} mais específicas,
adequadas e \textbf{\emph{inovadoras}} em relação a dados colhidos e
informações extraídas na interface do Direito com as Políticas Públicas.

\textbf{\emph{Articular}} a pesquisa \emph{jurídica} empírica às
atividades e vivências da \emph{prática jurídica}, considerando que
Direito é uma Ciência Social \emph{aplicada}.

\textbf{\emph{Preparar}} o discente para \textbf{\emph{aprimorar}} a
escolha, a problematização, a delimitação e a
\ul{\textbf{\emph{reelaboração}}} de um Projeto de Pesquisa (PP)
jurídico-política, como guia de apoio metodológico, bem como para
compreender o papel do \textbf{método} e a importância do referencial
teórico na pesquisa aplicada e na pesquisa empírica, focos do PPGDP.

\textbf{\emph{Promover}} um \textbf{\emph{diálogo}}, pelo
\textbf{\emph{domínio/letramento}} em diversas
\textbf{\emph{linguagens}} (Ciência, Ciência de Dados e Probabilidade e
Estatística -- com suas Linguagens \emph{Formais}; Direito e Políticas
Públicas -- com suas Linguagens \emph{Naturais diferenciadas}), entre os
pesquisadores do PPGDP \emph{sobre} os \emph{projeto de pesquisa
institucionais} a que são \textbf{\emph{aderentes}}, tomando como
\textbf{\emph{eixo condutor}} uma \emph{ideia} ou
\textbf{\emph{conceito}} de \ul{\textbf{\emph{ciclo de}}}
\textbf{\emph{\ldots{}}}, ou iteratividade (reiteratividade), nas 4
áreas de conhecimento cujos \emph{pontos de vista/de partida}
pretende-se valer esta disciplina \emph{multidisciplinar}: CD \^{}
E::Dir\&PP's.

\textbf{\emph{Fornecer}} orientações iniciais sobre as técnicas de
pesquisa \emph{predominantemente quantitativas} mais utilizadas no
\emph{campo jurídico} e na \emph{abordagem profissional} que o PPGDP
adota e vem aplicando; todavia com \textbf{\emph{foco}} naqueloutras
\textbf{técnicas adequadas} que ainda \textbf{\emph{não}} foram
\textbf{\emph{aplicadas}} ou \emph{incipientemente utilizadas}, a fim de
avançar na
\href{https://docs.google.com/document/d/1xDVEnbLBXm4tfnX2dvRWpL5_pQdGxwkhoJHAieg0glQ/mobilebasic\#heading=h.5n2rkdaavr5x}{\emph{fronteira
do conhecimento}} empírico até aqui adquirido (criação e rearranjo
teóricos e inovação no domínio da técnica).

\textbf{\emph{Estimular}} o estudante/pesquisador na prossecução de uma
nova perspectiva metodológica e acadêmica para permitir o conhecimento e
a prática do Direito mais voltados para \emph{superação} dos problemas
diagnosticados na experiência jurídica brasileira\footnote{A vida do
  Direito como ela é: pisar o chão de fábrica da arena jurídica. Não
  como deveria ser!} com suporte a \emph{decisões} tomadas com
\emph{base} em \emph{evidências ciêntíficas}.

\textbf{\emph{Auxiliar}} o pesquisador para que ele prepare \emph{um
relatório parcial de pesquisa} de seu produto de pesquisa empírica em um
seminário de pesquisa (pré-qualificação).

\bookmarksetup{startatroot}

\chapter{AEID - Análise Exploratória e Inferencial de
Dados}\label{sec-AEID}

\section{\texorpdfstring{AED - Análise \emph{Exploratória} de
Dados}{AED - Análise Exploratória de Dados}}\label{aed---anuxe1lise-exploratuxf3ria-de-dados}

\begin{quote}
A \textbf{análise de dados} se refere aos métodos e estratégias para se
olhar para os dados -- a \textbf{\emph{exploração}},
\textbf{\emph{organização}} e \textbf{\emph{descrição}} de dados com
auxílio de \textbf{\emph{gráficos e resumos numéricos}}. \emph{Sua
exploração conscienciosa permite que os dados iluminem a realidade}. Os
Capítulos 1 ao 6 discutem a análise de dados (MOORE; NOTZ; FLIGNER, 2023
, p.~6).
\end{quote}

\section{Exploração de Dados (cap.
1)}\label{explorauxe7uxe3o-de-dados-cap.-1}

\begin{quote}
\textbf{``O que {[}nos{]} dizem os dados?''} é a primeira pergunta que
fazemos em qualquer estudo estatístico. A \textbf{análise de dados}
responde a essa questão por \textbf{\emph{meio}} de uma
\textbf{\emph{exploração}} ampla dos dados. As
\textbf{\emph{ferramentas}} da análise de dados são
\textbf{\emph{gráficos}}, como os \emph{histogramas} e os
\emph{diagramas de dispersão}, \textbf{\emph{e medidas numéricas}}, como
as \emph{médias} e as \emph{correlações}. No entanto, ao menos tão
importantes quanto as \emph{ferramentas}, são os
\textbf{\emph{princípios}} que \emph{\textbf{organizam nosso pensamento}
no exame dos dados} (MOORE; NOTZ; FLIGNER, 2023 , p.~9).
\end{quote}

Quanto aos \textbf{\emph{princípios organizadores}} de um
\textbf{\emph{letramento}} ou pensamento estatístico, dois destacam-se
na AED - Análise Exploratória de Dados:

\begin{quote}
Um dos princípios organizadores da análise de dados consiste em
{[}\textbf{P\textsubscript{1}}{]} \textbf{\emph{olhar, primeiro, um item
de cada vez e,}} {[}\textbf{P\textsubscript{2}}{]} \textbf{\emph{depois,
as relações entre estes}}. Nossa apresentação segue esse princípio. Nos
\textbf{Capítulos 1 a 3}, você estudará \textbf{\emph{variáveis e suas
distribuições}}. Os \textbf{Capítulos 4 a 6} referem-se a
\textbf{\emph{relações}} entre variáveis. O Capítulo 7 faz uma revisão
dessa parte do texto (MOORE; NOTZ; FLIGNER, 2023 , p.~9).
\end{quote}

Abaixo uma figura que iluistra o \textbf{Ciclo da Ciência de Dados}. Não
se esqueça que a Estatística é ``\ldots a Ciência dos Dados! (MOORE;
NOTZ; FLIGNER, 2023 , p.~162).

\begin{figure}[H]

{\centering \pandocbounded{\includegraphics[keepaspectratio]{fig/2-data-science-wrangle-01.png}}

}

\caption{Ciclo da Ciência de Dados, suas 3 fases: 1 - Wrangle (importar,
organizar e transformar), 2 - Understand (Transformar, Visualizar e
Modelar; buscar melhor ajuste) e 3 - Communicate (relatar) and Replicate
(automatizar; app)}

\end{figure}%

A fase \textbf{1 -} \textbf{\emph{Wrangle}} , quando inclusa um
\emph{levantamento de dados primários}, consome cerca de \textbf{80\%}
do tempo de uma pesquisa empírica (WICKHAM; GROLEMUND, 2017, p. ix, xi,
117). Essa tarefa é uma verdadeira luta, que não costuma nos agradar.

Um similar \textbf{\emph{conceito}} de \textbf{Ciclo da Ciência de
Dados} (cf.~\href{https://livro.curso-r.com/livro-curso-r}{Curso-R}),
agora associando-o aos principais \texttt{pacotes} do \texttt{R} que
auxiliam cada fase ou etapa dentro de cada fase: notadamente o pacote
\texttt{tidyverse}.

\begin{figure}[H]

{\centering \pandocbounded{\includegraphics[keepaspectratio]{fig/Conceito-Pacotes-R-ciclo-ciencia-de-dados.png}}

}

\caption{O tidyverse é um pacote guarda-chuva que consolida uma série de
ferramentas que fazem parte do ciclo da ciência de dados. Fazem parte do
\{tidyverse\} os pacotes \{ggplot2\}, \{dplyr\}, \{tidyr\}, \{purrr\},
\{readr\}, entre muitos outros, como é possível observar na figura.}

\end{figure}%

Conferir uma \textbf{\emph{cheat sheet}} com pacotes em \textbf{R} bem
mais completa em:
\url{https://www.business-science.io/r-cheatsheet.html}.

As habilidades que, assim, espera-se de um \textbf{\emph{cientista de
dados}} são assim resumidas (GROLEMUND, 2014):

\begin{figure}[H]

{\centering \pandocbounded{\includegraphics[keepaspectratio]{fig/hopr_fig-10-04-3Core-SkillSets-OfDataScience_CP-DC-SR-p185-grolemund-2014.png}}

}

\caption{Three Core Skill sets of Data Science (GROLEMUND, 2017,
p.~185)}

\end{figure}%

Já a \textbf{Estatística}, melhor é referir-se à \textbf{Probabilidade e
Estatística}, é conceituada como: ramo da Matemática aplicada que reune
\textbf{\emph{um conjunto de métodos}} para:

\begin{itemize}
\tightlist
\item
  \textbf{planejar} estudos observacionais e experimentos aleatorizados
  em qualquer área do conhecimento científico, notadamente para
  pesquisas empíricas;
\item
  \textbf{coletar} dados \textbf{válidos} e \textbf{fidedignos};
\item
  \textbf{organizar},
\item
  \textbf{resumir},
\item
  \textbf{apresentar} (listas, tabelas, diagramas, fórmulas, gráficos,
  grafos etc),
\item
  \textbf{analisar},
\item
  \textbf{formular} e \textbf{testar} \textbf{\emph{hipóteses}} e
\item
  \textbf{interpretar} conjuntos de \textbf{dados} e
  \textbf{informações};
\item
  \textbf{elaborar} conclusões baseadas em \textbf{evidências} {[}dados
  e informações válidos e fidedignos{]} para
\item
  \textbf{apoiar} tomadas de \textbf{decisão} e para
\item
  \textbf{gerir} ou \textbf{controlar} um conjunto de \textbf{ações} em
  curso: qualidade, escala, cobertura, custos financeiros, eficiência,
  eficácia, efetividade etc. por meio de \textbf{indicadores} e de
  índices cuja \textbf{\emph{aplicabilidade, comparabilidade,
  consistência e difusão}} possam ser \textbf{testadas e validadas} por
  uma comunidade de experts.
\end{itemize}

O Professor João Luiz Becker (BECKER, 2015) promove uma clara
conceituação e distinção entre \textbf{dados} e \textbf{informações} e
ilustra o ciclo em que a coleta e a extração deles inserem-se num
processo mais amplo de obtenção de \textbf{conhecimento}, que prossegue
e passa pela \textbf{decisão} e pela \textbf{ação}.

\begin{center}
\pandocbounded{\includegraphics[keepaspectratio]{fig/Dados-Informacao-Decisao-Acao-Becker-fig11-p37-01.JPG}}
\end{center}

Outros conceitos importantes são o de validade e de fidedignidade dos
dados coletados, que podem ser compreendidos através da ilustração a
seguir:

\begin{figure}[H]

{\centering \pandocbounded{\includegraphics[keepaspectratio]{fig/preciscao-ou-fidedignidade-x-exatidao-ou-acuracia.png}}

}

\caption{Precisão ou Fidedignidade -x- Validade, Exatidão ou Acurácia}

\end{figure}%

Conferir também a fig.~2.1 - distinção entre confiabilidade e validade,
usando tiro ao alvo (Poldrack, 2025 , p.~13), que ainda conceitua
\emph{validade aparente}, \emph{de constructo} e \emph{preditiva}.
Segundo esse autor:

\begin{quote}
``A \textbf{\emph{confiabilidade}} se refere à \emph{consistência da
localização dos disparos}, e a \textbf{\emph{validade}} se refere à
\emph{acurácia} {[}em sentido estrito{]} \emph{dos disparos em relação
ao alvo}''.
\end{quote}

Perceba que, na prática, quando coletamos dados, desconhecemos a
localização do alvo, ou seja, não sabemos onde se localiza o
\emph{verdadeiro e desconhecido valor do parâmetro populacional} de
interesse.

Todavia, por meio de \ul{\textbf{\emph{uma}}} \textbf{\emph{amostra
probabilística}} de \textbf{\emph{tamanho adequado}} (n), é possível
\ul{\textbf{\emph{estimar}}} esse valor desconhecido de interesse da
pesquisa, bem como sua \ul{\textbf{\emph{acurácia}}} sob a
\ul{\textbf{\emph{suposição}}} uma \textbf{\emph{coleta}}
\textbf{\emph{válida}} de \textbf{\emph{dados fidedignos}}.

Essa suposição só se sustenta se forem tomados todos os cuidados
qualitativos de um adequado processo de amostragem probabilística, como,
por exemplo, \textbf{\emph{baixo}} \ul{\textbf{\emph{viés não
amostral}}} caracterizado por (BOLFARINE; BUSSAB, 2005 , p.~9, 27-28):

\begin{itemize}
\tightlist
\item
  \emph{baixa proporção} de \emph{viés de subcobertura} (indivíduos não
  previstos no SR - Sistema de Referência), que são os \emph{erros por}
  \emph{omissão};
\item
  baixa proporção de \emph{erros por comissão}, pela \emph{inclusão de
  elementos de outras populações} que \emph{não} a \emph{população alvo}
  ou \emph{inclusão} ou \emph{substituição voluntária} de elementos não
  sorteados na amostra probabilística (uma espécie de \emph{subamostra},
  de conveniência, pela \emph{escolha voluntária} do pesquisador, ou
  mesmo probabilística, ambas \emph{contaminantes} da \emph{amostra
  probabilística}, ex.: substituir todas as \texttt{k} observações
  perdidas -- não respostas ou NA's -- da \emph{amostra probabilística}
  de \emph{tamanho} \texttt{n} por outras \texttt{k} observações,
  escolhidas por conveniência ou mesmo quando produto de \emph{outra
  amostra aleatória de tamanho menor} \texttt{k} obtida da mesma
  população disponível);
\item
  \emph{baixa proporção} de não respondentes ou \emph{NA} (observações
  perdidas, parcial ou totalmente, no momento da coleta da amostra
  probabilística);
\item
  é necessário \emph{avaliar os efeitos} (quantitativos, por
  subestimativa ou superestimativa, e qualitativos nos dados coletados)
  da \emph{diferença de perfil} entre os \emph{respondentes} e os
  \emph{não respondentes} (NA's);
\item
  é necessário \emph{avaliar os efeitos} do \emph{eventual}
  \emph{processo de imputação de dados}, caso tenha sido usado para
  suprir informações não coletadas (NA's), como, por exemplo,
  substituição dos NA's parciais pelo valor médio ou mediano do restante
  dos elementos da amostra probabilística;
\item
  \emph{ausência} de \emph{viés de resposta voluntária} (os indivíduos
  da amostra escolheram-se);
\item
  \emph{ausência} de \emph{viés de insuficiênia do questionário}, por
  problemas em sua redação (má compreensão do sentido da pergunta);
\item
  \emph{ausência} de \emph{viés de fraseado} das perguntas no
  levantamento por survey (questionário fechado);
\item
  \emph{ausência} de \emph{viés} por \emph{efeito do entrevistador},
  como seu seu modo de vestir, de postar ou de falar;
\item
  \emph{ausência} de \emph{viés de fraseado} das perguntas no
  levantamento por survey (por indução de resposta no questionário
  fechado, ou mesmo no aberto);
\item
  \emph{ausência} de \emph{erros de codificação} e de digitação dos
  dados tabulados;
\item
  \emph{minimização} dos \emph{erros de observação}, ocorridos durante o
  processo de levantamento de dados;
\item
  \emph{minimização} dos \emph{erros de medição}, em razão da
  \emph{descalibração} do \emph{instrumento utilizado} ou mesmo
  decorrentes de sua \emph{inadequada aplicação} pelo
  \emph{instrumentador}, ocorridos durante o \emph{intervalo de tempo}
  decorrido no processo de levantamento de dados;
\item
  etc.
\end{itemize}

Trata-se de uma \emph{lista meramente enumerativa}, que não afasta a
\textbf{\emph{necessidade}}, por exemplo, de uma \textbf{\emph{reflexão
crítica}} quanto às \emph{possíveis} \textbf{\emph{variáveis ocultas}}
ou \textbf{\emph{não observadas}} pelo \ul{\textbf{\emph{quadro de
variáveis}}} que foi \ul{\textbf{\emph{escolhido}}} pelo
\textbf{\emph{pesquisador}} (geralmente produto de sua conveniência e
\emph{não} de uma \emph{revisão sistemática da literatura} - RSL) que
serviu de base para a \emph{coleta e tabulação} dos \emph{dados
primários} levantados, seja diretamente por ele (de preferência), ou por
pessoa por ele para tanto bem treinada e testada.

\begin{tcolorbox}[enhanced jigsaw, arc=.35mm, opacitybacktitle=0.6, colframe=quarto-callout-warning-color-frame, titlerule=0mm, leftrule=.75mm, left=2mm, colbacktitle=quarto-callout-warning-color!10!white, breakable, toprule=.15mm, bottomtitle=1mm, opacityback=0, coltitle=black, title=\textcolor{quarto-callout-warning-color}{\faExclamationTriangle}\hspace{0.5em}{Warning}, rightrule=.15mm, bottomrule=.15mm, toptitle=1mm, colback=white]

Apenas quando \ul{\textbf{\emph{todos}}} esses
\ul{\textbf{\emph{cuidados qualitativos}}} forem tomados para
\ul{\textbf{\emph{prevenção}}} contra a presença de
\ul{\textbf{\emph{vieses não amostrais}}} é que se poderá
\ul{\textbf{\emph{validar}}} uma \emph{coleta probabílistica} de
\emph{dados amostrais} \ul{\textbf{\emph{fidedignos}}}.

\end{tcolorbox}

A seguir a ideia de Ciclo da \textbf{Estatística Básica Inferencial},
que, após uma boa Análise Estatística Descritiva (AED) e Exploratória
(AEE) dos dados, busca \emph{chegar}, por meio do
\ul{\textbf{\emph{método indutivo}}}, a \ul{\textbf{\emph{conclusões}}}
\ul{\textbf{\emph{válidas}}} \textbf{\emph{e \ul{confiáveis}}} para
\textbf{\emph{toda}} a \ul{\textbf{\emph{população amostrada}}} a partir
de \ul{\textbf{\emph{uma}}} \ul{\textbf{\emph{amostra probabilística}}}
\ul{\textbf{\emph{válida}}} \emph{e \ul{\textbf{fidedígna}}} daquela
coletada.

\begin{figure}[H]

{\centering \pandocbounded{\includegraphics[keepaspectratio]{fig/inferencia.png}}

}

\caption{Ciclo da Inferência ou indução Estatística.}

\end{figure}%

O emprego desse método indutivo na inferência estatística pode ser visto
na indicação da seta inferior esquerda ilustrada em um mais amplo
\textbf{Conceito} de \textbf{Ciclo da Ciência}, cujo formato costuma ser
designado como \textbf{\emph{Diamond Shape}} (DONOVAN; MICKEY, 2019 ,
p.~274-275):

\begin{figure}[H]

{\centering \includegraphics[width=6.54167in,height=\textheight,keepaspectratio]{fig/DiamondShape-Donovan.jpg}

}

\caption{Conceito de Ciclo da Ciência na forma de um Diamond Shape
(DONOVAN; MICKEY, 2019, p.~274-275))}

\end{figure}%

Perceba-se, na ilustração acima, a importância da
\textbf{\emph{articulação}} da prévia da \ul{\textbf{\emph{teoria}}}, de
\emph{base} juntamente com \emph{uma teoria rival}, na busca de
\textbf{\emph{extração}} de \ul{\textbf{\emph{evidências}}} a
\emph{partir} de \ul{\textbf{\emph{um conjunto de dados validamente
coletados}}}.

Primordial, para evitar \textbf{\emph{Pràticas de Pesquisa
Questionáveis}} -- \textbf{PPQ}, que o pesquisador realize um
\ul{\textbf{pré-registro}} (ex.: sistema
\href{https://clinicaltrials.gov./}{\emph{ClinicalTrials.gov.}}) de pelo
menos um par de hipóteses, \emph{nula} (H\textsubscript{0}) e
alternativa (H\textsubscript{a}), \emph{adequadamente formuladas}, que
ele pretende testar. Isso bem \textbf{\emph{antes}} dele
\textbf{\emph{iniciar}} sua \textbf{\emph{coleta de dados}}, para
prevenir e \ul{\textbf{\emph{evitar}}} qualquer possibilidade de prática
de \ul{\textbf{\emph{HARKing}}} (\emph{hypothesizing after the results
are known}) ou de \emph{p-hacking}, o que possibilita que o pesquisador
\ul{\emph{reformule}} uma \ul{conclusão \emph{post-hoc}}
\textbf{\emph{como se}} fora uma \ul{predição \emph{a priori}}, que goza
de maior confiança, dado que ele, nesse caso, estaria indevidamente
\emph{reescrevendo sua teoria} de partida (Poldrack, 2025 , p.~266-267),
a de base e a rival, com \ul{\emph{base nos conjunto de dados
coletados}} (uma espécie de \emph{Teorização Fundamentada em Dados} -
\textbf{\emph{TFD ad hoc}}, que é obtida por meio da indução e que seria
então ``testada'' com base no mesmo conjunto de dados donde ela, \emph{a
posteriori}, proveio, ou seja, sob um indevido e \emph{inválido viés de
confirmação}, uma vez que as hipóteses, assim reelaboradas, claramamente
hão de ajustar-se aos dados colhidos), ao invés de elaborar
\ul{\textbf{\emph{predições}}} ou \ul{\textbf{\emph{conjecturas}}}
\ul{\textbf{\emph{deduzidas}}} das \ul{\textbf{\emph{teorias}}}
\ul{\textbf{\emph{previamente escolhidas}}}, conforme muito bem
ilustrado pelo \emph{retângulo superior} e \emph{seta superior direita}
do \textbf{\emph{diamond shape}} da figura acima.

Logo, é somente a \ul{\textbf{\emph{qualidade}}} do
\ul{\textbf{\emph{desenho do experimento}}} (\emph{Design of Experiment}
-- \emph{D.o.E.} referido na \emph{seta inferior direita} do mesmo
\emph{diamond shape}) que \ul{\textbf{\emph{garante}}}, através do
\ul{\textbf{\emph{consenso}}} dos \ul{\textbf{\emph{experts}}} que atuam
na \emph{específica área de conhecimento científico da pesquisa}
proposta, muitas vezes denominado por \ul{\textbf{\emph{crivo dos
pares}}}, que vai muito além de uma \emph{avaliação independente duplo
cego} pelos \emph{referees} dos periódicos científicos, porquanto é
\ul{\textbf{\emph{ônus}}} de \ul{\textbf{\emph{qualquer pesquisador}}}
\emph{demonstrar} que \ul{\textbf{\emph{aderiu}}} às seguintes
\ul{\textbf{\emph{boas práticas reproduzíveis}}} de pesquisa:

\begin{quote}
▶Decidir regras para finalizar a coleta de dados antes do seu início e
elencá-la no artigo.

▶Coletar, pelo menos, 20 observações por unidade de análise ou fornecer
uma justificativa convincente do custo da coleta de dados.

▶Listar todas as variáveis coletadas em um estudo.

▶Relatar todas as condições experimentais, incluindo manipulações
malsucedidas.

▶Elencar como seriam os resultados estatísticos incluindo as observações
eliminadas, se houver.

▶Relatar os resultados estatísticos sem a covariável se uma análise
incluir uma covariável.

\textbf{Replicação}

Um dos balizadores da ciência é o conceito de replicação --- ou seja,
\textbf{\emph{outros pesquisadores devem ser capazes de realizar o mesmo
estudo e obter o mesmo resultado}}. \emph{Infelizmente}, conforme vimos
o que aconteceu com o \emph{Reproducibility Project} analisado
anteriormente neste capítulo, \emph{muitas descobertas \ul{não} são
replicáveis}.

A \emph{melhor forma} de \emph{assegurar} a \emph{replicabilidade} de
sua \emph{pesquisa} é, primeiro, \ul{\emph{replicá-la por conta
própria}}; para alguns estudos, isso não será possível, mas \emph{sempre
que possível}, você \emph{deve garantir} que \ul{\emph{sua descoberta se
sustente em uma amostra nova}}, que deve ter \ul{\emph{potência
suficiente}} para \ul{\emph{encontrar}} o \ul{\emph{tamanho do efeito}}
de \emph{interesse}; em \emph{muitos casos}, isso \ul{\emph{exigirá}}
\ul{\emph{uma amostra maior do que a original}}.

É importante considerar alguns pontos quando se trata de replicação.

\ul{\textbf{Primeiro}}, o fato de \ul{\textbf{\emph{uma tentativa de
replicação ser malsucedida não significa necessariamente que a
descoberta original era falsa}}}; lembre que, com o \ul{\emph{nível
padrão de 80\% de potência}}, \ul{\textbf{\emph{ainda existe 1 chance em
5 de que o resultado não seja significativo}}}, \ul{\textbf{\emph{mesmo
que exista um efeito verdadeiro}}}.

Por esse motivo, \emph{queremos normalmente observar
\ul{\textbf{múltiplas replicações}}} em \emph{qualquer descoberta
importante} \ul{\textbf{\emph{antes de decidir se devemos ou não
acreditar nela}}} e, em geral, queremos que as
\ul{\textbf{\emph{tentativas de replicação tenham níveis de potência
maiores do que o original}}}.

Infelizmente, \ul{\textbf{\emph{muitas áreas}}}, incluindo a
\emph{Psicologia}, \ul{\textbf{\emph{não adotaram}}} esse conselho
{[}essa \emph{Regra de Boas Práticas de Pesquisa Reproduzível - PPQ}{]}
ao \emph{longo} dos \ul{\textbf{\emph{últimos anos}}},
\ul{\textbf{\emph{levando}}} a \ul{\textbf{\emph{descobertas}}}
\textbf{\emph{``\ul{amplamente aceitas}''}} que
\ul{\textbf{\emph{provavelmente se revelaram falsas}}}.

No que diz respeito aos estudos de PES de Daryl Bem, uma grande
tentativa de replicação envolvendo sete estudos \emph{não conseguiu
replicar as descobertas} (Galak et al, 2012).

\ul{\textbf{Segundo}}, lembre-se de que o \ul{\textbf{\emph{valor-p não
nos fornece uma medida da verossimilhança de replicação de uma
descoberta}}}.

Conforme examinamos anteriormente, o \ul{\textbf{\emph{valor-p}}} é
\ul{\textbf{\emph{uma afirmação sobre a verossimilhança dos dados}}},
\ul{\textbf{\emph{considerando uma hipótese nula específica}}}
\textbf{\emph{{[}sob a suposição de que Ho fosse verdadeira{]}}}; ele
\ul{\textbf{\emph{não indica nada}}} sobre a
\ul{\textbf{\emph{probabilidade}}} de que a
\ul{\textbf{\emph{descoberta}}} seja \ul{\textbf{\emph{efetivamente
verdadeira}}} (conforme aprendemos no Capítulo 11). Para
\ul{\textbf{\emph{saber}}} a \ul{\textbf{\emph{verossimilhança de
replicação}}}, \ul{\textbf{\emph{precisamos saber a probabilidade de a
descoberta ser verdadeira}}}, o que \textbf{\emph{geralmente não
sabemos}}. (Poldrack, 2025 , p.~269-270)
\end{quote}

Tudo isso sob \ul{\textbf{\emph{pena}}} de sua
\ul{\textbf{\emph{pesquisa empírica}}} ser
\ul{\textbf{\emph{classificada}}} como \ul{\textbf{\emph{facilmente
contestável}}}, por \textbf{\emph{adesão}}, \emph{voluntária} ou
\emph{involuntária}, às \ul{\textbf{Práticas de Pesquisa Questionáveis
-- PPQ}} (Poldrack, 2025 , p.~266-267).

Uma das modalidades de conceituar e aplicar a Probabilidade e
Estatística é pela denominada \textbf{Estatística bayesiana}, ou seja,
aquela apoiada no \textbf{conceito} de \textbf{\emph{probabilidade
condicional}} e no \textbf{\emph{Teorema de Bayes}}, ilustrado na figura
a seguir:

\begin{figure}[H]

{\centering \pandocbounded{\includegraphics[keepaspectratio]{fig/fig5.4_BayesTheorem.JPG}}

}

\caption{Teorema de Bayes para testar um par de Hipóteses (Ho:
\textasciitilde A e Ha: A) com suas probabilidades a priori mediadas e
atualizadas por uma coleta de dados B que permite inferir as respectivas
probabilidades a posteriori desse par. Que pode ser reiterado com novas
coletas de dados C, D, \ldots{}}

\end{figure}%

A partir do conceito de \ul{\emph{probabilidade condicional}} e de
\ul{\textbf{probabilidade \emph{a priori}}}, que se localiza no
\emph{primeiro fator} do \emph{numerador} do \emph{lado direito} da
\emph{igualdade} da \emph{fórmula} da figura acima que \emph{presenta} o
\ul{\textbf{\emph{Teorema de Byes}}}, chegou-se ao
\textbf{\emph{conceito}} de \ul{\textbf{\emph{Fator de Bayes}}}
(\textbf{FB}), que \ul{\textbf{\emph{mede}}} as \ul{\textbf{\emph{razões
de chance}}} (\emph{odds ratio}) entre a
\ul{\textbf{\emph{verossimilhança}}} ou \ul{\textbf{\emph{probabilidade
a priori}}} \ul{\textbf{\emph{dos dados}}} sob a \textbf{\emph{Hipótese
Alternativa}} (a probabilidade dos dados empíricos colhidos
\ul{\emph{supondo}} que \textbf{H\textsubscript{a}} fosse Verdadeira) em
\ul{\emph{relação à}} \ul{\textbf{\emph{verossimilhança}}} ou
\ul{\textbf{\emph{probabilidade a priori}}} \ul{\textbf{\emph{dos
dados}}} sob a \emph{Hipótese Nula} (a probabilidade dos dados empíricos
colhidos \ul{\emph{supondo}} que \textbf{H\textsubscript{0}} fosse
Verdadeira), traduzida pela seguinte expressão:

\[ FB = \frac{p(dados|H_a)}{p(dados|H_0)} \]

Ou seja, o \textbf{FB} ``\ldots{}\textbf{\emph{Fator de Bayes
caracteriza a verossimilhança relativa dos dados}},
\textbf{\emph{considerando duas hipóteses diferentes}}'' (Poldrack, 2025
, p.~140).

Esse \textbf{FB} - \textbf{\emph{Fator de Bayes}}, agora sim, é capaz de
\ul{\textbf{\emph{medir}}} a \ul{\textbf{\emph{força}}} da
\ul{\textbf{\emph{evidência}}} do \ul{\textbf{\emph{resultado}}}
consistente na \ul{\textbf{\emph{decisão}}} pela
\ul{\textbf{\emph{rejeição}}} da \textbf{H\textsubscript{0}} em
\emph{um} \ul{\textbf{\emph{Teste de Significância da Hipótese Nula}}}
(\textbf{\emph{NHST}} - \emph{Null Huphoteses Significant Test})
(Poldrack, 2025 , p.~140-142), pois:

\begin{quote}
Desse modo, \ul{\textbf{\emph{um valor maior que 1 refletirá maior
evidência}}} para Smith {[}\textbf{H\textsubscript{a}}{]}, e
\ul{\textbf{\emph{um valor menor que 1 refletirá maior evidência}}} para
Jones {[}\textbf{H\textsubscript{0}}{]}.

O \ul{\textbf{\emph{fator de Bayes}}} resultante (\textbf{3325,26})
fornece \ul{\textbf{\emph{uma medida das evidências}}} que
\ul{\textbf{\emph{os dados fornecem}}} em \ul{\textbf{\emph{relação às
duas hipóteses}}} --- aqui, \ul{\textbf{\emph{informa}}} que a
\ul{\textbf{\emph{hipótese do senador Smith}}} \ul{\textbf{\emph{é mais
fortemente respaldada pelos dados}}} do \ul{\textbf{\emph{que}}} a
\ul{\textbf{\emph{hipótese do senador Jones}}}. (Poldrack, 2025 ,
p.~140)
\end{quote}

Há um critério consensual na literatura de como se deve interpretar o
Fator de Bayes.

\begin{quote}
\textsc{interpretando os fatores de Bayes}

\textbf{\emph{Como sabemos se um fator de Bayes de 2 ou de 20 é eficaz
ou ineficaz?}}

Existe \ul{\textbf{\emph{uma diretriz geral}}} para a
\ul{\textbf{\emph{interpretação}}} dos \ul{\textbf{\emph{fatores de
Bayes}}} sugerida por
\href{https://www.andrew.cmu.edu/user/kk3n/simplicity/KassRaftery1995.pdf}{Kass
e Raftery (1995)}:
\end{quote}

A tabela a seguir, cf.~cap. 11 (Poldrack, 2025 , cap. 11, p.~143),
resume esse \ul{\textbf{\emph{critério de interpretação não
discricionário}}}, pois consensuado na literatura.

A seguir carrega-se o conjunto de dados \texttt{NHANES}, que é muito
trabalho ao longo de todo o livro de Poldrack.

Bem como o conjunto de pacotes necessários para gerar a tabela referida.

\begin{Shaded}
\begin{Highlighting}[numbers=left,,]
\InformationTok{\textasciigrave{}\textasciigrave{}\textasciigrave{}\{r\}}
\FunctionTok{library}\NormalTok{(tidyverse)}
\FunctionTok{library}\NormalTok{(ggplot2)}
\FunctionTok{library}\NormalTok{(cowplot)}
\FunctionTok{library}\NormalTok{(boot)}
\FunctionTok{library}\NormalTok{(MASS)}
\FunctionTok{library}\NormalTok{(BayesFactor)}
\FunctionTok{library}\NormalTok{(knitr)}
\FunctionTok{theme\_set}\NormalTok{(}\FunctionTok{theme\_minimal}\NormalTok{(}\AttributeTok{base\_size =} \DecValTok{14}\NormalTok{))}

\FunctionTok{set.seed}\NormalTok{(}\DecValTok{123456}\NormalTok{) }\CommentTok{\# set random seed to exactly replicate results}

\CommentTok{\# load the NHANES data library}
\FunctionTok{library}\NormalTok{(NHANES)}

\CommentTok{\# drop duplicated IDs within the NHANES dataset}
\NormalTok{NHANES }\OtherTok{\textless{}{-}}
\NormalTok{  NHANES }\SpecialCharTok{\%\textgreater{}\%}
\NormalTok{  dplyr}\SpecialCharTok{::}\FunctionTok{distinct}\NormalTok{(ID, }\AttributeTok{.keep\_all =} \ConstantTok{TRUE}\NormalTok{)}

\NormalTok{NHANES\_adult }\OtherTok{\textless{}{-}}
\NormalTok{  NHANES }\SpecialCharTok{\%\textgreater{}\%}
  \FunctionTok{drop\_na}\NormalTok{(Weight) }\SpecialCharTok{\%\textgreater{}\%}
  \FunctionTok{subset}\NormalTok{(Age }\SpecialCharTok{\textgreater{}=} \DecValTok{18}\NormalTok{)}
\InformationTok{\textasciigrave{}\textasciigrave{}\textasciigrave{}}
\end{Highlighting}
\end{Shaded}

Salvar o \emph{data set} \texttt{NHANES} como arquivo \texttt{.csv} na
pasta \texttt{out} deste Projeto.

\begin{Shaded}
\begin{Highlighting}[numbers=left,,]
\InformationTok{\textasciigrave{}\textasciigrave{}\textasciigrave{}\{r\}}
\CommentTok{\# Salvar esse dataframe no formato binário do R na pasta out}
\CommentTok{\# Sua próxima importação ela virá com todos os tratamentos até aqui realizados:}
\CommentTok{\# tipos de colunas preservados: \textless{}char\textgreater{}, \textless{}date\textgreater{}, \textless{}time\textgreater{}, \textless{}fctr\textgreater{}, \textless{}int\textgreater{},}
\CommentTok{\# \textquotesingle{}A tibble\textquotesingle{}:   6,779 × 76 [mais de 6 mil observações e 76 variáveis]}
\FunctionTok{write\_rds}\NormalTok{(NHANES, }\AttributeTok{file =} \StringTok{"out/NHANES.rds"}\NormalTok{) }\CommentTok{\# formato binário do R}

\CommentTok{\# Salvar esse conjunto de dados tratado no formato .csv:}
\FunctionTok{write.csv}\NormalTok{(NHANES,}
          \AttributeTok{file      =} \StringTok{"out/NHANES.csv"}\NormalTok{,}
          \AttributeTok{na        =} \StringTok{""}\NormalTok{, }\CommentTok{\# salvar campos NA como espaço vazio \textless{}blank\textgreater{}}
          \AttributeTok{row.names =} \ConstantTok{FALSE}\NormalTok{) }\CommentTok{\# não salvar coluna com números das linhas}
\InformationTok{\textasciigrave{}\textasciigrave{}\textasciigrave{}}
\end{Highlighting}
\end{Shaded}

Conferir tabela da força do Fator de Bayes (Poldrack, 2025 , p.~143).

\begin{figure}[H]

{\centering \pandocbounded{\includegraphics[keepaspectratio]{fig/Forca-Evidencia-FBayes-tab.png}}

}

\caption{Interpretando os Fatores de Bayes (FB): Força da Evidência}

\end{figure}%

As 2 fases, de \textbf{AED} (Análise Exploratória \emph{Descritiva}) e
de \textbf{AEI} (Análise Exploratória \emph{Inferencial}), demandarão
\textbf{20\%} restante do tempo de uma pesquisa (80\% é gasto com a
coleta, organização e tratamento dos dados primários), costumam ser-nos
bem \emph{mais prazerosas}.

Nessa nova fase, bem mais atraente, o objetivo é \emph{explorar os
dados} em busca do \textbf{reconhecimento de padrões perceptíveis}.

Todavia, cuidado com a possibilidade do \emph{erro percepcional} ou com
a prática de \emph{HARKing} ou de \emph{p-haking}, como já visto.

Gerar vários \textbf{gráficos} (barras, colunas, pizza, diagrama de ramo
e folha, histogramas, \emph{boxplot}, dispersão etc.) que permitam essa
\textbf{\emph{visualização}} e \textbf{\emph{captura de padrões}} para
cada \textbf{\emph{tipo variável}} observada: categórica (nominal ou
ordinal) ou quantitativa (discreta ou contínua).

A figura a seguir ilustra essa classificação dos tipos de variáveis
(Escovedo, 2024 , p.~17).

\begin{figure}[H]

{\centering \pandocbounded{\includegraphics[keepaspectratio]{fig/TiposDeVariaveis.png}}

}

\caption{Tipos de Variáveis}

\end{figure}%

\textbf{\emph{Resumos dos dados}} são muito úteis: média e desvio
padrão; mediana e AIQ (Amplitude Interquartil); resumo dos 5 números,
coeficiente de variação, assimetria, curtose etc.

Bem como para investigar possibilidade de \textbf{\emph{associação}}
entre elas: a depender da combinação dos seus \textbf{\emph{tipos}}, por
meio de \textbf{\emph{testes estatísticos}} formais.

Gerar tabelas de dupla entrada ou de contigência, quando ambas forem
categóricas, do tipo \textbf{factor}
\textless{}\texttt{fctr}\textgreater.

Todavia, todas essas possibilidades de recorrer à \emph{Data Science} e
à Probabilidade e Estatística, no nosso caso, tomarão por
\textbf{domínio} a interface do Direito com as Políticas Públicas.

\subsection{Até breve}\label{atuxe9-breve}

Dúvidas serão debeladas a cada aula!

\begin{figure}[H]

{\centering \pandocbounded{\includegraphics[keepaspectratio]{fig/ValeuGalera.png}}

}

\caption{Até nosso pRRRóximo RRRencontro!}

\end{figure}%

\bookmarksetup{startatroot}

\chapter{AED - cap 3 - As Distribuições Normais}\label{sec-dist-Normal}

\begin{quote}
Nos Capítulos 1 e 2, discutimos \textbf{\emph{métodos}} para
\emph{\textbf{resumir} um grande conjunto de dados}. Esses incluem
\textbf{\emph{apresentações gráficas}}. Gráficos de barras e histogramas
\emph{e resumos numéricos}, como \emph{média}, \emph{mediana},
\emph{quartis} e \emph{desvio-padrão}. Neste capítulo, veremos
\emph{outra maneira de resumir um grande conjunto de dados} usando
\textbf{\emph{uma curva suave}} para dar a \textbf{\emph{forma geral}}
de \textbf{\emph{sua distribuição}}.

Dispomos, então, de \textbf{\emph{um conjunto de ferramentas gráficas e
numéricas}} para \textbf{\emph{descrever distribuições
{[}empíricas{]}}}. Mais ainda, temos uma estratégia clara para a
\textbf{\emph{exploração de dados de uma única variável quantitativa}}.
(MOORE; NOTZ; FLIGNER, 2023 , p.~59).
\end{quote}

\section{AED - Análise Exploratória de Dados (cap.
1)}\label{aed---anuxe1lise-exploratuxf3ria-de-dados-cap.-1}

\begin{quote}
\textbf{``O que {[}nos{]} dizem os dados?''} é a primeira pergunta que
fazemos em qualquer estudo estatístico. A \textbf{análise de dados}
responde a essa questão por \textbf{\emph{meio}} de uma
\textbf{\emph{exploração}} ampla dos dados. As
\textbf{\emph{ferramentas}} da análise de dados são
\textbf{\emph{gráficos}}, como os \emph{histogramas} e os
\emph{diagramas de dispersão}, \textbf{\emph{e medidas numéricas}}, como
as \emph{médias} e as \emph{correlações}. No entanto, ao menos tão
importantes quanto as \emph{ferramentas}, são os
\textbf{\emph{princípios}} que \emph{\textbf{organizam nosso pensamento}
no exame dos dados} (MOORE; NOTZ; FLIGNER, 2023 , p.~9).
\end{quote}

Essa pergunta torna-se mais precisa se a recolocarmos d oseguinte modo:

Considerando o modo como os dados foram coletados, o que nós estamos
autorizados a dizer sobre o \ul{\textbf{\emph{que}}} esses mesmo
\textbf{\emph{dados}} \textbf{\emph{``nos dizem''}}?

Uma vez que dados não dizem.

Eles não falam por si.

Nós é que os interpretamos.

E os limites dessa interpretação importa! E muito!

Sendo que esses limites dependem de como os dados foram coletados!

\begin{quote}
\textbf{8.4 Credibilidade da inferência a partir de amostras}

O objetivo de uma amostra é dar informação sobre uma população maior. O
processo de extração de conclusões sobre a população com base na amostra
de dados se chama \emph{inferência}, porque \emph{inferimos} informação
sobre a população a partir do que \emph{sabemos} sobre a amostra.

Inferência a partir de amostras de conveniência ou de resposta
voluntária seria enganosa, pois esses métodos de escolha de amostra são
viesados. Nesses casos, estamos quase absolutamente certos de que a
amostra \emph{não} representa precisamente a população. \emph{A
\textbf{primeira razão} para nos apoiarmos em amostragem aleatória é a
\textbf{eliminação do viés na seleção de amostras} de uma lista de
indivíduos disponíveis}.

No entanto, é pouco provável que os resultados a partir de uma amostra
aleatória sejam exatamente os mesmos para toda a população. Resultados
amostrais, como as taxas mensais de desemprego obtidas pela Pesquisa da
População Corrente, são apenas \emph{estimativas da verdade sobre a
população}. Se selecionarmos duas amostras aleatórias da mesma
população, iremos, quase certamente, selecionar indivíduos diferentes.
Assim, os \emph{resultados irão diferir de alguma forma, apenas pelo
acaso}. \emph{Amostras} \emph{adequadamente planejadas evitam viés
sistemático}, mas raramente seus resultados são exatamente precisos e
\emph{variam de amostra para amostra}.

\textbf{\emph{Por que podemos confiar em amostras aleatórias?}} A grande
ideia é de que os resultados de amostragem aleatória não mudam de
maneira fortuita de amostra para amostra. Como \ul{\textbf{\emph{usamos
o acaso deliberadamente}}}, os \ul{\textbf{\emph{resultados obedecem às
leis da probabilidade}}} que \ul{\textbf{\emph{governam o comportamento
aleatório}}}. Essas leis nos \ul{\textbf{\emph{permitem dizer quão
provavelmente os resultados amostrais estarão próximos da verdade sobre
a população}}}. A \textbf{\emph{segunda razão}} para o uso de amostragem
aleatória é que as \ul{\textbf{\emph{leis da probabilidade permitem
inferência confiável sobre a população}}}. \textbf{\emph{Resultados de
amostras aleatórias vêm com uma margem de erro que delimita o tamanho do
erro provável}}. Como fazer isso é parte da técnica da inferência
estatística. Apresentaremos o raciocínio no Capítulo 16 e detalhes em
todo o restante do livro.

Um ponto merece nota: \textbf{\emph{amostras aleatórias maiores fornecem
resultados mais precisos do que amostras menores}}. Tomando uma amostra
aleatória muito grande, você pode ter certeza de que o resultado
amostral está muito próximo da verdade sobre a população. A Pesquisa da
População Corrente contata cerca de \emph{60 mil} residências, de modo
que \emph{estima a taxa nacional de desemprego de modo muito preciso}.
Pesquisas de opinião que contatam \emph{1.000 ou 1.500} pessoas
apresentam resultados menos precisos.

\ul{\textbf{\emph{É uma ideia errada a de que tamanhos amostrais maiores
sempre dão resultados mais precisos}}}. Depois do debate democrático em
julho de 2019, uma pesquisa online em Nova Jersey listou Bernie Sanders
como vencedor do debate, obtendo 53\% dos 13.468 votos na pesquisa. No
entanto, uma pesquisa da Quinnipiac University, em 29 de julho, de uma
amostra aleatória de 807 democratas eleitores encontrou que apenas 8\%
escolheram Sanders como vencedor. Outras pesquisas obtiveram resultados
semelhantes.

Ao \ul{\textbf{\emph{ler resultados de uma pesquisa, não suponha que a
pesquisa seja exata porque o tamanho amostral é grande}}}. Você deve
\ul{\textbf{\emph{prestar mais atenção}}} ao \ul{\textbf{\emph{modo}}}
como a \ul{\textbf{\emph{amostra foi selecionada}}}.
\ul{\textbf{\emph{Técnicas amostrais viesadas continuam a fornecer
resultados viesados}}}, \ul{\textbf{\emph{não importando o tamanho da
amostra}}}. (MOORE; NOTZ; FLIGNER, 2023 , p.~168-169)
\end{quote}

Sob pena de nossas observações, achados, \emph{inferências} e
\emph{conclusões} (causais e não causais) de pesquisa tornarem-se
\textbf{\emph{facilmente contestáveis}}.

Quanto aos \textbf{\emph{princípios organizadores}} de um
\textbf{\emph{letramento}} ou pensamento estatístico, dois destacam-se
na AED - Análise Exploratória de Dados:

\begin{quote}
Um dos princípios organizadores da análise de dados consiste em
{[}\textbf{P\textsubscript{1}}{]} \textbf{\emph{olhar, primeiro, um item
de cada vez e,}} {[}\textbf{P\textsubscript{2}}{]} \textbf{\emph{depois,
as relações entre estes}}. Nossa apresentação segue esse princípio. Nos
\textbf{Capítulos 1 a 3}, você estudará \textbf{\emph{variáveis e suas
distribuições}}. Os \textbf{Capítulos 4 a 6} referem-se a
\textbf{\emph{relações}} entre variáveis. O Capítulo 7 faz uma revisão
dessa parte do texto (MOORE; NOTZ; FLIGNER, 2023 , p.~9).
\end{quote}

\textbf{Estatística}, melhor é referir-se à \textbf{Probabilidade e
Estatística}, é conceituada como: ramo da \emph{Matemática} aplicada que
reune \textbf{\emph{um conjunto de métodos}} para:

\begin{itemize}
\tightlist
\item
  \textbf{planejar} \emph{estudos observacionais} e \emph{experimentos
  aleatorizados} em qualquer área do conhecimento científico,
  notadamente para pesquisas empíricas;
\item
  \textbf{coletar} dados \textbf{válidos} e \textbf{fidedignos}; ou
  seja, com alta \textbf{\emph{acurácia}}.
\item
  \textbf{organizar},
\item
  \textbf{apresentar} (listas, tabelas, diagramas, fórmulas, gráficos,
  grafos etc),
\item
  \textbf{resumir},
\item
  \textbf{analisar},
\item
  \textbf{formular} e \textbf{testar} \textbf{\emph{hipóteses}}
  {[}\emph{NHST-Null Hypotesis Significant Test}{]} e
\item
  \textbf{interpretar} conjuntos de \textbf{dados} e
  \textbf{informações}; {[}O que ``\emph{nos dizem}'' os dados?{]}
\item
  \textbf{elaborar} conclusões baseadas em \textbf{evidências} {[}dados
  e informações válidos e fidedignos{]} para
\item
  \textbf{apoiar} tomadas de \textbf{decisão} e para
\item
  \textbf{gerir} ou \textbf{controlar} um conjunto de \textbf{ações} em
  curso: qualidade, escala, cobertura, custos financeiros,
  \emph{eficiência, eficácia, efetividade} etc. por meio de
  \textbf{indicadores} e de índices cuja \textbf{\emph{aplicabilidade,
  comparabilidade, consistência e difusão}} possam ser \textbf{testadas
  e validadas} por uma comunidade de experts.
\end{itemize}

O Professor João Luiz Becker (BECKER, 2015) promove uma clara
conceituação e distinção entre \textbf{dados} e \textbf{informações} e
ilustra o ciclo em que a coleta e a extração deles inserem-se num
processo mais amplo de obtenção de \textbf{conhecimento}, que prossegue
e passa pela \textbf{decisão} e pela \textbf{ação}.

\begin{center}
\pandocbounded{\includegraphics[keepaspectratio]{fig/Dados-Informacao-Decisao-Acao-Becker-fig11-p37-01.JPG}}
\end{center}

Outros conceitos importantes são o de \textbf{\emph{validade}} e de
\textbf{\emph{fidedignidade}} dos dados coletados, que podem ser
compreendidos através da ilustração a seguir:

\begin{figure}[H]

{\centering \pandocbounded{\includegraphics[keepaspectratio]{fig/preciscao-ou-fidedignidade-x-exatidao-ou-acuracia.png}}

}

\caption{Precisão ou Fidedignidade -x- Validade, Exatidão ou Acurácia}

\end{figure}%

Conferir também a fig.~2.1 - distinção entre confiabilidade e validade,
usando a \emph{metáfora do tiro ao alvo} (Poldrack, 2025 , p.~13), que
ainda conceitua validade aparente, de constructo e preditiva. Segundo
esse autor:

\begin{quote}
``A \emph{confiabilidade} se refere à consistência da localização dos
disparos, e a \emph{validade} se refere à \emph{acurácia} {[}em
\emph{sentido estrito}{]} dos disparos em relação ao alvo''.
\end{quote}

Outro conceito importante é o de \textbf{\emph{acurácia}}, em
\emph{sentido amplo}, que agrega tanto o de \emph{confiabilidade}
(precisão) como o de \emph{validade} ou \emph{veracidade} (exatidão).

\begin{figure}[H]

{\centering \pandocbounded{\includegraphics[keepaspectratio]{fig/Veracidade-Precisao-Acuracia-Cartografia-UFU-IFSC.png}}

}

\caption{Validade, precisão e acurácia (sentido amplo). Cf. UFU,
p.~\ldots{}}

\end{figure}%

Perceba que, na prática, quando coletamos dados, desconhecemos o alvo,
ou seja, o verdadeiro e desconhecido valor do parâmetro populacional de
interesse.

Por meio de \ul{\textbf{\emph{uma}}} \textbf{\emph{amostra
probabilística}} de \textbf{\emph{tamanho adequado}} (n), é possível
estimar esse valor desconhecido de interesse da pesquisa.

A seguir a ideia de Ciclo da \textbf{Estatística Básica Inferencial},
que, após uma boa Análise Estatística Descritiva (AED) e Exploratória
(AEE), busca chegar a conclusões para toda a população amostrada a
partir de \textbf{\emph{uma amostra probabilística}} \emph{válida e
fidedígna} daquela coletada.

\begin{figure}[H]

{\centering \pandocbounded{\includegraphics[keepaspectratio]{fig/inferencia.png}}

}

\caption{Ciclo da Inferência ou indução Estatística.}

\end{figure}%

É importante fixar os principais conceitos que serão trabalhados nesta
disciplina \textbf{CDE-a-DPP}, o que reclama incursionar em
\textbf{\emph{conceitos simples}} de \textbf{Estatística Básica}, como:
população, amostra, plano amostral, tabelas, variáveis (seus tipos),
gráficos (seus tipos), resumos numéricos como média, desvio padrão,
mediana, Amplitude Interquartil (AIQ), correlação, regressão etc.

A ferramenta \textbf{\emph{statkey}} é um bom aplicativo \emph{free on
line} para exercitar esses conceitos. Experimente ela com nosso
\textbf{\emph{data set}} já organizado \texttt{obitjcsv.csv}, que se
encontra na pasta \texttt{out} de nosso Projeto
\texttt{CDE-a-DPP-2.Rproj}; clique aqui:
\url{https://www.lock5stat.com/StatKey/index.html} para acessar esse
aplicativo. No plano de ensino há diversas outras (cf.~aula n.~2).

Uma opção \emph{free and open} é o:
\href{https://jasp-stats.org/}{JASP}, desenvolvido pela Universidade de
Amsterdan, sem necessidade de aprender uma linguagem de programação.

As duas 2 fases (a 1ª é a fase da coleta, organização e tratamento dos
dados, que demanda 80\% do tempo da pesquisa de campo), de \textbf{AED}
(Análise Exploratória Descritiva) e de \textbf{AEI} (Análise
Exploratória Inferencial), demandarão os \textbf{20\%} restante do tempo
de uma pesquisa e costumam ser-nos bem \emph{mais prazerosas}.

Nessa nova fase, bem mais atraente, o objetivo é explorar os dados em
busca do \textbf{reconhecimento de padrões perceptíveis} (cuidado com a
possibilidade do \emph{erro percepcional}).

Gerar vários \textbf{gráficos} (barras, colunas, pizza, diagrama de ramo
e folha, histogramas, \emph{boxplot}, dispersão etc.) que permitam essa
\textbf{\emph{visualização}} e \textbf{\emph{captura de padrões}} para
cada \textbf{\emph{tipo variável}} observada: categórica (nominal ou
ordinal) ou quantitativa (discreta ou contínua).

Bem como \textbf{\emph{resumos numéricos}} dos conjuntos de dados
coletados, que podem ser grandes.

A figura a seguir ilustra essa classificação dos tipos de variáveis
(Escovedo, 2024 , p.~17).

\begin{figure}[H]

{\centering \pandocbounded{\includegraphics[keepaspectratio]{fig/TiposDeVariaveis.png}}

}

\caption{Tipos de Variáveis}

\end{figure}%

\textbf{\emph{Resumos dos dados}} são muito úteis: média e desvio
padrão; mediana e AIQ (Amplitude Interquartil); resumo dos 5 números,
coeficiente de variação, assimetria, curtose etc.

Bem como para investigar possibilidade de \textbf{\emph{associação}}
entre elas: a depender da combinação dos seus \textbf{\emph{tipos}}, por
meio de \textbf{\emph{testes estatísticos}} formais.

Gerar \textbf{\emph{tabelas de dupla entrada}} ou de contigência, quando
ambas forem categóricas, do tipo \textbf{factor}
\textless{}\texttt{fctr}\textgreater.

\section{Um rápido exemplo}\label{um-ruxe1pido-exemplo}

Procurar fazer uso dos pacotes do R acima mencionados.

\subsection{Preparar}\label{preparar}

Limpar e \emph{setup} do ambiente a ser utilizado: limpar e preparar a
\emph{Environment}.

\begin{Shaded}
\begin{Highlighting}[numbers=left,,]
\InformationTok{\textasciigrave{}\textasciigrave{}\textasciigrave{}\{r setup, include=TRUE\}}
\CommentTok{\# Deletar os objetos da Global Environment}
\FunctionTok{rm}\NormalTok{(}\AttributeTok{list=}\FunctionTok{ls}\NormalTok{())}

\CommentTok{\# Padrão de saídas Rmarkdown}
\NormalTok{knitr}\SpecialCharTok{::}\NormalTok{opts\_chunk}\SpecialCharTok{$}\FunctionTok{set}\NormalTok{(}\AttributeTok{echo =} \ConstantTok{TRUE}\NormalTok{, }\AttributeTok{warning =} \ConstantTok{FALSE}\NormalTok{)}

\CommentTok{\# Instalar tidyverse caso não esteja já instalado}
\ControlFlowTok{if}\NormalTok{ (}\SpecialCharTok{!}\FunctionTok{require}\NormalTok{(}\StringTok{\textquotesingle{}tidyverse\textquotesingle{}}\NormalTok{)) }\FunctionTok{install.packages}\NormalTok{(}\StringTok{\textquotesingle{}tidyverse\textquotesingle{}}\NormalTok{)}
\CommentTok{\# Instalar pacote magrittr caso não esteja já instalado}
\ControlFlowTok{if}\NormalTok{ (}\SpecialCharTok{!}\FunctionTok{require}\NormalTok{(}\StringTok{"magrittr"}\NormalTok{)) }\FunctionTok{install.packages}\NormalTok{(}\StringTok{"magrittr"}\NormalTok{)}
\CommentTok{\# Instalar pacote mlr caso não esteja já instalado}
\ControlFlowTok{if}\NormalTok{ (}\SpecialCharTok{!}\FunctionTok{require}\NormalTok{(}\StringTok{"mlr"}\NormalTok{)) }\FunctionTok{install.packages}\NormalTok{(}\StringTok{"mlr"}\NormalTok{)}

\CommentTok{\# Carregar o pacote DBI na Global Environment: disponível para uso direto}
\FunctionTok{library}\NormalTok{(}\StringTok{\textquotesingle{}tidyverse\textquotesingle{}}\NormalTok{)}
\CommentTok{\# Warning: package ‘tidyverse’ was built under R version 4.2.3}
\CommentTok{\# Warning: package ‘ggplot2’ was built under R version 4.2.3}
\CommentTok{\# Warning: package ‘tibble’ was built under R version 4.2.3}
\CommentTok{\# Warning: package ‘tidyr’ was built under R version 4.2.3}
\CommentTok{\# Warning: package ‘readr’ was built under R version 4.2.3}
\CommentTok{\# Warning: package ‘purrr’ was built under R version 4.2.3}
\CommentTok{\# Warning: package ‘dplyr’ was built under R version 4.2.3}
\CommentTok{\# Warning: package ‘stringr’ was built under R version 4.2.3}
\CommentTok{\# Warning: package ‘forcats’ was built under R version 4.2.3}
\CommentTok{\# Warning: package ‘lubridate’ was built under R version 4.2.3}
\CommentTok{\# ── Attaching core tidyverse packages \# ──────────────────────────────────────────── tidyverse 2.0.0 ──}
\CommentTok{\# ✔ dplyr     1.1.2     ✔ readr     2.1.5}
\CommentTok{\# ✔ forcats   1.0.0     ✔ stringr   1.5.1}
\CommentTok{\# ✔ ggplot2   3.5.1     ✔ tibble    3.2.1}
\CommentTok{\# ✔ lubridate 1.9.3     ✔ tidyr     1.3.1}
\CommentTok{\# ✔ purrr     1.0.2     }
\CommentTok{\# ── Conflicts ────────────────────────────────────────────────────────────── \# tidyverse\_conflicts() ──}
\CommentTok{\# ✖ dplyr::filter() masks stats::filter()}
\CommentTok{\# ✖ dplyr::lag()    masks stats::lag()}
\CommentTok{\# ℹ Use the conflicted package to force all conflicts to become errors}

\CommentTok{\# Carregar o pacote magrittr na Global Environment: disponível para uso direto}
\FunctionTok{library}\NormalTok{(}\StringTok{"magrittr"}\NormalTok{)}
\CommentTok{\# Attaching package: ‘magrittr’}
\CommentTok{\# }
\CommentTok{\# The following object is masked from ‘package:purrr’:}
\CommentTok{\# }
\CommentTok{\# set\_names}
\CommentTok{\# }
\CommentTok{\# The following object is masked from ‘package:tidyr’:}
\CommentTok{\# }
\CommentTok{\#     extract}

\CommentTok{\# Carregar o pacote mlr na Global Environment: disponível para uso direto}
\FunctionTok{library}\NormalTok{(}\StringTok{"mlr"}\NormalTok{)}

\CommentTok{\# Carregar o pacote rmarkdown na Global Environment: disponível para uso direto}
\CommentTok{\# library("rmarkdown")}
\InformationTok{\textasciigrave{}\textasciigrave{}\textasciigrave{}}
\end{Highlighting}
\end{Shaded}

\subsection{Importar}\label{importar}

\textbf{1}. \textbf{Importar} o \emph{data set}, o arquivo
\texttt{obitjcsv.csv}, que se encontra na pasta \texttt{out} de nosso
Projeto \texttt{CDE-a-DPP.Rproj}. Recomenda-se baixar a atualizar a
última versão desse nosso projeto que se encontra compartilhado no
google drive:
\url{https://drive.google.com/drive/u/1/folders/1wm9jUo5XlBHqbQDRf9XevFbXcqkWogqt}

\begin{Shaded}
\begin{Highlighting}[numbers=left,,]
\InformationTok{\textasciigrave{}\textasciigrave{}\textasciigrave{}\{r\}}
\CommentTok{\# Importar como tibble o arquivo de dentro da pasta chamada out.}
\NormalTok{obitj\_csv }\OtherTok{\textless{}{-}}\NormalTok{ readr}\SpecialCharTok{::}\FunctionTok{read\_csv}\NormalTok{(}\AttributeTok{file   =} \StringTok{"out/obitjcsv.csv"}\NormalTok{,}
                             \CommentTok{\# delim  = ",",}
                             \AttributeTok{quote  =} \StringTok{"}\SpecialCharTok{\textbackslash{}"}\StringTok{"}\NormalTok{,}
                             \AttributeTok{locale =} \FunctionTok{locale}\NormalTok{(}
                               \AttributeTok{decimal\_mark =} \StringTok{"."}\NormalTok{,}
                               \AttributeTok{encoding     =} \StringTok{"UTF{-}8"}
\NormalTok{                               )}
\NormalTok{                             )}

\CommentTok{\# cat {-} Concatenate And Print}
\FunctionTok{cat}\NormalTok{(}\StringTok{"}\SpecialCharTok{\textbackslash{}n}\StringTok{"}\NormalTok{) }\CommentTok{\# imprime no console (saída) uma linha em branco}
\FunctionTok{cat}\NormalTok{(}\StringTok{"Estrutura do objeto R denominado obitj\_csv:}\SpecialCharTok{\textbackslash{}n}\StringTok{"}\NormalTok{)}
\FunctionTok{str}\NormalTok{(obitj\_csv)}

\FunctionTok{cat}\NormalTok{(}\StringTok{"}\SpecialCharTok{\textbackslash{}n}\StringTok{"}\NormalTok{)}
\FunctionTok{cat}\NormalTok{(}\StringTok{"Nomes das 24 colunas do objeto obitj\_csv:}\SpecialCharTok{\textbackslash{}n}\StringTok{"}\NormalTok{)}
\FunctionTok{names}\NormalTok{(obitj\_csv)}

\NormalTok{obitj\_csv }\CommentTok{\# tibble:447 × 24}
\InformationTok{\textasciigrave{}\textasciigrave{}\textasciigrave{}}
\end{Highlighting}
\end{Shaded}

\begin{verbatim}

Estrutura do objeto R denominado obitj_csv:
spc_tbl_ [447 x 24] (S3: spec_tbl_df/tbl_df/tbl/data.frame)
 $ nomean   : chr [1:447] "A.J.D.S.A." "A.G.M.B." "A.G.D.S." "A.J.D.S.S." ...
 $ maean    : chr [1:447] "L.M.D.S.A." "A.D.S.M.B.B." "F.A.G." "S.D.S.S." ...
 $ nasc     : Date[1:447], format: "2001-04-21" "2000-09-04" ...
 $ sexo     : chr [1:447] "m" "m" "m" "m" ...
 $ cor      : chr [1:447] "branco" "pardo" NA "pardo" ...
 $ corag    : chr [1:447] "branco" "negro" NA "negro" ...
 $ dom      : chr [1:447] "Bairro São Francisco" NA NA "Vila Finsocial" ...
 $ dataesc1 : Date[1:447], format: "2018-09-28" NA ...
 $ esc1     : chr [1:447] "6 ano" NA NA "1 série EM" ...
 $ esc2     : chr [1:447] NA "8 ano" "9 ano" "1 série EM" ...
 $ compfam  : chr [1:447] "parentes" NA NA "mãe" ...
 $ relpai   : chr [1:447] "ausente" NA NA "auxílio" ...
 $ usudrog  : chr [1:447] "s" NA NA "s" ...
 $ subst    : chr [1:447] "maconha" NA NA "cocaína / crack" ...
 $ orgcrim  : chr [1:447] "n" NA NA NA ...
 $ sitdiv   : chr [1:447] NA NA NA "TDAH" ...
 $ dataobt  : Date[1:447], format: "2021-10-13" "2018-08-19" ...
 $ morte    : chr [1:447] "nat" "viol" "viol" "viol" ...
 $ paf      : chr [1:447] "n" "s" "n" "n" ...
 $ circobt  : chr [1:447] "Outros" "conflitos entre criminalidade" "conflitos entre criminalidade" "Outros" ...
 $ obsobt   : chr [1:447] NA NA NA NA ...
 $ idadeobtd: num [1:447] 7480 6558 7384 6614 8353 ...
 $ idadeobta: num [1:447] 20.5 18 20.2 18.1 22.9 ...
 $ npassag  : num [1:447] 5 4 10 2 8 2 2 2 5 2 ...
 - attr(*, "spec")=
  .. cols(
  ..   nomean = col_character(),
  ..   maean = col_character(),
  ..   nasc = col_date(format = ""),
  ..   sexo = col_character(),
  ..   cor = col_character(),
  ..   corag = col_character(),
  ..   dom = col_character(),
  ..   dataesc1 = col_date(format = ""),
  ..   esc1 = col_character(),
  ..   esc2 = col_character(),
  ..   compfam = col_character(),
  ..   relpai = col_character(),
  ..   usudrog = col_character(),
  ..   subst = col_character(),
  ..   orgcrim = col_character(),
  ..   sitdiv = col_character(),
  ..   dataobt = col_date(format = ""),
  ..   morte = col_character(),
  ..   paf = col_character(),
  ..   circobt = col_character(),
  ..   obsobt = col_character(),
  ..   idadeobtd = col_double(),
  ..   idadeobta = col_double(),
  ..   npassag = col_double()
  .. )
 - attr(*, "problems")=<externalptr> 

Nomes das 24 colunas do objeto obitj_csv:
 [1] "nomean"    "maean"     "nasc"      "sexo"      "cor"       "corag"    
 [7] "dom"       "dataesc1"  "esc1"      "esc2"      "compfam"   "relpai"   
[13] "usudrog"   "subst"     "orgcrim"   "sitdiv"    "dataobt"   "morte"    
[19] "paf"       "circobt"   "obsobt"    "idadeobtd" "idadeobta" "npassag"  
# A tibble: 447 x 24
   nomean     maean    nasc       sexo  cor   corag dom   dataesc1   esc1  esc2 
   <chr>      <chr>    <date>     <chr> <chr> <chr> <chr> <date>     <chr> <chr>
 1 A.J.D.S.A. L.M.D.S~ 2001-04-21 m     bran~ bran~ Bair~ 2018-09-28 6 ano <NA> 
 2 A.G.M.B.   A.D.S.M~ 2000-09-04 m     pardo negro <NA>  NA         <NA>  8 ano
 3 A.G.D.S.   F.A.G.   1997-06-22 m     <NA>  <NA>  <NA>  NA         <NA>  9 ano
 4 A.J.D.S.S. S.D.S.S. 2002-05-01 m     pardo negro Vila~ 2020-02-09 1 sé~ 1 sé~
 5 A.A.D.A.   A.A.D.A. 1999-10-13 m     <NA>  <NA>  <NA>  NA         <NA>  8 ano
 6 A.E.M.     M.E.M.   2001-03-25 m     bran~ bran~ <NA>  2018-10-23 8 ano 8 ano
 7 A.C.A.     J.A.D.A. 2000-12-18 m     pardo negro Seto~ 2018-11-18 8 ano 8 ano
 8 A.O.       D.G.B.S. 2003-09-30 m     bran~ bran~ Bair~ 2019-10-01 8 ano 8 ano
 9 A.R.C.     M.D.R.D~ 2001-02-24 m     pardo negro <NA>  NA         <NA>  <NA> 
10 A.O.S.     K.R.S.   2003-02-06 m     pardo negro Parq~ 2018-07-17 9 ano 8 ano
# i 437 more rows
# i 14 more variables: compfam <chr>, relpai <chr>, usudrog <chr>, subst <chr>,
#   orgcrim <chr>, sitdiv <chr>, dataobt <date>, morte <chr>, paf <chr>,
#   circobt <chr>, obsobt <chr>, idadeobtd <dbl>, idadeobta <dbl>,
#   npassag <dbl>
\end{verbatim}

\subsection{Transformar}\label{transformar}

\textbf{2}. \textbf{Transformar} esse \emph{data set} para que criar as
seguintes variáveis categóricas:

\textbf{\emph{Transformar}}, antes, as variáveis tipo \texttt{char} que
enquadram-se como factor: \texttt{fctr}.

\begin{Shaded}
\begin{Highlighting}[numbers=left,,]
\InformationTok{\textasciigrave{}\textasciigrave{}\textasciigrave{}\{r\}}
\CommentTok{\# para explicitar a ordem das categorias nas variáveis}
\CommentTok{\# que medem níveis de escolaridades: esc1 e esc2}
\CommentTok{\# variável categórica ordinal com 12 levels}
\NormalTok{series }\OtherTok{\textless{}{-}} \FunctionTok{c}\NormalTok{(}
  \StringTok{"1 ano"}\NormalTok{,}
  \StringTok{"2 ano"}\NormalTok{,}
  \StringTok{"4 ano"}\NormalTok{,}
  \StringTok{"5 ano"}\NormalTok{,}
  \StringTok{"6 ano"}\NormalTok{,}
  \StringTok{"7 ano"}\NormalTok{,}
  \StringTok{"8 ano"}\NormalTok{,}
  \StringTok{"9 ano"}\NormalTok{,}
  \StringTok{"1 série EM"}\NormalTok{,}
  \StringTok{"2 série EM"}\NormalTok{,}
  \StringTok{"3 série EM"}
\NormalTok{  )}

\CommentTok{\# para explicitar a ordem das categorias nas variáveis}
\CommentTok{\# que medem apenas 2 níveis (levels): s {-} sim / n {-} não}
\CommentTok{\# nessas ordem (e não na ordem alfabética)}
\NormalTok{sim\_n }\OtherTok{\textless{}{-}} \FunctionTok{c}\NormalTok{(}
  \StringTok{"s"}\NormalTok{,}
  \StringTok{"n"}
\NormalTok{  )}

\CommentTok{\# Declaração de Variáveis tipo char já existentes como categóricas}
\NormalTok{obitj\_csv }\OtherTok{\textless{}{-}}\NormalTok{ obitj\_csv }\SpecialCharTok{\%\textgreater{}\%} 
  \FunctionTok{mutate}\NormalTok{(}\AttributeTok{sexo =}                   \CommentTok{\# nova variável tipo \textless{}fctr\textgreater{}}
\NormalTok{           sexo }\SpecialCharTok{\%\textgreater{}\%}               \CommentTok{\# a partir da variável original sexo}
           \FunctionTok{factor}\NormalTok{() }\SpecialCharTok{\%\textgreater{}\%}           \CommentTok{\# converte para o tipo factor}
\NormalTok{           forcats}\SpecialCharTok{::}\FunctionTok{fct\_recode}\NormalTok{(   }\CommentTok{\# forcats função para recodificar labels}
             \StringTok{"F"} \OtherTok{=} \StringTok{"f"}\NormalTok{,      }\CommentTok{\# novo à esquerda, antigo à direita}
             \StringTok{"M"} \OtherTok{=} \StringTok{"m"}\NormalTok{),     }\CommentTok{\# F = Feminino, M = Masculino}
         
         \CommentTok{\# mesma coisa com código mais condensado:}
         \AttributeTok{cor =} \FunctionTok{factor}\NormalTok{(cor), }\CommentTok{\# mantidos os levels originais: branco, pardo, preto}
         
         \CommentTok{\# variável corag = cor agragada em apenas 2 categorias}
         \AttributeTok{corag =} \FunctionTok{factor}\NormalTok{(corag), }\CommentTok{\# mantidos os levels originais: branco, negro}
         
         \CommentTok{\# variável esc1 = escolaridade 1, com 11 categorias}
         \AttributeTok{esc1 =}                  \CommentTok{\# nova variável tipo \textless{}fctr\textgreater{}}
\NormalTok{           esc1 }\SpecialCharTok{|\textgreater{}}               \CommentTok{\# a partir da variável original esc2}
           \FunctionTok{factor}\NormalTok{( series ) }\SpecialCharTok{|\textgreater{}}   \CommentTok{\# converte para o tipo factor: 11/12 categorias}
\NormalTok{           forcats}\SpecialCharTok{::}\FunctionTok{fct\_recode}\NormalTok{(  }\CommentTok{\# forcats função para recodificar labels}
             \StringTok{"1ano"} \OtherTok{=} \StringTok{"1 ano"}\NormalTok{,   }\CommentTok{\# novo à esquerda, antigo à direita}
             \StringTok{"2ano"} \OtherTok{=} \StringTok{"2 ano"}\NormalTok{,}
             \StringTok{"3ano"} \OtherTok{=} \StringTok{"3 ano"}\NormalTok{,   }\CommentTok{\# embora esta categoria ñ ocorra}
             \StringTok{"4ano"} \OtherTok{=} \StringTok{"4 ano"}\NormalTok{,}
             \StringTok{"5ano"} \OtherTok{=} \StringTok{"5 ano"}\NormalTok{,}
             \StringTok{"6ano"} \OtherTok{=} \StringTok{"6 ano"}\NormalTok{,}
             \StringTok{"7ano"} \OtherTok{=} \StringTok{"7 ano"}\NormalTok{,}
             \StringTok{"8ano"} \OtherTok{=} \StringTok{"8 ano"}\NormalTok{,}
             \StringTok{"9ano"} \OtherTok{=} \StringTok{"9 ano"}\NormalTok{,}
             \StringTok{"1serieEM"} \OtherTok{=} \StringTok{"1 série EM"}\NormalTok{,}
             \StringTok{"2serieEM"} \OtherTok{=} \StringTok{"2 série EM"}\NormalTok{,}
             \StringTok{"3serieEM"} \OtherTok{=} \StringTok{"3 série EM"}
\NormalTok{           ), }\CommentTok{\# mantido nenhum label original}
         
         \CommentTok{\# variável esc2 = escolaridade 2, com 10 categorias}
         \AttributeTok{esc2 =}                  \CommentTok{\# nova variável tipo \textless{}fctr\textgreater{}}
\NormalTok{           esc2 }\SpecialCharTok{|\textgreater{}}               \CommentTok{\# a partir da variável original esc2}
           \FunctionTok{factor}\NormalTok{( series ) }\SpecialCharTok{|\textgreater{}}   \CommentTok{\# converte para o tipo factor: 12 categorias}
\NormalTok{           forcats}\SpecialCharTok{::}\FunctionTok{fct\_recode}\NormalTok{(  }\CommentTok{\# forcats função para recodificar labels}
             \StringTok{"1ano"} \OtherTok{=} \StringTok{"1 ano"}\NormalTok{,   }\CommentTok{\# embora esta categoria ñ ocorra}
             \StringTok{"2ano"} \OtherTok{=} \StringTok{"2 ano"}\NormalTok{,   }\CommentTok{\# novo à esquerda, antigo à direita}
             \StringTok{"3ano"} \OtherTok{=} \StringTok{"3 ano"}\NormalTok{,   }\CommentTok{\# categoria que ñ ocorre}
             \StringTok{"4ano"} \OtherTok{=} \StringTok{"4 ano"}\NormalTok{,}
             \StringTok{"5ano"} \OtherTok{=} \StringTok{"5 ano"}\NormalTok{,}
             \StringTok{"6ano"} \OtherTok{=} \StringTok{"6 ano"}\NormalTok{,}
             \StringTok{"7ano"} \OtherTok{=} \StringTok{"7 ano"}\NormalTok{,}
             \StringTok{"8ano"} \OtherTok{=} \StringTok{"8 ano"}\NormalTok{,}
             \StringTok{"9ano"} \OtherTok{=} \StringTok{"9 ano"}\NormalTok{,}
             \StringTok{"1serieEM"} \OtherTok{=} \StringTok{"1 série EM"}\NormalTok{,}
             \StringTok{"2serieEM"} \OtherTok{=} \StringTok{"2 série EM"}\NormalTok{,}
             \StringTok{"3serieEM"} \OtherTok{=} \StringTok{"3 série EM"}
\NormalTok{           ), }\CommentTok{\# mantido nenhum label original}
         
         \CommentTok{\# variável compfam = composição familiar, com 6 categorias}
         \AttributeTok{compfam =}               \CommentTok{\# nova variável tipo \textless{}fctr\textgreater{}}
\NormalTok{           compfam }\SpecialCharTok{|\textgreater{}}            \CommentTok{\# a partir da variável original compfam}
           \FunctionTok{factor}\NormalTok{() }\SpecialCharTok{|\textgreater{}}           \CommentTok{\# converte para o tipo factor: 6 categorias}
\NormalTok{           forcats}\SpecialCharTok{::}\FunctionTok{fct\_recode}\NormalTok{(  }\CommentTok{\# forcats função para recodificar labels}
             \StringTok{"mae"}      \OtherTok{=} \StringTok{"mãe"}\NormalTok{,      }\CommentTok{\# novo à esquerda, antigo à direita}
             \StringTok{"mae\_padr"} \OtherTok{=} \StringTok{"mãe + padrasto"}\NormalTok{,}
             \StringTok{"pai\_mae"}  \OtherTok{=} \StringTok{"pai + mãe"}\NormalTok{,}
             \StringTok{"pai\_madr"} \OtherTok{=} \StringTok{"pai + madrasta"}\NormalTok{,}
\NormalTok{           ), }\CommentTok{\# mantidos só 2 labels originais: pai, parentes}
         
         \CommentTok{\# variável relpai = relação com pai, com 3 categorias}
         \AttributeTok{relpai =}                \CommentTok{\# nova variável tipo \textless{}fctr\textgreater{}}
\NormalTok{           relpai }\SpecialCharTok{|\textgreater{}}             \CommentTok{\# a partir da variável original relpai}
           \FunctionTok{factor}\NormalTok{() }\SpecialCharTok{|\textgreater{}}           \CommentTok{\# converte para o tipo factor: 3 categorias}
\NormalTok{           forcats}\SpecialCharTok{::}\FunctionTok{fct\_recode}\NormalTok{(  }\CommentTok{\# forcats função para recodificar labels}
             \StringTok{"auxilio"}     \OtherTok{=} \StringTok{"auxílio"}\NormalTok{, }\CommentTok{\# novo à esquerda, antigo à direita}
             \StringTok{"mesma\_resid"} \OtherTok{=} \StringTok{"mesma residência"}\NormalTok{,}
             \StringTok{"ausente"}     \OtherTok{=} \StringTok{"ausente"}
\NormalTok{           ), }\CommentTok{\# mantidos só um label original: ausente}

         \CommentTok{\# variável usudrog = usuário de droga, com 2 categorias: s / n}
         \AttributeTok{usudrog =}                 \CommentTok{\# nova variável tipo \textless{}fctr\textgreater{}}
\NormalTok{           usudrog }\SpecialCharTok{|\textgreater{}}              \CommentTok{\# a partir da variável original usudrog}
           \FunctionTok{factor}\NormalTok{( sim\_n ),        }\CommentTok{\# converte para o tipo factor: 2 categorias}
           
        \CommentTok{\# variável subst = Substância entorpecente, com 4 categorias}
         \AttributeTok{subst =}                 \CommentTok{\# nova variável tipo \textless{}fctr\textgreater{}}
\NormalTok{           subst }\SpecialCharTok{\%\textgreater{}\%}             \CommentTok{\# a partir da variável original subst}
           \FunctionTok{factor}\NormalTok{() }\SpecialCharTok{\%\textgreater{}\%}          \CommentTok{\# converte para o tipo factor: 4 categorias}
\NormalTok{           forcats}\SpecialCharTok{::}\FunctionTok{fct\_recode}\NormalTok{(  }\CommentTok{\# forcats função para recodificar labels}
             \StringTok{"coca\_crack"}  \OtherTok{=} \StringTok{"cocaína / crack"}\NormalTok{, }\CommentTok{\# novo à esquerda, antigo à direita}
             \StringTok{"lsd\_ecstasy"} \OtherTok{=} \StringTok{"lsd, ecstasy"}\NormalTok{,}
             \StringTok{"licita"}      \OtherTok{=} \StringTok{"lícitas"}
\NormalTok{           ), }\CommentTok{\# mantidos só um level original: maconha}
         
         \CommentTok{\# variável orgcrim = organização criminosa, com 2 categorias: s / n}
         \AttributeTok{orgcrim =}                 \CommentTok{\# nova variável tipo \textless{}fctr\textgreater{}}
\NormalTok{           orgcrim }\SpecialCharTok{|\textgreater{}}              \CommentTok{\# a partir da variável original orgcrim}
           \FunctionTok{factor}\NormalTok{( sim\_n ),        }\CommentTok{\# converte para o tipo factor: 2 categorias}
         
         \CommentTok{\# variável morte = tipo de morte, com 2 categorias: nat / viol}
         \AttributeTok{morte =} \FunctionTok{factor}\NormalTok{(morte),    }\CommentTok{\# mantida ordem dos 2 levels originais}
         
         \CommentTok{\# variável paf = morte por perfuração de arma de fogo, com 2 cat: s / n}
         \AttributeTok{paf =}                     \CommentTok{\# nova variável tipo \textless{}fctr\textgreater{}}
\NormalTok{           paf }\SpecialCharTok{|\textgreater{}}                  \CommentTok{\# a partir da variável original paf}
           \FunctionTok{factor}\NormalTok{( sim\_n ),        }\CommentTok{\# converte para o tipo factor: 2 categorias}
         
         \CommentTok{\# variável circobt = circunstância do óbito, com 5 categorias}
         \AttributeTok{circobt =}                \CommentTok{\# nova variável tipo \textless{}fctr\textgreater{}}
\NormalTok{           circobt }\SpecialCharTok{\%\textgreater{}\%}            \CommentTok{\# a partir da variável original circobt}
           \FunctionTok{factor}\NormalTok{() }\SpecialCharTok{\%\textgreater{}\%}           \CommentTok{\# converte para o tipo factor: 5 categorias}
\NormalTok{           forcats}\SpecialCharTok{::}\FunctionTok{fct\_recode}\NormalTok{(   }\CommentTok{\# forcats função para recodificar labels}
             \StringTok{"MDIP"} \OtherTok{=} \StringTok{"intervenção policial"}\NormalTok{, }\CommentTok{\# novo à esquerda, antigo à direita}
             \CommentTok{\# MDIP = Morte Decorrente Intervenção Policial}
             \StringTok{"MDCC"} \OtherTok{=} \StringTok{"conflitos entre criminalidade"}\NormalTok{,}
             \CommentTok{\# MDIP = Morte Decorrente Conflitos entre Criminalidade}
             \StringTok{"transito"}      \OtherTok{=} \StringTok{"trânsito"}\NormalTok{, }
             \StringTok{"outros"}        \OtherTok{=} \StringTok{"Outros"}\NormalTok{,}
             \StringTok{"conf\_fam\_afet"} \OtherTok{=} \StringTok{"conflito familiar / afetivo"}
\NormalTok{             )}
\NormalTok{         )}

\NormalTok{obitj\_csv }\SpecialCharTok{|\textgreater{}} 
  \FunctionTok{head}\NormalTok{(}\DecValTok{25}\NormalTok{)}
\InformationTok{\textasciigrave{}\textasciigrave{}\textasciigrave{}}
\end{Highlighting}
\end{Shaded}

\begin{verbatim}
# A tibble: 25 x 24
   nomean     maean    nasc       sexo  cor   corag dom   dataesc1   esc1  esc2 
   <chr>      <chr>    <date>     <fct> <fct> <fct> <chr> <date>     <fct> <fct>
 1 A.J.D.S.A. L.M.D.S~ 2001-04-21 M     bran~ bran~ Bair~ 2018-09-28 6ano  <NA> 
 2 A.G.M.B.   A.D.S.M~ 2000-09-04 M     pardo negro <NA>  NA         <NA>  8ano 
 3 A.G.D.S.   F.A.G.   1997-06-22 M     <NA>  <NA>  <NA>  NA         <NA>  9ano 
 4 A.J.D.S.S. S.D.S.S. 2002-05-01 M     pardo negro Vila~ 2020-02-09 1ser~ 1ser~
 5 A.A.D.A.   A.A.D.A. 1999-10-13 M     <NA>  <NA>  <NA>  NA         <NA>  8ano 
 6 A.E.M.     M.E.M.   2001-03-25 M     bran~ bran~ <NA>  2018-10-23 8ano  8ano 
 7 A.C.A.     J.A.D.A. 2000-12-18 M     pardo negro Seto~ 2018-11-18 8ano  8ano 
 8 A.O.       D.G.B.S. 2003-09-30 M     bran~ bran~ Bair~ 2019-10-01 8ano  8ano 
 9 A.R.C.     M.D.R.D~ 2001-02-24 M     pardo negro <NA>  NA         <NA>  <NA> 
10 A.O.S.     K.R.S.   2003-02-06 M     pardo negro Parq~ 2018-07-17 9ano  8ano 
# i 15 more rows
# i 14 more variables: compfam <fct>, relpai <fct>, usudrog <fct>, subst <fct>,
#   orgcrim <fct>, sitdiv <chr>, dataobt <date>, morte <fct>, paf <fct>,
#   circobt <fct>, obsobt <chr>, idadeobtd <dbl>, idadeobta <dbl>,
#   npassag <dbl>
\end{verbatim}

\subsection{Inspecionar}\label{inspecionar}

Uma rápida inspeção em \texttt{esc1} e \texttt{esc2}.

Por meio de uma contagem das categorias presentes em \texttt{esc1}, que
corresponde à escolaridade do adolescente na data da primeira passagem
pela DePAI, resumidas em uma \textbf{tabela}.

\begin{Shaded}
\begin{Highlighting}[numbers=left,,]
\InformationTok{\textasciigrave{}\textasciigrave{}\textasciigrave{}\{r\}}
\NormalTok{obitj\_csv }\SpecialCharTok{|\textgreater{}} 
  \FunctionTok{count}\NormalTok{(esc1)}
\InformationTok{\textasciigrave{}\textasciigrave{}\textasciigrave{}}
\end{Highlighting}
\end{Shaded}

\begin{verbatim}
# A tibble: 12 x 2
   esc1         n
   <fct>    <int>
 1 1ano         1
 2 2ano         1
 3 4ano         6
 4 5ano         9
 5 6ano        43
 6 7ano        55
 7 8ano        80
 8 9ano        92
 9 1serieEM    81
10 2serieEM    26
11 3serieEM     7
12 <NA>        46
\end{verbatim}

Constata-se que a categoria \texttt{3ano} não ocorreu nos dados
coletados para \texttt{esc1}.

A mesma tabela em um formato mais adequado para impressão em
\texttt{.pdf}.

\begin{Shaded}
\begin{Highlighting}[numbers=left,,]
\InformationTok{\textasciigrave{}\textasciigrave{}\textasciigrave{}\{r\}}
\FunctionTok{library}\NormalTok{(gt)}

\NormalTok{tab.esc1 }\OtherTok{\textless{}{-}}\NormalTok{ obitj\_csv }\SpecialCharTok{|\textgreater{}} 
  \FunctionTok{count}\NormalTok{(esc1) }\SpecialCharTok{|\textgreater{}} 
  \FunctionTok{mutate}\NormalTok{(}\AttributeTok{p =}\NormalTok{ n }\SpecialCharTok{/} \FunctionTok{sum}\NormalTok{(n) }\SpecialCharTok{*} \DecValTok{100}\NormalTok{) }\SpecialCharTok{|\textgreater{}} 
  \FunctionTok{mutate}\NormalTok{(}\AttributeTok{p =} \FunctionTok{round}\NormalTok{(p, }\DecValTok{1}\NormalTok{) )}

\NormalTok{actual\_colnames }\OtherTok{\textless{}{-}} \FunctionTok{colnames}\NormalTok{(tab.esc1) }\CommentTok{\# [{-}1]}
\CommentTok{\# actual\_colnames}
\DocumentationTok{\#\# [1] "esc1" "n"    "p"}

\NormalTok{spanners\_and\_header }\OtherTok{\textless{}{-}} \ControlFlowTok{function}\NormalTok{(gt\_tbl) \{}
\NormalTok{  gt\_tbl }\SpecialCharTok{|\textgreater{}} 
    \FunctionTok{cols\_label}\NormalTok{(}
    \AttributeTok{esc1 =} \StringTok{"Escolaridade na data 1ª passagem"}\NormalTok{,}
    \AttributeTok{n    =} \StringTok{"Frequência Absoluta"}\NormalTok{,}
    \AttributeTok{p    =} \StringTok{"Em relação ao total"}
\NormalTok{    ) }\SpecialCharTok{|\textgreater{}} 
    \FunctionTok{tab\_spanner}\NormalTok{(}
      \AttributeTok{label   =} \FunctionTok{md}\NormalTok{(}\StringTok{"**2016{-}2023**"}\NormalTok{),}
      \AttributeTok{columns =} \DecValTok{1}
\NormalTok{    ) }\SpecialCharTok{|\textgreater{}}
    \FunctionTok{tab\_spanner}\NormalTok{(}
      \AttributeTok{label   =} \FunctionTok{md}\NormalTok{(}\StringTok{"**Contagem por séries**"}\NormalTok{),}
      \AttributeTok{columns =} \FunctionTok{c}\NormalTok{(}\DecValTok{2}\NormalTok{)}
\NormalTok{    ) }\SpecialCharTok{|\textgreater{}} 
    \FunctionTok{tab\_spanner}\NormalTok{(}
      \AttributeTok{label   =} \FunctionTok{md}\NormalTok{(}\StringTok{"**Proporção percentual (\%)**"}\NormalTok{),}
      \AttributeTok{columns =} \FunctionTok{c}\NormalTok{(}\DecValTok{3}\NormalTok{)}
\NormalTok{    ) }\SpecialCharTok{|\textgreater{}} 
    \FunctionTok{tab\_header}\NormalTok{(}
      \AttributeTok{title =} \StringTok{"Goiânia (DePAI): Escolaridade de jovens em conflito com a lei"}\NormalTok{,}
      \AttributeTok{subtitle =} \StringTok{"Na data da 1ª passagem pela Delegacia de Apuração de Atos Infracionais"}
\NormalTok{    ) }
\NormalTok{\}}

\NormalTok{tab.esc1 }\SpecialCharTok{|\textgreater{}} 
  \FunctionTok{gt}\NormalTok{() }\SpecialCharTok{|\textgreater{}} 
  \CommentTok{\# cols\_label(.list = desired\_colnames) |\textgreater{} }
  \FunctionTok{spanners\_and\_header}\NormalTok{()}
\InformationTok{\textasciigrave{}\textasciigrave{}\textasciigrave{}}
\end{Highlighting}
\end{Shaded}

\begin{table}
\caption*{
{\large Goiânia (DePAI): Escolaridade de jovens em conflito com a lei} \\ 
{\small Na data da 1ª passagem pela Delegacia de Apuração de Atos Infracionais}
} 
\fontsize{12.0pt}{14.4pt}\selectfont
\begin{tabular*}{\linewidth}{@{\extracolsep{\fill}}crr}
\toprule
\textbf{2016-2023} & \textbf{Contagem por séries} & \textbf{Proporção percentual (\%)} \\ 
\cmidrule(lr){1-1} \cmidrule(lr){2-2} \cmidrule(lr){3-3}
Escolaridade na data 1ª passagem & Frequência Absoluta & Em relação ao total \\ 
\midrule\addlinespace[2.5pt]
1ano & 1 & 0.2 \\ 
2ano & 1 & 0.2 \\ 
4ano & 6 & 1.3 \\ 
5ano & 9 & 2.0 \\ 
6ano & 43 & 9.6 \\ 
7ano & 55 & 12.3 \\ 
8ano & 80 & 17.9 \\ 
9ano & 92 & 20.6 \\ 
1serieEM & 81 & 18.1 \\ 
2serieEM & 26 & 5.8 \\ 
3serieEM & 7 & 1.6 \\ 
NA & 46 & 10.3 \\ 
\bottomrule
\end{tabular*}
\end{table}

Agora a mesma tabela acima é reproduzida para a \textbf{\emph{variável
categórica}} \texttt{esc2}, que faz a contagem dos jovens em conflito
com a lei (com passagem na DePAI), na data do seu respectivo óbito.

\begin{Shaded}
\begin{Highlighting}[numbers=left,,]
\InformationTok{\textasciigrave{}\textasciigrave{}\textasciigrave{}\{r\}}
\NormalTok{tab.esc2 }\OtherTok{\textless{}{-}}\NormalTok{ obitj\_csv }\SpecialCharTok{|\textgreater{}} 
  \FunctionTok{count}\NormalTok{(esc2) }\SpecialCharTok{|\textgreater{}} 
  \FunctionTok{mutate}\NormalTok{(}\AttributeTok{p =}\NormalTok{ n }\SpecialCharTok{/} \FunctionTok{sum}\NormalTok{(n) }\SpecialCharTok{*} \DecValTok{100}\NormalTok{) }\SpecialCharTok{|\textgreater{}} 
  \FunctionTok{mutate}\NormalTok{(}\AttributeTok{p =} \FunctionTok{round}\NormalTok{(p, }\DecValTok{1}\NormalTok{) )}

\NormalTok{actual\_colnames }\OtherTok{\textless{}{-}} \FunctionTok{colnames}\NormalTok{(tab.esc2) }\CommentTok{\# [{-}1]}

\CommentTok{\# sum(tab.esc2$p) \# 100.00\%}

\NormalTok{spanners\_and\_header }\OtherTok{\textless{}{-}} \ControlFlowTok{function}\NormalTok{(gt\_tbl) \{}
\NormalTok{  gt\_tbl }\SpecialCharTok{|\textgreater{}} 
    \FunctionTok{cols\_label}\NormalTok{(}
    \AttributeTok{esc2 =} \StringTok{"Escolaridade na data do óbito"}\NormalTok{,}
    \AttributeTok{n    =} \StringTok{"Frequência Absoluta"}\NormalTok{,}
    \AttributeTok{p    =} \StringTok{"Em relação ao total"}
\NormalTok{    ) }\SpecialCharTok{|\textgreater{}} 
    \FunctionTok{tab\_spanner}\NormalTok{(}
      \AttributeTok{label   =} \FunctionTok{md}\NormalTok{(}\StringTok{"**2016{-}2023**"}\NormalTok{),}
      \AttributeTok{columns =} \DecValTok{1}
\NormalTok{    ) }\SpecialCharTok{|\textgreater{}}
    \FunctionTok{tab\_spanner}\NormalTok{(}
      \AttributeTok{label   =} \FunctionTok{md}\NormalTok{(}\StringTok{"**Contagem por séries**"}\NormalTok{),}
      \AttributeTok{columns =} \FunctionTok{c}\NormalTok{(}\DecValTok{2}\NormalTok{)}
\NormalTok{    ) }\SpecialCharTok{|\textgreater{}} 
    \FunctionTok{tab\_spanner}\NormalTok{(}
      \AttributeTok{label   =} \FunctionTok{md}\NormalTok{(}\StringTok{"**Proporção percentual (\%)**"}\NormalTok{),}
      \AttributeTok{columns =} \FunctionTok{c}\NormalTok{(}\DecValTok{3}\NormalTok{)}
\NormalTok{    ) }\SpecialCharTok{|\textgreater{}} 
    \FunctionTok{tab\_header}\NormalTok{(}
      \AttributeTok{title =} \StringTok{"Goiânia (DePAI): Escolaridade de jovens em conflito com a lei"}\NormalTok{,}
      \AttributeTok{subtitle =} \StringTok{"Com passagem pela DePAI {-} Delegacia de Apuração de Atos Infracionais, na data do respectivo óbito"}
\NormalTok{    ) }
\NormalTok{\}}

\NormalTok{tab.esc2 }\SpecialCharTok{|\textgreater{}} 
  \FunctionTok{gt}\NormalTok{() }\SpecialCharTok{|\textgreater{}} 
  \CommentTok{\# cols\_label(.list = desired\_colnames) |\textgreater{} }
  \FunctionTok{spanners\_and\_header}\NormalTok{()}
\InformationTok{\textasciigrave{}\textasciigrave{}\textasciigrave{}}
\end{Highlighting}
\end{Shaded}

\begin{table}
\caption*{
{\large Goiânia (DePAI): Escolaridade de jovens em conflito com a lei} \\ 
{\small Com passagem pela DePAI - Delegacia de Apuração de Atos Infracionais, na data do respectivo óbito}
} 
\fontsize{12.0pt}{14.4pt}\selectfont
\begin{tabular*}{\linewidth}{@{\extracolsep{\fill}}crr}
\toprule
\textbf{2016-2023} & \textbf{Contagem por séries} & \textbf{Proporção percentual (\%)} \\ 
\cmidrule(lr){1-1} \cmidrule(lr){2-2} \cmidrule(lr){3-3}
Escolaridade na data do óbito & Frequência Absoluta & Em relação ao total \\ 
\midrule\addlinespace[2.5pt]
2ano & 2 & 0.4 \\ 
4ano & 7 & 1.6 \\ 
5ano & 9 & 2.0 \\ 
6ano & 33 & 7.4 \\ 
7ano & 53 & 11.9 \\ 
8ano & 81 & 18.1 \\ 
9ano & 62 & 13.9 \\ 
1serieEM & 103 & 23.0 \\ 
2serieEM & 36 & 8.1 \\ 
3serieEM & 21 & 4.7 \\ 
NA & 40 & 8.9 \\ 
\bottomrule
\end{tabular*}
\end{table}

Agora por meio de um \textbf{gráfico} de \textbf{Colunas} da variável
categórica: \texttt{esc2}, que corresponde à escolaridade do adolescente
na data do seu óbito.

\begin{Shaded}
\begin{Highlighting}[numbers=left,,]
\InformationTok{\textasciigrave{}\textasciigrave{}\textasciigrave{}\{r\}}
\CommentTok{\# uma primeira inspeção rápida}
\NormalTok{obitj\_csv }\SpecialCharTok{|\textgreater{}} 
  \FunctionTok{ggplot}\NormalTok{( }\FunctionTok{aes}\NormalTok{( esc2 ) ) }\SpecialCharTok{+}
  \FunctionTok{geom\_bar}\NormalTok{() }\CommentTok{\# orientation = "x"}
\InformationTok{\textasciigrave{}\textasciigrave{}\textasciigrave{}}
\end{Highlighting}
\end{Shaded}

\pandocbounded{\includegraphics[keepaspectratio]{cap3-DistNorm_files/figure-pdf/unnamed-chunk-6-1.pdf}}

Inverter os eixos x e y cima e também a ordem da variável categórica
\texttt{esc2} plotada no eixo y.

Para obter um \textbf{gráfico} de \textbf{Barras}.

\begin{Shaded}
\begin{Highlighting}[numbers=left,,]
\InformationTok{\textasciigrave{}\textasciigrave{}\textasciigrave{}\{r\}}
\NormalTok{obitj\_csv }\SpecialCharTok{|\textgreater{}} 
  \FunctionTok{ggplot}\NormalTok{( }\FunctionTok{aes}\NormalTok{( }\AttributeTok{y =}\NormalTok{ forcats}\SpecialCharTok{::}\FunctionTok{fct\_rev}\NormalTok{(esc2) ) ) }\SpecialCharTok{+} \CommentTok{\# fct\_rev() reverte a ordem das categorias}
  \FunctionTok{geom\_bar}\NormalTok{() }\CommentTok{\# orientação barras horizontais}
\InformationTok{\textasciigrave{}\textasciigrave{}\textasciigrave{}}
\end{Highlighting}
\end{Shaded}

\pandocbounded{\includegraphics[keepaspectratio]{cap3-DistNorm_files/figure-pdf/unnamed-chunk-7-1.pdf}}

Que, neste caso, é \textbf{\emph{mais legível}} que o gráfico de
colunas.

Constata-se que as categorias \texttt{1ano} e \texttt{3ano} não
ocorreram nos dados coletados para \texttt{esc2}.

\subsection{Curva Normal}\label{curva-normal}

Inspecionar a variável \textbf{numérica contínua} \textbf{\emph{idade na
data do óbito}}, expressa em anos (com uma casa decimal), dos jovens em
conflito com a lei: \texttt{idadeobta}.

\begin{Shaded}
\begin{Highlighting}[numbers=left,,]
\InformationTok{\textasciigrave{}\textasciigrave{}\textasciigrave{}\{r\}}
\FunctionTok{names}\NormalTok{(obitj\_csv) }\CommentTok{\# nomes de todas as variáveis do data set}

\NormalTok{obitj\_csv }\CommentTok{\# exibe o data set (conjunto de dados)}

\FunctionTok{levels}\NormalTok{(obitj\_csv}\SpecialCharTok{$}\NormalTok{circobt) }\CommentTok{\# nomes das categorias em ordem alfabética}
\InformationTok{\textasciigrave{}\textasciigrave{}\textasciigrave{}}
\end{Highlighting}
\end{Shaded}

\begin{verbatim}
 [1] "nomean"    "maean"     "nasc"      "sexo"      "cor"       "corag"    
 [7] "dom"       "dataesc1"  "esc1"      "esc2"      "compfam"   "relpai"   
[13] "usudrog"   "subst"     "orgcrim"   "sitdiv"    "dataobt"   "morte"    
[19] "paf"       "circobt"   "obsobt"    "idadeobtd" "idadeobta" "npassag"  
# A tibble: 447 x 24
   nomean     maean    nasc       sexo  cor   corag dom   dataesc1   esc1  esc2 
   <chr>      <chr>    <date>     <fct> <fct> <fct> <chr> <date>     <fct> <fct>
 1 A.J.D.S.A. L.M.D.S~ 2001-04-21 M     bran~ bran~ Bair~ 2018-09-28 6ano  <NA> 
 2 A.G.M.B.   A.D.S.M~ 2000-09-04 M     pardo negro <NA>  NA         <NA>  8ano 
 3 A.G.D.S.   F.A.G.   1997-06-22 M     <NA>  <NA>  <NA>  NA         <NA>  9ano 
 4 A.J.D.S.S. S.D.S.S. 2002-05-01 M     pardo negro Vila~ 2020-02-09 1ser~ 1ser~
 5 A.A.D.A.   A.A.D.A. 1999-10-13 M     <NA>  <NA>  <NA>  NA         <NA>  8ano 
 6 A.E.M.     M.E.M.   2001-03-25 M     bran~ bran~ <NA>  2018-10-23 8ano  8ano 
 7 A.C.A.     J.A.D.A. 2000-12-18 M     pardo negro Seto~ 2018-11-18 8ano  8ano 
 8 A.O.       D.G.B.S. 2003-09-30 M     bran~ bran~ Bair~ 2019-10-01 8ano  8ano 
 9 A.R.C.     M.D.R.D~ 2001-02-24 M     pardo negro <NA>  NA         <NA>  <NA> 
10 A.O.S.     K.R.S.   2003-02-06 M     pardo negro Parq~ 2018-07-17 9ano  8ano 
# i 437 more rows
# i 14 more variables: compfam <fct>, relpai <fct>, usudrog <fct>, subst <fct>,
#   orgcrim <fct>, sitdiv <chr>, dataobt <date>, morte <fct>, paf <fct>,
#   circobt <fct>, obsobt <chr>, idadeobtd <dbl>, idadeobta <dbl>,
#   npassag <dbl>
[1] "conf_fam_afet" "MDCC"          "MDIP"          "outros"       
[5] "transito"     
\end{verbatim}

Primeiro um histograma da variável idade na data do óbito em anos.

\begin{Shaded}
\begin{Highlighting}[numbers=left,,]
\InformationTok{\textasciigrave{}\textasciigrave{}\textasciigrave{}\{r\}}
\NormalTok{obitj }\OtherTok{=}\NormalTok{ obitj\_csv}

\NormalTok{media }\OtherTok{\textless{}{-}} \FunctionTok{mean}\NormalTok{(obitj}\SpecialCharTok{$}\NormalTok{idadeobta, }\AttributeTok{na.rm =} \ConstantTok{TRUE}\NormalTok{) }\SpecialCharTok{|\textgreater{}} 
  \FunctionTok{round}\NormalTok{(}\DecValTok{1}\NormalTok{)}

\FunctionTok{ggplot}\NormalTok{(}\AttributeTok{data =} \ConstantTok{NULL}\NormalTok{,}
       \FunctionTok{aes}\NormalTok{(}\AttributeTok{x =}\NormalTok{ obitj}\SpecialCharTok{$}\NormalTok{idadeobta)) }\SpecialCharTok{+}
  \FunctionTok{geom\_histogram}\NormalTok{(}\AttributeTok{binwidth =} \DecValTok{1}\NormalTok{, }\AttributeTok{fill =} \StringTok{"white"}\NormalTok{, }\AttributeTok{color =} \StringTok{"gray80"}\NormalTok{) }\SpecialCharTok{+}
  \FunctionTok{scale\_x\_continuous}\NormalTok{(}\AttributeTok{breaks =} \FunctionTok{seq}\NormalTok{(}\SpecialCharTok{{-}}\DecValTok{5}\NormalTok{, }\DecValTok{35}\NormalTok{, }\DecValTok{5}\NormalTok{)) }\SpecialCharTok{+}
  \FunctionTok{stat\_bin}\NormalTok{(}\AttributeTok{binwidth  =} \DecValTok{1}\NormalTok{,}
           \AttributeTok{geom      =} \StringTok{"text"}\NormalTok{,}
           \FunctionTok{aes}\NormalTok{(}\AttributeTok{label =}\NormalTok{ ..count..),}
           \AttributeTok{vjust     =} \SpecialCharTok{{-}}\FloatTok{0.3}\NormalTok{,}
           \AttributeTok{size      =} \FloatTok{3.5}\NormalTok{) }\SpecialCharTok{+}
  \FunctionTok{stat\_bin}\NormalTok{(}
    \AttributeTok{binwidth =} \DecValTok{1}\NormalTok{, }\AttributeTok{geom =} \StringTok{"text"}\NormalTok{, }\AttributeTok{color =} \StringTok{"black"}\NormalTok{, }\AttributeTok{cex =} \FloatTok{2.2}\NormalTok{,}
    \FunctionTok{aes}\NormalTok{(}\AttributeTok{y =} \FunctionTok{after\_stat}\NormalTok{(count }\SpecialCharTok{/} \FunctionTok{sum}\NormalTok{(count)), }
        \AttributeTok{label =}\NormalTok{ scales}\SpecialCharTok{::}\FunctionTok{percent}\NormalTok{(}\FunctionTok{round}\NormalTok{(}\FunctionTok{after\_stat}\NormalTok{(count }\SpecialCharTok{/} \FunctionTok{sum}\NormalTok{(count)), }\DecValTok{3}\NormalTok{))) ,}
    \AttributeTok{position =} \FunctionTok{position\_stack}\NormalTok{(}\AttributeTok{vjust =} \FloatTok{350.0}\NormalTok{)}
\NormalTok{  ) }\SpecialCharTok{+}
  \FunctionTok{geom\_vline}\NormalTok{(}\AttributeTok{xintercept =} \FunctionTok{mean}\NormalTok{(obitj}\SpecialCharTok{$}\NormalTok{idadeobta,  }\AttributeTok{na.rm =} \ConstantTok{TRUE}\NormalTok{),}
             \AttributeTok{color =} \StringTok{"red"}\NormalTok{, }\AttributeTok{size =} \FloatTok{0.4}\NormalTok{ , }\AttributeTok{linetype =} \StringTok{"dotdash"}\NormalTok{, }\AttributeTok{alpha =} \FloatTok{0.5}\NormalTok{) }\SpecialCharTok{+}
  \FunctionTok{labs}\NormalTok{(}\AttributeTok{title    =} \StringTok{"Histograma da idade na data do óbito (N = 449 obs.)"}\NormalTok{,}
       \AttributeTok{subtitle =} \StringTok{"Jovens c/passagem na DEPAI {-} Goiânia (2016{-}2023)"}\NormalTok{,}
       \AttributeTok{y        =} \StringTok{"Frequencia Absoluta (n) e Relativa (\%)"}\NormalTok{,}
       \AttributeTok{x        =} \FunctionTok{paste0}\NormalTok{(}\StringTok{"Idade na data do óbito (média em vermelho = "}\NormalTok{,}
\NormalTok{                         media, }\StringTok{" anos)"}\NormalTok{),}
       \AttributeTok{caption  =} \StringTok{"Fonte: dados primários levantados por Queops (2024)."}\NormalTok{)}
\InformationTok{\textasciigrave{}\textasciigrave{}\textasciigrave{}}
\end{Highlighting}
\end{Shaded}

\pandocbounded{\includegraphics[keepaspectratio]{cap3-DistNorm_files/figure-pdf/unnamed-chunk-9-1.pdf}}

Um boxplot para análise esploratória descritiva da mesma variável
numérica.

\begin{Shaded}
\begin{Highlighting}[numbers=left,,]
\InformationTok{\textasciigrave{}\textasciigrave{}\textasciigrave{}\{r\}}
\NormalTok{obitj }\SpecialCharTok{|\textgreater{}} 
  \FunctionTok{ggplot}\NormalTok{( }\FunctionTok{aes}\NormalTok{(}\AttributeTok{x =}\NormalTok{ idadeobta) ) }\SpecialCharTok{+}
  \FunctionTok{geom\_boxplot}\NormalTok{() }\SpecialCharTok{+}
  \FunctionTok{scale\_x\_continuous}\NormalTok{(}\AttributeTok{breaks =} \FunctionTok{seq}\NormalTok{(}\SpecialCharTok{{-}}\DecValTok{5}\NormalTok{, }\DecValTok{35}\NormalTok{, }\DecValTok{5}\NormalTok{)) }\SpecialCharTok{+}
  \FunctionTok{geom\_vline}\NormalTok{(}\AttributeTok{xintercept =} \FunctionTok{mean}\NormalTok{(obitj}\SpecialCharTok{$}\NormalTok{idadeobta, }\AttributeTok{na.rm =} \ConstantTok{TRUE}\NormalTok{),}
             \AttributeTok{color =} \StringTok{"red"}\NormalTok{, }\AttributeTok{size =} \FloatTok{0.7}\NormalTok{ , }\AttributeTok{linetype =} \StringTok{"dotdash"}\NormalTok{) }\SpecialCharTok{+}
  \FunctionTok{labs}\NormalTok{(}
       \AttributeTok{title =} \StringTok{"Boxplot da idade na data do óbito (N = 449 obs.) }\SpecialCharTok{\textbackslash{}n}\StringTok{Jovens c/passagem na DEPAI {-} Goiânia (2016{-}2023)"}\NormalTok{,}
       \AttributeTok{y =} \StringTok{""}\NormalTok{,}
       \AttributeTok{x =} \FunctionTok{paste0}\NormalTok{(}\StringTok{"Idade na data do óbito (média em vermelho = "}\NormalTok{, media, }\StringTok{" anos)"}\NormalTok{)}
\NormalTok{       )}
\InformationTok{\textasciigrave{}\textasciigrave{}\textasciigrave{}}
\end{Highlighting}
\end{Shaded}

\pandocbounded{\includegraphics[keepaspectratio]{cap3-DistNorm_files/figure-pdf/unnamed-chunk-10-1.pdf}}

Gráfico de barras lado a lado ou de barras empilhadas para verificar uma
possível relação entre as variáveis categóricas: \texttt{morte}
(nat/viol) e \texttt{usudrog} (s/n).

\begin{Shaded}
\begin{Highlighting}[numbers=left,,]
\InformationTok{\textasciigrave{}\textasciigrave{}\textasciigrave{}\{r\}}
\CommentTok{\#| label: fig{-}plot{-}barras{-}lado{-}a{-}lado{-}morte{-}usudrog}
\CommentTok{\#| warning: false}
\CommentTok{\#| fig{-}cap: "Gráfico de Barras Lado a Lado: Tipo de morte dos jovens que vieram a óbito (natural/violenta) segundo uso de drogas (sim/não)\textbackslash{}ncom passagem na DEPAI de Goiânia\textbackslash{}nPeríodo: 2016 a 2023 (N = 447)."}

\NormalTok{obitj }\SpecialCharTok{|\textgreater{}} 
  \FunctionTok{select}\NormalTok{(morte, usudrog) }\SpecialCharTok{|\textgreater{}} 
  \FunctionTok{filter}\NormalTok{(}\SpecialCharTok{!}\FunctionTok{is.na}\NormalTok{(morte)) }\SpecialCharTok{|\textgreater{}} 
  \FunctionTok{filter}\NormalTok{(}\SpecialCharTok{!}\FunctionTok{is.na}\NormalTok{(usudrog)) }\SpecialCharTok{|\textgreater{}} 
  \FunctionTok{group\_by}\NormalTok{(morte, usudrog) }\SpecialCharTok{|\textgreater{}} 
\NormalTok{  dplyr}\SpecialCharTok{::}\FunctionTok{summarize}\NormalTok{(}\AttributeTok{n =}\NormalTok{ dplyr}\SpecialCharTok{::}\FunctionTok{n}\NormalTok{()) }\SpecialCharTok{|\textgreater{}} 
  \FunctionTok{ggplot}\NormalTok{(}\FunctionTok{aes}\NormalTok{(}\AttributeTok{x    =}\NormalTok{ morte, }\AttributeTok{y =}\NormalTok{ n,}
             \AttributeTok{fill =}\NormalTok{ usudrog, }\AttributeTok{group =}\NormalTok{ usudrog)) }\SpecialCharTok{+}
  \FunctionTok{scale\_y\_continuous}\NormalTok{(}\AttributeTok{limits =} \FunctionTok{c}\NormalTok{(}\DecValTok{0}\NormalTok{, }\DecValTok{300}\NormalTok{), }\AttributeTok{breaks =} \FunctionTok{seq}\NormalTok{(}\DecValTok{0}\NormalTok{, }\DecValTok{300}\NormalTok{, }\DecValTok{50}\NormalTok{)) }\SpecialCharTok{+}
  \FunctionTok{geom\_bar}\NormalTok{(}\AttributeTok{stat =} \StringTok{"identity"}\NormalTok{, }\AttributeTok{binwidth =} \DecValTok{1}\NormalTok{,}
           \AttributeTok{position =} \StringTok{"dodge"}\NormalTok{,}
           \AttributeTok{color =} \StringTok{"black"}\NormalTok{, }
           \AttributeTok{na.rm =} \ConstantTok{TRUE}\NormalTok{) }\SpecialCharTok{+}
  \FunctionTok{scale\_fill\_grey}\NormalTok{(}\AttributeTok{start =} \FloatTok{0.0}\NormalTok{, }\AttributeTok{end =} \FloatTok{0.7}\NormalTok{) }\SpecialCharTok{+}
  \FunctionTok{geom\_text}\NormalTok{(}\FunctionTok{aes}\NormalTok{(}\AttributeTok{label =}\NormalTok{ n), }
            \AttributeTok{position =} \FunctionTok{position\_dodge}\NormalTok{(}\AttributeTok{width =} \FloatTok{0.9}\NormalTok{), }\AttributeTok{vjust =} \SpecialCharTok{{-}}\FloatTok{0.25}\NormalTok{) }\SpecialCharTok{+}
  \FunctionTok{geom\_text}\NormalTok{(}\FunctionTok{aes}\NormalTok{(}\AttributeTok{y =}\NormalTok{ n }\SpecialCharTok{{-}} \FloatTok{0.5}\NormalTok{,}
                \AttributeTok{label =}\NormalTok{ scales}\SpecialCharTok{::}\FunctionTok{percent}\NormalTok{(}\FunctionTok{round}\NormalTok{(n}\SpecialCharTok{/}\FunctionTok{sum}\NormalTok{(n), }\DecValTok{3}\NormalTok{))),}
            \AttributeTok{color =} \StringTok{"white"}\NormalTok{,}
            \AttributeTok{position =} \FunctionTok{position\_dodge}\NormalTok{(}\AttributeTok{width =} \FloatTok{0.9}\NormalTok{), }\AttributeTok{vjust =} \SpecialCharTok{+}\FloatTok{1.0}\NormalTok{,}
            \AttributeTok{size =} \DecValTok{3}\NormalTok{) }\SpecialCharTok{+}
  \FunctionTok{labs}\NormalTok{(}\AttributeTok{title    =} \StringTok{"Gráfico de Barras Lado a Lado: tipo de morte"}\NormalTok{,}
       \AttributeTok{subtitle =} \StringTok{"Quantidade por tipo de morte natual ou violenta (N = 399)}\SpecialCharTok{\textbackslash{}n}\StringTok{segundo usuário de droga (n = não / s = sim)."}\NormalTok{,}
       \AttributeTok{y        =} \StringTok{"Frequencia Abs./Relativa"}\NormalTok{,}
       \AttributeTok{x        =} \StringTok{"Tipo de morte (natural/violenta)"}\NormalTok{,}
       \AttributeTok{caption  =} \StringTok{"Fonte:  1. dados primários coletados por Queops (2024)."}\NormalTok{)}
\InformationTok{\textasciigrave{}\textasciigrave{}\textasciigrave{}}
\end{Highlighting}
\end{Shaded}

\pandocbounded{\includegraphics[keepaspectratio]{cap3-DistNorm_files/figure-pdf/unnamed-chunk-11-1.pdf}}

Depois uma curva suave (densidade) nas \textbf{5 (cinco) categorias} da
variável tipo \texttt{factor} cognominada \texttt{circobt}, a saber:

\begin{enumerate}
\def\labelenumi{\arabic{enumi}.}
\tightlist
\item
  MDCC
\item
  MDIP
\item
  transito
\item
  conf\_fam\_afet
\item
  outros
\end{enumerate}

Gráfiico com curvas suaves de densidade para os dados coletados de idade
na data do óbito para as 5 categorias observadas.

\begin{Shaded}
\begin{Highlighting}[numbers=left,,]
\InformationTok{\textasciigrave{}\textasciigrave{}\textasciigrave{}\{r\}}
\CommentTok{\# Suponha que você tenha um data frame chamado \textquotesingle{}dados\textquotesingle{} com colunas para \textquotesingle{}categoria\textquotesingle{} e para \textquotesingle{}valor\textquotesingle{}}
\NormalTok{dados }\OtherTok{\textless{}{-}}\NormalTok{ obitj\_csv}

\CommentTok{\# Remove as linhas onde a coluna \textquotesingle{}circobt\textquotesingle{} apresenta NA}
\NormalTok{dados\_sem\_na }\OtherTok{\textless{}{-}}\NormalTok{ dados[}\SpecialCharTok{!}\FunctionTok{is.na}\NormalTok{(dados}\SpecialCharTok{$}\NormalTok{circobt), ]}

\CommentTok{\# Carrega o pacote ggplot2 para gráficos avançados}
\FunctionTok{library}\NormalTok{(ggplot2)}

\CommentTok{\# Gera o gráfico de curvas de densidade para cada categoria}
\FunctionTok{ggplot}\NormalTok{(dados\_sem\_na, }\FunctionTok{aes}\NormalTok{(}\AttributeTok{x     =}\NormalTok{ idadeobta,}
                         \AttributeTok{color =}\NormalTok{ circobt,}
                         \AttributeTok{fill  =}\NormalTok{ circobt)) }\SpecialCharTok{+}
  \FunctionTok{geom\_density}\NormalTok{(}\AttributeTok{alpha =} \FloatTok{0.4}\NormalTok{) }\SpecialCharTok{+} \CommentTok{\# Plota as curvas de densidade com transparência}
  \FunctionTok{labs}\NormalTok{(}\AttributeTok{title =} \StringTok{"Curvas de Densidade da idade em anos na data óbito por Categorias"}\NormalTok{,}
       \AttributeTok{x =} \StringTok{"Idade em anos na data óbito"}\NormalTok{,}
       \AttributeTok{y =} \StringTok{"Densidade"}\NormalTok{) }\SpecialCharTok{+}
  \FunctionTok{theme\_minimal}\NormalTok{() }\SpecialCharTok{+}           \CommentTok{\# Usa um tema limpo}
  \FunctionTok{scale\_fill\_manual}\NormalTok{(}\AttributeTok{values =} \FunctionTok{c}\NormalTok{(}\StringTok{"skyblue"}\NormalTok{, }\StringTok{"orange"}\NormalTok{, }\StringTok{"lightgreen"}\NormalTok{, }\StringTok{"pink"}\NormalTok{, }\StringTok{"yellow"}\NormalTok{)) }\SpecialCharTok{+} \CommentTok{\# Cores das áreas}
  \FunctionTok{scale\_color\_manual}\NormalTok{(}\AttributeTok{values =} \FunctionTok{c}\NormalTok{(}\StringTok{"blue"}\NormalTok{, }\StringTok{"red"}\NormalTok{, }\StringTok{"green"}\NormalTok{,  }\StringTok{"purple"}\NormalTok{, }\StringTok{"brown"}\NormalTok{))          }\CommentTok{\# Define cores das linhas}
\InformationTok{\textasciigrave{}\textasciigrave{}\textasciigrave{}}
\end{Highlighting}
\end{Shaded}

\pandocbounded{\includegraphics[keepaspectratio]{cap3-DistNorm_files/figure-pdf/unnamed-chunk-12-1.pdf}}

Agora remover do data set as linhas com idades menor que 5 anos.

Porque tratam-se claramente de dados com erros de digitação.

\begin{Shaded}
\begin{Highlighting}[numbers=left,,]
\InformationTok{\textasciigrave{}\textasciigrave{}\textasciigrave{}\{r\}}
\CommentTok{\# Suponha que você tenha um data frame chamado \textquotesingle{}dados\textquotesingle{} com colunas para \textquotesingle{}categoria\textquotesingle{} e para \textquotesingle{}valor\textquotesingle{}}
\NormalTok{dados }\OtherTok{\textless{}{-}}\NormalTok{ obitj\_csv}

\CommentTok{\# Remove as linhas onde a coluna \textquotesingle{}circobt\textquotesingle{} apresenta NA}
\NormalTok{dados\_sem\_na }\OtherTok{\textless{}{-}}\NormalTok{ dados[}\SpecialCharTok{!}\FunctionTok{is.na}\NormalTok{(dados}\SpecialCharTok{$}\NormalTok{circobt), ]}

\CommentTok{\# Remove, em seguida, as linhas onde a coluna \textquotesingle{}idadeobta\textquotesingle{} apresenta valores menor ou igual a 5 anos}
\NormalTok{dados\_sem\_na }\OtherTok{\textless{}{-}}\NormalTok{ dados\_sem\_na[dados\_sem\_na}\SpecialCharTok{$}\NormalTok{idadeobta }\SpecialCharTok{\textgreater{}} \DecValTok{5}\NormalTok{, ]}

\CommentTok{\# Carrega o pacote ggplot2 para gráficos avançados}
\FunctionTok{library}\NormalTok{(ggplot2)}

\CommentTok{\# Gera o gráfico de curvas de densidade para cada categoria}
\FunctionTok{ggplot}\NormalTok{(dados\_sem\_na, }\FunctionTok{aes}\NormalTok{(}\AttributeTok{x     =}\NormalTok{ idadeobta,}
                         \AttributeTok{color =}\NormalTok{ circobt,}
                         \AttributeTok{fill  =}\NormalTok{ circobt)) }\SpecialCharTok{+}
  \FunctionTok{geom\_density}\NormalTok{(}\AttributeTok{alpha =} \FloatTok{0.4}\NormalTok{) }\SpecialCharTok{+} \CommentTok{\# Plota as curvas de densidade com transparência}
  \FunctionTok{labs}\NormalTok{(}\AttributeTok{title =} \StringTok{"Curvas de Densidade da idade em anos na data óbito por Categorias"}\NormalTok{,}
       \AttributeTok{x =} \StringTok{"Idade em anos na data óbito"}\NormalTok{,}
       \AttributeTok{y =} \StringTok{"Densidade"}\NormalTok{) }\SpecialCharTok{+}
  \FunctionTok{theme\_minimal}\NormalTok{() }\SpecialCharTok{+}           \CommentTok{\# Usa um tema limpo}
  \FunctionTok{scale\_fill\_manual}\NormalTok{(}\AttributeTok{values =} \FunctionTok{c}\NormalTok{(}\StringTok{"skyblue"}\NormalTok{, }\StringTok{"orange"}\NormalTok{, }\StringTok{"lightgreen"}\NormalTok{, }\StringTok{"pink"}\NormalTok{, }\StringTok{"yellow"}\NormalTok{)) }\SpecialCharTok{+} \CommentTok{\# Cores das áreas}
  \FunctionTok{scale\_color\_manual}\NormalTok{(}\AttributeTok{values =} \FunctionTok{c}\NormalTok{(}\StringTok{"blue"}\NormalTok{, }\StringTok{"red"}\NormalTok{, }\StringTok{"green"}\NormalTok{,  }\StringTok{"purple"}\NormalTok{, }\StringTok{"brown"}\NormalTok{))          }\CommentTok{\# Define cores das linhas}
\InformationTok{\textasciigrave{}\textasciigrave{}\textasciigrave{}}
\end{Highlighting}
\end{Shaded}

\pandocbounded{\includegraphics[keepaspectratio]{cap3-DistNorm_files/figure-pdf/unnamed-chunk-13-1.pdf}}

Mesmo gráfico acima, só que com cada curva de densidade em uma linha
horizontal diferente, numa mesma escala no eixo x.

\begin{Shaded}
\begin{Highlighting}[numbers=left,,]
\InformationTok{\textasciigrave{}\textasciigrave{}\textasciigrave{}\{r\}}
\FunctionTok{library}\NormalTok{(ggridges) }\CommentTok{\# Permite criar gráficos de densidade empilhados (ridges)}

\NormalTok{obitj }\SpecialCharTok{|\textgreater{}} 
  \FunctionTok{filter}\NormalTok{(idadeobta }\SpecialCharTok{\textgreater{}}  \DecValTok{2}\NormalTok{) }\SpecialCharTok{|\textgreater{}} 
  \FunctionTok{filter}\NormalTok{( }\SpecialCharTok{!}\FunctionTok{is.na}\NormalTok{(circobt) ) }\SpecialCharTok{|\textgreater{}} 
  \FunctionTok{ggplot}\NormalTok{(}
       \FunctionTok{aes}\NormalTok{(}\AttributeTok{x =}\NormalTok{ idadeobta,}
           \AttributeTok{y =}\NormalTok{ circobt, }\AttributeTok{fill =}\NormalTok{ circobt, }\AttributeTok{color =}\NormalTok{ circobt)) }\SpecialCharTok{+}
  \FunctionTok{geom\_density\_ridges}\NormalTok{(}\AttributeTok{alpha =} \FloatTok{0.5}\NormalTok{, }\AttributeTok{show.legend =} \ConstantTok{FALSE}\NormalTok{) }\SpecialCharTok{+} \CommentTok{\# Plota as curvas de densidade com transparência}
  \FunctionTok{scale\_x\_continuous}\NormalTok{(}\AttributeTok{breaks =} \DecValTok{12}\SpecialCharTok{:}\DecValTok{29}\NormalTok{) }\SpecialCharTok{+}
  \FunctionTok{labs}\NormalTok{(}\AttributeTok{title    =} \StringTok{"Densidade de probabilidade: Idade na data do óbito agrupada pela}\SpecialCharTok{\textbackslash{}n}\StringTok{circunstância do óbito dos jovens c/passagem na DEPAI de Goiânia"}\NormalTok{,}
       \AttributeTok{subtitle =} \StringTok{"Período: 2016 a 2023 (N = 391)"}\NormalTok{,}
       \AttributeTok{x        =} \StringTok{"Idade (anos)"}\NormalTok{,}
       \AttributeTok{y        =} \StringTok{"circobtj"}\NormalTok{,}
       \AttributeTok{caption  =} \StringTok{"Fonte: Dados primários coletados por Queops (2024)}\SpecialCharTok{\textbackslash{}n}\StringTok{MDIP {-} Morte Decorrente Intervenção Policial}\SpecialCharTok{\textbackslash{}n}\StringTok{MDCC {-} Morte Decorrente Conflitos entre Criminalidade"}
\NormalTok{       )}

\CommentTok{\# MDIP = Morte Decorrente Intervenção Policial}
\CommentTok{\# MDCC = Morte Decorrente Conflitos entre Criminalidade}
\InformationTok{\textasciigrave{}\textasciigrave{}\textasciigrave{}}
\end{Highlighting}
\end{Shaded}

\pandocbounded{\includegraphics[keepaspectratio]{cap3-DistNorm_files/figure-pdf/unnamed-chunk-14-1.pdf}}

Observa-se um perfil da distribuição da variável empírica
\texttt{idadeobta} que \ul{\textbf{\emph{assemelha-se muito}}} a uma
\textbf{curva Normal}, na 2 categorias:

MDIP = Morte Decorrente Intervenção Policial

MDCC = Morte Decorrente Conflitos entre Criminalidade

Ou seja, esses dois fenômenos apresentam um
\ul{\textbf{\emph{comportamento aleatório}}}.

Acrescentar o mesmo tipo \textbf{\emph{gráfico de densidade}} acima
agora para a variável categórica \texttt{esc2}.

Ou seja, segundo a categórica da série até então cursada pelo jojvem em
conflito com a lei na data do seu óbito.

\begin{Shaded}
\begin{Highlighting}[numbers=left,,]
\InformationTok{\textasciigrave{}\textasciigrave{}\textasciigrave{}\{r\}}
\NormalTok{obitj }\SpecialCharTok{|\textgreater{}} 
  \FunctionTok{filter}\NormalTok{(idadeobta }\SpecialCharTok{\textgreater{}}  \DecValTok{2}\NormalTok{) }\SpecialCharTok{|\textgreater{}} 
  \FunctionTok{filter}\NormalTok{( }\SpecialCharTok{!}\FunctionTok{is.na}\NormalTok{(esc2) ) }\SpecialCharTok{|\textgreater{}} 
  \FunctionTok{ggplot}\NormalTok{(}
       \FunctionTok{aes}\NormalTok{(}\AttributeTok{x =}\NormalTok{ idadeobta,}
           \AttributeTok{y =}\NormalTok{ esc2, }\AttributeTok{fill =}\NormalTok{ esc2, }\AttributeTok{color =}\NormalTok{ esc2)) }\SpecialCharTok{+}
  \FunctionTok{geom\_density\_ridges}\NormalTok{(}\AttributeTok{alpha =} \FloatTok{0.5}\NormalTok{, }\AttributeTok{show.legend =} \ConstantTok{FALSE}\NormalTok{) }\SpecialCharTok{+}
  \FunctionTok{scale\_x\_continuous}\NormalTok{(}\AttributeTok{breaks =} \DecValTok{13}\SpecialCharTok{:}\DecValTok{28}\NormalTok{) }\SpecialCharTok{+}
  \FunctionTok{labs}\NormalTok{(}\AttributeTok{title    =} \StringTok{"Densidade de probabilidade: Idade na data do óbito (média=18.7 anos) }\SpecialCharTok{\textbackslash{}n}\StringTok{Escolaridade no óbito (var. esc2) dos jovens c/passagem na DEPAI de Goiânia"}\NormalTok{,}
       \AttributeTok{subtitle =} \StringTok{"Período: 2016 a 2023 (N = 379)"}\NormalTok{,}
       \AttributeTok{x        =} \StringTok{"Idade (anos)"}\NormalTok{,}
       \AttributeTok{y        =} \StringTok{"esc2 (na data óbito)"}\NormalTok{,}
       \AttributeTok{caption  =} \StringTok{"Fonte: 1. dados primários coletados por Queops (2024)."}
\NormalTok{       )}
\InformationTok{\textasciigrave{}\textasciigrave{}\textasciigrave{}}
\end{Highlighting}
\end{Shaded}

\pandocbounded{\includegraphics[keepaspectratio]{cap3-DistNorm_files/figure-pdf/unnamed-chunk-15-1.pdf}}

O perfil de \ul{\textbf{\emph{uma distribuição Normal}}} novamente
apareceu para as séries so 6º ano até a 3ª série do Ensino Médio.

Ou seja, um comportamento aleatório para esse fenômeno observado nessas
séries.

Por fim um gráfico de densidades para a mesma variável idade em anos na
data do óbito estratificada ano a ano, para análise da evolução do
fenômeno ao longo do passar do tempo em Goiânia: 2016 a 2023 (8 anos).

\begin{Shaded}
\begin{Highlighting}[numbers=left,,]
\InformationTok{\textasciigrave{}\textasciigrave{}\textasciigrave{}\{r\}}
\NormalTok{obitj }\OtherTok{\textless{}{-}}\NormalTok{ obitj }\SpecialCharTok{|\textgreater{}} 
  \FunctionTok{mutate}\NormalTok{(}\AttributeTok{anobit =} \FunctionTok{year}\NormalTok{(dataobt) )}

\NormalTok{obitj }\SpecialCharTok{|\textgreater{}} 
  \FunctionTok{filter}\NormalTok{(idadeobta }\SpecialCharTok{\textgreater{}}  \DecValTok{2}\NormalTok{) }\SpecialCharTok{|\textgreater{}} 
  \FunctionTok{filter}\NormalTok{( }\SpecialCharTok{!}\FunctionTok{is.na}\NormalTok{(idadeobta) ) }\SpecialCharTok{|\textgreater{}} 
  \FunctionTok{filter}\NormalTok{( }\SpecialCharTok{!}\FunctionTok{is.na}\NormalTok{(anobit) ) }\SpecialCharTok{|\textgreater{}} 
  \FunctionTok{mutate}\NormalTok{(}\AttributeTok{anobit =} \FunctionTok{as.factor}\NormalTok{(anobit) ) }\SpecialCharTok{|\textgreater{}} 
  \FunctionTok{ggplot}\NormalTok{(}
       \FunctionTok{aes}\NormalTok{(}\AttributeTok{x =}\NormalTok{ idadeobta,}
           \AttributeTok{y =}\NormalTok{ anobit, }\AttributeTok{fill =}\NormalTok{ anobit, }\AttributeTok{color =}\NormalTok{ anobit)) }\SpecialCharTok{+}
  \FunctionTok{geom\_density\_ridges}\NormalTok{(}\FunctionTok{aes}\NormalTok{(}\AttributeTok{y =}\NormalTok{ anobit),}
                      \AttributeTok{alpha =} \FloatTok{0.5}\NormalTok{, }\AttributeTok{show.legend =} \ConstantTok{FALSE}\NormalTok{) }\SpecialCharTok{+}
  \FunctionTok{scale\_x\_continuous}\NormalTok{(}\AttributeTok{breaks =} \DecValTok{12}\SpecialCharTok{:}\DecValTok{29}\NormalTok{) }\SpecialCharTok{+}
  \FunctionTok{labs}\NormalTok{(}\AttributeTok{title    =} \StringTok{"Densidade de probabilidade: Idade na data do óbito agrupada pelo}\SpecialCharTok{\textbackslash{}n}\StringTok{ano do óbito dos jovens c/passagem na DEPAI de Goiânia"}\NormalTok{,}
       \AttributeTok{subtitle =} \StringTok{"Período: 2016 a 2023 (N = 396)"}\NormalTok{,}
       \AttributeTok{x        =} \StringTok{"Idade (anos)"}\NormalTok{,}
       \AttributeTok{y        =} \StringTok{"ano"}\NormalTok{,}
       \AttributeTok{caption  =} \StringTok{"Fonte: Dados primários coletados por Queops (2024)"}
\NormalTok{       )}

\CommentTok{\# MDIP = Morte Decorrente Intervenção Policial}
\CommentTok{\# MDCC = Morte Decorrente Conflitos entre Criminalidade}
\InformationTok{\textasciigrave{}\textasciigrave{}\textasciigrave{}}
\end{Highlighting}
\end{Shaded}

\pandocbounded{\includegraphics[keepaspectratio]{cap3-DistNorm_files/figure-pdf/unnamed-chunk-16-1.pdf}}

A proximidade a \ul{\textbf{\emph{uma distribuição Normal}}} apareceu
nos anos de \textbf{2016, 2017 e 2018}.

Nos anos subsequentes, 2019 até 2023, as distribuiões ficaram bem mais
dispersas e as três últimas (2021, 2022 e 2023) mostraram-se
\ul{\textbf{\emph{bimodais}}}, \emph{afastando-se} de um perfil
\emph{Normal}.

\section{Exercício n.~8.4 - Amostragem no
Campus}\label{exercuxedcio-n.-8.4---amostragem-no-campus}

Você gostaria de iniciar um clube no campus para os que fazem
psicologia, e você está interessado na \textbf{\emph{proporção dos que
fazem psicologia que adeririam}}. A taxa seria de US\$35 e usada para
pagar palestrantes convidados.

Você pergunta a cinco estudantes que fazem psicologia e que fazem seu
seminário de psicologia se eles estariam interessados em aderir ao clube
e quatro, dos cinco, respondem que sim. Esse método de amostragem é
viesado? Se for, qual é a direção provável do viés?

\begin{Shaded}
\begin{Highlighting}[numbers=left,,]
\InformationTok{\textasciigrave{}\textasciigrave{}\textasciigrave{}\{r\}}
\CommentTok{\# Variável binária: 0 = não e 1 = sim}
\CommentTok{\# amostra de tamanho n = 5}
\NormalTok{am }\OtherTok{=} \FunctionTok{c}\NormalTok{(}\DecValTok{1}\NormalTok{, }\DecValTok{1}\NormalTok{, }\DecValTok{1}\NormalTok{, }\DecValTok{1}\NormalTok{, }\DecValTok{0}\NormalTok{)}

\CommentTok{\# calcular tamanho da amostra}
\CommentTok{\# sum(am)}
\FunctionTok{cat}\NormalTok{(}\StringTok{"tamanho da amostra"}\NormalTok{, }\StringTok{"}\SpecialCharTok{\textbackslash{}n}\StringTok{"}\NormalTok{)}
\FunctionTok{cat}\NormalTok{(}\StringTok{"n = "}\NormalTok{, }\FunctionTok{sum}\NormalTok{(am))}
\FunctionTok{cat}\NormalTok{(}\StringTok{"}\SpecialCharTok{\textbackslash{}n}\StringTok{"}\NormalTok{)}

\CommentTok{\# mean(am) \%\textgreater{}\% round(4) * 100}
\FunctionTok{cat}\NormalTok{(}\StringTok{"proporção dos que fazem psicologia que adeririam:"}\NormalTok{, }\StringTok{"}\SpecialCharTok{\textbackslash{}n}\StringTok{"}\NormalTok{)}
\FunctionTok{cat}\NormalTok{(}\FunctionTok{mean}\NormalTok{(am) }\SpecialCharTok{\%\textgreater{}\%} \FunctionTok{round}\NormalTok{(}\DecValTok{4}\NormalTok{) }\SpecialCharTok{*} \DecValTok{100}\NormalTok{, }\StringTok{"\%"}\NormalTok{)}
\InformationTok{\textasciigrave{}\textasciigrave{}\textasciigrave{}}
\end{Highlighting}
\end{Shaded}

\begin{verbatim}
tamanho da amostra 
n =  4
proporção dos que fazem psicologia que adeririam: 
80 %
\end{verbatim}

O método é \ul{\textbf{\emph{viesado}}} porque se trata de \emph{uma
amostra de conveniência}.

É esperada uma direção de \emph{superepresentação} dessa amostra.

Logo, \emph{80\%} apresenta um \textbf{\emph{viés de superestimativa}}
da proporção dos que fazem psicologia que adeririam ao Clube proposto.

\section{Exercício n.~8.12 - Desonestidade
acadêmica}\label{exercuxedcio-n.-8.12---desonestidade-acaduxeamica}

Como estrair uma AAE-c/TPPP no R.

\begin{Shaded}
\begin{Highlighting}[numbers=left,,]
\InformationTok{\textasciigrave{}\textasciigrave{}\textasciigrave{}\{r\}}
\CommentTok{\# Suponha que temos um data frame chamado \textquotesingle{}dados\textquotesingle{} com uma coluna \textquotesingle{}estrato\textquotesingle{} indicando o estrato de cada observação}

\CommentTok{\# Exemplo de criação do data frame}
\FunctionTok{set.seed}\NormalTok{(}\DecValTok{123}\NormalTok{) }\CommentTok{\# Para reprodutibilidade}
\NormalTok{dados }\OtherTok{\textless{}{-}} \FunctionTok{data.frame}\NormalTok{(}
  \AttributeTok{id =} \DecValTok{1}\SpecialCharTok{:}\DecValTok{3954}\NormalTok{,}
  \AttributeTok{estrato =} \FunctionTok{c}\NormalTok{(}
    \FunctionTok{rep}\NormalTok{(}\DecValTok{1}\NormalTok{, }\DecValTok{1127}\NormalTok{),}
    \FunctionTok{rep}\NormalTok{(}\DecValTok{2}\NormalTok{,  }\DecValTok{989}\NormalTok{),}
    \FunctionTok{rep}\NormalTok{(}\DecValTok{3}\NormalTok{,  }\DecValTok{943}\NormalTok{),}
    \FunctionTok{rep}\NormalTok{(}\DecValTok{4}\NormalTok{,  }\DecValTok{895}\NormalTok{)}
\NormalTok{    )}
\NormalTok{)}

\CommentTok{\# Defina o tamanho total da amostra desejada}
\NormalTok{n\_total }\OtherTok{\textless{}{-}} \DecValTok{40}

\CommentTok{\# Calcule o tamanho de cada estrato}
\NormalTok{tamanho\_estrato }\OtherTok{\textless{}{-}} \FunctionTok{table}\NormalTok{(dados}\SpecialCharTok{$}\NormalTok{estrato)}
\CommentTok{\# Visualize o tamanho de cada estrato}
\FunctionTok{cat}\NormalTok{(}\StringTok{"Cálculo do Tamanho de cada estrato na População Amostrada"}\NormalTok{, }\StringTok{"}\SpecialCharTok{\textbackslash{}n}\StringTok{"}\NormalTok{)}
\FunctionTok{print}\NormalTok{(tamanho\_estrato)}

\CommentTok{\# Calcule o tamanho da amostra para cada estrato (proporcional ao tamanho do estrato)}
\NormalTok{n\_estrato }\OtherTok{\textless{}{-}}\NormalTok{ ( n\_total }\SpecialCharTok{*}\NormalTok{ tamanho\_estrato }\SpecialCharTok{/} \FunctionTok{sum}\NormalTok{(tamanho\_estrato) )}

\CommentTok{\# Utilizando a função ceiling() para arredondar para o inteiro superior}
\NormalTok{n\_estrato\_arredondados }\OtherTok{\textless{}{-}} \FunctionTok{ceiling}\NormalTok{(n\_estrato)}

\NormalTok{n\_estrato}
\NormalTok{n\_estrato\_arredondados}

\CommentTok{\# Realize a amostragem estratificada}
\NormalTok{amostra }\OtherTok{\textless{}{-}} \FunctionTok{do.call}\NormalTok{(rbind, }\FunctionTok{lapply}\NormalTok{(}\DecValTok{1}\SpecialCharTok{:}\DecValTok{4}\NormalTok{, }\ControlFlowTok{function}\NormalTok{(e) \{}
\NormalTok{  subset\_estrato }\OtherTok{\textless{}{-}} \FunctionTok{subset}\NormalTok{(dados, estrato }\SpecialCharTok{==}\NormalTok{ e)}
\NormalTok{  subset\_estrato[}\FunctionTok{sample}\NormalTok{(}\FunctionTok{nrow}\NormalTok{(subset\_estrato), n\_estrato\_arredondados[e]), ]}
\NormalTok{\}))}

\CommentTok{\# Visualize a amostra}
\FunctionTok{print}\NormalTok{(amostra)}

\CommentTok{\# Visualize o tamanho de cada estrato na amostra: AAE c/TPP}
\FunctionTok{cat}\NormalTok{(}\StringTok{"}\SpecialCharTok{\textbackslash{}n}\StringTok{"}\NormalTok{) }\CommentTok{\# pular uma linha na saída}
\FunctionTok{cat}\NormalTok{(}\StringTok{"Tamanho de cada estrato na Amostra piloto (n\_inic = 40): 1 AAE c/TPP"}\NormalTok{, }\StringTok{"}\SpecialCharTok{\textbackslash{}n}\StringTok{"}\NormalTok{)}
\FunctionTok{print}\NormalTok{(}\FunctionTok{table}\NormalTok{(amostra}\SpecialCharTok{$}\NormalTok{estrato))}
\InformationTok{\textasciigrave{}\textasciigrave{}\textasciigrave{}}
\end{Highlighting}
\end{Shaded}

\begin{verbatim}
Cálculo do Tamanho de cada estrato na População Amostrada 

   1    2    3    4 
1127  989  943  895 

        1         2         3         4 
11.401113 10.005058  9.539707  9.054122 

 1  2  3  4 
12 11 10 10 
       id estrato
415   415       1
463   463       1
179   179       1
526   526       1
195   195       1
938   938       1
1038 1038       1
665   665       1
602   602       1
709   709       1
1011 1011       1
1115 1115       1
2080 2080       2
1475 1475       2
1776 1776       2
1482 1482       2
1967 1967       2
1153 1153       2
1646 1646       2
1553 1553       2
2114 2114       2
1893 1893       2
1338 1338       2
3048 3048       3
2706 2706       3
2709 2709       3
2671 2671       3
2987 2987       3
2489 2489       3
2960 2960       3
2259 2259       3
2660 2660       3
2606 2606       3
3680 3680       4
3834 3834       4
3901 3901       4
3082 3082       4
3368 3368       4
3194 3194       4
3880 3880       4
3283 3283       4
3225 3225       4
3276 3276       4

Tamanho de cada estrato na Amostra piloto (n_inic = 40): 1 AAE c/TPP 

 1  2  3  4 
12 11 10 10 
\end{verbatim}

\section{Exercício n.~1.42 - Ela soa
alta}\label{exercuxedcio-n.-1.42---ela-soa-alta}

\textbf{1}. \textbf{Importar} o \emph{data set}, o arquivo
\texttt{ex01-42hearing.csv}, que se encontra na pasta
\texttt{dat\ \textgreater{}\ csv} de nosso Projeto
\texttt{CDE-a-DPP.Rproj}. Recomenda-se baixar a atualizar a última
versão desse nosso projeto que se encontra compartilhado no google
drive:
\url{https://drive.google.com/drive/u/1/folders/1wm9jUo5XlBHqbQDRf9XevFbXcqkWogqt}

\begin{Shaded}
\begin{Highlighting}[numbers=left,,]
\InformationTok{\textasciigrave{}\textasciigrave{}\textasciigrave{}\{r\}}
\CommentTok{\# Importar como tibble o arquivo de dentro da pasta chamada out.}
\NormalTok{audicao }\OtherTok{\textless{}{-}}\NormalTok{ readr}\SpecialCharTok{::}\FunctionTok{read\_csv}\NormalTok{(}\AttributeTok{file   =} \StringTok{"dat/csv/ex01{-}42hearing.csv"}\NormalTok{,}
                           \CommentTok{\# delim  = ",",}
                           \AttributeTok{quote  =} \StringTok{"}\SpecialCharTok{\textbackslash{}"}\StringTok{"}\NormalTok{,}
                           \AttributeTok{locale =} \FunctionTok{locale}\NormalTok{(}
                             \AttributeTok{decimal\_mark =} \StringTok{"."}\NormalTok{,}
                             \AttributeTok{encoding     =} \StringTok{"UTF{-}8"}
\NormalTok{                             )}
\NormalTok{                           )}

\CommentTok{\# cat {-} Concatenate And Print}
\FunctionTok{cat}\NormalTok{(}\StringTok{"}\SpecialCharTok{\textbackslash{}n}\StringTok{"}\NormalTok{) }\CommentTok{\# imprime no console (saída) uma linha em branco}
\FunctionTok{cat}\NormalTok{(}\StringTok{"}\SpecialCharTok{\textbackslash{}n}\StringTok{"}\NormalTok{)}
\FunctionTok{cat}\NormalTok{(}\StringTok{"Estrutura do objeto R denominado audicao:}\SpecialCharTok{\textbackslash{}n}\StringTok{"}\NormalTok{)}
\FunctionTok{str}\NormalTok{(audicao)}

\FunctionTok{cat}\NormalTok{(}\StringTok{"}\SpecialCharTok{\textbackslash{}n}\StringTok{"}\NormalTok{)}
\FunctionTok{cat}\NormalTok{(}\StringTok{"Nomes da única coluna do objeto audicao:}\SpecialCharTok{\textbackslash{}n}\StringTok{"}\NormalTok{)}
\FunctionTok{names}\NormalTok{(audicao)}

\NormalTok{audicao }\CommentTok{\# tibble:24 × 1}
\InformationTok{\textasciigrave{}\textasciigrave{}\textasciigrave{}}
\end{Highlighting}
\end{Shaded}

\begin{verbatim}


Estrutura do objeto R denominado audicao:
spc_tbl_ [24 x 1] (S3: spec_tbl_df/tbl_df/tbl/data.frame)
 $ numcorrect: num [1:24] 65 61 67 59 58 62 56 67 61 67 ...
 - attr(*, "spec")=
  .. cols(
  ..   numcorrect = col_double()
  .. )
 - attr(*, "problems")=<externalptr> 

Nomes da única coluna do objeto audicao:
[1] "numcorrect"
# A tibble: 24 x 1
   numcorrect
        <dbl>
 1         65
 2         61
 3         67
 4         59
 5         58
 6         62
 7         56
 8         67
 9         61
10         67
# i 14 more rows
\end{verbatim}

\subsection{letra a}\label{letra-a}

\textbf{2}. \textbf{Gerar} dois \emph{diagramas de ramo e folha}, como
pedido na letra do exercício 1.42 (p.~36).

Cf. \url{https://www.geeksforgeeks.org/r-stem-and-leaf-plots/}

\begin{Shaded}
\begin{Highlighting}[numbers=left,,]
\InformationTok{\textasciigrave{}\textasciigrave{}\textasciigrave{}\{r\}}
\CommentTok{\# R program to illustrate}
\CommentTok{\# Stem and Leaf Plot}

\CommentTok{\# using stem()}
\FunctionTok{stem}\NormalTok{(audicao}\SpecialCharTok{$}\NormalTok{numcorrect, }\AttributeTok{scale =} \FloatTok{0.5}\NormalTok{) }\CommentTok{\# nessa escala os ramos não se dividem}
\InformationTok{\textasciigrave{}\textasciigrave{}\textasciigrave{}}
\end{Highlighting}
\end{Shaded}

\begin{verbatim}

  The decimal point is 1 digit(s) to the right of the |

  4 | 9
  5 | 36668889
  6 | 1123556777889
  7 | 00
\end{verbatim}

Outro diagrama de árvore, agora com divisão dos ramos.

\begin{Shaded}
\begin{Highlighting}[numbers=left,,]
\InformationTok{\textasciigrave{}\textasciigrave{}\textasciigrave{}\{r\}}
\CommentTok{\# using stem()}
\FunctionTok{stem}\NormalTok{(audicao}\SpecialCharTok{$}\NormalTok{numcorrect, }\AttributeTok{scale =} \DecValTok{1}\NormalTok{)  }\CommentTok{\# nessa escala os ramos dividem{-}se em dois}
\InformationTok{\textasciigrave{}\textasciigrave{}\textasciigrave{}}
\end{Highlighting}
\end{Shaded}

\begin{verbatim}

  The decimal point is 1 digit(s) to the right of the |

  4 | 9
  5 | 3
  5 | 6668889
  6 | 1123
  6 | 556777889
  7 | 00
\end{verbatim}

O segundo diagrama de ramos e folhas é mais informativo que o primeiro.

Porque mostra uma distribuição bimodal.

Todavia o \emph{primeiro resume melhor os dados}, com forma de sino
levemente assimétrica à esquerda. E com moda igual à mediana.

\section{Exercício n.~2.31 - Tempos de sobrevivência de
cobaias}\label{exercuxedcio-n.-2.31---tempos-de-sobrevivuxeancia-de-cobaias}

\begin{quote}
Eis os \emph{tempos de sobrevivência, em dias}, de 72 cobaias depois de
serem infectadas, por injeção, com uma bactéria infecciosa, em um
experimento médico.\textsuperscript{17} Os \ul{\emph{\textbf{tempos de
sobrevivência}, sejam de máquinas sob estresse ou pacientes de câncer
após o tratamento, \textbf{em geral têm distribuições que são
assimétricas à direita}}}.

43 45 53 56 56 57 58 66 67 73 74 79 80 80 81 81 81 82 83 83 84 88

89 91 91 92 92 97 99 99 100 100 101 102 102 102 103 104 107 108

109 113 114 118 121 123 126 128 137 138 139 144 145 147 156 162

174 178 179 184 191 198 211 214 243 249 329 380 403 511 522 598

(a) Faça o \emph{gráfico da distribuição} e \emph{descreva} suas
\emph{principais características}. O gráfico mostra a \emph{assimetria à
direita \ul{esperada}}?

(b) Qual \emph{resumo numérico} você \emph{escolheria} para esses dados?
\emph{Calcule} seu resumo escolhido. Como \emph{ele reflete a
assimetria} da \emph{distribuição}?
\end{quote}

Carregar o arquivo \texttt{cobaias}.

Nome correto: \texttt{ex02-31guinpigs}

Extensão: \texttt{.csv} ou \texttt{.xls}

\textbf{1}. \textbf{Importar} o \textbf{\emph{data set}}, que se
encontra na pasta \texttt{dat\ \textgreater{}\ sobr} de nosso Projeto
\texttt{CDE-a-DPP.Rproj}. Recomenda-se baixar a atualizar a última
versão desse nosso projeto que se encontra compartilhado no google
drive:
\url{https://drive.google.com/drive/u/1/folders/1wm9jUo5XlBHqbQDRf9XevFbXcqkWogqt}

\begin{Shaded}
\begin{Highlighting}[numbers=left,,]
\InformationTok{\textasciigrave{}\textasciigrave{}\textasciigrave{}\{r\}}
\CommentTok{\# Importar como tibble o arquivo de dentro da pasta chamada sobr.}
\NormalTok{cobaias }\OtherTok{\textless{}{-}}\NormalTok{ readr}\SpecialCharTok{::}\FunctionTok{read\_csv}\NormalTok{(}\AttributeTok{file   =} \StringTok{"dat/sobr/ex02{-}31guinpigs.csv"}\NormalTok{,}
                           \CommentTok{\# delim  = NULL, \# delimitador é new line}
                           \AttributeTok{quote  =} \StringTok{"}\SpecialCharTok{\textbackslash{}"}\StringTok{"}\NormalTok{,}
                           \AttributeTok{locale =} \FunctionTok{locale}\NormalTok{(}
                             \AttributeTok{decimal\_mark =} \StringTok{"."}\NormalTok{,}
                             \AttributeTok{encoding     =} \StringTok{"UTF{-}8"}
\NormalTok{                             )}
\NormalTok{                           )}

\CommentTok{\# cat {-} Concatenate And Print}
\FunctionTok{cat}\NormalTok{(}\StringTok{"}\SpecialCharTok{\textbackslash{}n}\StringTok{"}\NormalTok{) }\CommentTok{\# imprime no console (saída) uma linha em branco}
\FunctionTok{cat}\NormalTok{(}\StringTok{"}\SpecialCharTok{\textbackslash{}n}\StringTok{"}\NormalTok{)}
\CommentTok{\# imprime no console (saída) uma string}
\FunctionTok{cat}\NormalTok{(}\StringTok{"Estrutura do objeto R denominado cobaias:}\SpecialCharTok{\textbackslash{}n}\StringTok{"}\NormalTok{)}
\FunctionTok{str}\NormalTok{(cobaias)}

\FunctionTok{cat}\NormalTok{(}\StringTok{"}\SpecialCharTok{\textbackslash{}n}\StringTok{"}\NormalTok{)}
\FunctionTok{cat}\NormalTok{(}\StringTok{"Nome da única coluna do objeto cobaias:}\SpecialCharTok{\textbackslash{}n}\StringTok{"}\NormalTok{)}
\FunctionTok{names}\NormalTok{(cobaias) }\CommentTok{\# days}

\CommentTok{\# mudar o nome da variável days para dias}
\FunctionTok{names}\NormalTok{(cobaias) }\OtherTok{=} \StringTok{"dias"}

\NormalTok{cobaias }\CommentTok{\# tibble:72 × 1}
\InformationTok{\textasciigrave{}\textasciigrave{}\textasciigrave{}}
\end{Highlighting}
\end{Shaded}

\begin{verbatim}


Estrutura do objeto R denominado cobaias:
spc_tbl_ [72 x 1] (S3: spec_tbl_df/tbl_df/tbl/data.frame)
 $ days: num [1:72] 43 45 53 56 56 57 58 66 67 73 ...
 - attr(*, "spec")=
  .. cols(
  ..   days = col_double()
  .. )
 - attr(*, "problems")=<externalptr> 

Nome da única coluna do objeto cobaias:
[1] "days"
# A tibble: 72 x 1
    dias
   <dbl>
 1    43
 2    45
 3    53
 4    56
 5    56
 6    57
 7    58
 8    66
 9    67
10    73
# i 62 more rows
\end{verbatim}

\textbf{a1. Fazer o gráfico da distribuição} e descrever suas principais
características.

\begin{Shaded}
\begin{Highlighting}[numbers=left,,]
\InformationTok{\textasciigrave{}\textasciigrave{}\textasciigrave{}\{r\}}
\CommentTok{\# definir o número de colunas do histograma}
\NormalTok{nbreaks }\OtherTok{=} \DecValTok{10}

\CommentTok{\# Calcula a densidade dos dados}
\NormalTok{densidade }\OtherTok{\textless{}{-}} \FunctionTok{density}\NormalTok{(cobaias}\SpecialCharTok{$}\NormalTok{dias)}

\CommentTok{\# Encontra o valor máximo entre o histograma e a curva de densidade}
\NormalTok{hist\_info }\OtherTok{\textless{}{-}} \FunctionTok{hist}\NormalTok{(cobaias}\SpecialCharTok{$}\NormalTok{dias, }
                  \AttributeTok{breaks =}\NormalTok{ nbreaks, }\CommentTok{\# Número de barras do histograma}
                  \AttributeTok{plot =} \ConstantTok{FALSE}\NormalTok{, }
                  \AttributeTok{probability =} \ConstantTok{TRUE}\NormalTok{)}
\NormalTok{max\_y }\OtherTok{\textless{}{-}} \FunctionTok{max}\NormalTok{(}\FunctionTok{max}\NormalTok{(hist\_info}\SpecialCharTok{$}\NormalTok{density), }\FunctionTok{max}\NormalTok{(densidade}\SpecialCharTok{$}\NormalTok{y))}

\CommentTok{\# Cria o histograma dos dados, com densidade no eixo y}
\CommentTok{\# Plota o histograma com o eixo y ajustado}
\FunctionTok{hist}\NormalTok{(cobaias}\SpecialCharTok{$}\NormalTok{dias, }
     \AttributeTok{breaks =}\NormalTok{ nbreaks,          }\CommentTok{\# Número de barras do histograma}
     \AttributeTok{probability =} \ConstantTok{TRUE}\NormalTok{,        }\CommentTok{\# Mostra densidade ao invés de frequência}
     \AttributeTok{main =} \StringTok{"Histograma e Curva de Densidade}\SpecialCharTok{\textbackslash{}n}\StringTok{Tempo de Sobrevivência de cobaias após infecção"}\NormalTok{,}
     \AttributeTok{xlab =} \StringTok{"Tempo (dias)"}\NormalTok{,}
     \AttributeTok{ylab =} \StringTok{"densidade (admensional)"}\NormalTok{,}
     \AttributeTok{col =} \StringTok{"lightblue"}\NormalTok{, }
     \AttributeTok{border =} \StringTok{"black"}\NormalTok{,}
     \AttributeTok{ylim =} \FunctionTok{c}\NormalTok{(}\DecValTok{0}\NormalTok{, max\_y }\SpecialCharTok{*} \FloatTok{1.05}\NormalTok{) }\CommentTok{\# Ajuste para garantir espaço para o pico}
\NormalTok{     )}

\CommentTok{\# Adiciona a curva de densidade ao histograma}
\FunctionTok{lines}\NormalTok{(}\FunctionTok{density}\NormalTok{(cobaias}\SpecialCharTok{$}\NormalTok{dias), }
      \AttributeTok{col =} \StringTok{"red"}\NormalTok{, }
      \AttributeTok{lwd =} \DecValTok{2}\NormalTok{)}

\CommentTok{\# Adiciona uma legenda}
\FunctionTok{legend}\NormalTok{(}\StringTok{"topright"}\NormalTok{, }
       \AttributeTok{legend =} \FunctionTok{c}\NormalTok{(}\StringTok{"Curva de Densidade"}\NormalTok{), }
       \AttributeTok{col =} \FunctionTok{c}\NormalTok{(}\StringTok{"red"}\NormalTok{), }
       \AttributeTok{lwd =} \DecValTok{2}\NormalTok{)}
\InformationTok{\textasciigrave{}\textasciigrave{}\textasciigrave{}}
\end{Highlighting}
\end{Shaded}

\pandocbounded{\includegraphics[keepaspectratio]{cap3-DistNorm_files/figure-pdf/unnamed-chunk-23-1.pdf}}

\textbf{a2.} Fazer o gráfico da distribuição e \textbf{descrever suas
principais características}.

Tanto o histograma como a curva suave de densidade (que o aproxima)
ilustram uma \textbf{\emph{distribuição dos dados empíricos}} com
\emph{um pico} em aproxiamdamente 100 dias e com \textbf{\emph{uma
acentuada assimetria à direita}}.

\textbf{a3.} O gráfico mostra a assimetria à direita esperada?

Sim, o gráfico, com as duas curvas, ilustra assimetria acentuada à
direita esperada para \ul{\textbf{\emph{tempos de sobrevivência}}} de
cobaias infectadas com diferentes doses do bacilo virulento da
tuberculose.

Cf. T. Bjerkedal, ``Acquisition of resistance in guinea pigs infected
with different doses of virulent tubercle bacilli,'' \emph{American
Journal of Hygiene}, 72 (1960), p.~130-148. In: (MOORE; NOTZ; FLIGNER,
2023 , ex. 2.31, p.~53 e Notas e fontes de dados, p.~e-243, nota 17).

(b) \textbf{Qual resumo numérico você escolheria para esses dados?}
Calcule seu resumo escolhido. Como ele reflete a assimetria da
distribuição?

\textbf{b.1.} \textbf{Qual resumo numérico você escolheria para esses
dados?}

Resumo dos cinco números, porque os dados empíricos ampresentam, como
esperado pela literatura, forte assimetria à esquerda.

\textbf{b.2 Calcule seu resumo escolhido.}

\begin{Shaded}
\begin{Highlighting}[numbers=left,,]
\InformationTok{\textasciigrave{}\textasciigrave{}\textasciigrave{}\{r\}}
\FunctionTok{summary}\NormalTok{(cobaias}\SpecialCharTok{$}\NormalTok{dias)}

\CommentTok{\# calcular}
\NormalTok{AIQ }\OtherTok{=} \FunctionTok{summary}\NormalTok{(cobaias}\SpecialCharTok{$}\NormalTok{dias)[}\DecValTok{5}\NormalTok{] }\SpecialCharTok{{-}} \FunctionTok{summary}\NormalTok{(cobaias}\SpecialCharTok{$}\NormalTok{dias)[}\DecValTok{2}\NormalTok{]}
\NormalTok{AQ2 }\OtherTok{=} \FunctionTok{summary}\NormalTok{(cobaias}\SpecialCharTok{$}\NormalTok{dias)[}\DecValTok{3}\NormalTok{] }\SpecialCharTok{{-}} \FunctionTok{summary}\NormalTok{(cobaias}\SpecialCharTok{$}\NormalTok{dias)[}\DecValTok{2}\NormalTok{]}
\NormalTok{AQ3 }\OtherTok{=} \FunctionTok{summary}\NormalTok{(cobaias}\SpecialCharTok{$}\NormalTok{dias)[}\DecValTok{5}\NormalTok{] }\SpecialCharTok{{-}} \FunctionTok{summary}\NormalTok{(cobaias}\SpecialCharTok{$}\NormalTok{dias)[}\DecValTok{3}\NormalTok{]}
\NormalTok{razaoAQ3\_2 }\OtherTok{=}\NormalTok{ AQ3 }\SpecialCharTok{/}\NormalTok{AQ2}

\CommentTok{\# exibir}
\FunctionTok{cat}\NormalTok{(}\StringTok{"AIQ = "}\NormalTok{, AIQ, }\StringTok{"}\SpecialCharTok{\textbackslash{}n}\StringTok{"}\NormalTok{)}
\FunctionTok{cat}\NormalTok{(}\StringTok{"AQ2 = "}\NormalTok{, AQ2, }\StringTok{"}\SpecialCharTok{\textbackslash{}n}\StringTok{"}\NormalTok{)}
\FunctionTok{cat}\NormalTok{(}\StringTok{"AQ3 = "}\NormalTok{, AQ3, }\StringTok{"}\SpecialCharTok{\textbackslash{}n}\StringTok{"}\NormalTok{)}
\FunctionTok{cat}\NormalTok{(}\StringTok{"AQ3/AQ2 = "}\NormalTok{, razaoAQ3\_2, }\StringTok{"}\SpecialCharTok{\textbackslash{}n}\StringTok{"}\NormalTok{)}
\InformationTok{\textasciigrave{}\textasciigrave{}\textasciigrave{}}
\end{Highlighting}
\end{Shaded}

\begin{verbatim}
   Min. 1st Qu.  Median    Mean 3rd Qu.    Max. 
  43.00   82.75  102.50  141.85  149.25  598.00 
AIQ =  66.5 
AQ2 =  19.75 
AQ3 =  46.75 
AQ3/AQ2 =  2.367089 
\end{verbatim}

\textbf{b.3} \textbf{Como ele reflete a assimetria da distribuição?}

A \textbf{\emph{média}} (141,85 dias) é \ul{\textbf{\emph{muito maior}}}
que a \textbf{\emph{mediana}} (102,50 dias).

O \emph{valor máximo} (598 dias) é \emph{muito maior} que \emph{ambas},
indicando presença de \textbf{\emph{outliers à direita}}, mas não se
sabe, ainda, \emph{quantos outliers há}.

A \textbf{\emph{Amplitude Interquartil}}: \textbf{AIQ} = Q3 - Q1 =
149,25 - 82,75 = 66,5 dias.

A \textbf{\emph{amplitude do 2º quartil}} (102,50-82,75 =
\textbf{19,75}) é \ul{\textbf{\emph{bem menor}}} que a do \textbf{3º
quartil} (149,25-102,50 = \textbf{46,75}). Esta é \textbf{\emph{mais que
o dobro}} daquela.

Ou seja, a \textbf{\emph{mediana}} (Q2) está \textbf{\emph{mais
próxima}} de \textbf{\emph{Q1}} do que de \textbf{\emph{Q3}}.

Olhar para um \textbf{\emph{boxplot}} é a melhor forma de interpretar um
resumo de 5 números.

\begin{Shaded}
\begin{Highlighting}[numbers=left,,]
\InformationTok{\textasciigrave{}\textasciigrave{}\textasciigrave{}\{r\}}
\CommentTok{\# Tem{-}se um data frame com uma coluna: sobrevivência em dias}
\CommentTok{\# Não foram fornecidos os grupos e suas dosagens de infectação}
\CommentTok{\# Suponha que foram aplicadas doses diferentes para 3 grupos}

\CommentTok{\# Como não foi repassada essa informação sobre o estudo,}
\CommentTok{\# Vamos supor que todas as cobaias são de um mesmo grupo: A, por exemplo}
\NormalTok{cobaias}\SpecialCharTok{$}\NormalTok{grupo }\OtherTok{=} \StringTok{"A"}

\CommentTok{\# Define as cores para cada grupo}
\NormalTok{cores }\OtherTok{\textless{}{-}} \FunctionTok{c}\NormalTok{(}\StringTok{"A"} \OtherTok{=} \StringTok{"skyblue"}\NormalTok{, }\StringTok{"B"} \OtherTok{=} \StringTok{"orange"}\NormalTok{, }\StringTok{"C"} \OtherTok{=} \StringTok{"lightgreen"}\NormalTok{)}

\CommentTok{\# Gera o boxplot, usando as cores definidas}
\FunctionTok{boxplot}\NormalTok{(dias }\SpecialCharTok{\textasciitilde{}}\NormalTok{ grupo, }
        \AttributeTok{data =}\NormalTok{ cobaias,}
        \AttributeTok{col  =}\NormalTok{ cores[cobaias}\SpecialCharTok{$}\NormalTok{grupo],}
        \AttributeTok{main =} \StringTok{"Boxplot com um só Grupo de dosagem, por Cores"}\NormalTok{,}
        \AttributeTok{xlab =} \StringTok{"Tempo de sobrevivência de cobaias (dias)"}\NormalTok{,}
        \AttributeTok{ylab =} \StringTok{"Grupo A"}\NormalTok{,  }\CommentTok{\# supondo que receberam a mesma dose do }
        \AttributeTok{horizontal =} \ConstantTok{TRUE}\NormalTok{) }\CommentTok{\# bacilo virulento da tuberculose}
                           \CommentTok{\# virulent tubercle bacilli}

\CommentTok{\# Adiciona a legenda}
\FunctionTok{legend}\NormalTok{(}\StringTok{"topright"}\NormalTok{,            }\CommentTok{\# posiciona a legenda acima à direita}
       \AttributeTok{title  =} \StringTok{"Grupo"}\NormalTok{,      }\CommentTok{\# É o título da legenda.}
       \AttributeTok{legend =} \FunctionTok{names}\NormalTok{(cores), }\CommentTok{\# serão os nomes dos grupos: A, B e C.}
       \AttributeTok{fill   =}\NormalTok{ cores         }\CommentTok{\# serão as cores de preencimento escolhidas}
\NormalTok{       )}

\CommentTok{\# Calcula a média dos valores}
\NormalTok{media }\OtherTok{\textless{}{-}} \FunctionTok{mean}\NormalTok{(cobaias}\SpecialCharTok{$}\NormalTok{dias)}

\CommentTok{\# Adiciona a média como um ponto vermelho no boxplot}
\FunctionTok{points}\NormalTok{(}\AttributeTok{y =} \DecValTok{1}\NormalTok{, }\AttributeTok{x =}\NormalTok{ media, }\AttributeTok{col =} \StringTok{"red"}\NormalTok{, }\AttributeTok{pch =} \DecValTok{19}\NormalTok{, }\AttributeTok{cex =} \FloatTok{1.5}\NormalTok{)}

\CommentTok{\# Adiciona uma legenda para o ponto da média}
\FunctionTok{legend}\NormalTok{(}\StringTok{"topleft"}\NormalTok{,}
       \AttributeTok{legend =} \StringTok{"Média=145,85"}\NormalTok{,}
       \AttributeTok{col =} \StringTok{"red"}\NormalTok{,}
       \AttributeTok{pch =} \DecValTok{19}\NormalTok{)}
\InformationTok{\textasciigrave{}\textasciigrave{}\textasciigrave{}}
\end{Highlighting}
\end{Shaded}

\pandocbounded{\includegraphics[keepaspectratio]{cap3-DistNorm_files/figure-pdf/unnamed-chunk-25-1.pdf}}

Para leitura e \ul{\textbf{\emph{interpretação adequada}}} de um
\textbf{\emph{boxplot}}, lembre-se de como ele é construído.

As duas figuras abaixo ilustram e operacionalizam os constructos
teóricos que são mobilizados na construção de um ou mais boxplots.

\begin{figure}[H]

{\centering \pandocbounded{\includegraphics[keepaspectratio]{fig/Moore-boxplot-fig2.1-p43.jpg}}

}

\caption{Diagramas em caixa que comparam os tempos de viagem para o
trabalho de amostras de trabalhadores na Carolina do Norte e em Nova
York. (Moore et al., 2023, p.~43)}

\end{figure}%

Agora o boxplot modificado para apresentar \textbf{\emph{outliers}}.

\begin{figure}[H]

{\centering \includegraphics[width=6.04167in,height=\textheight,keepaspectratio]{fig/Moore-boxplot-modificado-fig2.2-oulier-p45.jpg}

}

\caption{Diagramas horizontais modificados que comparam os tempos de
viagem para o trabalho de amostras de trabalhadores na Carolina do Norte
e em Nova York. (Moore et al., 2023, p.~45)}

\end{figure}%

Com essa recaptulação em mente, pode-se extrair as seguintes
\ul{\textbf{\emph{interpretações adequadas}}}.

O que \textbf{\emph{corrobora}} uma \textbf{\emph{forte assimetria à
direita}}, como era esperado segundo a \emph{literatura}.

Ver a \textbf{\emph{amplitude}} do \textbf{3º Q} (é o \emph{primeiro
segmento da AIQ}, de Q1 até mediana) em comparação à
\textbf{\emph{amplitude}} do \textbf{2º Q} (é o \emph{segundo segmento
da AIQ}, da mediana até Q3): \emph{mais que} o \emph{dobro}
(46,75/19,75≅2,4).

\emph{Idem} para a \textbf{\emph{área da caixa}} (\emph{box}) associado
ao \textbf{3º Q} (área em azul) em comparação à \textbf{\emph{área da
caixa}} (\emph{box})do \textbf{2º Q}: mais que o dobro (≅2,4, pois a
altura dos boxes é a mesma).

E o \textbf{\emph{bigode}} \textbf{\emph{à direita}} (tracejado desde o
\emph{box} até a cerca direita), mais que o dobro (≅2,4 neste caso) do
\textbf{\emph{bigode à esquerda}}.

Os \textbf{\emph{bigodes}} são as \emph{linhas que se estendem das
extremidades da caixa para os valores mínimo e máximo dos dados},
\emph{excluindo os valores atípicos} (\emph{outliers}).

E a presença de \textbf{\emph{6 outilers à direita}}, o que puxa a média
para a direita (média \textbf{\textgreater{}} mediana).

Com isso podemos considerar alcançados os objetivos da \textbf{8ª aula}
- Distribuições Normais (cap. 3) e Distribuições Amostrais (cap 15) dos
dados; nesta disciplina \textbf{CDE-a-DPP}.

\subsection{Curva Normal}\label{curva-normal-1}

\begin{Shaded}
\begin{Highlighting}[numbers=left,,]
\InformationTok{\textasciigrave{}\textasciigrave{}\textasciigrave{}\{r\}}
\CommentTok{\# Define os valores de z}
\NormalTok{z1 }\OtherTok{\textless{}{-}} \SpecialCharTok{{-}}\FloatTok{1.0}  \CommentTok{\# Limite inferior}
\NormalTok{z2 }\OtherTok{\textless{}{-}}  \FloatTok{1.0}  \CommentTok{\# Limite superior}

\CommentTok{\# Calcula a área sob a curva normal padrão entre z1 e z2}
\NormalTok{area }\OtherTok{\textless{}{-}} \FunctionTok{pnorm}\NormalTok{(z2, }\AttributeTok{mean =} \DecValTok{0}\NormalTok{, }\AttributeTok{sd =} \DecValTok{1}\NormalTok{) }\SpecialCharTok{{-}} \FunctionTok{pnorm}\NormalTok{(z1, }\AttributeTok{mean =} \DecValTok{0}\NormalTok{, }\AttributeTok{sd =} \DecValTok{1}\NormalTok{)}

\CommentTok{\# Exibe o resultado}
\FunctionTok{cat}\NormalTok{(}\StringTok{"A área sob a curva normal padrão entre"}\NormalTok{, z1, }\StringTok{"e"}\NormalTok{, z2, }\StringTok{"é:"}\NormalTok{, area, }\StringTok{"}\SpecialCharTok{\textbackslash{}n}\StringTok{"}\NormalTok{)}

\CommentTok{\# Gera sequência de valores para o eixo x}
\NormalTok{x }\OtherTok{\textless{}{-}} \FunctionTok{seq}\NormalTok{(}\SpecialCharTok{{-}}\DecValTok{3}\NormalTok{, }\DecValTok{3}\NormalTok{, }\AttributeTok{length =} \DecValTok{1000}\NormalTok{)}

\CommentTok{\# Calcula a densidade da normal padrão para cada x}
\NormalTok{y }\OtherTok{\textless{}{-}} \FunctionTok{dnorm}\NormalTok{(x, }\AttributeTok{mean =} \DecValTok{0}\NormalTok{, }\AttributeTok{sd =} \DecValTok{1}\NormalTok{)}

\CommentTok{\# Plota a curva normal padrão}
\FunctionTok{plot}\NormalTok{(x, y, }\AttributeTok{type =} \StringTok{"l"}\NormalTok{, }\AttributeTok{lwd =} \DecValTok{2}\NormalTok{, }\AttributeTok{col =} \StringTok{"blue"}\NormalTok{,}
     \AttributeTok{main =} \StringTok{"Curva Normal Padrão e Área entre z1 e z2"}\NormalTok{,}
     \AttributeTok{ylab =} \StringTok{"Densidade"}\NormalTok{, }\AttributeTok{xlab =} \StringTok{"z"}\NormalTok{)}

\CommentTok{\# Destaca a área sob a curva entre z1 e z2}
\NormalTok{x\_fill }\OtherTok{\textless{}{-}} \FunctionTok{seq}\NormalTok{(z1, z2, }\AttributeTok{length =} \DecValTok{500}\NormalTok{)}
\NormalTok{y\_fill }\OtherTok{\textless{}{-}} \FunctionTok{dnorm}\NormalTok{(x\_fill, }\AttributeTok{mean =} \DecValTok{0}\NormalTok{, }\AttributeTok{sd =} \DecValTok{1}\NormalTok{)}
\FunctionTok{polygon}\NormalTok{(}\FunctionTok{c}\NormalTok{(z1, x\_fill, z2), }\FunctionTok{c}\NormalTok{(}\DecValTok{0}\NormalTok{, y\_fill, }\DecValTok{0}\NormalTok{), }\AttributeTok{col =} \FunctionTok{rgb}\NormalTok{(}\DecValTok{1}\NormalTok{, }\DecValTok{0}\NormalTok{, }\DecValTok{0}\NormalTok{, }\FloatTok{0.5}\NormalTok{), }\AttributeTok{border =} \ConstantTok{NA}\NormalTok{)}

\CommentTok{\# Adiciona linhas verticais nos limites}
\FunctionTok{abline}\NormalTok{(}\AttributeTok{v =} \FunctionTok{c}\NormalTok{(z1, z2), }\AttributeTok{col =} \StringTok{"red"}\NormalTok{, }\AttributeTok{lty =} \DecValTok{2}\NormalTok{)}
\InformationTok{\textasciigrave{}\textasciigrave{}\textasciigrave{}}
\end{Highlighting}
\end{Shaded}

\begin{verbatim}
A área sob a curva normal padrão entre -1 e 1 é: 0.6826895 
\end{verbatim}

\pandocbounded{\includegraphics[keepaspectratio]{cap3-DistNorm_files/figure-pdf/unnamed-chunk-26-1.pdf}}

\subsection{Até breve}\label{atuxe9-breve-1}

Dúvidas serão debeladas a cada aula!

\begin{figure}[H]

{\centering \pandocbounded{\includegraphics[keepaspectratio]{fig/ValeuGalera.png}}

}

\caption{Até nosso pRRRóximo RRRencontro!}

\end{figure}%

\bookmarksetup{startatroot}

\chapter{AED - cap 5 pldr - Ajustando Modelos aos
Dados}\label{sec-model-dados}

\begin{quote}
Uma das atividades fundamentais da estatística é criar modelos que
possam sumarizar os dados com um conjunto pequeno de números,
fornecendo, assim, uma descrição compacta deles, aliada a uma medida de
nossa incerteza sobre essa descrição. Neste capítulo, analisaremos o
conceito de modelo estatístico e como ele pode ser usado para descrever
os dados.. (Poldrack, 2025 , p.~41).
\end{quote}

\begin{Shaded}
\begin{Highlighting}[numbers=left,,]
\InformationTok{\textasciigrave{}\textasciigrave{}\textasciigrave{}\{r\}}
\FunctionTok{library}\NormalTok{(tidyverse)}
\FunctionTok{library}\NormalTok{(NHANES)}
\FunctionTok{library}\NormalTok{(cowplot)}
\FunctionTok{library}\NormalTok{(mapproj)}
\FunctionTok{library}\NormalTok{(pander)}
\FunctionTok{library}\NormalTok{(knitr)}
\FunctionTok{library}\NormalTok{(modelr)}

\FunctionTok{panderOptions}\NormalTok{(}\StringTok{\textquotesingle{}round\textquotesingle{}}\NormalTok{,}\DecValTok{2}\NormalTok{)}
\FunctionTok{panderOptions}\NormalTok{(}\StringTok{\textquotesingle{}digits\textquotesingle{}}\NormalTok{,}\DecValTok{7}\NormalTok{)}
\FunctionTok{theme\_set}\NormalTok{(}\FunctionTok{theme\_minimal}\NormalTok{(}\AttributeTok{base\_size =} \DecValTok{14}\NormalTok{))}

\FunctionTok{options}\NormalTok{(}\AttributeTok{digits =} \DecValTok{2}\NormalTok{)}
\FunctionTok{set.seed}\NormalTok{(}\DecValTok{123456}\NormalTok{) }\CommentTok{\# set random seed to exactly replicate results}
\InformationTok{\textasciigrave{}\textasciigrave{}\textasciigrave{}}
\end{Highlighting}
\end{Shaded}

\section{Objetivos da Aprendizagem}\label{objetivos-da-aprendizagem}

\begin{quote}
Após ler este capítulo, você deve ser capaz de:

▶Descrever a equação básica para modelos estatísticos (dados = modelo +
erro).

▶Descrever diferentes medidas de tendência central e dispersão, como são
calculadas e quais são apropriadas em quais circunstâncias.

▶Calcular o Z-score {[}escore-Z{]} e descrever por que é útil.
(Poldrack, 2025 , p.~41).
\end{quote}

\section{O que É um Modelo?}\label{o-que-uxe9-um-modelo}

\begin{quote}
No mundo tangível, \textbf{\emph{modelos geralmente são simplificações}}
de \textbf{\emph{objetos}} do \textbf{\emph{mundo real}} que, ainda
assim, \textbf{\emph{expressam a essência}} do que está sendo
\emph{modelado}.

Um \emph{modelo} de um edifício traduz a estrutura da sua construção,
sendo ao mesmo tempo pequeno e leve o suficiente para ser manuseado com
as mãos; em biologia, um modelo de célula é muito maior que a célula
propriamente dita, mas, novamente, representa suas partes principais e
suas relações.

Em \emph{estatística}, um \textbf{\emph{modelo}} tem o
\textbf{\emph{objetivo}} de \textbf{\emph{fornecer uma descrição
condensada}} parecida, porém \textbf{\emph{aplicada a dados}} e não a
uma estrutura física.

Assim como modelos físicos, em geral, um modelo estatístico é bem
\textbf{\emph{mais simples}} do \textbf{\emph{que}} os
\textbf{\emph{dados que descreve}}; ele serve para
\textbf{\emph{capturar a estrutura}} deles da \textbf{\emph{forma mais
simples possível}}.

Em ambos os casos, percebemos que o \ul{\textbf{\emph{modelo}}}
\textbf{\emph{é uma ficção conveniente}} que
\textbf{\emph{necessariamente ignora alguns detalhes}} do
\textbf{\emph{objeto real}} que está sendo \textbf{\emph{modelado}}.

Como disse o famoso estatístico George Box: \ul{\textbf{``Todos os
modelos estão errados, mas alguns são úteis''}}.

Além disso, \textbf{\emph{pode ser útil}} \ul{\textbf{\emph{considerar
um modelo estatístico como uma teoria de como os dados observados foram
gerados}}}; nosso \textbf{\emph{objetivo}} então se torna
\ul{\textbf{\emph{encontrar aquele que sumariza}}} \textbf{\emph{com
mais \ul{eficiência e acurácia}}} a maneira \ul{\textbf{\emph{como os
dados}}} foram verdadeiramente \ul{\textbf{\emph{gerados}}}.

No entanto, como veremos a seguir, os desejos de eficiência e de
acurácia, não raro, são radicalmente contrários entre si.

A estrutura básica de um modelo estatístico é:

\[
dados = modelo + erros
\]

Isso \emph{expressa a \textbf{ideia}} de que os
\ul{\textbf{\emph{dados}}} podem ser \ul{\textbf{\emph{divididos}}} em
\ul{\textbf{\emph{duas partes}}}: \textbf{\emph{uma que é descrita por}}
um \ul{\textbf{\emph{modelo estatístico}}}, expressando os
\textbf{\emph{valores que esperamos que os dados assumam devido ao nosso
conhecimento}}, e \ul{\textbf{\emph{outra parte}}} que chamamos de
\ul{\textbf{\emph{erro}}}, retratando a \textbf{\emph{diferença entre as
predições do modelo e os dados observados}}.

Basicamente, gostaríamos de usar nosso modelo a fim de
\ul{\textbf{\emph{predizer}}} o valor dos dados para qualquer
observação. Escreveríamos a equação assim:

\[
\widehat{dados_i} = modelo_i
\]

O \textbf{``chapéu''} sobre \emph{dados} indica que se trata de uma
estimativa, e não do valor real deles.

Ou seja, o \textbf{\emph{valor predito}} dos dados para a
\emph{observação i} é \emph{igual} ao \emph{valor do modelo} para
\emph{essa observação}.

Assim que tivermos uma predição do modelo, podemos calcular o
\ul{\textbf{\emph{erro}}} \ul{{[}}resíduo\ul{{]}}: Ou seja, o erro para
qualquer \emph{observação i} é a \textbf{\emph{diferença entre o valor
observado dos dados e o valor predito dos dados a partir do modelo}}.

(Poldrack, 2025 , p.~41-42).
\end{quote}

\section{Modelagem Estatística: Um
Exemplo}\label{modelagem-estatuxedstica-um-exemplo}

\begin{quote}
Vejamos um exemplo de criação de um modelo para os dados, usando os
dados do NHANES. Em termos específicos, tentaremos criar um modelo da
altura das crianças na amostra do NHANES. Primeiro, carregaremos e
plotaremos os dados (Figura 5.1). (Poldrack, 2025 , p.~42)
\end{quote}

\begin{Shaded}
\begin{Highlighting}[numbers=left,,]
\InformationTok{\textasciigrave{}\textasciigrave{}\textasciigrave{}\{r\}}
\CommentTok{\# drop duplicated IDs within the NHANES dataset}
\NormalTok{NHANES }\OtherTok{\textless{}{-}}
\NormalTok{  NHANES }\SpecialCharTok{\%\textgreater{}\%}
\NormalTok{  dplyr}\SpecialCharTok{::}\FunctionTok{distinct}\NormalTok{(ID, }\AttributeTok{.keep\_all =} \ConstantTok{TRUE}\NormalTok{)}

\CommentTok{\# select the appropriate children with good height measurements}

\NormalTok{NHANES\_child }\OtherTok{\textless{}{-}}
\NormalTok{  NHANES }\SpecialCharTok{\%\textgreater{}\%}
  \FunctionTok{drop\_na}\NormalTok{(Height) }\SpecialCharTok{\%\textgreater{}\%}
  \FunctionTok{subset}\NormalTok{(Age }\SpecialCharTok{\textless{}} \DecValTok{18}\NormalTok{)}

\NormalTok{NHANES\_child }\SpecialCharTok{\%\textgreater{}\%} \FunctionTok{nrow}\NormalTok{()}

\NormalTok{NHANES\_child }\SpecialCharTok{\%\textgreater{}\%}
  \FunctionTok{ggplot}\NormalTok{(}\FunctionTok{aes}\NormalTok{(Height)) }\SpecialCharTok{+}
  \FunctionTok{geom\_histogram}\NormalTok{(}\AttributeTok{bins =} \DecValTok{100}\NormalTok{) }\SpecialCharTok{+} 
  \FunctionTok{labs}\NormalTok{(}
  \AttributeTok{title =} \StringTok{"Histograma da altura [Height]: 1691 crianças no NHANES"}\NormalTok{,}
  \AttributeTok{subtitle =} \StringTok{"filtro: idade (age) \textless{} 18 anos"}\NormalTok{,}
  \AttributeTok{caption =} \StringTok{"Fonte: Poldrack, 2025, p. 42, fig. 5.1"}\NormalTok{,}
  \AttributeTok{x =} \StringTok{"Altura (Height)"}\NormalTok{,}
  \AttributeTok{y =} \StringTok{"Contagem (Count)"}
\NormalTok{)}
\InformationTok{\textasciigrave{}\textasciigrave{}\textasciigrave{}}
\end{Highlighting}
\end{Shaded}

\begin{verbatim}
[1] 1691
\end{verbatim}

\pandocbounded{\includegraphics[keepaspectratio]{cap5-pldr-modelos-dados_files/figure-pdf/unnamed-chunk-2-1.pdf}}

Lembre que queremos descrever os dados do modo mais simples possível,
porém capturando suas features mais importantes. O modelo mais simples
imaginável teria apenas um número; ou seja, prediria o mesmo valor para
cada observação, independentemente de qualquer outra informação que
temos sobre essas observações. Em geral, descrevemos um modelo conforme
seus parâmetros, valores que podemos alterar para modificar as predições
dele. No decorrer desta obra, usamos a letra grega beta (β) para
representá-los; quando o modelo tem mais de um parâmetro, usamos números
subscritos para diferenciar as letras betas (por exemplo, β1). É comum
também referenciar os valores dos dados usando a letra y e observações
individuais usando uma versão com subscrita (yi).

Como geralmente não conhecemos os valores lógicos dos parâmetros,
precisamos estimá-los a partir dos dados. Por isso, costumamos colocar
um ``chapéu'' sobre o símbolo β para indicar que estamos usando uma
estimativa do valor do parâmetro, e não o seu valor lógico.

Assim, nosso \textbf{\emph{modelo simples}} para \textbf{\emph{altura
considerando um único parâmetro}} seria:

\[
y_i = \hat{\beta} + \epsilon
\]

O i subscrito não aparece ao lado direito da equação, significando que a
predição do modelo não depende da observação específica que estamos
analisando --- é a mesma para todas. A pergunta então é: como estimamos
os melhores valores do(s) parâmetro(s) no modelo? Nesse caso específico,
qual valor único seria a melhor estimativa para β? E, mais importante,
como definimos melhor?

\textbf{\emph{Um estimador simples}} seria a \textbf{\emph{moda}}, que é
simplesmente o valor mais comum no conjunto de dados. Ela redescreveria
todo o conjunto de 1.691 crianças em termos de um único número. Se
quiséssemos predizer a altura de qualquer nova criança, nosso valor
predito seria o mesmo número:

\[
y_i = 166.5 \text{ para todos os i}
\]

O \textbf{\emph{erro}} {[}também denominado por \emph{resíduo}{]} para
cada sujeito de pesquisa seria então a diferença entre o valor predito
\((\hat{y}_i)\) e sua altura real \((y_i)\):

\[
erro_i = y_i - \hat{y}_i
\]

Qual a eficácia deste modelo? Em geral, definimos a qualidade de um
modelo em termos da magnitude do erro, que representa o quanto os dados
divergem das predições; em condições iguais, aquele que gera o menor
erro é o melhor modelo. (Mas, como veremos mais tarde, as condições nem
sempre são iguais\ldots). No caso de colocar a moda como estimador para
β, descobrimos que o erro médio individual é de consideráveis −28,8
centímetros, o que não parece muito bom à primeira vista.

Como podemos encontrar um melhor estimador para o parâmetro do nosso
modelo? Podemos começar tentando encontrar um estimador que nos forneça
um erro médio de 0. Um bom candidato é a média aritmética (também
chamada simplesmente de média, geralmente representada por uma barra
sobre a variável, como \(\overline{X}\)), calculada como a soma de todos
os valores, dividida pelo número de valores. Matematicamente, ela é
expressada assim:

\[
\bar{X} = \frac{\sum_{i=1}^{n} x_i}{n}
\]

Acontece que, se usarmos a média aritmética como nosso estimador, o erro
médio é matematicamente garantido como 0 (veja a comprovação simples no
final do capítulo, caso esteja interessado).

Embora a média dos erros a partir da média seja 0, no histograma da
Figura 5.2, é possível ver que cada item individual ainda tem algum grau
de erro; alguns são positivos, outros, negativos, e \textbf{\emph{eles
se anulam mutuamente para fornecer um erro médio de 0}}.

\begin{Shaded}
\begin{Highlighting}[numbers=left,,]
\InformationTok{\textasciigrave{}\textasciigrave{}\textasciigrave{}\{r\}}
\CommentTok{\# compute error compared to the mean and plot histogram}

\NormalTok{error\_mean }\OtherTok{\textless{}{-}}\NormalTok{ NHANES\_child}\SpecialCharTok{$}\NormalTok{Height }\SpecialCharTok{{-}} \FunctionTok{mean}\NormalTok{(NHANES\_child}\SpecialCharTok{$}\NormalTok{Height)}

\FunctionTok{ggplot}\NormalTok{(}\ConstantTok{NULL}\NormalTok{, }\FunctionTok{aes}\NormalTok{(error\_mean)) }\SpecialCharTok{+}
  \FunctionTok{geom\_histogram}\NormalTok{(}\AttributeTok{bins =} \DecValTok{100}\NormalTok{) }\SpecialCharTok{+}
  \FunctionTok{xlim}\NormalTok{(}\SpecialCharTok{{-}}\DecValTok{60}\NormalTok{, }\DecValTok{60}\NormalTok{) }\SpecialCharTok{+}
  \FunctionTok{labs}\NormalTok{(}
    \AttributeTok{title =} \StringTok{"Histograma da distribuição de erros a partir da média"}\NormalTok{,}
  \AttributeTok{subtitle =} \StringTok{"1691 crianças no NHANES com filtro: idade (age) \textless{} 18 anos"}\NormalTok{,}
  \AttributeTok{caption =} \StringTok{"Fonte: Poldrack, 2025, p. 44, fig. 5.2"}\NormalTok{,}
  \AttributeTok{x =} \StringTok{"Erro ao predizer a Altura (Height) com a média"}\NormalTok{,}
  \AttributeTok{y =} \StringTok{"Contagem (Count)"}
\NormalTok{  )}
\InformationTok{\textasciigrave{}\textasciigrave{}\textasciigrave{}}
\end{Highlighting}
\end{Shaded}

\pandocbounded{\includegraphics[keepaspectratio]{cap5-pldr-modelos-dados_files/figure-pdf/unnamed-chunk-3-1.pdf}}

Gerar um boxplot para verificar a existência de
\textbf{\emph{outliers}}.

Não foram detectados NA's nem outliers no conjunto de dados plotados.

\begin{Shaded}
\begin{Highlighting}[numbers=left,,]
\InformationTok{\textasciigrave{}\textasciigrave{}\textasciigrave{}\{r\}}
\FunctionTok{is.na}\NormalTok{(error\_mean) }\SpecialCharTok{\%\textgreater{}\%} \FunctionTok{sum}\NormalTok{() }\CommentTok{\# verifica se há NA\textquotesingle{}s}

\NormalTok{error\_mean }\SpecialCharTok{\%\textgreater{}\%} \FunctionTok{summary}\NormalTok{() }\CommentTok{\# resumo dos 5 números e média}

\FunctionTok{ggplot}\NormalTok{(}\ConstantTok{NULL}\NormalTok{, }\FunctionTok{aes}\NormalTok{(error\_mean)) }\SpecialCharTok{+}
  \FunctionTok{geom\_boxplot}\NormalTok{() }\SpecialCharTok{+}
  \FunctionTok{xlim}\NormalTok{(}\SpecialCharTok{{-}}\DecValTok{60}\NormalTok{, }\DecValTok{60}\NormalTok{) }\SpecialCharTok{+}
  \FunctionTok{labs}\NormalTok{(}
  \AttributeTok{title    =} \StringTok{"Boxplot da altura [Height]: 1691 crianças no NHANES"}\NormalTok{,}
  \AttributeTok{subtitle =} \StringTok{"filtro: idade (age) \textless{} 18 anos"}\NormalTok{,}
  \AttributeTok{caption  =} \StringTok{"Fonte: cleuler"}\NormalTok{,}
  \AttributeTok{x =} \StringTok{"Erro ao predizer a Altura (Height) com a média"}\NormalTok{,}
  \AttributeTok{y =} \StringTok{"Contagem (Count)"}
\NormalTok{  )}
\InformationTok{\textasciigrave{}\textasciigrave{}\textasciigrave{}}
\end{Highlighting}
\end{Shaded}

\begin{verbatim}
[1] 0
   Min. 1st Qu.  Median    Mean 3rd Qu.    Max. 
  -54.1   -23.0     2.5     0.0    23.0    55.6 
\end{verbatim}

\pandocbounded{\includegraphics[keepaspectratio]{cap5-pldr-modelos-dados_files/figure-pdf/unnamed-chunk-4-1.pdf}}

O fato de os erros negativos e positivos se anularem implica que dois
modelos diferentes poderiam ter erros de magnitude bem distintos em
termos absolutos, mas que, ainda assim, teriam o mesmo erro médio.

É exatamente por isso que o \textbf{\emph{erro médio não é um bom
critério para nosso estimador}}; \textbf{\emph{queremos um critério que
minimize o erro geral, independentemente da direção}}.

Por esse motivo, normalmente sumarizamos os erros usando algum tipo de
medida que considere indesejáveis aqueles positivos e negativos.

Poderíamos usar o valor absoluto de cada erro, porém é
\textbf{\emph{mais comum usar os erros quadráticos}}, por razões que
analisaremos mais adiante no livro.

Ao longo desta obra, os leitores verão \textbf{\emph{diversas formas
comuns de sumarizar o erro quadrático}}.

Por isso, é importante entender como elas se relacionam.

\ul{\textbf{Primeiro}}, poderíamos simplesmente somá-los; isso se chama
\textbf{\emph{soma dos erros quadráticos}}.

Normalmente, o motivo pelo qual não usamos essa medida é que
\textbf{\emph{sua magnitude depende do número de pontos de dados}},
podendo \textbf{\emph{dificultar a interpretação}}, \textbf{\emph{a
menos que estejamos analisando o mesmo número de observações}}.

\ul{\textbf{Segundo}}, poderíamos calcular a \textbf{\emph{média dos
valores do erro quadrático}}, conhecida como \ul{\textbf{\emph{erro
quadrático médio}}} (\ul{\textbf{MSE}} \ul{\textbf{{[}Mean Squared
Error{]}}}).

No entanto, como elevamos os valores ao quadrado antes de calcular a
média, eles \textbf{\emph{não estão na mesma escala dos dados
originais}}; estão em centímetros ao quadrado.

Por isso, também \ul{\textbf{\emph{é comum obter a raiz quadrada do
MSE}}}, que chamamos de \ul{\textbf{\emph{raiz do erro quadrático
médio}}} (\ul{\textbf{RMSE}} - {[}\textbf{\emph{Root Mean Squared
Error}}{]}), de modo que o erro seja calculado \ul{\textbf{\emph{nas
mesmas unidades dos valores originais}}} (nesse exemplo, em
centímetros).

A \ul{\textbf{média}} tem \textbf{\emph{erros consideráveis}} ---
\emph{qualquer ponto individual de dado estará}, em \emph{média}, a
cerca de \textbf{\emph{27 cm da média}} ---, porém ainda \textbf{\emph{é
melhor que a \ul{moda}}}, a qual tem \textbf{\emph{um erro quadrático
médio}} de cerca de \textbf{\emph{39 cm}}.

\section{Aprimorando Nosso Modelo}\label{aprimorando-nosso-modelo}

Podemos imaginar um modelo melhor?

Lembre que esses dados abrangem todas as crianças na amostra do NHANES,
com idades entre 2 e 17 anos {[}filtro foi variável \texttt{age}
\textless{} 18{]}.

Dada essa \emph{ampla faixa etária}, poderíamos \textbf{\emph{esperar}}
que nosso \ul{\textbf{\emph{modelo}}} \textbf{\emph{de \ul{altura}
também considerasse a idade}} {[}\texttt{age}, e \emph{não apenas} a
\ul{\emph{média}} ou \ul{\emph{moda}} do \emph{conjunto de dados de
alturas das crianças e adolescentes}{]}.

\textbf{\emph{Plotaremos os dados de altura em relação à idade}} para
\textbf{\emph{verificar se}} essa \ul{\textbf{\emph{relação realmente
existe}}}.

No painel A da Figura 5.3, os \emph{pontos demonstram pessoas no
conjunto de dados}, e, como seria \textbf{\emph{de esperar}},
aparentemente existe \textbf{\emph{uma forte relação entre altura e
idade}}.

Desse modo, podemos \textbf{\emph{criar um modelo que relacione as
duas}}:

\[
\hat{y}_i = \hat{\beta} \times idade_i
\]

em que \(\hat{\beta}\) é a nossa \ul{\textbf{\emph{estimativa}}} do
\ul{\textbf{\emph{parâmetro}}}, que multiplicamos pela \texttt{idade} a
fim de gerar a \ul{\textbf{\emph{predição}}} do modelo.

Talvez você se recorde de que, em álgebra, \textbf{\emph{uma linha é
definida}} da seguinte \ul{\textbf{\emph{forma}}}:

\[
y = inclinação \times x + intercepto
\]

Se a \textbf{\emph{idade}} for a variável \texttt{x}, significa que
nossa \textbf{\emph{predição de altura}} a \emph{partir da idade} será
\emph{uma \ul{\textbf{linha}}} {[}uma \textbf{reta}{]} \emph{com}
\ul{\textbf{\emph{inclinação}}} β e \ul{\textbf{\emph{intercepto}}} 0
{[}zero{]}.

Para visualizar isso, plotaremos a linha {[}a reta{]} com o ajuste mais
adequado no topo dos dados (painel B da Figura 5.3).

\begin{Shaded}
\begin{Highlighting}[numbers=left,,]
\InformationTok{\textasciigrave{}\textasciigrave{}\textasciigrave{}\{r\}}
\FunctionTok{library}\NormalTok{(gridExtra)}
\FunctionTok{library}\NormalTok{(grid)}

\CommentTok{\# compute and print RMSE for mean and mode}
\NormalTok{rmse\_mean }\OtherTok{\textless{}{-}} \FunctionTok{sqrt}\NormalTok{(}\FunctionTok{mean}\NormalTok{(error\_mean}\SpecialCharTok{**}\DecValTok{2}\NormalTok{))}

\CommentTok{\# from https://www.tutorialspoint.com/r/r\_mean\_median\_mode.htm}
\NormalTok{getmode }\OtherTok{\textless{}{-}} \ControlFlowTok{function}\NormalTok{(v) \{}
\NormalTok{   uniqv }\OtherTok{\textless{}{-}} \FunctionTok{unique}\NormalTok{(v)}
\NormalTok{   uniqv[}\FunctionTok{which.max}\NormalTok{(}\FunctionTok{tabulate}\NormalTok{(}\FunctionTok{match}\NormalTok{(v, uniqv)))]}
\NormalTok{\}}

\NormalTok{error\_mode }\OtherTok{\textless{}{-}}\NormalTok{ NHANES\_child}\SpecialCharTok{$}\NormalTok{Height }\SpecialCharTok{{-}} \FunctionTok{getmode}\NormalTok{(NHANES\_child}\SpecialCharTok{$}\NormalTok{Height)}
\NormalTok{rmse\_mode  }\OtherTok{\textless{}{-}} \FunctionTok{sqrt}\NormalTok{(}\FunctionTok{mean}\NormalTok{(error\_mode}\SpecialCharTok{**}\DecValTok{2}\NormalTok{))}

\NormalTok{p1 }\OtherTok{\textless{}{-}}\NormalTok{ NHANES\_child }\SpecialCharTok{\%\textgreater{}\%}
  \FunctionTok{ggplot}\NormalTok{(}\FunctionTok{aes}\NormalTok{(}\AttributeTok{x =}\NormalTok{ Age, }\AttributeTok{y =}\NormalTok{ Height)) }\SpecialCharTok{+}
  \FunctionTok{geom\_point}\NormalTok{(}\AttributeTok{position =} \StringTok{"jitter"}\NormalTok{,}\AttributeTok{size=}\FloatTok{0.05}\NormalTok{) }\SpecialCharTok{+}
  \FunctionTok{scale\_x\_continuous}\NormalTok{(}\AttributeTok{breaks =} \FunctionTok{seq.int}\NormalTok{(}\DecValTok{0}\NormalTok{, }\DecValTok{20}\NormalTok{, }\DecValTok{2}\NormalTok{)) }\SpecialCharTok{+}
  \FunctionTok{ggtitle}\NormalTok{(}\StringTok{\textquotesingle{}A: original data\textquotesingle{}}\NormalTok{)}

\NormalTok{lmResultHeightOnly }\OtherTok{\textless{}{-}} \FunctionTok{lm}\NormalTok{(Height }\SpecialCharTok{\textasciitilde{}}\NormalTok{ Age }\SpecialCharTok{+} \DecValTok{0}\NormalTok{, }\AttributeTok{data=}\NormalTok{NHANES\_child)}
\NormalTok{rmse\_heightOnly    }\OtherTok{\textless{}{-}} \FunctionTok{sqrt}\NormalTok{(}\FunctionTok{mean}\NormalTok{(lmResultHeightOnly}\SpecialCharTok{$}\NormalTok{residuals}\SpecialCharTok{**}\DecValTok{2}\NormalTok{))}

\NormalTok{p2 }\OtherTok{\textless{}{-}}\NormalTok{ NHANES\_child }\SpecialCharTok{\%\textgreater{}\%}
  \FunctionTok{ggplot}\NormalTok{(}\FunctionTok{aes}\NormalTok{(}\AttributeTok{x =}\NormalTok{ Age, }\AttributeTok{y =}\NormalTok{ Height)) }\SpecialCharTok{+}
  \FunctionTok{geom\_point}\NormalTok{(}\AttributeTok{position =} \StringTok{"jitter"}\NormalTok{,}\AttributeTok{size=}\FloatTok{0.05}\NormalTok{) }\SpecialCharTok{+}
  \FunctionTok{scale\_x\_continuous}\NormalTok{(}\AttributeTok{breaks =} \FunctionTok{seq.int}\NormalTok{(}\DecValTok{0}\NormalTok{, }\DecValTok{20}\NormalTok{, }\DecValTok{2}\NormalTok{)) }\SpecialCharTok{+}
  \FunctionTok{annotate}\NormalTok{(}\StringTok{\textquotesingle{}segment\textquotesingle{}}\NormalTok{,}\AttributeTok{x=}\DecValTok{0}\NormalTok{,}\AttributeTok{xend=}\FunctionTok{max}\NormalTok{(NHANES\_child}\SpecialCharTok{$}\NormalTok{Age),}
           \AttributeTok{y=}\DecValTok{0}\NormalTok{,}\AttributeTok{yend=}\FunctionTok{max}\NormalTok{(lmResultHeightOnly}\SpecialCharTok{$}\NormalTok{fitted.values),}
           \AttributeTok{color=}\StringTok{\textquotesingle{}blue\textquotesingle{}}\NormalTok{,}\AttributeTok{lwd=}\DecValTok{1}\NormalTok{) }\SpecialCharTok{+}
  \FunctionTok{ggtitle}\NormalTok{(}\StringTok{\textquotesingle{}B: age\textquotesingle{}}\NormalTok{)}

\NormalTok{p3 }\OtherTok{\textless{}{-}}\NormalTok{ NHANES\_child }\SpecialCharTok{\%\textgreater{}\%}
  \FunctionTok{ggplot}\NormalTok{(}\FunctionTok{aes}\NormalTok{(}\AttributeTok{x =}\NormalTok{ Age, }\AttributeTok{y =}\NormalTok{ Height)) }\SpecialCharTok{+}
  \FunctionTok{geom\_point}\NormalTok{(}\AttributeTok{position =} \StringTok{"jitter"}\NormalTok{,}\AttributeTok{size=}\FloatTok{0.05}\NormalTok{) }\SpecialCharTok{+}
  \FunctionTok{scale\_x\_continuous}\NormalTok{(}\AttributeTok{breaks =} \FunctionTok{seq.int}\NormalTok{(}\DecValTok{0}\NormalTok{, }\DecValTok{20}\NormalTok{, }\DecValTok{2}\NormalTok{)) }\SpecialCharTok{+}
  \FunctionTok{geom\_smooth}\NormalTok{(}\AttributeTok{method=}\StringTok{\textquotesingle{}lm\textquotesingle{}}\NormalTok{,}\AttributeTok{se=}\ConstantTok{FALSE}\NormalTok{) }\SpecialCharTok{+}
  \FunctionTok{ggtitle}\NormalTok{(}\StringTok{\textquotesingle{}C: age + constant\textquotesingle{}}\NormalTok{)}

\NormalTok{p4 }\OtherTok{\textless{}{-}}\NormalTok{ NHANES\_child }\SpecialCharTok{\%\textgreater{}\%}
  \FunctionTok{ggplot}\NormalTok{(}\FunctionTok{aes}\NormalTok{(}\AttributeTok{x =}\NormalTok{ Age, }\AttributeTok{y =}\NormalTok{ Height)) }\SpecialCharTok{+}
  \FunctionTok{geom\_point}\NormalTok{(}\FunctionTok{aes}\NormalTok{(}\AttributeTok{colour =} \FunctionTok{factor}\NormalTok{(Gender)),}
             \AttributeTok{position =} \StringTok{"jitter"}\NormalTok{,}
             \AttributeTok{alpha =} \FloatTok{0.8}\NormalTok{,}
             \AttributeTok{size=}\FloatTok{0.05}\NormalTok{) }\SpecialCharTok{+}
  \FunctionTok{geom\_smooth}\NormalTok{(}\AttributeTok{method=}\StringTok{\textquotesingle{}lm\textquotesingle{}}\NormalTok{,}\FunctionTok{aes}\NormalTok{(}\AttributeTok{group =} \FunctionTok{factor}\NormalTok{(Gender),}
                              \AttributeTok{colour =} \FunctionTok{factor}\NormalTok{(Gender))) }\SpecialCharTok{+}
  \FunctionTok{theme}\NormalTok{(}\AttributeTok{legend.position =} \FunctionTok{c}\NormalTok{(}\FloatTok{0.25}\NormalTok{,}\FloatTok{0.8}\NormalTok{)) }\SpecialCharTok{+}
  \FunctionTok{scale\_color\_discrete}\NormalTok{(}\AttributeTok{name =} \ConstantTok{NULL}\NormalTok{) }\SpecialCharTok{+}
  \FunctionTok{ggtitle}\NormalTok{(}\StringTok{\textquotesingle{}D: age + constant + gender\textquotesingle{}}\NormalTok{)}


\CommentTok{\# Criando a nota de rodapé como grob com tamanho de fonte personalizado}
\NormalTok{nota\_rodape }\OtherTok{\textless{}{-}} \FunctionTok{textGrob}\NormalTok{(}
  \StringTok{"Altura [Height] das crianças no NHANES, plotadas sem um modelo (A, Original data [Dados originais]); com um modelo linear incluindo apenas}\SpecialCharTok{\textbackslash{}n}\StringTok{idade (B, Age [Idade]) ou incluindo idade e uma constante (C, Age + constant [Idade + constante]) e com um modelo linear que ajusta efeitos}\SpecialCharTok{\textbackslash{}n}\StringTok{diferentes de idade para homens e mulheres (D, Age + constant + gender [Idade + constante + gênero]). Fonte: Poldrack, 2025, p. 46, fig. 5.3"}\NormalTok{,}
  \AttributeTok{gp =} \FunctionTok{gpar}\NormalTok{(}\AttributeTok{fontsize =} \DecValTok{8}\NormalTok{,}
            \AttributeTok{fontface =} \StringTok{"italic"}\NormalTok{),}
  \AttributeTok{x =} \FloatTok{0.5}\NormalTok{,}
  \AttributeTok{hjust =} \FloatTok{0.5}
\NormalTok{)}

\CommentTok{\# Usando grid.arrange com a nota de rodapé customizada}
\CommentTok{\# usando grid.arrange() (do pacote gridExtra), pode usar o argumento bottom:}
\CommentTok{\# para gerar um nota de rodapé ´no gráfico final}
\FunctionTok{grid.arrange}\NormalTok{(}
\NormalTok{  p1, p2, p3, p4,}
  \AttributeTok{ncol =} \DecValTok{2}\NormalTok{,}
  \AttributeTok{bottom =}\NormalTok{ nota\_rodape}
\NormalTok{)}
\InformationTok{\textasciigrave{}\textasciigrave{}\textasciigrave{}}
\end{Highlighting}
\end{Shaded}

\pandocbounded{\includegraphics[keepaspectratio]{cap5-pldr-modelos-dados_files/figure-pdf/unnamed-chunk-5-1.pdf}}

Algo claramente está errado com esse \textbf{segundo modelo {[}B{]}},
pois a \emph{linha não parece seguir os dados muito bem}.

Na verdade, sua \textbf{RMSE} (39,16 cm) é bem \textbf{\emph{maior}} do
que a do \textbf{\emph{modelo que inclui apenas a média}}!

O problema surge do fato de \textbf{B} considerar
\textbf{\emph{somente}} a \texttt{idade}, significando que o
\textbf{\emph{valor predito}} de \texttt{altura} \([\hat{y}_i]\) a
partir dele deve \textbf{\emph{assumir}} um valor de \textbf{0}
{[}zero{]} quando a \texttt{idade} for \textbf{0} {[}zero{]}.

Ainda que os dados não incluam crianças com idade 0, matematicamente,
exige-se que a linha tenha um valor y de 0 quando x é 0, o que explica
por que a ela é forçada para baixo dos pontos de dados que representam
pessoas mais jovens.

Podemos \textbf{\emph{corrigir}} isso \textbf{\emph{incluindo um
intercepto}} em \textbf{\emph{nosso modelo}}, que basicamente
\textbf{\emph{representa}} a \texttt{altura} \textbf{\emph{estimada}}
\([\hat{y}_i]\) quando a \texttt{idade} é igual a 0; \emph{embora},
\emph{nesse conjunto de dados}, \emph{uma idade de 0 não seja
plausível}, é \emph{uma estratégia matemática} que permite ao
\emph{modelo considerar a magnitude geral dos dados}. Assim:

\[
\hat{y}_i = \hat{\beta_0} + \hat{\beta_1} \times idade_i
\]

Em que \(\hat{\beta_0}\) é nossa \textbf{\emph{estimativa}} para o
\textbf{\emph{intercepto}}, \emph{um valor constante adicionado à
predição para cada indivíduo}; chamamos isso de intercepto, pois ele
mapeia o intercepto na equação para uma linha reta.

Posteriormente, aprenderemos como estimamos esses valores de parâmetros
para um conjunto de dados específico; por ora, \textbf{\emph{usaremos
nosso software estatístico para estimar os valores dos parâmetros que
nos fornecem o menor erro para esses dados específicos}}.

Na Figura 5.3, o painel \textbf{C} mostra \textbf{\emph{esse}}
\textbf{\emph{modelo aplicado aos dados do NHANES}}.

Vemos que a \textbf{\emph{linha se ajusta melhor aos dados do que a
anterior}}, \emph{sem uma constante}.

Nosso \textbf{\emph{erro}} é \textbf{\emph{bem menor com esse novo
modelo}} --- apenas \textbf{8,36 cm} em média {[}\textbf{RMSE}{]}.

Consegue \ul{\textbf{\emph{pensar em outras variáveis}}}
{[}\textbf{ocultas}{]} que também possam estar
\textbf{\emph{relacionadas}} à \texttt{altura}?

Que tal o gênero? No painel \textbf{D} da Figura 5.3, plotamos os
\textbf{\emph{dados com linhas ajustadas separadamente para homens e
mulheres}}.

A partir do gráfico, parece haver \textbf{\emph{uma diferença entre
meninos e meninas}}, mesmo que \emph{relativamente pequena} e
\textbf{\emph{só presente após a puberdade}}.

Na Figura 5.4, plotamos os \textbf{\emph{valores}} da
\ul{\textbf{\emph{raiz do erro quadrático médio}}}
{[}\ul{\textbf{RMSE}}{]} para os \textbf{diferentes modelos},
\emph{incluindo um com um parâmetro adicional que modela o efeito de
gênero}.

A partir disso, observamos que o \textbf{\emph{modelo melhorou um
pouco}} ao \textbf{\emph{passar}} da \ul{\textbf{moda}}
\textbf{\emph{para}} \ul{\textbf{média}}; \textbf{\emph{muito}} ao
\textbf{\emph{somar}} a \texttt{idade} \textbf{\emph{à}}
\ul{\textbf{média}} e \textbf{\emph{ligeiramente}} ao
\textbf{\emph{incluir}} o \texttt{gênero}.

\begin{Shaded}
\begin{Highlighting}[numbers=left,,]
\InformationTok{\textasciigrave{}\textasciigrave{}\textasciigrave{}\{r\}}
\CommentTok{\# find the best fitting model to predict height given age}
\NormalTok{model\_age }\OtherTok{\textless{}{-}} \FunctionTok{lm}\NormalTok{(Height }\SpecialCharTok{\textasciitilde{}}\NormalTok{ Age, }\AttributeTok{data =}\NormalTok{ NHANES\_child)}

\CommentTok{\# the add\_predictions() function uses the fitted model to add the predicted values for each person to our dataset}
\NormalTok{NHANES\_child }\OtherTok{\textless{}{-}}
\NormalTok{  NHANES\_child }\SpecialCharTok{\%\textgreater{}\%}
  \FunctionTok{add\_predictions}\NormalTok{(model\_age, }\AttributeTok{var =} \StringTok{"predicted\_age"}\NormalTok{) }\SpecialCharTok{\%\textgreater{}\%}
  \FunctionTok{mutate}\NormalTok{(}
    \AttributeTok{error\_age =}\NormalTok{ Height }\SpecialCharTok{{-}}\NormalTok{ predicted\_age }\CommentTok{\#calculate each individual\textquotesingle{}s difference from the predicted value}
\NormalTok{  )}

\NormalTok{rmse\_age }\OtherTok{\textless{}{-}}
\NormalTok{  NHANES\_child }\SpecialCharTok{\%\textgreater{}\%}
  \FunctionTok{summarise}\NormalTok{(}
    \FunctionTok{sqrt}\NormalTok{(}\FunctionTok{mean}\NormalTok{((error\_age)}\SpecialCharTok{**}\DecValTok{2}\NormalTok{)) }\CommentTok{\#calculate the root mean squared error}
\NormalTok{  ) }\SpecialCharTok{\%\textgreater{}\%}
  \FunctionTok{pull}\NormalTok{()}

\CommentTok{\# compute model fit for modeling with age and gender}

\NormalTok{model\_age\_gender }\OtherTok{\textless{}{-}} \FunctionTok{lm}\NormalTok{(Height }\SpecialCharTok{\textasciitilde{}}\NormalTok{ Age }\SpecialCharTok{+}\NormalTok{ Gender, }\AttributeTok{data =}\NormalTok{ NHANES\_child)}

\NormalTok{rmse\_age\_gender }\OtherTok{\textless{}{-}}
\NormalTok{  NHANES\_child }\SpecialCharTok{\%\textgreater{}\%}
  \FunctionTok{add\_predictions}\NormalTok{(model\_age\_gender, }\AttributeTok{var =} \StringTok{"predicted\_age\_gender"}\NormalTok{) }\SpecialCharTok{\%\textgreater{}\%}
  \FunctionTok{summarise}\NormalTok{(}
    \FunctionTok{sqrt}\NormalTok{(}\FunctionTok{mean}\NormalTok{((Height }\SpecialCharTok{{-}}\NormalTok{ predicted\_age\_gender)}\SpecialCharTok{**}\DecValTok{2}\NormalTok{))}
\NormalTok{  ) }\SpecialCharTok{\%\textgreater{}\%}
  \FunctionTok{pull}\NormalTok{()}

\NormalTok{error\_df }\OtherTok{\textless{}{-}} \CommentTok{\#build a dataframe using the function tribble()}
  \FunctionTok{tribble}\NormalTok{(}
    \SpecialCharTok{\textasciitilde{}}\NormalTok{model, }\SpecialCharTok{\textasciitilde{}}\NormalTok{error,}
    \StringTok{"mode"}\NormalTok{, rmse\_mode,}
    \StringTok{"mean"}\NormalTok{, rmse\_mean,}
    \StringTok{"constant + age"}\NormalTok{, rmse\_age,}
    \StringTok{"constant + age + gender"}\NormalTok{, rmse\_age\_gender}
\NormalTok{  ) }\SpecialCharTok{\%\textgreater{}\%}
  \FunctionTok{mutate}\NormalTok{(}
    \AttributeTok{RMSE =}\NormalTok{ error}
\NormalTok{  )}

\NormalTok{error\_df }\SpecialCharTok{\%\textgreater{}\%}
  \FunctionTok{ggplot}\NormalTok{(}\FunctionTok{aes}\NormalTok{(}\AttributeTok{x =}\NormalTok{ model, }\AttributeTok{y =}\NormalTok{ RMSE)) }\SpecialCharTok{+}
  \FunctionTok{geom\_col}\NormalTok{() }\SpecialCharTok{+}
  \FunctionTok{scale\_x\_discrete}\NormalTok{(}\AttributeTok{limits =} \FunctionTok{c}\NormalTok{(}\StringTok{"mode"}\NormalTok{, }\StringTok{"mean"}\NormalTok{, }\StringTok{"constant + age"}\NormalTok{, }\StringTok{"constant + age + gender"}\NormalTok{)) }\SpecialCharTok{+}
  \FunctionTok{labs}\NormalTok{(}
    \AttributeTok{y =} \StringTok{"root mean squared error [RMSE]"}
\NormalTok{  ) }\SpecialCharTok{+}
  \FunctionTok{labs}\NormalTok{(}
  \AttributeTok{title    =} \StringTok{"Gráfico de Barras: altura [Height] 1691 crianças no NHANES"}\NormalTok{,}
  \AttributeTok{subtitle =} \StringTok{"filtro: idade (age) \textless{} 18 anos"}\NormalTok{,}
  \AttributeTok{caption  =} \StringTok{"Raiz do Erro Quadrático Médio [Root Mean Squared Error {-} RMSE] plotado para cada um dos}\SpecialCharTok{\textbackslash{}n}\StringTok{modelos testados anteriormente. Aqui, temos: Constante + idade + gênero [Constant + age + gender],}\SpecialCharTok{\textbackslash{}n}\StringTok{Constante + idade [Constant + age], Média [Mean] e Moda [Mode]. Fonte: Poldrack, 2025, p. 47, fig. 5.4"}\NormalTok{,}
  \AttributeTok{x =} \StringTok{"Modelos p/predizer Altura (Height)"}\NormalTok{,}
  \AttributeTok{y =} \StringTok{"Root Mean Squared Error [RMSE, cm]"}
\NormalTok{  ) }\SpecialCharTok{+}
  \FunctionTok{coord\_flip}\NormalTok{()}
\InformationTok{\textasciigrave{}\textasciigrave{}\textasciigrave{}}
\end{Highlighting}
\end{Shaded}

\pandocbounded{\includegraphics[keepaspectratio]{cap5-pldr-modelos-dados_files/figure-pdf/unnamed-chunk-6-1.pdf}}

\section{Quais São os Critérios para um ``Bom''
Modelo?}\label{quais-suxe3o-os-crituxe9rios-para-um-bom-modelo}

Via de regra, \textbf{\emph{queremos duas coisas}} do nosso
\ul{\textbf{modelo estatístico}}.

\ul{\textbf{Primeiro}}, queremos que ele \textbf{\emph{descreva bem os
nossos dados}}; ou seja, que tenha o \textbf{\emph{menor erro possível}}
ao modelar nossos dados.

\ul{\textbf{Segundo}}, queremos que generalize com eficácia em conjuntos
novos de dados; ou seja, \textbf{\emph{esperamos}} o \textbf{\emph{menor
número possível de erros}} quando formos \textbf{\emph{aplicá-lo}} a
\ul{\textbf{\emph{um conjunto novo de dados}}} para
\textbf{\emph{realizar}} uma \ul{\textbf{predição}}.

Acontece que esses dois aspectos podem entrar em conflito.

Para entender isso, antes de mais nada, é necessário analisarmos qual é
a \textbf{\emph{origem}} do \textbf{\emph{erro}}
{[}\textbf{\emph{resíduo}}{]}.

\ul{\textbf{\emph{Primeiro}}}, ele pode ocorrer \textbf{\emph{se nosso
modelo estiver errado}}; por exemplo, \emph{se dissermos,
inadequadamente}, que a \texttt{altura} \emph{diminui com a idade em vez
de aumentar}, nosso \emph{erro} será \emph{maior} do que seria para o
\emph{modelo correto}. {[}Erro percepsional{]}

Do mesmo modo, \emph{se estiver faltando um fator importante em nosso
modelo}, isso também \emph{aumentará nosso erro} (como aconteceu quando
\emph{desconsideramos} a \texttt{idade} do \emph{modelo} de
\texttt{altura}).

No entanto, o \textbf{\emph{erro}} também \textbf{\emph{pode ocorrer
mesmo quando o modelo está correto}}, devido à \textbf{\emph{variação
aleatória nos dados}}, o qual geralmente chamamos de \ul{\textbf{erro de
medição}} ou de \ul{\textbf{ruído}}. {[}Erro Amostral{]}

Às vezes, isso se deve a um \emph{erro em nossas medições} --- por
exemplo, quando elas \emph{dependem de um humano} {[}Erro do
\emph{instrumentador}{]}, como usar um cronômetro {[}Erro ou limite de
\emph{precisão} intrínseco do \emph{instrumento}{]} para medir o tempo
decorrido em uma corrida de rua.

Em outros casos, nosso \textbf{\emph{dispositivo de medição}} tem
\textbf{\emph{alta acurácia}} (como \emph{uma balança digital} para
medir o \emph{peso corporal}), \textbf{\emph{mas o que está sendo medido
é afetado por muitos fatores diferentes}} que \textbf{\emph{o tornam
variável}}.

\emph{Se conhecêssemos todos esses fatores}, poderíamos \emph{criar um
modelo} com \emph{maior acurácia} {[}\emph{menor RMSE}{]}, mas, na
\emph{prática}, isso \emph{raramente é possível}.

Usaremos \emph{um exemplo para ilustrar} o caso.

Em vez de usar dados reais, \emph{geraremos alguns dados com uma
\textbf{simulação}} de computador (no Capítulo 8, falaremos mais a
respeito desse assunto).

Suponha que queremos \emph{entender} a \emph{relação} entre o
\textbf{\emph{teor de álcool no sangue}} (\texttt{TAS}) de uma pessoa e
seu \textbf{\emph{tempo de reação}} em \emph{um teste computadorizado de
\textbf{condução}}.

Podemos \textbf{\emph{gerar alguns dados simulados}} e
\textbf{\emph{plotar}} a \textbf{\emph{relação}} (painel \textbf{A} da
Figura 5.5).

Nesse exemplo, o \texttt{tempo\ de\ reação} \emph{aumenta de modo
sistemático} com o \texttt{teor\ de\ álcool\ no\ sangue} --- a linha
mostra o \textbf{\emph{modelo com o ajuste mais adequado}}.

Podemos notar que \textbf{\emph{existem poucos erros}}, observação
\textbf{\emph{evidenciada}} pelo fato de \textbf{\emph{todos os pontos
estarem muito próximos à linha}}.

Além disso, \emph{poderíamos imaginar dados que mostram a mesma relação
linear}, porém \emph{com muito mais erros}, como no painel \textbf{B} da
Figura 5.5.

Aqui, verificamos que ainda \emph{existe um aumento sistemático} do
\texttt{tempo\ de\ reação} com o \texttt{TAS}, {[}\texttt{BAC} -
\emph{Blood Alcoholic Concentration}{]}, porém ele \emph{varia muito
entre os indivíduos}.

\begin{Shaded}
\begin{Highlighting}[numbers=left,,]
\InformationTok{\textasciigrave{}\textasciigrave{}\textasciigrave{}\{r\}}
\NormalTok{dataDf }\OtherTok{\textless{}{-}}
  \FunctionTok{tibble}\NormalTok{(}
    \AttributeTok{BAC =} \FunctionTok{runif}\NormalTok{(}\DecValTok{100}\NormalTok{) }\SpecialCharTok{*} \FloatTok{0.3}\NormalTok{,}
    \AttributeTok{ReactionTime =}\NormalTok{ BAC }\SpecialCharTok{*} \DecValTok{1} \SpecialCharTok{+} \DecValTok{1} \SpecialCharTok{+} \FunctionTok{rnorm}\NormalTok{(}\DecValTok{100}\NormalTok{) }\SpecialCharTok{*} \FloatTok{0.01}
\NormalTok{  )}

\NormalTok{p1 }\OtherTok{\textless{}{-}}\NormalTok{ dataDf }\SpecialCharTok{\%\textgreater{}\%}
  \FunctionTok{ggplot}\NormalTok{(}\FunctionTok{aes}\NormalTok{(}\AttributeTok{x =}\NormalTok{ BAC, }\AttributeTok{y =}\NormalTok{ ReactionTime)) }\SpecialCharTok{+}
  \FunctionTok{geom\_point}\NormalTok{() }\SpecialCharTok{+}
  \FunctionTok{geom\_smooth}\NormalTok{(}\AttributeTok{method =} \StringTok{"lm"}\NormalTok{, }\AttributeTok{se =} \ConstantTok{FALSE}\NormalTok{) }\SpecialCharTok{+}
  \FunctionTok{ggtitle}\NormalTok{(}\StringTok{\textquotesingle{}A: linear, low noise\textquotesingle{}}\NormalTok{)}

\CommentTok{\# noisy version}
\NormalTok{dataDf }\OtherTok{\textless{}{-}}
  \FunctionTok{tibble}\NormalTok{(}
    \AttributeTok{BAC =} \FunctionTok{runif}\NormalTok{(}\DecValTok{100}\NormalTok{) }\SpecialCharTok{*} \FloatTok{0.3}\NormalTok{,}
    \AttributeTok{ReactionTime =}\NormalTok{ BAC }\SpecialCharTok{*} \DecValTok{2} \SpecialCharTok{+} \DecValTok{1} \SpecialCharTok{+} \FunctionTok{rnorm}\NormalTok{(}\DecValTok{100}\NormalTok{) }\SpecialCharTok{*} \FloatTok{0.2}
\NormalTok{  )}

\NormalTok{p2 }\OtherTok{\textless{}{-}}\NormalTok{ dataDf }\SpecialCharTok{\%\textgreater{}\%}
  \FunctionTok{ggplot}\NormalTok{(}\FunctionTok{aes}\NormalTok{(}\AttributeTok{x =}\NormalTok{ BAC, }\AttributeTok{y =}\NormalTok{ ReactionTime)) }\SpecialCharTok{+}
  \FunctionTok{geom\_point}\NormalTok{() }\SpecialCharTok{+}
  \FunctionTok{geom\_smooth}\NormalTok{(}\AttributeTok{method =} \StringTok{"lm"}\NormalTok{, }\AttributeTok{se =} \ConstantTok{FALSE}\NormalTok{) }\SpecialCharTok{+}
  \FunctionTok{ggtitle}\NormalTok{(}\StringTok{\textquotesingle{}B: linear, high noise\textquotesingle{}}\NormalTok{)}

\CommentTok{\# nonlinear (inverted{-}U) function}

\NormalTok{dataDf }\OtherTok{\textless{}{-}}
\NormalTok{  dataDf }\SpecialCharTok{\%\textgreater{}\%}
  \FunctionTok{mutate}\NormalTok{(}
    \AttributeTok{caffeineLevel =} \FunctionTok{runif}\NormalTok{(}\DecValTok{100}\NormalTok{) }\SpecialCharTok{*} \DecValTok{10}\NormalTok{,}
    \AttributeTok{caffeineLevelInvertedU =}\NormalTok{ (caffeineLevel }\SpecialCharTok{{-}} \FunctionTok{mean}\NormalTok{(caffeineLevel))}\SpecialCharTok{**}\DecValTok{2}\NormalTok{,}
    \AttributeTok{testPerformance =} \SpecialCharTok{{-}}\DecValTok{1} \SpecialCharTok{*}\NormalTok{ caffeineLevelInvertedU }\SpecialCharTok{+} \FunctionTok{rnorm}\NormalTok{(}\DecValTok{100}\NormalTok{) }\SpecialCharTok{*} \FloatTok{0.5}
\NormalTok{  )}

\NormalTok{p3 }\OtherTok{\textless{}{-}}\NormalTok{ dataDf }\SpecialCharTok{\%\textgreater{}\%}
  \FunctionTok{ggplot}\NormalTok{(}\FunctionTok{aes}\NormalTok{(}\AttributeTok{x =}\NormalTok{ caffeineLevel, }\AttributeTok{y =}\NormalTok{ testPerformance)) }\SpecialCharTok{+}
  \FunctionTok{geom\_point}\NormalTok{() }\SpecialCharTok{+}
  \FunctionTok{geom\_smooth}\NormalTok{(}\AttributeTok{method =} \StringTok{"lm"}\NormalTok{, }\AttributeTok{se =} \ConstantTok{FALSE}\NormalTok{) }\SpecialCharTok{+}
  \FunctionTok{ggtitle}\NormalTok{(}\StringTok{\textquotesingle{}C: nonlinear\textquotesingle{}}\NormalTok{)}

\CommentTok{\# plot\_grid(p1, p2, p3)}

\CommentTok{\# Criando a nota de rodapé como grob com tamanho de fonte personalizado}
\NormalTok{nota\_rodape }\OtherTok{\textless{}{-}} \FunctionTok{textGrob}\NormalTok{(}
  \StringTok{"Relação simulada entre o teor de álcool no sangue [BAC] e o tempo de reação [Reaction time] em um teste de condução [Test performance],}\SpecialCharTok{\textbackslash{}n}\StringTok{estando o modelo linear mais adequado representado pela linha. (A) Linear, low noise [Linear, ruído baixo]: relação linear com baixo erro de}\SpecialCharTok{\textbackslash{}n}\StringTok{medição. (B), Linear, high noise [Linear, ruído alto]: relação linear com maior erro de medição. (C), Nonlinear [Não linear]: relação não linear}\SpecialCharTok{\textbackslash{}n}\StringTok{com baixo erro de medição e modelo linear (incorreto). Fonte: Poldrack, 2025, p. 48, fig. 5.5"}\NormalTok{,}
  \AttributeTok{gp =} \FunctionTok{gpar}\NormalTok{(}\AttributeTok{fontsize =} \DecValTok{8}\NormalTok{,}
            \AttributeTok{fontface =} \StringTok{"italic"}\NormalTok{),}
  \AttributeTok{x =} \FloatTok{0.5}\NormalTok{,}
  \AttributeTok{hjust =} \FloatTok{0.5}
\NormalTok{)}

\CommentTok{\# Usando grid.arrange com a nota de rodapé customizada}
\CommentTok{\# usando grid.arrange() (do pacote gridExtra), pode usar o argumento bottom:}
\CommentTok{\# para gerar um nota de rodapé ´no gráfico final}
\FunctionTok{grid.arrange}\NormalTok{(}
\NormalTok{  p1, p2, p3,}
  \AttributeTok{ncol =} \DecValTok{2}\NormalTok{,}
  \AttributeTok{bottom =}\NormalTok{ nota\_rodape}
\NormalTok{)}
\InformationTok{\textasciigrave{}\textasciigrave{}\textasciigrave{}}
\end{Highlighting}
\end{Shaded}

\pandocbounded{\includegraphics[keepaspectratio]{cap5-pldr-modelos-dados_files/figure-pdf/unnamed-chunk-7-1.pdf}}

Ambos {[}\textbf{A e B}{]} são \textbf{\emph{exemplos}} em que a
\textbf{\emph{relação entre as duas variáveis}} representadas parece ser
\ul{\textbf{\emph{linear}}}, e o \ul{\textbf{\emph{erro}}} retrata
\ul{\textbf{\emph{ruído}}} em \ul{\textbf{\emph{nossa medição}}}.

Em contrapartida, existem situações {[}\textbf{C}{]} em que a
\textbf{\emph{relação entre as variáveis \ul{não é linear}}}, e o
\textbf{\emph{erro aumenta}}, já que o \ul{\textbf{\emph{modelo não está
devidamente especificado}}}.

Suponha que estamos interessados na \textbf{\emph{relação}} entre a
\texttt{ingestão\ de\ cafeína} e o \texttt{desempenho\ em\ um\ teste}.

A \textbf{\emph{relação}} entre \emph{estimulantes} como a
\texttt{cafeína} e o \texttt{desempenho\ na\ prova} \emph{geralmente} é
\ul{\textbf{\emph{não linear}}} --- ou seja, \emph{não segue uma linha
reta}.

Isso ocorre porque \textbf{\emph{ele aumenta com pequenas quantidades}}
de \texttt{cafeína} (à medida que a \emph{pessoa fica mais alerta}),
mas, \textbf{\emph{depois, começa a diminuir com quantidades maiores}}
(à medida que a \emph{pessoa fica nervosa} e \emph{agitada}).

É possível \ul{\textbf{simular}} \emph{dados dessa forma} e, em seguida,
\textbf{\emph{ajustar}} um \textbf{\emph{modelo linear}} \emph{aos
dados} (painel \textbf{C} da Figura 5.5).

A \ul{\textbf{\emph{linha reta}}} que \ul{\textbf{\emph{melhor se
ajusta}}} a \ul{\textbf{\emph{esses dados}}} {[}Método dos Minímos
Quadrados - MMQ ; OMS - \emph{Ordinary Minimum Square}{]} claramente
\textbf{\emph{retrata}} um \textbf{\emph{alto grau}} de
\ul{\textbf{\emph{erro}}}.

Ainda que exista uma relação muito consistente entre o
\texttt{desempenho\ no\ teste} {[}Test performance{]} e a
\texttt{ingestão\ de\ cafeína} {[}Caffeine level{]}, ela
\ul{\textbf{\emph{segue uma curva}}} e \ul{\textbf{\emph{não uma linha
reta}}}.

O \ul{\textbf{modelo}}, que \ul{\textbf{assume}} uma \ul{\textbf{relação
linear}}, apresenta \ul{\textbf{\emph{erro elevado}}} porque é o
\ul{\textbf{\emph{modelo inadequado}}} para esses dados. {[}Erro
percepcional: está no Modelo, mas não está no Mundo real{]}

\section{Um Modelo Pode Ser Demasiadamente
Bom?}\label{um-modelo-pode-ser-demasiadamente-bom}

A impressão que temos é do \ul{\textbf{erro}} \ul{\textbf{\emph{como
algo ruim}}}, e, normalmente, \ul{\textbf{\emph{preferimos um modelo com
menor a outro com maior erro}}}.

No entanto, mencionamos anteriormente o \ul{\textbf{\emph{conflito}}}
entre a \ul{\textbf{\emph{capacidade de um modelo se ajustar com
acurácia ao conjunto atual de dados}}} e sua
\ul{\textbf{\emph{capacidade de generalizar em conjuntos novos de
dados}}}.

Em geral, acontece que \ul{\textbf{\emph{aquele com o menor erro é bem
pior em generalizar}}} em \emph{conjuntos novos de dados}!

Para visualizar isso, \textbf{\emph{geraremos}} mais uma vez alguns
\textbf{\emph{dados}} para que \emph{possamos saber a verdadeira relação
entre as variáveis}.

Criaremos \emph{dois conjuntos} de \textbf{\emph{dados simulados}},
gerados exatamente da mesma forma: ou seja, a equação para ambos é

\[
y = \beta \times X + \epsilon
\]

a \textbf{\emph{única diferença}} é que \textbf{\emph{um}}
\ul{\textbf{\emph{ruído aleatório}}} \textbf{\emph{diferente}} foi
\textbf{\emph{usado para}} \(\epsilon\){[}letra grega \emph{epsilon} que
simboliza o \emph{erro} ou \emph{resíduo} ou \emph{ruído}{]}
\textbf{\emph{em cada caso}}.

Em seguida, \textbf{\emph{ajustamos dois modelos aos dados}}:
\ul{\textbf{\emph{um simples}}} (com somente dois parâmetros, inclinação
e intercepto) e um mais complexo que contém um total de oito parâmetros
(inclinação e intercepto com parâmetros de tamanho que retratam
polinômios de grau crescente, como X\textsuperscript{2},
X\textsuperscript{3}, e assim por diante).

\begin{Shaded}
\begin{Highlighting}[numbers=left,,]
\InformationTok{\textasciigrave{}\textasciigrave{}\textasciigrave{}\{r\}}
\CommentTok{\#parameters for simulation}
\FunctionTok{set.seed}\NormalTok{(}\DecValTok{1122}\NormalTok{)}
\NormalTok{sampleSize }\OtherTok{\textless{}{-}} \DecValTok{16}


\CommentTok{\#build a dataframe of simulated data}
\NormalTok{simData }\OtherTok{\textless{}{-}}
  \FunctionTok{tibble}\NormalTok{(}
    \AttributeTok{X =} \FunctionTok{rnorm}\NormalTok{(sampleSize),}
    \AttributeTok{Y =}\NormalTok{ X }\SpecialCharTok{+} \FunctionTok{rnorm}\NormalTok{(sampleSize, }\AttributeTok{sd =} \DecValTok{1}\NormalTok{),}
    \AttributeTok{Ynew =}\NormalTok{ X }\SpecialCharTok{+} \FunctionTok{rnorm}\NormalTok{(sampleSize, }\AttributeTok{sd =} \DecValTok{1}\NormalTok{)}
\NormalTok{  )}

\CommentTok{\#fit models to these data}
\NormalTok{simpleModel }\OtherTok{\textless{}{-}} \FunctionTok{lm}\NormalTok{(Y }\SpecialCharTok{\textasciitilde{}}\NormalTok{ X, }\AttributeTok{data =}\NormalTok{ simData)}
\NormalTok{complexModel }\OtherTok{\textless{}{-}} \FunctionTok{lm}\NormalTok{(Y }\SpecialCharTok{\textasciitilde{}} \FunctionTok{poly}\NormalTok{(X, }\DecValTok{8}\NormalTok{), }\AttributeTok{data =}\NormalTok{ simData)}

\CommentTok{\#calculate root mean squared error for "current" dataset}
\NormalTok{rmse\_simple }\OtherTok{\textless{}{-}} \FunctionTok{sqrt}\NormalTok{(}\FunctionTok{mean}\NormalTok{(simpleModel}\SpecialCharTok{$}\NormalTok{residuals}\SpecialCharTok{**}\DecValTok{2}\NormalTok{))}
\NormalTok{rmse\_complex }\OtherTok{\textless{}{-}} \FunctionTok{sqrt}\NormalTok{(}\FunctionTok{mean}\NormalTok{(complexModel}\SpecialCharTok{$}\NormalTok{residuals}\SpecialCharTok{**}\DecValTok{2}\NormalTok{))}

\CommentTok{\#calculate root mean squared error for "new" dataset}
\NormalTok{rmse\_prediction\_simple }\OtherTok{\textless{}{-}} \FunctionTok{sqrt}\NormalTok{(}\FunctionTok{mean}\NormalTok{((simpleModel}\SpecialCharTok{$}\NormalTok{fitted.values }\SpecialCharTok{{-}}\NormalTok{ simData}\SpecialCharTok{$}\NormalTok{Ynew)}\SpecialCharTok{**}\DecValTok{2}\NormalTok{))}
\NormalTok{rmse\_prediction\_complex }\OtherTok{\textless{}{-}} \FunctionTok{sqrt}\NormalTok{(}\FunctionTok{mean}\NormalTok{((complexModel}\SpecialCharTok{$}\NormalTok{fitted.values }\SpecialCharTok{{-}}\NormalTok{ simData}\SpecialCharTok{$}\NormalTok{Ynew)}\SpecialCharTok{**}\DecValTok{2}\NormalTok{))}

\CommentTok{\#visualize}
\NormalTok{plot\_original\_data }\OtherTok{\textless{}{-}}
\NormalTok{  simData }\SpecialCharTok{\%\textgreater{}\%}
  \FunctionTok{ggplot}\NormalTok{(}\FunctionTok{aes}\NormalTok{(X, Y)) }\SpecialCharTok{+}
  \FunctionTok{geom\_point}\NormalTok{() }\SpecialCharTok{+}
  \FunctionTok{geom\_smooth}\NormalTok{(}
    \AttributeTok{method =} \StringTok{"lm"}\NormalTok{,}
    \AttributeTok{formula =}\NormalTok{ y }\SpecialCharTok{\textasciitilde{}} \FunctionTok{poly}\NormalTok{(x, }\DecValTok{8}\NormalTok{),}
    \AttributeTok{color =} \StringTok{"red"}\NormalTok{,}
    \AttributeTok{se =} \ConstantTok{FALSE}
\NormalTok{  ) }\SpecialCharTok{+}
  \FunctionTok{geom\_smooth}\NormalTok{(}
    \AttributeTok{method =} \StringTok{"lm"}\NormalTok{,}
    \AttributeTok{color =} \StringTok{"blue"}\NormalTok{,}
    \AttributeTok{se =} \ConstantTok{FALSE}
\NormalTok{  ) }\SpecialCharTok{+}
  \FunctionTok{ylim}\NormalTok{(}\SpecialCharTok{{-}}\DecValTok{3}\NormalTok{, }\DecValTok{3}\NormalTok{) }\SpecialCharTok{+}
  \FunctionTok{annotate}\NormalTok{(}
    \StringTok{"text"}\NormalTok{,}
    \AttributeTok{x =} \SpecialCharTok{{-}}\FloatTok{1.25}\NormalTok{,}
    \AttributeTok{y =} \FloatTok{2.5}\NormalTok{,}
    \AttributeTok{label =} \FunctionTok{sprintf}\NormalTok{(}\StringTok{"RMSE=\%0.1f"}\NormalTok{, rmse\_simple),}
    \AttributeTok{color =} \StringTok{"blue"}\NormalTok{,}
    \AttributeTok{hjust =} \DecValTok{0}\NormalTok{,}
    \AttributeTok{cex =} \DecValTok{4}
\NormalTok{  ) }\SpecialCharTok{+}
  \FunctionTok{annotate}\NormalTok{(}
    \StringTok{"text"}\NormalTok{,}
    \AttributeTok{x =} \SpecialCharTok{{-}}\FloatTok{1.25}\NormalTok{,}
    \AttributeTok{y =} \DecValTok{2}\NormalTok{,}
    \AttributeTok{label =} \FunctionTok{sprintf}\NormalTok{(}\StringTok{"RMSE=\%0.1f"}\NormalTok{, rmse\_complex),}
    \AttributeTok{color =} \StringTok{"red"}\NormalTok{,}
    \AttributeTok{hjust =} \DecValTok{0}\NormalTok{,}
    \AttributeTok{cex =} \DecValTok{4}
\NormalTok{  ) }\SpecialCharTok{+}
  \FunctionTok{ggtitle}\NormalTok{(}\StringTok{"original data"}\NormalTok{)}

\NormalTok{plot\_new\_data  }\OtherTok{\textless{}{-}}
\NormalTok{  simData }\SpecialCharTok{\%\textgreater{}\%}
  \FunctionTok{ggplot}\NormalTok{(}\FunctionTok{aes}\NormalTok{(X, Ynew)) }\SpecialCharTok{+}
  \FunctionTok{geom\_point}\NormalTok{() }\SpecialCharTok{+}
  \FunctionTok{geom\_smooth}\NormalTok{(}
    \FunctionTok{aes}\NormalTok{(X, Y),}
    \AttributeTok{method =} \StringTok{"lm"}\NormalTok{,}
    \AttributeTok{formula =}\NormalTok{ y }\SpecialCharTok{\textasciitilde{}} \FunctionTok{poly}\NormalTok{(x, }\DecValTok{8}\NormalTok{),}
    \AttributeTok{color =} \StringTok{"red"}\NormalTok{,}
    \AttributeTok{se =} \ConstantTok{FALSE}
\NormalTok{  ) }\SpecialCharTok{+}
  \FunctionTok{geom\_smooth}\NormalTok{(}
    \FunctionTok{aes}\NormalTok{(X, Y),}
    \AttributeTok{method =} \StringTok{"lm"}\NormalTok{,}
    \AttributeTok{color =} \StringTok{"blue"}\NormalTok{,}
    \AttributeTok{se =} \ConstantTok{FALSE}
\NormalTok{  ) }\SpecialCharTok{+}
  \FunctionTok{ylim}\NormalTok{(}\SpecialCharTok{{-}}\DecValTok{3}\NormalTok{, }\DecValTok{3}\NormalTok{) }\SpecialCharTok{+}
  \FunctionTok{annotate}\NormalTok{(}
    \StringTok{"text"}\NormalTok{,}
    \AttributeTok{x =} \SpecialCharTok{{-}}\FloatTok{1.25}\NormalTok{,}
    \AttributeTok{y =} \FloatTok{2.5}\NormalTok{,}
    \AttributeTok{label =} \FunctionTok{sprintf}\NormalTok{(}\StringTok{"RMSE=\%0.1f"}\NormalTok{, rmse\_prediction\_simple),}
    \AttributeTok{color =} \StringTok{"blue"}\NormalTok{,}
    \AttributeTok{hjust =} \DecValTok{0}\NormalTok{,}
    \AttributeTok{cex =} \DecValTok{4}
\NormalTok{  ) }\SpecialCharTok{+}
  \FunctionTok{annotate}\NormalTok{(}
    \StringTok{"text"}\NormalTok{,}
    \AttributeTok{x =} \SpecialCharTok{{-}}\FloatTok{1.25}\NormalTok{,}
    \AttributeTok{y =} \DecValTok{2}\NormalTok{,}
    \AttributeTok{label =} \FunctionTok{sprintf}\NormalTok{(}\StringTok{"RMSE=\%0.1f"}\NormalTok{, rmse\_prediction\_complex),}
    \AttributeTok{color =} \StringTok{"red"}\NormalTok{,}
    \AttributeTok{hjust =} \DecValTok{0}\NormalTok{,}
    \AttributeTok{cex =} \DecValTok{4}
\NormalTok{  ) }\SpecialCharTok{+}
  \FunctionTok{ggtitle}\NormalTok{(}\StringTok{"new data"}\NormalTok{)}

\CommentTok{\# plot\_grid(plot\_original\_data, plot\_new\_data)}

\CommentTok{\# Criando a nota de rodapé como grob com tamanho de fonte personalizado}
\NormalTok{nota\_rodape }\OtherTok{\textless{}{-}} \FunctionTok{textGrob}\NormalTok{(}
  \StringTok{"Um exemplo de sobreajuste. Ambos os conjuntos de dados foram gerados com o mesmo modelo, com diferentes ruídos aleatórios}\SpecialCharTok{\textbackslash{}n}\StringTok{adicionados para gerar cada conjunto. O painel A (Original data [Dados originais]) mostra os dados usados para ajustar o modelo, com um}\SpecialCharTok{\textbackslash{}n}\StringTok{ajuste linear simples (linha reta) e um ajuste polinomial complexo de oitavo grau (linha curva). Os valores da raiz do erro quadrático médio}\SpecialCharTok{\textbackslash{}n}\StringTok{(RMSE) para cada modelo são mostrados na figura; nesse caso, o complexo tem uma RMSE menor do que o simples. O painel B (New}\SpecialCharTok{\textbackslash{}n}\StringTok{data [Dados novos]) mostra o segundo conjunto de dados, considerando o mesmo modelo sobreposto e os valores RMSE calculados com}\SpecialCharTok{\textbackslash{}n}\StringTok{o modelo obtido a partir do primeiro conjunto de dados. Aqui, vemos que o mais simples se ajusta melhor ao conjunto novo de dados}\SpecialCharTok{\textbackslash{}n}\StringTok{do que o mais complexo, que foi sobreajustado ao primeiro conjunto de dados. Fonte: Poldrack, 2025, p. 49, fig. 5.6"}\NormalTok{,}
  \AttributeTok{gp =} \FunctionTok{gpar}\NormalTok{(}\AttributeTok{fontsize =} \DecValTok{8}\NormalTok{,}
            \AttributeTok{fontface =} \StringTok{"italic"}\NormalTok{),}
  \AttributeTok{x =} \FloatTok{0.5}\NormalTok{,}
  \AttributeTok{hjust =} \FloatTok{0.5}
\NormalTok{)}

\CommentTok{\# Usando grid.arrange com a nota de rodapé customizada}
\CommentTok{\# usando grid.arrange() (do pacote gridExtra), pode usar o argumento bottom:}
\CommentTok{\# para gerar um nota de rodapé ´no gráfico final}
\FunctionTok{grid.arrange}\NormalTok{(}
\NormalTok{  plot\_original\_data, plot\_new\_data,}
  \AttributeTok{ncol =} \DecValTok{2}\NormalTok{,}
  \AttributeTok{bottom =}\NormalTok{ nota\_rodape}
\NormalTok{)}
\InformationTok{\textasciigrave{}\textasciigrave{}\textasciigrave{}}
\end{Highlighting}
\end{Shaded}

\pandocbounded{\includegraphics[keepaspectratio]{cap5-pldr-modelos-dados_files/figure-pdf/unnamed-chunk-8-1.pdf}}

Na Figura 5.6, o painel \textbf{A} mostra que o \textbf{\emph{modelo
mais complexo (não linear) se ajusta melhor aos dados do que o mais
simples (linear)}}.

No entanto, vemos o \textbf{\emph{contrário quando o mesmo modelo é
aplicado a um conjunto novo de dados gerados da mesma forma}} (painel
\textbf{B}).

Aqui, o \ul{\textbf{\emph{mais simples se ajusta melhor aos dados novos
do que o mais complexo}}}.

Por \emph{intuição}, podemos \textbf{\emph{observar que o modelo mais
complexo é extremamente influenciado pelos pontos de dados específicos
do primeiro conjunto de dados}}; como a \emph{posição exata desses
pontos} de dados foi \emph{determinada por ruído aleatório}, isso
\emph{leva o modelo mais complexo a se ajustar indevidamente ao conjunto
novo de dados}.

Chamamos esse fenômeno de \ul{\textbf{\emph{sobreajuste}}}
{[}\ul{\textbf{\emph{overfitting}}}{]}.

Por enquanto, é importante não esquecer que o \textbf{\emph{ajuste do
nosso modelo precisa ser bom, não demasiadamente bom}}.

Como disse Albert \textbf{Einstein} (\textbf{\emph{1934}}):
``Dificilmente se pode negar que o \textbf{\emph{objetivo supremo de
toda teoria é tornar os elementos básicos irredutíveis tão simples e tão
poucos quanto possível sem ter que renunciar à representação adequada de
um único dado da experiência}}''.

Em geral, isso é \textbf{\emph{parafraseado}} como \ul{\textbf{``tudo
deve ser o mais simples possível, mas não demasiadamente simples''}}.

\section{Sumarização de Dados Usando a
Média}\label{sumarizauxe7uxe3o-de-dados-usando-a-muxe9dia}

Já falamos sobre a média (ou valor médio) anteriormente, e, de fato, a
maioria das pessoas a conhece, mesmo que nunca tenham feito um curso de
estatística.

Ela é comumente usada para descrever o que chamamos de tendência central
de um conjunto de dados --- ou seja, em torno de qual valor os dados
estão centralizados?

A maioria das pessoas não pensa em calcular a média como um meio para
ajustar um modelo aos dados.

No entanto, é exatamente isso que estamos fazendo quando a calculamos.

Já vimos a fórmula para calcular a média de uma amostra de dados:

\[
\bar{X} = \frac{\sum_{i=1}^{n} x_i}{n}
\]

É importante ressaltar que essa fórmula \ul{\textbf{\emph{é específica
para uma amostra de dados}}}, um conjunto de pontos de dados
selecionados a partir de uma população maior.

Com uma amostra, desejamos caracterizar uma população maior --- o
conjunto completo de indivíduos em que estamos interessados.
{[}\textbf{\emph{Inferência}}{]}

Por exemplo, se fôssemos analistas de pesquisas eleitorais, nossa
população de interesse poderia ser todos os eleitores
registrados\footnote{Esse exemplo hipotético se refere às eleições nos
  Estados Unidos. Lá, cada estado tem autonomia para criar suas regras
  de votação no âmbito permitido pelas leis federais. Ou seja, já que o
  voto não é obrigatório, dependendo da localização, os eleitores podem
  se registrar antes de votar ou podem se registrar no mesmo dia.
  Estadunidenses podem votar pelo correio, ou presencialmente, seja por
  cédulas de papel ou por urnas eletrônicas. {[}N. da T.{]}}, ao passo
que nossa amostra pode incluir apenas algumas milhares de pessoas dessa
população.

No Capítulo 7, falaremos com mais detalhes sobre amostragem, mas, por
ora, vale ressaltar que os estatísticos geralmente gostam de
\textbf{\emph{usar símbolos distintos}} a fim de
\textbf{\emph{diferenciar}} os cálculos estatísticos que descrevem
\textbf{\emph{valores para uma amostra}} a partir de
\textbf{\emph{parâmetros}} que descrevem os {[}verdadeiros e
desconhecidos{]} \textbf{\emph{valores}} lógicos para \textbf{\emph{uma
população}}; nesse caso, a fórmula para a \ul{\textbf{\emph{média da
população}}} (denotada como μ) é: {[}N maiúscula denota o tamanho da
população{]}

\[
\mu = \frac{\sum_{i=1}^{N} x_i}{N}
\]

Em que N é o tamanho de toda a população.

Nesse caso, os cálculos matemáticos são exatamente os mesmos para a
amostra e para a população; apenas os símbolos diferem.

Mais tarde, veremos casos em que os cálculos matemáticos são diferentes,
dependendo se estamos calculando um parâmetro da população ou uma
estatística amostral.

Já vimos que a \textbf{\emph{média é o \ul{estimador}}} que
\textbf{\emph{garante um erro médio de 0}}. {[}\textbf{\emph{MSE}}{]}

No entanto, também aprendemos que o \textbf{\emph{erro médio não é o
melhor critério}}; queremos, na verdade, \ul{\textbf{\emph{um
estimador}}} que nos forneça \ul{\textbf{\emph{a menor soma dos erros
quadráticos}}} (\textbf{\emph{SSE}}), o que a média também faz.

Isso poderia ser provado usando cálculo, mas demonstraremos graficamente
na Figura 5.7.

\begin{Shaded}
\begin{Highlighting}[numbers=left,,]
\InformationTok{\textasciigrave{}\textasciigrave{}\textasciigrave{}\{r\}}
\NormalTok{df\_error }\OtherTok{\textless{}{-}}
  \FunctionTok{tibble}\NormalTok{(}
    \AttributeTok{val =} \FunctionTok{seq}\NormalTok{(}\DecValTok{100}\NormalTok{, }\DecValTok{175}\NormalTok{, }\FloatTok{0.05}\NormalTok{),}
    \AttributeTok{sse =} \ConstantTok{NA}
\NormalTok{  )}

\ControlFlowTok{for}\NormalTok{ (i }\ControlFlowTok{in} \DecValTok{1}\SpecialCharTok{:}\FunctionTok{dim}\NormalTok{(df\_error)[}\DecValTok{1}\NormalTok{]) \{}
\NormalTok{  err }\OtherTok{\textless{}{-}}\NormalTok{ NHANES\_child}\SpecialCharTok{$}\NormalTok{Height }\SpecialCharTok{{-}}\NormalTok{ df\_error}\SpecialCharTok{$}\NormalTok{val[i]}
\NormalTok{  df\_error}\SpecialCharTok{$}\NormalTok{sse[i] }\OtherTok{\textless{}{-}} \FunctionTok{sum}\NormalTok{(err}\SpecialCharTok{**}\DecValTok{2}\NormalTok{)}
\NormalTok{\}}

\NormalTok{df\_error }\SpecialCharTok{\%\textgreater{}\%}
  \FunctionTok{ggplot}\NormalTok{(}\FunctionTok{aes}\NormalTok{(val, sse)) }\SpecialCharTok{+}
  \FunctionTok{geom\_vline}\NormalTok{(}\AttributeTok{xintercept =} \FunctionTok{mean}\NormalTok{(NHANES\_child}\SpecialCharTok{$}\NormalTok{Height), }\AttributeTok{color =} \StringTok{"blue"}\NormalTok{) }\SpecialCharTok{+}
  \FunctionTok{geom\_point}\NormalTok{(}\AttributeTok{size =} \FloatTok{0.1}\NormalTok{) }\SpecialCharTok{+}
  \FunctionTok{annotate}\NormalTok{(}
    \StringTok{"text"}\NormalTok{,}
    \AttributeTok{x =} \FunctionTok{mean}\NormalTok{(NHANES\_child}\SpecialCharTok{$}\NormalTok{Height) }\SpecialCharTok{+} \DecValTok{8}\NormalTok{,}
    \AttributeTok{y =} \FunctionTok{max}\NormalTok{(df\_error}\SpecialCharTok{$}\NormalTok{sse),}
    \AttributeTok{label =} \StringTok{"mean"}\NormalTok{,}
    \AttributeTok{color =} \StringTok{"blue"}
\NormalTok{  ) }\SpecialCharTok{+}
  \FunctionTok{labs}\NormalTok{(}
  \AttributeTok{title    =} \StringTok{"Gráfico com parábola (SSE): média [reta vertical]"}\NormalTok{,}
  \AttributeTok{subtitle =} \StringTok{"SSE: Sum of Squared Error (Soma dos Erros Quadráticos {-} SEQ)"}\NormalTok{,}
  \AttributeTok{caption  =} \StringTok{"Uma demonstração de como a média é a medida estatística que minimiza a soma dos erros quadráticos}\SpecialCharTok{\textbackslash{}n}\StringTok{[Sum of squared errors]. Usando os dados de altura infantil do NHANES, calculamos a média (denotada}\SpecialCharTok{\textbackslash{}n}\StringTok{pela linha vertical, Mean [Média]). Em seguida, testamos um intervalo de possíveis estimativas de}\SpecialCharTok{\textbackslash{}n}\StringTok{parâmetros e, para cada um, calculamos a soma dos erros quadráticos para cada ponto de dados desse}\SpecialCharTok{\textbackslash{}n}\StringTok{valor, que são indicados pela curva. Observamos que a média se situa no ponto mínimo}\SpecialCharTok{\textbackslash{}n}\StringTok{do gráfico de erro quadrático. Fonte: Poldrack, 2025, p. 51, fig. 5.7"}\NormalTok{,}
  \AttributeTok{x =} \StringTok{"Test value"}\NormalTok{,}
  \AttributeTok{y =} \StringTok{"Sum of Squared Error [SSE]"}
\NormalTok{  )}
\InformationTok{\textasciigrave{}\textasciigrave{}\textasciigrave{}}
\end{Highlighting}
\end{Shaded}

\pandocbounded{\includegraphics[keepaspectratio]{cap5-pldr-modelos-dados_files/figure-pdf/unnamed-chunk-9-1.pdf}}

Como a minimização da SSE é uma boa \emph{feature} {[}característica
desejada{]}, a \ul{\textbf{média}} é \textbf{\emph{a medida estatística
mais utilizada para sumarizar os dados}}.

No entanto, ela também \textbf{\emph{tem um lado sombrio}}.

Suponha que cinco pessoas estejam em um bar e que examinaremos a renda
de cada uma delas (Tabela 5.1).

A média (US\$61.600,00) parece ser uma boa sumarização da renda das
cinco.

Agora, analisaremos o que acontece se Beyoncé Knowles entrar no bar
(Tabela 5.2).

A média agora é de quase US\$10 milhões, o que não é representativo de
nenhuma das pessoas do bar --- em particular, \textbf{é extremamente
influenciada pelo \emph{outlier} de Beyoncé}, que \emph{foge, e muito,
da média}.

De modo geral, \textbf{\emph{essa medida estatística é altamente
sensível a valores extremos}}, por isso é sempre
\ul{\textbf{\emph{importante garantir que não haja valores extremos ao
usá-la para sumarizar os dados}}}.

\begin{figure}[H]

{\centering \pandocbounded{\includegraphics[keepaspectratio]{fig/Poldrack-tab-5.2-renda-5-pessoas-bar-com-Beyonce.png}}

}

\caption{Renda de cinco clientes do bar sem e com renda da Beyoncé}

\end{figure}%

\section{Sumarização Robusta de Dados Usando a
Mediana}\label{sumarizauxe7uxe3o-robusta-de-dados-usando-a-mediana}

Se quisermos \textbf{sumarizar} os dados de uma forma \textbf{menos
sensível a \emph{outliers}}, podemos usar outra medida estatística
chamada \ul{\textbf{\emph{mediana}}}.

Se \textbf{\emph{ordenarmos todos os valores por ordem de magnitude}}, a
\ul{\textbf{\emph{mediana é o valor que fica no meio}}}.

Caso exista \textbf{\emph{um número par}} de valores, existirão dois
valores coincidentes para a posição do meio.

Nesse caso, usamos a \textbf{\emph{média (ou seja, o ponto
intermediário) desses dois números}}.

Vejamos um exemplo. Suponha que queremos sumarizar os seguintes valores:

8 6 3 14 12 7 6 4 9

Se os ordenarmos assim:

3 4 6 6 7 8 9 12 14

A \ul{\textbf{\emph{mediana é o valor do meio}}} --- neste caso, o
\emph{quinto dos nove valores}.

Considerando que a \ul{\textbf{\emph{média minimiza a soma dos erros
quadráticos}}}, a \ul{\textbf{\emph{mediana minimiza}}} uma
\emph{quantidade ligeiramente diferente}: a \ul{\textbf{\emph{soma do
valor absoluto dos erros}}}.

Isso explica por que \textbf{é menos sensível a \emph{outliers}} ---
\emph{elevar ao quadrado potencializa o efeito de grandes erros} em
\emph{comparação} ao \emph{uso} do \emph{valor absoluto}.

É possível observar isso no exemplo da renda: a renda mediana
(US\$65.000) é mais representativa do grupo como um todo do que a média
(US\$9.051.333) e menos sensível a um único \emph{outlier} maior.

\textbf{Sendo assim, por que usaríamos a média?}

Conforme analisaremos em um próximo capítulo, \textbf{\emph{ela é o
``melhor'' estimador no sentido de que varia menos de amostra para
amostra}}, em \emph{comparação} a \textbf{\emph{outros estimadores}}.

Cabe a \textbf{nós decidir se a sensibilidade a potenciais
\emph{outliers}} vale a pena --- afinal de contas, a estatística tem
tudo a ver com \emph{trade-offs}. {[}trocas que caracterizam dilemas de
escolha: melhorar um atributo piora outro e vice-vsera{]}

\section{A Moda}\label{a-moda}

Não raro, queremos descrever a tendência central de \textbf{\emph{um
conjunto de dados que não é numérico}}.

Por exemplo, imagine que queremos saber quais modelos de iPhone são mais
usados.

Para testar isso, poderíamos perguntar a um grande grupo de usuários
desse celular qual modelo cada um tem.

Se calculássemos a média desses valores, poderíamos observar que a média
do modelo de iPhone é 9,51, o que \textbf{\emph{claramente não faz
sentido}}, pois esses números não podem ser interpretados como medidas
quantitativas.

Neste caso, \textbf{\emph{uma medida de tendência central mais
adequada}} é a \ul{\textbf{\emph{moda}}}, o \ul{\textbf{\emph{valor mais
comum}}} no \emph{conjunto de dados}, conforme vimos anteriormente.

\section{Variabilidade: Até que Ponto a Média se Ajusta Bem aos
Dados?}\label{variabilidade-atuxe9-que-ponto-a-muxe9dia-se-ajusta-bem-aos-dados}

Após descrevermos a tendência central dos dados, em geral,
\textbf{\emph{também queremos descrever até que ponto eles podem
variar}}.

Às vezes, também chamamos essa descrição de \textbf{\emph{dispersão}}
pelo fato de descrever o quanto os dados estão amplamente dispersos.

Já exploramos a \emph{soma dos erros quadráticos}, a \emph{base para} as
medidas de variabilidade mais utilizadas: a \textbf{\emph{variância}} e
o \textbf{\emph{desvio-padrão}}.

A \ul{\textbf{\emph{variância}}} para \ul{\textbf{\emph{uma população}}}
(referenciada como σ\textsuperscript{2}) é simplesmente a
\textbf{\emph{soma dos erros quadráticos}} \textbf{{[}SSE{]}
\emph{dividida pelo número de observações}} --- ou seja, é exatamente
\ul{\textbf{\emph{o mesmo que o erro quadrático médio}}} que já vimos:

\[
\sigma^2 = \frac{SSE}{N} = \frac{\sum_{i=1}^N (x_i - \mu)^2}{N}
\]

Em que μ é a média da população.

O \ul{\textbf{desvio-padrão}} da população é simplesmente a
\textbf{\emph{raiz quadrada disso}} --- ou seja, a \textbf{\emph{raiz do
erro quadrático}} {[}\textbf{RMSE}{]} que já vimos.

Ele é útil porque \textbf{\emph{apresenta os erros nas mesmas unidades
que os dados originais}} (desfazendo a quadratura que aplicamos aos
erros).

Como normalmente não temos acesso a toda a população, temos que calcular
a \textbf{variância} usando uma amostra, que chamamos de
\(\hat{\sigma}^2\), com o ``\textbf{\emph{chapéu}}'' representando o
\ul{\textbf{\emph{fato de se tratar de uma estimativa baseada em uma
amostra}}}.

A equação para \(\hat{\sigma}^2\) (às vezes, também chamada de
s\textsuperscript{2}) é semelhante à de σ\textsuperscript{2}:
{[}\textbf{variância amostral}{]}

\[
s^2 = \hat{\sigma}^2 = \frac{\sum_{i=1}^n (x_i - \bar{X})^2}{n-1}
\]

A única diferença entre as duas equações é que dividimos por n − 1 a
mesma variância em vez de dividir por N a variância da população.

Isso se relaciona com \textbf{\emph{um conceito fundamental de
estatística}}: \ul{\textbf{graus de liberdade}}.

Lembre que, para calcular a \ul{\textbf{variância amostral}},
\emph{primeiro precisamos estimar} a \ul{\textbf{\emph{média amostral}}}
\(\bar{X}\).

Ao fazer isso, \emph{um valor nos dados não pode mais variar}.

Por exemplo, digamos que temos os seguintes pontos de dados para a
variável x: {[}3, 5, 7, 9, 11{]}, cuja média é 7.

\emph{Como já sabemos que média deste conjunto de dados é 7},
\emph{podemos calcular o valor específico de qualquer dado ausente}.

Por exemplo, \emph{suponha que ocultamos o primeiro valor (3)}.

Mesmo sem vê-lo, \emph{sabemos que seu valor deve ser 3, pois a média de
7 implica que a soma de todos os valores é}

7 * n = 35 e 35 − (5 + 7 + 9 + 11) = 3.

\begin{figure}[H]

{\centering \pandocbounded{\includegraphics[keepaspectratio]{fig/Poldrack-tab-5.3-estimativas-variancias-n-versus-n-1.png}}

}

\caption{Estimativas de Variância com n versus n-1}

\end{figure}%

Assim sendo, quando dizemos que \ul{\textbf{\emph{``perdemos'' um grau
de liberdade}}}, \emph{significa} que \emph{há um valor que não é livre
para variar após ajustarmos o modelo}.

No contexto da \ul{\textbf{variância amostral}}, \ul{\textbf{se
desconsiderarmos o grau de liberdade perdido}}, nossa
\ul{\textbf{estimativa da variância amostral será enviesada}}, fazendo
com que \ul{\textbf{\emph{subestimemos a incerteza}}} de
\ul{\textbf{\emph{nossa estimativa da média}}}.

\section{Usando Simulações para Entender
Estatística}\label{usando-simulauxe7uxf5es-para-entender-estatuxedstica}

Sou defensor ferrenho do uso de \ul{\textbf{simulações}} computacionais
\textbf{\emph{para entender conceitos estatísticos}}.

Nos capítulos posteriores, nós o exploraremos de forma mais aprofundada.

Aqui, apresentamos a ideia \textbf{\emph{questionando se podemos
confirmar a necessidade de subtrair 1}} {[}do tamanho n{]} da amostra ao
calcular a \ul{\textbf{variância amostral}}.

\textbf{\emph{Consideraremos toda a amostra de crianças do conjunto de
dados do NHANES}} como nossa ``\ul{\textbf{população}}''.

Queremos \textbf{\emph{avaliar}} a \textbf{\emph{eficiência}} dos
cálculos de \ul{\textbf{variância amostral}}, \ul{\textbf{usando n ou n
− 1 no denominador}} \ul{\textbf{\emph{para estimar a variância dessa
população}}}, considerando \emph{um grande número de amostras
aleatórias} \emph{simuladas} a \emph{partir dos dados}.

No Capítulo 8, abordaremos os detalhes de como fazer isso.

Na Tabela 5.3 {[}acima{]}, os \ul{\textbf{\emph{resultados mostram que a
teoria descrita anteriormente estava correta}}}: a
\textbf{\emph{estimativa da variância usando n − 1 como denominador é
bem próxima da variância calculada em todos os dados}} (ou seja,
\textbf{\emph{a população}}), enquanto \textbf{\emph{aquela calculada
usando n como denominador é enviesada}} (menor) em comparação ao
{[}\ul{\textbf{verdadeiro e, neste caso, conhecido}}{]}
\ul{\textbf{valor}} lógico {[}do \textbf{parâmetro populacional}{]}.

Como analisaremos mais tarde, \ul{\textbf{\emph{subestimar a variância é
bastante complicado porque nos torna excessivamente confiantes}}} em
relação às nossas \ul{\textbf{decisões estatísticas}}. {[}os Testes de
Significância da Hipótese Nula; \ul{\textbf{NHST}} - \emph{Null
Hypotesis Signifcant Test}{]}

\section{Z-Scores}\label{z-scores}

Após caracterizar uma distribuição em termos de tendência central e de
variabilidade, geralmente é útil expressar os \textbf{\emph{escores
individuais em termos de sua posição relativa à distribuição geral}}.

Suponha que estamos interessados em caracterizar o \textbf{\emph{nível
relativo de crimes em diferentes estados}}, a fim de determinar se a
Califórnia é um lugar especialmente perigoso.

Podemos responder a essa pergunta com os dados de 2014 do site
\emph{Uniform Crime Reporting} do \textbf{FBI}.

O painel \textbf{A} da Figura 5.8 mostra \textbf{\emph{um histograma do
número de crimes violentos por estado}}, destacando o valor da
Califórnia.

Observando esses dados, parece que a Califórnia é absurdamente perigosa,
com 153.709 crimes em 2014.

Podemos \textbf{\emph{visualizar esses dados gerando um mapa}}, que
mostra a distribuição de uma variável entre os estados, apresentado no
painel \textbf{B} da Figura 5.8 {[}abaixo{]}.

\begin{Shaded}
\begin{Highlighting}[numbers=left,,]
\InformationTok{\textasciigrave{}\textasciigrave{}\textasciigrave{}\{r\}}
\NormalTok{crimeData }\OtherTok{\textless{}{-}}
  \FunctionTok{read.table}\NormalTok{(}
    \StringTok{"https://raw.githubusercontent.com/statsthinking21/statsthinking21{-}figures{-}data/main/CrimeOneYearofData\_clean.csv"}\NormalTok{,}
    \AttributeTok{header =} \ConstantTok{TRUE}\NormalTok{,}
    \AttributeTok{sep =} \StringTok{","}
\NormalTok{  )}

\CommentTok{\# let\textquotesingle{}s drop DC since it is so small}
\NormalTok{crimeData }\OtherTok{\textless{}{-}}
\NormalTok{  crimeData }\SpecialCharTok{\%\textgreater{}\%}
\NormalTok{  dplyr}\SpecialCharTok{::}\FunctionTok{filter}\NormalTok{(State }\SpecialCharTok{!=} \StringTok{"District of Columbia"}\NormalTok{)}

\NormalTok{caCrimeData }\OtherTok{\textless{}{-}}
\NormalTok{  crimeData }\SpecialCharTok{\%\textgreater{}\%}
\NormalTok{  dplyr}\SpecialCharTok{::}\FunctionTok{filter}\NormalTok{(State }\SpecialCharTok{==} \StringTok{"California"}\NormalTok{)}

\NormalTok{p1 }\OtherTok{\textless{}{-}}\NormalTok{ crimeData }\SpecialCharTok{\%\textgreater{}\%}
  \FunctionTok{ggplot}\NormalTok{(}\FunctionTok{aes}\NormalTok{(Violent.crime.total)) }\SpecialCharTok{+}
  \FunctionTok{geom\_histogram}\NormalTok{(}\AttributeTok{bins =} \DecValTok{25}\NormalTok{) }\SpecialCharTok{+}
  \FunctionTok{geom\_vline}\NormalTok{(}\AttributeTok{xintercept =}\NormalTok{ caCrimeData}\SpecialCharTok{$}\NormalTok{Violent.crime.total, }\AttributeTok{color =} \StringTok{"blue"}\NormalTok{) }\SpecialCharTok{+}
  \FunctionTok{xlab}\NormalTok{(}\StringTok{"Number of violent crimes in 2014"}\NormalTok{)}

\FunctionTok{library}\NormalTok{(mapproj)}
\FunctionTok{library}\NormalTok{(fiftystater)}
\CommentTok{\# É necessário instalar antes o pacote devtools}
\CommentTok{\# retirar o hashtag do início da próxima linha e executar ela só 1 vez}
\CommentTok{\# install.packages("devtools")}

\CommentTok{\# Para depois instalar o pacote fiftystater}
\CommentTok{\# Baixando{-}o do Git Hub:}
\CommentTok{\# retirar o hashtag do início da próxima linha e executar ela só 1 vez}
\CommentTok{\# devtools::install\_github("wmurphyrd/fiftystater")}

\FunctionTok{data}\NormalTok{(}\StringTok{"fifty\_states"}\NormalTok{) }\CommentTok{\# this line is optional due to lazy data loading}

\NormalTok{crimeData }\OtherTok{\textless{}{-}}
\NormalTok{  crimeData }\SpecialCharTok{\%\textgreater{}\%}
  \FunctionTok{mutate}\NormalTok{(}\AttributeTok{StateLower =} \FunctionTok{tolower}\NormalTok{(State),}
         \AttributeTok{Violent.crime.thousands =}\NormalTok{ Violent.crime.total}\SpecialCharTok{/}\DecValTok{1000}\NormalTok{)}

\CommentTok{\# map\_id creates the aesthetic mapping to the state name column in your data}
\NormalTok{plot\_map }\OtherTok{\textless{}{-}}
  \FunctionTok{ggplot}\NormalTok{(crimeData, }\FunctionTok{aes}\NormalTok{(}\AttributeTok{map\_id =}\NormalTok{ StateLower)) }\SpecialCharTok{+}
  \CommentTok{\# map points to the fifty\_states shape data}
  \FunctionTok{geom\_map}\NormalTok{(}\FunctionTok{aes}\NormalTok{(}\AttributeTok{fill =}\NormalTok{ Violent.crime.thousands), }\AttributeTok{map =}\NormalTok{ fifty\_states) }\SpecialCharTok{+}
  \FunctionTok{scale\_x\_continuous}\NormalTok{(}\AttributeTok{breaks =} \ConstantTok{NULL}\NormalTok{) }\SpecialCharTok{+}
  \FunctionTok{scale\_y\_continuous}\NormalTok{(}\AttributeTok{breaks =} \ConstantTok{NULL}\NormalTok{) }\SpecialCharTok{+}
  \FunctionTok{theme}\NormalTok{(}
    \AttributeTok{legend.title =} \FunctionTok{element\_text}\NormalTok{(}
      \AttributeTok{size =} \DecValTok{10}\NormalTok{,         }\CommentTok{\# tamanho da fonte}
      \AttributeTok{face =} \StringTok{"bold"}\NormalTok{,     }\CommentTok{\# negrito}
      \AttributeTok{color =} \StringTok{"blue"}\NormalTok{,    }\CommentTok{\# cor do texto}
      \AttributeTok{hjust =} \DecValTok{0}          \CommentTok{\# alinhamento horizontal (0 = esquerda, 0.5 = centro, 1 = direita)}
\NormalTok{      ),}
    \AttributeTok{legend.position  =} \StringTok{"bottom"}\NormalTok{,}
    \AttributeTok{panel.background =} \FunctionTok{element\_blank}\NormalTok{()}
\NormalTok{  ) }\SpecialCharTok{+}
  \FunctionTok{coord\_map}\NormalTok{() }\SpecialCharTok{+}
  \FunctionTok{expand\_limits}\NormalTok{(}\AttributeTok{x =}\NormalTok{ fifty\_states}\SpecialCharTok{$}\NormalTok{long, }\AttributeTok{y =}\NormalTok{ fifty\_states}\SpecialCharTok{$}\NormalTok{lat) }\SpecialCharTok{+}
  \FunctionTok{labs}\NormalTok{(}
    \AttributeTok{x =} \StringTok{""}\NormalTok{,}
    \AttributeTok{y =} \StringTok{""}
\NormalTok{  )  }\SpecialCharTok{+}
  \CommentTok{\# inverter o gradiente de cores para estados com mais crimes violentos}
  \CommentTok{\# serem representados pela cor azul mais escura.}
  \CommentTok{\# O oposto pela cor azul mais claro.}
  \FunctionTok{scale\_fill\_gradient}\NormalTok{(}
    \AttributeTok{low =} \StringTok{"lightblue"}\NormalTok{, }\CommentTok{\# cor para valores baixos}
    \AttributeTok{high =} \StringTok{"darkblue"}  \CommentTok{\# cor para valores altos}
\NormalTok{  )}

\CommentTok{\# add border boxes to AK/HI}
\NormalTok{p2 }\OtherTok{\textless{}{-}}\NormalTok{ plot\_map }\SpecialCharTok{+} \FunctionTok{fifty\_states\_inset\_boxes}\NormalTok{()}

\CommentTok{\# plot\_grid(p1, p2)}

\CommentTok{\# Criando a nota de rodapé como grob com tamanho de fonte personalizado}
\NormalTok{nota\_rodape }\OtherTok{\textless{}{-}} \FunctionTok{textGrob}\NormalTok{(}
  \StringTok{"(A) Histograma do número de crimes violentos em 2014 [Number of violent crimes in 2014]. O valor para a Califórnia aparece como}\SpecialCharTok{\textbackslash{}n}\StringTok{uma linha vertical ao lado direito do gráfico. (B) Um mapa dos mesmos dados, com o número de crimes (em milhares) plotados por estado.}\SpecialCharTok{\textbackslash{}n}\StringTok{Fonte: Poldrack, 2025, p. 55, fig. 5.8 (apenas com a inversão do gradiente de cores do original)"}\NormalTok{,}
  \AttributeTok{gp =} \FunctionTok{gpar}\NormalTok{(}\AttributeTok{fontsize =} \DecValTok{8}\NormalTok{,}
            \AttributeTok{fontface =} \StringTok{"italic"}\NormalTok{),}
  \AttributeTok{x =} \FloatTok{0.5}\NormalTok{,}
  \AttributeTok{hjust =} \FloatTok{0.5}
\NormalTok{)}

\CommentTok{\# Usando grid.arrange com a nota de rodapé customizada}
\CommentTok{\# usando grid.arrange() (do pacote gridExtra), pode usar o argumento bottom:}
\CommentTok{\# para gerar um nota de rodapé ´no gráfico final}
\FunctionTok{grid.arrange}\NormalTok{(}
\NormalTok{  p1, p2,}
  \AttributeTok{ncol =} \DecValTok{2}\NormalTok{,}
  \AttributeTok{bottom =}\NormalTok{ nota\_rodape}
\NormalTok{)}
\InformationTok{\textasciigrave{}\textasciigrave{}\textasciigrave{}}
\end{Highlighting}
\end{Shaded}

\pandocbounded{\includegraphics[keepaspectratio]{cap5-pldr-modelos-dados_files/figure-pdf/unnamed-chunk-10-1.pdf}}

No entanto, pode ter lhe ocorrido que a \texttt{Califórnia} também tem
\textbf{\emph{a maior população entre os estados dos EUA}}.

Logo, é \textbf{\emph{plausível que também tenha um número maior de
crimes}}.

Se plotarmos o \textbf{\emph{número de crimes em relação à população de
cada estado}} (painel A da figura 5.9 {[}abaixo{]}),
\textbf{\emph{veremos uma relação direta entre as duas variáveis}}.

\begin{Shaded}
\begin{Highlighting}[numbers=left,,]
\InformationTok{\textasciigrave{}\textasciigrave{}\textasciigrave{}\{r\}}
\NormalTok{p1 }\OtherTok{\textless{}{-}}\NormalTok{ crimeData }\SpecialCharTok{\%\textgreater{}\%}
  \FunctionTok{ggplot}\NormalTok{(}\FunctionTok{aes}\NormalTok{(Population, Violent.crime.total)) }\SpecialCharTok{+}
  \FunctionTok{geom\_point}\NormalTok{() }\SpecialCharTok{+}
  \FunctionTok{annotate}\NormalTok{(}
    \StringTok{"point"}\NormalTok{,}
    \AttributeTok{x =}\NormalTok{ caCrimeData}\SpecialCharTok{$}\NormalTok{Population,}
    \AttributeTok{y =}\NormalTok{ caCrimeData}\SpecialCharTok{$}\NormalTok{Violent.crime.total,}
    \AttributeTok{color =} \StringTok{"blue"}
\NormalTok{  ) }\SpecialCharTok{+}
  \FunctionTok{annotate}\NormalTok{(}
    \StringTok{"text"}\NormalTok{,}
    \AttributeTok{x =}\NormalTok{ caCrimeData}\SpecialCharTok{$}\NormalTok{Population }\SpecialCharTok{{-}} \DecValTok{1000000}\NormalTok{,}
    \AttributeTok{y =}\NormalTok{ caCrimeData}\SpecialCharTok{$}\NormalTok{Violent.crime.total }\SpecialCharTok{+} \DecValTok{8000}\NormalTok{,}
    \AttributeTok{label =} \StringTok{"CA"}\NormalTok{,}
    \AttributeTok{color =} \StringTok{"blue"}
\NormalTok{  ) }\SpecialCharTok{+}
  \FunctionTok{ylab}\NormalTok{(}\StringTok{"Number of violent crimes in 2014"}\NormalTok{)}

\NormalTok{p2 }\OtherTok{\textless{}{-}}\NormalTok{ crimeData }\SpecialCharTok{\%\textgreater{}\%}
  \FunctionTok{ggplot}\NormalTok{(}\FunctionTok{aes}\NormalTok{(Violent.Crime.rate)) }\SpecialCharTok{+}
  \FunctionTok{geom\_histogram}\NormalTok{(}\AttributeTok{binwidth =} \DecValTok{80}\NormalTok{) }\SpecialCharTok{+}
  \FunctionTok{geom\_vline}\NormalTok{(}\AttributeTok{xintercept =}\NormalTok{ caCrimeData}\SpecialCharTok{$}\NormalTok{Violent.Crime.rate, }\AttributeTok{color =} \StringTok{"blue"}\NormalTok{) }\SpecialCharTok{+}
  \FunctionTok{annotate}\NormalTok{(}
    \StringTok{"text"}\NormalTok{,}
    \AttributeTok{x =}\NormalTok{ caCrimeData}\SpecialCharTok{$}\NormalTok{Violent.Crime.rate}\SpecialCharTok{+}\DecValTok{25}\NormalTok{,}
    \AttributeTok{y =} \DecValTok{12}\NormalTok{,}
    \AttributeTok{label =} \StringTok{"CA"}\NormalTok{,}
    \AttributeTok{color =} \StringTok{"blue"}
\NormalTok{  ) }\SpecialCharTok{+}
  \FunctionTok{scale\_x\_continuous}\NormalTok{(}\AttributeTok{breaks =} \FunctionTok{seq.int}\NormalTok{(}\DecValTok{0}\NormalTok{, }\DecValTok{700}\NormalTok{, }\DecValTok{100}\NormalTok{)) }\SpecialCharTok{+}
  \FunctionTok{scale\_y\_continuous}\NormalTok{(}\AttributeTok{breaks =} \FunctionTok{seq.int}\NormalTok{(}\DecValTok{0}\NormalTok{, }\DecValTok{13}\NormalTok{, }\DecValTok{2}\NormalTok{)) }\SpecialCharTok{+}
  \FunctionTok{xlab}\NormalTok{(}\StringTok{"Rate of violent crimes per 100,000 people"}\NormalTok{) }\SpecialCharTok{+}
  \FunctionTok{labs}\NormalTok{(}\AttributeTok{x =} \StringTok{"Rate of violent crimes per 100,000 people"}\NormalTok{) }\SpecialCharTok{+} \CommentTok{\# label do eixo x}
  \FunctionTok{theme}\NormalTok{(}
    \AttributeTok{axis.title.x =} \FunctionTok{element\_text}\NormalTok{(}\AttributeTok{size =} \DecValTok{12}\NormalTok{) }\CommentTok{\# tamanho da fonte do título do eixo x}
\NormalTok{    )}

\CommentTok{\# plot\_grid(p1, p2)}

\CommentTok{\# Criando a nota de rodapé como grob com tamanho de fonte personalizado}
\NormalTok{nota\_rodape }\OtherTok{\textless{}{-}} \FunctionTok{textGrob}\NormalTok{(}
  \StringTok{"(A) Um gráfico com o número de crimes violentos em 2014 [Z{-}scored rate of violent crimes] e a população [Population] por estado.}\SpecialCharTok{\textbackslash{}n}\StringTok{(B) Um histograma das taxas de crimes violentos per capita, expressos como crimes a cada 100.000 habitantes [Rate of violent crimes}\SpecialCharTok{\textbackslash{}n}\StringTok{per 100,000 people]. Fonte: Poldrack, 2025, p. 56, fig. 5.9 (apenas com ajuste do tamanho da fonte do eixo x original)"}\NormalTok{,}
  \AttributeTok{gp =} \FunctionTok{gpar}\NormalTok{(}\AttributeTok{fontsize =} \DecValTok{8}\NormalTok{,}
            \AttributeTok{fontface =} \StringTok{"italic"}\NormalTok{),}
  \AttributeTok{x =} \FloatTok{0.5}\NormalTok{,}
  \AttributeTok{hjust =} \FloatTok{0.5}
\NormalTok{  )}

\CommentTok{\# Usando grid.arrange com a nota de rodapé customizada}
\CommentTok{\# usando grid.arrange() (do pacote gridExtra), pode usar o argumento bottom:}
\CommentTok{\# para gerar um nota de rodapé ´no gráfico final}
\FunctionTok{grid.arrange}\NormalTok{(}
\NormalTok{  p1, p2,}
  \AttributeTok{ncol =} \DecValTok{2}\NormalTok{,}
  \AttributeTok{bottom =}\NormalTok{ nota\_rodape}
\NormalTok{)}
\InformationTok{\textasciigrave{}\textasciigrave{}\textasciigrave{}}
\end{Highlighting}
\end{Shaded}

\pandocbounded{\includegraphics[keepaspectratio]{cap5-pldr-modelos-dados_files/figure-pdf/unnamed-chunk-11-1.pdf}}

Em vez de usar o número bruto de crimes, \ul{\textbf{devemos usar a taxa
de crimes violentos \emph{per capita}}}, obtida
\textbf{\emph{dividindo}} o \ul{\textbf{\emph{número de crimes por
estado pela população de cada um}}}.

O \emph{conjunto de dados do FBI} já \emph{inclui} esse valor (expresso
como \emph{taxa a cada 100.000 habitantes}).

Observando o painel \textbf{B} da Figura 5.9 {[}acima{]}, percebemos que
a \textbf{Califórnia} \ul{\textbf{\emph{não é tão perigosa assim}}}.

A \textbf{\emph{taxa de criminalidade}} de \ul{\textbf{396,10 por
100.000 pessoas}} está \textbf{\emph{um pouco acima da média}} de
\textbf{346,81} entre os estados, mas dentro do intervalo de muitos
outros.

Mas e \textbf{\emph{se quisermos ter uma visão mais clara da distância
da Califórnia em relação ao restante da distribuição?}}

O \ul{\textbf{Z-score}} nos possibilita expressar os dados, de modo que
forneça mais \emph{insights} sobre a \textbf{\emph{relação de cada ponto
de dados com a distribuição geral}}.

A fórmula que calcula um \ul{\textbf{Z-score}} para \ul{\textbf{\emph{um
ponto individual de dados}}}, considerando a \textbf{\emph{média da
população}} (μ) e o \textbf{\emph{desvio-padrão}} (σ)
\textbf{\emph{conhecidos}}, é:

\[
Z(x) = \frac{x - \mu}{\sigma}
\]

Intuitivamente, podemos considerar o \ul{\textbf{Z-score}} como
\ul{\textbf{\emph{um indicador da distância}}} de \ul{\textbf{qualquer
ponto de dados em relação à média}}, \ul{\textbf{mensurada}}
\ul{\textbf{\emph{em unidades de desvio-padrão}}}.

Podemos \emph{calcular esse escore para os dados da taxa de
criminalidade}, conforme mostrado na Figura 5.10, que \textbf{plota os
Z-scores} em \textbf{\emph{relação aos escores originais}}.

A seguir, o \ul{\textbf{gráfico de dispersão mostra que o processo de
Z-score não altera a distribuição relativa dos pontos de dados}}
(visível porque os \emph{dados originais e os com Z-score ficam em uma
linha reta} quando plotados comparativamente em um gráfico) ---
\ul{\textbf{apenas os desloca para ter uma média de 0 e um desvio padrão
de 1}}.

A Figura 5.11 demonstra os \ul{\textbf{dados de crimes com Z-score}}
usando \textbf{\emph{uma representação geográfica}}.

Ela nos \textbf{\emph{proporciona uma perspectiva}} um pouco
\textbf{\emph{mais interpretável dos dados}}.

Por exemplo, podemos observar que \texttt{Nevada}, \texttt{Tennessee} e
\texttt{Novo\ México} apresentam \textbf{\emph{taxas de criminalidade
que estão aproximadamente dois desvios-padrão acima da média}}.

\begin{Shaded}
\begin{Highlighting}[numbers=left,,]
\InformationTok{\textasciigrave{}\textasciigrave{}\textasciigrave{}\{r\}}
\CommentTok{\# calcuar o Z{-}Score}
\NormalTok{crimeData }\OtherTok{\textless{}{-}}
\NormalTok{  crimeData }\SpecialCharTok{\%\textgreater{}\%}
  \FunctionTok{mutate}\NormalTok{(}
    \AttributeTok{ViolentCrimeRateZscore =}
\NormalTok{      (Violent.Crime.rate }\SpecialCharTok{{-}} \FunctionTok{mean}\NormalTok{(Violent.Crime.rate)) }\SpecialCharTok{/}
      \FunctionTok{sd}\NormalTok{(crimeData}\SpecialCharTok{$}\NormalTok{Violent.Crime.rate)}
\NormalTok{    )}

\NormalTok{caCrimeData }\OtherTok{\textless{}{-}}
\NormalTok{  crimeData }\SpecialCharTok{\%\textgreater{}\%}
\NormalTok{  dplyr}\SpecialCharTok{::}\FunctionTok{filter}\NormalTok{(State }\SpecialCharTok{==} \StringTok{"California"}\NormalTok{)}

\NormalTok{media\_x }\OtherTok{\textless{}{-}} \FunctionTok{mean}\NormalTok{(crimeData}\SpecialCharTok{$}\NormalTok{Violent.Crime.rate)}

\NormalTok{crimeData }\SpecialCharTok{\%\textgreater{}\%}
  \FunctionTok{ggplot}\NormalTok{(}\FunctionTok{aes}\NormalTok{(Violent.Crime.rate, ViolentCrimeRateZscore)) }\SpecialCharTok{+}
  \FunctionTok{geom\_point}\NormalTok{() }\SpecialCharTok{+}
  \FunctionTok{annotate}\NormalTok{(}
    \StringTok{"point"}\NormalTok{,}
    \AttributeTok{x =}\NormalTok{ caCrimeData}\SpecialCharTok{$}\NormalTok{Violent.Crime.rate,}
    \AttributeTok{y =}\NormalTok{ caCrimeData}\SpecialCharTok{$}\NormalTok{ViolentCrimeRateZscore,}
    \AttributeTok{color =} \StringTok{"blue"}\NormalTok{,}
    \AttributeTok{size  =} \DecValTok{3}\NormalTok{,}
    \AttributeTok{alpha =} \FloatTok{0.5}
\NormalTok{  ) }\SpecialCharTok{+}
  \FunctionTok{annotate}\NormalTok{(}
    \StringTok{"text"}\NormalTok{,}
    \AttributeTok{x =}\NormalTok{ caCrimeData}\SpecialCharTok{$}\NormalTok{Violent.Crime.rate }\SpecialCharTok{{-}} \DecValTok{5}\NormalTok{,}
    \AttributeTok{y =}\NormalTok{ caCrimeData}\SpecialCharTok{$}\NormalTok{ViolentCrimeRateZscore }\SpecialCharTok{+} \FloatTok{0.30}\NormalTok{,}
    \AttributeTok{label =} \StringTok{"CA"}\NormalTok{,}
    \AttributeTok{color =} \StringTok{"blue"}
\NormalTok{  ) }\SpecialCharTok{+}
  \FunctionTok{labs}\NormalTok{(}
    \AttributeTok{x =} \StringTok{"Rate of violent crimes per 100,000 people"}\NormalTok{,}
    \AttributeTok{y =} \StringTok{"Z{-}scored of rate of violent crimes"}\NormalTok{,}
    \AttributeTok{caption  =} \StringTok{"Gráfico de dispersão dos dados originais da taxa de criminalidade em relação aos dados com Z{-}score.}\SpecialCharTok{\textbackslash{}n}\StringTok{Fonte: Poldrack, 2025, p. 57, fig. 5.10 (acrescida a taxa média e a California {-} CA como ponto azul claro)"}
\NormalTok{  ) }\SpecialCharTok{+}
  \FunctionTok{geom\_vline}\NormalTok{(}\AttributeTok{xintercept =}\NormalTok{ media\_x,}
             \AttributeTok{color      =} \StringTok{"blue"}\NormalTok{,}
             \AttributeTok{linetype   =} \StringTok{"dashed"}\NormalTok{,}
             \AttributeTok{linewidth  =} \DecValTok{1}
\NormalTok{             ) }\SpecialCharTok{+}
  \FunctionTok{annotate}\NormalTok{(}
    \StringTok{"text"}\NormalTok{,}
    \AttributeTok{x =}\NormalTok{ media\_x }\SpecialCharTok{+} \DecValTok{70}\NormalTok{, }\CommentTok{\# ajuste conforme necessário para a posição horizontal}
    \AttributeTok{y =} \DecValTok{2}\NormalTok{,            }\CommentTok{\# ajuste conforme necessário para a posição vertical}
    \AttributeTok{label =} \FunctionTok{paste0}\NormalTok{(}\StringTok{"Média = "}\NormalTok{, }\FunctionTok{round}\NormalTok{(media\_x, }\DecValTok{1}\NormalTok{)),}
    \AttributeTok{color =} \StringTok{"blue"}\NormalTok{,}
    \AttributeTok{vjust =} \SpecialCharTok{{-}}\FloatTok{0.5}\NormalTok{,}
    \AttributeTok{fontface =} \StringTok{"bold"}
\NormalTok{  )}
\InformationTok{\textasciigrave{}\textasciigrave{}\textasciigrave{}}
\end{Highlighting}
\end{Shaded}

\pandocbounded{\includegraphics[keepaspectratio]{cap5-pldr-modelos-dados_files/figure-pdf/unnamed-chunk-12-1.pdf}}

Agora os dados de crime renderizados em um mapa dos EUA, apresentados
como \ul{\textbf{Z-scores}}.

\begin{Shaded}
\begin{Highlighting}[numbers=left,,]
\InformationTok{\textasciigrave{}\textasciigrave{}\textasciigrave{}\{r\}}
\NormalTok{plot\_map\_z }\OtherTok{\textless{}{-}}
  \FunctionTok{ggplot}\NormalTok{(crimeData, }\FunctionTok{aes}\NormalTok{(}\AttributeTok{map\_id =}\NormalTok{ StateLower)) }\SpecialCharTok{+}
  \CommentTok{\# map points to the fifty\_states shape data}
  \FunctionTok{geom\_map}\NormalTok{(}\FunctionTok{aes}\NormalTok{(}\AttributeTok{fill =}\NormalTok{ ViolentCrimeRateZscore), }\AttributeTok{map =}\NormalTok{ fifty\_states) }\SpecialCharTok{+}
  \FunctionTok{expand\_limits}\NormalTok{(}\AttributeTok{x =}\NormalTok{ fifty\_states}\SpecialCharTok{$}\NormalTok{long, }\AttributeTok{y =}\NormalTok{ fifty\_states}\SpecialCharTok{$}\NormalTok{lat) }\SpecialCharTok{+}
  \FunctionTok{scale\_x\_continuous}\NormalTok{(}\AttributeTok{breaks =} \ConstantTok{NULL}\NormalTok{) }\SpecialCharTok{+}
  \FunctionTok{scale\_y\_continuous}\NormalTok{(}\AttributeTok{breaks =} \ConstantTok{NULL}\NormalTok{) }\SpecialCharTok{+}
  \FunctionTok{theme}\NormalTok{(}
    \AttributeTok{legend.position =} \StringTok{"top"}\NormalTok{,}
    \AttributeTok{panel.background =} \FunctionTok{element\_blank}\NormalTok{()}
\NormalTok{  ) }\SpecialCharTok{+}
  \FunctionTok{coord\_map}\NormalTok{() }\SpecialCharTok{+}
  \FunctionTok{expand\_limits}\NormalTok{(}\AttributeTok{x =}\NormalTok{ fifty\_states}\SpecialCharTok{$}\NormalTok{long, }\AttributeTok{y =}\NormalTok{ fifty\_states}\SpecialCharTok{$}\NormalTok{lat) }\SpecialCharTok{+}
  \FunctionTok{labs}\NormalTok{(}\AttributeTok{x =} \StringTok{""}\NormalTok{,}
       \AttributeTok{y =} \StringTok{""}\NormalTok{) }\SpecialCharTok{+}
  \CommentTok{\# inverter o gradiente de cores para estados com mais crimes violentos}
  \CommentTok{\# serem representados pela cor azul mais escura.}
  \CommentTok{\# O oposto pela cor azul mais claro.}
  \FunctionTok{scale\_fill\_gradient}\NormalTok{(}
    \AttributeTok{low =} \StringTok{"lightblue"}\NormalTok{,   }\CommentTok{\# cor para valores baixos}
    \AttributeTok{high =} \StringTok{"darkblue"}  \CommentTok{\# cor para valores altos}
\NormalTok{  )}
  
\NormalTok{final\_plot }\OtherTok{\textless{}{-}} \FunctionTok{ggdraw}\NormalTok{(plot\_map\_z) }\SpecialCharTok{+}
  \FunctionTok{draw\_label}\NormalTok{(}
    \StringTok{"Dados de crime renderizados em um mapa dos EUA, apresentados como Z{-}scores.}\SpecialCharTok{\textbackslash{}n}\StringTok{Fonte: Poldrack, 2025, p. 57, fig. 5.11 (apenas com a inversão do gradiente de cores do original)"}\NormalTok{,}
    \AttributeTok{x =} \FloatTok{0.5}\NormalTok{, }\AttributeTok{y =} \FloatTok{0.02}\NormalTok{, }\AttributeTok{hjust =} \FloatTok{0.5}\NormalTok{, }\AttributeTok{vjust =} \DecValTok{0}\NormalTok{,}
    \AttributeTok{fontface =} \StringTok{"italic"}\NormalTok{, }\AttributeTok{size =} \DecValTok{10}
\NormalTok{  )}

\CommentTok{\# print(final\_plot)}

\CommentTok{\# add border boxes to AK/HI}
\NormalTok{final\_plot }\SpecialCharTok{+} \FunctionTok{fifty\_states\_inset\_boxes}\NormalTok{()}
\InformationTok{\textasciigrave{}\textasciigrave{}\textasciigrave{}}
\end{Highlighting}
\end{Shaded}

\pandocbounded{\includegraphics[keepaspectratio]{cap5-pldr-modelos-dados_files/figure-pdf/unnamed-chunk-13-1.pdf}}

\section{Interpretando Z-scores}\label{interpretando-z-scores}

\textbf{O ``Z'' em Z-score se origina do fato de a distribuição normal
padrão} (ou seja, uma distribuição normal com uma média de 0 e um
desvio-padrão de 1) frequentemente ser chamada de
\ul{\textbf{\emph{distribuição Z}}}.

Podemos usar a distribuição normal padrão para nos ajudar a compreender
\textbf{\emph{o que os Z-scores específicos nos revelam}} sobre a
posição de um ponto de dados em relação ao restante da distribuição.

Na Figura 5.12, a coluna da esquerda mostra que \textbf{\emph{esperamos
que cerca de 16\% dos valores estejam em Z ≥ 1 e a mesma proporção, em Z
≤ −1}}.

Na Figura 5.12, a coluna da direita apresenta o mesmo gráfico, mas
\textbf{\emph{para dois desvios-padrão}}.

\begin{Shaded}
\begin{Highlighting}[numbers=left,,]
\InformationTok{\textasciigrave{}\textasciigrave{}\textasciigrave{}\{r\}}
\CommentTok{\# First, create a function to generate plots of the density and CDF}
\NormalTok{dnormfun }\OtherTok{\textless{}{-}} \ControlFlowTok{function}\NormalTok{(x) \{}
  \FunctionTok{return}\NormalTok{(}\FunctionTok{dnorm}\NormalTok{(x, }\DecValTok{248}\NormalTok{))}
\NormalTok{\}}

\NormalTok{plot\_density\_and\_cdf }\OtherTok{\textless{}{-}}
  \ControlFlowTok{function}\NormalTok{(zcut, }\AttributeTok{zmin =} \SpecialCharTok{{-}}\DecValTok{4}\NormalTok{, }\AttributeTok{zmax =} \DecValTok{4}\NormalTok{, }\AttributeTok{plot\_cdf =} \ConstantTok{TRUE}\NormalTok{, }\AttributeTok{zmean =} \DecValTok{0}\NormalTok{, }\AttributeTok{zsd =} \DecValTok{1}\NormalTok{) \{}
\NormalTok{    zmin }\OtherTok{\textless{}{-}}\NormalTok{ zmin }\SpecialCharTok{*}\NormalTok{ zsd }\SpecialCharTok{+}\NormalTok{ zmean}
\NormalTok{    zmax }\OtherTok{\textless{}{-}}\NormalTok{ zmax }\SpecialCharTok{*}\NormalTok{ zsd }\SpecialCharTok{+}\NormalTok{ zmean}
\NormalTok{    x }\OtherTok{\textless{}{-}} \FunctionTok{seq}\NormalTok{(zmin, zmax, }\FloatTok{0.1} \SpecialCharTok{*}\NormalTok{ zsd)}
\NormalTok{    zdist }\OtherTok{\textless{}{-}} \FunctionTok{dnorm}\NormalTok{(x, }\AttributeTok{mean =}\NormalTok{ zmean, }\AttributeTok{sd =}\NormalTok{ zsd)}
\NormalTok{    area }\OtherTok{\textless{}{-}} \FunctionTok{pnorm}\NormalTok{(zcut) }\SpecialCharTok{{-}} \FunctionTok{pnorm}\NormalTok{(}\SpecialCharTok{{-}}\NormalTok{zcut)}

\NormalTok{    p2 }\OtherTok{\textless{}{-}}
      \FunctionTok{tibble}\NormalTok{(}
        \AttributeTok{zdist =}\NormalTok{ zdist,}
        \AttributeTok{x =}\NormalTok{ x}
\NormalTok{      ) }\SpecialCharTok{\%\textgreater{}\%}
      \FunctionTok{ggplot}\NormalTok{(}\FunctionTok{aes}\NormalTok{(x, zdist)) }\SpecialCharTok{+}
      \FunctionTok{geom\_line}\NormalTok{(}
        \FunctionTok{aes}\NormalTok{(x, zdist),}
        \AttributeTok{color =} \StringTok{"red"}\NormalTok{,}
        \AttributeTok{size =} \DecValTok{2}
\NormalTok{      ) }\SpecialCharTok{+}
      \FunctionTok{stat\_function}\NormalTok{(}
        \AttributeTok{fun =}\NormalTok{ dnorm, }\AttributeTok{args =} \FunctionTok{list}\NormalTok{(}\AttributeTok{mean =}\NormalTok{ zmean, }\AttributeTok{sd =}\NormalTok{ zsd),}
        \AttributeTok{xlim =} \FunctionTok{c}\NormalTok{(zmean }\SpecialCharTok{{-}}\NormalTok{ zcut }\SpecialCharTok{*}\NormalTok{ zsd, zmean }\SpecialCharTok{+}\NormalTok{ zsd }\SpecialCharTok{*}\NormalTok{ zcut),}
        \AttributeTok{geom =} \StringTok{"area"}\NormalTok{, }\AttributeTok{fill =} \StringTok{"orange"}
\NormalTok{      ) }\SpecialCharTok{+}
      \FunctionTok{stat\_function}\NormalTok{(}
        \AttributeTok{fun =}\NormalTok{ dnorm, }\AttributeTok{args =} \FunctionTok{list}\NormalTok{(}\AttributeTok{mean =}\NormalTok{ zmean, }\AttributeTok{sd =}\NormalTok{ zsd),}
        \AttributeTok{xlim =} \FunctionTok{c}\NormalTok{(zmin, zmean }\SpecialCharTok{{-}}\NormalTok{ zcut }\SpecialCharTok{*}\NormalTok{ zsd),}
        \AttributeTok{geom =} \StringTok{"area"}\NormalTok{, }\AttributeTok{fill =} \StringTok{"green"}
\NormalTok{      ) }\SpecialCharTok{+}
      \FunctionTok{stat\_function}\NormalTok{(}
        \AttributeTok{fun =}\NormalTok{ dnorm, }\AttributeTok{args =} \FunctionTok{list}\NormalTok{(}\AttributeTok{mean =}\NormalTok{ zmean, }\AttributeTok{sd =}\NormalTok{ zsd),}
        \AttributeTok{xlim =} \FunctionTok{c}\NormalTok{(zmean }\SpecialCharTok{+}\NormalTok{ zcut }\SpecialCharTok{*}\NormalTok{ zsd, zmax),}
        \AttributeTok{geom =} \StringTok{"area"}\NormalTok{, }\AttributeTok{fill =} \StringTok{"green"}
\NormalTok{      ) }\SpecialCharTok{+}
      \FunctionTok{annotate}\NormalTok{(}
        \StringTok{"text"}\NormalTok{,}
        \AttributeTok{x =}\NormalTok{ zmean,}
        \AttributeTok{y =} \FunctionTok{dnorm}\NormalTok{(zmean, }\AttributeTok{mean =}\NormalTok{ zmean, }\AttributeTok{sd =}\NormalTok{ zsd) }\SpecialCharTok{/} \DecValTok{2}\NormalTok{,}
        \AttributeTok{size =} \DecValTok{5}\NormalTok{,}
        \AttributeTok{label =} \FunctionTok{sprintf}\NormalTok{(}\StringTok{"\%0.1f\%\%"}\NormalTok{, area }\SpecialCharTok{*} \DecValTok{100}\NormalTok{)}
\NormalTok{      ) }\SpecialCharTok{+}
      \FunctionTok{annotate}\NormalTok{(}
        \StringTok{"text"}\NormalTok{,}
        \AttributeTok{x =}\NormalTok{ zmean }\SpecialCharTok{{-}}\NormalTok{ zsd }\SpecialCharTok{*}\NormalTok{ zcut }\SpecialCharTok{{-}} \FloatTok{0.5} \SpecialCharTok{*}\NormalTok{ zsd,}
        \AttributeTok{y =} \FunctionTok{dnorm}\NormalTok{(zmean }\SpecialCharTok{{-}}\NormalTok{ zcut }\SpecialCharTok{*}\NormalTok{ zsd, }\AttributeTok{mean =}\NormalTok{ zmean, }\AttributeTok{sd =}\NormalTok{ zsd) }\SpecialCharTok{+} \FloatTok{0.01} \SpecialCharTok{/}\NormalTok{ zsd,}
        \AttributeTok{size =} \DecValTok{3}\NormalTok{,}
        \AttributeTok{label =} \FunctionTok{sprintf}\NormalTok{(}\StringTok{"\%0.1f\%\%"}\NormalTok{, }\FunctionTok{pnorm}\NormalTok{(zmean }\SpecialCharTok{{-}}\NormalTok{ zsd }\SpecialCharTok{*}\NormalTok{ zcut, }\AttributeTok{mean =}\NormalTok{ zmean, }\AttributeTok{sd =}\NormalTok{ zsd) }\SpecialCharTok{*} \DecValTok{100}\NormalTok{)}
\NormalTok{      ) }\SpecialCharTok{+}
      \FunctionTok{annotate}\NormalTok{(}
        \StringTok{"text"}\NormalTok{,}
        \AttributeTok{x =}\NormalTok{ zmean }\SpecialCharTok{+}\NormalTok{ zsd }\SpecialCharTok{*}\NormalTok{ zcut }\SpecialCharTok{+} \FloatTok{0.5} \SpecialCharTok{*}\NormalTok{ zsd,}
        \AttributeTok{y =} \FunctionTok{dnorm}\NormalTok{(zmean }\SpecialCharTok{{-}}\NormalTok{ zcut }\SpecialCharTok{*}\NormalTok{ zsd, }\AttributeTok{mean =}\NormalTok{ zmean, }\AttributeTok{sd =}\NormalTok{ zsd) }\SpecialCharTok{+} \FloatTok{0.01} \SpecialCharTok{/}\NormalTok{ zsd,}
        \AttributeTok{size =} \DecValTok{3}\NormalTok{,}
        \AttributeTok{label =} \FunctionTok{sprintf}\NormalTok{(}\StringTok{"\%0.1f\%\%"}\NormalTok{, (}\DecValTok{1} \SpecialCharTok{{-}} \FunctionTok{pnorm}\NormalTok{(zmean }\SpecialCharTok{+}\NormalTok{ zsd }\SpecialCharTok{*}\NormalTok{ zcut, }\AttributeTok{mean =}\NormalTok{ zmean, }\AttributeTok{sd =}\NormalTok{ zsd)) }\SpecialCharTok{*} \DecValTok{100}\NormalTok{)}
\NormalTok{      ) }\SpecialCharTok{+}
      \FunctionTok{xlim}\NormalTok{(zmin, zmax) }\SpecialCharTok{+}
      \FunctionTok{labs}\NormalTok{(}
        \AttributeTok{x =} \StringTok{"Z score"}\NormalTok{,}
        \AttributeTok{y =} \StringTok{"density"}
\NormalTok{      )}

\NormalTok{      cdf2 }\OtherTok{\textless{}{-}}
        \FunctionTok{tibble}\NormalTok{(}
          \AttributeTok{zdist =}\NormalTok{ zdist,}
          \AttributeTok{x =}\NormalTok{ x,}
          \AttributeTok{zcdf =} \FunctionTok{pnorm}\NormalTok{(x, }\AttributeTok{mean =}\NormalTok{ zmean, }\AttributeTok{sd =}\NormalTok{ zsd)}
\NormalTok{        ) }\SpecialCharTok{\%\textgreater{}\%}
        \FunctionTok{ggplot}\NormalTok{(}\FunctionTok{aes}\NormalTok{(x, zcdf)) }\SpecialCharTok{+}
        \FunctionTok{geom\_line}\NormalTok{() }\SpecialCharTok{+}
        \FunctionTok{annotate}\NormalTok{(}
          \StringTok{"segment"}\NormalTok{,}
          \AttributeTok{x =}\NormalTok{ zmin,}
          \AttributeTok{xend =}\NormalTok{ zmean }\SpecialCharTok{+}\NormalTok{ zsd }\SpecialCharTok{*}\NormalTok{ zcut,}
          \AttributeTok{y =} \FunctionTok{pnorm}\NormalTok{(zmean }\SpecialCharTok{+}\NormalTok{ zsd }\SpecialCharTok{*}\NormalTok{ zcut, }\AttributeTok{mean =}\NormalTok{ zmean, }\AttributeTok{sd =}\NormalTok{ zsd),}
          \AttributeTok{yend =} \FunctionTok{pnorm}\NormalTok{(zmean }\SpecialCharTok{+}\NormalTok{ zsd }\SpecialCharTok{*}\NormalTok{ zcut, }\AttributeTok{mean =}\NormalTok{ zmean, }\AttributeTok{sd =}\NormalTok{ zsd),}
          \AttributeTok{color =} \StringTok{"red"}\NormalTok{,}
          \AttributeTok{linetype =} \StringTok{"dashed"}
\NormalTok{        ) }\SpecialCharTok{+}
        \FunctionTok{annotate}\NormalTok{(}
          \StringTok{"segment"}\NormalTok{,}
          \AttributeTok{x =}\NormalTok{ zmean }\SpecialCharTok{+}\NormalTok{ zsd }\SpecialCharTok{*}\NormalTok{ zcut,}
          \AttributeTok{xend =}\NormalTok{ zmean }\SpecialCharTok{+}\NormalTok{ zsd }\SpecialCharTok{*}\NormalTok{ zcut,}
          \AttributeTok{y =} \DecValTok{0}\NormalTok{, }\AttributeTok{yend =} \FunctionTok{pnorm}\NormalTok{(zmean }\SpecialCharTok{+}\NormalTok{ zsd }\SpecialCharTok{*}\NormalTok{ zcut, }\AttributeTok{mean =}\NormalTok{ zmean, }\AttributeTok{sd =}\NormalTok{ zsd),}
          \AttributeTok{color =} \StringTok{"red"}\NormalTok{,}
          \AttributeTok{linetype =} \StringTok{"dashed"}
\NormalTok{        ) }\SpecialCharTok{+}
        \FunctionTok{annotate}\NormalTok{(}
          \StringTok{"segment"}\NormalTok{,}
          \AttributeTok{x =}\NormalTok{ zmin,}
          \AttributeTok{xend =}\NormalTok{ zmean }\SpecialCharTok{{-}}\NormalTok{ zcut }\SpecialCharTok{*}\NormalTok{ zsd,}
          \AttributeTok{y =} \FunctionTok{pnorm}\NormalTok{(zmean }\SpecialCharTok{{-}}\NormalTok{ zcut }\SpecialCharTok{*}\NormalTok{ zsd, }\AttributeTok{mean =}\NormalTok{ zmean, }\AttributeTok{sd =}\NormalTok{ zsd),}
          \AttributeTok{yend =} \FunctionTok{pnorm}\NormalTok{(zmean }\SpecialCharTok{{-}}\NormalTok{ zcut }\SpecialCharTok{*}\NormalTok{ zsd, }\AttributeTok{mean =}\NormalTok{ zmean, }\AttributeTok{sd =}\NormalTok{ zsd),}
          \AttributeTok{color =} \StringTok{"blue"}\NormalTok{,}
          \AttributeTok{linetype =} \StringTok{"dashed"}
\NormalTok{        ) }\SpecialCharTok{+}
        \FunctionTok{annotate}\NormalTok{(}
          \StringTok{"segment"}\NormalTok{,}
          \AttributeTok{x =}\NormalTok{ zmean }\SpecialCharTok{{-}}\NormalTok{ zcut }\SpecialCharTok{*}\NormalTok{ zsd,}
          \AttributeTok{xend =}\NormalTok{ zmean }\SpecialCharTok{{-}}\NormalTok{ zcut }\SpecialCharTok{*}\NormalTok{ zsd,}
          \AttributeTok{y =} \DecValTok{0}\NormalTok{,}
          \AttributeTok{yend =} \FunctionTok{pnorm}\NormalTok{(zmean }\SpecialCharTok{{-}}\NormalTok{ zcut }\SpecialCharTok{*}\NormalTok{ zsd, }\AttributeTok{mean =}\NormalTok{ zmean, }\AttributeTok{sd =}\NormalTok{ zsd),}
          \AttributeTok{color =} \StringTok{"blue"}\NormalTok{,}
          \AttributeTok{linetype =} \StringTok{"dashed"}
\NormalTok{        ) }\SpecialCharTok{+}
        \FunctionTok{ylab}\NormalTok{(}\StringTok{"Cumulative density"}\NormalTok{)}

    \FunctionTok{return}\NormalTok{(}\FunctionTok{list}\NormalTok{(}\AttributeTok{pdf =}\NormalTok{ p2, }\AttributeTok{cdf =}\NormalTok{ cdf2))}
\NormalTok{  \}}

\NormalTok{plots1 }\OtherTok{=} \FunctionTok{plot\_density\_and\_cdf}\NormalTok{(}\DecValTok{1}\NormalTok{)}
\NormalTok{plots2 }\OtherTok{=} \FunctionTok{plot\_density\_and\_cdf}\NormalTok{(}\DecValTok{2}\NormalTok{)}

\CommentTok{\# plot\_grid(plots1$pdf, plots2$pdf, plots1$cdf, plots2$cdf, nrow=2, ncol=2)}

\CommentTok{\# Criando a nota de rodapé como grob com tamanho de fonte personalizado}
\NormalTok{nota\_rodape }\OtherTok{\textless{}{-}} \FunctionTok{textGrob}\NormalTok{(}
  \StringTok{"Na parte superior temos a densidade [Density] e, na parte inferior, a distribuição cumulativa [Cumulative density]}\SpecialCharTok{\textbackslash{}n}\StringTok{de uma distribuição normal padrão, com pontos de cutoffs em um desvio{-}padrão acima/abaixo da média (coluna da esquerda) e}\SpecialCharTok{\textbackslash{}n}\StringTok{dois desvios{-}padrão (coluna da direita). Fonte: Poldrack, 2025, p. 58, fig. 5.12 (apenas com ajuste do tamanho da fonte das proporções \%)"}\NormalTok{,}
  \AttributeTok{gp =} \FunctionTok{gpar}\NormalTok{(}\AttributeTok{fontsize =} \DecValTok{8}\NormalTok{,}
            \AttributeTok{fontface =} \StringTok{"italic"}\NormalTok{),}
  \AttributeTok{x =} \FloatTok{0.5}\NormalTok{,}
  \AttributeTok{hjust =} \FloatTok{0.5}
\NormalTok{  )}

\CommentTok{\# Usando grid.arrange com a nota de rodapé customizada}
\CommentTok{\# usando grid.arrange() (do pacote gridExtra), pode usar o argumento bottom:}
\CommentTok{\# para gerar um nota de rodapé ´no gráfico final}
\FunctionTok{grid.arrange}\NormalTok{(}
\NormalTok{  plots1}\SpecialCharTok{$}\NormalTok{pdf, plots2}\SpecialCharTok{$}\NormalTok{pdf, plots1}\SpecialCharTok{$}\NormalTok{cdf, plots2}\SpecialCharTok{$}\NormalTok{cdf,}
  \AttributeTok{nrow =} \DecValTok{2}\NormalTok{,}
  \AttributeTok{ncol =} \DecValTok{2}\NormalTok{,}
  \AttributeTok{bottom =}\NormalTok{ nota\_rodape)}
\InformationTok{\textasciigrave{}\textasciigrave{}\textasciigrave{}}
\end{Highlighting}
\end{Shaded}

\pandocbounded{\includegraphics[keepaspectratio]{cap5-pldr-modelos-dados_files/figure-pdf/unnamed-chunk-14-1.pdf}}

Aqui, vemos que \textbf{\emph{apenas 2,3\% dos valores estão em Z ≤ −2 e
o mesmo em Z ≥ 2}}.

Portanto, \ul{\textbf{\emph{se soubermos o Z-score de um determinado
ponto de dados}}}, podemos \ul{\textbf{\emph{estimar}}} a
\ul{\textbf{probabilidade}} ou a \textbf{\emph{improbabilidade de
encontrarmos um valor tão extremo quanto esse}}.

Isso nos possibilita contextualizar melhor os valores.

No caso das taxas de criminalidade, observamos que a \texttt{Califórnia}
tem um \textbf{Z-score} de \textbf{0.38} em relação à \textbf{\emph{sua
taxa de crimes violentos per capita}}, indicando que \textbf{\emph{está
próxima da média de outros estados}}, pois \textbf{\emph{cerca de 35\%
desses apresentam taxas maiores}} e \textbf{\emph{cerca de 65\% deles
apresentam taxas menores}}.

\section{Escores Padronizados}\label{escores-padronizados}

Um \ul{\textbf{escore padronizado}} é \textbf{\emph{um Z-score que foi
transformado para ter média e desvio-padrão diferentes da distribuição
normal padrão}}.

Suponha que, \emph{em vez de Z-scores}, quiséssemos \textbf{\emph{gerar
escores padronizados de criminalidade com média de 100 e desvio-padrão
de 10}} (Figura 5.13).

Isso \textbf{\emph{se assemelha à padronização em notas de testes de
inteligência}} para gerar o \ul{\textbf{quociente de inteligência
(QI)}}.

Podemos fazer isso \textbf{\emph{simplesmente multiplicando os Z-scores
por 10 e, em seguida, somando 100}}.

\begin{Shaded}
\begin{Highlighting}[numbers=left,,]
\InformationTok{\textasciigrave{}\textasciigrave{}\textasciigrave{}\{r\}}
\NormalTok{crimeData }\OtherTok{\textless{}{-}}
\NormalTok{  crimeData }\SpecialCharTok{\%\textgreater{}\%}
  \FunctionTok{mutate}\NormalTok{(}
    \AttributeTok{ViolentCrimeRateStdScore =}\NormalTok{ (ViolentCrimeRateZscore) }\SpecialCharTok{*} \DecValTok{10} \SpecialCharTok{+} \DecValTok{100}
\NormalTok{  )}

\NormalTok{caCrimeData }\OtherTok{\textless{}{-}}
\NormalTok{  crimeData }\SpecialCharTok{\%\textgreater{}\%}
  \FunctionTok{filter}\NormalTok{(State }\SpecialCharTok{==} \StringTok{"California"}\NormalTok{)}

\NormalTok{crimeData }\SpecialCharTok{\%\textgreater{}\%}
  \FunctionTok{ggplot}\NormalTok{(}\FunctionTok{aes}\NormalTok{(ViolentCrimeRateStdScore)) }\SpecialCharTok{+}
  \FunctionTok{geom\_histogram}\NormalTok{(}\AttributeTok{binwidth =} \DecValTok{5}\NormalTok{,}
                 \AttributeTok{alpha    =} \FloatTok{0.6}\NormalTok{,}
                 \AttributeTok{fill  =} \StringTok{"gray"}\NormalTok{,  }\CommentTok{\# cor de preenchimento das barras}
                 \AttributeTok{color =} \StringTok{"black"}     \CommentTok{\# cor das linhas das bordas das barras}
\NormalTok{                 ) }\SpecialCharTok{+}
  \FunctionTok{geom\_vline}\NormalTok{(}\AttributeTok{xintercept =}\NormalTok{ caCrimeData}\SpecialCharTok{$}\NormalTok{ViolentCrimeRateStdScore,}
             \AttributeTok{color =} \StringTok{"darkblue"}\NormalTok{,}
             \AttributeTok{linetype =} \StringTok{"dashed"}\NormalTok{,}
             \AttributeTok{size =} \DecValTok{1}
\NormalTok{             ) }\SpecialCharTok{+}
  \FunctionTok{scale\_y\_continuous}\NormalTok{(}\AttributeTok{breaks =} \FunctionTok{seq.int}\NormalTok{(}\DecValTok{0}\NormalTok{, }\DecValTok{13}\NormalTok{, }\DecValTok{2}\NormalTok{)) }\SpecialCharTok{+}
  \FunctionTok{annotate}\NormalTok{(}
    \StringTok{"text"}\NormalTok{,}
    \AttributeTok{x =}\NormalTok{ caCrimeData}\SpecialCharTok{$}\NormalTok{ViolentCrimeRateStdScore }\SpecialCharTok{+} \DecValTok{6}\NormalTok{,}
    \AttributeTok{y =} \DecValTok{12}\NormalTok{,}
    \AttributeTok{label =} \FunctionTok{paste0}\NormalTok{(}\StringTok{"California = "}\NormalTok{, }\FunctionTok{round}\NormalTok{(caCrimeData}\SpecialCharTok{$}\NormalTok{ViolentCrimeRateStdScore, }\DecValTok{1}\NormalTok{)),}
    \AttributeTok{color =} \StringTok{"darkblue"}
\NormalTok{  ) }\SpecialCharTok{+}
  \FunctionTok{labs}\NormalTok{(}
    \AttributeTok{x =} \StringTok{"Standardized rate of violent crimes"}\NormalTok{,}
    \AttributeTok{caption  =} \StringTok{"Dados de crimes apresentados como escores padronizados [Standardized rate of violent crimes]}\SpecialCharTok{\textbackslash{}n}\StringTok{com média de 100 e desvio{-}padrão de 10 (acrescido o da California {-} CA junto à linha vertical azul clara).}\SpecialCharTok{\textbackslash{}n}\StringTok{Fonte: Poldrack, 2025, p. 59, fig. 5.13"}
\NormalTok{  )}
\InformationTok{\textasciigrave{}\textasciigrave{}\textasciigrave{}}
\end{Highlighting}
\end{Shaded}

\pandocbounded{\includegraphics[keepaspectratio]{cap5-pldr-modelos-dados_files/figure-pdf/unnamed-chunk-15-1.pdf}}

\section{Usando Z-scores para Comparar
Distribuições}\label{usando-z-scores-para-comparar-distribuiuxe7uxf5es}

Uma aplicação vantajosa dos \ul{\textbf{Z-scores}} é
\textbf{\emph{comparar distribuições de diferentes variáveis}}.

Suponha que queiramos \emph{comparar} a \textbf{\emph{distribuição de
crimes violentos}} e \textbf{\emph{de crimes contra o patrimônio}} entre
os \textbf{\emph{estados}}.

No painel \textbf{A} da Figura 5.15, \textbf{\emph{representamos
comparativamente ambas}}, sendo que a Califórnia é mostrada na forma de
um ponto enorme.

Como podemos observar, as \textbf{\emph{taxas brutas de crimes contra o
patrimônio são bem maiores do que as taxas brutas de crimes violentos}}.

Assim, \textbf{\emph{não podemos comparar os números diretamente}}.

No entanto, \textbf{\emph{podemos plotar os Z-scores desses dados entre
si}} (painel \textbf{B} da Figura 5.14) --- aqui, mais uma vez,
observamos que a \textbf{\emph{distribuição dos dados não muda}}.

\begin{Shaded}
\begin{Highlighting}[numbers=left,,]
\InformationTok{\textasciigrave{}\textasciigrave{}\textasciigrave{}\{r\}}
\NormalTok{p1 }\OtherTok{\textless{}{-}}\NormalTok{ crimeData }\SpecialCharTok{\%\textgreater{}\%}
  \FunctionTok{ggplot}\NormalTok{(}\FunctionTok{aes}\NormalTok{(Violent.Crime.rate, Property.crime.rate)) }\SpecialCharTok{+}
  \FunctionTok{geom\_point}\NormalTok{(}\AttributeTok{size =} \DecValTok{2}\NormalTok{) }\SpecialCharTok{+}
  \FunctionTok{annotate}\NormalTok{(}
    \StringTok{"point"}\NormalTok{,}
    \AttributeTok{x =}\NormalTok{ caCrimeData}\SpecialCharTok{$}\NormalTok{Violent.Crime.rate,}
    \AttributeTok{y =}\NormalTok{ caCrimeData}\SpecialCharTok{$}\NormalTok{Property.crime.rate,}
    \AttributeTok{color =} \StringTok{"blue"}\NormalTok{,}
    \AttributeTok{size =} \DecValTok{5}
\NormalTok{  ) }\SpecialCharTok{+}
  \FunctionTok{annotate}\NormalTok{(}
    \StringTok{"text"}\NormalTok{,}
    \AttributeTok{x =}\NormalTok{ caCrimeData}\SpecialCharTok{$}\NormalTok{Violent.Crime.rate }\SpecialCharTok{+} \DecValTok{100}\NormalTok{,}
    \AttributeTok{y =}\NormalTok{ caCrimeData}\SpecialCharTok{$}\NormalTok{Property.crime.rate }\SpecialCharTok{{-}} \DecValTok{50}\NormalTok{,}
    \AttributeTok{label =} \StringTok{"California"}\NormalTok{,}
    \AttributeTok{color =} \StringTok{"blue"}\NormalTok{,}
    \AttributeTok{size =} \DecValTok{5}
\NormalTok{  ) }\SpecialCharTok{+}
  \FunctionTok{labs}\NormalTok{(}
    \AttributeTok{x =} \StringTok{"Violent crime rate (per 100,000)"}\NormalTok{,}
    \AttributeTok{y =} \StringTok{"Property crime rate (per 100,000)"}
\NormalTok{  )}

\CommentTok{\# plot z scores}

\NormalTok{crimeData }\OtherTok{\textless{}{-}}
\NormalTok{  crimeData }\SpecialCharTok{\%\textgreater{}\%}
  \FunctionTok{mutate}\NormalTok{(}
    \AttributeTok{PropertyCrimeRateZscore =}
\NormalTok{      (Property.crime.rate }\SpecialCharTok{{-}} \FunctionTok{mean}\NormalTok{(Property.crime.rate)) }\SpecialCharTok{/}
      \FunctionTok{sd}\NormalTok{(Property.crime.rate)}
\NormalTok{  )}

\NormalTok{caCrimeData }\OtherTok{\textless{}{-}}
\NormalTok{  crimeData }\SpecialCharTok{\%\textgreater{}\%}
\NormalTok{  dplyr}\SpecialCharTok{::}\FunctionTok{filter}\NormalTok{(State }\SpecialCharTok{==} \StringTok{"California"}\NormalTok{)}


\NormalTok{p2 }\OtherTok{\textless{}{-}}\NormalTok{ crimeData }\SpecialCharTok{\%\textgreater{}\%}
  \FunctionTok{ggplot}\NormalTok{(}\FunctionTok{aes}\NormalTok{(ViolentCrimeRateZscore, PropertyCrimeRateZscore)) }\SpecialCharTok{+}
  \FunctionTok{geom\_point}\NormalTok{(}\AttributeTok{size =} \DecValTok{2}\NormalTok{) }\SpecialCharTok{+}
  \FunctionTok{scale\_y\_continuous}\NormalTok{(}\AttributeTok{breaks =} \FunctionTok{seq.int}\NormalTok{(}\SpecialCharTok{{-}}\DecValTok{2}\NormalTok{, }\DecValTok{2}\NormalTok{, .}\DecValTok{5}\NormalTok{)) }\SpecialCharTok{+}
  \FunctionTok{scale\_x\_continuous}\NormalTok{(}\AttributeTok{breaks =} \FunctionTok{seq.int}\NormalTok{(}\SpecialCharTok{{-}}\DecValTok{2}\NormalTok{, }\DecValTok{2}\NormalTok{, .}\DecValTok{5}\NormalTok{)) }\SpecialCharTok{+}
  \FunctionTok{annotate}\NormalTok{(}
    \StringTok{"point"}\NormalTok{,}
    \AttributeTok{x =}\NormalTok{ caCrimeData}\SpecialCharTok{$}\NormalTok{ViolentCrimeRateZscore,}
    \AttributeTok{y =}\NormalTok{ caCrimeData}\SpecialCharTok{$}\NormalTok{PropertyCrimeRateZscore,}
    \AttributeTok{color =} \StringTok{"blue"}\NormalTok{, }\AttributeTok{size =} \DecValTok{5}
\NormalTok{  ) }\SpecialCharTok{+}
  \FunctionTok{annotate}\NormalTok{(}
    \StringTok{"text"}\NormalTok{,}
    \AttributeTok{x =}\NormalTok{ caCrimeData}\SpecialCharTok{$}\NormalTok{ViolentCrimeRateZscore }\SpecialCharTok{+} \FloatTok{0.8}\NormalTok{,}
    \AttributeTok{y =}\NormalTok{ caCrimeData}\SpecialCharTok{$}\NormalTok{PropertyCrimeRateZscore  }\SpecialCharTok{{-}} \FloatTok{0.2}\NormalTok{,}
    \AttributeTok{label =} \StringTok{"California"}\NormalTok{,}
    \AttributeTok{color =} \StringTok{"blue"}\NormalTok{,}
    \AttributeTok{size =} \DecValTok{5}
\NormalTok{  ) }\SpecialCharTok{+}
  \FunctionTok{theme}\NormalTok{(}
    \AttributeTok{axis.title =} \FunctionTok{element\_text}\NormalTok{(}\AttributeTok{size =} \DecValTok{16}\NormalTok{)}
\NormalTok{  ) }\SpecialCharTok{+}
  \FunctionTok{labs}\NormalTok{(}
    \AttributeTok{x =} \StringTok{"z{-}scored rate of violent crimes"}\NormalTok{,}
    \AttributeTok{y =} \StringTok{"z{-}scored rate of property crimes"}
\NormalTok{  )}

\CommentTok{\# plot\_grid(p1, p2)}

\CommentTok{\# Criando a nota de rodapé como grob com tamanho de fonte personalizado}
\NormalTok{nota\_rodape }\OtherTok{\textless{}{-}} \FunctionTok{textGrob}\NormalTok{(}
  \StringTok{"(A) Gráfico de taxas de crimes contra o patrimônio [Z{-}scored rate of property crimes] e}\SpecialCharTok{\textbackslash{}n}\StringTok{(B) taxas de crimes contra o patrimônio e violentos [Z{-}scored rate of violent crimes] com Z{-}score.}\SpecialCharTok{\textbackslash{}n}\StringTok{Fonte: Poldrack, 2025, p. 60, fig. 5.14"}\NormalTok{,}
  \AttributeTok{gp =} \FunctionTok{gpar}\NormalTok{(}\AttributeTok{fontsize =} \DecValTok{8}\NormalTok{,}
            \AttributeTok{fontface =} \StringTok{"italic"}\NormalTok{),}
  \AttributeTok{x =} \FloatTok{0.5}\NormalTok{,}
  \AttributeTok{hjust =} \FloatTok{0.5}
\NormalTok{  )}

\CommentTok{\# Usando grid.arrange com a nota de rodapé customizada}
\CommentTok{\# usando grid.arrange() (do pacote gridExtra), pode usar o argumento bottom:}
\CommentTok{\# para gerar um nota de rodapé ´no gráfico final}
\FunctionTok{grid.arrange}\NormalTok{(}
\NormalTok{  p1, p2,}
  \AttributeTok{ncol =} \DecValTok{2}\NormalTok{,}
  \AttributeTok{bottom =}\NormalTok{ nota\_rodape)}
\InformationTok{\textasciigrave{}\textasciigrave{}\textasciigrave{}}
\end{Highlighting}
\end{Shaded}

\pandocbounded{\includegraphics[keepaspectratio]{cap5-pldr-modelos-dados_files/figure-pdf/unnamed-chunk-16-1.pdf}}

\ul{\textbf{Convertê-los em Z-scores para cada variável faz com que
sejam comparáveis}} e nos \emph{possibilita avaliar} que a
\textbf{\emph{Califórnia está no meio da distribuição em termos de
crimes violentos e de crimes contra o patrimônio}}.

\emph{Adicionaremos mais um fator} ao gráfico: a
\textbf{\emph{população}}.

\begin{Shaded}
\begin{Highlighting}[numbers=left,,]
\InformationTok{\textasciigrave{}\textasciigrave{}\textasciigrave{}\{r\}}
\NormalTok{p1 }\OtherTok{\textless{}{-}}\NormalTok{ crimeData }\SpecialCharTok{\%\textgreater{}\%}
  \FunctionTok{ggplot}\NormalTok{(}\FunctionTok{aes}\NormalTok{(ViolentCrimeRateZscore, PropertyCrimeRateZscore)) }\SpecialCharTok{+}
  \FunctionTok{geom\_point}\NormalTok{(}\FunctionTok{aes}\NormalTok{(}\AttributeTok{size =}\NormalTok{ Population)) }\SpecialCharTok{+}
  \FunctionTok{annotate}\NormalTok{(}
    \StringTok{"point"}\NormalTok{,}
    \AttributeTok{x =}\NormalTok{ caCrimeData}\SpecialCharTok{$}\NormalTok{ViolentCrimeRateZscore,}
    \AttributeTok{y =}\NormalTok{ caCrimeData}\SpecialCharTok{$}\NormalTok{PropertyCrimeRateZscore,}
    \AttributeTok{color =} \StringTok{"blue"}\NormalTok{,}
    \AttributeTok{size =} \DecValTok{5}
\NormalTok{  ) }\SpecialCharTok{+}
  \FunctionTok{labs}\NormalTok{(}
    \AttributeTok{x =} \StringTok{"z{-}scored rate of violent crimes"}\NormalTok{,}
    \AttributeTok{y =} \StringTok{"z{-}scored rate of property crimes"}
\NormalTok{  ) }\SpecialCharTok{+}
  \FunctionTok{theme}\NormalTok{(}\AttributeTok{legend.position =} \FunctionTok{c}\NormalTok{(}\FloatTok{0.2}\NormalTok{,}\FloatTok{0.8}\NormalTok{))}

\NormalTok{crimeData }\OtherTok{\textless{}{-}}\NormalTok{ crimeData }\SpecialCharTok{\%\textgreater{}\%}
  \FunctionTok{mutate}\NormalTok{(}
    \AttributeTok{ViolenceDiff =}\NormalTok{ ViolentCrimeRateZscore }\SpecialCharTok{{-}}\NormalTok{ PropertyCrimeRateZscore}
\NormalTok{  )}

\NormalTok{p2 }\OtherTok{\textless{}{-}}\NormalTok{ crimeData }\SpecialCharTok{\%\textgreater{}\%}
  \FunctionTok{ggplot}\NormalTok{(}\FunctionTok{aes}\NormalTok{(Population, ViolenceDiff)) }\SpecialCharTok{+}
  \FunctionTok{geom\_point}\NormalTok{() }\SpecialCharTok{+}
  \FunctionTok{ylab}\NormalTok{(}\StringTok{"Violence difference"}\NormalTok{)}

\CommentTok{\# plot\_grid(p1, p2)}

\CommentTok{\# Criando a nota de rodapé como grob com tamanho de fonte personalizado}
\NormalTok{nota\_rodape }\OtherTok{\textless{}{-}} \FunctionTok{textGrob}\NormalTok{(}
  \StringTok{"(A) Gráfico de taxas de crimes violentos [Z{-}scored rate of violent crimes] comparadas a taxas de crimes contra o patrimônio}\SpecialCharTok{\textbackslash{}n}\StringTok{[Z{-}scored rate of property crimes], com o tamanho da população retratado pelo tamanho do símbolo gráfico.}\SpecialCharTok{\textbackslash{}n}\StringTok{(B) Escores de diferença [Violence difference] para crimes violentos e crimes contra o patrimônio, plotados em relação à população [Population].}\SpecialCharTok{\textbackslash{}n}\StringTok{Fonte: Poldrack, 2025, p. 60, fig. 5.15"}\NormalTok{,}
  \AttributeTok{gp =} \FunctionTok{gpar}\NormalTok{(}\AttributeTok{fontsize =} \DecValTok{8}\NormalTok{,}
            \AttributeTok{fontface =} \StringTok{"italic"}\NormalTok{),}
  \AttributeTok{x =} \FloatTok{0.5}\NormalTok{,}
  \AttributeTok{hjust =} \FloatTok{0.5}
\NormalTok{  )}

\CommentTok{\# Usando grid.arrange com a nota de rodapé customizada}
\CommentTok{\# usando grid.arrange() (do pacote gridExtra), pode usar o argumento bottom:}
\CommentTok{\# para gerar um nota de rodapé ´no gráfico final}
\FunctionTok{grid.arrange}\NormalTok{(}
\NormalTok{  p1, p2,}
  \AttributeTok{ncol =} \DecValTok{2}\NormalTok{,}
  \AttributeTok{bottom =}\NormalTok{ nota\_rodape)}
\InformationTok{\textasciigrave{}\textasciigrave{}\textasciigrave{}}
\end{Highlighting}
\end{Shaded}

\pandocbounded{\includegraphics[keepaspectratio]{cap5-pldr-modelos-dados_files/figure-pdf/unnamed-chunk-17-1.pdf}}

No painel \textbf{A} da Figura 5.15, demonstramos isso com o tamanho do
símbolo gráfico, que geralmente é uma forma útil de adicionar
informações a um gráfico.

Como os \ul{\textbf{Z-scores são diretamente comparáveis}}, também
podemos \textbf{\emph{calcular um escore de diferença}} que expresse a
\textbf{\emph{taxa relativa de crimes violentos e crimes não violentos}}
(contra o patrimônio) entre os \textbf{\emph{estados}}.

Em seguida, podemos \textbf{\emph{plotar esses escores em relação à
população}} (painel \textbf{B} da Figura 5.15).

Isso \ul{\textbf{mostra como podemos usar Z-scores a fim de relacionar
diferentes variáveis em uma escala comum}}.

Vale ressaltar que os \textbf{\emph{estados menores parecem ter as
maiores diferenças em ambas as direções}}.

\emph{Mesmo que seja tentador observar cada estado e tentar determinar
por que eles têm um escore de diferença alto ou baixo}, isso
\textbf{\emph{provavelmente retrata o fato de que as estimativas obtidas
a partir de amostras menores serão necessariamente mais variáveis}},
conforme analisaremos no Capítulo 7.

\section{Problemas}\label{problemas}

\begin{enumerate}
\def\labelenumi{\arabic{enumi}.}
\tightlist
\item
  Descreva as três partes do modelo básico de estatística e como elas se
  relacionam.
\item
  Um pesquisador quer criar um modelo para predizer a altura, usando uma
  amostra com 8 pessoas com as seguintes alturas (em centímetros): 170;
  176; 168; 188; 178; 168; 179; 181.
\end{enumerate}

\begin{itemize}
\tightlist
\item
  ▶Determine a moda desses dados.
\item
  ▶Calcule o erro da moda para cada pessoa e depois calcule a média
  desses erros.
\item
  ▶Calcule a média dos dados e, em seguida, calcule o erro médio a
  partir da média.
\end{itemize}

\begin{enumerate}
\def\labelenumi{\arabic{enumi}.}
\setcounter{enumi}{2}
\tightlist
\item
  Descreva as duas possíveis fontes de erro ao comparar as predições de
  um modelo com os dados.
\item
  Descreva o conceito de sobreajuste e como saber se o sobreajuste
  ocorreu.
\item
  Se estimarmos a média e a mediana de um conjunto de dados e, em
  seguida, calcularmos a soma dos erros quadráticos para cada uma dessas
  estimativas em comparação aos dados, qual das duas é necessariamente
  menor ou igual à outra?
\item
  Qual é a razão pela qual alguém pode querer usar a mediana em vez da
  média para descrever um conjunto específico de dados?
\item
  Calcule a mediana dos dados descritos na questão 2.
\item
  O que significa quando um símbolo estatístico tem um ``chapéu'' (como
  )?
\item
  Qual é a diferença entre a forma como o desvio-padrão é calculado para
  uma população e para uma amostra, e qual é o conceito fundamental de
  estatística relacionado a essa diferença?
\item
  Calcule o desvio-padrão para os dados da amostra descritos na questão
  2.
\item
  Calcule os Z-scores para cada uma das pessoas descritas na questão 2.
\item
  Qual das afirmações a seguir é verdadeira em relação à média? Escolha
  todas as opções que se aplicam.
\end{enumerate}

\begin{itemize}
\tightlist
\item
  ▶A soma dos erros de cada amostra com a média (amostral) é 0.
\item
  ▶Ela minimiza a soma dos erros quadráticos.
\item
  ▶Ela não é sensível a outliers.
\item
  ▶Ela retrata o quinquagésimo percentil nos dados.
\end{itemize}

\begin{enumerate}
\def\labelenumi{\arabic{enumi}.}
\setcounter{enumi}{12}
\tightlist
\item
  O modelo que melhor se ajusta a um determinado conjunto de dados (ou
  seja, aquele com a menor soma de erros quadráticos) geralmente também
  é o modelo que melhor se ajusta a um conjunto novo de dados.
  Verdadeiro ou falso?
\item
  Qual desses conceitos é mais diretamente relevante para a pergunta
  anterior?
\end{enumerate}

\begin{itemize}
\tightlist
\item
  ▶Sobreajuste
\item
  ▶Graus de liberdade
\item
  ▶Variabilidade
\item
  ▶Escores padronizados
\end{itemize}

\section{Coeficiente de Correlação - cap. 10
pldr}\label{coeficiente-de-correlauxe7uxe3o---cap.-10-pldr}

O \ul{\textbf{coeficiente de correlação}} (\textbf{r}) é
\ul{\textbf{\emph{uma medida}}} da \ul{\textbf{\emph{força}}} da
\ul{\textbf{\emph{relação linear}}} \ul{\textbf{\emph{entre duas
variáveis contínuas}}}.

No Capítulo 13, analisaremos a correlação mais detalhadamente; por ora,
\emph{simplesmente apresentaremos} \textbf{\emph{r}} como
\textbf{\emph{uma forma de quantificar a relação entre duas variáveis}}.

Trata-se de \emph{uma} \textbf{\emph{medida}} que varia de \textbf{− 1 a
1}, em que um \textbf{\emph{valor}} de \textbf{\emph{1 representa uma
relação positiva perfeita}} \emph{entre as variáveis},
\ul{\textbf{\emph{0 representa nenhuma relação}}}, e \textbf{\emph{− 1
representa uma relação negativa perfeita}}.

A Figura 10.4 ilustra exemplos de diversos níveis de correlação usando
dados gerados aleatoriamente. (Poldrack, 2025 , cap. 10 - , p.~123-124)

Carregar pacotes e o conjunto de dados \texttt{NHANES}. Selecionar o
subset apenas de adultos (\texttt{age} \textgreater= 18 anos).

\begin{Shaded}
\begin{Highlighting}[numbers=left,,]
\InformationTok{\textasciigrave{}\textasciigrave{}\textasciigrave{}\{r\}}
\FunctionTok{library}\NormalTok{(tidyverse)}
\FunctionTok{library}\NormalTok{(ggplot2)}
\FunctionTok{library}\NormalTok{(cowplot)}
\FunctionTok{library}\NormalTok{(boot)}
\FunctionTok{library}\NormalTok{(MASS)}
\FunctionTok{library}\NormalTok{(pwr)}
\FunctionTok{set.seed}\NormalTok{(}\DecValTok{123456}\NormalTok{) }\CommentTok{\# set random seed to exactly replicate results}
\FunctionTok{theme\_set}\NormalTok{(}\FunctionTok{theme\_minimal}\NormalTok{(}\AttributeTok{base\_size =} \DecValTok{14}\NormalTok{))}

\FunctionTok{library}\NormalTok{(knitr)}

\CommentTok{\# load the NHANES data library}
\FunctionTok{library}\NormalTok{(NHANES)}

\CommentTok{\# drop duplicated IDs within the NHANES dataset}
\NormalTok{NHANES }\OtherTok{\textless{}{-}}
\NormalTok{  NHANES }\SpecialCharTok{\%\textgreater{}\%}
\NormalTok{  dplyr}\SpecialCharTok{::}\FunctionTok{distinct}\NormalTok{(ID,}\AttributeTok{.keep\_all=}\ConstantTok{TRUE}\NormalTok{)}

\NormalTok{NHANES\_adult }\OtherTok{\textless{}{-}} \CommentTok{\# Selecionar o subset apenas de adultos (age \textgreater{}= 18 anos)}
\NormalTok{  NHANES }\SpecialCharTok{\%\textgreater{}\%}
  \FunctionTok{drop\_na}\NormalTok{(Weight) }\SpecialCharTok{\%\textgreater{}\%} \CommentTok{\# descartar as observações com NA na variável Weight (peso).}
  \FunctionTok{subset}\NormalTok{(Age }\SpecialCharTok{\textgreater{}=} \DecValTok{18}\NormalTok{)}
\InformationTok{\textasciigrave{}\textasciigrave{}\textasciigrave{}}
\end{Highlighting}
\end{Shaded}

Gerar um painel com cinco \textbf{\emph{gráficos de dispersão}} para
ilustrar a relação entre a \textbf{\emph{forma}} desses gráficos e o
\textbf{\emph{valor}} do \ul{\textbf{coeficiente de correlação}} de
Pierson, que mede a \ul{\textbf{força}} da \ul{\textbf{relação linear}}
entre \ul{\textbf{duas variáveis quantitativas}}: a variável resposta
(Y) e a variável explicativa (X).

\begin{Shaded}
\begin{Highlighting}[numbers=left,,]
\InformationTok{\textasciigrave{}\textasciigrave{}\textasciigrave{}\{r\}}
\FunctionTok{set.seed}\NormalTok{(}\DecValTok{123456789}\NormalTok{)}
\NormalTok{p }\OtherTok{\textless{}{-}} \FunctionTok{list}\NormalTok{()}
\NormalTok{corrvals }\OtherTok{\textless{}{-}} \FunctionTok{c}\NormalTok{(}\DecValTok{1}\NormalTok{, }\FloatTok{0.5}\NormalTok{, }\DecValTok{0}\NormalTok{, }\SpecialCharTok{{-}}\FloatTok{0.5}\NormalTok{, }\SpecialCharTok{{-}}\DecValTok{1}\NormalTok{) }\CommentTok{\# um vetor com uma gama típica de valores de correlação}

\ControlFlowTok{for}\NormalTok{ (i }\ControlFlowTok{in} \DecValTok{1}\SpecialCharTok{:}\FunctionTok{length}\NormalTok{(corrvals))\{}
\NormalTok{  simdata }\OtherTok{\textless{}{-}} \FunctionTok{data.frame}\NormalTok{(}\FunctionTok{mvrnorm}\NormalTok{(}\AttributeTok{n  =} \DecValTok{50}\NormalTok{, }\CommentTok{\# coleta uma AAS de tamanho n=50}
                                \AttributeTok{mu =} \FunctionTok{c}\NormalTok{(}\DecValTok{0}\NormalTok{, }\DecValTok{0}\NormalTok{), }\CommentTok{\# um vetor com duas médias padronizadas}
                                \AttributeTok{Sigma =} \FunctionTok{matrix}\NormalTok{(}\FunctionTok{c}\NormalTok{(          }\DecValTok{1}\NormalTok{, corrvals[i],}
\NormalTok{                                                 corrvals[i], }\DecValTok{1}\NormalTok{)}
\NormalTok{                                               ,}\DecValTok{2}\NormalTok{ ,}\DecValTok{2}\NormalTok{)) }\CommentTok{\# gerar uma matriz de covariâncias 2x2}
\NormalTok{                )}
\NormalTok{  tmp }\OtherTok{\textless{}{-}} \FunctionTok{ggplot}\NormalTok{(simdata, }\FunctionTok{aes}\NormalTok{(X1,X2)) }\SpecialCharTok{+}
    \FunctionTok{geom\_point}\NormalTok{(}\AttributeTok{size=}\FloatTok{0.5}\NormalTok{) }\SpecialCharTok{+}
    \FunctionTok{ggtitle}\NormalTok{(}\FunctionTok{sprintf}\NormalTok{(}\StringTok{\textquotesingle{}r = \%.02f\textquotesingle{}}\NormalTok{, }\FunctionTok{cor}\NormalTok{(simdata)[}\DecValTok{1}\NormalTok{,}\DecValTok{2}\NormalTok{]))}
\NormalTok{  p[[i]] }\OtherTok{=}\NormalTok{ tmp}
\NormalTok{\}}
\CommentTok{\# plot\_grid(p[[1]],p[[2]],p[[3]],p[[4]],p[[5]])}

\CommentTok{\# Criando a nota de rodapé como textGrob com tamanho de fonte personalizado}
\NormalTok{nota\_rodape }\OtherTok{\textless{}{-}} \FunctionTok{textGrob}\NormalTok{(}
  \StringTok{"Exemplos de diversos níveis do coeficiente de correlação.}\SpecialCharTok{\textbackslash{}n}\StringTok{Fonte: Poldrack, 2025, p. 124, fig. 10.4"}\NormalTok{,}
  \AttributeTok{gp =} \FunctionTok{gpar}\NormalTok{(}\AttributeTok{fontsize =} \DecValTok{8}\NormalTok{,}
            \AttributeTok{fontface =} \StringTok{"italic"}\NormalTok{),}
  \AttributeTok{x =} \FloatTok{0.5}\NormalTok{,}
  \AttributeTok{hjust =} \FloatTok{0.5}
\NormalTok{  )}

\CommentTok{\# Usando grid.arrange com a nota de rodapé customizada}
\CommentTok{\# usando grid.arrange() (do pacote gridExtra), pode usar o argumento bottom:}
\CommentTok{\# para gerar um nota de rodapé ´no gráfico final}
\FunctionTok{grid.arrange}\NormalTok{(}
\NormalTok{  p[[}\DecValTok{1}\NormalTok{]],p[[}\DecValTok{2}\NormalTok{]],p[[}\DecValTok{3}\NormalTok{]],p[[}\DecValTok{4}\NormalTok{]],p[[}\DecValTok{5}\NormalTok{]],}
  \AttributeTok{ncol =} \DecValTok{3}\NormalTok{,}
  \AttributeTok{bottom =}\NormalTok{ nota\_rodape)}
\InformationTok{\textasciigrave{}\textasciigrave{}\textasciigrave{}}
\end{Highlighting}
\end{Shaded}

\pandocbounded{\includegraphics[keepaspectratio]{cap5-pldr-modelos-dados_files/figure-pdf/unnamed-chunk-19-1.pdf}}

\section{Correlação: micro e pequena
empresas}\label{correlauxe7uxe3o-micro-e-pequena-empresas}

Carregar pacotes e set up.

\begin{Shaded}
\begin{Highlighting}[numbers=left,,]
\InformationTok{\textasciigrave{}\textasciigrave{}\textasciigrave{}\{r\}}
\CommentTok{\# import MASS first because it otherwise will mask dplyr::select}
\FunctionTok{library}\NormalTok{(MASS)}

\FunctionTok{library}\NormalTok{(tidyverse)}
\FunctionTok{library}\NormalTok{(ggdendro)}
\FunctionTok{library}\NormalTok{(psych)}
\FunctionTok{library}\NormalTok{(gplots)}
\FunctionTok{library}\NormalTok{(pdist)}
\FunctionTok{library}\NormalTok{(factoextra)}
\FunctionTok{library}\NormalTok{(viridis)}
\FunctionTok{library}\NormalTok{(mclust)}
\FunctionTok{library}\NormalTok{(knitr)}

\FunctionTok{theme\_set}\NormalTok{(}\FunctionTok{theme\_minimal}\NormalTok{())}
\InformationTok{\textasciigrave{}\textasciigrave{}\textasciigrave{}}
\end{Highlighting}
\end{Shaded}

Carregar dados primários e secundários coletados por (BARZELLAY; DAS
NEVES, 2022).

\subsection{Importar}\label{importar-1}

\begin{Shaded}
\begin{Highlighting}[numbers=left,,]
\InformationTok{\textasciigrave{}\textasciigrave{}\textasciigrave{}\{r\}}
\CommentTok{\# Importar como tibble o arquivo de dentro da pasta chamada: dat/csv.}
\NormalTok{mpe }\OtherTok{\textless{}{-}}\NormalTok{ readr}\SpecialCharTok{::}\FunctionTok{read\_csv}\NormalTok{(}\AttributeTok{file   =} \StringTok{"dat/csv/MPE{-}GO\_DP\_PIB\_Caged\_Rais.csv"}\NormalTok{,}
                       \CommentTok{\# delim  = ",",}
                       \AttributeTok{quote  =} \StringTok{"}\SpecialCharTok{\textbackslash{}"}\StringTok{"}\NormalTok{,}
                       \AttributeTok{locale =} \FunctionTok{locale}\NormalTok{(}
                         \AttributeTok{decimal\_mark =} \StringTok{"."}\NormalTok{,}
                         \AttributeTok{encoding     =} \StringTok{"UTF{-}8"}
\NormalTok{                         )}
\NormalTok{                       )}

\CommentTok{\# cat {-} Concatenate And Print}
\FunctionTok{cat}\NormalTok{(}\StringTok{"}\SpecialCharTok{\textbackslash{}n}\StringTok{"}\NormalTok{) }\CommentTok{\# imprime no console (saída) uma linha em branco}
\FunctionTok{cat}\NormalTok{(}\StringTok{"Estrutura do objeto R denominado mpe:}\SpecialCharTok{\textbackslash{}n}\StringTok{"}\NormalTok{)}
\FunctionTok{str}\NormalTok{(mpe)}

\FunctionTok{cat}\NormalTok{(}\StringTok{"}\SpecialCharTok{\textbackslash{}n}\StringTok{"}\NormalTok{)}
\FunctionTok{cat}\NormalTok{(}\StringTok{"Nomes das 8 colunas do objeto mpe:}\SpecialCharTok{\textbackslash{}n}\StringTok{"}\NormalTok{)}
\FunctionTok{names}\NormalTok{(mpe)}
\CommentTok{\# [1] "ano"  "PgtoGO\_MPE"  "PIB"  "CAGED"  "RAIS"  "Pop"  "DPGO\_MPE\_pc"  "PIB\_pc"}
\CommentTok{\# ano: vai de 2006 até 2019 (14 linhas de observações para as 8 colunas de variáveis)}

\NormalTok{mpe }\CommentTok{\# tibble: 14 × 8}
\InformationTok{\textasciigrave{}\textasciigrave{}\textasciigrave{}}
\end{Highlighting}
\end{Shaded}

\begin{verbatim}

Estrutura do objeto R denominado mpe:
spc_tbl_ [14 x 8] (S3: spec_tbl_df/tbl_df/tbl/data.frame)
 $ ano        : num [1:14] 2006 2007 2008 2009 2010 ...
 $ PgtoGO_MPE : num [1:14] 1.65e+08 1.76e+08 2.05e+08 3.30e+08 4.53e+08 ...
 $ PIB        : num [1:14] 6.14e+07 7.14e+07 8.24e+07 9.29e+07 1.07e+08 ...
 $ CAGED      : num [1:14] 21061 41153 47347 34404 83975 ...
 $ RAIS       : num [1:14] 992822 1061426 1135046 1209310 1313641 ...
 $ Pop        : num [1:14] 5730762 5840650 5844996 5926308 6003788 ...
 $ DPGO_MPE_pc: num [1:14] 28.9 30.1 35.1 55.6 75.5 ...
 $ PIB_pc     : num [1:14] 10710 12226 14101 15670 17784 ...
 - attr(*, "spec")=
  .. cols(
  ..   ano = col_double(),
  ..   PgtoGO_MPE = col_double(),
  ..   PIB = col_double(),
  ..   CAGED = col_double(),
  ..   RAIS = col_double(),
  ..   Pop = col_double(),
  ..   DPGO_MPE_pc = col_double(),
  ..   PIB_pc = col_double()
  .. )
 - attr(*, "problems")=<externalptr> 

Nomes das 8 colunas do objeto mpe:
[1] "ano"         "PgtoGO_MPE"  "PIB"         "CAGED"       "RAIS"       
[6] "Pop"         "DPGO_MPE_pc" "PIB_pc"     
# A tibble: 14 x 8
     ano PgtoGO_MPE       PIB  CAGED    RAIS     Pop DPGO_MPE_pc PIB_pc
   <dbl>      <dbl>     <dbl>  <dbl>   <dbl>   <dbl>       <dbl>  <dbl>
 1  2006 165475952.  61375409  21061  992822 5730762        28.9 10710.
 2  2007 175555748.  71410569  41153 1061426 5840650        30.1 12226.
 3  2008 205135955.  82417571  47347 1135046 5844996        35.1 14101.
 4  2009 329706125.  92865740  34404 1209310 5926308        55.6 15670.
 5  2010 453279122. 106770107  83975 1313641 6003788        75.5 17784.
 6  2011 333961964. 121296722  69552 1385230 6080716        54.9 19948.
 7  2012 472699129. 138757833  66230 1439341 6154996        76.8 22544.
 8  2013 640670509. 151300180  60831 1509395 6434048        99.6 23516.
 9  2014 562863907. 165015307  25333 1514532 6523222        86.3 25297.
10  2015 370979808. 173632448 -24551 1501397 6610681        56.1 26265.
11  2016 393978072. 181692438 -19354 1445943 6695855        58.8 27135.
12  2017 490777617. 191898682  25370 1515422 6778772        72.4 28309.
13  2018 422735929. 195682000  17293 1507648 6921161        61.1 28273 
14  2019 241392288. 208672000  21550 1524304 6939629        34.8 30070.
\end{verbatim}

\subsection{Dicionário de Dados}\label{dicionuxe1rio-de-dados}

O significado das 8 variáveis coletadas neste Estudo Observacional de
267 do TCE-GO: o Poder das compras públicas pelo Estado de Goiás como
instrumento de Política Pública de fomento às MPE's - Micro e Pequenas
Empresas (art. 179, CF/1988; arts. 44 e 45,
\href{https://www.planalto.gov.br/ccivil_03/leis/lcp/lcp123.htm}{LC
n.~123/2006} - Estatuto Nacional da Microempresa e da Empresa de Pequeno
Porte).\footnote{Art. 44. Nas licitações será assegurada, como critério
  de desempate, \textbf{\emph{preferência de contratação para as
  microempresas e empresas de pequeno porte}}. (Vide Lei nº 14.133, de
  2021)

  § 1º Entende-se por \textbf{\emph{empate}} aquelas situações em que as
  \textbf{\emph{propostas apresentadas pelas microempresas e empresas de
  pequeno porte sejam iguais ou até 10\% (dez por cento) superiores à
  proposta mais bem classificada}}.

  § 2º Na modalidade de \textbf{\emph{pregão}}, o intervalo percentual
  estabelecido no § 1º deste artigo será de \textbf{\emph{até 5\% (cinco
  por cento) superior ao melhor preço}}.

  Art. 45. Para efeito do disposto no art. 44 desta Lei Complementar,
  ocorrendo o \textbf{\emph{empate}}, proceder-se-á da seguinte forma:
  (Vide Lei nº 14.133, de 2021

  I - a \textbf{\emph{microempresa ou empresa de pequeno porte mais bem
  classificada poderá apresentar proposta de preço inferior àquela
  considerada vencedora do certame}}, situação em que será
  \textbf{\emph{adjudicado em seu favor o objeto licitado}};

  II - \textbf{\emph{não}} ocorrendo a \textbf{\emph{contratação da
  microempresa ou empresa de pequeno porte}}, na forma do
  \textbf{\emph{inciso I}} do caput deste artigo, serão
  \textbf{\emph{convocadas as remanescentes que porventura se enquadrem
  na hipótese dos §§ 1º e 2º do art. 44}} desta Lei Complementar, na
  \textbf{\emph{ordem classificatória}}, para o exercício do
  \textbf{\emph{mesmo direito}};

  III - no caso de \textbf{\emph{equivalência dos valores}} apresentados
  pelas microempresas e empresas de pequeno porte que se encontrem nos
  intervalos estabelecidos nos §§ 1º e 2º do art. 44 desta Lei
  Complementar, será \textbf{\emph{realizado sorteio}} entre elas para
  que \textbf{\emph{se identifique aquela que primeiro poderá apresentar
  melhor oferta}}.

  § 1º Na hipótese da \textbf{\emph{não-contratação}} nos termos
  previstos no caput deste artigo, o \textbf{\emph{objeto licitado será
  adjudicado}} em \textbf{\emph{favor da proposta originalmente
  vencedora}} do certame.

  § 2º O disposto neste artigo \textbf{\emph{somente se aplicará quando
  a melhor oferta inicial não tiver sido apresentada por microempresa ou
  empresa de pequeno porte}}.

  § 3º No caso de \textbf{\emph{pregão}}, a \emph{microempresa ou
  empresa de pequeno porte mais bem classificada será convocada} para
  apresentar \textbf{\emph{nova proposta no prazo máximo de 5 (cinco)
  minutos após o encerramento dos lances}}, sob \textbf{\emph{pena de
  preclusão}}.}

\begin{enumerate}
\def\labelenumi{\arabic{enumi}.}
\item
  ``\texttt{ano}'' - vai de 2006 até 2019 (são 14 linhas de observações
  para as 8 colunas de variáveis)
\item
  ``\texttt{PgtoGO\_MPE}'' - Despesas anuais do Estado de Goiás com MPE
  - Micro e Pequenas Empresas (BARZELLAY; DAS NEVES, 2022 , p.~144,
  tabela 12), obtido do Sistema SIOFNet - Sistema de Elabroação e
  Execução Orcamentária e Financeira do Estado de Goiás.
\item
  ``\texttt{PIB}'' - Produto Interno Bruto do Esatado de Goiás, obtido
  junto ao IMB - Instituto Mauro Borges;
\item
  ``\texttt{CAGED}'' - Cadastro Geral de Empregados e Desempregados,
  obtido junto ao IMB - Instituto Mauro Borges, que fornece o
  \textbf{\emph{saldo anual}} de empregos, ou seja, Contratações -
  Demissões (BARZELLAY; DAS NEVES, 2022 , p.~155);
\item
  ``\texttt{RAIS}'' - Relação Anual de Informações Sociais, obtido junto
  ao IMB - Instituto Mauro Borges, que fornece o \textbf{\emph{número
  total}} de \textbf{\emph{vínculos empregatícios}} ano a ano
  (BARZELLAY; DAS NEVES, 2022 , p.~154-155);
\item
  ``\texttt{Pop}'' - População do Estado de Goiás, {[}obtido junto ao
  IMB - Instituto Mauro Borges{]};
\item
  ``\texttt{DPGO\_MPE\_pc}'' - Despesas anuais \emph{per capta} do
  Estado de Goiás com MPE - Micro e Pequenas Empresas, calculada, ano a
  ano, de 2006 a 2019, pela seguinte fórmula:

  \[
  DPGO\_MPE\_pc = \frac{PgtoGO\_MPE}{Pop}
  \]
\item
  ``\texttt{PIB\_pc}'' - Produto Interno Bruto \emph{per capta} do
  Esatado de Goiás, calculado, ano a ano, de 2006 a 2019, pela seguinte
  fórmula:

  \[
  PIB\_pc = \frac{PIB}{Pop}
  \]
\end{enumerate}

\subsection{Contexto dos dados}\label{contexto-dos-dados}

A figura a seguir ilustra o contexto em que se deve interpretar os dados
sobre CAGED e a MPE's no Brasil (BARZELLAY; DAS NEVES, 2022 , p.~100).

\begin{figure}[H]

{\centering \pandocbounded{\includegraphics[keepaspectratio]{fig/CAGED-2006-2019-MPE-x-demEmpresas-MinEcon.png}}

}

\caption{CAGED (2006 a 2019) por porte das empresas: MPE's versus demais
Empresas. Fonte: Min. Economia}

\end{figure}%

Observe-se que, mesmo em príodo de \textbf{\emph{crise}} de
\emph{empregos}, 2015 a 2019, as MPE's - Micro e Pequenas Empresas são
\emph{menos afetadas} e \emph{tendem} a sofrer \textbf{\emph{quedas não
tão acentuadas}} e mesmo \textbf{\emph{recuperar-se mais rapidamente}}
que as empresas dos demais portes (médias e grandes).

Outro aspceto relevante é a participação das MPEs no PIB Brasil
(1985-2017), conforme estudo realizado pela FGV e SEBRAE.

\begin{figure}[H]

{\centering \pandocbounded{\includegraphics[keepaspectratio]{fig/MPE-participacao-PIB-Br-FGV-SEBRAE.png}}

}

\caption{Participação das MPEs no PIB-Br (1985 a 2019). Fonte: FGV e
SEBRAE}

\end{figure}%

Percebe-se pouca dispersão em torno da reta tracejada (provável reta de
regressão), que representa uma \textbf{\emph{correlação positiva}} entre
a \textbf{proporção} \textbf{(\%) da participação} das MPE's no PIB-Br
ao longo do período observados, de 1985 a 2017, de forma
\textbf{\emph{consistente}} ao longos desses \textbf{32 anos}.

Agora vamos olhar para os \textbf{\emph{Valores}}, em reais (R\$), da
\textbf{\emph{participação das MPEs beneficiárias de contratos nas
compras públicas}} (licitações) de \textbf{\emph{entes federais}}, de
2016 a 2020 (BARZELLAY; DAS NEVES, 2022 , p.~107, gráfico 2).

\pandocbounded{\includegraphics[keepaspectratio]{images/clipboard-3769383876.png}}

E verificar como a \textbf{\emph{proporção (\%) dessa participação
evoluiu}} nesse mesmo período de \textbf{\emph{tempo}}, (BARZELLAY; DAS
NEVES, 2022 , p.~108, gráfico 4).

\pandocbounded{\includegraphics[keepaspectratio]{images/clipboard-1656164584.png}}

\subsection{Explorar}\label{explorar}

Ver um \textbf{\emph{resumo}} das \textbf{\emph{possíveis relações entre
cada par dessas 8 variáveis quantitativas}}.

\begin{Shaded}
\begin{Highlighting}[numbers=left,,]
\InformationTok{\textasciigrave{}\textasciigrave{}\textasciigrave{}\{r\}}
\FunctionTok{pairs.panels}\NormalTok{(mpe, }\AttributeTok{lm=}\ConstantTok{TRUE}\NormalTok{)}
\InformationTok{\textasciigrave{}\textasciigrave{}\textasciigrave{}}
\end{Highlighting}
\end{Shaded}

\pandocbounded{\includegraphics[keepaspectratio]{cap5-pldr-modelos-dados_files/figure-pdf/unnamed-chunk-22-1.pdf}}

A matriz de gr{[}aficos acima fornece uma visão geral rápida das
relações entre as variáveis e identificar quais são mais importantes.

Exibir o \ul{\textbf{Nível de Significância}} dos
\textbf{\emph{coeficientes de correlação}} acima através de um
\ul{\textbf{correlograma}} com \emph{mapa de calor} (de cores) ou
\textbf{\emph{hit-map}} para corroborar essas evidências iniciais.

\begin{Shaded}
\begin{Highlighting}[numbers=left,,]
\InformationTok{\textasciigrave{}\textasciigrave{}\textasciigrave{}\{r\}}
\FunctionTok{library}\NormalTok{(psych)}
\FunctionTok{library}\NormalTok{(Hmisc)}
\FunctionTok{library}\NormalTok{(corrplot)}

\CommentTok{\# Calculando correlações e p{-}valores}
\NormalTok{res }\OtherTok{\textless{}{-}} \FunctionTok{rcorr}\NormalTok{(}\FunctionTok{as.matrix}\NormalTok{(mpe))}

\FunctionTok{library}\NormalTok{(Hmisc)}
\FunctionTok{library}\NormalTok{(corrplot)}

\CommentTok{\# Calcular a matriz de correlação e os p{-}valores}
\NormalTok{res }\OtherTok{\textless{}{-}} \FunctionTok{rcorr}\NormalTok{(}\FunctionTok{as.matrix}\NormalTok{(mpe)) }\CommentTok{\# retorna lista com r (correlações) e P (p{-}valores)}

\CommentTok{\# Gera o corrplot com personalização}
\FunctionTok{corrplot}\NormalTok{(}
\NormalTok{  res}\SpecialCharTok{$}\NormalTok{r,                        }\CommentTok{\# matriz de correlação}
  \AttributeTok{method =} \StringTok{"color"}\NormalTok{,             }\CommentTok{\# método de visualização (pode ser "number", "circle", etc.)}
  \AttributeTok{type =} \StringTok{"upper"}\NormalTok{,               }\CommentTok{\# mostra apenas a metade superior}
  \AttributeTok{order =} \StringTok{"hclust"}\NormalTok{,             }\CommentTok{\# ordena as variáveis por similaridade}
  \AttributeTok{addCoef.col =} \StringTok{"black"}\NormalTok{,        }\CommentTok{\# adiciona os valores das correlações}
  \AttributeTok{tl.col =} \StringTok{"black"}\NormalTok{,             }\CommentTok{\# cor dos nomes das variáveis}
  \AttributeTok{tl.srt =} \DecValTok{45}\NormalTok{,                  }\CommentTok{\# rotação dos nomes das variáveis}
  \AttributeTok{tl.cex =} \FloatTok{0.8}\NormalTok{,                 }\CommentTok{\# tamanho dos nomes das variáveis}
  \AttributeTok{col =} \FunctionTok{colorRampPalette}\NormalTok{(}\FunctionTok{c}\NormalTok{(}\StringTok{"red"}\NormalTok{, }\StringTok{"white"}\NormalTok{, }\StringTok{"blue"}\NormalTok{))(}\DecValTok{200}\NormalTok{), }\CommentTok{\# gradiente de cores}
  \AttributeTok{number.cex =} \FloatTok{0.7}\NormalTok{,             }\CommentTok{\# tamanho dos números e asteriscos}
  \AttributeTok{mar =} \FunctionTok{c}\NormalTok{(}\DecValTok{0}\NormalTok{,}\DecValTok{0}\NormalTok{,}\DecValTok{1}\NormalTok{,}\DecValTok{0}\NormalTok{)              }\CommentTok{\# margens do gráfico}
\NormalTok{)}

\CommentTok{\# Adiciona um título ao gráfico}
\CommentTok{\# title("Mapa de Correlação {-} MPE\textquotesingle{}s Goiás", line = 0.5, cex.main = 1.5)}

\CommentTok{\# Adiciona asteriscos manualmente}
\NormalTok{n }\OtherTok{\textless{}{-}} \FunctionTok{ncol}\NormalTok{(res}\SpecialCharTok{$}\NormalTok{r)}
\ControlFlowTok{for}\NormalTok{(i }\ControlFlowTok{in} \DecValTok{1}\SpecialCharTok{:}\NormalTok{(n}\DecValTok{{-}1}\NormalTok{)) \{}
  \ControlFlowTok{for}\NormalTok{(j }\ControlFlowTok{in}\NormalTok{ (i}\SpecialCharTok{+}\DecValTok{1}\NormalTok{)}\SpecialCharTok{:}\NormalTok{n) \{}
\NormalTok{    pval }\OtherTok{\textless{}{-}}\NormalTok{ res}\SpecialCharTok{$}\NormalTok{P[i, j]}
    \ControlFlowTok{if}\NormalTok{(}\SpecialCharTok{!}\FunctionTok{is.na}\NormalTok{(pval)) \{}
      \ControlFlowTok{if}\NormalTok{(pval }\SpecialCharTok{\textless{}} \FloatTok{0.001}\NormalTok{) \{}
\NormalTok{        ast }\OtherTok{\textless{}{-}} \StringTok{"***"}
\NormalTok{      \} }\ControlFlowTok{else} \ControlFlowTok{if}\NormalTok{(pval }\SpecialCharTok{\textless{}} \FloatTok{0.01}\NormalTok{) \{}
\NormalTok{        ast }\OtherTok{\textless{}{-}} \StringTok{"**"}
\NormalTok{      \} }\ControlFlowTok{else} \ControlFlowTok{if}\NormalTok{(pval }\SpecialCharTok{\textless{}} \FloatTok{0.05}\NormalTok{) \{}
\NormalTok{        ast }\OtherTok{\textless{}{-}} \StringTok{"*"}
\NormalTok{      \} }\ControlFlowTok{else}\NormalTok{ \{}
\NormalTok{        ast }\OtherTok{\textless{}{-}} \StringTok{""}
\NormalTok{      \}}
      \ControlFlowTok{if}\NormalTok{(ast }\SpecialCharTok{!=} \StringTok{""}\NormalTok{) \{}
        \CommentTok{\# Ajuste os valores de x e y para posicionar acima e à direita}
\NormalTok{        x }\OtherTok{\textless{}{-}}\NormalTok{ j }\SpecialCharTok{+} \FloatTok{0.25}
\NormalTok{        y }\OtherTok{\textless{}{-}}\NormalTok{ n }\SpecialCharTok{{-}}\NormalTok{ i }\SpecialCharTok{+} \DecValTok{1} \SpecialCharTok{+} \FloatTok{0.25}
        \FunctionTok{text}\NormalTok{(x, y, }\AttributeTok{labels =}\NormalTok{ ast, }\AttributeTok{col =} \StringTok{"red"}\NormalTok{, }\AttributeTok{cex =} \FloatTok{1.2}\NormalTok{, }\AttributeTok{font =} \DecValTok{2}\NormalTok{)}
\NormalTok{      \}}
\NormalTok{    \}}
\NormalTok{  \}}
\NormalTok{\}}

\CommentTok{\# Adiciona a nota de rodapé explicando os asteriscos}
\FunctionTok{mtext}\NormalTok{(}
  \StringTok{"*** p \textless{} 0.001   ** p \textless{} 0.01   * p \textless{} 0.05"}\NormalTok{,}
  \AttributeTok{side =} \DecValTok{1}\NormalTok{,         }\CommentTok{\# parte inferior do gráfico}
  \AttributeTok{line =} \DecValTok{3}\NormalTok{,         }\CommentTok{\# distância da margem}
  \AttributeTok{cex =} \FloatTok{0.7}\NormalTok{,        }\CommentTok{\# tamanho da fonte}
  \AttributeTok{adj =} \DecValTok{0}           \CommentTok{\# centralizado}
\NormalTok{)}
\InformationTok{\textasciigrave{}\textasciigrave{}\textasciigrave{}}
\end{Highlighting}
\end{Shaded}

\pandocbounded{\includegraphics[keepaspectratio]{cap5-pldr-modelos-dados_files/figure-pdf/unnamed-chunk-23-1.pdf}}

Há correlações que são \textbf{\emph{esperadas}}, como PIB e PIB\_pc,
entre PIB\_pc e Pop ou entre DPGO\_MPE\_pc e Pop, dada ao modo como
foram calculados esses indicadores \emph{per capta}.

\subsection{Contexto MPE no Estado de
Goiás}\label{contexto-mpe-no-estado-de-goiuxe1s}

Proporção de órgãos públicos estaduais no valor total de compras
públicas realizadas pelo Estado de Goiás ao contratar com MPEs, no
período 2006 a 2019, com uma análise de Pareto.

Comparação do valor total, em reais (R\$) gasto nas licitações do estado
de Goiás, de 2009 a 2019, com a participação das MPEs em relação a esse
total de compras públicas, (BARZELLAY; DAS NEVES, 2022 , p.~146, gráfico
17).

\pandocbounded{\includegraphics[keepaspectratio]{images/clipboard-2788289314.png}}

Agora a \emph{mesma informação} acima expressa como \textbf{proporção
(\%)} do \textbf{\emph{valor das compras públicas pagos às MPE's}} em
\ul{\textbf{\emph{relação}}} ao \textbf{\emph{valor total das despesas
do Estado de Goiás com licitações}} (2009 a 2019), no mesmo período de
11 anos (BARZELLAY; DAS NEVES, 2022 , p.~146, gráfico 18).

\pandocbounded{\includegraphics[keepaspectratio]{images/clipboard-862037786.png}}

Agora é nítida a \ul{\textbf{tendência}} de \textbf{\emph{queda
sistemática}} dessa \textbf{\emph{proporção (\%)}} desde 2010, quando
chegou a \textbf{\emph{41,9\%}}, em um ano em que claramente foi o de
menor despesa pública com licitações; até 2019, quando terminou com a
menor proporção registrada no período de 11 anos, com
\textbf{\emph{6,4\%}}.

\textbf{\emph{Tendência}} essa em \textbf{\emph{clara divergência}} com
a \textbf{\emph{Política Pública de fomento às MPPs}} preconizada no
art. 179, CF/1988 e arts. 44 e 45,
\href{https://www.planalto.gov.br/ccivil_03/leis/lcp/lcp123.htm}{LC
n.~123/2006} - Estatuto Nacional da Microempresa e da Empresa de Pequeno
Porte).

Uma \textbf{Análise de Pareto} referente ao volume de recursos
fiscalizados (VTF, R\$) de cada um dos \textbf{28 órgãos licitantes},
dos n = 267 processos cujos acórdãos foram analisados (possível viés de
busca por palavras chave no site do TCE-GO), ilusta em que órgãos
concentram-se as despesas públicas com MPE's.

\begin{figure}[H]

{\centering \pandocbounded{\includegraphics[keepaspectratio]{fig/MPE-analise-Pareto-VRF.png}}

}

\caption{Análise de Pareto referente ao volume de recursos fiscalizados
(VRF, R\$) de cada órgão licitante dos 267 processos cujos acórdãos
foram analisados.}

\end{figure}%

Esse gráfico evidencia que \textbf{20\%} dos \textbf{28 órgãos}, que
corresponde a 0,20 x 28 = 5,6 = \textbf{\emph{6 primeiros órgãos}},
\textbf{concentram 80\% do VRF} - Volume de Recursos Fiscalizados (R\$)
pelo TCE-GO quanto às despesas com licitação do Estado de Goiás (2006 a
2019, um período de 14 anos).

Os órgãos em que o Controle Externo do TCE-GO deveria
\textbf{\emph{priorizar}} o \textbf{\emph{controle}} do
\textbf{\emph{poder das compras públicas}} tendo em vista o
monitoramento do cumprimento da Política Pública de fomento às MPE's
são:

\begin{enumerate}
\def\labelenumi{\arabic{enumi}.}
\tightlist
\item
  \texttt{SES} - Secretaria de Estado da Saúde
\item
  \texttt{GOINFRA}
\item
  \texttt{AGEHAB}
\item
  \texttt{SSP}
\item
  \texttt{SANEAGO}
\item
  \texttt{SEAD} - Secretaria de Estado da Administração
\end{enumerate}

Para se ter uma ideia do alcance dessa política em relação a todas as
MPE's ativas sediadas no Estado de Goiás, do total das \textbf{629.359}
(seiscentos e vinte e nove mil, trezentos e cinquenta e nove) empresas
registradas como empresa de pequeno porte com endereço no Estado de
Goiás, apenas \textbf{6.893} (seis mil, oitocentos e noventa e três), ou
seja, \textbf{1,1\%} (um vírgula zero nove por cento)
\textbf{\emph{consta da lista das empresas beneficiadas com os empenhos
realizados pelo Estado de Goiás entre 2006 e 2020}}.

O gráfico a seguir ilustra esse cenário.

\begin{figure}[H]

{\centering \pandocbounded{\includegraphics[keepaspectratio]{fig/MPE-proporcao-GO-contratadas-x-naoContr.png}}

}

\caption{Relação MPEs ativas, com endereço registrado no Estado de
Goiás, contratadas e não contratadas pelo Estado de Goiás (2006 a 2020),
em números.}

\end{figure}%

Apenas 1\% do total de MPE's ativas situadas no Estado de Goiás tiveram
acesso à Política Pública de fomento prevista na
\href{https://www.planalto.gov.br/ccivil_03/leis/lcp/lcp123.htm}{LC
n.~123/2006}, o que denota um amplo espaço de alcance dessa política
pública ainda sem cobertura.

\subsection{Análise Exploratória Explicativa
bivariada}\label{anuxe1lise-exploratuxf3ria-explicativa-bivariada}

\subsubsection{\texorpdfstring{Y = \texttt{CAGED} e X =
\texttt{PgtoGO\_MPE}}{Y = CAGED e X = PgtoGO\_MPE}}\label{y-caged-e-x-pgtogo_mpe}

Baseia-se no estudo de uma possível \textbf{\emph{relação linear}} entre
\textbf{uma variável resposta Y} e \textbf{uma variável explicativa X},
ambas \textbf{quantitativas}.

Vamos considerar, inicialmente:

\begin{itemize}
\tightlist
\item
  Y = \texttt{CAGED}
\item
  X = \texttt{PgtoGO\_MPE}
\end{itemize}

Script a seguir gera um gráfico de dispersão com a reta de regressão.

\begin{Shaded}
\begin{Highlighting}[numbers=left,,]
\InformationTok{\textasciigrave{}\textasciigrave{}\textasciigrave{}\{r\}}
\FunctionTok{library}\NormalTok{(ggpubr)}

\CommentTok{\# dados do data frame chamado mpe}

\FunctionTok{summary}\NormalTok{(mpe}\SpecialCharTok{$}\NormalTok{CAGED)}

\CommentTok{\# Gráfico de dispersão com reta de regressão linear, r e R²}
\FunctionTok{ggplot}\NormalTok{(mpe, }\FunctionTok{aes}\NormalTok{(}\AttributeTok{x =}\NormalTok{ PgtoGO\_MPE }\SpecialCharTok{/} \DecValTok{1000000}\NormalTok{,}
                \AttributeTok{y =}\NormalTok{ CAGED }\SpecialCharTok{/} \DecValTok{1000}\NormalTok{)) }\SpecialCharTok{+}
  \FunctionTok{geom\_point}\NormalTok{(}\AttributeTok{color =} \StringTok{"blue"}\NormalTok{, }\AttributeTok{alpha =} \FloatTok{0.7}\NormalTok{) }\SpecialCharTok{+}           \CommentTok{\# pontos de dispersão}
  \FunctionTok{geom\_smooth}\NormalTok{(}\AttributeTok{method =} \StringTok{"lm"}\NormalTok{, }\AttributeTok{se =} \ConstantTok{TRUE}\NormalTok{, }\AttributeTok{color =} \StringTok{"red"}\NormalTok{) }\SpecialCharTok{+} \CommentTok{\# reta de regressão linear com intervalo de confiança}
  \FunctionTok{stat\_cor}\NormalTok{(}
  \FunctionTok{aes}\NormalTok{(}\AttributeTok{label =} \FunctionTok{paste}\NormalTok{(..r.label.., ..p.label.., }\AttributeTok{sep =} \StringTok{"\textasciitilde{}\textasciigrave{},\textasciigrave{}\textasciitilde{}"}\NormalTok{)),}
  \AttributeTok{label.x =} \ConstantTok{Inf}\NormalTok{, }\AttributeTok{label.y =} \SpecialCharTok{{-}}\ConstantTok{Inf}\NormalTok{, }\AttributeTok{hjust =} \FloatTok{1.1}\NormalTok{, }\AttributeTok{vjust =} \SpecialCharTok{{-}}\FloatTok{0.5}\NormalTok{, }\AttributeTok{size =} \DecValTok{5}
\NormalTok{) }\SpecialCharTok{+}
\FunctionTok{stat\_regline\_equation}\NormalTok{(}
  \FunctionTok{aes}\NormalTok{(}\AttributeTok{label =} \FunctionTok{paste}\NormalTok{(..eq.label.., ..rr.label.., }\AttributeTok{sep =} \StringTok{"\textasciitilde{}\textasciitilde{}\textasciitilde{}"}\NormalTok{)),}
  \AttributeTok{label.x =} \ConstantTok{Inf}\NormalTok{, }\AttributeTok{label.y =} \ConstantTok{Inf}\NormalTok{, }\AttributeTok{hjust =} \FloatTok{1.1}\NormalTok{, }\AttributeTok{vjust =} \DecValTok{2}\NormalTok{, }\AttributeTok{size =} \DecValTok{5}
\NormalTok{) }\SpecialCharTok{+}
  \FunctionTok{labs}\NormalTok{(}
    \AttributeTok{title =} \StringTok{"Gráfico de Dispersão c/Reta Regressão"}\NormalTok{,}
    \AttributeTok{subtitle =} \StringTok{"Período: 2006 a 2019 (n = 14 obs.)"}\NormalTok{,}
    \AttributeTok{x =} \StringTok{"PgtoGO\_MPE (despesa pública GO c/MPE, milhões R$)"}\NormalTok{,}
    \AttributeTok{y =} \StringTok{"CAGED (saldo em milhares de empregos)"}
\NormalTok{  ) }\SpecialCharTok{+}
  \FunctionTok{ylim}\NormalTok{(}\SpecialCharTok{{-}}\DecValTok{30}\NormalTok{, }\DecValTok{85}\NormalTok{) }\SpecialCharTok{+}
  \FunctionTok{geom\_hline}\NormalTok{(}\AttributeTok{yintercept =} \DecValTok{0}\NormalTok{, }\AttributeTok{linetype =} \StringTok{"dashed"}\NormalTok{) }\SpecialCharTok{+}
  \FunctionTok{theme\_minimal}\NormalTok{(}\AttributeTok{base\_size =} \DecValTok{14}\NormalTok{)         }\CommentTok{\# tema visual limpo e fonte maior}
\InformationTok{\textasciigrave{}\textasciigrave{}\textasciigrave{}}
\end{Highlighting}
\end{Shaded}

\begin{verbatim}
   Min. 1st Qu.  Median    Mean 3rd Qu.    Max. 
 -24551   21183   29887   33585   57460   83975 
\end{verbatim}

\pandocbounded{\includegraphics[keepaspectratio]{cap5-pldr-modelos-dados_files/figure-pdf/unnamed-chunk-24-1.pdf}}

\subsubsection{Z-score}\label{z-score}

Calcular o Z-score para os dados do data frame \texttt{mpe}.

Exceto para a coluna \texttt{ano}, para a qual não faz sentido
determinar o Z-score.

\begin{Shaded}
\begin{Highlighting}[numbers=left,,]
\InformationTok{\textasciigrave{}\textasciigrave{}\textasciigrave{}\{r\}}
\CommentTok{\# Função para calcular o z{-}score de um vetor numérico}
\NormalTok{z\_score }\OtherTok{\textless{}{-}} \ControlFlowTok{function}\NormalTok{(x) \{}
  \CommentTok{\# Remove valores ausentes (NA) do cálculo da média e do desvio padrão}
\NormalTok{  media }\OtherTok{\textless{}{-}} \FunctionTok{mean}\NormalTok{(x, }\AttributeTok{na.rm =} \ConstantTok{TRUE}\NormalTok{)}
\NormalTok{  desvio }\OtherTok{\textless{}{-}} \FunctionTok{sd}\NormalTok{(x, }\AttributeTok{na.rm =} \ConstantTok{TRUE}\NormalTok{)}
  \CommentTok{\# Calcula o z{-}score para cada elemento de x}
\NormalTok{  z }\OtherTok{\textless{}{-}}\NormalTok{ (x }\SpecialCharTok{{-}}\NormalTok{ media) }\SpecialCharTok{/}\NormalTok{ desvio}
  \FunctionTok{return}\NormalTok{(z)}
\NormalTok{\}}

\CommentTok{\# Interpretação: valores positivos estão acima da média, negativos abaixo.}

\CommentTok{\# Se usar essa função com um vetor contendo NA}
\CommentTok{\# Os NAs permanecem como NA no resultado.}

\CommentTok{\# deletar a variável mpe\_z, caso ela exista}
\ControlFlowTok{if}\NormalTok{ (}\FunctionTok{exists}\NormalTok{(}\StringTok{"mpe\_z"}\NormalTok{)) \{}
  \FunctionTok{rm}\NormalTok{(mpe\_z) }\CommentTok{\# remove mpe\_z, somente se ela existir}
\NormalTok{\}}

\NormalTok{mpe }\CommentTok{\# exibe o data frame mpe}

\CommentTok{\# Aplicar a função z\_score em cada coluna do data frame mpe:}
\CommentTok{\# Exceto à primeira coluna ano, porque não faz sentido.}
\NormalTok{mpe\_z }\OtherTok{\textless{}{-}} \FunctionTok{as.data.frame}\NormalTok{(}\FunctionTok{apply}\NormalTok{(mpe[, }\SpecialCharTok{{-}}\DecValTok{1}\NormalTok{], }\AttributeTok{MARGIN=}\DecValTok{2}\NormalTok{, }\AttributeTok{FUN=}\NormalTok{z\_score))}
\NormalTok{mpe\_z }\OtherTok{\textless{}{-}}\NormalTok{ mpe\_z }\SpecialCharTok{\%\textgreater{}\%} 
\NormalTok{  dplyr}\SpecialCharTok{::}\FunctionTok{mutate}\NormalTok{(}\AttributeTok{ano =}\NormalTok{ mpe}\SpecialCharTok{$}\NormalTok{ano, }\AttributeTok{.before =}\NormalTok{ PgtoGO\_MPE)}
\NormalTok{mpe\_z }\CommentTok{\# exibe o data frame de z{-}socores, acrescido da 1ª coluna ano.}
\InformationTok{\textasciigrave{}\textasciigrave{}\textasciigrave{}}
\end{Highlighting}
\end{Shaded}

\begin{verbatim}
# A tibble: 14 x 8
     ano PgtoGO_MPE       PIB  CAGED    RAIS     Pop DPGO_MPE_pc PIB_pc
   <dbl>      <dbl>     <dbl>  <dbl>   <dbl>   <dbl>       <dbl>  <dbl>
 1  2006 165475952.  61375409  21061  992822 5730762        28.9 10710.
 2  2007 175555748.  71410569  41153 1061426 5840650        30.1 12226.
 3  2008 205135955.  82417571  47347 1135046 5844996        35.1 14101.
 4  2009 329706125.  92865740  34404 1209310 5926308        55.6 15670.
 5  2010 453279122. 106770107  83975 1313641 6003788        75.5 17784.
 6  2011 333961964. 121296722  69552 1385230 6080716        54.9 19948.
 7  2012 472699129. 138757833  66230 1439341 6154996        76.8 22544.
 8  2013 640670509. 151300180  60831 1509395 6434048        99.6 23516.
 9  2014 562863907. 165015307  25333 1514532 6523222        86.3 25297.
10  2015 370979808. 173632448 -24551 1501397 6610681        56.1 26265.
11  2016 393978072. 181692438 -19354 1445943 6695855        58.8 27135.
12  2017 490777617. 191898682  25370 1515422 6778772        72.4 28309.
13  2018 422735929. 195682000  17293 1507648 6921161        61.1 28273 
14  2019 241392288. 208672000  21550 1524304 6939629        34.8 30070.
    ano PgtoGO_MPE      PIB  CAGED  RAIS   Pop DPGO_MPE_pc PIB_pc
1  2006     -1.454 -1.56017 -0.400 -1.97 -1.38     -1.3900  -1.67
2  2007     -1.384 -1.35788  0.241 -1.61 -1.12     -1.3356  -1.44
3  2008     -1.179 -1.13599  0.439 -1.21 -1.11     -1.1030  -1.15
4  2009     -0.318 -0.92537  0.026 -0.81 -0.92     -0.1555  -0.91
5  2010      0.537 -0.64508  1.608 -0.25 -0.74      0.7616  -0.58
6  2011     -0.288 -0.35225  1.147  0.13 -0.56     -0.1882  -0.25
7  2012      0.671 -0.00026  1.041  0.42 -0.39      0.8216   0.15
8  2013      1.833  0.25258  0.869  0.79  0.27      1.8729   0.30
9  2014      1.295  0.52906 -0.263  0.82  0.47      1.2595   0.58
10 2015     -0.032  0.70277 -1.855  0.75  0.68     -0.1329   0.73
11 2016      0.127  0.86524 -1.689  0.45  0.88     -0.0073   0.86
12 2017      0.796  1.07099 -0.262  0.83  1.07      0.6185   1.04
13 2018      0.326  1.14725 -0.520  0.79  1.40      0.0961   1.03
14 2019     -0.929  1.40911 -0.384  0.87  1.45     -1.1177   1.31
\end{verbatim}

Mesmo grágfico de dispersão acima, mas agora para os escores-z das
variáveis X e Y.

\begin{Shaded}
\begin{Highlighting}[numbers=left,,]
\InformationTok{\textasciigrave{}\textasciigrave{}\textasciigrave{}\{r\}}
\CommentTok{\# dados do data frame chamado mpe}

\FunctionTok{summary}\NormalTok{(mpe\_z}\SpecialCharTok{$}\NormalTok{CAGED)}

\CommentTok{\# Gráfico de dispersão com reta de regressão linear, r e R²}
\FunctionTok{ggplot}\NormalTok{(mpe\_z, }\FunctionTok{aes}\NormalTok{(}\AttributeTok{x =}\NormalTok{ PgtoGO\_MPE,}
                  \AttributeTok{y =}\NormalTok{ CAGED)) }\SpecialCharTok{+}
  \FunctionTok{geom\_point}\NormalTok{(}\AttributeTok{color =} \StringTok{"blue"}\NormalTok{, }\AttributeTok{alpha =} \FloatTok{0.7}\NormalTok{) }\SpecialCharTok{+}           \CommentTok{\# pontos de dispersão}
  \FunctionTok{geom\_smooth}\NormalTok{(}\AttributeTok{method =} \StringTok{"lm"}\NormalTok{, }\AttributeTok{se =} \ConstantTok{TRUE}\NormalTok{, }\AttributeTok{color =} \StringTok{"red"}\NormalTok{) }\SpecialCharTok{+} \CommentTok{\# reta de regressão linear com intervalo de confiança}
  \FunctionTok{stat\_cor}\NormalTok{(}
  \FunctionTok{aes}\NormalTok{(}\AttributeTok{label =} \FunctionTok{paste}\NormalTok{(..r.label.., ..p.label.., }\AttributeTok{sep =} \StringTok{"\textasciitilde{}\textasciigrave{},\textasciigrave{}\textasciitilde{}"}\NormalTok{)),}
  \AttributeTok{label.x =} \ConstantTok{Inf}\NormalTok{, }\AttributeTok{label.y =} \SpecialCharTok{{-}}\ConstantTok{Inf}\NormalTok{, }\AttributeTok{hjust =} \FloatTok{1.1}\NormalTok{, }\AttributeTok{vjust =} \SpecialCharTok{{-}}\FloatTok{0.5}\NormalTok{, }\AttributeTok{size =} \DecValTok{5}
\NormalTok{) }\SpecialCharTok{+}
\FunctionTok{stat\_regline\_equation}\NormalTok{(}
  \FunctionTok{aes}\NormalTok{(}\AttributeTok{label =} \FunctionTok{paste}\NormalTok{(..eq.label.., ..rr.label.., }\AttributeTok{sep =} \StringTok{"\textasciitilde{}\textasciitilde{}\textasciitilde{}"}\NormalTok{)),}
  \AttributeTok{label.x =} \ConstantTok{Inf}\NormalTok{, }\AttributeTok{label.y =} \ConstantTok{Inf}\NormalTok{, }\AttributeTok{hjust =} \FloatTok{1.1}\NormalTok{, }\AttributeTok{vjust =} \DecValTok{2}\NormalTok{, }\AttributeTok{size =} \DecValTok{5}
\NormalTok{) }\SpecialCharTok{+}
  \FunctionTok{labs}\NormalTok{(}
    \AttributeTok{title =} \StringTok{"Gráfico de Dispersão c/Reta Regressão p/Z{-}scores"}\NormalTok{,}
    \AttributeTok{subtitle =} \StringTok{"Período: 2006 a 2019 (n = 14 obs.)"}\NormalTok{,}
    \AttributeTok{x =} \StringTok{"Z{-}PgtoGO\_MPE (Z{-}score da despesa pública GO c/MPE)"}\NormalTok{,}
    \AttributeTok{y =} \StringTok{"Z{-}CAGED (Z{-}score do saldo de empregos)"}
\NormalTok{  ) }\SpecialCharTok{+}
  \FunctionTok{xlim}\NormalTok{(}\SpecialCharTok{{-}}\DecValTok{2}\NormalTok{, }\DecValTok{2}\NormalTok{) }\SpecialCharTok{+}
  \FunctionTok{ylim}\NormalTok{(}\SpecialCharTok{{-}}\DecValTok{2}\NormalTok{, }\DecValTok{2}\NormalTok{) }\SpecialCharTok{+}
  \FunctionTok{theme\_minimal}\NormalTok{(}\AttributeTok{base\_size =} \DecValTok{14}\NormalTok{)         }\CommentTok{\# tema visual limpo e fonte maior}
\InformationTok{\textasciigrave{}\textasciigrave{}\textasciigrave{}}
\end{Highlighting}
\end{Shaded}

\begin{verbatim}
   Min. 1st Qu.  Median    Mean 3rd Qu.    Max. 
 -1.855  -0.396  -0.118   0.000   0.762   1.608 
\end{verbatim}

\pandocbounded{\includegraphics[keepaspectratio]{cap5-pldr-modelos-dados_files/figure-pdf/unnamed-chunk-26-1.pdf}}

Observar que ao trasformar os valores originais em score-Z, os gráficos
de dispersão e a reta de regressão não mudam de forma.

Apenas a \emph{reta de regressão} fica \textbf{\emph{centrada na
origem}}, ponto (X = 0, Y = 0).

O resultado final é uma \ul{\textbf{fraca correção}} (r = 0,16) entre Y
= \texttt{CAGED} e X = \texttt{PgtoGO\_MPE}.

Além disso essa \textbf{\emph{correlação não é estatisticamente
significativa}} para um \emph{nível de significância} de 5\% (Erro tipo
I, alpha = 0,05 = 5\%), poi seu valor-P = 0,59 = 59\%.

\subsubsection{\texorpdfstring{Y = \texttt{RAIS} e X =
\texttt{PgtoGO\_MPE}}{Y = RAIS e X = PgtoGO\_MPE}}\label{y-rais-e-x-pgtogo_mpe}

Vamos considerar, agora:

\begin{itemize}
\tightlist
\item
  Y = \texttt{RAIS}
\item
  X = \texttt{PgtoGO\_MPE}
\end{itemize}

Script a seguir gera um gráfico de dispersão com a reta de regressão.

\begin{Shaded}
\begin{Highlighting}[numbers=left,,]
\InformationTok{\textasciigrave{}\textasciigrave{}\textasciigrave{}\{r\}}
\CommentTok{\# dados do data frame chamado mpe}

\FunctionTok{summary}\NormalTok{(mpe}\SpecialCharTok{$}\NormalTok{RAIS)}

\CommentTok{\# Gráfico de dispersão com reta de regressão linear, r e R²}
\FunctionTok{ggplot}\NormalTok{(mpe, }\FunctionTok{aes}\NormalTok{(}\AttributeTok{x =}\NormalTok{ PgtoGO\_MPE }\SpecialCharTok{/} \DecValTok{1000000}\NormalTok{,}
                \AttributeTok{y =}\NormalTok{ RAIS }\SpecialCharTok{/} \DecValTok{1000000}\NormalTok{)) }\SpecialCharTok{+}
  \FunctionTok{geom\_point}\NormalTok{(}\AttributeTok{color =} \StringTok{"blue"}\NormalTok{, }\AttributeTok{alpha =} \FloatTok{0.7}\NormalTok{) }\SpecialCharTok{+}           \CommentTok{\# pontos de dispersão}
  \FunctionTok{geom\_smooth}\NormalTok{(}\AttributeTok{method =} \StringTok{"lm"}\NormalTok{, }\AttributeTok{se =} \ConstantTok{TRUE}\NormalTok{, }\AttributeTok{color =} \StringTok{"red"}\NormalTok{) }\SpecialCharTok{+} \CommentTok{\# reta de regressão linear com intervalo de confiança}
  \FunctionTok{stat\_cor}\NormalTok{(}
  \FunctionTok{aes}\NormalTok{(}\AttributeTok{label =} \FunctionTok{paste}\NormalTok{(..r.label.., ..p.label.., }\AttributeTok{sep =} \StringTok{"\textasciitilde{}\textasciigrave{},\textasciigrave{}\textasciitilde{}"}\NormalTok{)),}
  \AttributeTok{label.x =} \ConstantTok{Inf}\NormalTok{, }\AttributeTok{label.y =} \SpecialCharTok{{-}}\ConstantTok{Inf}\NormalTok{, }\AttributeTok{hjust =} \FloatTok{1.1}\NormalTok{, }\AttributeTok{vjust =} \SpecialCharTok{{-}}\FloatTok{0.5}\NormalTok{, }\AttributeTok{size =} \DecValTok{5}
\NormalTok{) }\SpecialCharTok{+}
\FunctionTok{stat\_regline\_equation}\NormalTok{(}
  \FunctionTok{aes}\NormalTok{(}\AttributeTok{label =} \FunctionTok{paste}\NormalTok{(..eq.label.., ..rr.label.., }\AttributeTok{sep =} \StringTok{"\textasciitilde{}\textasciitilde{}\textasciitilde{}"}\NormalTok{)),}
  \AttributeTok{label.x =} \ConstantTok{Inf}\NormalTok{, }\AttributeTok{label.y =} \ConstantTok{Inf}\NormalTok{, }\AttributeTok{hjust =} \FloatTok{1.1}\NormalTok{, }\AttributeTok{vjust =} \DecValTok{2}\NormalTok{, }\AttributeTok{size =} \DecValTok{5}
\NormalTok{) }\SpecialCharTok{+}
  \FunctionTok{labs}\NormalTok{(}
    \AttributeTok{title =} \StringTok{"Gráfico de Dispersão c/Reta Regressão"}\NormalTok{,}
    \AttributeTok{subtitle =} \StringTok{"Período: 2006 a 2019 (n = 14 obs.)"}\NormalTok{,}
    \AttributeTok{x =} \StringTok{"PgtoGO\_MPE (despesa pública GO c/MPE, milhões R$)"}\NormalTok{,}
    \AttributeTok{y =} \StringTok{"RAIS (núm. empregos, em milhões)"}
\NormalTok{  ) }\SpecialCharTok{+}
  \FunctionTok{ylim}\NormalTok{(}\DecValTok{0}\NormalTok{, }\DecValTok{2}\NormalTok{) }\SpecialCharTok{+}
  \FunctionTok{theme\_minimal}\NormalTok{(}\AttributeTok{base\_size =} \DecValTok{14}\NormalTok{)         }\CommentTok{\# tema visual limpo e fonte maior}
\InformationTok{\textasciigrave{}\textasciigrave{}\textasciigrave{}}
\end{Highlighting}
\end{Shaded}

\begin{verbatim}
   Min. 1st Qu.  Median    Mean 3rd Qu.    Max. 
 992822 1235393 1442642 1361104 1508958 1524304 
\end{verbatim}

\pandocbounded{\includegraphics[keepaspectratio]{cap5-pldr-modelos-dados_files/figure-pdf/unnamed-chunk-27-1.pdf}}

O mesmo gráfico para os já calculados escores-Z.

\begin{Shaded}
\begin{Highlighting}[numbers=left,,]
\InformationTok{\textasciigrave{}\textasciigrave{}\textasciigrave{}\{r\}}
\CommentTok{\# dados do data frame chamado mpe}

\FunctionTok{summary}\NormalTok{(mpe\_z}\SpecialCharTok{$}\NormalTok{RAIS)}

\CommentTok{\# Gráfico de dispersão com reta de regressão linear, r e R²}
\FunctionTok{ggplot}\NormalTok{(mpe\_z, }\FunctionTok{aes}\NormalTok{(}\AttributeTok{x =}\NormalTok{ PgtoGO\_MPE,}
                  \AttributeTok{y =}\NormalTok{ RAIS)) }\SpecialCharTok{+}
  \FunctionTok{geom\_point}\NormalTok{(}\AttributeTok{color =} \StringTok{"blue"}\NormalTok{, }\AttributeTok{alpha =} \FloatTok{0.7}\NormalTok{) }\SpecialCharTok{+}           \CommentTok{\# pontos de dispersão}
  \FunctionTok{geom\_smooth}\NormalTok{(}\AttributeTok{method =} \StringTok{"lm"}\NormalTok{, }\AttributeTok{se =} \ConstantTok{TRUE}\NormalTok{, }\AttributeTok{color =} \StringTok{"red"}\NormalTok{) }\SpecialCharTok{+} \CommentTok{\# reta de regressão linear com intervalo de confiança}
  \FunctionTok{stat\_cor}\NormalTok{(}
  \FunctionTok{aes}\NormalTok{(}\AttributeTok{label =} \FunctionTok{paste}\NormalTok{(..r.label.., ..p.label.., }\AttributeTok{sep =} \StringTok{"\textasciitilde{}\textasciigrave{},\textasciigrave{}\textasciitilde{}"}\NormalTok{)),}
  \AttributeTok{label.x =} \ConstantTok{Inf}\NormalTok{, }\AttributeTok{label.y =} \SpecialCharTok{{-}}\ConstantTok{Inf}\NormalTok{, }\AttributeTok{hjust =} \FloatTok{1.1}\NormalTok{, }\AttributeTok{vjust =} \SpecialCharTok{{-}}\FloatTok{0.5}\NormalTok{, }\AttributeTok{size =} \DecValTok{5}
\NormalTok{) }\SpecialCharTok{+}
\FunctionTok{stat\_regline\_equation}\NormalTok{(}
  \FunctionTok{aes}\NormalTok{(}\AttributeTok{label =} \FunctionTok{paste}\NormalTok{(..eq.label.., ..rr.label.., }\AttributeTok{sep =} \StringTok{"\textasciitilde{}\textasciitilde{}\textasciitilde{}"}\NormalTok{)),}
  \AttributeTok{label.x =} \ConstantTok{Inf}\NormalTok{, }\AttributeTok{label.y =} \ConstantTok{Inf}\NormalTok{, }\AttributeTok{hjust =} \FloatTok{1.1}\NormalTok{, }\AttributeTok{vjust =} \DecValTok{2}\NormalTok{, }\AttributeTok{size =} \DecValTok{5}
\NormalTok{) }\SpecialCharTok{+}
  \FunctionTok{labs}\NormalTok{(}
    \AttributeTok{title =} \StringTok{"Gráfico de Dispersão c/Reta Regressão p/Z{-}scores"}\NormalTok{,}
    \AttributeTok{subtitle =} \StringTok{"Período: 2006 a 2019 (n = 14 obs.)"}\NormalTok{,}
    \AttributeTok{x =} \StringTok{"Z{-}PgtoGO\_MPE (Z{-}score da despesa pública GO c/MPE)"}\NormalTok{,}
    \AttributeTok{y =} \StringTok{"Z{-}CAGED (Z{-}score do saldo de empregos)"}
\NormalTok{  ) }\SpecialCharTok{+}
  \FunctionTok{xlim}\NormalTok{(}\SpecialCharTok{{-}}\DecValTok{2}\NormalTok{, }\DecValTok{2}\NormalTok{) }\SpecialCharTok{+}
  \FunctionTok{ylim}\NormalTok{(}\SpecialCharTok{{-}}\DecValTok{2}\NormalTok{, }\DecValTok{2}\NormalTok{) }\SpecialCharTok{+}
  \FunctionTok{theme\_minimal}\NormalTok{(}\AttributeTok{base\_size =} \DecValTok{14}\NormalTok{)         }\CommentTok{\# tema visual limpo e fonte maior}
\InformationTok{\textasciigrave{}\textasciigrave{}\textasciigrave{}}
\end{Highlighting}
\end{Shaded}

\begin{verbatim}
   Min. 1st Qu.  Median    Mean 3rd Qu.    Max. 
 -1.973  -0.674   0.437   0.000   0.792   0.874 
\end{verbatim}

\pandocbounded{\includegraphics[keepaspectratio]{cap5-pldr-modelos-dados_files/figure-pdf/unnamed-chunk-28-1.pdf}}

O resultado final é uma \ul{\textbf{forte correção}} (r = 0,73) entre Y
= \texttt{RAIS} e X = \texttt{PgtoGO\_MPE}.

Além disso essa \textbf{\emph{correlação \ul{é estatisticamente
significativa}}} para um \emph{nível de significância} de 5\% (Erro tipo
I, alpha = 0,05 = 5\%), poi seu \textbf{valor-P} = 0,0028 = 0,3\% é
\ul{\textbf{menor que}} 5,0\% = alpha.

\subsubsection{\texorpdfstring{Y = \texttt{PIB} e X =
\texttt{PgtoGO\_MPE}}{Y = PIB e X = PgtoGO\_MPE}}\label{y-pib-e-x-pgtogo_mpe}

Vamos considerar, agora:

\begin{itemize}
\tightlist
\item
  Y = \texttt{PIB}
\item
  X = \texttt{PgtoGO\_MPE}
\end{itemize}

Script a seguir gera um gráfico de dispersão com a reta de regressão.

\begin{Shaded}
\begin{Highlighting}[numbers=left,,]
\InformationTok{\textasciigrave{}\textasciigrave{}\textasciigrave{}\{r\}}
\CommentTok{\# dados do data frame chamado mpe}

\FunctionTok{summary}\NormalTok{(mpe}\SpecialCharTok{$}\NormalTok{PIB)}

\CommentTok{\# Gráfico de dispersão com reta de regressão linear, r e R²}
\FunctionTok{ggplot}\NormalTok{(mpe, }\FunctionTok{aes}\NormalTok{(}\AttributeTok{x =}\NormalTok{ PgtoGO\_MPE }\SpecialCharTok{/} \DecValTok{1000000}\NormalTok{,}
                \AttributeTok{y =}\NormalTok{ PIB }\SpecialCharTok{/} \DecValTok{1000000}\NormalTok{)) }\SpecialCharTok{+}
  \FunctionTok{geom\_point}\NormalTok{(}\AttributeTok{color =} \StringTok{"blue"}\NormalTok{, }\AttributeTok{alpha =} \FloatTok{0.7}\NormalTok{) }\SpecialCharTok{+}           \CommentTok{\# pontos de dispersão}
  \FunctionTok{geom\_smooth}\NormalTok{(}\AttributeTok{method =} \StringTok{"lm"}\NormalTok{, }\AttributeTok{se =} \ConstantTok{TRUE}\NormalTok{, }\AttributeTok{color =} \StringTok{"red"}\NormalTok{) }\SpecialCharTok{+} \CommentTok{\# reta de regressão linear com intervalo de confiança}
  \FunctionTok{stat\_cor}\NormalTok{(}
  \FunctionTok{aes}\NormalTok{(}\AttributeTok{label =} \FunctionTok{paste}\NormalTok{(..r.label.., ..p.label.., }\AttributeTok{sep =} \StringTok{"\textasciitilde{}\textasciigrave{},\textasciigrave{}\textasciitilde{}"}\NormalTok{)),}
  \AttributeTok{label.x =} \ConstantTok{Inf}\NormalTok{, }\AttributeTok{label.y =} \SpecialCharTok{{-}}\ConstantTok{Inf}\NormalTok{, }\AttributeTok{hjust =} \FloatTok{1.1}\NormalTok{, }\AttributeTok{vjust =} \SpecialCharTok{{-}}\FloatTok{0.5}\NormalTok{, }\AttributeTok{size =} \DecValTok{5}
\NormalTok{) }\SpecialCharTok{+}
\FunctionTok{stat\_regline\_equation}\NormalTok{(}
  \FunctionTok{aes}\NormalTok{(}\AttributeTok{label =} \FunctionTok{paste}\NormalTok{(..eq.label.., ..rr.label.., }\AttributeTok{sep =} \StringTok{"\textasciitilde{}\textasciitilde{}\textasciitilde{}"}\NormalTok{)),}
  \AttributeTok{label.x =} \ConstantTok{Inf}\NormalTok{, }\AttributeTok{label.y =} \ConstantTok{Inf}\NormalTok{, }\AttributeTok{hjust =} \FloatTok{1.1}\NormalTok{, }\AttributeTok{vjust =} \DecValTok{2}\NormalTok{, }\AttributeTok{size =} \DecValTok{5}
\NormalTok{) }\SpecialCharTok{+}
  \FunctionTok{labs}\NormalTok{(}
    \AttributeTok{title =} \StringTok{"Gráfico de Dispersão c/Reta Regressão"}\NormalTok{,}
    \AttributeTok{subtitle =} \StringTok{"Período: 2006 a 2019 (n = 14 obs.)"}\NormalTok{,}
    \AttributeTok{x =} \StringTok{"PgtoGO\_MPE (despesa pública GO c/MPE, milhões R$)"}\NormalTok{,}
    \AttributeTok{y =} \StringTok{"PIB (em milhões R$)"}
\NormalTok{  ) }\SpecialCharTok{+}
  \FunctionTok{ylim}\NormalTok{(}\DecValTok{50}\NormalTok{, }\DecValTok{200}\NormalTok{) }\SpecialCharTok{+}
  \FunctionTok{theme\_minimal}\NormalTok{(}\AttributeTok{base\_size =} \DecValTok{14}\NormalTok{)         }\CommentTok{\# tema visual limpo e fonte maior}
\InformationTok{\textasciigrave{}\textasciigrave{}\textasciigrave{}}
\end{Highlighting}
\end{Shaded}

\begin{verbatim}
    Min.  1st Qu.   Median     Mean  3rd Qu.     Max. 
6.14e+07 9.63e+07 1.45e+08 1.39e+08 1.80e+08 2.09e+08 
\end{verbatim}

\pandocbounded{\includegraphics[keepaspectratio]{cap5-pldr-modelos-dados_files/figure-pdf/unnamed-chunk-29-1.pdf}}

O mesmo gráfico para os já calculados escores-Z.

\begin{Shaded}
\begin{Highlighting}[numbers=left,,]
\InformationTok{\textasciigrave{}\textasciigrave{}\textasciigrave{}\{r\}}
\CommentTok{\# dados do data frame chamado mpe}

\FunctionTok{summary}\NormalTok{(mpe\_z}\SpecialCharTok{$}\NormalTok{PIB)}

\CommentTok{\# Gráfico de dispersão com reta de regressão linear, r e R²}
\FunctionTok{ggplot}\NormalTok{(mpe\_z, }\FunctionTok{aes}\NormalTok{(}\AttributeTok{x =}\NormalTok{ PgtoGO\_MPE,}
                  \AttributeTok{y =}\NormalTok{ PIB)) }\SpecialCharTok{+}
  \FunctionTok{geom\_point}\NormalTok{(}\AttributeTok{color =} \StringTok{"blue"}\NormalTok{, }\AttributeTok{alpha =} \FloatTok{0.7}\NormalTok{) }\SpecialCharTok{+}           \CommentTok{\# pontos de dispersão}
  \FunctionTok{geom\_smooth}\NormalTok{(}\AttributeTok{method =} \StringTok{"lm"}\NormalTok{, }\AttributeTok{se =} \ConstantTok{TRUE}\NormalTok{, }\AttributeTok{color =} \StringTok{"red"}\NormalTok{) }\SpecialCharTok{+} \CommentTok{\# reta de regressão linear com intervalo de confiança}
  \FunctionTok{stat\_cor}\NormalTok{(}
  \FunctionTok{aes}\NormalTok{(}\AttributeTok{label =} \FunctionTok{paste}\NormalTok{(..r.label.., ..p.label.., }\AttributeTok{sep =} \StringTok{"\textasciitilde{}\textasciigrave{},\textasciigrave{}\textasciitilde{}"}\NormalTok{)),}
  \AttributeTok{label.x =} \ConstantTok{Inf}\NormalTok{, }\AttributeTok{label.y =} \SpecialCharTok{{-}}\ConstantTok{Inf}\NormalTok{, }\AttributeTok{hjust =} \FloatTok{1.1}\NormalTok{, }\AttributeTok{vjust =} \SpecialCharTok{{-}}\FloatTok{0.5}\NormalTok{, }\AttributeTok{size =} \DecValTok{5}
\NormalTok{) }\SpecialCharTok{+}
\FunctionTok{stat\_regline\_equation}\NormalTok{(}
  \FunctionTok{aes}\NormalTok{(}\AttributeTok{label =} \FunctionTok{paste}\NormalTok{(..eq.label.., ..rr.label.., }\AttributeTok{sep =} \StringTok{"\textasciitilde{}\textasciitilde{}\textasciitilde{}"}\NormalTok{)),}
  \AttributeTok{label.x =} \ConstantTok{Inf}\NormalTok{, }\AttributeTok{label.y =} \ConstantTok{Inf}\NormalTok{, }\AttributeTok{hjust =} \FloatTok{1.1}\NormalTok{, }\AttributeTok{vjust =} \DecValTok{2}\NormalTok{, }\AttributeTok{size =} \DecValTok{5}
\NormalTok{) }\SpecialCharTok{+}
  \FunctionTok{labs}\NormalTok{(}
    \AttributeTok{title =} \StringTok{"Gráfico de Dispersão c/Reta Regressão p/Z{-}scores"}\NormalTok{,}
    \AttributeTok{subtitle =} \StringTok{"Período: 2006 a 2019 (n = 14 obs.)"}\NormalTok{,}
    \AttributeTok{x =} \StringTok{"Z{-}PgtoGO\_MPE (Z{-}score da despesa pública GO c/MPE)"}\NormalTok{,}
    \AttributeTok{y =} \StringTok{"Z{-}PIB (Z{-}score do PIB{-}GO)"}
\NormalTok{  ) }\SpecialCharTok{+}
  \FunctionTok{xlim}\NormalTok{(}\SpecialCharTok{{-}}\DecValTok{2}\NormalTok{, }\DecValTok{2}\NormalTok{) }\SpecialCharTok{+}
  \FunctionTok{ylim}\NormalTok{(}\SpecialCharTok{{-}}\DecValTok{2}\NormalTok{, }\DecValTok{2}\NormalTok{) }\SpecialCharTok{+}
  \FunctionTok{theme\_minimal}\NormalTok{(}\AttributeTok{base\_size =} \DecValTok{14}\NormalTok{)         }\CommentTok{\# tema visual limpo e fonte maior}
\InformationTok{\textasciigrave{}\textasciigrave{}\textasciigrave{}}
\end{Highlighting}
\end{Shaded}

\begin{verbatim}
   Min. 1st Qu.  Median    Mean 3rd Qu.    Max. 
 -1.560  -0.855   0.126   0.000   0.825   1.409 
\end{verbatim}

\pandocbounded{\includegraphics[keepaspectratio]{cap5-pldr-modelos-dados_files/figure-pdf/unnamed-chunk-30-1.pdf}}

O resultado final é uma \ul{\textbf{correção moderada a forte}} (r =
0,51) entre Y = \texttt{PIB} e X = \texttt{PgtoGO\_MPE}.

Além disso essa \textbf{\emph{correlação \ul{nãoé estatisticamente
significativa}}} para um \emph{nível de significância} de 5\% (Erro tipo
I, alpha = 0,05 = 5\%), poi seu valor-P = 0,06 = 6,0\% é
\ul{\textbf{menor que}} 5,0\% = alpha.

\subsection{Conclusão}\label{conclusuxe3o}

Pelos resultados ancançados e pelos testes de significância da hipótese
nula (r = 0), pode-se afrimar que há \textbf{\emph{forte correlação}}
entre Y = \texttt{RAIS} e X = \texttt{PgtoGO\_MPE}, pois \textbf{r =
0,73} e seu correspondente \textbf{valor-P} = 0,0028 = 0,3\% é
\ul{\textbf{menor que}} 5,0\% = alpha.

Pela reta de regressão, pode-se \textbf{interpretar} que
\textbf{\emph{para cada 1 milhão de R\$ gastos anualmente com a despesa
pública do Estado de Goiás em contratos públicos com MPE's}} o
\ul{\textbf{\emph{total de empregos eleva-se}}}, em média, 0,00095
milhões = 0,95 mil = \textbf{\emph{950 empregos formais}}
(\texttt{RAIS}).

O que equivaleria a um \ul{\textbf{\emph{gasto médio marginal}}} de
\textbf{R\$1.052,63} com MPE's pelo Estado de Goiás para
\textbf{\emph{cada vaga de emprego formal alcançada}} no período
observado de \textbf{2006 até 2019}.

Evidência que corrobora a intenção contitucional e legal de fomentar as
MPE's, a partir da consideração de que são elas as principais
responsáveis pela geração de postos de trabalho no meio urbano (no meio
rural esse papel fica com a agricultura familiar).

\bookmarksetup{startatroot}

\chapter{AID - cap 13 - Modelagem de Relações
Contínuas}\label{sec-model-rel-contin}

\begin{quote}
13 Modelagem de Relações Contínuas

A maioria das pessoas conhece a ideia de correlação e, neste capítulo,
apresentaremos uma compreensão mais formal desse conceito normalmente
usado e incompreendido. (Poldrack, 2025 , cap. 13, p.~157).
\end{quote}

\section{Objetivos da Aprendizagem}\label{objetivos-da-aprendizagem-1}

\begin{quote}
▶\emph{Descrever} o \textbf{conceito} de \ul{\textbf{coeficiente de
correlação}} e \textbf{\emph{sua interpretação}}.

▶\emph{Calcular} a correlação entre \textbf{\emph{duas variáveis
contínuas}}.

▶\emph{Descrever} o \textbf{\emph{efeito}} de pontos de dados com
\emph{outliers} e como abordá-los.

▶\emph{Descrever} as \textbf{\emph{possíveis influências causais}} que
\emph{podem originar} uma \emph{correlação observada}. (Poldrack, 2025 ,
p.~157).
\end{quote}

\subsection{Carregar pacotes e conjunto de dados
.}\label{carregar-pacotes-e-conjunto-de-dados-.}

\begin{Shaded}
\begin{Highlighting}[numbers=left,,]
\InformationTok{\textasciigrave{}\textasciigrave{}\textasciigrave{}\{r\}}
\FunctionTok{library}\NormalTok{(tidyverse)}
\FunctionTok{library}\NormalTok{(ggplot2)}
\FunctionTok{library}\NormalTok{(fivethirtyeight)}
\FunctionTok{library}\NormalTok{(BayesFactor)}
\FunctionTok{library}\NormalTok{(bayestestR)}
\FunctionTok{library}\NormalTok{(cowplot)}
\FunctionTok{library}\NormalTok{(knitr)}
\FunctionTok{library}\NormalTok{(DiagrammeR)}
\FunctionTok{library}\NormalTok{(htmltools)}
\FunctionTok{library}\NormalTok{(webshot)}
\FunctionTok{theme\_set}\NormalTok{(}\FunctionTok{theme\_minimal}\NormalTok{(}\AttributeTok{base\_size =} \DecValTok{14}\NormalTok{))}

\FunctionTok{set.seed}\NormalTok{(}\DecValTok{123456}\NormalTok{) }\CommentTok{\# set random seed to exactly replicate results}

\CommentTok{\# load the NHANES data library}
\FunctionTok{library}\NormalTok{(NHANES)}

\CommentTok{\# drop duplicated IDs within the NHANES dataset}
\NormalTok{NHANES }\OtherTok{\textless{}{-}}
\NormalTok{  NHANES }\SpecialCharTok{\%\textgreater{}\%}
\NormalTok{  dplyr}\SpecialCharTok{::}\FunctionTok{distinct}\NormalTok{(ID,}\AttributeTok{.keep\_all=}\ConstantTok{TRUE}\NormalTok{)}

\NormalTok{NHANES\_adult }\OtherTok{\textless{}{-}}
\NormalTok{  NHANES }\SpecialCharTok{\%\textgreater{}\%}
  \FunctionTok{drop\_na}\NormalTok{(Weight) }\SpecialCharTok{\%\textgreater{}\%}
  \FunctionTok{subset}\NormalTok{(Age}\SpecialCharTok{\textgreater{}=}\DecValTok{18}\NormalTok{)}
\InformationTok{\textasciigrave{}\textasciigrave{}\textasciigrave{}}
\end{Highlighting}
\end{Shaded}

\section{Crimes de Ódio e Desigualdade de Renda: Um
Exemplo}\label{crimes-de-uxf3dio-e-desigualdade-de-renda-um-exemplo}

\begin{quote}
Em 2017, o site fivethirtyeight.com publicou uma matéria chamada
``\emph{Higher Rates of Hate Crimes Are Tied to Income Inequality}
{[}Taxas Mais Altas de Crimes de Ódio Estão Relacionadas à Desigualdade
de Renda{]}'', que explorava a \textbf{\emph{relação}} entre a
\textbf{\emph{prevalência de crimes de ódio}} \ul{\textbf{e}} a
\textbf{\emph{desigualdade de renda}} \emph{após a eleição presidencial
de 2016 nos Estados Unidos}.

A matéria apresentava uma análise de dados sobre crimes de ódio,
\emph{feita pelo} \textbf{FBI} e pelo \textbf{\emph{Southern Poverty Law
Center}}, a partir da qual relatava o seguinte:

\textbf{\emph{Constatamos que a desigualdade de renda foi o fator
determinante mais significativo dos crimes e incidentes de ódio}},
\emph{ponderados pela população dos Estados Unidos}. (Majumder, 2017)

. (Poldrack, 2025 , p.~157).
\end{quote}

Os dados para essa análise estão disponíveis como parte do pacote
\texttt{fivethirtyeight} do software estatístico R, o que facilita o
acesso.

A análise apresentada na matéria focava a relação entre a desigualdade
de renda (definida por uma medida chamada índice de Gini ou coeficiente
de Gini --- confira o apêndice deste capítulo para mais detalhes) e a
prevalência de crimes de ódio em cada estado.

Essa relação é mostrada na Figura 13.1.

\begin{Shaded}
\begin{Highlighting}[numbers=left,,]
\InformationTok{\textasciigrave{}\textasciigrave{}\textasciigrave{}\{r\}}
\NormalTok{hateCrimes }\OtherTok{\textless{}{-}}
\NormalTok{  hate\_crimes }\SpecialCharTok{\%\textgreater{}\%}
  \FunctionTok{mutate}\NormalTok{(}\AttributeTok{state\_abb =}\NormalTok{ state.abb[}\FunctionTok{match}\NormalTok{(state, state.name)]) }\SpecialCharTok{\%\textgreater{}\%}
  \FunctionTok{drop\_na}\NormalTok{(avg\_hatecrimes\_per\_100k\_fbi)}

\NormalTok{hateCrimes}\SpecialCharTok{$}\NormalTok{state\_abb[hateCrimes}\SpecialCharTok{$}\NormalTok{state}\SpecialCharTok{==}\StringTok{"District of Columbia"}\NormalTok{] }\OtherTok{=} \StringTok{\textquotesingle{}DC\textquotesingle{}}

\FunctionTok{ggplot}\NormalTok{(hateCrimes,}\FunctionTok{aes}\NormalTok{(gini\_index,avg\_hatecrimes\_per\_100k\_fbi,}\AttributeTok{label=}\NormalTok{state\_abb)) }\SpecialCharTok{+}
  \FunctionTok{geom\_point}\NormalTok{(}\AttributeTok{size =} \FloatTok{0.8}\NormalTok{, }\AttributeTok{color =} \StringTok{"blue"}\NormalTok{, }\AttributeTok{alpha=}\FloatTok{0.4}\NormalTok{) }\SpecialCharTok{+}
  \FunctionTok{geom\_text}\NormalTok{(}\FunctionTok{aes}\NormalTok{(}\AttributeTok{label=}\NormalTok{state\_abb), }\AttributeTok{hjust=}\DecValTok{0}\NormalTok{, }\AttributeTok{vjust=}\DecValTok{0}\NormalTok{, }\AttributeTok{size =} \DecValTok{2}\NormalTok{, }\CommentTok{\# tamanho do texto}
            \AttributeTok{position =} \FunctionTok{position\_jitter}\NormalTok{(}\AttributeTok{width =} \FloatTok{0.003}\NormalTok{, }\AttributeTok{height =} \FloatTok{0.003}\NormalTok{)) }\SpecialCharTok{+}
  \FunctionTok{theme}\NormalTok{(}\AttributeTok{plot.title =} \FunctionTok{element\_text}\NormalTok{(}\AttributeTok{size =} \DecValTok{20}\NormalTok{, }\AttributeTok{face =} \StringTok{"bold"}\NormalTok{)) }\SpecialCharTok{+}
  \FunctionTok{xlab}\NormalTok{(}\StringTok{\textquotesingle{}Gini index\textquotesingle{}}\NormalTok{) }\SpecialCharTok{+}
  \FunctionTok{ylab}\NormalTok{(}\StringTok{\textquotesingle{}Avg hate crimes per 100K population (FBI)\textquotesingle{}}\NormalTok{) }\SpecialCharTok{+}
  \FunctionTok{labs}\NormalTok{(}
    \AttributeTok{title =} \StringTok{"Gráfico de Dispersão"}\NormalTok{,}
    \AttributeTok{subtitle =} \StringTok{"Ano: 2017 (n = 51 obs.)"}\NormalTok{,}
    \AttributeTok{x =} \StringTok{"Gini index"}\NormalTok{,}
    \AttributeTok{y =} \StringTok{"Avg hate crimes per 100K population (FBI)"}\NormalTok{,}
    \AttributeTok{caption  =} \StringTok{"Gráfico do FBI das taxas médias de crimes de ódio por 100 mil habitantes [Average}\SpecialCharTok{\textbackslash{}n}\StringTok{hate crimes per 100K population (FBI)] em relação ao índice de Gini [Gini index].}\SpecialCharTok{\textbackslash{}n}\StringTok{Fonte: Poldrack(2025, p. 158, fig. 13.1)"}
\NormalTok{  ) }\SpecialCharTok{+}
  \FunctionTok{theme}\NormalTok{(}\AttributeTok{plot.margin =} \FunctionTok{unit}\NormalTok{(}\FunctionTok{c}\NormalTok{(}\DecValTok{1}\NormalTok{,}\DecValTok{1}\NormalTok{,}\DecValTok{1}\NormalTok{,}\DecValTok{1}\NormalTok{), }\StringTok{"cm"}\NormalTok{)) }\SpecialCharTok{+}
  \FunctionTok{xlim}\NormalTok{(}\FloatTok{0.40}\NormalTok{, }\FloatTok{0.55}\NormalTok{)}
\InformationTok{\textasciigrave{}\textasciigrave{}\textasciigrave{}}
\end{Highlighting}
\end{Shaded}

\pandocbounded{\includegraphics[keepaspectratio]{cap13-pldr-model-rel-contin_files/figure-pdf/unnamed-chunk-2-1.pdf}}

Analisando os dados, parece \textbf{\emph{possível}} existir
\textbf{\emph{uma relação positiva entre as duas variáveis}}.

Como podemos quantificá-la?

\section{Covariância e
Correlação}\label{covariuxe2ncia-e-correlauxe7uxe3o}

Uma maneira de quantificar a relação entre duas variáveis
\hspace{0pt}\hspace{0pt}é a \ul{\textbf{covariância}}.

Lembre-se de que a \ul{\textbf{variância}} \ul{\textbf{\emph{amostral}}}
para uma única variável é calculada como a diferença média quadrática
entre cada ponto de dados e a média, dividida pelo tamanho amostral
menos 1:

\[
s^2 = \frac{\sum_{i=1}^n (x_i - \bar{x})^2}{n-1}
\]

Isso nos diz o \emph{quão distante cada observação está da}
\emph{média}, em \emph{unidades quadradas}.

A \ul{\textbf{covariância}} nos diz \emph{\textbf{se existe uma relação
entre os desvios} de \textbf{duas variáveis
\hspace{0pt}\hspace{0pt}diferentes} ao longo das observações}. Ela é
definida assim:

\[
covariância =  \frac{\sum_{i=1}^n (x_i - \bar{x}) (y_i - \bar{y}) }{n-1}
\]

Esse valor estará \textbf{\emph{longe de zero quando x e y forem ambos
altamente desviantes da média}}; \textbf{\emph{se forem desviantes na
mesma direção}}, a covariância será \textbf{\emph{positiva}}, enquanto
se forem desviantes \textbf{\emph{em direções opostas}}, a covariância
\textbf{\emph{será negativa}}.

Vejamos primeiro um exemplo prático.

Os dados são mostrados na tabela, juntamente com seus desvios
individuais da média e seus produtos cruzados.

\begin{Shaded}
\begin{Highlighting}[numbers=left,,]
\InformationTok{\textasciigrave{}\textasciigrave{}\textasciigrave{}\{r\}}
\CommentTok{\# create data for toy example of covariance}
\FunctionTok{set.seed}\NormalTok{(}\DecValTok{123456789}\NormalTok{)}
\NormalTok{df }\OtherTok{\textless{}{-}}
  \FunctionTok{tibble}\NormalTok{(}\AttributeTok{x =} \FunctionTok{c}\NormalTok{(}\DecValTok{3}\NormalTok{, }\DecValTok{5}\NormalTok{, }\DecValTok{8}\NormalTok{, }\DecValTok{10}\NormalTok{, }\DecValTok{12}\NormalTok{)) }\SpecialCharTok{\%\textgreater{}\%}
  \FunctionTok{mutate}\NormalTok{(}\AttributeTok{y =}\NormalTok{ x }\SpecialCharTok{+} \FunctionTok{round}\NormalTok{(}\FunctionTok{rnorm}\NormalTok{(}\AttributeTok{n =} \DecValTok{5}\NormalTok{, }\AttributeTok{mean =} \DecValTok{0}\NormalTok{, }\AttributeTok{sd =} \DecValTok{2}\NormalTok{))) }\SpecialCharTok{\%\textgreater{}\%}
  \FunctionTok{mutate}\NormalTok{(}
    \AttributeTok{x\_dev =}\NormalTok{ x }\SpecialCharTok{{-}} \FunctionTok{mean}\NormalTok{(x),}
    \AttributeTok{y\_dev =}\NormalTok{ y }\SpecialCharTok{{-}} \FunctionTok{mean}\NormalTok{(y)}
\NormalTok{  ) }\SpecialCharTok{\%\textgreater{}\%}
  \FunctionTok{mutate}\NormalTok{(}\AttributeTok{crossproduct =}\NormalTok{ x\_dev }\SpecialCharTok{*}\NormalTok{ y\_dev)}

\NormalTok{covXY }\OtherTok{\textless{}{-}} \FunctionTok{sum}\NormalTok{(df}\SpecialCharTok{$}\NormalTok{crossproduct) }\SpecialCharTok{/}\NormalTok{ (}\FunctionTok{nrow}\NormalTok{(df) }\SpecialCharTok{{-}} \DecValTok{1}\NormalTok{)}
\NormalTok{corXY }\OtherTok{\textless{}{-}} \FunctionTok{sum}\NormalTok{(df}\SpecialCharTok{$}\NormalTok{crossproduct) }\SpecialCharTok{/}\NormalTok{ ( (}\FunctionTok{nrow}\NormalTok{(df) }\SpecialCharTok{{-}} \DecValTok{1}\NormalTok{) }\SpecialCharTok{*} \FunctionTok{sd}\NormalTok{(df}\SpecialCharTok{$}\NormalTok{x) }\SpecialCharTok{*} \FunctionTok{sd}\NormalTok{(df}\SpecialCharTok{$}\NormalTok{y) )}

\CommentTok{\# calcular Z{-}escores: Z\_x e Z\_y}
\NormalTok{df }\OtherTok{\textless{}{-}}\NormalTok{ df }\SpecialCharTok{\%\textgreater{}\%} 
  \FunctionTok{mutate}\NormalTok{(}
    \AttributeTok{z\_x =}\NormalTok{ x\_dev }\SpecialCharTok{/} \FunctionTok{sd}\NormalTok{(x),}
    \AttributeTok{z\_y =}\NormalTok{ y\_dev }\SpecialCharTok{/} \FunctionTok{sd}\NormalTok{(y)}
\NormalTok{  ) }\SpecialCharTok{\%\textgreater{}\%}
  \FunctionTok{mutate}\NormalTok{(}\AttributeTok{z\_xy =}\NormalTok{ z\_x }\SpecialCharTok{*}\NormalTok{ z\_y)}

\NormalTok{corZxy }\OtherTok{\textless{}{-}} \FunctionTok{sum}\NormalTok{(df}\SpecialCharTok{$}\NormalTok{z\_xy) }\SpecialCharTok{/}\NormalTok{ ( }\FunctionTok{nrow}\NormalTok{(df) }\SpecialCharTok{{-}} \DecValTok{1}\NormalTok{ )}

\FunctionTok{cat}\NormalTok{(}\StringTok{"Dados para o exemplo ilustrativo de Covariância:}\SpecialCharTok{\textbackslash{}n}\StringTok{"}\NormalTok{)}
\FunctionTok{cat}\NormalTok{(}\StringTok{"Covariância covXY  = "}\NormalTok{, covXY , }\StringTok{"}\SpecialCharTok{\textbackslash{}n}\StringTok{"}\NormalTok{)}
\FunctionTok{cat}\NormalTok{(}\StringTok{"Correlação  corXY  = "}\NormalTok{, corXY , }\StringTok{"}\SpecialCharTok{\textbackslash{}n}\StringTok{"}\NormalTok{)}
\FunctionTok{cat}\NormalTok{(}\StringTok{"Correlação  corZxy = "}\NormalTok{, corZxy, }\StringTok{"}\SpecialCharTok{\textbackslash{}n}\StringTok{"}\NormalTok{)}
\NormalTok{df}
\InformationTok{\textasciigrave{}\textasciigrave{}\textasciigrave{}}
\end{Highlighting}
\end{Shaded}

\begin{verbatim}
Dados para o exemplo ilustrativo de Covariância:
Covariância covXY  =  10.1 
Correlação  corXY  =  0.889221 
Correlação  corZxy =  0.889221 
# A tibble: 5 x 8
      x     y  x_dev  y_dev crossproduct    z_x    z_y   z_xy
  <dbl> <dbl>  <dbl>  <dbl>        <dbl>  <dbl>  <dbl>  <dbl>
1     3     4 -4.6   -4.2          19.3  -1.26  -1.35  1.70  
2     5     6 -2.6   -2.2           5.72 -0.713 -0.706 0.504 
3     8    11  0.400  2.8           1.12  0.110  0.899 0.0986
4    10     9  2.4    0.800         1.92  0.658  0.257 0.169 
5    12    11  4.4    2.8          12.3   1.21   0.899 1.08  
\end{verbatim}

A \ul{\textbf{covariância}} é simplesmente a \textbf{\emph{média dos
produtos cruzados}}, nesse caso é de 17,05 {[}no livro texto{]}.

No nosso exemplo acima a \textbf{covariância} é: \textbf{10,1}.

Normalmente, \textbf{\emph{não a usamos para descrever as relações}}
entre as variáveis, porque ela varia conforme o nível geral de variância
nos dados.

Ao contrário, em geral, \textbf{\emph{usaríamos}} o
\ul{\textbf{coeficiente de correlação}}.

A \ul{\textbf{correlação}} (\emph{r}) é calculada
\textbf{\emph{escalando a covariância pelos desvios-padrão}} das
\textbf{\emph{duas variáveis}}:

\[
r=\frac{covariância}{s_x \cdot s_y} =  \frac{\sum_{i=1}^n (x_i - \bar{x}) (y_i - \bar{y}) }{(n-1)s_x \cdot s_y} = \frac{\sum_{i=1}^n Z_x \cdot Z_y}{(n-1)}
\]

No exemplo ilustrativo da Tabela 13.1, o seu valor é de 0,89 {[}no livro
texto{]}.

No nosso exemplo acima a \textbf{correlação} é: \textbf{0.889221}.
Obtendo o mesmo resultado por dois métodos distintos equivalentes: a)
dividindo a conavirância pelos produtos dos desvios padrão de x e de y;
b) pelo produto cruzado dos Z-escores de x e de y, dividido pelo número
de \emph{graus} \emph{de liberadade} da amostra (n-1).

O \ul{\textbf{coeficiente de correlação}} é útil porque varia entre −1 e
1, \emph{independentemente da natureza dos dados}, sendo
\emph{facilmente interpretável} --- na verdade, já o tínhamos visto
quando analisamos os tamanhos dos efeitos no Capítulo 10.

Conforme analisado, uma correlação de \emph{1 indica uma relação linear
perfeita}, uma correlação de \emph{−1 indica uma relação negativa
perfeita} e uma \emph{correlação de 0 indica nenhuma relação linear}.

\section{Reta de Regressão: Um
Exemplo}\label{reta-de-regressuxe3o-um-exemplo}

\begin{quote}
Por que algumas pessoas acham fácil permanecer magras? Seguindo o
processo de quatro passos, mostramos o relato de um estudo que clareia
um pouco o assunto de ganho de peso.

\texttt{ESTABELEÇA}: algumas pessoas não ganham peso, mesmo quando comem
muito. Talvez a agitação e outras ``atividades de não exercício'' (ANE)
expliquem por quê. De fato, algumas pessoas podem, espontaneamente,
aumentar a atividade de não exercício quando comem mais, reduzindo,
assim, a quantidade de peso que ganham com o excesso de comida. Para
investigar o efeito de ANE no ganho de peso, pesquisadores,
deliberadamente, superalimentaram 16 jovens adultos saudáveis durante
oito semanas. Mediram o ganho de gordura (em quilogramas) e, como
variável explicativa, mudanças no uso da energia (em calorias) em
atividades diferentes de exercício deliberado -- agitação, vida diária e
semelhantes. A mudança no uso da energia foi a energia medida no último
dia do período de oito semanas, menos o uso de energia medida no dia
antes do início da superalimentação. Eis os dados:1

As pessoas com os maiores aumentos em ANE tendem a ganhar menos gordura?

\texttt{PLANEJE}: faça um diagrama de dispersão dos dados e examine o
padrão. Se for linear, use a correlação para medir sua intensidade e
desenhe uma reta de regressão no diagrama para predizer o ganho de
gordura a partir de mudança na ANE.

\texttt{RESOLVA}: a Figura 5.1 é um diagrama de dispersão desses dados.
O gráfico mostra uma associação linear negativa ligeiramente forte, sem
valores atípicos. A correlação é r = --0,7786. A reta no gráfico é uma
reta de regressão para predição do ganho de gordura a partir de mudanças
na ANE.

\texttt{CONCLUA}: pessoas com maiores aumentos em ANE realmente ganham
menos gordura. Para acrescentarmos mais a essa conclusão, devemos
estudar retas de regressão com mais detalhe.

No entanto, já podemos usar a reta de regressão para predizer o ganho de
gordura a partir do valor de ANE. Suponha que a ANE de um indivíduo
cresça de 400 calorias quando ele se superalimenta. ``Suba e vire'' no
gráfico da Figura 5.1. A partir de 400 calorias no eixo x, suba até a
reta de regressão e, então, vá para o eixo y. O gráfico mostra que o
ganho de gordura predito é um pouco maior do que 2 quilogramas. (MOORE;
NOTZ; FLIGNER, 2023 , cap. 5, exemplo 5.1, p.~101)
\end{quote}

\subsection{carregar}\label{carregar}

\begin{Shaded}
\begin{Highlighting}[numbers=left,,]
\InformationTok{\textasciigrave{}\textasciigrave{}\textasciigrave{}\{r\}}
\FunctionTok{library}\NormalTok{(readr)}

\CommentTok{\# Importar como tibble o arquivo de dentro da pasta chamada: dat/csv.}
\NormalTok{gangord }\OtherTok{\textless{}{-}}\NormalTok{ readr}\SpecialCharTok{::}\FunctionTok{read\_csv}\NormalTok{(}\AttributeTok{file   =} \StringTok{"dat/csv/eg05{-}01fatgain.csv"}\NormalTok{,}
                           \CommentTok{\# delim  = ",",}
                           \AttributeTok{quote  =} \StringTok{"}\SpecialCharTok{\textbackslash{}"}\StringTok{"}\NormalTok{,}
                           \AttributeTok{locale =} \FunctionTok{locale}\NormalTok{(}
                             \AttributeTok{decimal\_mark =} \StringTok{"."}\NormalTok{,}
                             \AttributeTok{encoding     =} \StringTok{"UTF{-}8"}
\NormalTok{                             )}
\NormalTok{                           )}

\CommentTok{\# cat {-} Concatenate And Print}
\FunctionTok{cat}\NormalTok{(}\StringTok{"}\SpecialCharTok{\textbackslash{}n}\StringTok{"}\NormalTok{) }\CommentTok{\# imprime no console (saída) uma linha em branco}
\FunctionTok{cat}\NormalTok{(}\StringTok{"Estrutura do objeto R denominado gangord:}\SpecialCharTok{\textbackslash{}n}\StringTok{"}\NormalTok{)}
\FunctionTok{str}\NormalTok{(gangord)}

\FunctionTok{cat}\NormalTok{(}\StringTok{"}\SpecialCharTok{\textbackslash{}n}\StringTok{"}\NormalTok{)}
\FunctionTok{cat}\NormalTok{(}\StringTok{"Nomes das 2 colunas do objeto gangord:}\SpecialCharTok{\textbackslash{}n}\StringTok{"}\NormalTok{)}
\FunctionTok{names}\NormalTok{(gangord)}
\CommentTok{\# [1] "NEA" "Fat"}
\CommentTok{\# NEA: NonExercise Activit}
\CommentTok{\# Fat: Fat gain}

\NormalTok{gangord }\CommentTok{\# tibble: 6 obs × 2 colunas (variáveis)}

\CommentTok{\# renomear variáveis: português}
\FunctionTok{names}\NormalTok{(gangord) }\OtherTok{\textless{}{-}} \FunctionTok{c}\NormalTok{(}\StringTok{"ANE"}\NormalTok{,    }\CommentTok{\# Atividade de Não Exercício (cal)}
                    \StringTok{"gangor"}\NormalTok{) }\CommentTok{\# Ganho de gordura (kg)}

\NormalTok{gangord }\CommentTok{\# tibble: 6 obs × 2 colunas (variáveis)}
\InformationTok{\textasciigrave{}\textasciigrave{}\textasciigrave{}}
\end{Highlighting}
\end{Shaded}

\begin{verbatim}

Estrutura do objeto R denominado gangord:
spc_tbl_ [16 x 2] (S3: spec_tbl_df/tbl_df/tbl/data.frame)
 $ NEA: num [1:16] -94 -57 -29 135 143 151 245 355 392 473 ...
 $ Fat: num [1:16] 4.2 3 3.7 2.7 3.2 3.6 2.4 1.3 3.8 1.7 ...
 - attr(*, "spec")=
  .. cols(
  ..   NEA = col_double(),
  ..   Fat = col_double()
  .. )
 - attr(*, "problems")=<externalptr> 

Nomes das 2 colunas do objeto gangord:
[1] "NEA" "Fat"
# A tibble: 16 x 2
     NEA   Fat
   <dbl> <dbl>
 1   -94   4.2
 2   -57   3  
 3   -29   3.7
 4   135   2.7
 5   143   3.2
 6   151   3.6
 7   245   2.4
 8   355   1.3
 9   392   3.8
10   473   1.7
11   486   1.6
12   535   2.2
13   571   1  
14   580   0.4
15   620   2.3
16   690   1.1
# A tibble: 16 x 2
     ANE gangor
   <dbl>  <dbl>
 1   -94    4.2
 2   -57    3  
 3   -29    3.7
 4   135    2.7
 5   143    3.2
 6   151    3.6
 7   245    2.4
 8   355    1.3
 9   392    3.8
10   473    1.7
11   486    1.6
12   535    2.2
13   571    1  
14   580    0.4
15   620    2.3
16   690    1.1
\end{verbatim}

\subsection{gráfico dispersão c/reta
regressão}\label{gruxe1fico-dispersuxe3o-creta-regressuxe3o}

\begin{Shaded}
\begin{Highlighting}[numbers=left,,]
\InformationTok{\textasciigrave{}\textasciigrave{}\textasciigrave{}\{r\}}
\FunctionTok{library}\NormalTok{(ggpubr)}

\CommentTok{\# dados do data frame chamado gangord}

\FunctionTok{summary}\NormalTok{(gangord}\SpecialCharTok{$}\NormalTok{ANE)}
\FunctionTok{summary}\NormalTok{(gangord}\SpecialCharTok{$}\NormalTok{gangor)}

\CommentTok{\# Gráfico de dispersão com reta de regressão linear, r e R²}
\FunctionTok{ggplot}\NormalTok{(gangord, }\FunctionTok{aes}\NormalTok{(}\AttributeTok{x =}\NormalTok{ ANE,}
                    \AttributeTok{y =}\NormalTok{ gangor)) }\SpecialCharTok{+}
  \FunctionTok{geom\_point}\NormalTok{(}\AttributeTok{color =} \StringTok{"blue"}\NormalTok{, }\AttributeTok{alpha =} \FloatTok{0.7}\NormalTok{) }\SpecialCharTok{+}           \CommentTok{\# pontos de dispersão}
  \FunctionTok{geom\_smooth}\NormalTok{(}\AttributeTok{method =} \StringTok{"lm"}\NormalTok{, }\AttributeTok{se =} \ConstantTok{TRUE}\NormalTok{, }\AttributeTok{color =} \StringTok{"red"}\NormalTok{) }\SpecialCharTok{+} \CommentTok{\# reta de regressão linear com intervalo de confiança}
  \FunctionTok{stat\_cor}\NormalTok{(}
  \FunctionTok{aes}\NormalTok{(}\AttributeTok{label =} \FunctionTok{paste}\NormalTok{(..r.label.., ..p.label.., }\AttributeTok{sep =} \StringTok{"\textasciitilde{}\textasciigrave{},\textasciigrave{}\textasciitilde{}"}\NormalTok{)),}
  \AttributeTok{label.x =} \ConstantTok{Inf}\NormalTok{, }\AttributeTok{label.y =} \SpecialCharTok{{-}}\ConstantTok{Inf}\NormalTok{, }\AttributeTok{hjust =} \FloatTok{1.1}\NormalTok{, }\AttributeTok{vjust =} \SpecialCharTok{{-}}\FloatTok{0.5}\NormalTok{, }\AttributeTok{size =} \DecValTok{5}
\NormalTok{) }\SpecialCharTok{+}
\FunctionTok{stat\_regline\_equation}\NormalTok{(}
  \FunctionTok{aes}\NormalTok{(}\AttributeTok{label =} \FunctionTok{paste}\NormalTok{(..eq.label.., ..rr.label.., }\AttributeTok{sep =} \StringTok{"\textasciitilde{}\textasciitilde{}\textasciitilde{}"}\NormalTok{)),}
  \AttributeTok{label.x =} \ConstantTok{Inf}\NormalTok{, }\AttributeTok{label.y =} \ConstantTok{Inf}\NormalTok{, }\AttributeTok{hjust =} \FloatTok{1.1}\NormalTok{, }\AttributeTok{vjust =} \DecValTok{2}\NormalTok{, }\AttributeTok{size =} \DecValTok{5}
\NormalTok{) }\SpecialCharTok{+}
  \FunctionTok{labs}\NormalTok{(}
    \AttributeTok{title =} \StringTok{"Gráfico de Dispersão c/Reta Regressão"}\NormalTok{,}
    \AttributeTok{subtitle =} \StringTok{"Período: 16 jovens por 8 semanas (n = 16 obs.)"}\NormalTok{,}
    \AttributeTok{x =} \StringTok{"mudança na ANE (cal.)"}\NormalTok{,}
    \AttributeTok{y =} \StringTok{"ganho de gordura (Kg)"}
\NormalTok{  ) }\SpecialCharTok{+}
  \FunctionTok{ylim}\NormalTok{(}\SpecialCharTok{{-}}\DecValTok{1}\NormalTok{, }\DecValTok{6}\NormalTok{) }\SpecialCharTok{+}
  \FunctionTok{xlim}\NormalTok{(}\SpecialCharTok{{-}}\DecValTok{200}\NormalTok{, }\DecValTok{1000}\NormalTok{) }\SpecialCharTok{+}
  \FunctionTok{theme\_minimal}\NormalTok{(}\AttributeTok{base\_size =} \DecValTok{14}\NormalTok{)       }\CommentTok{\# tema visual limpo e fonte maior}
\InformationTok{\textasciigrave{}\textasciigrave{}\textasciigrave{}}
\end{Highlighting}
\end{Shaded}

\begin{verbatim}
   Min. 1st Qu.  Median    Mean 3rd Qu.    Max. 
  -94.0   141.0   373.5   324.8   544.0   690.0 
   Min. 1st Qu.  Median    Mean 3rd Qu.    Max. 
  0.400   1.525   2.350   2.388   3.300   4.200 
\end{verbatim}

\pandocbounded{\includegraphics[keepaspectratio]{cap13-pldr-model-rel-contin_files/figure-pdf/unnamed-chunk-5-1.pdf}}

\subsection{predição}\label{prediuxe7uxe3o}

Recolocando \textbf{\emph{uma questão de pesquisa subsequente}}:

\textbf{Quanta ANE (cal) é necessária para reduzir o granho de peso a
zero mesmo sob uma dieta hipercalórica?}

\begin{Shaded}
\begin{Highlighting}[numbers=left,,]
\InformationTok{\textasciigrave{}\textasciigrave{}\textasciigrave{}\{r\}}
\FunctionTok{library}\NormalTok{(ggpubr)}
\FunctionTok{library}\NormalTok{(ggplot2)}

\CommentTok{\# ajuste do modelo}
\NormalTok{lm\_mod }\OtherTok{\textless{}{-}} \FunctionTok{lm}\NormalTok{(gangor }\SpecialCharTok{\textasciitilde{}}\NormalTok{ ANE, }\AttributeTok{data =}\NormalTok{ gangord)}

\CommentTok{\# intervalos}
\NormalTok{x\_obs }\OtherTok{\textless{}{-}} \FunctionTok{range}\NormalTok{(gangord}\SpecialCharTok{$}\NormalTok{ANE, }\AttributeTok{na.rm =} \ConstantTok{TRUE}\NormalTok{)}
\NormalTok{x\_plot }\OtherTok{\textless{}{-}} \FunctionTok{c}\NormalTok{(}\SpecialCharTok{{-}}\DecValTok{200}\NormalTok{, }\DecValTok{1050}\NormalTok{)  }\CommentTok{\# limites desejados na visualização}

\CommentTok{\# sequências para cada segmento}
\NormalTok{left\_seq }\OtherTok{\textless{}{-}} \FunctionTok{seq}\NormalTok{(x\_plot[}\DecValTok{1}\NormalTok{], x\_obs[}\DecValTok{1}\NormalTok{], }\AttributeTok{length.out =} \DecValTok{100}\NormalTok{)}
\NormalTok{mid\_seq  }\OtherTok{\textless{}{-}} \FunctionTok{seq}\NormalTok{(x\_obs[}\DecValTok{1}\NormalTok{], x\_obs[}\DecValTok{2}\NormalTok{], }\AttributeTok{length.out =} \DecValTok{200}\NormalTok{)}
\NormalTok{right\_seq}\OtherTok{\textless{}{-}} \FunctionTok{seq}\NormalTok{(x\_obs[}\DecValTok{2}\NormalTok{], x\_plot[}\DecValTok{2}\NormalTok{], }\AttributeTok{length.out =} \DecValTok{100}\NormalTok{)}

\CommentTok{\# predições}
\NormalTok{left\_df  }\OtherTok{\textless{}{-}} \FunctionTok{data.frame}\NormalTok{(}\AttributeTok{ANE =}\NormalTok{ left\_seq,  }\AttributeTok{fit =} \FunctionTok{predict}\NormalTok{(lm\_mod, }\AttributeTok{newdata =} \FunctionTok{data.frame}\NormalTok{(}\AttributeTok{ANE =}\NormalTok{ left\_seq)))}
\NormalTok{mid\_pred  }\OtherTok{\textless{}{-}} \FunctionTok{predict}\NormalTok{(lm\_mod, }\AttributeTok{newdata =} \FunctionTok{data.frame}\NormalTok{(}\AttributeTok{ANE =}\NormalTok{ mid\_seq), }\AttributeTok{interval =} \StringTok{"confidence"}\NormalTok{, }\AttributeTok{level =} \FloatTok{0.95}\NormalTok{)}
\NormalTok{mid\_df   }\OtherTok{\textless{}{-}} \FunctionTok{data.frame}\NormalTok{(}\AttributeTok{ANE =}\NormalTok{ mid\_seq, }\AttributeTok{fit =}\NormalTok{ mid\_pred[, }\StringTok{"fit"}\NormalTok{], }\AttributeTok{lwr =}\NormalTok{ mid\_pred[, }\StringTok{"lwr"}\NormalTok{], }\AttributeTok{upr =}\NormalTok{ mid\_pred[, }\StringTok{"upr"}\NormalTok{])}
\NormalTok{right\_df }\OtherTok{\textless{}{-}} \FunctionTok{data.frame}\NormalTok{(}\AttributeTok{ANE =}\NormalTok{ right\_seq, }\AttributeTok{fit =} \FunctionTok{predict}\NormalTok{(lm\_mod, }\AttributeTok{newdata =} \FunctionTok{data.frame}\NormalTok{(}\AttributeTok{ANE =}\NormalTok{ right\_seq)))}

\CommentTok{\# plot}
\FunctionTok{ggplot}\NormalTok{(gangord, }\FunctionTok{aes}\NormalTok{(}\AttributeTok{x =}\NormalTok{ ANE, }\AttributeTok{y =}\NormalTok{ gangor)) }\SpecialCharTok{+}
  \FunctionTok{geom\_point}\NormalTok{(}\AttributeTok{color =} \StringTok{"blue"}\NormalTok{, }\AttributeTok{alpha =} \FloatTok{0.7}\NormalTok{) }\SpecialCharTok{+}
  \CommentTok{\# IC apenas no trecho observado}
  \FunctionTok{geom\_ribbon}\NormalTok{(}\AttributeTok{data =}\NormalTok{ mid\_df, }\FunctionTok{aes}\NormalTok{(}\AttributeTok{x =}\NormalTok{ ANE, }\AttributeTok{ymin =}\NormalTok{ lwr, }\AttributeTok{ymax =}\NormalTok{ upr), }\AttributeTok{fill =} \StringTok{"red"}\NormalTok{, }\AttributeTok{alpha =} \FloatTok{0.15}\NormalTok{, }\AttributeTok{inherit.aes =} \ConstantTok{FALSE}\NormalTok{) }\SpecialCharTok{+}
  \CommentTok{\# linha sólida no trecho observado}
  \FunctionTok{geom\_line}\NormalTok{(}\AttributeTok{data =}\NormalTok{ mid\_df, }\FunctionTok{aes}\NormalTok{(}\AttributeTok{x =}\NormalTok{ ANE, }\AttributeTok{y =}\NormalTok{ fit), }\AttributeTok{color =} \StringTok{"red"}\NormalTok{, }\AttributeTok{size =} \DecValTok{1}\NormalTok{) }\SpecialCharTok{+}
  \CommentTok{\# extensões tracejadas esquerda e direita}
  \FunctionTok{geom\_line}\NormalTok{(}\AttributeTok{data =}\NormalTok{ left\_df, }\FunctionTok{aes}\NormalTok{(}\AttributeTok{x =}\NormalTok{ ANE, }\AttributeTok{y =}\NormalTok{ fit), }\AttributeTok{color =} \StringTok{"red"}\NormalTok{, }\AttributeTok{linetype =} \StringTok{"dashed"}\NormalTok{, }\AttributeTok{size =} \DecValTok{1}\NormalTok{) }\SpecialCharTok{+}
  \FunctionTok{geom\_line}\NormalTok{(}\AttributeTok{data =}\NormalTok{ right\_df, }\FunctionTok{aes}\NormalTok{(}\AttributeTok{x =}\NormalTok{ ANE, }\AttributeTok{y =}\NormalTok{ fit), }\AttributeTok{color =} \StringTok{"red"}\NormalTok{, }\AttributeTok{linetype =} \StringTok{"dashed"}\NormalTok{, }\AttributeTok{size =} \DecValTok{1}\NormalTok{) }\SpecialCharTok{+}
  \CommentTok{\# estatísticas (r, p, equação) continuam funcionando}
  \FunctionTok{stat\_cor}\NormalTok{(}
    \FunctionTok{aes}\NormalTok{(}\AttributeTok{label =} \FunctionTok{paste}\NormalTok{(..r.label.., ..p.label.., }\AttributeTok{sep =} \StringTok{"\textasciitilde{}\textasciigrave{},\textasciigrave{}\textasciitilde{}"}\NormalTok{)),}
    \AttributeTok{label.x =} \ConstantTok{Inf}\NormalTok{, }\AttributeTok{label.y =} \SpecialCharTok{{-}}\ConstantTok{Inf}\NormalTok{, }\AttributeTok{hjust =} \FloatTok{1.1}\NormalTok{, }\AttributeTok{vjust =} \SpecialCharTok{{-}}\FloatTok{0.5}\NormalTok{, }\AttributeTok{size =} \DecValTok{5}
\NormalTok{  ) }\SpecialCharTok{+}
  \FunctionTok{stat\_regline\_equation}\NormalTok{(}
    \FunctionTok{aes}\NormalTok{(}\AttributeTok{label =} \FunctionTok{paste}\NormalTok{(..eq.label.., ..rr.label.., }\AttributeTok{sep =} \StringTok{"\textasciitilde{}\textasciitilde{}\textasciitilde{}"}\NormalTok{)),}
    \AttributeTok{label.x =} \ConstantTok{Inf}\NormalTok{, }\AttributeTok{label.y =} \ConstantTok{Inf}\NormalTok{, }\AttributeTok{hjust =} \FloatTok{1.1}\NormalTok{, }\AttributeTok{vjust =} \DecValTok{2}\NormalTok{, }\AttributeTok{size =} \DecValTok{5}
\NormalTok{  ) }\SpecialCharTok{+}
  \FunctionTok{labs}\NormalTok{(}
    \AttributeTok{title =} \StringTok{"Gráfico de Dispersão c/Reta Regressão extrapolada"}\NormalTok{,}
    \AttributeTok{subtitle =} \StringTok{"Período: 16 jovens por 8 semanas (n = 16 obs.)"}\NormalTok{,}
    \AttributeTok{x =} \StringTok{"mudança na ANE (cal.) [predição: (x=1029.4, y=0)]"}\NormalTok{,}
    \AttributeTok{y =} \StringTok{"ganho de gordura (Kg)"}
\NormalTok{  ) }\SpecialCharTok{+}
  \FunctionTok{coord\_cartesian}\NormalTok{(}\AttributeTok{xlim =}\NormalTok{ x\_plot, }\AttributeTok{ylim =} \FunctionTok{c}\NormalTok{(}\SpecialCharTok{{-}}\DecValTok{1}\NormalTok{, }\DecValTok{6}\NormalTok{), }\AttributeTok{expand =} \ConstantTok{FALSE}\NormalTok{) }\SpecialCharTok{+}
  \CommentTok{\# ponto adicional fixo em (1029.4, 0)}
  \FunctionTok{geom\_point}\NormalTok{(}
    \AttributeTok{data =} \FunctionTok{data.frame}\NormalTok{(}\AttributeTok{ANE =} \FloatTok{1029.4}\NormalTok{, }\AttributeTok{gangor =} \DecValTok{0}\NormalTok{),}
    \AttributeTok{mapping =} \FunctionTok{aes}\NormalTok{(}\AttributeTok{x =}\NormalTok{ ANE, }\AttributeTok{y =}\NormalTok{ gangor),}
    \AttributeTok{color =} \StringTok{"black"}\NormalTok{, }\AttributeTok{alpha =} \FloatTok{1.0}\NormalTok{, }\AttributeTok{size =} \DecValTok{3}\NormalTok{, }\AttributeTok{shape =} \DecValTok{3}\NormalTok{,}
    \AttributeTok{inherit.aes =} \ConstantTok{FALSE}
\NormalTok{  ) }\SpecialCharTok{+}
  \FunctionTok{theme\_minimal}\NormalTok{(}\AttributeTok{base\_size =} \DecValTok{14}\NormalTok{)}
\InformationTok{\textasciigrave{}\textasciigrave{}\textasciigrave{}}
\end{Highlighting}
\end{Shaded}

\pandocbounded{\includegraphics[keepaspectratio]{cap13-pldr-model-rel-contin_files/figure-pdf/unnamed-chunk-6-1.pdf}}

Observa-se, no gráfico acima, que o \ul{\textbf{intercepto}} da
\ul{\textbf{reta de regressão}} é: \textbf{+3,5 Kg}. ou seja, é a
\textbf{\emph{estimativa}} do \texttt{ganho\ de\ gordura}
\textbf{\emph{se}} o valor de \texttt{ANE} \emph{não muda durante as 2
semanas que a pessoa superalimenta-se.}

E que a \ul{\textbf{inclinação}} da \ul{\textbf{reta de regressão}} é:
\textbf{-0,0034} (um número puro ou sem unidade). Equivale à tangente do
ângulo que essa reta forma com o eixo x.

Isso significa que, quando a \textbf{variável explicativa X}
\emph{aumentar 1 unidade}, no caso, \textbf{\emph{aumentar}}
\textbf{+1,0 cal}, então a \textbf{variável resposta Y}, \emph{segundo}
a \emph{reta de regressão de mínimos quadrados}, irá
\textbf{\emph{diminuir, \ul{em média},}} \textbf{-0,0034 Kg} =
\textbf{-3,4 g}.

Ou seja, para \textbf{diminuir 1 kg} na variável
\texttt{ganho\ de\ gordura} sob dieta hipercalórica, é necessário
\textbf{aumentar} a \texttt{ANE}:

\[
-0.0034 x = -1.0 \text{ kg} \Rightarrow x = \frac{1.0}{0.0034} \Rightarrow x = +294.2 \text{ cal}
\]

\begin{tcolorbox}[enhanced jigsaw, arc=.35mm, opacitybacktitle=0.6, colframe=quarto-callout-warning-color-frame, titlerule=0mm, leftrule=.75mm, left=2mm, colbacktitle=quarto-callout-warning-color!10!white, breakable, toprule=.15mm, bottomtitle=1mm, opacityback=0, coltitle=black, title=\textcolor{quarto-callout-warning-color}{\faExclamationTriangle}\hspace{0.5em}{inclinação da rela de regressão}, rightrule=.15mm, bottomrule=.15mm, toptitle=1mm, colback=white]

\textbf{\emph{Não}} se pode dizer \textbf{\emph{quão importante}} é
\textbf{\emph{uma relação}} pelo \textbf{\emph{simples exame}} do
\textbf{\emph{tamanho}} da \textbf{\emph{inclinação}} da
\textbf{\emph{reta de regressão}}. (MOORE; NOTZ; FLIGNER, 2023 , cap. 5,
p.~102)

\end{tcolorbox}

Todavia a amostra de \texttt{n=\ 16} jovens é \emph{pequena}.

Portanto sujeita à maior \ul{\textbf{variabilidade amostral}}, aquela
\emph{variação} nos dados observada \emph{de amostra para amostra}.

Isso implica \textbf{\emph{uma reflexão}} sobre a
\textbf{\emph{predição}}.

Voltando à questão: \textbf{Quanta ANE (cal) é necessária para reduzir o
granho de peso a zero mesmo sob uma dieta hipercalórica?}

Então queremos descobrir: qual valor de \(x\) para que \(\hat{y}=0\)
(lê-se \emph{ypsilon chapeu})?

\[
-0.0034 x + 3.50 = 0.0 \text{ kg} \Rightarrow x = \frac{3.50}{0.0034} \Rightarrow x = +1029.4 \text{ cal}
\]

\textbf{O quanto podemos confiar na predição:} \(x=\) \textbf{1029,4
cal. ⟹} \(\hat{y}=\) \textbf{0,0 Kg} (lê-se \emph{ypsilon
chapeu})\textbf{?}

É preciso ficar atento ao fato de que, \textbf{\emph{quanto mais nos
afastamos do intervalo de dados coletados}}, \textbf{\emph{mais largo
será o intervalo de predição}} (para um Nível de Confiança de 95\%).
Esse \emph{intervalo} atinge sua menos expessura no seu \emph{ponto
coordenado médio}: \((\bar{x},\bar{y})\).

Ou seja, o \textbf{\emph{verdadeiro e desconhecido valor de x
(}}\texttt{ANE}\textbf{\emph{)}} poderá encontra-se, aproximadamente
(pela \emph{leitura do gráfico acima}), para o valor de y = 0
(pré-estabelecido), em \ul{\textbf{\emph{algum valor de x}}}
\ul{\textbf{\emph{pertencente}}} ao seguinte \textbf{intervalo de
predição (NC=95\%) para x: (750 cal, 1250 cal)}.

\subsection{Estatísticas de
Regressão}\label{estatuxedsticas-de-regressuxe3o}

\begin{Shaded}
\begin{Highlighting}[numbers=left,,]
\InformationTok{\textasciigrave{}\textasciigrave{}\textasciigrave{}\{r\}}
\CommentTok{\# Pacotes necessários}
\FunctionTok{library}\NormalTok{(broom)}
\FunctionTok{library}\NormalTok{(lmtest)   }\CommentTok{\# dwtest}
\FunctionTok{library}\NormalTok{(ggplot2)  }\CommentTok{\# só se for anotar no gráfico}

\CommentTok{\# Ajuste do modelo}
\NormalTok{lm\_mod }\OtherTok{\textless{}{-}} \FunctionTok{lm}\NormalTok{(gangor }\SpecialCharTok{\textasciitilde{}}\NormalTok{ ANE, }\AttributeTok{data =}\NormalTok{ gangord)}

\CommentTok{\# Resumo base}
\NormalTok{summary\_lm }\OtherTok{\textless{}{-}} \FunctionTok{summary}\NormalTok{(lm\_mod)}

\CommentTok{\# Tabela de coeficientes (estimate, std.error, t, p)}
\NormalTok{coef\_table }\OtherTok{\textless{}{-}}\NormalTok{ summary\_lm}\SpecialCharTok{$}\NormalTok{coefficients}

\CommentTok{\# R² e R² ajustado}
\NormalTok{r\_sq      }\OtherTok{\textless{}{-}}\NormalTok{ summary\_lm}\SpecialCharTok{$}\NormalTok{r.squared}
\NormalTok{adj\_r\_sq  }\OtherTok{\textless{}{-}}\NormalTok{ summary\_lm}\SpecialCharTok{$}\NormalTok{adj.r.squared}

\CommentTok{\# Estatística F e p{-}valor do teste F}
\NormalTok{fstat     }\OtherTok{\textless{}{-}}\NormalTok{ summary\_lm}\SpecialCharTok{$}\NormalTok{fstatistic}
\NormalTok{f\_pvalue  }\OtherTok{\textless{}{-}} \FunctionTok{pf}\NormalTok{(fstat[}\DecValTok{1}\NormalTok{], fstat[}\DecValTok{2}\NormalTok{], fstat[}\DecValTok{3}\NormalTok{], }\AttributeTok{lower.tail =} \ConstantTok{FALSE}\NormalTok{)}

\CommentTok{\# Erro padrão residual e RMSE}
\NormalTok{sigma\_hat }\OtherTok{\textless{}{-}}\NormalTok{ summary\_lm}\SpecialCharTok{$}\NormalTok{sigma}
\NormalTok{rmse      }\OtherTok{\textless{}{-}} \FunctionTok{sqrt}\NormalTok{(}\FunctionTok{mean}\NormalTok{(}\FunctionTok{residuals}\NormalTok{(lm\_mod)}\SpecialCharTok{\^{}}\DecValTok{2}\NormalTok{))}

\CommentTok{\# AIC / BIC}
\NormalTok{model\_aic }\OtherTok{\textless{}{-}} \FunctionTok{AIC}\NormalTok{(lm\_mod)}
\NormalTok{model\_bic }\OtherTok{\textless{}{-}} \FunctionTok{BIC}\NormalTok{(lm\_mod)}

\CommentTok{\# Intervalos de confiança dos coeficientes (95\%)}
\NormalTok{conf\_int }\OtherTok{\textless{}{-}} \FunctionTok{confint}\NormalTok{(lm\_mod, }\AttributeTok{level =} \FloatTok{0.95}\NormalTok{)}

\CommentTok{\# Durbin{-}Watson (autocorrelação dos resíduos)}
\NormalTok{dw }\OtherTok{\textless{}{-}} \FunctionTok{tryCatch}\NormalTok{(}\FunctionTok{dwtest}\NormalTok{(lm\_mod), }\AttributeTok{error =} \ControlFlowTok{function}\NormalTok{(e) }\ConstantTok{NULL}\NormalTok{)}

\CommentTok{\# Saídas com broom para fácil uso programático}
\NormalTok{tidy\_coefs }\OtherTok{\textless{}{-}}\NormalTok{ broom}\SpecialCharTok{::}\FunctionTok{tidy}\NormalTok{(lm\_mod)    }\CommentTok{\# coeficientes com estatísticas}
\NormalTok{glance\_mod }\OtherTok{\textless{}{-}}\NormalTok{ broom}\SpecialCharTok{::}\FunctionTok{glance}\NormalTok{(lm\_mod)  }\CommentTok{\# R², AIC, BIC, etc.}
\NormalTok{augment\_df }\OtherTok{\textless{}{-}}\NormalTok{ broom}\SpecialCharTok{::}\FunctionTok{augment}\NormalTok{(lm\_mod) }\CommentTok{\# observações, resíduos, fitted, .se.fit etc.}

\CommentTok{\# Previsões com intervalo de confiança para um grid (útil para plot)}
\NormalTok{x\_grid }\OtherTok{\textless{}{-}} \FunctionTok{seq}\NormalTok{(}\SpecialCharTok{{-}}\DecValTok{200}\NormalTok{, }\DecValTok{1000}\NormalTok{, }\AttributeTok{length.out =} \DecValTok{200}\NormalTok{)}
\NormalTok{pred\_df }\OtherTok{\textless{}{-}} \FunctionTok{predict}\NormalTok{(lm\_mod, }\AttributeTok{newdata =} \FunctionTok{data.frame}\NormalTok{(}\AttributeTok{ANE =}\NormalTok{ x\_grid), }\AttributeTok{interval =} \StringTok{"confidence"}\NormalTok{, }\AttributeTok{level =} \FloatTok{0.95}\NormalTok{)}
\NormalTok{pred\_df }\OtherTok{\textless{}{-}} \FunctionTok{data.frame}\NormalTok{(}\AttributeTok{ANE =}\NormalTok{ x\_grid, }\AttributeTok{fit =}\NormalTok{ pred\_df[, }\StringTok{"fit"}\NormalTok{], }\AttributeTok{lwr =}\NormalTok{ pred\_df[, }\StringTok{"lwr"}\NormalTok{], }\AttributeTok{upr =}\NormalTok{ pred\_df[, }\StringTok{"upr"}\NormalTok{])}

\CommentTok{\# Preparar etiquetas formatadas para anotar no ggplot}
\NormalTok{eq\_label }\OtherTok{\textless{}{-}} \FunctionTok{sprintf}\NormalTok{(}\StringTok{"y = \%.3f \%+.3f*x"}\NormalTok{, coef\_table[}\StringTok{"(Intercept)"}\NormalTok{, }\StringTok{"Estimate"}\NormalTok{], coef\_table[}\StringTok{"ANE"}\NormalTok{, }\StringTok{"Estimate"}\NormalTok{])}
\NormalTok{r\_label  }\OtherTok{\textless{}{-}} \FunctionTok{sprintf}\NormalTok{(}\StringTok{"R² = \%.3f"}\NormalTok{, r\_sq)}
\NormalTok{p\_label  }\OtherTok{\textless{}{-}} \FunctionTok{sprintf}\NormalTok{(}\StringTok{"p (slope) = \%.3g"}\NormalTok{, coef\_table[}\StringTok{"ANE"}\NormalTok{, }\StringTok{"Pr(\textgreater{}|t|)"}\NormalTok{])}

\CommentTok{\# Resultado: lista com principais objetos}
\NormalTok{resultados\_reg }\OtherTok{\textless{}{-}} \FunctionTok{list}\NormalTok{(}
  \AttributeTok{lm\_mod =}\NormalTok{ lm\_mod,}
  \AttributeTok{summary =}\NormalTok{ summary\_lm,}
  \AttributeTok{coef\_table =}\NormalTok{ coef\_table,}
  \AttributeTok{tidy =}\NormalTok{ tidy\_coefs,}
  \AttributeTok{glance =}\NormalTok{ glance\_mod,}
  \AttributeTok{augment =}\NormalTok{ augment\_df,}
  \AttributeTok{r\_squared =}\NormalTok{ r\_sq,}
  \AttributeTok{adj\_r\_squared =}\NormalTok{ adj\_r\_sq,}
  \AttributeTok{f\_statistic =}\NormalTok{ fstat,}
  \AttributeTok{f\_pvalue =}\NormalTok{ f\_pvalue,}
  \AttributeTok{sigma =}\NormalTok{ sigma\_hat,}
  \AttributeTok{rmse =}\NormalTok{ rmse,}
  \AttributeTok{aic =}\NormalTok{ model\_aic,}
  \AttributeTok{bic =}\NormalTok{ model\_bic,}
  \AttributeTok{confint =}\NormalTok{ conf\_int,}
  \AttributeTok{durbin\_watson =}\NormalTok{ dw,}
  \AttributeTok{prediction\_grid =}\NormalTok{ pred\_df,}
  \AttributeTok{labels =} \FunctionTok{list}\NormalTok{(}\AttributeTok{equation =}\NormalTok{ eq\_label, }\AttributeTok{r2 =}\NormalTok{ r\_label, }\AttributeTok{pvalue =}\NormalTok{ p\_label)}
\NormalTok{)}

\CommentTok{\# Exibir sumário conciso no console}
\FunctionTok{print}\NormalTok{(eq\_label)}
\FunctionTok{print}\NormalTok{(r\_label)}
\FunctionTok{print}\NormalTok{(p\_label)}
\FunctionTok{print}\NormalTok{(}\FunctionTok{sprintf}\NormalTok{(}\StringTok{"RMSE = \%.3f | AIC = \%.2f | BIC = \%.2f"}\NormalTok{, rmse, model\_aic, model\_bic))}
\FunctionTok{print}\NormalTok{(}\StringTok{"Um teste de hipótese para verificar autocorrelação dos resíduos:"}\NormalTok{)}
\ControlFlowTok{if}\NormalTok{ (}\SpecialCharTok{!}\FunctionTok{is.null}\NormalTok{(dw)) }\FunctionTok{print}\NormalTok{(dw)}
\InformationTok{\textasciigrave{}\textasciigrave{}\textasciigrave{}}
\end{Highlighting}
\end{Shaded}

\begin{verbatim}
[1] "y = 3.505 -0.003*x"
[1] "R² = 0.606"
[1] "p (slope) = 0.000381"
[1] "RMSE = 0.692 | AIC = 39.63 | BIC = 41.95"
[1] "Um teste de hipótese para verificar autocorrelação dos resíduos:"

    Durbin-Watson test

data:  lm_mod
DW = 2.7523, p-value = 0.9058
alternative hypothesis: true autocorrelation is greater than 0
\end{verbatim}

Pela estatística de teste de Durbin-Watson, que varia de zero (0,0) até
4,0, que haja autocorrelação dos resíduos do modelo.

Ou seja, não se pode rejeitar, para um erro tipo I de 5,0\%, a hipótese
nula de ausência de autocorrelação dos resíduos do modelo.

O que significa decidir pela ausência de correlação entre os resíduos, o
que uma evidência a favor do modelo linear que foi contruído para
representar a relação entre X e Y.

E que os resíduos, assim, aproxiam-se de uma distribuição Normal, que é
um pressuposto para obtenção da reta de regressão.

Pressuposto esse que foi verificado poe meio de um teste de hipótese no
presente caso.

\bookmarksetup{startatroot}

\chapter{AID - cap 16 - Estatística
Multivariada}\label{sec-stat-multivar}

\begin{quote}
A palavra multivariada se refere a análises que envolvem mais de uma
variável aleatória. Apesar de termos visto exemplos em que o modelo
incluía múltiplas variáveis (como na regressão linear), estávamos
especificamente interessados em explicar a variação de uma variável
dependente em termos de uma ou mais variáveis independentes, as quais,
em geral, são especificadas pelo responsável do experimento em vez de
serem especificadas por medidas. Em uma análise multivariada, tratamos
normalmente todas as variáveis como iguais e procuramos entender como se
relacionam entre si como um grupo. (Poldrack, 2025 , p.~201).
\end{quote}

\section{Objetivos da Aprendizagem}\label{objetivos-da-aprendizagem-2}

\begin{quote}
▶Descrever a diferença entre aprendizado supervisionado e não
supervisionado.

▶Usar técnicas de visualização, incluindo mapas de calor {[}heatmaps{]}
a fim de visualizar a estrutura de dados multivariados.

▶Entender o conceito de clusterização e como ele pode ser usado para
identificar a estrutura nos dados.

▶Entender o conceito de redução de dimensionalidade.

▶Descrever como a análise de componentes principais e a análise fatorial
podem ser usadas para executar a redução de dimensionalidade. (Poldrack,
2025 , p.~201).
\end{quote}

\section{Variedades de Análise
Multivariada}\label{variedades-de-anuxe1lise-multivariada}

\begin{quote}
Existem inúmeros tipos diferentes de análise multivariada, mas, neste
capítulo, nós nos concentraremos em duas abordagens principais.

Na primeira, podemos simplesmente querer compreender e visualizar a
estrutura que existe nos dados, ou seja, quais variáveis ou observações
estão relacionadas. Em geral, definimos relacionadas em termos de alguma
medida que indexa a distância entre os valores de diferentes variáveis.
Um método importante que se encaixa nessa categoria é conhecido como
clusterização, cujo intuito é encontrar clusters de variáveis ou
observações semelhantes entre variáveis.

Na segunda, podemos querer usar um grande número de variáveis e
reduzi-lo, de modo que retenhamos o máximo de informações possíveis.
Chamamos isso de redução de dimensionalidade, em que a dimensionalidade
se refere ao número de variáveis no conjunto de dados.

Analisaremos duas técnicas comumente usadas para redução de
dimensionalidade: análise de componentes principais e análise fatorial.
Com frequência, a clusterização e a redução de dimensionalidade são
consideradas modalidades de aprendizado não supervisionado,
diferentemente do aprendizado supervisionado, que caracteriza modelos
como a regressão linear, sobre a qual você já aprendeu. A razão para
considerarmos a regressão linear como ``supervisionada'' é que sabemos o
valor daquilo que estamos tentando predizer (ou seja, a variável
dependente) e estamos tentando encontrar o modelo que melhor prediz
esses valores. No aprendizado não supervisionado, não estamos tentando
predizer um valor específico; pelo contrário, estamos tentando descobrir
estruturas nos dados as quais possam ser úteis para entendermos o que
está acontecendo; isso, em geral, exige algumas suposições sobre o tipo
de estrutura que queremos encontrar.

Uma informação que você descobrirá neste capítulo é que, no aprendizado
supervisionado, apesar de geralmente existir uma resposta ``correta''
(uma vez que chegamos a um consenso para determinar o ``melhor'' modelo,
como a soma dos erros quadráticos), não raro, no aprendizado não
supervisionado, não existe uma resposta ``correta'' consensual.
Diferentes métodos de aprendizado não supervisionado podem fornecer
respostas totalmente distintas sobre os mesmos dados. Em geral, não é
possível, a princípio, determinar qual delas é ``correta'', pois isso
depende dos objetivos da análise e das suposições que estamos dispostos
a fazer sobre os processos ou sistemas os quais originam os dados.
Algumas pessoas ficam frustradas com isso, enquanto outras ficam
entusiasmadas; caberá a você descobrir em qual desses grupos se encaixa.
(Poldrack, 2025 , p.~201-202).
\end{quote}

\section{Dados Multivariados: Um
Exemplo}\label{dados-multivariados-um-exemplo}

\begin{quote}
Para exemplificar a análise multivariada, examinaremos um conjunto de
dados coletado pela minha equipe e publicado em Eisenberg \emph{et al}.,
2019 (Eisenberg et al., 2019). Ele é valioso por dois motivos: tem um
grande número de variáveis interessantes, coletadas a partir de uma
quantidade relativamente alta de indivíduos, e está disponível
gratuitamente e online, possibilitando que você possa explorá-lo ainda
mais.

Realizamos este estudo porque estávamos interessados em compreender como
diversos aspectos diferentes da função psicológica estão relacionados,
com foco específico em medidas relacionadas à psicologia de autocontrole
e conceitos correlatos. Os participantes realizaram uma bateria de
testes cognitivos e questionários ao longo de uma semana, totalizando
dez horas de experimentos. Nesse primeiro exemplo, focaremos as
variáveis relacionadas a dois aspectos específicos do autocontrole; no
Capítulo 17, veremos outra análise desses dados.

A \textbf{\emph{inibição de resposta}} é definida como a habilidade de
interromper rapidamente uma ação e, nesse estudo, foi medida por meio de
um conjunto de tarefas conhecidas como \textbf{\emph{tarefas de sinal de
parada}} {[}\emph{stop-signal tasks}{]}. A variável de interesse para
elas é uma estimativa de quanto tempo os indivíduos levam para se
pararem, conhecida como o \textbf{\emph{tempo de reação do sinal de
parada}} (SSRT {[}\emph{stop-signal reaction time}{]}).

No conjunto de dados, existem quatro medidas diferentes de SSRT. A
\textbf{\emph{impulsividade}} é definida como a tendência de tomar
decisões por impulso, sem considerar as consequências potenciais e os
objetivos de longo prazo.

O estudo inclui uma série de questionários diferentes que medem a
impulsividade, mas nosso foco será o questionário UPPS-P que avalia
cinco facetas diferentes da impulsividade.

Após o cálculo dos escores para cada um dos 522 participantes do estudo
de Eisenberg, obtemos 9 números para cada indivíduo. Tratamos cada uma
dessas variáveis como uma dimensão do conjunto de dados; embora os dados
multivariados possam, às vezes, ter milhares ou até milhões de
dimensões, é útil observar primeiro como os métodos funcionam com um
número reduzido de dimensões. (Poldrack, 2025 , p.~202-203)
\end{quote}

\section{Visualizando Dados
Multivariados}\label{visualizando-dados-multivariados}

\begin{quote}
Um dos principais desafios dos dados multivariados é que o olho e o
cérebro humanos simplesmente não têm os recursos necessários para
visualizar dados com mais de três dimensões.

Podemos usar diversas ferramentas para tentar visualizar dados
multivariados, porém todas elas perdem a eficiência à medida que o
número de variáveis aumenta.

Quando ele se torna muito grande para ser visualizado diretamente, em
geral, a abordagem mais produtiva é reduzir primeiro o número de
dimensões (conforme analisaremos mais adiante) e, em seguida, visualizar
esse conjunto reduzido de dados. (Poldrack, 2025 , p.~203)
\end{quote}

\subsection{Gráfico de Dispersão de
Matrizes}\label{gruxe1fico-de-dispersuxe3o-de-matrizes}

\begin{quote}
Uma forma útil de visualizar um pequeno número de variáveis é plotar
cada par de variáveis entre si, às vezes conhecido como gráfico de
dispersão de matrizes; um exemplo é mostrado na Figura 16.1. Cada
linha/coluna na figura se refere a uma única variável --- nesse caso, é
uma de nossas variáveis sobre psicologia do exemplo de conjunto de dados
de autocontrole visto anteriormente. Os elementos diagonais do gráfico
mostram a distribuição de cada variável como um histograma. Os elementos
abaixo da diagonal mostram um gráfico de dispersão para cada par de
matrizes, sobreposto com uma linha de regressão que descreve a relação
entre as variáveis. Os elementos acima da diagonal mostram o coeficiente
de correlação para cada par de variáveis. Quando o número de variáveis é
relativamente pequeno (cerca de 10 ou menos), essa pode ser uma forma
útil de obter bons insights a partir de um conjunto de dados
multivariados. De imediato, podemos observar que as correlações são
altas entre cada uma das variáveis do SSRT e entre cada uma das
variáveis de impulsividade do UPPS. No entanto, as correlações entre as
duas são todas muito baixas. Essa é nossa primeira suspeita de que
existem dois conjuntos de variáveis relacionadas nesse conjunto de
dados. (Poldrack, 2025 , p.~203)
\end{quote}

\begin{Shaded}
\begin{Highlighting}[numbers=left,,]
\InformationTok{\textasciigrave{}\textasciigrave{}\textasciigrave{}\{r\}}
\CommentTok{\# import MASS first because it otherwise will mask dplyr::select}
\FunctionTok{library}\NormalTok{(MASS)}

\FunctionTok{library}\NormalTok{(tidyverse)}
\FunctionTok{library}\NormalTok{(ggdendro)}
\FunctionTok{library}\NormalTok{(psych)}
\FunctionTok{library}\NormalTok{(gplots)}
\FunctionTok{library}\NormalTok{(pdist)}
\FunctionTok{library}\NormalTok{(factoextra)}
\FunctionTok{library}\NormalTok{(viridis)}
\FunctionTok{library}\NormalTok{(mclust)}
\FunctionTok{library}\NormalTok{(knitr)}
\FunctionTok{theme\_set}\NormalTok{(}\FunctionTok{theme\_minimal}\NormalTok{())}
\InformationTok{\textasciigrave{}\textasciigrave{}\textasciigrave{}}
\end{Highlighting}
\end{Shaded}

\subsection{\texorpdfstring{Dados Multivariados -
\emph{setup}}{Dados Multivariados - setup}}\label{dados-multivariados---setup}

Código para preparar os dados para gerar a fig.~16.1.

\begin{Shaded}
\begin{Highlighting}[numbers=left,,]
\InformationTok{\textasciigrave{}\textasciigrave{}\textasciigrave{}\{r\}}
\NormalTok{behavdata }\OtherTok{\textless{}{-}} \FunctionTok{read\_csv}\NormalTok{(}\StringTok{\textquotesingle{}https://raw.githubusercontent.com/statsthinking21/statsthinking21{-}figures{-}data/main/Eisenberg/meaningful\_variables.csv\textquotesingle{}}\NormalTok{,}
                      \AttributeTok{show\_col\_types =} \ConstantTok{FALSE}\NormalTok{)}
\NormalTok{demoghealthdata }\OtherTok{\textless{}{-}} \FunctionTok{read\_csv}\NormalTok{(}\StringTok{\textquotesingle{}https://raw.githubusercontent.com/statsthinking21/statsthinking21{-}figures{-}data/main/Eisenberg/demographic\_health.csv\textquotesingle{}}\NormalTok{,}
                            \AttributeTok{show\_col\_types =} \ConstantTok{FALSE}\NormalTok{)}

\CommentTok{\# recode Sex variable from 0/1 to Male/Female}
\NormalTok{demoghealthdata }\OtherTok{\textless{}{-}}\NormalTok{ demoghealthdata }\SpecialCharTok{\%\textgreater{}\%}
  \FunctionTok{mutate}\NormalTok{(}\AttributeTok{Sex =} \FunctionTok{recode\_factor}\NormalTok{(Sex, }\StringTok{\textasciigrave{}}\AttributeTok{0}\StringTok{\textasciigrave{}}\OtherTok{=}\StringTok{"Male"}\NormalTok{, }\StringTok{\textasciigrave{}}\AttributeTok{1}\StringTok{\textasciigrave{}}\OtherTok{=}\StringTok{"Female"}\NormalTok{))}

\CommentTok{\# combine the data into a single data frame by subcode}
\NormalTok{alldata }\OtherTok{\textless{}{-}} \FunctionTok{merge}\NormalTok{(behavdata, demoghealthdata, }\AttributeTok{by=}\StringTok{\textquotesingle{}subcode\textquotesingle{}}\NormalTok{)}

\NormalTok{rename\_list }\OtherTok{=} \FunctionTok{list}\NormalTok{(}\StringTok{\textquotesingle{}upps\_impulsivity\_survey\textquotesingle{}}  \OtherTok{=} \StringTok{\textquotesingle{}UPPS\textquotesingle{}}\NormalTok{,}
                   \StringTok{\textquotesingle{}sensation\_seeking\_survey\textquotesingle{}} \OtherTok{=} \StringTok{\textquotesingle{}SSS\textquotesingle{}}\NormalTok{,}
                   \StringTok{\textquotesingle{}dickman\_survey\textquotesingle{}}    \OtherTok{=} \StringTok{\textquotesingle{}Dickman\textquotesingle{}}\NormalTok{,}
                   \StringTok{\textquotesingle{}bis11\_survey\textquotesingle{}}      \OtherTok{=} \StringTok{\textquotesingle{}BIS11\textquotesingle{}}\NormalTok{,}
                   \StringTok{\textquotesingle{}spatial\_span\textquotesingle{}}      \OtherTok{=} \StringTok{\textquotesingle{}spatial\textquotesingle{}}\NormalTok{,}
                   \StringTok{\textquotesingle{}digit\_span\textquotesingle{}}        \OtherTok{=} \StringTok{\textquotesingle{}digit\textquotesingle{}}\NormalTok{,}
                   \StringTok{\textquotesingle{}adaptive\_n\_back\textquotesingle{}}   \OtherTok{=} \StringTok{\textquotesingle{}nback\textquotesingle{}}\NormalTok{,}
                   \StringTok{\textquotesingle{}dospert\_rt\_survey\textquotesingle{}} \OtherTok{=} \StringTok{\textquotesingle{}dospert\textquotesingle{}}\NormalTok{,}
                   \StringTok{\textquotesingle{}motor\_selective\_stop\_signal.SSRT\textquotesingle{}} \OtherTok{=} \StringTok{\textquotesingle{}SSRT\_motorsel\textquotesingle{}}\NormalTok{,}
                   \StringTok{\textquotesingle{}stim\_selective\_stop\_signal.SSRT\textquotesingle{}}  \OtherTok{=} \StringTok{\textquotesingle{}SSRT\_stimsel\textquotesingle{}}\NormalTok{,}
                   \StringTok{\textquotesingle{}stop\_signal.SSRT\_low\textquotesingle{}}  \OtherTok{=} \StringTok{\textquotesingle{}SSRT\_low\textquotesingle{}}\NormalTok{,}
                   \StringTok{\textquotesingle{}stop\_signal.SSRT\_high\textquotesingle{}} \OtherTok{=} \StringTok{\textquotesingle{}SSRT\_high\textquotesingle{}}\NormalTok{)}

\NormalTok{impulsivity\_variables }\OtherTok{=} \FunctionTok{c}\NormalTok{(}\StringTok{\textquotesingle{}Sex\textquotesingle{}}\NormalTok{)}

\NormalTok{keep\_variables }\OtherTok{\textless{}{-}} \FunctionTok{c}\NormalTok{(}\StringTok{"spatial.forward\_span"}\NormalTok{,}
                    \StringTok{"spatial.reverse\_span"}\NormalTok{,}
                    \StringTok{"digit.forward\_span"}\NormalTok{,}
                    \StringTok{"digit.reverse\_span"}\NormalTok{,}
                    \StringTok{"nback.mean\_load"}\NormalTok{)}

\ControlFlowTok{for}\NormalTok{ (potential\_match }\ControlFlowTok{in} \FunctionTok{names}\NormalTok{(alldata))\{}
  \ControlFlowTok{for}\NormalTok{ (n }\ControlFlowTok{in} \FunctionTok{names}\NormalTok{(rename\_list))\{}
    \ControlFlowTok{if}\NormalTok{ (}\FunctionTok{str\_detect}\NormalTok{(potential\_match, n))\{}
      \CommentTok{\# print(sprintf(\textquotesingle{}found match: \%s \%s\textquotesingle{}, n, potential\_match))}
\NormalTok{      replacement\_name }\OtherTok{\textless{}{-}} \FunctionTok{str\_replace}\NormalTok{(potential\_match, n, }\FunctionTok{toString}\NormalTok{(rename\_list[n]))}
      \FunctionTok{names}\NormalTok{(alldata)[}\FunctionTok{names}\NormalTok{(alldata) }\SpecialCharTok{==}\NormalTok{ potential\_match] }\OtherTok{\textless{}{-}}\NormalTok{ replacement\_name}
\NormalTok{      impulsivity\_variables }\OtherTok{\textless{}{-}} \FunctionTok{c}\NormalTok{(impulsivity\_variables, replacement\_name)}
\NormalTok{    \}}
\NormalTok{  \}}
\NormalTok{\}}

\NormalTok{impulsivity\_data }\OtherTok{\textless{}{-}}\NormalTok{ alldata[, impulsivity\_variables] }\SpecialCharTok{\%\textgreater{}\%}
  \FunctionTok{drop\_na}\NormalTok{()}


\NormalTok{ssrtdata }\OtherTok{=}\NormalTok{ alldata[,}\FunctionTok{c}\NormalTok{(}\StringTok{\textquotesingle{}subcode\textquotesingle{}}\NormalTok{, }\FunctionTok{names}\NormalTok{(alldata)[}\FunctionTok{grep}\NormalTok{(}\StringTok{\textquotesingle{}SSRT\_\textquotesingle{}}\NormalTok{, }\FunctionTok{names}\NormalTok{(alldata))])] }\SpecialCharTok{\%\textgreater{}\%}
  \FunctionTok{drop\_na}\NormalTok{() }\SpecialCharTok{\%\textgreater{}\%}
\NormalTok{  dplyr}\SpecialCharTok{::}\FunctionTok{select}\NormalTok{(}\SpecialCharTok{{-}}\NormalTok{stop\_signal.proactive\_SSRT\_speeding)}

\NormalTok{upps\_data }\OtherTok{\textless{}{-}}\NormalTok{ alldata }\SpecialCharTok{\%\textgreater{}\%}
\NormalTok{  dplyr}\SpecialCharTok{::}\FunctionTok{select}\NormalTok{(}\FunctionTok{starts\_with}\NormalTok{(}\StringTok{\textquotesingle{}UPPS\textquotesingle{}}\NormalTok{), }\StringTok{\textquotesingle{}subcode\textquotesingle{}}\NormalTok{) }\SpecialCharTok{\%\textgreater{}\%}
  \FunctionTok{setNames}\NormalTok{(}\FunctionTok{gsub}\NormalTok{(}\StringTok{"UPPS."}\NormalTok{, }\StringTok{""}\NormalTok{, }\FunctionTok{names}\NormalTok{(.)))}

\NormalTok{impdata }\OtherTok{\textless{}{-}} \FunctionTok{inner\_join}\NormalTok{(ssrtdata, upps\_data) }\SpecialCharTok{\%\textgreater{}\%}
  \FunctionTok{drop\_na}\NormalTok{() }\SpecialCharTok{\%\textgreater{}\%}
\NormalTok{  dplyr}\SpecialCharTok{::}\FunctionTok{select}\NormalTok{(}\SpecialCharTok{{-}}\NormalTok{subcode) }\SpecialCharTok{\%\textgreater{}\%}
  \FunctionTok{scale}\NormalTok{() }\SpecialCharTok{\%\textgreater{}\%}
  \FunctionTok{as.data.frame}\NormalTok{() }\SpecialCharTok{\%\textgreater{}\%}
\NormalTok{  dplyr}\SpecialCharTok{::}\FunctionTok{rename}\NormalTok{(}\AttributeTok{SSRT\_motor   =}\NormalTok{ SSRT\_motorsel,}
                \AttributeTok{SSRT\_stim    =}\NormalTok{ SSRT\_stimsel,}
                \AttributeTok{UPPS\_pers    =}\NormalTok{ lack\_of\_perseverance,}
                \AttributeTok{UPPS\_premed  =}\NormalTok{ lack\_of\_premeditation,}
                \AttributeTok{UPPS\_negurg  =}\NormalTok{ negative\_urgency,}
                \AttributeTok{UPPS\_posurg  =}\NormalTok{ positive\_urgency,}
                \AttributeTok{UPPS\_senseek =}\NormalTok{ sensation\_seeking}
\NormalTok{                )}
\InformationTok{\textasciigrave{}\textasciigrave{}\textasciigrave{}}
\end{Highlighting}
\end{Shaded}

\subsection{Figura 16.1 - Matriz de gráficos de dispersão (9
variáveis)}\label{figura-16.1---matriz-de-gruxe1ficos-de-dispersuxe3o-9-variuxe1veis}

\begin{Shaded}
\begin{Highlighting}[numbers=left,,]
\InformationTok{\textasciigrave{}\textasciigrave{}\textasciigrave{}\{r\}}
\FunctionTok{pairs.panels}\NormalTok{(impdata, }\AttributeTok{lm=}\ConstantTok{TRUE}\NormalTok{)}
\InformationTok{\textasciigrave{}\textasciigrave{}\textasciigrave{}}
\end{Highlighting}
\end{Shaded}

\pandocbounded{\includegraphics[keepaspectratio]{cap16-pldr-EstMultiv_files/figure-pdf/unnamed-chunk-3-1.pdf}}

Os gráficos em faceta acima permitem uma boa análise preliminar da
possível associação linear entre cada para de variáveis quantitativas do
conjunto de dados \texttt{autocontrole}.

\subsection{Figure 16.2 - mapa de
calor}\label{figure-16.2---mapa-de-calor}

\begin{Shaded}
\begin{Highlighting}[numbers=left,,]
\InformationTok{\textasciigrave{}\textasciigrave{}\textasciigrave{}\{r\}}
\NormalTok{cc }\OtherTok{=} \FunctionTok{cor}\NormalTok{(impdata)}
\FunctionTok{par}\NormalTok{(}\AttributeTok{mai=}\FunctionTok{c}\NormalTok{(}\DecValTok{2}\NormalTok{, }\DecValTok{1}\NormalTok{, }\DecValTok{1}\NormalTok{, }\DecValTok{1}\NormalTok{)}\SpecialCharTok{+}\FloatTok{0.1}\NormalTok{)}

\FunctionTok{heatmap.2}\NormalTok{(cc, }\AttributeTok{trace=}\StringTok{\textquotesingle{}none\textquotesingle{}}\NormalTok{, }\AttributeTok{dendrogram=}\StringTok{\textquotesingle{}none\textquotesingle{}}\NormalTok{,}
          \AttributeTok{cellnote=}\FunctionTok{round}\NormalTok{(cc, }\DecValTok{2}\NormalTok{), }\AttributeTok{notecol=}\StringTok{\textquotesingle{}black\textquotesingle{}}\NormalTok{, }\AttributeTok{key=}\ConstantTok{FALSE}\NormalTok{,}
          \AttributeTok{margins=}\FunctionTok{c}\NormalTok{(}\DecValTok{12}\NormalTok{,}\DecValTok{8}\NormalTok{), }\AttributeTok{srtCol=}\DecValTok{45}\NormalTok{, }\AttributeTok{symm=}\ConstantTok{TRUE}\NormalTok{, }\AttributeTok{revC=}\ConstantTok{TRUE}\NormalTok{, }\CommentTok{\#notecex=4,}
          \AttributeTok{cexRow=}\DecValTok{1}\NormalTok{, }\AttributeTok{cexCol=}\DecValTok{1}\NormalTok{, }\AttributeTok{offsetRow=}\SpecialCharTok{{-}}\DecValTok{150}\NormalTok{, }\AttributeTok{col=}\FunctionTok{viridis}\NormalTok{(}\DecValTok{50}\NormalTok{))}
\InformationTok{\textasciigrave{}\textasciigrave{}\textasciigrave{}}
\end{Highlighting}
\end{Shaded}

\pandocbounded{\includegraphics[keepaspectratio]{cap16-pldr-EstMultiv_files/figure-pdf/unnamed-chunk-4-1.pdf}}

Mapa de calor da Matriz de Correlação para 9 variáveis do data set
\texttt{autocontrole}.

\bookmarksetup{startatroot}

\chapter{AED - cap 6 moore - Tabelas de Dupla
Entrada}\label{sec-tab-dupla-entr}

\begin{quote}
Nos Capítulos 4 e 5, consideramos relações entre duas variáveis
quantitativas.

Neste capítulo, usaremos tabelas de dupla entrada para descrevermos
relações entre duas variáveis categóricas. Algumas variáveis -- tais
como gênero, raça e ocupação -- são categóricas por natureza. Outras
variáveis categóricas são criadas pelo agrupamento em classes dos
valores de uma variável quantitativa.

Para explorar relações entre duas variáveis categóricas, usamos as
contagens ou percentuais de indivíduos que se encaixam nas várias
categorias.

Assim como com variáveis quantitativas, devemos estar alertas para a
influência de variáveis ocultas, e tomar cuidado para não assumirmos que
os padrões que observamos continuem a valer para dados adicionais ou em
um contexto mais amplo. (MOORE; NOTZ; FLIGNER, 2023 , p.~130).
\end{quote}

\begin{Shaded}
\begin{Highlighting}[numbers=left,,]
\InformationTok{\textasciigrave{}\textasciigrave{}\textasciigrave{}\{r\}}
\FunctionTok{library}\NormalTok{(tidyverse)}
\FunctionTok{library}\NormalTok{(NHANES)}
\FunctionTok{library}\NormalTok{(cowplot)}
\FunctionTok{library}\NormalTok{(mapproj)}
\FunctionTok{library}\NormalTok{(pander)}
\FunctionTok{library}\NormalTok{(knitr)}
\FunctionTok{library}\NormalTok{(modelr)}

\FunctionTok{panderOptions}\NormalTok{(}\StringTok{\textquotesingle{}round\textquotesingle{}}\NormalTok{,}\DecValTok{2}\NormalTok{)}
\FunctionTok{panderOptions}\NormalTok{(}\StringTok{\textquotesingle{}digits\textquotesingle{}}\NormalTok{,}\DecValTok{7}\NormalTok{)}
\FunctionTok{theme\_set}\NormalTok{(}\FunctionTok{theme\_minimal}\NormalTok{(}\AttributeTok{base\_size =} \DecValTok{14}\NormalTok{))}

\FunctionTok{options}\NormalTok{(}\AttributeTok{digits =} \DecValTok{2}\NormalTok{)}
\FunctionTok{set.seed}\NormalTok{(}\DecValTok{123456}\NormalTok{) }\CommentTok{\# set random seed to exactly replicate results}
\InformationTok{\textasciigrave{}\textasciigrave{}\textasciigrave{}}
\end{Highlighting}
\end{Shaded}

\section{Objetivos da Aprendizagem}\label{objetivos-da-aprendizagem-3}

\begin{quote}
Após ler este capítulo, você deve ser capaz de:

▶ 6.1 Calcular e interpretar distribuições marginais em tabelas de dupla
entrada.

▶6.2 Calcular e interpretar distribuições condicionais em tabelas de
dupla entrada.

▶6.3Reconhecer e explicar o paradoxo de Simpson. (MOORE; NOTZ; FLIGNER,
2023 , p.~130).
\end{quote}

\subsection{Exemplo 6.1 Quem recebe graus
acadêmicos?}\label{exemplo-6.1-quem-recebe-graus-acaduxeamicos}

Em 2017, o órgão norte-americano National Center for Education
Statistics fez uma projeção do número de graus acadêmicos a serem dados
em 2020 e 2021 para homens e mulheres. A Tabela 6.1 mostra suas
projeções.\footnote{Os dados são do 2017 Digest of Education Statistics,
  no site do \emph{National Center for Education Statistics},
  \href{https://nces.ed.gov/}{nces.ed.gov}.} Essa é uma tabela de dupla
entrada porque descreve duas variáveis categóricas. Uma é o sexo de um
indivíduo. A outra é o grau acadêmico recebido. Sexo é a variável linha
porque cada linha na tabela descreve o sexo de um indivíduo. Grau
acadêmico conferido é a variável coluna porque cada coluna descreve um
grau. Como o grau acadêmico conferido tem uma ordem natural, desde
``Associado'' a ``Doutor'', as colunas estão nessa ordem. As entradas na
tabela são as contagens de indivíduos (em milhares) em cada classe de
sexo por grau acadêmico.

As entradas na margem direita são os totais das entradas das linhas, as
entradas na margem inferior são os totais das entradas das colunas, e a
entrada embaixo à direita é o total de todos os estudantes previstos
para receberem um grau acadêmico no período de 2020 e 2021.

\begin{Shaded}
\begin{Highlighting}[numbers=left,,]
\InformationTok{\textasciigrave{}\textasciigrave{}\textasciigrave{}\{r\}}
\CommentTok{\# Graus acadêmicos por sexo (com ordem explícita dos fatores)}
\NormalTok{grausex }\OtherTok{\textless{}{-}} \FunctionTok{data.frame}\NormalTok{(}
  \AttributeTok{sex  =} \FunctionTok{factor}\NormalTok{(}\FunctionTok{c}\NormalTok{(}\FunctionTok{rep}\NormalTok{(}\StringTok{"M"}\NormalTok{, }\DecValTok{2283}\NormalTok{), }\FunctionTok{rep}\NormalTok{(}\StringTok{"H"}\NormalTok{, }\DecValTok{1622}\NormalTok{)),}
                \AttributeTok{levels =} \FunctionTok{c}\NormalTok{(}\StringTok{"M"}\NormalTok{, }\StringTok{"H"}\NormalTok{)),}
  \AttributeTok{grau =} \FunctionTok{factor}\NormalTok{(}\FunctionTok{c}\NormalTok{(}\FunctionTok{rep}\NormalTok{(}\StringTok{"ass"}\NormalTok{, }\DecValTok{639}\NormalTok{), }\FunctionTok{rep}\NormalTok{(}\StringTok{"bach"}\NormalTok{, }\DecValTok{1087}\NormalTok{), }\FunctionTok{rep}\NormalTok{(}\StringTok{"ms"}\NormalTok{, }\DecValTok{460}\NormalTok{), }\FunctionTok{rep}\NormalTok{(}\StringTok{"dr"}\NormalTok{, }\DecValTok{97}\NormalTok{),}
                  \FunctionTok{rep}\NormalTok{(}\StringTok{"ass"}\NormalTok{, }\DecValTok{402}\NormalTok{), }\FunctionTok{rep}\NormalTok{(}\StringTok{"bach"}\NormalTok{,  }\DecValTok{804}\NormalTok{), }\FunctionTok{rep}\NormalTok{(}\StringTok{"ms"}\NormalTok{, }\DecValTok{329}\NormalTok{), }\FunctionTok{rep}\NormalTok{(}\StringTok{"dr"}\NormalTok{, }\DecValTok{87}\NormalTok{)),}
                \AttributeTok{levels =} \FunctionTok{c}\NormalTok{(}\StringTok{"ass"}\NormalTok{, }\StringTok{"bach"}\NormalTok{, }\StringTok{"ms"}\NormalTok{, }\StringTok{"dr"}\NormalTok{))}
\NormalTok{)}


\CommentTok{\# tabela multiway do data.frame (mantém a ordem dos níveis)}
\FunctionTok{table}\NormalTok{(grausex)}
\InformationTok{\textasciigrave{}\textasciigrave{}\textasciigrave{}}
\end{Highlighting}
\end{Shaded}

\begin{verbatim}
   grau
sex  ass bach   ms   dr
  M  639 1087  460   97
  H  402  804  329   87
\end{verbatim}

Acrescentar os totais marginais de linhas e de colunas.

\begin{Shaded}
\begin{Highlighting}[numbers=left,,]
\InformationTok{\textasciigrave{}\textasciigrave{}\textasciigrave{}\{r\}}
\CommentTok{\# Tabela de contingência (linhas = sexo, colunas = grau)}
\NormalTok{tab }\OtherTok{\textless{}{-}} \FunctionTok{table}\NormalTok{(grausex}\SpecialCharTok{$}\NormalTok{sex, grausex}\SpecialCharTok{$}\NormalTok{grau)}

\CommentTok{\# Adicionar totais marginais (linha e coluna)}
\NormalTok{tab\_totais }\OtherTok{\textless{}{-}} \FunctionTok{addmargins}\NormalTok{(tab)}

\CommentTok{\# Substituir o rótulo padrão "Sum" por "Total" nas margens (mais legível em pt{-}BR)}
\FunctionTok{rownames}\NormalTok{(tab\_totais)[}\FunctionTok{nrow}\NormalTok{(tab\_totais)] }\OtherTok{\textless{}{-}} \StringTok{"Total"}
\FunctionTok{colnames}\NormalTok{(tab\_totais)[}\FunctionTok{ncol}\NormalTok{(tab\_totais)] }\OtherTok{\textless{}{-}} \StringTok{"Total"}

\CommentTok{\# Exibir tabela com totais marginais}
\FunctionTok{cat}\NormalTok{(}\StringTok{"}\SpecialCharTok{\textbackslash{}n}\StringTok{Tabela com totais marginais:}\SpecialCharTok{\textbackslash{}n}\StringTok{"}\NormalTok{)}
\FunctionTok{print}\NormalTok{(tab\_totais)}

\CommentTok{\# Exemplo: acessar apenas totais de linha ou coluna (se precisar)}
\CommentTok{\# margin.table(tab, 1)  \# totais por grau (linhas)}
\CommentTok{\# margin.table(tab, 2)  \# totais por sexo (colunas)}
\InformationTok{\textasciigrave{}\textasciigrave{}\textasciigrave{}}
\end{Highlighting}
\end{Shaded}

\begin{verbatim}

Tabela com totais marginais:
       
         ass bach   ms   dr Total
  M      639 1087  460   97  2283
  H      402  804  329   87  1622
  Total 1041 1891  789  184  3905
\end{verbatim}

\begin{tcolorbox}[enhanced jigsaw, arc=.35mm, opacitybacktitle=0.6, colframe=quarto-callout-important-color-frame, titlerule=0mm, leftrule=.75mm, left=2mm, colbacktitle=quarto-callout-important-color!10!white, breakable, toprule=.15mm, bottomtitle=1mm, opacityback=0, coltitle=black, title=\textcolor{quarto-callout-important-color}{\faExclamation}\hspace{0.5em}{Tabela de dupla entrada}, rightrule=.15mm, bottomrule=.15mm, toptitle=1mm, colback=white]

Uma tabela de contagens usada para a \textbf{\emph{organização de dados
sobre duas variáveis categóricas}}.

Valores de cada variável linha percorrem horizontalmente a tabela, e
valores de cada variável coluna percorrem a tabela verticalmente.

Entradas na tabela são as contagens da frequência em que cada combinação
da linha e da coluna correspondentes ocorre.

Tabelas de dupla entrada são usadas, em geral, para o
\textbf{\emph{resumo}} de grandes quantidades de informação por meio do
agrupamento das observações em categorias.

\end{tcolorbox}

\section{Distribuições Marginais}\label{distribuiuxe7uxf5es-marginais}

\begin{quote}
Como podemos apreender a informação contida na Tabela 6.1? Primeiro,
olhe a distribuição de cada variável separadamente. A distribuição de
uma variável categórica diz com que frequência cada resultado ocorreu. A
coluna ``Total'' à direita da tabela contém os totais para as linhas.
Esses totais de linhas mostram a distribuição de sexo no grupo inteiro
de 3.905 mil estudantes: 2.283 mil são mulheres, e 1.622 mil são homens.

Se os totais de linhas e de colunas estiverem ausentes, a primeira coisa
a fazer no estudo de uma tabela de dupla entrada é calcular esses
totais. As distribuições de sexo apenas e de grau conferido apenas são
chamadas distribuições marginais, porque elas aparecem nas margens
direita e inferior da tabela de dupla entrada.

\begin{tcolorbox}[enhanced jigsaw, arc=.35mm, opacitybacktitle=0.6, colframe=quarto-callout-important-color-frame, titlerule=0mm, leftrule=.75mm, left=2mm, colbacktitle=quarto-callout-important-color!10!white, breakable, toprule=.15mm, bottomtitle=1mm, opacityback=0, coltitle=black, title=\textcolor{quarto-callout-important-color}{\faExclamation}\hspace{0.5em}{Distribuições marginais}, rightrule=.15mm, bottomrule=.15mm, toptitle=1mm, colback=white]

A distribuição marginal de uma das variáveis categóricas em uma tabela
de dupla entrada de contagens é a distribuição dos valores daquela
variável entre todos os indivíduos descritos pela tabela.

\end{tcolorbox}

Porcentagens são, em geral, mais informativas do que contagens.

Podemos apresentar a distribuição marginal de sexo em porcentagens,
dividindo cada total de linha pelo total da tabela e convertendo para
uma porcentagem. (MOORE; NOTZ; FLIGNER, 2023 , p.~131).
\end{quote}

\subsection{Exemplo 6.2 Cálculo de uma distribuição
marginal}\label{exemplo-6.2-cuxe1lculo-de-uma-distribuiuxe7uxe3o-marginal}

\begin{Shaded}
\begin{Highlighting}[numbers=left,,]
\InformationTok{\textasciigrave{}\textasciigrave{}\textasciigrave{}\{r\}}
\CommentTok{\# {-}{-}{-} Marginais em contagem {-}{-}{-}}
\CommentTok{\# Totais por linha (por sexo)}
\NormalTok{margem\_linhas }\OtherTok{\textless{}{-}} \FunctionTok{margin.table}\NormalTok{(tab, }\DecValTok{1}\NormalTok{)}

\CommentTok{\# Totais por coluna (por grau)}
\NormalTok{margem\_colunas }\OtherTok{\textless{}{-}} \FunctionTok{margin.table}\NormalTok{(tab, }\DecValTok{2}\NormalTok{)}

\FunctionTok{cat}\NormalTok{(}\StringTok{"}\SpecialCharTok{\textbackslash{}n}\StringTok{Marginal — por sexo (contagem):}\SpecialCharTok{\textbackslash{}n}\StringTok{"}\NormalTok{)}
\FunctionTok{print}\NormalTok{(margem\_linhas)}
\FunctionTok{cat}\NormalTok{(}\StringTok{"}\SpecialCharTok{\textbackslash{}n}\StringTok{Marginal — por grau (contagem):}\SpecialCharTok{\textbackslash{}n}\StringTok{"}\NormalTok{)}
\FunctionTok{print}\NormalTok{(margem\_colunas)}

\CommentTok{\# {-}{-}{-} Marginais em proporção (relativas ao total geral) {-}{-}{-}}
\NormalTok{prop\_linhas  }\OtherTok{\textless{}{-}} \FunctionTok{prop.table}\NormalTok{(margem\_linhas)  }\CommentTok{\# soma = 1}
\NormalTok{prop\_colunas }\OtherTok{\textless{}{-}} \FunctionTok{prop.table}\NormalTok{(margem\_colunas) }\CommentTok{\# soma = 1}

\FunctionTok{cat}\NormalTok{(}\StringTok{"}\SpecialCharTok{\textbackslash{}n}\StringTok{Marginal — por sexo (proporção \% do total):}\SpecialCharTok{\textbackslash{}n}\StringTok{"}\NormalTok{)}
\FunctionTok{print}\NormalTok{(}\FunctionTok{round}\NormalTok{(}\DecValTok{100} \SpecialCharTok{*}\NormalTok{ prop\_linhas, }\DecValTok{2}\NormalTok{))}
\FunctionTok{cat}\NormalTok{(}\StringTok{"}\SpecialCharTok{\textbackslash{}n}\StringTok{Marginal — por grau (proporção \% do total):}\SpecialCharTok{\textbackslash{}n}\StringTok{"}\NormalTok{)}
\FunctionTok{print}\NormalTok{(}\FunctionTok{round}\NormalTok{(}\DecValTok{100} \SpecialCharTok{*}\NormalTok{ prop\_colunas, }\DecValTok{2}\NormalTok{))}

\CommentTok{\# {-}{-}{-} Proporções condicionais e por célula {-}{-}{-}}
\NormalTok{prop\_celula\_total }\OtherTok{\textless{}{-}} \FunctionTok{prop.table}\NormalTok{(tab)        }\CommentTok{\# cada célula / total geral}
\NormalTok{prop\_cond\_linha }\OtherTok{\textless{}{-}} \FunctionTok{prop.table}\NormalTok{(tab, }\DecValTok{1}\NormalTok{)       }\CommentTok{\# cada linha soma 1 (condicional por sexo)}
\NormalTok{prop\_cond\_coluna }\OtherTok{\textless{}{-}} \FunctionTok{prop.table}\NormalTok{(tab, }\DecValTok{2}\NormalTok{)      }\CommentTok{\# cada coluna soma 1 (condicional por grau)}

\FunctionTok{cat}\NormalTok{(}\StringTok{"}\SpecialCharTok{\textbackslash{}n}\StringTok{Proporção de cada célula em relação ao total (em \%):}\SpecialCharTok{\textbackslash{}n}\StringTok{"}\NormalTok{)}
\FunctionTok{print}\NormalTok{(}\FunctionTok{round}\NormalTok{(}\DecValTok{100} \SpecialCharTok{*}\NormalTok{ prop\_celula\_total, }\DecValTok{2}\NormalTok{))}
\FunctionTok{cat}\NormalTok{(}\StringTok{"}\SpecialCharTok{\textbackslash{}n}\StringTok{Proporção condicional por sexo (cada linha soma 100\%):}\SpecialCharTok{\textbackslash{}n}\StringTok{"}\NormalTok{)}
\FunctionTok{print}\NormalTok{(}\FunctionTok{round}\NormalTok{(}\DecValTok{100} \SpecialCharTok{*}\NormalTok{ prop\_cond\_linha, }\DecValTok{2}\NormalTok{))}
\FunctionTok{cat}\NormalTok{(}\StringTok{"}\SpecialCharTok{\textbackslash{}n}\StringTok{Proporção condicional por grau (cada coluna soma 100\%):}\SpecialCharTok{\textbackslash{}n}\StringTok{"}\NormalTok{)}
\FunctionTok{print}\NormalTok{(}\FunctionTok{round}\NormalTok{(}\DecValTok{100} \SpecialCharTok{*}\NormalTok{ prop\_cond\_coluna, }\DecValTok{2}\NormalTok{))}

\CommentTok{\# {-}{-}{-} Data.frames resumidos para relatório {-}{-}{-}}
\NormalTok{resumo\_sexo }\OtherTok{\textless{}{-}} \FunctionTok{data.frame}\NormalTok{(}
  \AttributeTok{sexo =} \FunctionTok{names}\NormalTok{(margem\_linhas),}
  \AttributeTok{contagem =} \FunctionTok{as.integer}\NormalTok{(margem\_linhas),}
  \AttributeTok{proporcao\_total =} \FunctionTok{round}\NormalTok{(}\FunctionTok{as.numeric}\NormalTok{(prop\_linhas), }\DecValTok{4}\NormalTok{)}
\NormalTok{)}
\NormalTok{resumo\_grau }\OtherTok{\textless{}{-}} \FunctionTok{data.frame}\NormalTok{(}
  \AttributeTok{grau =} \FunctionTok{names}\NormalTok{(margem\_colunas),}
  \AttributeTok{contagem =} \FunctionTok{as.integer}\NormalTok{(margem\_colunas),}
  \AttributeTok{proporcao\_total =} \FunctionTok{round}\NormalTok{(}\FunctionTok{as.numeric}\NormalTok{(prop\_colunas), }\DecValTok{4}\NormalTok{)}
\NormalTok{)}

\FunctionTok{cat}\NormalTok{(}\StringTok{"}\SpecialCharTok{\textbackslash{}n}\StringTok{Resumo por sexo (contagem + proporção do total):}\SpecialCharTok{\textbackslash{}n}\StringTok{"}\NormalTok{)}
\FunctionTok{print}\NormalTok{(resumo\_sexo)}
\FunctionTok{cat}\NormalTok{(}\StringTok{"}\SpecialCharTok{\textbackslash{}n}\StringTok{Resumo por grau (contagem + proporção do total):}\SpecialCharTok{\textbackslash{}n}\StringTok{"}\NormalTok{)}
\FunctionTok{print}\NormalTok{(resumo\_grau)}
\InformationTok{\textasciigrave{}\textasciigrave{}\textasciigrave{}}
\end{Highlighting}
\end{Shaded}

\begin{verbatim}

Marginal — por sexo (contagem):

   M    H 
2283 1622 

Marginal — por grau (contagem):

 ass bach   ms   dr 
1041 1891  789  184 

Marginal — por sexo (proporção % do total):

 M  H 
58 42 

Marginal — por grau (proporção % do total):

 ass bach   ms   dr 
26.7 48.4 20.2  4.7 

Proporção de cada célula em relação ao total (em %):
   
     ass bach   ms   dr
  M 16.4 27.8 11.8  2.5
  H 10.3 20.6  8.4  2.2

Proporção condicional por sexo (cada linha soma 100%):
   
     ass bach   ms   dr
  M 28.0 47.6 20.1  4.2
  H 24.8 49.6 20.3  5.4

Proporção condicional por grau (cada coluna soma 100%):
   
    ass bach ms dr
  M  61   57 58 53
  H  39   43 42 47

Resumo por sexo (contagem + proporção do total):
  sexo contagem proporcao_total
1    M     2283            0.58
2    H     1622            0.42

Resumo por grau (contagem + proporção do total):
  grau contagem proporcao_total
1  ass     1041           0.267
2 bach     1891           0.484
3   ms      789           0.202
4   dr      184           0.047
\end{verbatim}

Cada distribuição marginal de uma tabela de dupla entrada é uma
distribuição para uma única variável categórica. Como vimos no Capítulo
1, podemos usar um gráfico de barras ou um gráfico de setores para
apresentar essa distribuição. A Figura 6.1 é um gráfico de barras da
distribuição de sexo entre os estudantes na amostra.

Ao trabalhar com uma tabela de dupla entrada, você deve calcular muitos
percentuais. Aqui está uma sugestão para ajudá-lo a decidir qual fração
dá o percentual que você deseja. Pergunte-se: ``qual grupo representa o
total do qual eu desejo uma porcentagem?'' A contagem para esse grupo é
o denominador da fração que leva à porcentagem. No Exemplo 6.2,
desejamos a porcentagem ``de estudantes'', de modo que a contagem de
estudantes (o total da tabela) é o denominador.

\begin{Shaded}
\begin{Highlighting}[numbers=left,,]
\InformationTok{\textasciigrave{}\textasciigrave{}\textasciigrave{}\{r\}}
\CommentTok{\# Pacotes necessários (instala se não existir)}
\ControlFlowTok{if}\NormalTok{ (}\SpecialCharTok{!}\FunctionTok{requireNamespace}\NormalTok{(}\StringTok{"ggplot2"}\NormalTok{, }\AttributeTok{quietly =} \ConstantTok{TRUE}\NormalTok{)) }\FunctionTok{install.packages}\NormalTok{(}\StringTok{"ggplot2"}\NormalTok{)}
\ControlFlowTok{if}\NormalTok{ (}\SpecialCharTok{!}\FunctionTok{requireNamespace}\NormalTok{(}\StringTok{"scales"}\NormalTok{, }\AttributeTok{quietly =} \ConstantTok{TRUE}\NormalTok{)) }\FunctionTok{install.packages}\NormalTok{(}\StringTok{"scales"}\NormalTok{)}
\FunctionTok{library}\NormalTok{(ggplot2)}
\FunctionTok{library}\NormalTok{(scales)}

\CommentTok{\# {-}{-}{-} Gráfico de barras para \textquotesingle{}sexo\textquotesingle{} (contagem) {-}{-}{-}}
\NormalTok{tab\_sexo }\OtherTok{\textless{}{-}} \FunctionTok{as.data.frame}\NormalTok{(}\FunctionTok{table}\NormalTok{(grausex}\SpecialCharTok{$}\NormalTok{sex))}
\FunctionTok{names}\NormalTok{(tab\_sexo) }\OtherTok{\textless{}{-}} \FunctionTok{c}\NormalTok{(}\StringTok{"sexo"}\NormalTok{, }\StringTok{"contagem"}\NormalTok{)}
\NormalTok{tab\_sexo}\SpecialCharTok{$}\NormalTok{proporcao }\OtherTok{\textless{}{-}}\NormalTok{ tab\_sexo}\SpecialCharTok{$}\NormalTok{contagem }\SpecialCharTok{/} \FunctionTok{sum}\NormalTok{(tab\_sexo}\SpecialCharTok{$}\NormalTok{contagem)}

\NormalTok{p\_sexo }\OtherTok{\textless{}{-}} \FunctionTok{ggplot}\NormalTok{(tab\_sexo, }\FunctionTok{aes}\NormalTok{(}\AttributeTok{x =}\NormalTok{ sexo, }\AttributeTok{y =}\NormalTok{ contagem, }\AttributeTok{fill =}\NormalTok{ sexo)) }\SpecialCharTok{+}
  \FunctionTok{geom\_col}\NormalTok{(}\AttributeTok{show.legend =} \ConstantTok{FALSE}\NormalTok{) }\SpecialCharTok{+}
  \FunctionTok{geom\_text}\NormalTok{(}\FunctionTok{aes}\NormalTok{(}\AttributeTok{label =}\NormalTok{ contagem), }\AttributeTok{vjust =} \SpecialCharTok{{-}}\FloatTok{0.4}\NormalTok{) }\SpecialCharTok{+}
  \FunctionTok{labs}\NormalTok{(}\AttributeTok{title =} \StringTok{"Contagem por sexo"}\NormalTok{, }\AttributeTok{x =} \StringTok{"Sexo"}\NormalTok{, }\AttributeTok{y =} \StringTok{"Contagem"}\NormalTok{) }\SpecialCharTok{+}
  \FunctionTok{theme\_minimal}\NormalTok{()}

\CommentTok{\# Exibir}
\FunctionTok{print}\NormalTok{(p\_sexo)}
\CommentTok{\# ggsave("barra\_sexo\_contagem.png", p\_sexo, width = 6, height = 4)}

\CommentTok{\# {-}{-}{-} Gráfico de barras para \textquotesingle{}sexo\textquotesingle{} (proporção) {-}{-}{-}}
\NormalTok{p\_sexo\_prop }\OtherTok{\textless{}{-}} \FunctionTok{ggplot}\NormalTok{(tab\_sexo, }\FunctionTok{aes}\NormalTok{(}\AttributeTok{x =}\NormalTok{ sexo, }\AttributeTok{y =}\NormalTok{ proporcao, }\AttributeTok{fill =}\NormalTok{ sexo)) }\SpecialCharTok{+}
  \FunctionTok{geom\_col}\NormalTok{(}\AttributeTok{show.legend =} \ConstantTok{FALSE}\NormalTok{) }\SpecialCharTok{+}
  \FunctionTok{geom\_text}\NormalTok{(}\FunctionTok{aes}\NormalTok{(}\AttributeTok{label =} \FunctionTok{percent}\NormalTok{(proporcao, }\AttributeTok{accuracy =} \FloatTok{0.1}\NormalTok{)), }\AttributeTok{vjust =} \SpecialCharTok{{-}}\FloatTok{0.4}\NormalTok{) }\SpecialCharTok{+}
  \FunctionTok{scale\_y\_continuous}\NormalTok{(}\AttributeTok{labels =} \FunctionTok{percent\_format}\NormalTok{(}\AttributeTok{accuracy =} \DecValTok{1}\NormalTok{)) }\SpecialCharTok{+}
  \FunctionTok{labs}\NormalTok{(}\AttributeTok{title =} \StringTok{"Proporção por sexo"}\NormalTok{, }\AttributeTok{x =} \StringTok{"Sexo"}\NormalTok{, }\AttributeTok{y =} \StringTok{"Proporção"}\NormalTok{) }\SpecialCharTok{+}
  \FunctionTok{theme\_minimal}\NormalTok{()}

\FunctionTok{print}\NormalTok{(p\_sexo\_prop)}
\CommentTok{\# ggsave("barra\_sexo\_proporcao.png", p\_sexo\_prop, width = 6, height = 4)}

\CommentTok{\# {-}{-}{-} Gráfico de barras para \textquotesingle{}grau\textquotesingle{} (contagem) {-}{-}{-}}
\NormalTok{tab\_grau }\OtherTok{\textless{}{-}} \FunctionTok{as.data.frame}\NormalTok{(}\FunctionTok{table}\NormalTok{(grausex}\SpecialCharTok{$}\NormalTok{grau))}
\FunctionTok{names}\NormalTok{(tab\_grau) }\OtherTok{\textless{}{-}} \FunctionTok{c}\NormalTok{(}\StringTok{"grau"}\NormalTok{, }\StringTok{"contagem"}\NormalTok{)}
\NormalTok{tab\_grau}\SpecialCharTok{$}\NormalTok{proporcao }\OtherTok{\textless{}{-}}\NormalTok{ tab\_grau}\SpecialCharTok{$}\NormalTok{contagem }\SpecialCharTok{/} \FunctionTok{sum}\NormalTok{(tab\_grau}\SpecialCharTok{$}\NormalTok{contagem)}

\NormalTok{p\_grau }\OtherTok{\textless{}{-}} \FunctionTok{ggplot}\NormalTok{(tab\_grau, }\FunctionTok{aes}\NormalTok{(}\AttributeTok{x =}\NormalTok{ grau, }\AttributeTok{y =}\NormalTok{ contagem, }\AttributeTok{fill =}\NormalTok{ grau)) }\SpecialCharTok{+}
  \FunctionTok{geom\_col}\NormalTok{(}\AttributeTok{show.legend =} \ConstantTok{FALSE}\NormalTok{) }\SpecialCharTok{+}
  \FunctionTok{geom\_text}\NormalTok{(}\FunctionTok{aes}\NormalTok{(}\AttributeTok{label =}\NormalTok{ contagem), }\AttributeTok{vjust =} \SpecialCharTok{{-}}\FloatTok{0.4}\NormalTok{) }\SpecialCharTok{+}
  \FunctionTok{labs}\NormalTok{(}\AttributeTok{title =} \StringTok{"Contagem por grau acadêmico"}\NormalTok{, }\AttributeTok{x =} \StringTok{"Grau"}\NormalTok{, }\AttributeTok{y =} \StringTok{"Contagem"}\NormalTok{) }\SpecialCharTok{+}
  \FunctionTok{theme\_minimal}\NormalTok{()}

\FunctionTok{print}\NormalTok{(p\_grau)}
\CommentTok{\# ggsave("barra\_grau\_contagem.png", p\_grau, width = 7, height = 4)}

\CommentTok{\# {-}{-}{-} Gráfico de barras para \textquotesingle{}grau\textquotesingle{} (proporção) {-}{-}{-}}
\NormalTok{p\_grau\_prop }\OtherTok{\textless{}{-}} \FunctionTok{ggplot}\NormalTok{(tab\_grau, }\FunctionTok{aes}\NormalTok{(}\AttributeTok{x =}\NormalTok{ grau, }\AttributeTok{y =}\NormalTok{ proporcao, }\AttributeTok{fill =}\NormalTok{ grau)) }\SpecialCharTok{+}
  \FunctionTok{geom\_col}\NormalTok{(}\AttributeTok{show.legend =} \ConstantTok{FALSE}\NormalTok{) }\SpecialCharTok{+}
  \FunctionTok{geom\_text}\NormalTok{(}\FunctionTok{aes}\NormalTok{(}\AttributeTok{label =} \FunctionTok{percent}\NormalTok{(proporcao, }\AttributeTok{accuracy =} \FloatTok{0.1}\NormalTok{)), }\AttributeTok{vjust =} \SpecialCharTok{{-}}\FloatTok{0.4}\NormalTok{) }\SpecialCharTok{+}
  \FunctionTok{scale\_y\_continuous}\NormalTok{(}\AttributeTok{labels =} \FunctionTok{percent\_format}\NormalTok{(}\AttributeTok{accuracy =} \DecValTok{1}\NormalTok{)) }\SpecialCharTok{+}
  \FunctionTok{labs}\NormalTok{(}\AttributeTok{title =} \StringTok{"Proporção por grau acadêmico"}\NormalTok{, }\AttributeTok{x =} \StringTok{"Grau"}\NormalTok{, }\AttributeTok{y =} \StringTok{"Proporção"}\NormalTok{) }\SpecialCharTok{+}
  \FunctionTok{theme\_minimal}\NormalTok{()}

\FunctionTok{print}\NormalTok{(p\_grau\_prop)}
\CommentTok{\# ggsave("barra\_grau\_proporcao.png", p\_grau\_prop, width = 7, height = 4)}
\InformationTok{\textasciigrave{}\textasciigrave{}\textasciigrave{}}
\end{Highlighting}
\end{Shaded}

\pandocbounded{\includegraphics[keepaspectratio]{cap6-moore-tab-dupla-entrada_files/figure-pdf/unnamed-chunk-5-1.pdf}}

\pandocbounded{\includegraphics[keepaspectratio]{cap6-moore-tab-dupla-entrada_files/figure-pdf/unnamed-chunk-5-2.pdf}}

\pandocbounded{\includegraphics[keepaspectratio]{cap6-moore-tab-dupla-entrada_files/figure-pdf/unnamed-chunk-5-3.pdf}}

\pandocbounded{\includegraphics[keepaspectratio]{cap6-moore-tab-dupla-entrada_files/figure-pdf/unnamed-chunk-5-4.pdf}}

\subsection{Aplique seu conhecimento}\label{aplique-seu-conhecimento}

\subsubsection{6.1 Videogames e
conceitos.}\label{videogames-e-conceitos.}

A popularidade do computador, vídeo, internet e de jogos de realidade
virtual tem aumentado a preocupação sobre sua capacidade de impactar
negativamente a juventude. Os dados neste exercício se baseiam em
pesquisa recente com alunos do Ensino Médio com idade entre 14 a 18
anos, em escolas de Connecticut. Eis as distribuições das notas dos
meninos que jogaram e que não jogaram videogame.

\begin{Shaded}
\begin{Highlighting}[numbers=left,,]
\InformationTok{\textasciigrave{}\textasciigrave{}\textasciigrave{}\{r\}}
\CommentTok{\# Tabela de contingência: Conceito x Jogaram videogame}
\CommentTok{\# Linhas: status (Jogaram / Nunca jogaram)}
\CommentTok{\# Colunas: Conceito (As e Bs, Cs, Ds e Fs)}

\CommentTok{\# Montar matriz com os dados (cada linha corresponde a um status)}
\CommentTok{\# Linha 1 = "Jogaram videogame": 736, 450, 193}
\CommentTok{\# Linha 2 = "Nunca jogaram videogame": 205, 144, 80}
\NormalTok{counts }\OtherTok{\textless{}{-}} \FunctionTok{matrix}\NormalTok{(}
  \FunctionTok{c}\NormalTok{(}\DecValTok{736}\NormalTok{, }\DecValTok{450}\NormalTok{, }\DecValTok{193}\NormalTok{,}
    \DecValTok{205}\NormalTok{, }\DecValTok{144}\NormalTok{,  }\DecValTok{80}\NormalTok{),}
  \AttributeTok{nrow =} \DecValTok{2}\NormalTok{,}
  \AttributeTok{byrow =} \ConstantTok{TRUE}
\NormalTok{)}

\FunctionTok{dimnames}\NormalTok{(counts) }\OtherTok{\textless{}{-}} \FunctionTok{list}\NormalTok{(}
  \AttributeTok{Status =} \FunctionTok{c}\NormalTok{(}\StringTok{"Jogaram videogame"}\NormalTok{, }\StringTok{"Nunca jogaram videogame"}\NormalTok{),}
  \AttributeTok{Conceito =} \FunctionTok{c}\NormalTok{(}\StringTok{"As e Bs"}\NormalTok{, }\StringTok{"Cs"}\NormalTok{, }\StringTok{"Ds e Fs"}\NormalTok{)}
\NormalTok{)}

\CommentTok{\# Converter para objeto table (mantém dimnames)}
\NormalTok{tab }\OtherTok{\textless{}{-}} \FunctionTok{as.table}\NormalTok{(counts)}

\CommentTok{\# Exibir tabela original}
\FunctionTok{cat}\NormalTok{(}\StringTok{"Tabela de contingência (contagens):}\SpecialCharTok{\textbackslash{}n}\StringTok{"}\NormalTok{)}
\FunctionTok{print}\NormalTok{(tab)}
\InformationTok{\textasciigrave{}\textasciigrave{}\textasciigrave{}}
\end{Highlighting}
\end{Shaded}

\begin{verbatim}
Tabela de contingência (contagens):
                         Conceito
Status                    As e Bs  Cs Ds e Fs
  Jogaram videogame           736 450     193
  Nunca jogaram videogame     205 144      80
\end{verbatim}

(a) Quantas pessoas essa tabela descreve? Quantas delas jogaram
videogame?

(b) Dê a distribuição marginal dos conceitos. Qual percentual dos
meninos representados na tabela teve conc eito C ou menos?

\begin{Shaded}
\begin{Highlighting}[numbers=left,,]
\InformationTok{\textasciigrave{}\textasciigrave{}\textasciigrave{}\{r\}}
\CommentTok{\# Adicionar totais marginais (linha e coluna) e renomear para "Total"}
\NormalTok{tab\_totais }\OtherTok{\textless{}{-}} \FunctionTok{addmargins}\NormalTok{(tab)}
\FunctionTok{rownames}\NormalTok{(tab\_totais)[}\FunctionTok{nrow}\NormalTok{(tab\_totais)] }\OtherTok{\textless{}{-}} \StringTok{"Total"}
\FunctionTok{colnames}\NormalTok{(tab\_totais)[}\FunctionTok{ncol}\NormalTok{(tab\_totais)] }\OtherTok{\textless{}{-}} \StringTok{"Total"}

\FunctionTok{cat}\NormalTok{(}\StringTok{"}\SpecialCharTok{\textbackslash{}n}\StringTok{Tabela com totais marginais:}\SpecialCharTok{\textbackslash{}n}\StringTok{"}\NormalTok{)}
\FunctionTok{print}\NormalTok{(tab\_totais)}

\CommentTok{\# {-}{-}{-} Marginais em contagem {-}{-}{-}}
\NormalTok{margem\_status  }\OtherTok{\textless{}{-}} \FunctionTok{margin.table}\NormalTok{(tab, }\DecValTok{1}\NormalTok{)  }\CommentTok{\# totais por status (linhas)}
\NormalTok{margem\_conceito }\OtherTok{\textless{}{-}} \FunctionTok{margin.table}\NormalTok{(tab, }\DecValTok{2}\NormalTok{) }\CommentTok{\# totais por conceito (colunas)}

\FunctionTok{cat}\NormalTok{(}\StringTok{"}\SpecialCharTok{\textbackslash{}n}\StringTok{Marginal — por status (contagem):}\SpecialCharTok{\textbackslash{}n}\StringTok{"}\NormalTok{)}
\FunctionTok{print}\NormalTok{(margem\_status)}
\FunctionTok{cat}\NormalTok{(}\StringTok{"}\SpecialCharTok{\textbackslash{}n}\StringTok{Marginal — por conceito (contagem):}\SpecialCharTok{\textbackslash{}n}\StringTok{"}\NormalTok{)}
\FunctionTok{print}\NormalTok{(margem\_conceito)}

\CommentTok{\# {-}{-}{-} Marginais em proporção (relativas ao total geral) {-}{-}{-}}
\NormalTok{prop\_status  }\OtherTok{\textless{}{-}} \FunctionTok{prop.table}\NormalTok{(margem\_status)   }\CommentTok{\# soma = 1}
\NormalTok{prop\_conceito }\OtherTok{\textless{}{-}} \FunctionTok{prop.table}\NormalTok{(margem\_conceito)}

\FunctionTok{cat}\NormalTok{(}\StringTok{"}\SpecialCharTok{\textbackslash{}n}\StringTok{Marginal — por status (porcentagem do total):}\SpecialCharTok{\textbackslash{}n}\StringTok{"}\NormalTok{)}
\FunctionTok{print}\NormalTok{(}\FunctionTok{round}\NormalTok{(}\DecValTok{100} \SpecialCharTok{*}\NormalTok{ prop\_status, }\DecValTok{2}\NormalTok{))}
\FunctionTok{cat}\NormalTok{(}\StringTok{"}\SpecialCharTok{\textbackslash{}n}\StringTok{Marginal — por conceito (porcentagem do total):}\SpecialCharTok{\textbackslash{}n}\StringTok{"}\NormalTok{)}
\FunctionTok{print}\NormalTok{(}\FunctionTok{round}\NormalTok{(}\DecValTok{100} \SpecialCharTok{*}\NormalTok{ prop\_conceito, }\DecValTok{2}\NormalTok{))}

\CommentTok{\# {-}{-}{-} Proporções por célula e condicionais {-}{-}{-}}
\NormalTok{prop\_celula\_total }\OtherTok{\textless{}{-}} \FunctionTok{prop.table}\NormalTok{(tab)    }\CommentTok{\# cada célula / total geral}
\NormalTok{prop\_cond\_status  }\OtherTok{\textless{}{-}} \FunctionTok{prop.table}\NormalTok{(tab, }\DecValTok{1}\NormalTok{) }\CommentTok{\# condicional por linha (status)}
\NormalTok{prop\_cond\_conceito }\OtherTok{\textless{}{-}} \FunctionTok{prop.table}\NormalTok{(tab, }\DecValTok{2}\NormalTok{)}\CommentTok{\# condicional por coluna (conceito)}

\FunctionTok{cat}\NormalTok{(}\StringTok{"}\SpecialCharTok{\textbackslash{}n}\StringTok{Proporção de cada célula em relação ao total (em \%):}\SpecialCharTok{\textbackslash{}n}\StringTok{"}\NormalTok{)}
\FunctionTok{print}\NormalTok{(}\FunctionTok{round}\NormalTok{(}\DecValTok{100} \SpecialCharTok{*}\NormalTok{ prop\_celula\_total, }\DecValTok{2}\NormalTok{))}
\FunctionTok{cat}\NormalTok{(}\StringTok{"}\SpecialCharTok{\textbackslash{}n}\StringTok{Proporção condicional por status (cada linha = 100\%):}\SpecialCharTok{\textbackslash{}n}\StringTok{"}\NormalTok{)}
\FunctionTok{print}\NormalTok{(}\FunctionTok{round}\NormalTok{(}\DecValTok{100} \SpecialCharTok{*}\NormalTok{ prop\_cond\_status, }\DecValTok{2}\NormalTok{))}
\FunctionTok{cat}\NormalTok{(}\StringTok{"}\SpecialCharTok{\textbackslash{}n}\StringTok{Proporção condicional por conceito (cada coluna = 100\%):}\SpecialCharTok{\textbackslash{}n}\StringTok{"}\NormalTok{)}
\FunctionTok{print}\NormalTok{(}\FunctionTok{round}\NormalTok{(}\DecValTok{100} \SpecialCharTok{*}\NormalTok{ prop\_cond\_conceito, }\DecValTok{2}\NormalTok{))}

\CommentTok{\# {-}{-}{-} Data.frames resumidos (para relatório/exportação) {-}{-}{-}}
\NormalTok{resumo\_status }\OtherTok{\textless{}{-}} \FunctionTok{data.frame}\NormalTok{(}
  \AttributeTok{status =} \FunctionTok{names}\NormalTok{(margem\_status),}
  \AttributeTok{contagem =} \FunctionTok{as.integer}\NormalTok{(margem\_status),}
  \AttributeTok{proporcao\_total =} \FunctionTok{round}\NormalTok{(}\FunctionTok{as.numeric}\NormalTok{(prop\_status), }\DecValTok{4}\NormalTok{),}
  \AttributeTok{porcentagem =} \FunctionTok{round}\NormalTok{(}\DecValTok{100} \SpecialCharTok{*} \FunctionTok{as.numeric}\NormalTok{(prop\_status), }\DecValTok{2}\NormalTok{)}
\NormalTok{)}

\NormalTok{resumo\_conceito }\OtherTok{\textless{}{-}} \FunctionTok{data.frame}\NormalTok{(}
  \AttributeTok{conceito =} \FunctionTok{names}\NormalTok{(margem\_conceito),}
  \AttributeTok{contagem =} \FunctionTok{as.integer}\NormalTok{(margem\_conceito),}
  \AttributeTok{proporcao\_total =} \FunctionTok{round}\NormalTok{(}\FunctionTok{as.numeric}\NormalTok{(prop\_conceito), }\DecValTok{4}\NormalTok{),}
  \AttributeTok{porcentagem =} \FunctionTok{round}\NormalTok{(}\DecValTok{100} \SpecialCharTok{*} \FunctionTok{as.numeric}\NormalTok{(prop\_conceito), }\DecValTok{2}\NormalTok{)}
\NormalTok{)}

\FunctionTok{cat}\NormalTok{(}\StringTok{"}\SpecialCharTok{\textbackslash{}n}\StringTok{Resumo por status:}\SpecialCharTok{\textbackslash{}n}\StringTok{"}\NormalTok{)}
\FunctionTok{print}\NormalTok{(resumo\_status)}
\FunctionTok{cat}\NormalTok{(}\StringTok{"}\SpecialCharTok{\textbackslash{}n}\StringTok{Resumo por conceito:}\SpecialCharTok{\textbackslash{}n}\StringTok{"}\NormalTok{)}
\FunctionTok{print}\NormalTok{(resumo\_conceito)}
\InformationTok{\textasciigrave{}\textasciigrave{}\textasciigrave{}}
\end{Highlighting}
\end{Shaded}

\begin{verbatim}

Tabela com totais marginais:
                         Conceito
Status                    As e Bs   Cs Ds e Fs Total
  Jogaram videogame           736  450     193  1379
  Nunca jogaram videogame     205  144      80   429
  Total                       941  594     273  1808

Marginal — por status (contagem):
Status
      Jogaram videogame Nunca jogaram videogame 
                   1379                     429 

Marginal — por conceito (contagem):
Conceito
As e Bs      Cs Ds e Fs 
    941     594     273 

Marginal — por status (porcentagem do total):
Status
      Jogaram videogame Nunca jogaram videogame 
                     76                      24 

Marginal — por conceito (porcentagem do total):
Conceito
As e Bs      Cs Ds e Fs 
     52      33      15 

Proporção de cada célula em relação ao total (em %):
                         Conceito
Status                    As e Bs   Cs Ds e Fs
  Jogaram videogame          40.7 24.9    10.7
  Nunca jogaram videogame    11.3  8.0     4.4

Proporção condicional por status (cada linha = 100%):
                         Conceito
Status                    As e Bs Cs Ds e Fs
  Jogaram videogame            53 33      14
  Nunca jogaram videogame      48 34      19

Proporção condicional por conceito (cada coluna = 100%):
                         Conceito
Status                    As e Bs Cs Ds e Fs
  Jogaram videogame            78 76      71
  Nunca jogaram videogame      22 24      29

Resumo por status:
                   status contagem proporcao_total porcentagem
1       Jogaram videogame     1379            0.76          76
2 Nunca jogaram videogame      429            0.24          24

Resumo por conceito:
  conceito contagem proporcao_total porcentagem
1  As e Bs      941            0.52          52
2       Cs      594            0.33          33
3  Ds e Fs      273            0.15          15
\end{verbatim}

Olhando para as saídas acima, temos as respostas.

(a) Total de pessoas descritas pela tabela: 1808

Total de pessoas que jogaram videogame: 1379

(b) A partir da distribuição marginal dos conceitos, o percentual de
meninos com conceito C ou menos é: 32,85\% + 15,10\% = 47,96\%.

Ou seja, a maioria dos meninos que jogam videogame apresentam conceito A
ou B: 52,05\%.

\subsubsection{6.2 Idades de
Universitários.}\label{idades-de-universituxe1rios.}

Eis uma tabela de dupla entrada de dados do U.S. Census Bureau que
descreve a idade e o gênero de todos os alunos universitários
americanos. As entradas na tabela são contagens em milhares de
estudantes.

\begin{Shaded}
\begin{Highlighting}[numbers=left,,]
\InformationTok{\textasciigrave{}\textasciigrave{}\textasciigrave{}\{r\}}
\CommentTok{\# Montar tabela Faixa etária x Sexo}
\CommentTok{\# Linhas: faixas etárias (na ordem desejada)}
\CommentTok{\# Colunas: Mulher, Homem}

\NormalTok{counts }\OtherTok{\textless{}{-}} \FunctionTok{matrix}\NormalTok{(}
  \FunctionTok{c}\NormalTok{(}
    \DecValTok{2348}\NormalTok{, }\DecValTok{1831}\NormalTok{,  }\CommentTok{\# 15 a 19 anos: Mulher, Homem}
    \DecValTok{4280}\NormalTok{, }\DecValTok{3713}\NormalTok{,  }\CommentTok{\# 20 a 24 anos}
    \DecValTok{2166}\NormalTok{, }\DecValTok{1714}\NormalTok{,  }\CommentTok{\# 25 a 34 anos}
    \DecValTok{1492}\NormalTok{,  }\DecValTok{853}   \CommentTok{\# 35 anos ou mais}
\NormalTok{  ),}
  \AttributeTok{nrow =} \DecValTok{4}\NormalTok{,}
  \AttributeTok{byrow =} \ConstantTok{TRUE}
\NormalTok{)}

\FunctionTok{rownames}\NormalTok{(counts) }\OtherTok{\textless{}{-}} \FunctionTok{c}\NormalTok{(}\StringTok{"15 a 19 anos"}\NormalTok{, }\StringTok{"20 a 24 anos"}\NormalTok{, }\StringTok{"25 a 34 anos"}\NormalTok{, }\StringTok{"35 anos ou mais"}\NormalTok{)}
\FunctionTok{colnames}\NormalTok{(counts) }\OtherTok{\textless{}{-}} \FunctionTok{c}\NormalTok{(}\StringTok{"Mulher"}\NormalTok{, }\StringTok{"Homem"}\NormalTok{)}

\CommentTok{\# Converter para objeto table (mantém dimnames)}
\NormalTok{tab\_faixa }\OtherTok{\textless{}{-}} \FunctionTok{as.table}\NormalTok{(counts)}

\CommentTok{\# Exibir tabela original}
\FunctionTok{cat}\NormalTok{(}\StringTok{"Tabela Faixa etária x Sexo (contagens):}\SpecialCharTok{\textbackslash{}n}\StringTok{"}\NormalTok{)}
\FunctionTok{print}\NormalTok{(tab\_faixa)}
\InformationTok{\textasciigrave{}\textasciigrave{}\textasciigrave{}}
\end{Highlighting}
\end{Shaded}

\begin{verbatim}
Tabela Faixa etária x Sexo (contagens):
                Mulher Homem
15 a 19 anos      2348  1831
20 a 24 anos      4280  3713
25 a 34 anos      2166  1714
35 anos ou mais   1492   853
\end{verbatim}

(a) Há quantos alunos universitários?

(b) Encontre a distribuição marginal das faixas etárias. Qual percentual
de universitários está na faixa etária de 20 a 24 anos?

\begin{Shaded}
\begin{Highlighting}[numbers=left,,]
\InformationTok{\textasciigrave{}\textasciigrave{}\textasciigrave{}\{r\}}
\CommentTok{\# Adicionar totais marginais (linhas e colunas)}
\NormalTok{tab\_com\_totais }\OtherTok{\textless{}{-}} \FunctionTok{addmargins}\NormalTok{(tab\_faixa)}
\CommentTok{\# Renomear as margens para "Total" (substitui o rótulo padrão)}
\FunctionTok{rownames}\NormalTok{(tab\_com\_totais)[}\FunctionTok{nrow}\NormalTok{(tab\_com\_totais)] }\OtherTok{\textless{}{-}} \StringTok{"Total"}
\FunctionTok{colnames}\NormalTok{(tab\_com\_totais)[}\FunctionTok{ncol}\NormalTok{(tab\_com\_totais)] }\OtherTok{\textless{}{-}} \StringTok{"Total"}

\FunctionTok{cat}\NormalTok{(}\StringTok{"}\SpecialCharTok{\textbackslash{}n}\StringTok{Tabela com totais marginais:}\SpecialCharTok{\textbackslash{}n}\StringTok{"}\NormalTok{)}
\FunctionTok{print}\NormalTok{(tab\_com\_totais)}

\CommentTok{\# Marginais em contagem}
\NormalTok{totais\_por\_faixa }\OtherTok{\textless{}{-}} \FunctionTok{margin.table}\NormalTok{(tab\_faixa, }\DecValTok{1}\NormalTok{)  }\CommentTok{\# soma por linha (faixa etária)}
\NormalTok{totais\_por\_sexo  }\OtherTok{\textless{}{-}} \FunctionTok{margin.table}\NormalTok{(tab\_faixa, }\DecValTok{2}\NormalTok{)  }\CommentTok{\# soma por coluna (sexo)}

\FunctionTok{cat}\NormalTok{(}\StringTok{"}\SpecialCharTok{\textbackslash{}n}\StringTok{Totais por faixa etária:}\SpecialCharTok{\textbackslash{}n}\StringTok{"}\NormalTok{); }\FunctionTok{print}\NormalTok{(totais\_por\_faixa)}
\FunctionTok{cat}\NormalTok{(}\StringTok{"}\SpecialCharTok{\textbackslash{}n}\StringTok{Totais por sexo:}\SpecialCharTok{\textbackslash{}n}\StringTok{"}\NormalTok{); }\FunctionTok{print}\NormalTok{(totais\_por\_sexo)}

\CommentTok{\# Proporções}
\NormalTok{prop\_total }\OtherTok{\textless{}{-}} \FunctionTok{prop.table}\NormalTok{(tab\_faixa)              }\CommentTok{\# cada célula / total geral}
\NormalTok{prop\_por\_faixa }\OtherTok{\textless{}{-}} \FunctionTok{prop.table}\NormalTok{(tab\_faixa, }\DecValTok{1}\NormalTok{)       }\CommentTok{\# condicional por faixa (linhas somam 1)}
\NormalTok{prop\_por\_sexo  }\OtherTok{\textless{}{-}} \FunctionTok{prop.table}\NormalTok{(tab\_faixa, }\DecValTok{2}\NormalTok{)       }\CommentTok{\# condicional por sexo (colunas somam 1)}

\FunctionTok{cat}\NormalTok{(}\StringTok{"}\SpecialCharTok{\textbackslash{}n}\StringTok{Proporção de cada célula em relação ao total (em \%):}\SpecialCharTok{\textbackslash{}n}\StringTok{"}\NormalTok{)}
\FunctionTok{print}\NormalTok{(}\FunctionTok{round}\NormalTok{(}\DecValTok{100} \SpecialCharTok{*}\NormalTok{ prop\_total, }\DecValTok{2}\NormalTok{))}

\CommentTok{\# Converter para data.frame em formato "long" (útil para ggplot2 ou export)}
\NormalTok{df\_long }\OtherTok{\textless{}{-}} \FunctionTok{as.data.frame}\NormalTok{(tab\_faixa)}
\FunctionTok{names}\NormalTok{(df\_long) }\OtherTok{\textless{}{-}} \FunctionTok{c}\NormalTok{(}\StringTok{"faixa\_etaria"}\NormalTok{, }\StringTok{"sexo"}\NormalTok{, }\StringTok{"contagem"}\NormalTok{)}

\FunctionTok{cat}\NormalTok{(}\StringTok{"}\SpecialCharTok{\textbackslash{}n}\StringTok{Data.frame (long) pronto para plotagem/exportação:}\SpecialCharTok{\textbackslash{}n}\StringTok{"}\NormalTok{)}
\FunctionTok{print}\NormalTok{(df\_long)}
\InformationTok{\textasciigrave{}\textasciigrave{}\textasciigrave{}}
\end{Highlighting}
\end{Shaded}

\begin{verbatim}

Tabela com totais marginais:
                Mulher Homem Total
15 a 19 anos      2348  1831  4179
20 a 24 anos      4280  3713  7993
25 a 34 anos      2166  1714  3880
35 anos ou mais   1492   853  2345
Total            10286  8111 18397

Totais por faixa etária:
   15 a 19 anos    20 a 24 anos    25 a 34 anos 35 anos ou mais 
           4179            7993            3880            2345 

Totais por sexo:
Mulher  Homem 
 10286   8111 

Proporção de cada célula em relação ao total (em %):
                Mulher Homem
15 a 19 anos      12.8   9.9
20 a 24 anos      23.3  20.2
25 a 34 anos      11.8   9.3
35 anos ou mais    8.1   4.6

Data.frame (long) pronto para plotagem/exportação:
     faixa_etaria   sexo contagem
1    15 a 19 anos Mulher     2348
2    20 a 24 anos Mulher     4280
3    25 a 34 anos Mulher     2166
4 35 anos ou mais Mulher     1492
5    15 a 19 anos  Homem     1831
6    20 a 24 anos  Homem     3713
7    25 a 34 anos  Homem     1714
8 35 anos ou mais  Homem      853
\end{verbatim}

Respostas:

(a) Há 18397 alunos universitários no total.

\begin{Shaded}
\begin{Highlighting}[numbers=left,,]
\InformationTok{\textasciigrave{}\textasciigrave{}\textasciigrave{}\{r\}}
\CommentTok{\# {-}{-}{-} Distribuição marginal por faixa etária (contagens) {-}{-}{-}}
\NormalTok{margem\_faixa }\OtherTok{\textless{}{-}} \FunctionTok{margin.table}\NormalTok{(tab\_faixa, }\DecValTok{1}\NormalTok{)   }\CommentTok{\# soma sobre colunas =\textgreater{} total por faixa}
\FunctionTok{cat}\NormalTok{(}\StringTok{"}\SpecialCharTok{\textbackslash{}n}\StringTok{Marginal (contagem) por faixa etária:}\SpecialCharTok{\textbackslash{}n}\StringTok{"}\NormalTok{)}
\FunctionTok{print}\NormalTok{(margem\_faixa)}

\CommentTok{\# {-}{-}{-} Proporção e porcentagem da distribuição marginal {-}{-}{-}}
\NormalTok{prop\_faixa }\OtherTok{\textless{}{-}} \FunctionTok{prop.table}\NormalTok{(margem\_faixa)       }\CommentTok{\# proporção relativa ao total geral (soma = 1)}
\NormalTok{pct\_faixa  }\OtherTok{\textless{}{-}} \DecValTok{100} \SpecialCharTok{*}\NormalTok{ prop\_faixa               }\CommentTok{\# porcentagem}

\FunctionTok{cat}\NormalTok{(}\StringTok{"}\SpecialCharTok{\textbackslash{}n}\StringTok{Marginal (proporção) por faixa etária (em \%):}\SpecialCharTok{\textbackslash{}n}\StringTok{"}\NormalTok{)}
\FunctionTok{print}\NormalTok{(}\FunctionTok{round}\NormalTok{(}\DecValTok{100} \SpecialCharTok{*}\NormalTok{ prop\_faixa, }\DecValTok{2}\NormalTok{))            }\CommentTok{\# exibir porcentagem com 2 casas}
\InformationTok{\textasciigrave{}\textasciigrave{}\textasciigrave{}}
\end{Highlighting}
\end{Shaded}

\begin{verbatim}

Marginal (contagem) por faixa etária:
   15 a 19 anos    20 a 24 anos    25 a 34 anos 35 anos ou mais 
           4179            7993            3880            2345 

Marginal (proporção) por faixa etária (em %):
   15 a 19 anos    20 a 24 anos    25 a 34 anos 35 anos ou mais 
             23              43              21              13 
\end{verbatim}

(b) A partir da distribuição marginal das faixas etárias, vê-se que o
percentual de universitários que está na faixa estária de 20 a 24 anos
é: 43,45\%.

\section{Distribuições
condicionais}\label{distribuiuxe7uxf5es-condicionais}

\begin{quote}
A Tabela 6.1 contém muito mais informação do que as duas distribuições
marginais de sexo e de grau conferido, separadas. Distribuições
marginais nada dizem sobre a relação entre duas variáveis. Para
descrevermos uma relação entre duas variáveis categóricas, devemos
calcular alguns percentuais bem escolhidos a partir das contagens
mostradas no corpo da tabela.

Digamos que você deseje comparar as proporções de mulheres e de homens
que recebem um grau de doutor. Para isso, compare os percentuais para
cada categoria de sexo. Para estudar as mulheres, examinamos apenas a
linha ``Mulheres'' na Tabela 6.1. Para encontrar o percentual de
mulheres que recebem o grau de doutor, divida a contagem dessas mulheres
pelo número total de mulheres (total da linha):

\[
\frac{\text{mulheres que recebem um grau de doutorado}}{\text{total da linha}} = \frac{97}{2283} = 0,042 = 4,2\%
\]

Fazendo isso para todas as quatro entradas na linha ``Mulheres'' obtemos
a \textbf{\emph{distribuição condicional de graus conferidos entre as
mulheres}}. Usamos o termo \ul{\textbf{\emph{condicional}}} porque essa
distribuição \textbf{\emph{descreve apenas estudantes que satisfazem a
condição de serem mulheres}}. (MOORE; NOTZ; FLIGNER, 2023 , p.~132)
\end{quote}

\begin{tcolorbox}[enhanced jigsaw, arc=.35mm, opacitybacktitle=0.6, colframe=quarto-callout-important-color-frame, titlerule=0mm, leftrule=.75mm, left=2mm, colbacktitle=quarto-callout-important-color!10!white, breakable, toprule=.15mm, bottomtitle=1mm, opacityback=0, coltitle=black, title=\textcolor{quarto-callout-important-color}{\faExclamation}\hspace{0.5em}{Distribuições condicionais}, rightrule=.15mm, bottomrule=.15mm, toptitle=1mm, colback=white]

Uma distribuição condicional de uma variável é a
\textbf{\emph{distribuição dos valores daquela variável apenas entre os
indivíduos que têm determinado valor na outra variável}}.

\textbf{\emph{Há uma distribuição condicional distinta para cada valor
da outra variável.}}

\end{tcolorbox}

\subsection{Exemplo 6.3 Comparação de mulheres e
homens}\label{exemplo-6.3-comparauxe7uxe3o-de-mulheres-e-homens}

\texttt{ESTABELEÇA}: como diferem homens e mulheres em relação aos graus
que pretendiam receber no período de 2020 e 2021?~

\texttt{PLANEJE}: faça uma tabela de dupla entrada das respostas pela
categoria sexo. Encontre a distribuição condicional para cada categoria
de sexo. Compare essas duas distribuições.

\texttt{RESOLVA}: a Tabela 6.1 é a tabela de dupla entrada de que
precisamos. Olhe primeiro apenas para a linha ``Mulheres'' para
encontrar a distribuição condicional para mulheres; depois, apenas para
a linha ``Homens'' para encontrar a distribuição condicional para
homens. Eis os cálculos e as duas distribuições condicionais:

\begin{figure}[H]

{\centering \pandocbounded{\includegraphics[keepaspectratio]{fig/Moore-tab-ex-6.3-comp-sexo-M-H-por-graus.png}}

}

\caption{Duas distribuições condicionais de graus por sexo (M, H)}

\end{figure}%

Gerar gráficos de barras empilhadas lado a lado para essas duas
distribuições condicionais.

\begin{Shaded}
\begin{Highlighting}[numbers=left,,]
\InformationTok{\textasciigrave{}\textasciigrave{}\textasciigrave{}\{r\}}
\CommentTok{\# Criar data.frame com fatores e ordem explícita dos níveis}
\NormalTok{grausex }\OtherTok{\textless{}{-}} \FunctionTok{data.frame}\NormalTok{(}
  \AttributeTok{sex  =} \FunctionTok{factor}\NormalTok{(}\FunctionTok{c}\NormalTok{(}\FunctionTok{rep}\NormalTok{(}\StringTok{"M"}\NormalTok{, }\DecValTok{2283}\NormalTok{), }\FunctionTok{rep}\NormalTok{(}\StringTok{"H"}\NormalTok{, }\DecValTok{1622}\NormalTok{)), }\AttributeTok{levels =} \FunctionTok{c}\NormalTok{(}\StringTok{"M"}\NormalTok{, }\StringTok{"H"}\NormalTok{)),}
  \AttributeTok{grau =} \FunctionTok{factor}\NormalTok{(}\FunctionTok{c}\NormalTok{(}\FunctionTok{rep}\NormalTok{(}\StringTok{"ass"}\NormalTok{, }\DecValTok{639}\NormalTok{), }\FunctionTok{rep}\NormalTok{(}\StringTok{"bach"}\NormalTok{, }\DecValTok{1087}\NormalTok{), }\FunctionTok{rep}\NormalTok{(}\StringTok{"ms"}\NormalTok{, }\DecValTok{460}\NormalTok{), }\FunctionTok{rep}\NormalTok{(}\StringTok{"dr"}\NormalTok{, }\DecValTok{97}\NormalTok{),}
                  \FunctionTok{rep}\NormalTok{(}\StringTok{"ass"}\NormalTok{, }\DecValTok{402}\NormalTok{), }\FunctionTok{rep}\NormalTok{(}\StringTok{"bach"}\NormalTok{,  }\DecValTok{804}\NormalTok{), }\FunctionTok{rep}\NormalTok{(}\StringTok{"ms"}\NormalTok{, }\DecValTok{329}\NormalTok{), }\FunctionTok{rep}\NormalTok{(}\StringTok{"dr"}\NormalTok{, }\DecValTok{87}\NormalTok{)),}
                \AttributeTok{levels =} \FunctionTok{c}\NormalTok{(}\StringTok{"ass"}\NormalTok{, }\StringTok{"bach"}\NormalTok{, }\StringTok{"ms"}\NormalTok{, }\StringTok{"dr"}\NormalTok{))}
\NormalTok{)}

\CommentTok{\# Tabela com linhas = sexo e colunas = grau (contagens)}
\NormalTok{tab }\OtherTok{\textless{}{-}} \FunctionTok{table}\NormalTok{(grausex}\SpecialCharTok{$}\NormalTok{sex, grausex}\SpecialCharTok{$}\NormalTok{grau)}
\FunctionTok{cat}\NormalTok{(}\StringTok{"Tabela de contagens (linhas = sexo, colunas = grau):}\SpecialCharTok{\textbackslash{}n}\StringTok{"}\NormalTok{)}
\FunctionTok{print}\NormalTok{(tab)}

\CommentTok{\# Distribuição condicional de grau dado o sexo:}
\CommentTok{\# para cada linha (sexo) as proporções somam 1}
\NormalTok{prop\_condicional }\OtherTok{\textless{}{-}} \FunctionTok{prop.table}\NormalTok{(tab, }\AttributeTok{margin =} \DecValTok{1}\NormalTok{)}

\FunctionTok{cat}\NormalTok{(}\StringTok{"}\SpecialCharTok{\textbackslash{}n}\StringTok{Distribuição condicional (grau | sexo) {-} proporções:}\SpecialCharTok{\textbackslash{}n}\StringTok{"}\NormalTok{)}
\FunctionTok{print}\NormalTok{(}\FunctionTok{round}\NormalTok{(prop\_condicional, }\DecValTok{4}\NormalTok{))   }\CommentTok{\# 4 casas decimais}

\FunctionTok{cat}\NormalTok{(}\StringTok{"}\SpecialCharTok{\textbackslash{}n}\StringTok{Distribuição condicional (grau | sexo) {-} porcentagens:}\SpecialCharTok{\textbackslash{}n}\StringTok{"}\NormalTok{)}
\FunctionTok{print}\NormalTok{(}\FunctionTok{round}\NormalTok{(}\DecValTok{100} \SpecialCharTok{*}\NormalTok{ prop\_condicional, }\DecValTok{2}\NormalTok{))  }\CommentTok{\# em \%}

\CommentTok{\# Data.frame com contagens e proporções (útil para relatórios/plotagem)}
\NormalTok{df\_counts }\OtherTok{\textless{}{-}} \FunctionTok{as.data.frame}\NormalTok{(tab)}
\FunctionTok{names}\NormalTok{(df\_counts) }\OtherTok{\textless{}{-}} \FunctionTok{c}\NormalTok{(}\StringTok{"sexo"}\NormalTok{, }\StringTok{"grau"}\NormalTok{, }\StringTok{"contagem"}\NormalTok{)}

\NormalTok{df\_props }\OtherTok{\textless{}{-}} \FunctionTok{as.data.frame}\NormalTok{(prop\_condicional)}
\FunctionTok{names}\NormalTok{(df\_props) }\OtherTok{\textless{}{-}} \FunctionTok{c}\NormalTok{(}\StringTok{"sexo"}\NormalTok{, }\StringTok{"grau"}\NormalTok{, }\StringTok{"proporcao"}\NormalTok{)}

\NormalTok{df\_condicional }\OtherTok{\textless{}{-}} \FunctionTok{merge}\NormalTok{(df\_counts, df\_props, }\AttributeTok{by =} \FunctionTok{c}\NormalTok{(}\StringTok{"sexo"}\NormalTok{, }\StringTok{"grau"}\NormalTok{))}
\NormalTok{df\_condicional}\SpecialCharTok{$}\NormalTok{porcentagem }\OtherTok{\textless{}{-}} \FunctionTok{round}\NormalTok{(}\DecValTok{100} \SpecialCharTok{*}\NormalTok{ df\_condicional}\SpecialCharTok{$}\NormalTok{proporcao, }\DecValTok{2}\NormalTok{)}

\CommentTok{\# Preservar ordem dos fatores e ordenar para exibição}
\NormalTok{df\_condicional}\SpecialCharTok{$}\NormalTok{sexo }\OtherTok{\textless{}{-}} \FunctionTok{factor}\NormalTok{(df\_condicional}\SpecialCharTok{$}\NormalTok{sexo, }\AttributeTok{levels =} \FunctionTok{levels}\NormalTok{(grausex}\SpecialCharTok{$}\NormalTok{sex))}
\NormalTok{df\_condicional}\SpecialCharTok{$}\NormalTok{grau }\OtherTok{\textless{}{-}} \FunctionTok{factor}\NormalTok{(df\_condicional}\SpecialCharTok{$}\NormalTok{grau, }\AttributeTok{levels =} \FunctionTok{levels}\NormalTok{(grausex}\SpecialCharTok{$}\NormalTok{grau))}
\NormalTok{df\_condicional }\OtherTok{\textless{}{-}}\NormalTok{ df\_condicional[}\FunctionTok{order}\NormalTok{(df\_condicional}\SpecialCharTok{$}\NormalTok{sexo, df\_condicional}\SpecialCharTok{$}\NormalTok{grau), ]}

\FunctionTok{cat}\NormalTok{(}\StringTok{"}\SpecialCharTok{\textbackslash{}n}\StringTok{Data.frame com contagem + proporção condicional (ordenado):}\SpecialCharTok{\textbackslash{}n}\StringTok{"}\NormalTok{)}
\FunctionTok{print}\NormalTok{(df\_condicional)}

\CommentTok{\# {-}{-}{-} Opcional: gráfico (ggplot2) mostrando proporção de graus por sexo {-}{-}{-}}
\FunctionTok{ggplot}\NormalTok{(df\_condicional, }\FunctionTok{aes}\NormalTok{(}\AttributeTok{x =}\NormalTok{ sexo, }\AttributeTok{y =}\NormalTok{ proporcao, }\AttributeTok{fill =}\NormalTok{ grau)) }\SpecialCharTok{+}
  \FunctionTok{geom\_col}\NormalTok{(}\AttributeTok{position =} \StringTok{"stack"}\NormalTok{) }\SpecialCharTok{+}                         \CommentTok{\# altura = proporção por sexo (soma = 1)}
  \FunctionTok{scale\_y\_continuous}\NormalTok{(}\AttributeTok{labels =}\NormalTok{ scales}\SpecialCharTok{::}\FunctionTok{percent\_format}\NormalTok{(}\AttributeTok{accuracy =} \DecValTok{1}\NormalTok{)) }\SpecialCharTok{+}
  \FunctionTok{labs}\NormalTok{(}\AttributeTok{title =} \StringTok{"Distribuição condicional: Grau dado o Sexo"}\NormalTok{,}
       \AttributeTok{x =} \StringTok{"Sexo"}\NormalTok{, }\AttributeTok{y =} \StringTok{"Proporção (grau | sexo)"}\NormalTok{) }\SpecialCharTok{+}
  \FunctionTok{theme\_minimal}\NormalTok{()}
\InformationTok{\textasciigrave{}\textasciigrave{}\textasciigrave{}}
\end{Highlighting}
\end{Shaded}

\begin{verbatim}
Tabela de contagens (linhas = sexo, colunas = grau):
   
     ass bach   ms   dr
  M  639 1087  460   97
  H  402  804  329   87

Distribuição condicional (grau | sexo) - proporções:
   
      ass  bach    ms    dr
  M 0.280 0.476 0.202 0.042
  H 0.248 0.496 0.203 0.054

Distribuição condicional (grau | sexo) - porcentagens:
   
     ass bach   ms   dr
  M 28.0 47.6 20.1  4.2
  H 24.8 49.6 20.3  5.4

Data.frame com contagem + proporção condicional (ordenado):
  sexo grau contagem proporcao porcentagem
5    M  ass      639     0.280        28.0
6    M bach     1087     0.476        47.6
8    M   ms      460     0.201        20.1
7    M   dr       97     0.042         4.2
1    H  ass      402     0.248        24.8
2    H bach      804     0.496        49.6
4    H   ms      329     0.203        20.3
3    H   dr       87     0.054         5.4
\end{verbatim}

\pandocbounded{\includegraphics[keepaspectratio]{cap6-moore-tab-dupla-entrada_files/figure-pdf/unnamed-chunk-11-1.pdf}}

\textbf{\emph{As porcentagens em cada linha devem somar 100\% porque,
para cada categoria de sexo, todos recebem um, {[}e somente um{]}, dos
quatro graus}}.

No entanto, em geral, as porcentagens podem não ter soma exatamente
100\% porque arredondamos para um número fixo de casas decimais. Esse é
o \ul{\textbf{erro de arredondamento}}, e vemos que há erro de
arredondamento aqui.

O mesmo gráfico mais elaborado: barras empilhadas lado a lado para essas
duas distribuições condicionais com indicação dos percentuais de cada
classe de \texttt{graus} dentro de cada classe de \texttt{sexo}.

\begin{Shaded}
\begin{Highlighting}[numbers=left,,]
\InformationTok{\textasciigrave{}\textasciigrave{}\textasciigrave{}\{r\}}
\CommentTok{\# Pacotes necessários}
\ControlFlowTok{if}\NormalTok{ (}\SpecialCharTok{!}\FunctionTok{requireNamespace}\NormalTok{(}\StringTok{"ggplot2"}\NormalTok{, }\AttributeTok{quietly =} \ConstantTok{TRUE}\NormalTok{)) }\FunctionTok{install.packages}\NormalTok{(}\StringTok{"ggplot2"}\NormalTok{)}
\ControlFlowTok{if}\NormalTok{ (}\SpecialCharTok{!}\FunctionTok{requireNamespace}\NormalTok{(}\StringTok{"scales"}\NormalTok{, }\AttributeTok{quietly =} \ConstantTok{TRUE}\NormalTok{)) }\FunctionTok{install.packages}\NormalTok{(}\StringTok{"scales"}\NormalTok{)}
\FunctionTok{library}\NormalTok{(ggplot2)}
\FunctionTok{library}\NormalTok{(scales)}

\CommentTok{\# Criar data.frame com fatores e ordem explícita dos níveis (preserva ordem nos gráficos)}
\NormalTok{grausex }\OtherTok{\textless{}{-}} \FunctionTok{data.frame}\NormalTok{(}
  \AttributeTok{sex  =} \FunctionTok{factor}\NormalTok{(}\FunctionTok{c}\NormalTok{(}\FunctionTok{rep}\NormalTok{(}\StringTok{"M"}\NormalTok{, }\DecValTok{2283}\NormalTok{), }\FunctionTok{rep}\NormalTok{(}\StringTok{"H"}\NormalTok{, }\DecValTok{1622}\NormalTok{)), }\AttributeTok{levels =} \FunctionTok{c}\NormalTok{(}\StringTok{"M"}\NormalTok{, }\StringTok{"H"}\NormalTok{)),}
  \AttributeTok{grau =} \FunctionTok{factor}\NormalTok{(}\FunctionTok{c}\NormalTok{(}\FunctionTok{rep}\NormalTok{(}\StringTok{"ass"}\NormalTok{, }\DecValTok{639}\NormalTok{), }\FunctionTok{rep}\NormalTok{(}\StringTok{"bach"}\NormalTok{, }\DecValTok{1087}\NormalTok{), }\FunctionTok{rep}\NormalTok{(}\StringTok{"ms"}\NormalTok{, }\DecValTok{460}\NormalTok{), }\FunctionTok{rep}\NormalTok{(}\StringTok{"dr"}\NormalTok{, }\DecValTok{97}\NormalTok{),}
                  \FunctionTok{rep}\NormalTok{(}\StringTok{"ass"}\NormalTok{, }\DecValTok{402}\NormalTok{), }\FunctionTok{rep}\NormalTok{(}\StringTok{"bach"}\NormalTok{,  }\DecValTok{804}\NormalTok{), }\FunctionTok{rep}\NormalTok{(}\StringTok{"ms"}\NormalTok{, }\DecValTok{329}\NormalTok{), }\FunctionTok{rep}\NormalTok{(}\StringTok{"dr"}\NormalTok{, }\DecValTok{87}\NormalTok{)),}
                \AttributeTok{levels =} \FunctionTok{c}\NormalTok{(}\StringTok{"ass"}\NormalTok{, }\StringTok{"bach"}\NormalTok{, }\StringTok{"ms"}\NormalTok{, }\StringTok{"dr"}\NormalTok{))}
\NormalTok{)}

\CommentTok{\# Preparar data.frame de contagens por sexo x grau}
\NormalTok{df }\OtherTok{\textless{}{-}} \FunctionTok{as.data.frame}\NormalTok{(}\FunctionTok{table}\NormalTok{(grausex}\SpecialCharTok{$}\NormalTok{sex, grausex}\SpecialCharTok{$}\NormalTok{grau))}
\FunctionTok{names}\NormalTok{(df) }\OtherTok{\textless{}{-}} \FunctionTok{c}\NormalTok{(}\StringTok{"sexo"}\NormalTok{, }\StringTok{"grau"}\NormalTok{, }\StringTok{"contagem"}\NormalTok{)}

\CommentTok{\# Calcular proporção de cada grau dentro de cada sexo: P(grau | sexo)}
\CommentTok{\# usa ave para somar por sexo sem depender de dplyr}
\NormalTok{df}\SpecialCharTok{$}\NormalTok{proporcao }\OtherTok{\textless{}{-}}\NormalTok{ df}\SpecialCharTok{$}\NormalTok{contagem }\SpecialCharTok{/} \FunctionTok{ave}\NormalTok{(df}\SpecialCharTok{$}\NormalTok{contagem, df}\SpecialCharTok{$}\NormalTok{sexo, }\AttributeTok{FUN =}\NormalTok{ sum)}

\CommentTok{\# Opcional: formatar rótulo em porcentagem (com 1 casa decimal)}
\NormalTok{df}\SpecialCharTok{$}\NormalTok{label\_pct }\OtherTok{\textless{}{-}} \FunctionTok{ifelse}\NormalTok{(df}\SpecialCharTok{$}\NormalTok{proporcao }\SpecialCharTok{\textgreater{}=} \FloatTok{0.03}\NormalTok{, }\FunctionTok{percent}\NormalTok{(df}\SpecialCharTok{$}\NormalTok{proporcao, }\AttributeTok{accuracy =} \FloatTok{0.1}\NormalTok{), }\StringTok{""}\NormalTok{) }
\CommentTok{\# (esconde rótulos muito pequenos para evitar sobreposição)}

\CommentTok{\# Gráfico: barras empilhadas com posição = "fill" para mostrar proporção dentro de cada sexo}
\NormalTok{p }\OtherTok{\textless{}{-}} \FunctionTok{ggplot}\NormalTok{(df, }\FunctionTok{aes}\NormalTok{(}\AttributeTok{x =}\NormalTok{ sexo, }\AttributeTok{y =}\NormalTok{ contagem, }\AttributeTok{fill =}\NormalTok{ grau)) }\SpecialCharTok{+}
  \FunctionTok{geom\_col}\NormalTok{(}\AttributeTok{position =} \StringTok{"fill"}\NormalTok{, }\AttributeTok{width =} \FloatTok{0.6}\NormalTok{, }\AttributeTok{colour =} \StringTok{"grey30"}\NormalTok{, }\AttributeTok{size =} \FloatTok{0.1}\NormalTok{) }\SpecialCharTok{+}
  \FunctionTok{geom\_text}\NormalTok{(}\FunctionTok{aes}\NormalTok{(}\AttributeTok{label =}\NormalTok{ label\_pct), }\AttributeTok{position =} \FunctionTok{position\_fill}\NormalTok{(}\AttributeTok{vjust =} \FloatTok{0.5}\NormalTok{), }\AttributeTok{colour =} \StringTok{"white"}\NormalTok{, }\AttributeTok{size =} \DecValTok{3}\NormalTok{) }\SpecialCharTok{+}
  \FunctionTok{scale\_y\_continuous}\NormalTok{(}\AttributeTok{labels =} \FunctionTok{percent\_format}\NormalTok{(}\AttributeTok{accuracy =} \DecValTok{1}\NormalTok{)) }\SpecialCharTok{+}
  \FunctionTok{scale\_fill\_brewer}\NormalTok{(}\AttributeTok{palette =} \StringTok{"Set2"}\NormalTok{) }\SpecialCharTok{+}
  \FunctionTok{labs}\NormalTok{(}
    \AttributeTok{title =} \StringTok{"Proporção de graus dentro de cada sexo"}\NormalTok{,}
    \AttributeTok{x =} \StringTok{"Sexo"}\NormalTok{,}
    \AttributeTok{y =} \StringTok{"Proporção (cada barra = 100\%)"}\NormalTok{,}
    \AttributeTok{fill =} \StringTok{"Grau"}
\NormalTok{  ) }\SpecialCharTok{+}
  \FunctionTok{theme\_minimal}\NormalTok{() }\SpecialCharTok{+}
  \FunctionTok{theme}\NormalTok{(}
    \AttributeTok{plot.title =} \FunctionTok{element\_text}\NormalTok{(}\AttributeTok{hjust =} \FloatTok{0.5}\NormalTok{)}
\NormalTok{  )}

\CommentTok{\# Exibir gráfico}
\FunctionTok{print}\NormalTok{(p)}

\CommentTok{\# Observações:}
\CommentTok{\# {-} position = "fill" transforma as alturas em proporções por sexo (P(grau | sexo)).}
\CommentTok{\# {-} Rótulos aparecem apenas quando proporção \textgreater{}= 3\% (ajuste em df$label\_pct).}
\CommentTok{\# {-} A ordem dos graus é preservada pelos levels definidos em grausex$grau.}
\InformationTok{\textasciigrave{}\textasciigrave{}\textasciigrave{}}
\end{Highlighting}
\end{Shaded}

\pandocbounded{\includegraphics[keepaspectratio]{cap6-moore-tab-dupla-entrada_files/figure-pdf/unnamed-chunk-12-1.pdf}}

\texttt{CONCLUA}: a porcentagem projetada de \texttt{mulheres} a receber
um \texttt{grau} de \texttt{associado} é \ul{maior do que} a porcentagem
projetada de \texttt{homens} a receber esse mesmo \texttt{grau},
enquanto a porcentagem projetada de \texttt{homens} a \ul{receber outros
graus diferentes} de \texttt{associado} é \ul{ligeiramente maior} do que
a porcentagem de \texttt{mulheres}.

\begin{tcolorbox}[enhanced jigsaw, arc=.35mm, opacitybacktitle=0.6, colframe=quarto-callout-important-color-frame, titlerule=0mm, leftrule=.75mm, left=2mm, colbacktitle=quarto-callout-important-color!10!white, breakable, toprule=.15mm, bottomtitle=1mm, opacityback=0, coltitle=black, title=\textcolor{quarto-callout-important-color}{\faExclamation}\hspace{0.5em}{Erro de Arredondamento}, rightrule=.15mm, bottomrule=.15mm, toptitle=1mm, colback=white]

O erro de arredondamento é a \textbf{\emph{pequena diferença entre um
número decimal arredondado e seu valor preciso antes do
arredondamento}}.

\end{tcolorbox}

Um programa de computador fará esses cálculos para você. A maioria dos
programas permite que você escolha quais distribuições condicionais você
quer comparar. A saída na Figura 6.2 apresenta as \ul{duas distribuições
condicionais} de \ul{graus conferidos}, \ul{uma para cada sexo}, e
também a distribuição marginal dos graus conferidos a todos os
estudantes. As distribuições coincidem (a menos de erro de
arredondamento) com os resultados nos Exemplos 6.2 e 6.3.

Lembre-se de que \ul{\textbf{há dois conjuntos de distribuições
condicionais para qualquer tabela de dupla entrada}}.

O Exemplo 6.3 examinou as \ul{distribuições condicionais de graus
conferidos para as duas categorias de sexo}. A Figura 6.3(a) faz essa
comparação em um gráfico de barras, com barras separadas para homens e
mulheres, lado a lado, para cada categoria de grau. Nesse gráfico, o
total das quatro barras cinza-escuro é 100\%, e o total das barras
cinza-claro também é 100\%.

Poderíamos também examinar as \ul{quatro distribuições condicionais de
sexo}, \ul{uma para cada categoria de grau conferido}, olhando
separadamente as quatro colunas na Tabela 6.1. A Figura 6.3(b) faz essa
comparação em um gráfico de barras, novamente com barras separadas para
homens e mulheres, lado a lado, para cada categoria de grau. Note que os
percentuais de cada par lado a lado têm soma 100\%. A Figura 6.3(c)
também faz essa comparação. Em (c), cada barra é dividida (segmentada)
em duas partes, representadas por dois tons de cinza. A porção superior
de cada barra representa a proporção de mulheres que receberam cada
grau. A porção inferior representa a proporção de homens. Cada barra tem
altura 1, porque cada barra representa todos os estudantes em cada grupo
diferente de pessoas. Gráficos de barras como esse na Figura 6.3(c), nos
quais cada barra é dividida em partes, cada parte representando uma
categoria diferente, são algumas vezes chamados de
\textbf{\emph{gráficos de barras segmentadas}}.

Gerar esses gráficos.

\begin{Shaded}
\begin{Highlighting}[numbers=left,,]
\InformationTok{\textasciigrave{}\textasciigrave{}\textasciigrave{}\{r\}}
\CommentTok{\# Preparar data.frame com contagens por grau x sexo}
\NormalTok{df }\OtherTok{\textless{}{-}} \FunctionTok{as.data.frame}\NormalTok{(}\FunctionTok{table}\NormalTok{(grausex}\SpecialCharTok{$}\NormalTok{grau, grausex}\SpecialCharTok{$}\NormalTok{sex))}
\FunctionTok{names}\NormalTok{(df) }\OtherTok{\textless{}{-}} \FunctionTok{c}\NormalTok{(}\StringTok{"grau"}\NormalTok{, }\StringTok{"sexo"}\NormalTok{, }\StringTok{"contagem"}\NormalTok{)}

\CommentTok{\# Converter sexo para rótulos mais legíveis (opcional)}
\NormalTok{df}\SpecialCharTok{$}\NormalTok{sexo }\OtherTok{\textless{}{-}} \FunctionTok{factor}\NormalTok{(df}\SpecialCharTok{$}\NormalTok{sexo, }\AttributeTok{levels =} \FunctionTok{c}\NormalTok{(}\StringTok{"M"}\NormalTok{, }\StringTok{"H"}\NormalTok{), }\AttributeTok{labels =} \FunctionTok{c}\NormalTok{(}\StringTok{"Mulher"}\NormalTok{, }\StringTok{"Homem"}\NormalTok{))}

\CommentTok{\# Gráfico: barras lado a lado (position = position\_dodge)}
\NormalTok{p }\OtherTok{\textless{}{-}} \FunctionTok{ggplot}\NormalTok{(df, }\FunctionTok{aes}\NormalTok{(}\AttributeTok{x =}\NormalTok{ grau, }\AttributeTok{y =}\NormalTok{ contagem, }\AttributeTok{fill =}\NormalTok{ sexo)) }\SpecialCharTok{+}
  \FunctionTok{geom\_col}\NormalTok{(}\AttributeTok{position =} \FunctionTok{position\_dodge}\NormalTok{(}\AttributeTok{width =} \FloatTok{0.8}\NormalTok{), }\AttributeTok{width =} \FloatTok{0.7}\NormalTok{, }\AttributeTok{colour =} \StringTok{"grey20"}\NormalTok{, }\AttributeTok{size =} \FloatTok{0.2}\NormalTok{) }\SpecialCharTok{+}
  \FunctionTok{geom\_text}\NormalTok{(}\FunctionTok{aes}\NormalTok{(}\AttributeTok{label =}\NormalTok{ contagem),}
            \AttributeTok{position =} \FunctionTok{position\_dodge}\NormalTok{(}\AttributeTok{width =} \FloatTok{0.8}\NormalTok{),}
            \AttributeTok{vjust =} \SpecialCharTok{{-}}\FloatTok{0.3}\NormalTok{, }\AttributeTok{size =} \DecValTok{3}\NormalTok{) }\SpecialCharTok{+}
  \FunctionTok{scale\_fill\_manual}\NormalTok{(}\AttributeTok{values =} \FunctionTok{c}\NormalTok{(}\StringTok{"\#4E79A7"}\NormalTok{, }\StringTok{"\#F28E2B"}\NormalTok{)) }\SpecialCharTok{+} \CommentTok{\# cores opcionais}
  \FunctionTok{labs}\NormalTok{(}
    \AttributeTok{title =} \StringTok{"Distribuição de graus por sexo"}\NormalTok{,}
    \AttributeTok{x =} \StringTok{"Grau acadêmico"}\NormalTok{,}
    \AttributeTok{y =} \StringTok{"Contagem"}\NormalTok{,}
    \AttributeTok{fill =} \StringTok{"Sexo"}
\NormalTok{  ) }\SpecialCharTok{+}
  \FunctionTok{theme\_minimal}\NormalTok{() }\SpecialCharTok{+}
  \FunctionTok{theme}\NormalTok{(}
    \AttributeTok{plot.title =} \FunctionTok{element\_text}\NormalTok{(}\AttributeTok{hjust =} \FloatTok{0.5}\NormalTok{),}
    \AttributeTok{axis.text.x =} \FunctionTok{element\_text}\NormalTok{(}\AttributeTok{angle =} \DecValTok{0}\NormalTok{, }\AttributeTok{vjust =} \FloatTok{0.5}\NormalTok{)}
\NormalTok{  )}

\CommentTok{\# Exibir gráfico}
\FunctionTok{print}\NormalTok{(p)}

\CommentTok{\# Observações:}
\CommentTok{\# {-} Cada grupo em x (grau) contém duas barras: Mulher e Homem, lado a lado.}
\CommentTok{\# {-} Ajuste width/position\_dodge para espaçamento diferente entre as barras.}
\InformationTok{\textasciigrave{}\textasciigrave{}\textasciigrave{}}
\end{Highlighting}
\end{Shaded}

\pandocbounded{\includegraphics[keepaspectratio]{cap6-moore-tab-dupla-entrada_files/figure-pdf/unnamed-chunk-13-1.pdf}}

Mesmo gráfico acima agora com proporções no eixo y ao invés de
contagens.

\begin{Shaded}
\begin{Highlighting}[numbers=left,,]
\InformationTok{\textasciigrave{}\textasciigrave{}\textasciigrave{}\{r\}}
\CommentTok{\# Criar data.frame com fatores e ordem explícita dos níveis}
\NormalTok{grausex }\OtherTok{\textless{}{-}} \FunctionTok{data.frame}\NormalTok{(}
  \AttributeTok{sex  =} \FunctionTok{factor}\NormalTok{(}\FunctionTok{c}\NormalTok{(}\FunctionTok{rep}\NormalTok{(}\StringTok{"M"}\NormalTok{, }\DecValTok{2283}\NormalTok{), }\FunctionTok{rep}\NormalTok{(}\StringTok{"H"}\NormalTok{, }\DecValTok{1622}\NormalTok{)), }\AttributeTok{levels =} \FunctionTok{c}\NormalTok{(}\StringTok{"M"}\NormalTok{, }\StringTok{"H"}\NormalTok{)),}
  \AttributeTok{grau =} \FunctionTok{factor}\NormalTok{(}\FunctionTok{c}\NormalTok{(}\FunctionTok{rep}\NormalTok{(}\StringTok{"ass"}\NormalTok{, }\DecValTok{639}\NormalTok{), }\FunctionTok{rep}\NormalTok{(}\StringTok{"bach"}\NormalTok{, }\DecValTok{1087}\NormalTok{), }\FunctionTok{rep}\NormalTok{(}\StringTok{"ms"}\NormalTok{, }\DecValTok{460}\NormalTok{), }\FunctionTok{rep}\NormalTok{(}\StringTok{"dr"}\NormalTok{, }\DecValTok{97}\NormalTok{),}
                  \FunctionTok{rep}\NormalTok{(}\StringTok{"ass"}\NormalTok{, }\DecValTok{402}\NormalTok{), }\FunctionTok{rep}\NormalTok{(}\StringTok{"bach"}\NormalTok{,  }\DecValTok{804}\NormalTok{), }\FunctionTok{rep}\NormalTok{(}\StringTok{"ms"}\NormalTok{, }\DecValTok{329}\NormalTok{), }\FunctionTok{rep}\NormalTok{(}\StringTok{"dr"}\NormalTok{, }\DecValTok{87}\NormalTok{)),}
                \AttributeTok{levels =} \FunctionTok{c}\NormalTok{(}\StringTok{"ass"}\NormalTok{, }\StringTok{"bach"}\NormalTok{, }\StringTok{"ms"}\NormalTok{, }\StringTok{"dr"}\NormalTok{))}
\NormalTok{)}

\CommentTok{\# Preparar data.frame com contagens por grau x sexo}
\NormalTok{df }\OtherTok{\textless{}{-}} \FunctionTok{as.data.frame}\NormalTok{(}\FunctionTok{table}\NormalTok{(grausex}\SpecialCharTok{$}\NormalTok{grau, grausex}\SpecialCharTok{$}\NormalTok{sex))}
\FunctionTok{names}\NormalTok{(df) }\OtherTok{\textless{}{-}} \FunctionTok{c}\NormalTok{(}\StringTok{"grau"}\NormalTok{, }\StringTok{"sexo"}\NormalTok{, }\StringTok{"contagem"}\NormalTok{)}

\CommentTok{\# Converter sexo para rótulos legíveis}
\NormalTok{df}\SpecialCharTok{$}\NormalTok{sexo }\OtherTok{\textless{}{-}} \FunctionTok{factor}\NormalTok{(df}\SpecialCharTok{$}\NormalTok{sexo, }\AttributeTok{levels =} \FunctionTok{c}\NormalTok{(}\StringTok{"M"}\NormalTok{, }\StringTok{"H"}\NormalTok{), }\AttributeTok{labels =} \FunctionTok{c}\NormalTok{(}\StringTok{"Mulher"}\NormalTok{, }\StringTok{"Homem"}\NormalTok{))}

\CommentTok{\# Converter grau para rótulos legíveis (opcional)}
\NormalTok{df}\SpecialCharTok{$}\NormalTok{grau }\OtherTok{\textless{}{-}} \FunctionTok{factor}\NormalTok{(df}\SpecialCharTok{$}\NormalTok{grau,}
                  \AttributeTok{levels =} \FunctionTok{c}\NormalTok{(}\StringTok{"ass"}\NormalTok{, }\StringTok{"bach"}\NormalTok{, }\StringTok{"ms"}\NormalTok{, }\StringTok{"dr"}\NormalTok{),}
                  \AttributeTok{labels =} \FunctionTok{c}\NormalTok{(}\StringTok{"Associado"}\NormalTok{, }\StringTok{"Bacharel"}\NormalTok{, }\StringTok{"Mestre"}\NormalTok{, }\StringTok{"Doutor"}\NormalTok{)}
\NormalTok{                  )}

\CommentTok{\# Calcular proporção de cada sexo dentro de cada grau: P(sexo | grau)}
\NormalTok{df}\SpecialCharTok{$}\NormalTok{proporcao }\OtherTok{\textless{}{-}}\NormalTok{ df}\SpecialCharTok{$}\NormalTok{contagem }\SpecialCharTok{/} \FunctionTok{ave}\NormalTok{(df}\SpecialCharTok{$}\NormalTok{contagem, df}\SpecialCharTok{$}\NormalTok{grau, }\AttributeTok{FUN =}\NormalTok{ sum)}

\CommentTok{\# Rótulo em porcentagem (mostrar somente se \textgreater{}= 2\% para evitar sobreposição)}
\NormalTok{df}\SpecialCharTok{$}\NormalTok{label }\OtherTok{\textless{}{-}} \FunctionTok{ifelse}\NormalTok{(df}\SpecialCharTok{$}\NormalTok{proporcao }\SpecialCharTok{\textgreater{}=} \FloatTok{0.02}\NormalTok{, }\FunctionTok{percent}\NormalTok{(df}\SpecialCharTok{$}\NormalTok{proporcao, }\AttributeTok{accuracy =} \FloatTok{0.1}\NormalTok{), }\StringTok{""}\NormalTok{)}

\CommentTok{\# Calcular proporção de cada sexo dentro de cada grau: P(sexo | grau)}
\NormalTok{df}\SpecialCharTok{$}\NormalTok{proporcao }\OtherTok{\textless{}{-}}\NormalTok{ df}\SpecialCharTok{$}\NormalTok{contagem }\SpecialCharTok{/} \FunctionTok{ave}\NormalTok{(df}\SpecialCharTok{$}\NormalTok{contagem, df}\SpecialCharTok{$}\NormalTok{grau, }\AttributeTok{FUN =}\NormalTok{ sum)}

\CommentTok{\# Rótulo em porcentagem}
\NormalTok{df}\SpecialCharTok{$}\NormalTok{label }\OtherTok{\textless{}{-}} \FunctionTok{percent}\NormalTok{(df}\SpecialCharTok{$}\NormalTok{proporcao, }\AttributeTok{accuracy =} \FloatTok{0.1}\NormalTok{)}

\CommentTok{\# Posicionamento do texto: garantir que fique totalmente dentro da barra}
\CommentTok{\# {-} Para barras pequenas, colocar no centro (mais legível)}
\CommentTok{\# {-} Para barras maiores, colocar próximo à extremidade interna direita (95\% do comprimento)}
\NormalTok{df}\SpecialCharTok{$}\NormalTok{label\_x }\OtherTok{\textless{}{-}} \FunctionTok{ifelse}\NormalTok{(df}\SpecialCharTok{$}\NormalTok{proporcao }\SpecialCharTok{\textless{}} \FloatTok{0.05}\NormalTok{,}
\NormalTok{                     df}\SpecialCharTok{$}\NormalTok{proporcao }\SpecialCharTok{*} \FloatTok{0.5}\NormalTok{,   }\CommentTok{\# centro para barras muito pequenas}
\NormalTok{                     df}\SpecialCharTok{$}\NormalTok{proporcao }\SpecialCharTok{*} \FloatTok{0.95}\NormalTok{)  }\CommentTok{\# perto da extremidade interna direita}

\CommentTok{\# Gráfico: barras agrupadas (lado a lado) por grau, depois inverte eixos (coord\_flip)}
\NormalTok{p }\OtherTok{\textless{}{-}} \FunctionTok{ggplot}\NormalTok{(df, }\FunctionTok{aes}\NormalTok{(}\AttributeTok{x =}\NormalTok{ grau, }\AttributeTok{y =}\NormalTok{ proporcao, }\AttributeTok{fill =}\NormalTok{ sexo)) }\SpecialCharTok{+}
  \FunctionTok{geom\_col}\NormalTok{(}\AttributeTok{position =} \FunctionTok{position\_dodge}\NormalTok{(}\AttributeTok{width =} \FloatTok{0.8}\NormalTok{), }\AttributeTok{width =} \FloatTok{0.7}\NormalTok{, }\AttributeTok{colour =} \StringTok{"grey20"}\NormalTok{, }\AttributeTok{size =} \FloatTok{0.2}\NormalTok{) }\SpecialCharTok{+}
  \CommentTok{\# geom\_text com y = label\_x posiciona o texto dentro da barra (posição horizontal definida por label\_x)}
  \FunctionTok{geom\_text}\NormalTok{(}\FunctionTok{aes}\NormalTok{(}\AttributeTok{y =}\NormalTok{ label\_x, }\AttributeTok{label =}\NormalTok{ label),}
            \AttributeTok{position =} \FunctionTok{position\_dodge}\NormalTok{(}\AttributeTok{width =} \FloatTok{0.8}\NormalTok{),   }\CommentTok{\# mantém alinhamento com as barras dodge}
            \AttributeTok{colour =} \StringTok{"white"}\NormalTok{, }\AttributeTok{size =} \DecValTok{3}\NormalTok{, }\AttributeTok{fontface =} \StringTok{"bold"}\NormalTok{) }\SpecialCharTok{+}
  \FunctionTok{scale\_y\_continuous}\NormalTok{(}\AttributeTok{labels =} \FunctionTok{percent\_format}\NormalTok{(}\AttributeTok{accuracy =} \DecValTok{1}\NormalTok{)) }\SpecialCharTok{+}
  \FunctionTok{scale\_fill\_manual}\NormalTok{(}\AttributeTok{values =} \FunctionTok{c}\NormalTok{(}\StringTok{"\#4E79A7"}\NormalTok{, }\StringTok{"\#F28E2B"}\NormalTok{)) }\SpecialCharTok{+}
  \FunctionTok{labs}\NormalTok{(}
    \AttributeTok{title =} \StringTok{"Proporção de sexo por grau (cada grau = 100\%)"}\NormalTok{,}
    \AttributeTok{x =} \StringTok{"Grau"}\NormalTok{,}
    \AttributeTok{y =} \StringTok{"Proporção"}\NormalTok{,}
    \AttributeTok{fill =} \StringTok{"Sexo"}
\NormalTok{  ) }\SpecialCharTok{+}
  \FunctionTok{theme\_minimal}\NormalTok{() }\SpecialCharTok{+}
  \FunctionTok{theme}\NormalTok{(}\AttributeTok{plot.title =} \FunctionTok{element\_text}\NormalTok{(}\AttributeTok{hjust =} \FloatTok{0.5}\NormalTok{)) }\SpecialCharTok{+}
  \FunctionTok{coord\_flip}\NormalTok{()  }\CommentTok{\# barras horizontais}

\CommentTok{\# Exibir gráfico}
\FunctionTok{print}\NormalTok{(p)}
\InformationTok{\textasciigrave{}\textasciigrave{}\textasciigrave{}}
\end{Highlighting}
\end{Shaded}

\pandocbounded{\includegraphics[keepaspectratio]{cap6-moore-tab-dupla-entrada_files/figure-pdf/unnamed-chunk-14-1.pdf}}

Mesmo gráfico acima com barras empilhadas.

\begin{Shaded}
\begin{Highlighting}[numbers=left,,]
\InformationTok{\textasciigrave{}\textasciigrave{}\textasciigrave{}\{r\}}
\CommentTok{\# Gerar gráfico de barras empilhadas (horizontais) mostrando P(sexo | grau)}
\CommentTok{\# Cada barra representa um grau; segmentos mostram a proporção de Mulher/Homem dentro desse grau.}

\CommentTok{\# Pacotes necessários}
\ControlFlowTok{if}\NormalTok{ (}\SpecialCharTok{!}\FunctionTok{requireNamespace}\NormalTok{(}\StringTok{"ggplot2"}\NormalTok{, }\AttributeTok{quietly =} \ConstantTok{TRUE}\NormalTok{)) }\FunctionTok{install.packages}\NormalTok{(}\StringTok{"ggplot2"}\NormalTok{)}
\ControlFlowTok{if}\NormalTok{ (}\SpecialCharTok{!}\FunctionTok{requireNamespace}\NormalTok{(}\StringTok{"scales"}\NormalTok{, }\AttributeTok{quietly =} \ConstantTok{TRUE}\NormalTok{)) }\FunctionTok{install.packages}\NormalTok{(}\StringTok{"scales"}\NormalTok{)}
\FunctionTok{library}\NormalTok{(ggplot2)}
\FunctionTok{library}\NormalTok{(scales)}

\CommentTok{\# Dados (preservar ordem dos níveis)}
\NormalTok{grausex }\OtherTok{\textless{}{-}} \FunctionTok{data.frame}\NormalTok{(}
  \AttributeTok{sex  =} \FunctionTok{factor}\NormalTok{(}\FunctionTok{c}\NormalTok{(}\FunctionTok{rep}\NormalTok{(}\StringTok{"M"}\NormalTok{, }\DecValTok{2283}\NormalTok{), }\FunctionTok{rep}\NormalTok{(}\StringTok{"H"}\NormalTok{, }\DecValTok{1622}\NormalTok{)), }\AttributeTok{levels =} \FunctionTok{c}\NormalTok{(}\StringTok{"M"}\NormalTok{, }\StringTok{"H"}\NormalTok{)),}
  \AttributeTok{grau =} \FunctionTok{factor}\NormalTok{(}\FunctionTok{c}\NormalTok{(}\FunctionTok{rep}\NormalTok{(}\StringTok{"ass"}\NormalTok{, }\DecValTok{639}\NormalTok{), }\FunctionTok{rep}\NormalTok{(}\StringTok{"bach"}\NormalTok{, }\DecValTok{1087}\NormalTok{), }\FunctionTok{rep}\NormalTok{(}\StringTok{"ms"}\NormalTok{, }\DecValTok{460}\NormalTok{), }\FunctionTok{rep}\NormalTok{(}\StringTok{"dr"}\NormalTok{, }\DecValTok{97}\NormalTok{),}
                  \FunctionTok{rep}\NormalTok{(}\StringTok{"ass"}\NormalTok{, }\DecValTok{402}\NormalTok{), }\FunctionTok{rep}\NormalTok{(}\StringTok{"bach"}\NormalTok{,  }\DecValTok{804}\NormalTok{), }\FunctionTok{rep}\NormalTok{(}\StringTok{"ms"}\NormalTok{, }\DecValTok{329}\NormalTok{), }\FunctionTok{rep}\NormalTok{(}\StringTok{"dr"}\NormalTok{, }\DecValTok{87}\NormalTok{)),}
                \AttributeTok{levels =} \FunctionTok{c}\NormalTok{(}\StringTok{"ass"}\NormalTok{, }\StringTok{"bach"}\NormalTok{, }\StringTok{"ms"}\NormalTok{, }\StringTok{"dr"}\NormalTok{))}
\NormalTok{)}

\CommentTok{\# Preparar data.frame de contagens por grau x sexo}
\NormalTok{df }\OtherTok{\textless{}{-}} \FunctionTok{as.data.frame}\NormalTok{(}\FunctionTok{table}\NormalTok{(grausex}\SpecialCharTok{$}\NormalTok{grau, grausex}\SpecialCharTok{$}\NormalTok{sex))}
\FunctionTok{names}\NormalTok{(df) }\OtherTok{\textless{}{-}} \FunctionTok{c}\NormalTok{(}\StringTok{"grau"}\NormalTok{, }\StringTok{"sexo"}\NormalTok{, }\StringTok{"contagem"}\NormalTok{)}

\CommentTok{\# Etiquetas legíveis para sexo}
\NormalTok{df}\SpecialCharTok{$}\NormalTok{sexo }\OtherTok{\textless{}{-}} \FunctionTok{factor}\NormalTok{(df}\SpecialCharTok{$}\NormalTok{sexo, }\AttributeTok{levels =} \FunctionTok{c}\NormalTok{(}\StringTok{"M"}\NormalTok{, }\StringTok{"H"}\NormalTok{), }\AttributeTok{labels =} \FunctionTok{c}\NormalTok{(}\StringTok{"Mulher"}\NormalTok{, }\StringTok{"Homem"}\NormalTok{))}

\CommentTok{\# Calcular proporção de cada sexo dentro de cada grau: P(sexo | grau)}
\NormalTok{df}\SpecialCharTok{$}\NormalTok{proporcao }\OtherTok{\textless{}{-}}\NormalTok{ df}\SpecialCharTok{$}\NormalTok{contagem }\SpecialCharTok{/} \FunctionTok{ave}\NormalTok{(df}\SpecialCharTok{$}\NormalTok{contagem, df}\SpecialCharTok{$}\NormalTok{grau, }\AttributeTok{FUN =}\NormalTok{ sum)}

\CommentTok{\# Rótulo em porcentagem (mostrar apenas quando segmento for suficientemente grande)}
\NormalTok{limite\_label }\OtherTok{\textless{}{-}} \FloatTok{0.03}   \CommentTok{\# exibir rótulo somente se proporção \textgreater{}= 3\%}
\NormalTok{df}\SpecialCharTok{$}\NormalTok{label }\OtherTok{\textless{}{-}} \FunctionTok{ifelse}\NormalTok{(df}\SpecialCharTok{$}\NormalTok{proporcao }\SpecialCharTok{\textgreater{}=}\NormalTok{ limite\_label, }\FunctionTok{percent}\NormalTok{(df}\SpecialCharTok{$}\NormalTok{proporcao, }\AttributeTok{accuracy =} \FloatTok{0.1}\NormalTok{), }\StringTok{""}\NormalTok{)}

\CommentTok{\# Escolher cor do rótulo para garantir contraste (branco em segmentos grandes, preto em pequenos)}
\NormalTok{df}\SpecialCharTok{$}\NormalTok{label\_color }\OtherTok{\textless{}{-}} \FunctionTok{ifelse}\NormalTok{(df}\SpecialCharTok{$}\NormalTok{proporcao }\SpecialCharTok{\textgreater{}=} \FloatTok{0.15}\NormalTok{, }\StringTok{"white"}\NormalTok{, }\StringTok{"black"}\NormalTok{)}

\CommentTok{\# Gráfico: barras empilhadas (cada barra soma 100\%) eixos invertidos (horizontais)}
\NormalTok{p }\OtherTok{\textless{}{-}} \FunctionTok{ggplot}\NormalTok{(df, }\FunctionTok{aes}\NormalTok{(}\AttributeTok{x =}\NormalTok{ grau, }\AttributeTok{y =}\NormalTok{ proporcao, }\AttributeTok{fill =}\NormalTok{ sexo)) }\SpecialCharTok{+}
  \FunctionTok{geom\_col}\NormalTok{(}\AttributeTok{position =} \StringTok{"stack"}\NormalTok{, }\AttributeTok{width =} \FloatTok{0.7}\NormalTok{, }\AttributeTok{colour =} \StringTok{"grey20"}\NormalTok{, }\AttributeTok{size =} \FloatTok{0.2}\NormalTok{) }\SpecialCharTok{+}
  \FunctionTok{geom\_text}\NormalTok{(}\FunctionTok{aes}\NormalTok{(}\AttributeTok{label =}\NormalTok{ label, }\AttributeTok{colour =}\NormalTok{ label\_color),}
            \AttributeTok{position =} \FunctionTok{position\_stack}\NormalTok{(}\AttributeTok{vjust =} \FloatTok{0.5}\NormalTok{), }\AttributeTok{size =} \DecValTok{3}\NormalTok{, }\AttributeTok{fontface =} \StringTok{"bold"}\NormalTok{) }\SpecialCharTok{+}
  \FunctionTok{scale\_colour\_identity}\NormalTok{() }\SpecialCharTok{+}  \CommentTok{\# usa as cores definidas em df$label\_color diretamente}
  \FunctionTok{scale\_y\_continuous}\NormalTok{(}\AttributeTok{labels =} \FunctionTok{percent\_format}\NormalTok{(}\AttributeTok{accuracy =} \DecValTok{1}\NormalTok{)) }\SpecialCharTok{+}
  \FunctionTok{scale\_fill\_manual}\NormalTok{(}\AttributeTok{values =} \FunctionTok{c}\NormalTok{(}\StringTok{"\#4E79A7"}\NormalTok{, }\StringTok{"\#F28E2B"}\NormalTok{)) }\SpecialCharTok{+}
  \FunctionTok{labs}\NormalTok{(}
    \AttributeTok{title =} \StringTok{"Proporção de sexo por grau (cada grau = 100\%) — barras empilhadas"}\NormalTok{,}
    \AttributeTok{x =} \StringTok{"Grau"}\NormalTok{,}
    \AttributeTok{y =} \StringTok{"Proporção"}\NormalTok{,}
    \AttributeTok{fill =} \StringTok{"Sexo"}
\NormalTok{  ) }\SpecialCharTok{+}
  \FunctionTok{theme\_minimal}\NormalTok{() }\SpecialCharTok{+}
  \FunctionTok{theme}\NormalTok{(}\AttributeTok{plot.title =} \FunctionTok{element\_text}\NormalTok{(}\AttributeTok{hjust =} \FloatTok{0.5}\NormalTok{)) }\SpecialCharTok{+}
  \FunctionTok{coord\_flip}\NormalTok{()  }\CommentTok{\# barras horizontais}

\CommentTok{\# Exibir gráfico}
\FunctionTok{print}\NormalTok{(p)}
\InformationTok{\textasciigrave{}\textasciigrave{}\textasciigrave{}}
\end{Highlighting}
\end{Shaded}

\pandocbounded{\includegraphics[keepaspectratio]{cap6-moore-tab-dupla-entrada_files/figure-pdf/unnamed-chunk-15-1.pdf}}

Mesmo gráfico anterior com as classes do grau ordenadas de cima para
baixo.

\begin{Shaded}
\begin{Highlighting}[numbers=left,,]
\InformationTok{\textasciigrave{}\textasciigrave{}\textasciigrave{}\{r\}}
\CommentTok{\# Gráfico de barras empilhadas horizontais com as classes de \textquotesingle{}grau\textquotesingle{}}
\CommentTok{\# ordenadas de cima para baixo (controle via levels do fator).}

\CommentTok{\# Pacotes necessários}
\ControlFlowTok{if}\NormalTok{ (}\SpecialCharTok{!}\FunctionTok{requireNamespace}\NormalTok{(}\StringTok{"ggplot2"}\NormalTok{, }\AttributeTok{quietly =} \ConstantTok{TRUE}\NormalTok{)) }\FunctionTok{install.packages}\NormalTok{(}\StringTok{"ggplot2"}\NormalTok{)}
\ControlFlowTok{if}\NormalTok{ (}\SpecialCharTok{!}\FunctionTok{requireNamespace}\NormalTok{(}\StringTok{"scales"}\NormalTok{, }\AttributeTok{quietly =} \ConstantTok{TRUE}\NormalTok{)) }\FunctionTok{install.packages}\NormalTok{(}\StringTok{"scales"}\NormalTok{)}
\FunctionTok{library}\NormalTok{(ggplot2)}
\FunctionTok{library}\NormalTok{(scales)}

\CommentTok{\# Dados (preservar ordem natural desejada das classes de grau)}
\NormalTok{ordem\_grau }\OtherTok{\textless{}{-}} \FunctionTok{c}\NormalTok{(}\StringTok{"ass"}\NormalTok{, }\StringTok{"bach"}\NormalTok{, }\StringTok{"ms"}\NormalTok{, }\StringTok{"dr"}\NormalTok{)  }\CommentTok{\# ordem desejada top {-}\textgreater{} bottom no gráfico final}

\NormalTok{grausex }\OtherTok{\textless{}{-}} \FunctionTok{data.frame}\NormalTok{(}
  \AttributeTok{sex  =} \FunctionTok{factor}\NormalTok{(}\FunctionTok{c}\NormalTok{(}\FunctionTok{rep}\NormalTok{(}\StringTok{"M"}\NormalTok{, }\DecValTok{2283}\NormalTok{), }\FunctionTok{rep}\NormalTok{(}\StringTok{"H"}\NormalTok{, }\DecValTok{1622}\NormalTok{)), }\AttributeTok{levels =} \FunctionTok{c}\NormalTok{(}\StringTok{"M"}\NormalTok{, }\StringTok{"H"}\NormalTok{)),}
  \AttributeTok{grau =} \FunctionTok{factor}\NormalTok{(}\FunctionTok{c}\NormalTok{(}\FunctionTok{rep}\NormalTok{(}\StringTok{"ass"}\NormalTok{, }\DecValTok{639}\NormalTok{), }\FunctionTok{rep}\NormalTok{(}\StringTok{"bach"}\NormalTok{, }\DecValTok{1087}\NormalTok{), }\FunctionTok{rep}\NormalTok{(}\StringTok{"ms"}\NormalTok{, }\DecValTok{460}\NormalTok{), }\FunctionTok{rep}\NormalTok{(}\StringTok{"dr"}\NormalTok{, }\DecValTok{97}\NormalTok{),}
                  \FunctionTok{rep}\NormalTok{(}\StringTok{"ass"}\NormalTok{, }\DecValTok{402}\NormalTok{), }\FunctionTok{rep}\NormalTok{(}\StringTok{"bach"}\NormalTok{,  }\DecValTok{804}\NormalTok{), }\FunctionTok{rep}\NormalTok{(}\StringTok{"ms"}\NormalTok{, }\DecValTok{329}\NormalTok{), }\FunctionTok{rep}\NormalTok{(}\StringTok{"dr"}\NormalTok{, }\DecValTok{87}\NormalTok{)))}
\NormalTok{)}

\CommentTok{\# Preparar data.frame de contagens por grau x sexo}
\NormalTok{df }\OtherTok{\textless{}{-}} \FunctionTok{as.data.frame}\NormalTok{(}\FunctionTok{table}\NormalTok{(grausex}\SpecialCharTok{$}\NormalTok{grau, grausex}\SpecialCharTok{$}\NormalTok{sex))}
\FunctionTok{names}\NormalTok{(df) }\OtherTok{\textless{}{-}} \FunctionTok{c}\NormalTok{(}\StringTok{"grau"}\NormalTok{, }\StringTok{"sexo"}\NormalTok{, }\StringTok{"contagem"}\NormalTok{)}

\CommentTok{\# Etiquetas legíveis para sexo}
\NormalTok{df}\SpecialCharTok{$}\NormalTok{sexo }\OtherTok{\textless{}{-}} \FunctionTok{factor}\NormalTok{(df}\SpecialCharTok{$}\NormalTok{sexo, }\AttributeTok{levels =} \FunctionTok{c}\NormalTok{(}\StringTok{"M"}\NormalTok{, }\StringTok{"H"}\NormalTok{), }\AttributeTok{labels =} \FunctionTok{c}\NormalTok{(}\StringTok{"Mulher"}\NormalTok{, }\StringTok{"Homem"}\NormalTok{))}

\CommentTok{\# Garantir que \textquotesingle{}grau\textquotesingle{} preserve a ordem desejada no gráfico:}
\CommentTok{\# Para que, após coord\_flip(), a ordem apareça de cima para baixo como em ordem\_grau,}
\CommentTok{\# definimos os níveis como o reverso da ordem desejada.}
\NormalTok{df}\SpecialCharTok{$}\NormalTok{grau }\OtherTok{\textless{}{-}} \FunctionTok{factor}\NormalTok{(}\FunctionTok{as.character}\NormalTok{(df}\SpecialCharTok{$}\NormalTok{grau), }\AttributeTok{levels =} \FunctionTok{rev}\NormalTok{(ordem\_grau))}

\CommentTok{\# Calcular proporção de cada sexo dentro de cada grau: P(sexo | grau)}
\NormalTok{df}\SpecialCharTok{$}\NormalTok{proporcao }\OtherTok{\textless{}{-}}\NormalTok{ df}\SpecialCharTok{$}\NormalTok{contagem }\SpecialCharTok{/} \FunctionTok{ave}\NormalTok{(df}\SpecialCharTok{$}\NormalTok{contagem, df}\SpecialCharTok{$}\NormalTok{grau, }\AttributeTok{FUN =}\NormalTok{ sum)}

\CommentTok{\# Rótulo em porcentagem (apenas quando segmento suficientemente grande)}
\NormalTok{limite\_label }\OtherTok{\textless{}{-}} \FloatTok{0.03}
\NormalTok{df}\SpecialCharTok{$}\NormalTok{label }\OtherTok{\textless{}{-}} \FunctionTok{ifelse}\NormalTok{(df}\SpecialCharTok{$}\NormalTok{proporcao }\SpecialCharTok{\textgreater{}=}\NormalTok{ limite\_label, }\FunctionTok{percent}\NormalTok{(df}\SpecialCharTok{$}\NormalTok{proporcao, }\AttributeTok{accuracy =} \FloatTok{0.1}\NormalTok{), }\StringTok{""}\NormalTok{)}

\CommentTok{\# Cor do rótulo para contraste (branco em segmentos grandes)}
\NormalTok{df}\SpecialCharTok{$}\NormalTok{label\_color }\OtherTok{\textless{}{-}} \FunctionTok{ifelse}\NormalTok{(df}\SpecialCharTok{$}\NormalTok{proporcao }\SpecialCharTok{\textgreater{}=} \FloatTok{0.15}\NormalTok{, }\StringTok{"white"}\NormalTok{, }\StringTok{"black"}\NormalTok{)}

\CommentTok{\# Gráfico: barras empilhadas horizontais (cada grau = 100\%)}
\NormalTok{p }\OtherTok{\textless{}{-}} \FunctionTok{ggplot}\NormalTok{(df, }\FunctionTok{aes}\NormalTok{(}\AttributeTok{x =}\NormalTok{ grau, }\AttributeTok{y =}\NormalTok{ proporcao, }\AttributeTok{fill =}\NormalTok{ sexo)) }\SpecialCharTok{+}
  \FunctionTok{geom\_col}\NormalTok{(}\AttributeTok{position =} \StringTok{"stack"}\NormalTok{, }\AttributeTok{width =} \FloatTok{0.7}\NormalTok{, }\AttributeTok{colour =} \StringTok{"grey20"}\NormalTok{, }\AttributeTok{size =} \FloatTok{0.2}\NormalTok{) }\SpecialCharTok{+}
  \FunctionTok{geom\_text}\NormalTok{(}\FunctionTok{aes}\NormalTok{(}\AttributeTok{label =}\NormalTok{ label, }\AttributeTok{colour =}\NormalTok{ label\_color),}
            \AttributeTok{position =} \FunctionTok{position\_stack}\NormalTok{(}\AttributeTok{vjust =} \FloatTok{0.5}\NormalTok{), }\AttributeTok{size =} \DecValTok{3}\NormalTok{, }\AttributeTok{fontface =} \StringTok{"bold"}\NormalTok{) }\SpecialCharTok{+}
  \FunctionTok{scale\_colour\_identity}\NormalTok{() }\SpecialCharTok{+}
  \FunctionTok{scale\_y\_continuous}\NormalTok{(}\AttributeTok{labels =} \FunctionTok{percent\_format}\NormalTok{(}\AttributeTok{accuracy =} \DecValTok{1}\NormalTok{)) }\SpecialCharTok{+}
  \FunctionTok{scale\_fill\_manual}\NormalTok{(}\AttributeTok{values =} \FunctionTok{c}\NormalTok{(}\StringTok{"\#4E79A7"}\NormalTok{, }\StringTok{"\#F28E2B"}\NormalTok{)) }\SpecialCharTok{+}
  \FunctionTok{labs}\NormalTok{(}
    \AttributeTok{title =} \StringTok{"Proporção de sexo por grau (cada grau = 100\%) — empilhadas"}\NormalTok{,}
    \AttributeTok{x =} \StringTok{"Grau"}\NormalTok{,}
    \AttributeTok{y =} \StringTok{"Proporção"}\NormalTok{,}
    \AttributeTok{fill =} \StringTok{"Sexo"}
\NormalTok{  ) }\SpecialCharTok{+}
  \FunctionTok{theme\_minimal}\NormalTok{() }\SpecialCharTok{+}
  \FunctionTok{theme}\NormalTok{(}\AttributeTok{plot.title =} \FunctionTok{element\_text}\NormalTok{(}\AttributeTok{hjust =} \FloatTok{0.5}\NormalTok{)) }\SpecialCharTok{+}
  \FunctionTok{coord\_flip}\NormalTok{()  }\CommentTok{\# barras horizontais; ordem por níveis de \textquotesingle{}grau\textquotesingle{} (já invertida acima)}

\CommentTok{\# Exibir gráfico}
\FunctionTok{print}\NormalTok{(p)}
\InformationTok{\textasciigrave{}\textasciigrave{}\textasciigrave{}}
\end{Highlighting}
\end{Shaded}

\pandocbounded{\includegraphics[keepaspectratio]{cap6-moore-tab-dupla-entrada_files/figure-pdf/unnamed-chunk-16-1.pdf}}

Fica claro agora que, nos EUA, a categoria Mulher na variável
\texttt{sexo} obterá \ul{\textbf{maior proporção}} (\textgreater{}
50,0\%) em cada uma das 4 categorias de \texttt{graus}.

No Brasil espera-se uma distribuição oposta a essa.

\begin{tcolorbox}[enhanced jigsaw, arc=.35mm, opacitybacktitle=0.6, colframe=quarto-callout-important-color-frame, titlerule=0mm, leftrule=.75mm, left=2mm, colbacktitle=quarto-callout-important-color!10!white, breakable, toprule=.15mm, bottomtitle=1mm, opacityback=0, coltitle=black, title=\textcolor{quarto-callout-important-color}{\faExclamation}\hspace{0.5em}{Gráfico de Barras Segmentadas}, rightrule=.15mm, bottomrule=.15mm, toptitle=1mm, colback=white]

Um gráfico de barras segmentadas é \textbf{\emph{um gráfico de barras
para a apresentação de dados sobre duas variáveis categóricas no qual
cada barra é dividida em partes}}. Cada barra representa as observações
que assumem determinado valor de uma variável, e o
\textbf{\emph{comprimento de cada parte da barra representa a proporção
daquelas observações que assumem um valor específico da segunda
variável}}.

\end{tcolorbox}

A Figura 6.4 mostra um gráfico de mosaico, que é uma variação de um
gráfico de barras segmentadas.

Agora, as barras têm larguras diferentes, e essas larguras correspondem
à proporção de estudantes em cada uma das quatro categorias de grau.
Assim, as larguras mostram a distribuição marginal do grau conferido.
Cada barra é, novamente, dividida (segmentada) em duas partes,
representadas por dois tons de cinza. A porção superior (cinza-claro) de
cada barra representa a proporção de mulheres entre os estudantes que
receberam cada um dos graus. A outra porção (cinza-escuro) representa a
proporção de homens. Cada barra tem altura de 100\%, porque cada barra
representa todos os adultos em cada grupo diferente de pessoas. O
gráfico de mosaico é mais informativo do que o gráfico de barras
segmentadas porque mostra a distribuição marginal do grau conferido, bem
como a distribuição condicional de sexo, dado o grau conferido.

\begin{Shaded}
\begin{Highlighting}[numbers=left,,]
\InformationTok{\textasciigrave{}\textasciigrave{}\textasciigrave{}\{r\}}
\CommentTok{\# Gráfico mosaico para Grau x Sexo (preserva ordem dos níveis)}
\CommentTok{\# Instala e carrega pacote vcd opcionalmente (melhor visual). Fallback para mosaicplot().}

\ControlFlowTok{if}\NormalTok{ (}\SpecialCharTok{!}\FunctionTok{requireNamespace}\NormalTok{(}\StringTok{"vcd"}\NormalTok{, }\AttributeTok{quietly =} \ConstantTok{TRUE}\NormalTok{)) \{}
  \FunctionTok{message}\NormalTok{(}\StringTok{"Pacote \textquotesingle{}vcd\textquotesingle{} não encontrado — usando mosaicplot() base. Para visual mais rico, instale: install.packages(\textquotesingle{}vcd\textquotesingle{})"}\NormalTok{)}
\NormalTok{\} }\ControlFlowTok{else}\NormalTok{ \{}
  \FunctionTok{library}\NormalTok{(vcd)}
\NormalTok{\}}

\CommentTok{\# Criar data.frame com fatores e ordem explícita dos níveis}
\NormalTok{grausex }\OtherTok{\textless{}{-}} \FunctionTok{data.frame}\NormalTok{(}
  \AttributeTok{sex  =} \FunctionTok{factor}\NormalTok{(}\FunctionTok{c}\NormalTok{(}\FunctionTok{rep}\NormalTok{(}\StringTok{"Mulher"}\NormalTok{, }\DecValTok{2283}\NormalTok{), }\FunctionTok{rep}\NormalTok{(}\StringTok{"Homem"}\NormalTok{, }\DecValTok{1622}\NormalTok{)), }\AttributeTok{levels =} \FunctionTok{c}\NormalTok{(}\StringTok{"Mulher"}\NormalTok{, }\StringTok{"Homem"}\NormalTok{)),}
  \AttributeTok{grau =} \FunctionTok{factor}\NormalTok{(}\FunctionTok{c}\NormalTok{(}\FunctionTok{rep}\NormalTok{(}\StringTok{"ass"}\NormalTok{, }\DecValTok{639}\NormalTok{), }\FunctionTok{rep}\NormalTok{(}\StringTok{"bach"}\NormalTok{, }\DecValTok{1087}\NormalTok{), }\FunctionTok{rep}\NormalTok{(}\StringTok{"ms"}\NormalTok{, }\DecValTok{460}\NormalTok{), }\FunctionTok{rep}\NormalTok{(}\StringTok{"dr"}\NormalTok{, }\DecValTok{97}\NormalTok{),}
                  \FunctionTok{rep}\NormalTok{(}\StringTok{"ass"}\NormalTok{, }\DecValTok{402}\NormalTok{), }\FunctionTok{rep}\NormalTok{(}\StringTok{"bach"}\NormalTok{,  }\DecValTok{804}\NormalTok{), }\FunctionTok{rep}\NormalTok{(}\StringTok{"ms"}\NormalTok{, }\DecValTok{329}\NormalTok{), }\FunctionTok{rep}\NormalTok{(}\StringTok{"dr"}\NormalTok{, }\DecValTok{87}\NormalTok{)),}
                \AttributeTok{levels =} \FunctionTok{c}\NormalTok{(}\StringTok{"ass"}\NormalTok{, }\StringTok{"bach"}\NormalTok{, }\StringTok{"ms"}\NormalTok{, }\StringTok{"dr"}\NormalTok{))}
\NormalTok{)}

\CommentTok{\# Tabela de contingência (linhas = grau, colunas = sexo)}
\NormalTok{tab }\OtherTok{\textless{}{-}} \FunctionTok{table}\NormalTok{(grausex}\SpecialCharTok{$}\NormalTok{grau, grausex}\SpecialCharTok{$}\NormalTok{sex)}

\CommentTok{\# Exibir tabela para conferência}
\FunctionTok{print}\NormalTok{(tab)}

\CommentTok{\# {-}{-}{-} Gráfico mosaico com vcd::mosaic (se disponível) {-}{-}{-}}
\ControlFlowTok{if}\NormalTok{ (}\StringTok{"vcd"} \SpecialCharTok{\%in\%} \FunctionTok{loadedNamespaces}\NormalTok{()) \{}
  \CommentTok{\# mosaic(\textasciitilde{} grau + sex, ...) mostra pedaços por grau primeiro; shade colore conforme associação}
\NormalTok{  vcd}\SpecialCharTok{::}\FunctionTok{mosaic}\NormalTok{(}\SpecialCharTok{\textasciitilde{}}\NormalTok{ grau }\SpecialCharTok{+}\NormalTok{ sex, }\AttributeTok{data =}\NormalTok{ grausex,}
              \AttributeTok{shade =} \ConstantTok{TRUE}\NormalTok{, }\AttributeTok{legend =} \ConstantTok{TRUE}\NormalTok{,}
              \AttributeTok{main =} \StringTok{"Gráfico mosaico — Grau vs Sexo"}\NormalTok{,}
              \AttributeTok{labeling\_args =} \FunctionTok{list}\NormalTok{(}\AttributeTok{set\_varnames =} \FunctionTok{c}\NormalTok{(}\AttributeTok{grau =} \StringTok{"Grau"}\NormalTok{, }\AttributeTok{sex =} \StringTok{"Sexo"}\NormalTok{))}
\NormalTok{  )}
\NormalTok{\} }\ControlFlowTok{else}\NormalTok{ \{}
  \CommentTok{\# fallback: mosaicplot da base R}
  \CommentTok{\# note: mosaicplot espera tabela com categorias nas dimensões; color=TRUE usa paleta padrão}
  \FunctionTok{mosaicplot}\NormalTok{(tab,}
             \AttributeTok{main =} \StringTok{"Gráfico mosaico — Grau vs Sexo (base)"}\NormalTok{,}
             \AttributeTok{xlab =} \StringTok{"Grau"}\NormalTok{, }\AttributeTok{ylab =} \StringTok{"Sexo"}\NormalTok{,}
             \AttributeTok{color =} \ConstantTok{TRUE}\NormalTok{, }\AttributeTok{las =} \DecValTok{1}\NormalTok{)}
\NormalTok{\}}
\InformationTok{\textasciigrave{}\textasciigrave{}\textasciigrave{}}
\end{Highlighting}
\end{Shaded}

\begin{verbatim}
      
       Mulher Homem
  ass     639   402
  bach   1087   804
  ms      460   329
  dr       97    87
\end{verbatim}

\pandocbounded{\includegraphics[keepaspectratio]{cap6-moore-tab-dupla-entrada_files/figure-pdf/unnamed-chunk-17-1.pdf}}

\begin{tcolorbox}[enhanced jigsaw, arc=.35mm, opacitybacktitle=0.6, colframe=quarto-callout-important-color-frame, titlerule=0mm, leftrule=.75mm, left=2mm, colbacktitle=quarto-callout-important-color!10!white, breakable, toprule=.15mm, bottomtitle=1mm, opacityback=0, coltitle=black, title=\textcolor{quarto-callout-important-color}{\faExclamation}\hspace{0.5em}{Gráfico de mosaico}, rightrule=.15mm, bottomrule=.15mm, toptitle=1mm, colback=white]

\textbf{\emph{Um gráfico de barras segmentadas}} no qual a
\textbf{\emph{largura de cada barra representa a proporção de todas as
observações que estão na categoria que a barra representa}}.

\end{tcolorbox}

Mesmo gráfico acima indicando as percentagens no eixo y e seu valor
dentro de cada mosaico.

\begin{Shaded}
\begin{Highlighting}[numbers=left,,]
\InformationTok{\textasciigrave{}\textasciigrave{}\textasciigrave{}\{r\}}
\CommentTok{\# Mosaic{-}like com eixo y em percentagem (corrigido)}
\CommentTok{\# Corrige erro: "subset(...) deve ser lógico" substituindo uso incorreto de subset por indexação com match()}

\ControlFlowTok{if}\NormalTok{ (}\SpecialCharTok{!}\FunctionTok{requireNamespace}\NormalTok{(}\StringTok{"ggplot2"}\NormalTok{, }\AttributeTok{quietly =} \ConstantTok{TRUE}\NormalTok{)) }\FunctionTok{install.packages}\NormalTok{(}\StringTok{"ggplot2"}\NormalTok{)}
\ControlFlowTok{if}\NormalTok{ (}\SpecialCharTok{!}\FunctionTok{requireNamespace}\NormalTok{(}\StringTok{"scales"}\NormalTok{, }\AttributeTok{quietly =} \ConstantTok{TRUE}\NormalTok{)) }\FunctionTok{install.packages}\NormalTok{(}\StringTok{"scales"}\NormalTok{)}
\FunctionTok{library}\NormalTok{(ggplot2)}
\FunctionTok{library}\NormalTok{(scales)}

\CommentTok{\# Dados (preservar ordem dos níveis)}
\NormalTok{grausex }\OtherTok{\textless{}{-}} \FunctionTok{data.frame}\NormalTok{(}
  \AttributeTok{sex  =} \FunctionTok{factor}\NormalTok{(}\FunctionTok{c}\NormalTok{(}\FunctionTok{rep}\NormalTok{(}\StringTok{"M"}\NormalTok{, }\DecValTok{2283}\NormalTok{), }\FunctionTok{rep}\NormalTok{(}\StringTok{"H"}\NormalTok{, }\DecValTok{1622}\NormalTok{)), }\AttributeTok{levels =} \FunctionTok{c}\NormalTok{(}\StringTok{"M"}\NormalTok{, }\StringTok{"H"}\NormalTok{)),}
  \AttributeTok{grau =} \FunctionTok{factor}\NormalTok{(}\FunctionTok{c}\NormalTok{(}\FunctionTok{rep}\NormalTok{(}\StringTok{"ass"}\NormalTok{, }\DecValTok{639}\NormalTok{), }\FunctionTok{rep}\NormalTok{(}\StringTok{"bach"}\NormalTok{, }\DecValTok{1087}\NormalTok{), }\FunctionTok{rep}\NormalTok{(}\StringTok{"ms"}\NormalTok{, }\DecValTok{460}\NormalTok{), }\FunctionTok{rep}\NormalTok{(}\StringTok{"dr"}\NormalTok{, }\DecValTok{97}\NormalTok{),}
                  \FunctionTok{rep}\NormalTok{(}\StringTok{"ass"}\NormalTok{, }\DecValTok{402}\NormalTok{), }\FunctionTok{rep}\NormalTok{(}\StringTok{"bach"}\NormalTok{,  }\DecValTok{804}\NormalTok{), }\FunctionTok{rep}\NormalTok{(}\StringTok{"ms"}\NormalTok{, }\DecValTok{329}\NormalTok{), }\FunctionTok{rep}\NormalTok{(}\StringTok{"dr"}\NormalTok{, }\DecValTok{87}\NormalTok{)),}
                \AttributeTok{levels =} \FunctionTok{c}\NormalTok{(}\StringTok{"ass"}\NormalTok{, }\StringTok{"bach"}\NormalTok{, }\StringTok{"ms"}\NormalTok{, }\StringTok{"dr"}\NormalTok{))}
\NormalTok{)}

\CommentTok{\# Tabela de contingência grau x sexo}
\NormalTok{tab }\OtherTok{\textless{}{-}} \FunctionTok{as.data.frame}\NormalTok{(}\FunctionTok{table}\NormalTok{(grausex}\SpecialCharTok{$}\NormalTok{grau, grausex}\SpecialCharTok{$}\NormalTok{sex))}
\FunctionTok{names}\NormalTok{(tab) }\OtherTok{\textless{}{-}} \FunctionTok{c}\NormalTok{(}\StringTok{"grau"}\NormalTok{, }\StringTok{"sexo"}\NormalTok{, }\StringTok{"contagem"}\NormalTok{)}

\CommentTok{\# Totais por grau e total geral}
\NormalTok{totais\_grau }\OtherTok{\textless{}{-}} \FunctionTok{aggregate}\NormalTok{(contagem }\SpecialCharTok{\textasciitilde{}}\NormalTok{ grau, }\AttributeTok{data =}\NormalTok{ tab, sum)}
\NormalTok{total\_geral }\OtherTok{\textless{}{-}} \FunctionTok{sum}\NormalTok{(totais\_grau}\SpecialCharTok{$}\NormalTok{contagem)}

\CommentTok{\# Garantir ordem desejada das categorias de grau (usa levels do fator original)}
\NormalTok{ord }\OtherTok{\textless{}{-}} \FunctionTok{levels}\NormalTok{(grausex}\SpecialCharTok{$}\NormalTok{grau)}

\CommentTok{\# Obter as larguras (proporção de cada grau no total) na ordem correta}
\CommentTok{\# CORREÇÃO: usar indexação com match() em vez de subset(...)}
\NormalTok{larguras }\OtherTok{\textless{}{-}}\NormalTok{ totais\_grau[}\FunctionTok{match}\NormalTok{(ord, totais\_grau}\SpecialCharTok{$}\NormalTok{grau), ]}
\NormalTok{larguras}\SpecialCharTok{$}\NormalTok{prop }\OtherTok{\textless{}{-}}\NormalTok{ larguras}\SpecialCharTok{$}\NormalTok{contagem }\SpecialCharTok{/}\NormalTok{ total\_geral}
\CommentTok{\# posições x para cada coluna (xmin/xmax)}
\NormalTok{larguras}\SpecialCharTok{$}\NormalTok{cum\_prev }\OtherTok{\textless{}{-}} \FunctionTok{c}\NormalTok{(}\DecValTok{0}\NormalTok{, }\FunctionTok{head}\NormalTok{(}\FunctionTok{cumsum}\NormalTok{(larguras}\SpecialCharTok{$}\NormalTok{prop), }\SpecialCharTok{{-}}\DecValTok{1}\NormalTok{))}
\NormalTok{larguras}\SpecialCharTok{$}\NormalTok{xmin }\OtherTok{\textless{}{-}}\NormalTok{ larguras}\SpecialCharTok{$}\NormalTok{cum\_prev}
\NormalTok{larguras}\SpecialCharTok{$}\NormalTok{xmax }\OtherTok{\textless{}{-}}\NormalTok{ larguras}\SpecialCharTok{$}\NormalTok{cum\_prev }\SpecialCharTok{+}\NormalTok{ larguras}\SpecialCharTok{$}\NormalTok{prop}

\CommentTok{\# juntar totais por grau ao dataframe por linha}
\NormalTok{tab }\OtherTok{\textless{}{-}} \FunctionTok{merge}\NormalTok{(tab, totais\_grau, }\AttributeTok{by =} \StringTok{"grau"}\NormalTok{, }\AttributeTok{suffixes =} \FunctionTok{c}\NormalTok{(}\StringTok{""}\NormalTok{, }\StringTok{"\_grau"}\NormalTok{))}
\FunctionTok{names}\NormalTok{(tab)[}\FunctionTok{names}\NormalTok{(tab) }\SpecialCharTok{==} \StringTok{"contagem\_grau"}\NormalTok{] }\OtherTok{\textless{}{-}} \StringTok{"total\_grau"}

\CommentTok{\# proporção dentro de cada grau (P(sexo | grau))}
\NormalTok{tab}\SpecialCharTok{$}\NormalTok{prop\_within\_grau }\OtherTok{\textless{}{-}}\NormalTok{ tab}\SpecialCharTok{$}\NormalTok{contagem }\SpecialCharTok{/}\NormalTok{ tab}\SpecialCharTok{$}\NormalTok{total\_grau}

\CommentTok{\# juntar posições x ao dataframe tab (mantendo ordem por grau)}
\NormalTok{tab }\OtherTok{\textless{}{-}} \FunctionTok{merge}\NormalTok{(tab, larguras[, }\FunctionTok{c}\NormalTok{(}\StringTok{"grau"}\NormalTok{, }\StringTok{"xmin"}\NormalTok{, }\StringTok{"xmax"}\NormalTok{)], }\AttributeTok{by =} \StringTok{"grau"}\NormalTok{)}

\CommentTok{\# calcular posições y (ymin, ymax) empilhadas dentro de cada grau}
\NormalTok{tab }\OtherTok{\textless{}{-}}\NormalTok{ tab[}\FunctionTok{order}\NormalTok{(}\FunctionTok{match}\NormalTok{(tab}\SpecialCharTok{$}\NormalTok{grau, ord), tab}\SpecialCharTok{$}\NormalTok{sexo), ]  }\CommentTok{\# ordenar por grau na ordem desejada}
\NormalTok{tab }\OtherTok{\textless{}{-}} \FunctionTok{do.call}\NormalTok{(rbind, }\FunctionTok{lapply}\NormalTok{(}\FunctionTok{split}\NormalTok{(tab, tab}\SpecialCharTok{$}\NormalTok{grau), }\ControlFlowTok{function}\NormalTok{(dfg) \{}
\NormalTok{  dfg}\SpecialCharTok{$}\NormalTok{ymin }\OtherTok{\textless{}{-}} \FunctionTok{c}\NormalTok{(}\DecValTok{0}\NormalTok{, }\FunctionTok{head}\NormalTok{(}\FunctionTok{cumsum}\NormalTok{(dfg}\SpecialCharTok{$}\NormalTok{prop\_within\_grau), }\SpecialCharTok{{-}}\DecValTok{1}\NormalTok{))}
\NormalTok{  dfg}\SpecialCharTok{$}\NormalTok{ymax }\OtherTok{\textless{}{-}}\NormalTok{ dfg}\SpecialCharTok{$}\NormalTok{ymin }\SpecialCharTok{+}\NormalTok{ dfg}\SpecialCharTok{$}\NormalTok{prop\_within\_grau}
\NormalTok{  dfg}
\NormalTok{\}))}

\CommentTok{\# posições para rótulos (centro de cada segmento)}
\NormalTok{tab}\SpecialCharTok{$}\NormalTok{xmid }\OtherTok{\textless{}{-}}\NormalTok{ (tab}\SpecialCharTok{$}\NormalTok{xmin }\SpecialCharTok{+}\NormalTok{ tab}\SpecialCharTok{$}\NormalTok{xmax) }\SpecialCharTok{/} \DecValTok{2}
\NormalTok{tab}\SpecialCharTok{$}\NormalTok{ymid }\OtherTok{\textless{}{-}}\NormalTok{ (tab}\SpecialCharTok{$}\NormalTok{ymin }\SpecialCharTok{+}\NormalTok{ tab}\SpecialCharTok{$}\NormalTok{ymax) }\SpecialCharTok{/} \DecValTok{2}

\CommentTok{\# rótulo em percentagem e cor de rótulo para contraste}
\NormalTok{tab}\SpecialCharTok{$}\NormalTok{label }\OtherTok{\textless{}{-}} \FunctionTok{ifelse}\NormalTok{(tab}\SpecialCharTok{$}\NormalTok{prop\_within\_grau }\SpecialCharTok{\textgreater{}=} \FloatTok{0.03}\NormalTok{, }\FunctionTok{percent}\NormalTok{(tab}\SpecialCharTok{$}\NormalTok{prop\_within\_grau, }\AttributeTok{accuracy =} \FloatTok{0.1}\NormalTok{), }\StringTok{""}\NormalTok{)}
\NormalTok{tab}\SpecialCharTok{$}\NormalTok{label\_col }\OtherTok{\textless{}{-}} \FunctionTok{ifelse}\NormalTok{(tab}\SpecialCharTok{$}\NormalTok{prop\_within\_grau }\SpecialCharTok{\textgreater{}=} \FloatTok{0.15}\NormalTok{, }\StringTok{"white"}\NormalTok{, }\StringTok{"black"}\NormalTok{)}

\CommentTok{\# Texto de rodapé (nota)}
\NormalTok{rodape }\OtherTok{\textless{}{-}} \StringTok{"Gráfico de mosaico comparando as proporções de mulheres e homens entre aqueles em cada categoria}\SpecialCharTok{\textbackslash{}n}\StringTok{de grau a ser conferido (n = 3905)."}

\CommentTok{\# Plot: retângulos com largura proporcional (mosaic{-}like) e y em percentagem}
\NormalTok{p }\OtherTok{\textless{}{-}} \FunctionTok{ggplot}\NormalTok{(tab) }\SpecialCharTok{+}
  \FunctionTok{geom\_rect}\NormalTok{(}\FunctionTok{aes}\NormalTok{(}\AttributeTok{xmin =}\NormalTok{ xmin, }\AttributeTok{xmax =}\NormalTok{ xmax, }\AttributeTok{ymin =}\NormalTok{ ymin, }\AttributeTok{ymax =}\NormalTok{ ymax, }\AttributeTok{fill =}\NormalTok{ sexo),}
            \AttributeTok{colour =} \StringTok{"grey30"}\NormalTok{, }\AttributeTok{size =} \FloatTok{0.2}\NormalTok{) }\SpecialCharTok{+}
  \FunctionTok{geom\_text}\NormalTok{(}\FunctionTok{aes}\NormalTok{(}\AttributeTok{x =}\NormalTok{ xmid, }\AttributeTok{y =}\NormalTok{ ymid, }\AttributeTok{label =}\NormalTok{ label, }\AttributeTok{colour =}\NormalTok{ label\_col),}
            \AttributeTok{size =} \DecValTok{3}\NormalTok{, }\AttributeTok{fontface =} \StringTok{"bold"}\NormalTok{) }\SpecialCharTok{+}
  \FunctionTok{scale\_colour\_identity}\NormalTok{() }\SpecialCharTok{+}
  \FunctionTok{scale\_y\_continuous}\NormalTok{(}\AttributeTok{labels =} \FunctionTok{percent\_format}\NormalTok{(}\AttributeTok{accuracy =} \DecValTok{1}\NormalTok{), }\AttributeTok{breaks =} \FunctionTok{seq}\NormalTok{(}\DecValTok{0}\NormalTok{, }\DecValTok{1}\NormalTok{, }\AttributeTok{by =} \FloatTok{0.25}\NormalTok{)) }\SpecialCharTok{+}
  \FunctionTok{scale\_x\_continuous}\NormalTok{(}\AttributeTok{breaks =}\NormalTok{ (larguras}\SpecialCharTok{$}\NormalTok{xmin }\SpecialCharTok{+}\NormalTok{ larguras}\SpecialCharTok{$}\NormalTok{xmax) }\SpecialCharTok{/} \DecValTok{2}\NormalTok{,}
                     \AttributeTok{labels =}\NormalTok{ larguras}\SpecialCharTok{$}\NormalTok{grau,}
                     \AttributeTok{expand =} \FunctionTok{c}\NormalTok{(}\DecValTok{0}\NormalTok{, }\DecValTok{0}\NormalTok{)) }\SpecialCharTok{+}
  \FunctionTok{scale\_fill\_manual}\NormalTok{(}\AttributeTok{values =} \FunctionTok{c}\NormalTok{(}\StringTok{"\#4E79A7"}\NormalTok{, }\StringTok{"\#F28E2B"}\NormalTok{)) }\SpecialCharTok{+}
  \FunctionTok{labs}\NormalTok{(}\AttributeTok{title =} \StringTok{"Mosaico (estilo) — Grau x Sexo"}\NormalTok{,}
       \AttributeTok{subtitle =} \StringTok{"Distribuição condicional do Sexo (\%), uma vez dada a classe do Grau"}\NormalTok{,}
       \AttributeTok{x =} \StringTok{"Grau (largura proporcional ao total do grau)"}\NormalTok{,}
       \AttributeTok{y =} \StringTok{"Proporção dentro do grau (percentagem)"}\NormalTok{,}
       \AttributeTok{caption =}\NormalTok{ rodape,}
       \AttributeTok{fill =} \StringTok{"Sexo"}\NormalTok{) }\SpecialCharTok{+}
  \FunctionTok{theme\_minimal}\NormalTok{() }\SpecialCharTok{+}
  \FunctionTok{theme}\NormalTok{(}\AttributeTok{plot.title =} \FunctionTok{element\_text}\NormalTok{(}\AttributeTok{hjust =} \FloatTok{0.5}\NormalTok{),}
        \AttributeTok{panel.grid =} \FunctionTok{element\_blank}\NormalTok{())}

\FunctionTok{print}\NormalTok{(p)}
\InformationTok{\textasciigrave{}\textasciigrave{}\textasciigrave{}}
\end{Highlighting}
\end{Shaded}

\pandocbounded{\includegraphics[keepaspectratio]{cap6-moore-tab-dupla-entrada_files/figure-pdf/unnamed-chunk-18-1.pdf}}

A Figura 6.4 acima \textbf{\emph{mostra apenas um dos dois conjuntos de
distribuições condicionais}}.

Precisaríamos de outro gráfico para apresentar o outro (a distribuição
condicional de grau conferido, dado o sexo). Também, os gráficos nas
Figuras 6.3 e 6.4 indicam apenas porcentagens ou proporções, não
contagens totais.

Script R comentado para o gráfico mosaico anterior agora para o
\texttt{grau} dado o \texttt{sexo}.

\begin{Shaded}
\begin{Highlighting}[numbers=left,,]
\InformationTok{\textasciigrave{}\textasciigrave{}\textasciigrave{}\{r\}}
\CommentTok{\# Mosaic{-}like: Grau dado o Sexo (P(grau | sexo)) com rótulos "Mulher" / "Homem"}
\CommentTok{\# Legenda de \textquotesingle{}Grau\textquotesingle{} invertida (ordem mostrada de cima para baixo invertida)}

\ControlFlowTok{if}\NormalTok{ (}\SpecialCharTok{!}\FunctionTok{requireNamespace}\NormalTok{(}\StringTok{"ggplot2"}\NormalTok{, }\AttributeTok{quietly =} \ConstantTok{TRUE}\NormalTok{)) }\FunctionTok{install.packages}\NormalTok{(}\StringTok{"ggplot2"}\NormalTok{)}
\ControlFlowTok{if}\NormalTok{ (}\SpecialCharTok{!}\FunctionTok{requireNamespace}\NormalTok{(}\StringTok{"scales"}\NormalTok{, }\AttributeTok{quietly =} \ConstantTok{TRUE}\NormalTok{)) }\FunctionTok{install.packages}\NormalTok{(}\StringTok{"scales"}\NormalTok{)}
\FunctionTok{library}\NormalTok{(ggplot2)}
\FunctionTok{library}\NormalTok{(scales)}

\CommentTok{\# Dados (sexos nomeados como "Mulher" e "Homem", ordem preservada)}
\NormalTok{grausex }\OtherTok{\textless{}{-}} \FunctionTok{data.frame}\NormalTok{(}
  \AttributeTok{sex  =} \FunctionTok{factor}\NormalTok{(}\FunctionTok{c}\NormalTok{(}\FunctionTok{rep}\NormalTok{(}\StringTok{"Mulher"}\NormalTok{, }\DecValTok{2283}\NormalTok{), }\FunctionTok{rep}\NormalTok{(}\StringTok{"Homem"}\NormalTok{, }\DecValTok{1622}\NormalTok{)), }\AttributeTok{levels =} \FunctionTok{c}\NormalTok{(}\StringTok{"Mulher"}\NormalTok{, }\StringTok{"Homem"}\NormalTok{)),}
  \AttributeTok{grau =} \FunctionTok{factor}\NormalTok{(}\FunctionTok{c}\NormalTok{(}\FunctionTok{rep}\NormalTok{(}\StringTok{"ass"}\NormalTok{, }\DecValTok{639}\NormalTok{), }\FunctionTok{rep}\NormalTok{(}\StringTok{"bach"}\NormalTok{, }\DecValTok{1087}\NormalTok{), }\FunctionTok{rep}\NormalTok{(}\StringTok{"ms"}\NormalTok{, }\DecValTok{460}\NormalTok{), }\FunctionTok{rep}\NormalTok{(}\StringTok{"dr"}\NormalTok{, }\DecValTok{97}\NormalTok{),}
                  \FunctionTok{rep}\NormalTok{(}\StringTok{"ass"}\NormalTok{, }\DecValTok{402}\NormalTok{), }\FunctionTok{rep}\NormalTok{(}\StringTok{"bach"}\NormalTok{,  }\DecValTok{804}\NormalTok{), }\FunctionTok{rep}\NormalTok{(}\StringTok{"ms"}\NormalTok{, }\DecValTok{329}\NormalTok{), }\FunctionTok{rep}\NormalTok{(}\StringTok{"dr"}\NormalTok{, }\DecValTok{87}\NormalTok{)),}
                \AttributeTok{levels =} \FunctionTok{c}\NormalTok{(}\StringTok{"ass"}\NormalTok{, }\StringTok{"bach"}\NormalTok{, }\StringTok{"ms"}\NormalTok{, }\StringTok{"dr"}\NormalTok{))}
\NormalTok{)}

\CommentTok{\# Tabela de contagens por sexo x grau}
\NormalTok{tab }\OtherTok{\textless{}{-}} \FunctionTok{as.data.frame}\NormalTok{(}\FunctionTok{table}\NormalTok{(grausex}\SpecialCharTok{$}\NormalTok{sex, grausex}\SpecialCharTok{$}\NormalTok{grau))}
\FunctionTok{names}\NormalTok{(tab) }\OtherTok{\textless{}{-}} \FunctionTok{c}\NormalTok{(}\StringTok{"sexo"}\NormalTok{, }\StringTok{"grau"}\NormalTok{, }\StringTok{"contagem"}\NormalTok{)}

\CommentTok{\# Totais por sexo e total geral (para larguras das colunas)}
\NormalTok{totais\_sexo }\OtherTok{\textless{}{-}} \FunctionTok{aggregate}\NormalTok{(contagem }\SpecialCharTok{\textasciitilde{}}\NormalTok{ sexo, }\AttributeTok{data =}\NormalTok{ tab, sum)}
\NormalTok{total\_geral }\OtherTok{\textless{}{-}} \FunctionTok{sum}\NormalTok{(totais\_sexo}\SpecialCharTok{$}\NormalTok{contagem)}

\CommentTok{\# Garantir ordem desejada dos sexos e graus}
\NormalTok{ord\_sexo }\OtherTok{\textless{}{-}} \FunctionTok{levels}\NormalTok{(grausex}\SpecialCharTok{$}\NormalTok{sex)   }\CommentTok{\# c("Mulher","Homem")}
\NormalTok{ord\_grau }\OtherTok{\textless{}{-}} \FunctionTok{levels}\NormalTok{(grausex}\SpecialCharTok{$}\NormalTok{grau)  }\CommentTok{\# c("ass","bach","ms","dr")}

\CommentTok{\# Larguras proporcionais por sexo (na ordem correta)}
\NormalTok{larguras }\OtherTok{\textless{}{-}}\NormalTok{ totais\_sexo[}\FunctionTok{match}\NormalTok{(ord\_sexo, totais\_sexo}\SpecialCharTok{$}\NormalTok{sexo), ]}
\NormalTok{larguras}\SpecialCharTok{$}\NormalTok{prop }\OtherTok{\textless{}{-}}\NormalTok{ larguras}\SpecialCharTok{$}\NormalTok{contagem }\SpecialCharTok{/}\NormalTok{ total\_geral}
\NormalTok{larguras}\SpecialCharTok{$}\NormalTok{cum\_prev }\OtherTok{\textless{}{-}} \FunctionTok{c}\NormalTok{(}\DecValTok{0}\NormalTok{, }\FunctionTok{head}\NormalTok{(}\FunctionTok{cumsum}\NormalTok{(larguras}\SpecialCharTok{$}\NormalTok{prop), }\SpecialCharTok{{-}}\DecValTok{1}\NormalTok{))}
\NormalTok{larguras}\SpecialCharTok{$}\NormalTok{xmin }\OtherTok{\textless{}{-}}\NormalTok{ larguras}\SpecialCharTok{$}\NormalTok{cum\_prev}
\NormalTok{larguras}\SpecialCharTok{$}\NormalTok{xmax }\OtherTok{\textless{}{-}}\NormalTok{ larguras}\SpecialCharTok{$}\NormalTok{cum\_prev }\SpecialCharTok{+}\NormalTok{ larguras}\SpecialCharTok{$}\NormalTok{prop}

\CommentTok{\# Juntar totais ao dataframe por linha}
\NormalTok{tab }\OtherTok{\textless{}{-}} \FunctionTok{merge}\NormalTok{(tab, totais\_sexo, }\AttributeTok{by =} \StringTok{"sexo"}\NormalTok{, }\AttributeTok{suffixes =} \FunctionTok{c}\NormalTok{(}\StringTok{""}\NormalTok{, }\StringTok{"\_sexo"}\NormalTok{))}
\FunctionTok{names}\NormalTok{(tab)[}\FunctionTok{names}\NormalTok{(tab) }\SpecialCharTok{==} \StringTok{"contagem\_sexo"}\NormalTok{] }\OtherTok{\textless{}{-}} \StringTok{"total\_sexo"}

\CommentTok{\# Proporção dentro de cada sexo: P(grau | sexo)}
\NormalTok{tab}\SpecialCharTok{$}\NormalTok{prop\_within\_sexo }\OtherTok{\textless{}{-}}\NormalTok{ tab}\SpecialCharTok{$}\NormalTok{contagem }\SpecialCharTok{/}\NormalTok{ tab}\SpecialCharTok{$}\NormalTok{total\_sexo}

\CommentTok{\# Juntar posições x (xmin/xmax) ao dataframe tab}
\NormalTok{tab }\OtherTok{\textless{}{-}} \FunctionTok{merge}\NormalTok{(tab, larguras[, }\FunctionTok{c}\NormalTok{(}\StringTok{"sexo"}\NormalTok{, }\StringTok{"xmin"}\NormalTok{, }\StringTok{"xmax"}\NormalTok{)], }\AttributeTok{by =} \StringTok{"sexo"}\NormalTok{)}

\CommentTok{\# Calcular posições y empilhadas (ymin/ymax) dentro de cada sexo, mantendo ordem de grau}
\NormalTok{tab }\OtherTok{\textless{}{-}}\NormalTok{ tab[}\FunctionTok{order}\NormalTok{(}\FunctionTok{match}\NormalTok{(tab}\SpecialCharTok{$}\NormalTok{sexo, ord\_sexo), }\FunctionTok{match}\NormalTok{(tab}\SpecialCharTok{$}\NormalTok{grau, ord\_grau)), ]}
\NormalTok{tab }\OtherTok{\textless{}{-}} \FunctionTok{do.call}\NormalTok{(rbind, }\FunctionTok{lapply}\NormalTok{(}\FunctionTok{split}\NormalTok{(tab, tab}\SpecialCharTok{$}\NormalTok{sexo), }\ControlFlowTok{function}\NormalTok{(dfg) \{}
\NormalTok{  dfg }\OtherTok{\textless{}{-}}\NormalTok{ dfg[}\FunctionTok{order}\NormalTok{(}\FunctionTok{match}\NormalTok{(dfg}\SpecialCharTok{$}\NormalTok{grau, ord\_grau)), ]}
\NormalTok{  dfg}\SpecialCharTok{$}\NormalTok{ymin }\OtherTok{\textless{}{-}} \FunctionTok{c}\NormalTok{(}\DecValTok{0}\NormalTok{, }\FunctionTok{head}\NormalTok{(}\FunctionTok{cumsum}\NormalTok{(dfg}\SpecialCharTok{$}\NormalTok{prop\_within\_sexo), }\SpecialCharTok{{-}}\DecValTok{1}\NormalTok{))}
\NormalTok{  dfg}\SpecialCharTok{$}\NormalTok{ymax }\OtherTok{\textless{}{-}}\NormalTok{ dfg}\SpecialCharTok{$}\NormalTok{ymin }\SpecialCharTok{+}\NormalTok{ dfg}\SpecialCharTok{$}\NormalTok{prop\_within\_sexo}
\NormalTok{  dfg}
\NormalTok{\}))}

\CommentTok{\# Posições para rótulos (centro de cada segmento)}
\NormalTok{tab}\SpecialCharTok{$}\NormalTok{xmid }\OtherTok{\textless{}{-}}\NormalTok{ (tab}\SpecialCharTok{$}\NormalTok{xmin }\SpecialCharTok{+}\NormalTok{ tab}\SpecialCharTok{$}\NormalTok{xmax) }\SpecialCharTok{/} \DecValTok{2}
\NormalTok{tab}\SpecialCharTok{$}\NormalTok{ymid }\OtherTok{\textless{}{-}}\NormalTok{ (tab}\SpecialCharTok{$}\NormalTok{ymin }\SpecialCharTok{+}\NormalTok{ tab}\SpecialCharTok{$}\NormalTok{ymax) }\SpecialCharTok{/} \DecValTok{2}

\CommentTok{\# Rótulos em percentagem (mostrar só quando segmento \textgreater{}= 3\% dentro do sexo)}
\NormalTok{tab}\SpecialCharTok{$}\NormalTok{label }\OtherTok{\textless{}{-}} \FunctionTok{ifelse}\NormalTok{(tab}\SpecialCharTok{$}\NormalTok{prop\_within\_sexo }\SpecialCharTok{\textgreater{}=} \FloatTok{0.03}\NormalTok{, }\FunctionTok{percent}\NormalTok{(tab}\SpecialCharTok{$}\NormalTok{prop\_within\_sexo, }\AttributeTok{accuracy =} \FloatTok{0.1}\NormalTok{), }\StringTok{""}\NormalTok{)}
\NormalTok{tab}\SpecialCharTok{$}\NormalTok{label\_col }\OtherTok{\textless{}{-}} \FunctionTok{ifelse}\NormalTok{(tab}\SpecialCharTok{$}\NormalTok{prop\_within\_sexo }\SpecialCharTok{\textgreater{}=} \FloatTok{0.15}\NormalTok{, }\StringTok{"white"}\NormalTok{, }\StringTok{"black"}\NormalTok{)}

\CommentTok{\# Texto de rodapé (opcional)}
\NormalTok{rodape }\OtherTok{\textless{}{-}} \StringTok{"Gráfico de mosaico: proporção de graus dentro de cada sexo (P(grau | sexo))."}

\CommentTok{\# Plot: a única diferença em relação ao script anterior é a inversão da legenda de \textquotesingle{}grau\textquotesingle{}}
\NormalTok{p }\OtherTok{\textless{}{-}} \FunctionTok{ggplot}\NormalTok{(tab) }\SpecialCharTok{+}
  \FunctionTok{geom\_rect}\NormalTok{(}\FunctionTok{aes}\NormalTok{(}\AttributeTok{xmin =}\NormalTok{ xmin, }\AttributeTok{xmax =}\NormalTok{ xmax, }\AttributeTok{ymin =}\NormalTok{ ymin, }\AttributeTok{ymax =}\NormalTok{ ymax, }\AttributeTok{fill =}\NormalTok{ grau),}
            \AttributeTok{colour =} \StringTok{"grey30"}\NormalTok{, }\AttributeTok{size =} \FloatTok{0.2}\NormalTok{) }\SpecialCharTok{+}
  \FunctionTok{geom\_text}\NormalTok{(}\FunctionTok{aes}\NormalTok{(}\AttributeTok{x =}\NormalTok{ xmid, }\AttributeTok{y =}\NormalTok{ ymid, }\AttributeTok{label =}\NormalTok{ label, }\AttributeTok{colour =}\NormalTok{ label\_col),}
            \AttributeTok{size =} \DecValTok{3}\NormalTok{, }\AttributeTok{fontface =} \StringTok{"bold"}\NormalTok{) }\SpecialCharTok{+}
  \FunctionTok{scale\_colour\_identity}\NormalTok{() }\SpecialCharTok{+}
  \FunctionTok{scale\_y\_continuous}\NormalTok{(}\AttributeTok{labels =} \FunctionTok{percent\_format}\NormalTok{(}\AttributeTok{accuracy =} \DecValTok{1}\NormalTok{), }\AttributeTok{breaks =} \FunctionTok{seq}\NormalTok{(}\DecValTok{0}\NormalTok{, }\DecValTok{1}\NormalTok{, }\AttributeTok{by =} \FloatTok{0.25}\NormalTok{)) }\SpecialCharTok{+}
  \FunctionTok{scale\_x\_continuous}\NormalTok{(}\AttributeTok{breaks =}\NormalTok{ (larguras}\SpecialCharTok{$}\NormalTok{xmin }\SpecialCharTok{+}\NormalTok{ larguras}\SpecialCharTok{$}\NormalTok{xmax) }\SpecialCharTok{/} \DecValTok{2}\NormalTok{,}
                     \AttributeTok{labels =}\NormalTok{ larguras}\SpecialCharTok{$}\NormalTok{sexo,}
                     \AttributeTok{expand =} \FunctionTok{c}\NormalTok{(}\DecValTok{0}\NormalTok{, }\DecValTok{0}\NormalTok{)) }\SpecialCharTok{+}
  \FunctionTok{scale\_fill\_brewer}\NormalTok{(}\AttributeTok{type =} \StringTok{"qual"}\NormalTok{, }\AttributeTok{palette =} \StringTok{"Set2"}\NormalTok{, }\AttributeTok{labels =}\NormalTok{ ord\_grau) }\SpecialCharTok{+}
  \FunctionTok{labs}\NormalTok{(}\AttributeTok{title =} \StringTok{"Mosaic (estilo) — Grau dado o Sexo"}\NormalTok{,}
       \AttributeTok{x =} \StringTok{"Sexo (largura proporcional ao total por sexo)"}\NormalTok{,}
       \AttributeTok{y =} \StringTok{"Proporção dentro do sexo (percentagem)"}\NormalTok{,}
       \AttributeTok{fill =} \StringTok{"Grau"}\NormalTok{,}
       \AttributeTok{caption =}\NormalTok{ rodape) }\SpecialCharTok{+}
  \FunctionTok{guides}\NormalTok{(}\AttributeTok{fill =} \FunctionTok{guide\_legend}\NormalTok{(}\AttributeTok{reverse =} \ConstantTok{TRUE}\NormalTok{)) }\SpecialCharTok{+}   \CommentTok{\# INVERTE apenas a ordem da legenda de \textquotesingle{}Grau\textquotesingle{}}
  \FunctionTok{theme\_minimal}\NormalTok{() }\SpecialCharTok{+}
  \FunctionTok{theme}\NormalTok{(}\AttributeTok{plot.title =} \FunctionTok{element\_text}\NormalTok{(}\AttributeTok{hjust =} \FloatTok{0.5}\NormalTok{),}
        \AttributeTok{panel.grid =} \FunctionTok{element\_blank}\NormalTok{(),}
        \AttributeTok{plot.caption =} \FunctionTok{element\_text}\NormalTok{(}\AttributeTok{hjust =} \DecValTok{0}\NormalTok{, }\AttributeTok{size =} \DecValTok{9}\NormalTok{, }\AttributeTok{face =} \StringTok{"italic"}\NormalTok{, }\AttributeTok{margin =} \FunctionTok{margin}\NormalTok{(}\AttributeTok{t =} \DecValTok{8}\NormalTok{)))}

\FunctionTok{print}\NormalTok{(p)}
\InformationTok{\textasciigrave{}\textasciigrave{}\textasciigrave{}}
\end{Highlighting}
\end{Shaded}

\pandocbounded{\includegraphics[keepaspectratio]{cap6-moore-tab-dupla-entrada_files/figure-pdf/unnamed-chunk-19-1.pdf}}

\begin{tcolorbox}[enhanced jigsaw, arc=.35mm, opacitybacktitle=0.6, colframe=quarto-callout-important-color-frame, titlerule=0mm, leftrule=.75mm, left=2mm, colbacktitle=quarto-callout-important-color!10!white, breakable, toprule=.15mm, bottomtitle=1mm, opacityback=0, coltitle=black, title=\textcolor{quarto-callout-important-color}{\faExclamation}\hspace{0.5em}{Gráficos!}, rightrule=.15mm, bottomrule=.15mm, toptitle=1mm, colback=white]

Nenhum gráfico único retrata a forma da relação entre variáveis
categóricas (como um diagrama de dispersão faz para variáveis
quantitativas).

Nenhuma medida numérica única (como a correlação) resume a intensidade
da associação.

Gráficos de barras são flexíveis o bastante para serem úteis, mas você
deve pensar sobre quais comparações você deseja apresentar.

Para medidas numéricas, confiamos em porcentagens bem escolhidas. Convém
você decidir de quais porcentagens você precisa.

Eis uma sugestão: \textbf{\emph{se houver uma relação
explicativa-resposta}}, \ul{\textbf{\emph{compare}}} as
\textbf{\emph{distribuições condicionais da variável resposta para os
valores separados da variável explicativa}}. Se você acha que
\ul{\textbf{\emph{sexo influencia o grau conferido}}},
\ul{\textbf{compare}} as \ul{\textbf{\emph{distribuições condicionais de
grau conferido para cada categoria de sexo}}}, como no Exemplo 6.3.

\end{tcolorbox}

\subsection{Aplique seu conhecimento}\label{aplique-seu-conhecimento-1}

\subsubsection{6.3 Videogames e
conceitos.}\label{videogames-e-conceitos.-1}

O Exercício 6.1 fornece os dados sobre a distribuição de conceitos de
meninos que jogaram e não jogaram videogames.

Para \textbf{\emph{ver a relação}} entre \texttt{conceitos} e
\texttt{jogar\ videogames}, \ul{\textbf{determine as distribuições
condicionais de conceitos (a variável resposta) para jogadores e não
jogadores}}. O que você conclui?

\begin{Shaded}
\begin{Highlighting}[numbers=left,,]
\InformationTok{\textasciigrave{}\textasciigrave{}\textasciigrave{}\{r\}}
\CommentTok{\# Criar tabela agregada: Conceito x Jogaram videogame}
\NormalTok{counts }\OtherTok{\textless{}{-}} \FunctionTok{matrix}\NormalTok{(}
  \FunctionTok{c}\NormalTok{(}
    \DecValTok{736}\NormalTok{, }\DecValTok{450}\NormalTok{, }\DecValTok{193}\NormalTok{,   }\CommentTok{\# "Jogaram videogame"}
    \DecValTok{205}\NormalTok{, }\DecValTok{144}\NormalTok{,  }\DecValTok{80}    \CommentTok{\# "Nunca jogaram videogame"}
\NormalTok{  ),}
  \AttributeTok{nrow =} \DecValTok{2}\NormalTok{,}
  \AttributeTok{byrow =} \ConstantTok{TRUE}
\NormalTok{)}

\FunctionTok{rownames}\NormalTok{(counts) }\OtherTok{\textless{}{-}} \FunctionTok{c}\NormalTok{(}\StringTok{"Jogaram videogame"}\NormalTok{, }\StringTok{"Nunca jogaram videogame"}\NormalTok{)}
\FunctionTok{colnames}\NormalTok{(counts) }\OtherTok{\textless{}{-}} \FunctionTok{c}\NormalTok{(}\StringTok{"As e Bs"}\NormalTok{, }\StringTok{"Cs"}\NormalTok{, }\StringTok{"Ds e Fs"}\NormalTok{)}

\CommentTok{\# Converter para objeto table}
\NormalTok{tab }\OtherTok{\textless{}{-}} \FunctionTok{as.table}\NormalTok{(counts)}

\CommentTok{\# Exibir tabela de contagens}
\FunctionTok{cat}\NormalTok{(}\StringTok{"Tabela de contagens (linhas = status, colunas = conceito):}\SpecialCharTok{\textbackslash{}n}\StringTok{"}\NormalTok{)}
\FunctionTok{print}\NormalTok{(tab)}

\CommentTok{\# {-}{-}{-} Distribuições condicionais: P(conceito | status) {-}{-}{-}}
\CommentTok{\# margin = 1 ==\textgreater{} soma por linha = 1 (cada status)}
\NormalTok{condicional }\OtherTok{\textless{}{-}} \FunctionTok{prop.table}\NormalTok{(tab, }\AttributeTok{margin =} \DecValTok{1}\NormalTok{)}

\FunctionTok{cat}\NormalTok{(}\StringTok{"}\SpecialCharTok{\textbackslash{}n}\StringTok{Distribuições condicionais P(conceito | status) (proporções):}\SpecialCharTok{\textbackslash{}n}\StringTok{"}\NormalTok{)}
\FunctionTok{print}\NormalTok{(}\FunctionTok{round}\NormalTok{(condicional, }\DecValTok{4}\NormalTok{))}

\FunctionTok{cat}\NormalTok{(}\StringTok{"}\SpecialCharTok{\textbackslash{}n}\StringTok{Distribuições condicionais P(conceito | status) (percentuais):}\SpecialCharTok{\textbackslash{}n}\StringTok{"}\NormalTok{)}
\FunctionTok{print}\NormalTok{(}\FunctionTok{round}\NormalTok{(}\DecValTok{100} \SpecialCharTok{*}\NormalTok{ condicional, }\DecValTok{2}\NormalTok{))}

\CommentTok{\# {-}{-}{-} Data.frame "long" com contagens e proporções (útil para plotagem) {-}{-}{-}}
\NormalTok{df }\OtherTok{\textless{}{-}} \FunctionTok{as.data.frame}\NormalTok{(tab)}
\FunctionTok{names}\NormalTok{(df) }\OtherTok{\textless{}{-}} \FunctionTok{c}\NormalTok{(}\StringTok{"status"}\NormalTok{, }\StringTok{"conceito"}\NormalTok{, }\StringTok{"contagem"}\NormalTok{)}

\CommentTok{\# proporção de cada conceito dentro do respectivo status}
\NormalTok{df}\SpecialCharTok{$}\NormalTok{proporcao }\OtherTok{\textless{}{-}}\NormalTok{ df}\SpecialCharTok{$}\NormalTok{contagem }\SpecialCharTok{/} \FunctionTok{ave}\NormalTok{(df}\SpecialCharTok{$}\NormalTok{contagem, df}\SpecialCharTok{$}\NormalTok{status, }\AttributeTok{FUN =}\NormalTok{ sum)}
\NormalTok{df}\SpecialCharTok{$}\NormalTok{percent }\OtherTok{\textless{}{-}} \FunctionTok{round}\NormalTok{(}\DecValTok{100} \SpecialCharTok{*}\NormalTok{ df}\SpecialCharTok{$}\NormalTok{proporcao, }\DecValTok{2}\NormalTok{)}

\FunctionTok{cat}\NormalTok{(}\StringTok{"}\SpecialCharTok{\textbackslash{}n}\StringTok{Data.frame com contagem + proporção condicional (ordenado):}\SpecialCharTok{\textbackslash{}n}\StringTok{"}\NormalTok{)}
\FunctionTok{print}\NormalTok{(df)}
\FunctionTok{cat}\NormalTok{(}\StringTok{"}\SpecialCharTok{\textbackslash{}n}\StringTok{Tamanho da amostra:"}\NormalTok{, }\FunctionTok{sum}\NormalTok{(df}\SpecialCharTok{$}\NormalTok{contagem),}\StringTok{"}\SpecialCharTok{\textbackslash{}n}\StringTok{"}\NormalTok{)}

\CommentTok{\# {-}{-}{-} Teste de independência (opcional) {-}{-}{-}}
\CommentTok{\# chisq \textless{}{-} chisq.test(tab)}
\CommentTok{\# print(chisq)}

\CommentTok{\# {-}{-}{-} Gráfico: barras empilhadas mostrando composição por status (cada barra = 100\%) {-}{-}{-}}
\ControlFlowTok{if}\NormalTok{ (}\SpecialCharTok{!}\FunctionTok{requireNamespace}\NormalTok{(}\StringTok{"ggplot2"}\NormalTok{, }\AttributeTok{quietly =} \ConstantTok{TRUE}\NormalTok{)) }\FunctionTok{install.packages}\NormalTok{(}\StringTok{"ggplot2"}\NormalTok{)}
\ControlFlowTok{if}\NormalTok{ (}\SpecialCharTok{!}\FunctionTok{requireNamespace}\NormalTok{(}\StringTok{"scales"}\NormalTok{, }\AttributeTok{quietly =} \ConstantTok{TRUE}\NormalTok{)) }\FunctionTok{install.packages}\NormalTok{(}\StringTok{"scales"}\NormalTok{)}
\FunctionTok{library}\NormalTok{(ggplot2)}
\FunctionTok{library}\NormalTok{(scales)}

\CommentTok{\# Preservar ordem das categorias}
\NormalTok{df}\SpecialCharTok{$}\NormalTok{status }\OtherTok{\textless{}{-}} \FunctionTok{factor}\NormalTok{(df}\SpecialCharTok{$}\NormalTok{status, }\AttributeTok{levels =} \FunctionTok{c}\NormalTok{(}\StringTok{"Jogaram videogame"}\NormalTok{, }\StringTok{"Nunca jogaram videogame"}\NormalTok{))}
\NormalTok{df}\SpecialCharTok{$}\NormalTok{conceito }\OtherTok{\textless{}{-}} \FunctionTok{factor}\NormalTok{(df}\SpecialCharTok{$}\NormalTok{conceito, }\AttributeTok{levels =} \FunctionTok{c}\NormalTok{(}\StringTok{"As e Bs"}\NormalTok{, }\StringTok{"Cs"}\NormalTok{, }\StringTok{"Ds e Fs"}\NormalTok{))}

\NormalTok{p }\OtherTok{\textless{}{-}} \FunctionTok{ggplot}\NormalTok{(df, }\FunctionTok{aes}\NormalTok{(}\AttributeTok{x =}\NormalTok{ status, }\AttributeTok{y =}\NormalTok{ contagem, }\AttributeTok{fill =}\NormalTok{ conceito)) }\SpecialCharTok{+}
  \FunctionTok{geom\_col}\NormalTok{(}\AttributeTok{position =} \StringTok{"fill"}\NormalTok{, }\AttributeTok{colour =} \StringTok{"grey30"}\NormalTok{, }\AttributeTok{size =} \FloatTok{0.2}\NormalTok{) }\SpecialCharTok{+}
  \FunctionTok{scale\_y\_continuous}\NormalTok{(}\AttributeTok{labels =} \FunctionTok{percent\_format}\NormalTok{(}\AttributeTok{accuracy =} \DecValTok{1}\NormalTok{)) }\SpecialCharTok{+}
  \FunctionTok{geom\_text}\NormalTok{(}\FunctionTok{aes}\NormalTok{(}\AttributeTok{label =} \FunctionTok{ifelse}\NormalTok{(proporcao }\SpecialCharTok{\textgreater{}=} \FloatTok{0.03}\NormalTok{, }\FunctionTok{percent}\NormalTok{(proporcao, }\AttributeTok{accuracy =} \FloatTok{0.1}\NormalTok{), }\StringTok{""}\NormalTok{)),}
            \AttributeTok{position =} \FunctionTok{position\_fill}\NormalTok{(}\AttributeTok{vjust =} \FloatTok{0.5}\NormalTok{), }\AttributeTok{size =} \DecValTok{3}\NormalTok{, }\AttributeTok{colour =} \StringTok{"white"}\NormalTok{, }\AttributeTok{fontface =} \StringTok{"bold"}\NormalTok{) }\SpecialCharTok{+}
  \FunctionTok{labs}\NormalTok{(}
    \AttributeTok{title =} \StringTok{"Composição por Conceito dentro de cada Status de Jogar videogame"}\NormalTok{,}
    \AttributeTok{x =} \StringTok{"Status"}\NormalTok{,}
    \AttributeTok{y =} \StringTok{"Proporção dentro do status"}\NormalTok{,}
    \AttributeTok{fill =} \StringTok{"Conceito"}
\NormalTok{  ) }\SpecialCharTok{+}
  \FunctionTok{theme\_minimal}\NormalTok{() }\SpecialCharTok{+}
  \FunctionTok{theme}\NormalTok{(}\AttributeTok{plot.title =} \FunctionTok{element\_text}\NormalTok{(}\AttributeTok{hjust =} \FloatTok{0.5}\NormalTok{))}

\FunctionTok{print}\NormalTok{(p)}
\InformationTok{\textasciigrave{}\textasciigrave{}\textasciigrave{}}
\end{Highlighting}
\end{Shaded}

\begin{verbatim}
Tabela de contagens (linhas = status, colunas = conceito):
                        As e Bs  Cs Ds e Fs
Jogaram videogame           736 450     193
Nunca jogaram videogame     205 144      80

Distribuições condicionais P(conceito | status) (proporções):
                        As e Bs   Cs Ds e Fs
Jogaram videogame          0.53 0.33    0.14
Nunca jogaram videogame    0.48 0.34    0.19

Distribuições condicionais P(conceito | status) (percentuais):
                        As e Bs Cs Ds e Fs
Jogaram videogame            53 33      14
Nunca jogaram videogame      48 34      19

Data.frame com contagem + proporção condicional (ordenado):
                   status conceito contagem proporcao percent
1       Jogaram videogame  As e Bs      736      0.53      53
2 Nunca jogaram videogame  As e Bs      205      0.48      48
3       Jogaram videogame       Cs      450      0.33      33
4 Nunca jogaram videogame       Cs      144      0.34      34
5       Jogaram videogame  Ds e Fs      193      0.14      14
6 Nunca jogaram videogame  Ds e Fs       80      0.19      19

Tamanho da amostra: 1808 
\end{verbatim}

\pandocbounded{\includegraphics[keepaspectratio]{cap6-moore-tab-dupla-entrada_files/figure-pdf/unnamed-chunk-20-1.pdf}}

Considerando a variável \texttt{Jogar\ Videogame} como explicativa e a
variável \texttt{Conceito} como resposta, então os meninos que já
jogaram apresentam Conceito A ou B em maior proporção (53,4\%) que
aqueles que nunca jograram (47,8\%).

Os jogadores tiveram conceitos um pouco maiores que os não jogadores, é
o que podemos dizer, \textbf{\emph{mas isso poderia se dever ao acaso}}.

Muito embora a amostra tenha um tamanho = 1808.

Não pode esquecer da reflexão sobre variáveis ocultas.

\textbf{Qual as possíveis variáveis ocultas nesse exemplo?}

\subsubsection{6.3 Idades de
universitários.}\label{idades-de-universituxe1rios.-1}

O Exercício 6.2 fornece dados do \emph{U.S. Census Bureau} que descrevem
a \texttt{idade} e o \texttt{sexo} de todos os estudantes universitários
americanos.

Suspeitamos de que o percentual de mulheres seja maior entre estudantes
na faixa etária de 25 a 34 anos do que na faixa etária de 20 a 24 anos.
Os dados apoiam essa suspeita? Siga o processo dos quatro passos, como
ilustrado no Exemplo 6.3.

\begin{Shaded}
\begin{Highlighting}[numbers=left,,]
\InformationTok{\textasciigrave{}\textasciigrave{}\textasciigrave{}\{r\}}
\CommentTok{\# Análise: comparar percentual de mulheres em 25{-}34 vs 20{-}24}
\CommentTok{\# Entrada: tabela longa com colunas Age, Sex, Count (conforme fornecido)}
\CommentTok{\#}
\CommentTok{\# Saídas:}
\CommentTok{\# {-} Distribuições condicionais P(Mulher | faixa)}
\CommentTok{\# {-} Teste de duas proporções (unilateral H1: p25{-}34 \textgreater{} p20{-}24)}
\CommentTok{\# {-} Gráfico das proporções com IC95\%}

\CommentTok{\# Pacotes}
\ControlFlowTok{if}\NormalTok{ (}\SpecialCharTok{!}\FunctionTok{requireNamespace}\NormalTok{(}\StringTok{"dplyr"}\NormalTok{, }\AttributeTok{quietly =} \ConstantTok{TRUE}\NormalTok{)) }\FunctionTok{install.packages}\NormalTok{(}\StringTok{"dplyr"}\NormalTok{)}
\ControlFlowTok{if}\NormalTok{ (}\SpecialCharTok{!}\FunctionTok{requireNamespace}\NormalTok{(}\StringTok{"ggplot2"}\NormalTok{, }\AttributeTok{quietly =} \ConstantTok{TRUE}\NormalTok{)) }\FunctionTok{install.packages}\NormalTok{(}\StringTok{"ggplot2"}\NormalTok{)}
\ControlFlowTok{if}\NormalTok{ (}\SpecialCharTok{!}\FunctionTok{requireNamespace}\NormalTok{(}\StringTok{"scales"}\NormalTok{, }\AttributeTok{quietly =} \ConstantTok{TRUE}\NormalTok{)) }\FunctionTok{install.packages}\NormalTok{(}\StringTok{"scales"}\NormalTok{)}
\FunctionTok{library}\NormalTok{(dplyr); }\FunctionTok{library}\NormalTok{(ggplot2); }\FunctionTok{library}\NormalTok{(scales)}

\CommentTok{\# {-}{-}{-} Criar data.frame a partir da tabela longa fornecida {-}{-}{-}}
\NormalTok{df\_long }\OtherTok{\textless{}{-}} \FunctionTok{data.frame}\NormalTok{(}
  \AttributeTok{Age  =} \FunctionTok{c}\NormalTok{(}\StringTok{"15to19"}\NormalTok{,}\StringTok{"20to24"}\NormalTok{,}\StringTok{"25to34"}\NormalTok{,}\StringTok{"35up"}\NormalTok{,}
           \StringTok{"15to19"}\NormalTok{,}\StringTok{"20to24"}\NormalTok{,}\StringTok{"25to34"}\NormalTok{,}\StringTok{"35up"}\NormalTok{),}
  \AttributeTok{Sex  =} \FunctionTok{c}\NormalTok{(}\StringTok{"Female"}\NormalTok{,}\StringTok{"Female"}\NormalTok{,}\StringTok{"Female"}\NormalTok{,}\StringTok{"Female"}\NormalTok{,}
           \StringTok{"Male"}\NormalTok{,}\StringTok{"Male"}\NormalTok{,}\StringTok{"Male"}\NormalTok{,}\StringTok{"Male"}\NormalTok{),}
  \AttributeTok{Count =} \FunctionTok{c}\NormalTok{(}\DecValTok{2348}\NormalTok{, }\DecValTok{4280}\NormalTok{, }\DecValTok{2166}\NormalTok{, }\DecValTok{1492}\NormalTok{,}
            \DecValTok{1831}\NormalTok{, }\DecValTok{3713}\NormalTok{, }\DecValTok{1714}\NormalTok{,  }\DecValTok{853}\NormalTok{),}
  \AttributeTok{stringsAsFactors =} \ConstantTok{FALSE}
\NormalTok{)}

\CommentTok{\# Exibir tamanho da amostra}
\FunctionTok{cat}\NormalTok{(}\StringTok{"}\SpecialCharTok{\textbackslash{}n}\StringTok{total das contagens (tamanho amostra) n:}\SpecialCharTok{\textbackslash{}n}\StringTok{"}\NormalTok{)}
\FunctionTok{print}\NormalTok{( }\FunctionTok{sum}\NormalTok{(df\_long}\SpecialCharTok{$}\NormalTok{Count) )}

\CommentTok{\# Padronizar rótulos de sexo para pt{-}BR (opcional)}
\NormalTok{df\_long }\OtherTok{\textless{}{-}}\NormalTok{ df\_long }\SpecialCharTok{\%\textgreater{}\%}
  \FunctionTok{mutate}\NormalTok{(}\AttributeTok{sexo =} \FunctionTok{case\_when}\NormalTok{(}
    \FunctionTok{tolower}\NormalTok{(Sex) }\SpecialCharTok{\%in\%} \FunctionTok{c}\NormalTok{(}\StringTok{"female"}\NormalTok{, }\StringTok{"f"}\NormalTok{) }\SpecialCharTok{\textasciitilde{}} \StringTok{"Mulher"}\NormalTok{,}
    \FunctionTok{tolower}\NormalTok{(Sex) }\SpecialCharTok{\%in\%} \FunctionTok{c}\NormalTok{(}\StringTok{"male"}\NormalTok{, }\StringTok{"m"}\NormalTok{)   }\SpecialCharTok{\textasciitilde{}} \StringTok{"Homem"}\NormalTok{,}
    \ConstantTok{TRUE} \SpecialCharTok{\textasciitilde{}}\NormalTok{ Sex}
\NormalTok{  ),}
  \AttributeTok{faixa =} \FunctionTok{case\_when}\NormalTok{(}
\NormalTok{    Age }\SpecialCharTok{==} \StringTok{"15to19"} \SpecialCharTok{\textasciitilde{}} \StringTok{"15{-}19"}\NormalTok{,}
\NormalTok{    Age }\SpecialCharTok{==} \StringTok{"20to24"} \SpecialCharTok{\textasciitilde{}} \StringTok{"20{-}24"}\NormalTok{,}
\NormalTok{    Age }\SpecialCharTok{==} \StringTok{"25to34"} \SpecialCharTok{\textasciitilde{}} \StringTok{"25{-}34"}\NormalTok{,}
\NormalTok{    Age }\SpecialCharTok{==} \StringTok{"35up"}   \SpecialCharTok{\textasciitilde{}} \StringTok{"35+"}\NormalTok{,}
    \ConstantTok{TRUE} \SpecialCharTok{\textasciitilde{}}\NormalTok{ Age}
\NormalTok{  )) }\SpecialCharTok{\%\textgreater{}\%}
  \FunctionTok{select}\NormalTok{(faixa, sexo, Count)}

\CommentTok{\# {-}{-}{-} Calcular totais por faixa e proporções condicionais P(Mulher | faixa) {-}{-}{-}}
\NormalTok{totais\_faixa }\OtherTok{\textless{}{-}}\NormalTok{ df\_long }\SpecialCharTok{\%\textgreater{}\%}
  \FunctionTok{group\_by}\NormalTok{(faixa) }\SpecialCharTok{\%\textgreater{}\%}
  \FunctionTok{summarise}\NormalTok{(}\AttributeTok{total\_faixa =} \FunctionTok{sum}\NormalTok{(Count), }\AttributeTok{.groups =} \StringTok{"drop"}\NormalTok{)}

\NormalTok{res }\OtherTok{\textless{}{-}}\NormalTok{ df\_long }\SpecialCharTok{\%\textgreater{}\%}
  \FunctionTok{left\_join}\NormalTok{(totais\_faixa, }\AttributeTok{by =} \StringTok{"faixa"}\NormalTok{) }\SpecialCharTok{\%\textgreater{}\%}
  \FunctionTok{mutate}\NormalTok{(}\AttributeTok{proporcao =}\NormalTok{ Count }\SpecialCharTok{/}\NormalTok{ total\_faixa) }\SpecialCharTok{\%\textgreater{}\%}
  \FunctionTok{arrange}\NormalTok{(faixa, }\FunctionTok{desc}\NormalTok{(sexo))}

\CommentTok{\# Exibir tabela resumida}
\FunctionTok{cat}\NormalTok{(}\StringTok{"}\SpecialCharTok{\textbackslash{}n}\StringTok{Distribuição por faixa e sexo (contagens e proporções):}\SpecialCharTok{\textbackslash{}n}\StringTok{"}\NormalTok{)}
\FunctionTok{print}\NormalTok{(res)}

\CommentTok{\# {-}{-}{-} Extrair números necessários para as duas faixas de interesse {-}{-}{-}}
\NormalTok{n\_20\_24 }\OtherTok{\textless{}{-}}\NormalTok{ res }\SpecialCharTok{\%\textgreater{}\%} \FunctionTok{filter}\NormalTok{(faixa }\SpecialCharTok{==} \StringTok{"20{-}24"}\NormalTok{, sexo }\SpecialCharTok{==} \StringTok{"Mulher"}\NormalTok{) }\SpecialCharTok{\%\textgreater{}\%} \FunctionTok{pull}\NormalTok{(Count)}
\NormalTok{N\_20\_24 }\OtherTok{\textless{}{-}}\NormalTok{ res }\SpecialCharTok{\%\textgreater{}\%} \FunctionTok{filter}\NormalTok{(faixa }\SpecialCharTok{==} \StringTok{"20{-}24"}\NormalTok{) }\SpecialCharTok{\%\textgreater{}\%} \FunctionTok{slice}\NormalTok{(}\DecValTok{1}\NormalTok{) }\SpecialCharTok{\%\textgreater{}\%} \FunctionTok{pull}\NormalTok{(total\_faixa)}
\NormalTok{n\_25\_34 }\OtherTok{\textless{}{-}}\NormalTok{ res }\SpecialCharTok{\%\textgreater{}\%} \FunctionTok{filter}\NormalTok{(faixa }\SpecialCharTok{==} \StringTok{"25{-}34"}\NormalTok{, sexo }\SpecialCharTok{==} \StringTok{"Mulher"}\NormalTok{) }\SpecialCharTok{\%\textgreater{}\%} \FunctionTok{pull}\NormalTok{(Count)}
\NormalTok{N\_25\_34 }\OtherTok{\textless{}{-}}\NormalTok{ res }\SpecialCharTok{\%\textgreater{}\%} \FunctionTok{filter}\NormalTok{(faixa }\SpecialCharTok{==} \StringTok{"25{-}34"}\NormalTok{) }\SpecialCharTok{\%\textgreater{}\%} \FunctionTok{slice}\NormalTok{(}\DecValTok{1}\NormalTok{) }\SpecialCharTok{\%\textgreater{}\%} \FunctionTok{pull}\NormalTok{(total\_faixa)}

\CommentTok{\# Segurança caso algum esteja NA}
\NormalTok{n\_20\_24 }\OtherTok{\textless{}{-}} \FunctionTok{ifelse}\NormalTok{(}\FunctionTok{length}\NormalTok{(n\_20\_24)}\SpecialCharTok{==}\DecValTok{0}\NormalTok{, }\DecValTok{0}\NormalTok{, n\_20\_24)}
\NormalTok{n\_25\_34 }\OtherTok{\textless{}{-}} \FunctionTok{ifelse}\NormalTok{(}\FunctionTok{length}\NormalTok{(n\_25\_34)}\SpecialCharTok{==}\DecValTok{0}\NormalTok{, }\DecValTok{0}\NormalTok{, n\_25\_34)}

\CommentTok{\# {-}{-}{-} Teste de duas proporções (unilateral)}
\CommentTok{\# H0: p20{-}24 = p25{-}34  vs  H1: p25{-}34 \textgreater{} p20{-}24}
\CommentTok{\# Usamos prop.test com x = c(x1, x2) e n = c(n1, n2); alternative = "less"}
\CommentTok{\# (colocamos p1 = 20{-}24, p2 = 25{-}34; alternative = "less" testa p1 \textless{} p2)}
\NormalTok{prop\_test }\OtherTok{\textless{}{-}} \FunctionTok{prop.test}\NormalTok{(}\AttributeTok{x =} \FunctionTok{c}\NormalTok{(n\_20\_24, n\_25\_34),}
                       \AttributeTok{n =} \FunctionTok{c}\NormalTok{(N\_20\_24, N\_25\_34),}
                       \AttributeTok{alternative =} \StringTok{"less"}\NormalTok{,}
                       \AttributeTok{correct =} \ConstantTok{FALSE}\NormalTok{)}

\FunctionTok{cat}\NormalTok{(}\StringTok{"}\SpecialCharTok{\textbackslash{}n}\StringTok{{-}{-}{-} Teste de duas proporções (unilateral H1: p25{-}34 \textgreater{} p20{-}24) {-}{-}{-}}\SpecialCharTok{\textbackslash{}n}\StringTok{"}\NormalTok{)}
\FunctionTok{print}\NormalTok{(prop\_test)}

\CommentTok{\# {-}{-}{-} IC95\% para cada proporção individual (para plotagem) {-}{-}{-}}
\NormalTok{ci\_20 }\OtherTok{\textless{}{-}} \FunctionTok{prop.test}\NormalTok{(n\_20\_24, N\_20\_24, }\AttributeTok{correct =} \ConstantTok{FALSE}\NormalTok{)}\SpecialCharTok{$}\NormalTok{conf.int}
\NormalTok{ci\_25 }\OtherTok{\textless{}{-}} \FunctionTok{prop.test}\NormalTok{(n\_25\_34, N\_25\_34, }\AttributeTok{correct =} \ConstantTok{FALSE}\NormalTok{)}\SpecialCharTok{$}\NormalTok{conf.int}

\NormalTok{plot\_df }\OtherTok{\textless{}{-}} \FunctionTok{data.frame}\NormalTok{(}
  \AttributeTok{faixa =} \FunctionTok{c}\NormalTok{(}\StringTok{"20{-}24"}\NormalTok{,}\StringTok{"25{-}34"}\NormalTok{),}
  \AttributeTok{proporcao =} \FunctionTok{c}\NormalTok{(n\_20\_24 }\SpecialCharTok{/}\NormalTok{ N\_20\_24, n\_25\_34 }\SpecialCharTok{/}\NormalTok{ N\_25\_34),}
  \AttributeTok{ci\_low =} \FunctionTok{c}\NormalTok{(ci\_20[}\DecValTok{1}\NormalTok{], ci\_25[}\DecValTok{1}\NormalTok{]),}
  \AttributeTok{ci\_upp =} \FunctionTok{c}\NormalTok{(ci\_20[}\DecValTok{2}\NormalTok{], ci\_25[}\DecValTok{2}\NormalTok{])}
\NormalTok{)}

\CommentTok{\# {-}{-}{-} Gráfico: proporções com IC95\% {-}{-}{-}}
\FunctionTok{ggplot}\NormalTok{(plot\_df, }\FunctionTok{aes}\NormalTok{(}\AttributeTok{x =}\NormalTok{ faixa, }\AttributeTok{y =}\NormalTok{ proporcao, }\AttributeTok{fill =}\NormalTok{ faixa)) }\SpecialCharTok{+}
  \FunctionTok{geom\_col}\NormalTok{(}\AttributeTok{width =} \FloatTok{0.5}\NormalTok{, }\AttributeTok{show.legend =} \ConstantTok{FALSE}\NormalTok{) }\SpecialCharTok{+}
  \FunctionTok{geom\_errorbar}\NormalTok{(}\FunctionTok{aes}\NormalTok{(}\AttributeTok{ymin =}\NormalTok{ ci\_low, }\AttributeTok{ymax =}\NormalTok{ ci\_upp), }\AttributeTok{width =} \FloatTok{0.12}\NormalTok{, }\AttributeTok{size =} \FloatTok{0.8}\NormalTok{) }\SpecialCharTok{+}
  \FunctionTok{geom\_text}\NormalTok{(}\FunctionTok{aes}\NormalTok{(}\AttributeTok{label =}\NormalTok{ scales}\SpecialCharTok{::}\FunctionTok{percent}\NormalTok{(proporcao, }\AttributeTok{accuracy =} \FloatTok{0.1}\NormalTok{)),}
            \AttributeTok{vjust =} \SpecialCharTok{{-}}\FloatTok{0.6}\NormalTok{, }\AttributeTok{size =} \FloatTok{3.5}\NormalTok{) }\SpecialCharTok{+}
  \FunctionTok{scale\_y\_continuous}\NormalTok{(}\AttributeTok{labels =} \FunctionTok{percent\_format}\NormalTok{(}\AttributeTok{accuracy =} \DecValTok{1}\NormalTok{), }\AttributeTok{limits =} \FunctionTok{c}\NormalTok{(}\DecValTok{0}\NormalTok{, }\DecValTok{1}\NormalTok{)) }\SpecialCharTok{+}
  \FunctionTok{labs}\NormalTok{(}\AttributeTok{title =} \StringTok{"Proporção de mulheres por faixa etária"}\NormalTok{,}
       \AttributeTok{subtitle =} \StringTok{"Comparação entre 20{-}24 e 25{-}34 anos"}\NormalTok{,}
       \AttributeTok{x =} \StringTok{"Faixa etária (anos)"}\NormalTok{,}
       \AttributeTok{y =} \StringTok{"Proporção de mulheres (IC95\%)"}\NormalTok{) }\SpecialCharTok{+}
  \FunctionTok{theme\_minimal}\NormalTok{()}

\CommentTok{\# {-}{-}{-} Interpretação simples automática {-}{-}{-}}
\NormalTok{pval }\OtherTok{\textless{}{-}}\NormalTok{ prop\_test}\SpecialCharTok{$}\NormalTok{p.value}
\FunctionTok{cat}\NormalTok{(}\StringTok{"}\SpecialCharTok{\textbackslash{}n}\StringTok{Interpretação (teste unilaeral):}\SpecialCharTok{\textbackslash{}n}\StringTok{"}\NormalTok{)}
\ControlFlowTok{if}\NormalTok{ (pval }\SpecialCharTok{\textless{}} \FloatTok{0.05}\NormalTok{) \{}
  \FunctionTok{cat}\NormalTok{(}\FunctionTok{sprintf}\NormalTok{(}\StringTok{"p = \%.4f \textless{} 0.05: evidência de que a proporção de mulheres é maior em 25{-}34 do que em 20{-}24.}\SpecialCharTok{\textbackslash{}n}\StringTok{"}\NormalTok{, pval))}
\NormalTok{\} }\ControlFlowTok{else}\NormalTok{ \{}
  \FunctionTok{cat}\NormalTok{(}\FunctionTok{sprintf}\NormalTok{(}\StringTok{"p = \%.4f \textgreater{}= 0.05: sem evidência suficiente de que a proporção de mulheres seja maior em 25{-}34.}\SpecialCharTok{\textbackslash{}n}\StringTok{"}\NormalTok{, pval))}
\NormalTok{\}}
\CommentTok{\# FIM}
\InformationTok{\textasciigrave{}\textasciigrave{}\textasciigrave{}}
\end{Highlighting}
\end{Shaded}

\begin{verbatim}

total das contagens (tamanho amostra) n:
[1] 18397

Distribuição por faixa e sexo (contagens e proporções):
  faixa   sexo Count total_faixa proporcao
1 15-19 Mulher  2348        4179      0.56
2 15-19  Homem  1831        4179      0.44
3 20-24 Mulher  4280        7993      0.54
4 20-24  Homem  3713        7993      0.46
5 25-34 Mulher  2166        3880      0.56
6 25-34  Homem  1714        3880      0.44
7   35+ Mulher  1492        2345      0.64
8   35+  Homem   853        2345      0.36

--- Teste de duas proporções (unilateral H1: p25-34 > p20-24) ---

    2-sample test for equality of proportions without continuity correction

data:  c(n_20_24, n_25_34) out of c(N_20_24, N_25_34)
X-squared = 5, df = 1, p-value = 0.01
alternative hypothesis: less
95 percent confidence interval:
 -1.0000 -0.0068
sample estimates:
prop 1 prop 2 
  0.54   0.56 


Interpretação (teste unilaeral):
p = 0.0097 < 0.05: evidência de que a proporção de mulheres é maior em 25-34 do que em 20-24.
\end{verbatim}

\pandocbounded{\includegraphics[keepaspectratio]{cap6-moore-tab-dupla-entrada_files/figure-pdf/unnamed-chunk-21-1.pdf}}

É preciso cuidado na interpretação dessas saídas, pois um grande tamanho
de amostra pode capturar pequenas diferenças entre as duas categorias,
que podem não ser \textbf{praticamente significativas}.

\section{Paradoxo de Simpson}\label{paradoxo-de-simpson}

Como no caso de variáveis quantitativas, os efeitos de
\ul{\textbf{variáveis ocultas}} podem mudar, ou mesmo inverter, relações
entre duas variáveis categóricas. Aqui está um exemplo que demonstra as
surpresas com as quais um usuário de dados menos avisado pode se
defrontar.

\begin{Shaded}
\begin{Highlighting}[numbers=left,,]
\InformationTok{\textasciigrave{}\textasciigrave{}\textasciigrave{}\{r\}}
\CommentTok{\# Exemplo do Paradoxo de Simpson: socorro por Helicóptero vs Ambulância}
\CommentTok{\# Em cada estrato (Leve / Grave) Helicóptero tem maior taxa de sobrevivência,}
\CommentTok{\# mas ao agregar os dados Ambulância apresenta taxa global maior.}
\CommentTok{\#}
\CommentTok{\# Dados hipotéticos construídos para ilustrar o paradoxo.}

\CommentTok{\# Pacotes necessários}
\ControlFlowTok{if}\NormalTok{ (}\SpecialCharTok{!}\FunctionTok{requireNamespace}\NormalTok{(}\StringTok{"dplyr"}\NormalTok{, }\AttributeTok{quietly =} \ConstantTok{TRUE}\NormalTok{)) }\FunctionTok{install.packages}\NormalTok{(}\StringTok{"dplyr"}\NormalTok{)}
\ControlFlowTok{if}\NormalTok{ (}\SpecialCharTok{!}\FunctionTok{requireNamespace}\NormalTok{(}\StringTok{"ggplot2"}\NormalTok{, }\AttributeTok{quietly =} \ConstantTok{TRUE}\NormalTok{)) }\FunctionTok{install.packages}\NormalTok{(}\StringTok{"ggplot2"}\NormalTok{)}
\ControlFlowTok{if}\NormalTok{ (}\SpecialCharTok{!}\FunctionTok{requireNamespace}\NormalTok{(}\StringTok{"scales"}\NormalTok{, }\AttributeTok{quietly =} \ConstantTok{TRUE}\NormalTok{)) }\FunctionTok{install.packages}\NormalTok{(}\StringTok{"scales"}\NormalTok{)}
\FunctionTok{library}\NormalTok{(dplyr)}
\FunctionTok{library}\NormalTok{(ggplot2)}
\FunctionTok{library}\NormalTok{(scales)}

\CommentTok{\# {-}{-}{-} Construção dos dados (contagens) {-}{-}{-}}
\CommentTok{\# Estratos: "Leve" e "Grave"}
\CommentTok{\# Notação: successes = sobreviventes, total = número de acidentados atendidos}
\CommentTok{\# Projeto intencional:}
\CommentTok{\# {-} Helicóptero tem taxas maiores em ambos os estratos}
\CommentTok{\# {-} Ambulância atende muito mais casos no estrato com alta taxa, invertendo o resultado agregado}

\NormalTok{df\_counts }\OtherTok{\textless{}{-}} \FunctionTok{data.frame}\NormalTok{(}
  \AttributeTok{estrato =} \FunctionTok{rep}\NormalTok{(}\FunctionTok{c}\NormalTok{(}\StringTok{"Leve"}\NormalTok{, }\StringTok{"Grave"}\NormalTok{), }\AttributeTok{each =} \DecValTok{2}\NormalTok{),}
  \AttributeTok{transporte =} \FunctionTok{rep}\NormalTok{(}\FunctionTok{c}\NormalTok{(}\StringTok{"Helicóptero"}\NormalTok{, }\StringTok{"Ambulância"}\NormalTok{), }\AttributeTok{times =} \DecValTok{2}\NormalTok{),}
  \AttributeTok{success =} \FunctionTok{c}\NormalTok{(}\DecValTok{9}\NormalTok{, }\DecValTok{72}\NormalTok{,    }\CommentTok{\# Leve: H = 9/10 (0.90), A = 72/90 (0.80)}
              \DecValTok{63}\NormalTok{, }\DecValTok{6}\NormalTok{),   }\CommentTok{\# Grave: H = 63/90 (0.70), A = 6/10  (0.60)}
  \AttributeTok{total =} \FunctionTok{c}\NormalTok{(}\DecValTok{10}\NormalTok{, }\DecValTok{90}\NormalTok{,     }\CommentTok{\# totais correspondentes}
            \DecValTok{90}\NormalTok{, }\DecValTok{10}\NormalTok{),}
  \AttributeTok{stringsAsFactors =} \ConstantTok{FALSE}
\NormalTok{)}

\CommentTok{\# Calcular taxas por célula}
\NormalTok{df\_counts }\OtherTok{\textless{}{-}}\NormalTok{ df\_counts }\SpecialCharTok{\%\textgreater{}\%}
  \FunctionTok{mutate}\NormalTok{(}\AttributeTok{rate =}\NormalTok{ success }\SpecialCharTok{/}\NormalTok{ total)}

\CommentTok{\# Mostrar tabela de contingência formatada}
\FunctionTok{cat}\NormalTok{(}\StringTok{"Tabela de contagens (cada linha = estrato x transporte):}\SpecialCharTok{\textbackslash{}n}\StringTok{"}\NormalTok{)}
\FunctionTok{print}\NormalTok{(df\_counts)}

\CommentTok{\# {-}{-}{-} Comparações por estrato (teste de proporção) {-}{-}{-}}
\FunctionTok{cat}\NormalTok{(}\StringTok{"}\SpecialCharTok{\textbackslash{}n}\StringTok{Testes por estrato (Helicóptero vs Ambulância):}\SpecialCharTok{\textbackslash{}n}\StringTok{"}\NormalTok{)}
\ControlFlowTok{for}\NormalTok{ (e }\ControlFlowTok{in} \FunctionTok{unique}\NormalTok{(df\_counts}\SpecialCharTok{$}\NormalTok{estrato)) \{}
\NormalTok{  sub }\OtherTok{\textless{}{-}}\NormalTok{ df\_counts }\SpecialCharTok{\%\textgreater{}\%} \FunctionTok{filter}\NormalTok{(estrato }\SpecialCharTok{==}\NormalTok{ e) }\SpecialCharTok{\%\textgreater{}\%} \FunctionTok{arrange}\NormalTok{(transporte)}
  \CommentTok{\# x: successes para os dois grupos; n: totals}
\NormalTok{  x }\OtherTok{\textless{}{-}}\NormalTok{ sub}\SpecialCharTok{$}\NormalTok{success}
\NormalTok{  n }\OtherTok{\textless{}{-}}\NormalTok{ sub}\SpecialCharTok{$}\NormalTok{total}
  \CommentTok{\# prop.test compara as duas proporções (2 grupos)}
\NormalTok{  tst }\OtherTok{\textless{}{-}} \FunctionTok{prop.test}\NormalTok{(}\AttributeTok{x =}\NormalTok{ x, }\AttributeTok{n =}\NormalTok{ n, }\AttributeTok{correct =} \ConstantTok{FALSE}\NormalTok{)}
  \FunctionTok{cat}\NormalTok{(}\FunctionTok{sprintf}\NormalTok{(}\StringTok{"}\SpecialCharTok{\textbackslash{}n}\StringTok{Estrato: \%s}\SpecialCharTok{\textbackslash{}n}\StringTok{"}\NormalTok{, e))}
  \FunctionTok{print}\NormalTok{(sub)}
  \FunctionTok{print}\NormalTok{(tst)}
\NormalTok{\}}

\CommentTok{\# {-}{-}{-} Agregado: somar successes e totals por transporte {-}{-}{-}}
\NormalTok{totais }\OtherTok{\textless{}{-}}\NormalTok{ df\_counts }\SpecialCharTok{\%\textgreater{}\%}
  \FunctionTok{group\_by}\NormalTok{(transporte) }\SpecialCharTok{\%\textgreater{}\%}
  \FunctionTok{summarise}\NormalTok{(}\AttributeTok{success =} \FunctionTok{sum}\NormalTok{(success), }\AttributeTok{total =} \FunctionTok{sum}\NormalTok{(total), }\AttributeTok{.groups =} \StringTok{"drop"}\NormalTok{) }\SpecialCharTok{\%\textgreater{}\%}
  \FunctionTok{mutate}\NormalTok{(}\AttributeTok{rate =}\NormalTok{ success }\SpecialCharTok{/}\NormalTok{ total)}

\FunctionTok{cat}\NormalTok{(}\StringTok{"}\SpecialCharTok{\textbackslash{}n}\StringTok{Totais agregados por transporte:}\SpecialCharTok{\textbackslash{}n}\StringTok{"}\NormalTok{)}
\FunctionTok{print}\NormalTok{(totais)}

\CommentTok{\# Teste agregado (Helicóptero vs Ambulância)}
\NormalTok{tst\_global }\OtherTok{\textless{}{-}} \FunctionTok{prop.test}\NormalTok{(}\AttributeTok{x =}\NormalTok{ totais}\SpecialCharTok{$}\NormalTok{success, }\AttributeTok{n =}\NormalTok{ totais}\SpecialCharTok{$}\NormalTok{total, }\AttributeTok{correct =} \ConstantTok{FALSE}\NormalTok{)}
\FunctionTok{cat}\NormalTok{(}\StringTok{"}\SpecialCharTok{\textbackslash{}n}\StringTok{Teste agregado (Helicóptero vs Ambulância):}\SpecialCharTok{\textbackslash{}n}\StringTok{"}\NormalTok{)}
\FunctionTok{print}\NormalTok{(tst\_global)}

\CommentTok{\# {-}{-}{-} Visualização: taxas por estrato e agregado {-}{-}{-}}
\CommentTok{\# Preparar dados para plot: taxas estrato{-}por{-}estrato + linha do agregado}
\NormalTok{plot\_df }\OtherTok{\textless{}{-}}\NormalTok{ df\_counts }\SpecialCharTok{\%\textgreater{}\%}
  \FunctionTok{mutate}\NormalTok{(}\AttributeTok{estrato =} \FunctionTok{factor}\NormalTok{(estrato, }\AttributeTok{levels =} \FunctionTok{c}\NormalTok{(}\StringTok{"Leve"}\NormalTok{, }\StringTok{"Grave"}\NormalTok{))) }\SpecialCharTok{\%\textgreater{}\%}
  \FunctionTok{select}\NormalTok{(estrato, transporte, rate, success, total)}

\CommentTok{\# Dados agregados para adicionar como painel "Agregado"}
\NormalTok{agregado\_panel }\OtherTok{\textless{}{-}}\NormalTok{ totais }\SpecialCharTok{\%\textgreater{}\%}
  \FunctionTok{mutate}\NormalTok{(}\AttributeTok{estrato =} \StringTok{"Agregado"}\NormalTok{) }\SpecialCharTok{\%\textgreater{}\%}
  \FunctionTok{select}\NormalTok{(estrato, transporte, rate, success, total)}

\NormalTok{plot\_df\_all }\OtherTok{\textless{}{-}} \FunctionTok{bind\_rows}\NormalTok{(plot\_df, agregado\_panel) }\SpecialCharTok{\%\textgreater{}\%}
  \FunctionTok{mutate}\NormalTok{(}\AttributeTok{estrato =} \FunctionTok{factor}\NormalTok{(estrato, }\AttributeTok{levels =} \FunctionTok{c}\NormalTok{(}\StringTok{"Leve"}\NormalTok{, }\StringTok{"Grave"}\NormalTok{, }\StringTok{"Agregado"}\NormalTok{)))}

\NormalTok{p }\OtherTok{\textless{}{-}} \FunctionTok{ggplot}\NormalTok{(plot\_df\_all, }\FunctionTok{aes}\NormalTok{(}\AttributeTok{x =}\NormalTok{ transporte, }\AttributeTok{y =}\NormalTok{ rate, }\AttributeTok{fill =}\NormalTok{ transporte)) }\SpecialCharTok{+}
  \FunctionTok{geom\_col}\NormalTok{(}\AttributeTok{position =} \StringTok{"dodge"}\NormalTok{, }\AttributeTok{width =} \FloatTok{0.6}\NormalTok{, }\AttributeTok{show.legend =} \ConstantTok{FALSE}\NormalTok{) }\SpecialCharTok{+}
  \FunctionTok{geom\_text}\NormalTok{(}\FunctionTok{aes}\NormalTok{(}\AttributeTok{label =} \FunctionTok{paste0}\NormalTok{(success, }\StringTok{"/"}\NormalTok{, total, }\StringTok{"}\SpecialCharTok{\textbackslash{}n}\StringTok{"}\NormalTok{, }\FunctionTok{percent}\NormalTok{(rate, }\AttributeTok{accuracy =} \FloatTok{0.1}\NormalTok{))),}
            \AttributeTok{position =} \FunctionTok{position\_dodge}\NormalTok{(}\AttributeTok{width =} \FloatTok{0.6}\NormalTok{), }\AttributeTok{vjust =} \SpecialCharTok{{-}}\FloatTok{0.5}\NormalTok{, }\AttributeTok{size =} \DecValTok{3}\NormalTok{) }\SpecialCharTok{+}
  \FunctionTok{facet\_wrap}\NormalTok{(}\SpecialCharTok{\textasciitilde{}}\NormalTok{ estrato, }\AttributeTok{nrow =} \DecValTok{1}\NormalTok{) }\SpecialCharTok{+}
  \FunctionTok{scale\_y\_continuous}\NormalTok{(}\AttributeTok{labels =} \FunctionTok{percent\_format}\NormalTok{(}\AttributeTok{accuracy =} \DecValTok{1}\NormalTok{), }\AttributeTok{limits =} \FunctionTok{c}\NormalTok{(}\DecValTok{0}\NormalTok{, }\DecValTok{1}\NormalTok{)) }\SpecialCharTok{+}
  \FunctionTok{scale\_fill\_manual}\NormalTok{(}\AttributeTok{values =} \FunctionTok{c}\NormalTok{(}\StringTok{"Helicóptero"} \OtherTok{=} \StringTok{"\#4E79A7"}\NormalTok{, }\StringTok{"Ambulância"} \OtherTok{=} \StringTok{"\#F28E2B"}\NormalTok{)) }\SpecialCharTok{+}
  \FunctionTok{labs}\NormalTok{(}\AttributeTok{title =} \StringTok{"Exemplo do Paradoxo de Simpson — Socorro por Helicóptero vs Ambulância"}\NormalTok{,}
       \AttributeTok{subtitle =} \StringTok{"Em cada estrato Helicóptero tem taxa maior; agregado favorece Ambulância"}\NormalTok{,}
       \AttributeTok{x =} \StringTok{""}\NormalTok{, }\AttributeTok{y =} \StringTok{"Taxa de sobrevivência"}\NormalTok{,}
       \AttributeTok{caption =} \StringTok{"Dados hipotéticos ilustrativos"}\NormalTok{) }\SpecialCharTok{+}
  \FunctionTok{theme\_minimal}\NormalTok{() }\SpecialCharTok{+}
  \FunctionTok{theme}\NormalTok{(}\AttributeTok{plot.title =} \FunctionTok{element\_text}\NormalTok{(}\AttributeTok{hjust =} \FloatTok{0.5}\NormalTok{))}

\FunctionTok{print}\NormalTok{(p)}

\CommentTok{\# {-}{-}{-} Interpretação curta {-}{-}{-}}
\FunctionTok{cat}\NormalTok{(}\StringTok{"}\SpecialCharTok{\textbackslash{}n}\StringTok{Interpretação:}\SpecialCharTok{\textbackslash{}n}\StringTok{"}\NormalTok{)}
\FunctionTok{cat}\NormalTok{(}\StringTok{"{-} Em cada estrato (Leve e Grave) a taxa de sobrevivência do Helicóptero é maior que a da Ambulância.}\SpecialCharTok{\textbackslash{}n}\StringTok{"}\NormalTok{)}
\FunctionTok{cat}\NormalTok{(}\StringTok{"{-} No entanto, devido à distribuição muito diferente do número de pacientes entre estratos para cada transporte}\SpecialCharTok{\textbackslash{}n}\StringTok{"}\NormalTok{)}
\FunctionTok{cat}\NormalTok{(}\StringTok{"  (ambulância atende muitos casos no estrato com alta taxa), a taxa global da Ambulância é maior.}\SpecialCharTok{\textbackslash{}n}\StringTok{"}\NormalTok{)}
\FunctionTok{cat}\NormalTok{(}\StringTok{"{-} Esse é o Paradoxo de Simpson: a tendência observada em cada estrato pode ser invertida ao agregar os dados.}\SpecialCharTok{\textbackslash{}n}\StringTok{"}\NormalTok{)}

\CommentTok{\# Fim}
\InformationTok{\textasciigrave{}\textasciigrave{}\textasciigrave{}}
\end{Highlighting}
\end{Shaded}

\begin{verbatim}
Tabela de contagens (cada linha = estrato x transporte):
  estrato  transporte success total rate
1    Leve Helicóptero       9    10  0.9
2    Leve  Ambulância      72    90  0.8
3   Grave Helicóptero      63    90  0.7
4   Grave  Ambulância       6    10  0.6

Testes por estrato (Helicóptero vs Ambulância):

Estrato: Leve
  estrato  transporte success total rate
1    Leve  Ambulância      72    90  0.8
2    Leve Helicóptero       9    10  0.9

    2-sample test for equality of proportions without continuity correction

data:  x out of n
X-squared = 0.6, df = 1, p-value = 0.4
alternative hypothesis: two.sided
95 percent confidence interval:
 -0.3  0.1
sample estimates:
prop 1 prop 2 
   0.8    0.9 


Estrato: Grave
  estrato  transporte success total rate
1   Grave  Ambulância       6    10  0.6
2   Grave Helicóptero      63    90  0.7

    2-sample test for equality of proportions without continuity correction

data:  x out of n
X-squared = 0.4, df = 1, p-value = 0.5
alternative hypothesis: two.sided
95 percent confidence interval:
 -0.42  0.22
sample estimates:
prop 1 prop 2 
   0.6    0.7 


Totais agregados por transporte:
# A tibble: 2 x 4
  transporte  success total  rate
  <chr>         <dbl> <dbl> <dbl>
1 Ambulância       78   100  0.78
2 Helicóptero      72   100  0.72

Teste agregado (Helicóptero vs Ambulância):

    2-sample test for equality of proportions without continuity correction

data:  totais$success out of totais$total
X-squared = 1, df = 1, p-value = 0.3
alternative hypothesis: two.sided
95 percent confidence interval:
 -0.06  0.18
sample estimates:
prop 1 prop 2 
  0.78   0.72 


Interpretação:
- Em cada estrato (Leve e Grave) a taxa de sobrevivência do Helicóptero é maior que a da Ambulância.
- No entanto, devido à distribuição muito diferente do número de pacientes entre estratos para cada transporte
  (ambulância atende muitos casos no estrato com alta taxa), a taxa global da Ambulância é maior.
- Esse é o Paradoxo de Simpson: a tendência observada em cada estrato pode ser invertida ao agregar os dados.
\end{verbatim}

\pandocbounded{\includegraphics[keepaspectratio]{cap6-moore-tab-dupla-entrada_files/figure-pdf/unnamed-chunk-22-1.pdf}}

\section{Verifique suas habilidades}\label{verifique-suas-habilidades}

Redes sociais

\begin{Shaded}
\begin{Highlighting}[numbers=left,,]
\InformationTok{\textasciigrave{}\textasciigrave{}\textasciigrave{}\{r\}}
\CommentTok{\# Script: calculo de distribuições marginal e condicional}
\CommentTok{\# Como usar: execute no R ou RStudio: source("scripts/distribuicoes\_marginais\_condicionais.R")}

\CommentTok{\# 1) Monta a tabela de contingência com os valores fornecidos}
\NormalTok{idades }\OtherTok{\textless{}{-}} \FunctionTok{c}\NormalTok{(}\StringTok{"18{-}29"}\NormalTok{, }\StringTok{"30{-}49"}\NormalTok{, }\StringTok{"50{-}64"}\NormalTok{, }\StringTok{"65+"}\NormalTok{)}
\NormalTok{respostas }\OtherTok{\textless{}{-}} \FunctionTok{c}\NormalTok{(}\StringTok{"Sim"}\NormalTok{, }\StringTok{"Não"}\NormalTok{)}

\CommentTok{\# Os valores estão em ordem por linha: (Sim, Não) para cada faixa etária}
\NormalTok{counts }\OtherTok{\textless{}{-}} \FunctionTok{matrix}\NormalTok{(}
  \FunctionTok{c}\NormalTok{(}
    \DecValTok{212}\NormalTok{, }\DecValTok{24}\NormalTok{,   }\CommentTok{\# Idade 18{-}29}
    \DecValTok{324}\NormalTok{, }\DecValTok{71}\NormalTok{,   }\CommentTok{\# Idade 30{-}49}
    \DecValTok{293}\NormalTok{, }\DecValTok{131}\NormalTok{,  }\CommentTok{\# Idade 50{-}64}
    \DecValTok{156}\NormalTok{, }\DecValTok{235}   \CommentTok{\# Idade 65+}
\NormalTok{  ),}
  \AttributeTok{nrow =} \FunctionTok{length}\NormalTok{(idades),}
  \AttributeTok{byrow =} \ConstantTok{TRUE}
\NormalTok{)}

\FunctionTok{rownames}\NormalTok{(counts) }\OtherTok{\textless{}{-}}\NormalTok{ idades}
\FunctionTok{colnames}\NormalTok{(counts) }\OtherTok{\textless{}{-}}\NormalTok{ respostas}

\CommentTok{\# Mostra a tabela bruta}
\FunctionTok{cat}\NormalTok{(}\StringTok{"Tabela de contingência (frequências absolutas):}\SpecialCharTok{\textbackslash{}n}\StringTok{"}\NormalTok{)}
\FunctionTok{print}\NormalTok{(counts)}
\FunctionTok{cat}\NormalTok{(}\StringTok{"}\SpecialCharTok{\textbackslash{}n}\StringTok{"}\NormalTok{)}

\CommentTok{\# Total geral}
\NormalTok{total }\OtherTok{\textless{}{-}} \FunctionTok{sum}\NormalTok{(counts)}
\FunctionTok{cat}\NormalTok{(}\StringTok{"Total geral:"}\NormalTok{, total, }\StringTok{"}\SpecialCharTok{\textbackslash{}n\textbackslash{}n}\StringTok{"}\NormalTok{)}

\CommentTok{\# 2) Distribuições marginais}
\CommentTok{\# {-} Marginal por idade (soma por linha)}
\NormalTok{marginal\_idade }\OtherTok{\textless{}{-}} \FunctionTok{rowSums}\NormalTok{(counts)}
\CommentTok{\# {-} Marginal por resposta (soma por coluna)}
\NormalTok{marginal\_resposta }\OtherTok{\textless{}{-}} \FunctionTok{colSums}\NormalTok{(counts)}

\FunctionTok{cat}\NormalTok{(}\StringTok{"Distribuição marginal por idade (frequências):}\SpecialCharTok{\textbackslash{}n}\StringTok{"}\NormalTok{)}
\FunctionTok{print}\NormalTok{(marginal\_idade)}
\FunctionTok{cat}\NormalTok{(}\StringTok{"}\SpecialCharTok{\textbackslash{}n}\StringTok{Distribuição marginal por resposta (frequências):}\SpecialCharTok{\textbackslash{}n}\StringTok{"}\NormalTok{)}
\FunctionTok{print}\NormalTok{(marginal\_resposta)}
\FunctionTok{cat}\NormalTok{(}\StringTok{"}\SpecialCharTok{\textbackslash{}n}\StringTok{"}\NormalTok{)}

\CommentTok{\# Também em proporções (frações do total) e em percentuais}
\NormalTok{prop\_marginal\_idade }\OtherTok{\textless{}{-}}\NormalTok{ marginal\_idade }\SpecialCharTok{/}\NormalTok{ total}
\NormalTok{prop\_marginal\_resposta }\OtherTok{\textless{}{-}}\NormalTok{ marginal\_resposta }\SpecialCharTok{/}\NormalTok{ total}

\FunctionTok{cat}\NormalTok{(}\StringTok{"Distribuição marginal por idade (percentual):}\SpecialCharTok{\textbackslash{}n}\StringTok{"}\NormalTok{)}
\FunctionTok{print}\NormalTok{(}\FunctionTok{round}\NormalTok{(prop\_marginal\_idade }\SpecialCharTok{*} \DecValTok{100}\NormalTok{, }\DecValTok{2}\NormalTok{))}
\FunctionTok{cat}\NormalTok{(}\StringTok{"}\SpecialCharTok{\textbackslash{}n}\StringTok{Distribuição marginal por resposta (percentual):}\SpecialCharTok{\textbackslash{}n}\StringTok{"}\NormalTok{)}
\FunctionTok{print}\NormalTok{(}\FunctionTok{round}\NormalTok{(prop\_marginal\_resposta }\SpecialCharTok{*} \DecValTok{100}\NormalTok{, }\DecValTok{2}\NormalTok{))}
\FunctionTok{cat}\NormalTok{(}\StringTok{"}\SpecialCharTok{\textbackslash{}n}\StringTok{"}\NormalTok{)}

\CommentTok{\# 3) Distribuição conjunta em proporções (tabela de proporções)}
\NormalTok{prop\_conjunta }\OtherTok{\textless{}{-}} \FunctionTok{prop.table}\NormalTok{(counts)  }\CommentTok{\# por padrão divide por total}
\FunctionTok{cat}\NormalTok{(}\StringTok{"Tabela conjunta (proporções):}\SpecialCharTok{\textbackslash{}n}\StringTok{"}\NormalTok{)}
\FunctionTok{print}\NormalTok{(}\FunctionTok{round}\NormalTok{(prop\_conjunta, }\DecValTok{4}\NormalTok{))}
\FunctionTok{cat}\NormalTok{(}\StringTok{"}\SpecialCharTok{\textbackslash{}n}\StringTok{Tabela conjunta (percentual):}\SpecialCharTok{\textbackslash{}n}\StringTok{"}\NormalTok{)}
\FunctionTok{print}\NormalTok{(}\FunctionTok{round}\NormalTok{(prop\_conjunta }\SpecialCharTok{*} \DecValTok{100}\NormalTok{, }\DecValTok{2}\NormalTok{))}
\FunctionTok{cat}\NormalTok{(}\StringTok{"}\SpecialCharTok{\textbackslash{}n}\StringTok{"}\NormalTok{)}

\CommentTok{\# 4) Distribuições condicionais}
\CommentTok{\# a) Condicional por linha: P(Resposta | Idade)  =\textgreater{} prop.table com margin = 1}
\NormalTok{condicional\_por\_idade }\OtherTok{\textless{}{-}} \FunctionTok{prop.table}\NormalTok{(counts, }\AttributeTok{margin =} \DecValTok{1}\NormalTok{)}
\FunctionTok{cat}\NormalTok{(}\StringTok{"P(Resposta | Idade) — probabilidade de \textquotesingle{}Sim\textquotesingle{} ou \textquotesingle{}Não\textquotesingle{} dado a faixa etária (linhas somam 1):}\SpecialCharTok{\textbackslash{}n}\StringTok{"}\NormalTok{)}
\FunctionTok{print}\NormalTok{(}\FunctionTok{round}\NormalTok{(condicional\_por\_idade, }\DecValTok{4}\NormalTok{))}
\FunctionTok{cat}\NormalTok{(}\StringTok{"}\SpecialCharTok{\textbackslash{}n}\StringTok{Em percentuais:}\SpecialCharTok{\textbackslash{}n}\StringTok{"}\NormalTok{)}
\FunctionTok{print}\NormalTok{(}\FunctionTok{round}\NormalTok{(condicional\_por\_idade }\SpecialCharTok{*} \DecValTok{100}\NormalTok{, }\DecValTok{2}\NormalTok{))}
\FunctionTok{cat}\NormalTok{(}\StringTok{"}\SpecialCharTok{\textbackslash{}n}\StringTok{"}\NormalTok{)}

\CommentTok{\# b) Condicional por coluna: P(Idade | Resposta) =\textgreater{} prop.table com margin = 2}
\NormalTok{condicional\_por\_resposta }\OtherTok{\textless{}{-}} \FunctionTok{prop.table}\NormalTok{(counts, }\AttributeTok{margin =} \DecValTok{2}\NormalTok{)}
\FunctionTok{cat}\NormalTok{(}\StringTok{"P(Idade | Resposta) — distribuição etária entre quem disse \textquotesingle{}Sim\textquotesingle{} ou \textquotesingle{}Não\textquotesingle{} (colunas somam 1):}\SpecialCharTok{\textbackslash{}n}\StringTok{"}\NormalTok{)}
\FunctionTok{print}\NormalTok{(}\FunctionTok{round}\NormalTok{(condicional\_por\_resposta, }\DecValTok{4}\NormalTok{))}
\FunctionTok{cat}\NormalTok{(}\StringTok{"}\SpecialCharTok{\textbackslash{}n}\StringTok{Em percentuais:}\SpecialCharTok{\textbackslash{}n}\StringTok{"}\NormalTok{)}
\FunctionTok{print}\NormalTok{(}\FunctionTok{round}\NormalTok{(condicional\_por\_resposta }\SpecialCharTok{*} \DecValTok{100}\NormalTok{, }\DecValTok{2}\NormalTok{))}
\FunctionTok{cat}\NormalTok{(}\StringTok{"}\SpecialCharTok{\textbackslash{}n}\StringTok{"}\NormalTok{)}

\CommentTok{\# 5) Exemplos de acesso aos valores:}
\CommentTok{\# P(Sim | 18{-}29)}
\NormalTok{p\_sim\_18\_29 }\OtherTok{\textless{}{-}}\NormalTok{ condicional\_por\_idade[}\StringTok{"18{-}29"}\NormalTok{, }\StringTok{"Sim"}\NormalTok{]}
\FunctionTok{cat}\NormalTok{(}\StringTok{"P(Sim | 18{-}29) ="}\NormalTok{, }\FunctionTok{round}\NormalTok{(p\_sim\_18\_29, }\DecValTok{4}\NormalTok{), }\StringTok{"("}\NormalTok{, }\FunctionTok{round}\NormalTok{(p\_sim\_18\_29}\SpecialCharTok{*}\DecValTok{100}\NormalTok{,}\DecValTok{2}\NormalTok{), }\StringTok{"\% )}\SpecialCharTok{\textbackslash{}n}\StringTok{"}\NormalTok{)}

\CommentTok{\# P(18{-}29 | Sim)}
\NormalTok{p\_18\_29\_sim }\OtherTok{\textless{}{-}}\NormalTok{ condicional\_por\_resposta[}\StringTok{"18{-}29"}\NormalTok{, }\StringTok{"Sim"}\NormalTok{]}
\FunctionTok{cat}\NormalTok{(}\StringTok{"P(18{-}29 | Sim) ="}\NormalTok{, }\FunctionTok{round}\NormalTok{(p\_18\_29\_sim, }\DecValTok{4}\NormalTok{), }\StringTok{"("}\NormalTok{, }\FunctionTok{round}\NormalTok{(p\_18\_29\_sim}\SpecialCharTok{*}\DecValTok{100}\NormalTok{,}\DecValTok{2}\NormalTok{), }\StringTok{"\% )}\SpecialCharTok{\textbackslash{}n}\StringTok{"}\NormalTok{)}

\CommentTok{\# Fim do script}
\InformationTok{\textasciigrave{}\textasciigrave{}\textasciigrave{}}
\end{Highlighting}
\end{Shaded}

\begin{verbatim}
Tabela de contingência (frequências absolutas):
      Sim Não
18-29 212  24
30-49 324  71
50-64 293 131
65+   156 235

Total geral: 1446 

Distribuição marginal por idade (frequências):
18-29 30-49 50-64   65+ 
  236   395   424   391 

Distribuição marginal por resposta (frequências):
Sim Não 
985 461 

Distribuição marginal por idade (percentual):
18-29 30-49 50-64   65+ 
   16    27    29    27 

Distribuição marginal por resposta (percentual):
Sim Não 
 68  32 

Tabela conjunta (proporções):
       Sim   Não
18-29 0.15 0.017
30-49 0.22 0.049
50-64 0.20 0.091
65+   0.11 0.163

Tabela conjunta (percentual):
      Sim  Não
18-29  15  1.7
30-49  22  4.9
50-64  20  9.1
65+    11 16.2

P(Resposta | Idade) — probabilidade de 'Sim' ou 'Não' dado a faixa etária (linhas somam 1):
       Sim  Não
18-29 0.90 0.10
30-49 0.82 0.18
50-64 0.69 0.31
65+   0.40 0.60

Em percentuais:
      Sim Não
18-29  90  10
30-49  82  18
50-64  69  31
65+    40  60

P(Idade | Resposta) — distribuição etária entre quem disse 'Sim' ou 'Não' (colunas somam 1):
       Sim   Não
18-29 0.22 0.052
30-49 0.33 0.154
50-64 0.30 0.284
65+   0.16 0.510

Em percentuais:
      Sim  Não
18-29  22  5.2
30-49  33 15.4
50-64  30 28.4
65+    16 51.0

P(Sim | 18-29) = 0.9 ( 90 % )
P(18-29 | Sim) = 0.22 ( 22 % )
\end{verbatim}

Gráficos

\begin{Shaded}
\begin{Highlighting}[numbers=left,,]
\InformationTok{\textasciigrave{}\textasciigrave{}\textasciigrave{}\{r\}}
\CommentTok{\# Script: gráficos de barras empilhadas (contagem e 100\% empilhado)}
\CommentTok{\# Como usar: source("scripts/graficos\_barras\_empilhadas.R") ou execute no RStudio}
\CommentTok{\# Saídas: exibe os gráficos e salva arquivos PNG no diretório de trabalho}

\CommentTok{\# 0) Instala/carrega pacotes necessários (instala apenas se não existir)}
\NormalTok{required\_pkgs }\OtherTok{\textless{}{-}} \FunctionTok{c}\NormalTok{(}\StringTok{"ggplot2"}\NormalTok{, }\StringTok{"dplyr"}\NormalTok{, }\StringTok{"scales"}\NormalTok{, }\StringTok{"RColorBrewer"}\NormalTok{)}
\NormalTok{installed }\OtherTok{\textless{}{-}} \FunctionTok{rownames}\NormalTok{(}\FunctionTok{installed.packages}\NormalTok{())}
\ControlFlowTok{for}\NormalTok{ (pkg }\ControlFlowTok{in}\NormalTok{ required\_pkgs) \{}
  \ControlFlowTok{if}\NormalTok{ (}\SpecialCharTok{!}\NormalTok{pkg }\SpecialCharTok{\%in\%}\NormalTok{ installed) }\FunctionTok{install.packages}\NormalTok{(pkg, }\AttributeTok{dependencies =} \ConstantTok{TRUE}\NormalTok{)}
  \FunctionTok{library}\NormalTok{(pkg, }\AttributeTok{character.only =} \ConstantTok{TRUE}\NormalTok{)}
\NormalTok{\}}

\CommentTok{\# 1) Monta o data.frame com os dados fornecidos}
\NormalTok{idades }\OtherTok{\textless{}{-}} \FunctionTok{c}\NormalTok{(}\StringTok{"18{-}29"}\NormalTok{, }\StringTok{"30{-}49"}\NormalTok{, }\StringTok{"50{-}64"}\NormalTok{, }\StringTok{"65+"}\NormalTok{)}
\NormalTok{respostas }\OtherTok{\textless{}{-}} \FunctionTok{c}\NormalTok{(}\StringTok{"Sim"}\NormalTok{, }\StringTok{"Não"}\NormalTok{)}
\CommentTok{\# Orden: para cada faixa etária {-}\textgreater{} (Sim, Não)}
\NormalTok{counts }\OtherTok{\textless{}{-}} \FunctionTok{c}\NormalTok{(}
  \DecValTok{212}\NormalTok{, }\DecValTok{24}\NormalTok{,    }\CommentTok{\# 18{-}29}
  \DecValTok{324}\NormalTok{, }\DecValTok{71}\NormalTok{,    }\CommentTok{\# 30{-}49}
  \DecValTok{293}\NormalTok{, }\DecValTok{131}\NormalTok{,   }\CommentTok{\# 50{-}64}
  \DecValTok{156}\NormalTok{, }\DecValTok{235}    \CommentTok{\# 65+}
\NormalTok{)}

\NormalTok{df }\OtherTok{\textless{}{-}} \FunctionTok{data.frame}\NormalTok{(}
  \AttributeTok{Idade =} \FunctionTok{factor}\NormalTok{(}\FunctionTok{rep}\NormalTok{(idades, }\AttributeTok{each =} \DecValTok{2}\NormalTok{), }\AttributeTok{levels =}\NormalTok{ idades),}
  \AttributeTok{Resposta =} \FunctionTok{factor}\NormalTok{(}\FunctionTok{rep}\NormalTok{(respostas, }\AttributeTok{times =} \FunctionTok{length}\NormalTok{(idades)), }\AttributeTok{levels =}\NormalTok{ respostas),}
  \AttributeTok{Count =}\NormalTok{ counts}
\NormalTok{)}

\CommentTok{\# 2) Gráfico 1: barras empilhadas (contagens absolutas)}
\NormalTok{p\_contagem }\OtherTok{\textless{}{-}} \FunctionTok{ggplot}\NormalTok{(df, }\FunctionTok{aes}\NormalTok{(}\AttributeTok{x =}\NormalTok{ Idade, }\AttributeTok{y =}\NormalTok{ Count, }\AttributeTok{fill =}\NormalTok{ Resposta)) }\SpecialCharTok{+}
  \FunctionTok{geom\_bar}\NormalTok{(}\AttributeTok{stat =} \StringTok{"identity"}\NormalTok{, }\AttributeTok{colour =} \StringTok{"black"}\NormalTok{, }\AttributeTok{width =} \FloatTok{0.7}\NormalTok{) }\SpecialCharTok{+}
  \FunctionTok{geom\_text}\NormalTok{(}\FunctionTok{aes}\NormalTok{(}\AttributeTok{label =}\NormalTok{ Count), }\AttributeTok{position =} \FunctionTok{position\_stack}\NormalTok{(}\AttributeTok{vjust =} \FloatTok{0.5}\NormalTok{), }\AttributeTok{size =} \DecValTok{3}\NormalTok{, }\AttributeTok{colour =} \StringTok{"white"}\NormalTok{) }\SpecialCharTok{+}
  \FunctionTok{scale\_fill\_brewer}\NormalTok{(}\AttributeTok{palette =} \StringTok{"Set2"}\NormalTok{) }\SpecialCharTok{+}
  \FunctionTok{labs}\NormalTok{(}\AttributeTok{title =} \StringTok{"Respostas por faixa etária (contagem)"}\NormalTok{,}
       \AttributeTok{x =} \StringTok{"Faixa etária"}\NormalTok{,}
       \AttributeTok{y =} \StringTok{"Contagem"}\NormalTok{,}
       \AttributeTok{fill =} \StringTok{"Resposta"}\NormalTok{) }\SpecialCharTok{+}
  \FunctionTok{theme\_minimal}\NormalTok{(}\AttributeTok{base\_size =} \DecValTok{12}\NormalTok{)}

\CommentTok{\# Exibe e salva}
\FunctionTok{print}\NormalTok{(p\_contagem)}
\FunctionTok{ggsave}\NormalTok{(}\AttributeTok{filename =} \StringTok{"barras\_empilhadas\_contagem.png"}\NormalTok{, }\AttributeTok{plot =}\NormalTok{ p\_contagem, }\AttributeTok{width =} \DecValTok{8}\NormalTok{, }\AttributeTok{height =} \DecValTok{5}\NormalTok{, }\AttributeTok{dpi =} \DecValTok{300}\NormalTok{)}

\CommentTok{\# 3) Preparação para gráfico 100\% empilhado (proporções dentro de cada idade)}
\NormalTok{df\_prop }\OtherTok{\textless{}{-}}\NormalTok{ df }\SpecialCharTok{\%\textgreater{}\%}
  \FunctionTok{group\_by}\NormalTok{(Idade) }\SpecialCharTok{\%\textgreater{}\%}
  \FunctionTok{mutate}\NormalTok{(}\AttributeTok{Prop =}\NormalTok{ Count }\SpecialCharTok{/} \FunctionTok{sum}\NormalTok{(Count)) }\SpecialCharTok{\%\textgreater{}\%}
  \FunctionTok{ungroup}\NormalTok{()}

\CommentTok{\# 4) Gráfico 2: barras empilhadas em porcentagem (cada barra soma 100\%)}
\NormalTok{p\_percent }\OtherTok{\textless{}{-}} \FunctionTok{ggplot}\NormalTok{(df\_prop, }\FunctionTok{aes}\NormalTok{(}\AttributeTok{x =}\NormalTok{ Idade, }\AttributeTok{y =}\NormalTok{ Count, }\AttributeTok{fill =}\NormalTok{ Resposta)) }\SpecialCharTok{+}
  \FunctionTok{geom\_bar}\NormalTok{(}\AttributeTok{stat =} \StringTok{"identity"}\NormalTok{, }\AttributeTok{position =} \StringTok{"fill"}\NormalTok{, }\AttributeTok{colour =} \StringTok{"black"}\NormalTok{, }\AttributeTok{width =} \FloatTok{0.7}\NormalTok{) }\SpecialCharTok{+}
  \CommentTok{\# rótulos com porcentagem dentro das fatias}
  \FunctionTok{geom\_text}\NormalTok{(}\FunctionTok{aes}\NormalTok{(}\AttributeTok{label =} \FunctionTok{ifelse}\NormalTok{(Prop }\SpecialCharTok{\textgreater{}=} \FloatTok{0.03}\NormalTok{, scales}\SpecialCharTok{::}\FunctionTok{percent}\NormalTok{(Prop, }\AttributeTok{accuracy =} \FloatTok{0.1}\NormalTok{), }\StringTok{""}\NormalTok{)),}
            \AttributeTok{position =} \FunctionTok{position\_fill}\NormalTok{(}\AttributeTok{vjust =} \FloatTok{0.5}\NormalTok{), }\AttributeTok{size =} \DecValTok{3}\NormalTok{, }\AttributeTok{colour =} \StringTok{"white"}\NormalTok{) }\SpecialCharTok{+}
  \FunctionTok{scale\_y\_continuous}\NormalTok{(}\AttributeTok{labels =}\NormalTok{ scales}\SpecialCharTok{::}\FunctionTok{percent\_format}\NormalTok{()) }\SpecialCharTok{+}
  \FunctionTok{scale\_fill\_brewer}\NormalTok{(}\AttributeTok{palette =} \StringTok{"Set2"}\NormalTok{) }\SpecialCharTok{+}
  \FunctionTok{labs}\NormalTok{(}\AttributeTok{title =} \StringTok{"Respostas por faixa etária (100\% empilhado)"}\NormalTok{,}
       \AttributeTok{x =} \StringTok{"Faixa etária"}\NormalTok{,}
       \AttributeTok{y =} \StringTok{"Percentual"}\NormalTok{,}
       \AttributeTok{fill =} \StringTok{"Resposta"}\NormalTok{) }\SpecialCharTok{+}
  \FunctionTok{theme\_minimal}\NormalTok{(}\AttributeTok{base\_size =} \DecValTok{12}\NormalTok{)}

\CommentTok{\# Exibe e salva}
\FunctionTok{print}\NormalTok{(p\_percent)}
\FunctionTok{ggsave}\NormalTok{(}\AttributeTok{filename =} \StringTok{"barras\_empilhadas\_percentual.png"}\NormalTok{, }\AttributeTok{plot =}\NormalTok{ p\_percent, }\AttributeTok{width =} \DecValTok{8}\NormalTok{, }\AttributeTok{height =} \DecValTok{5}\NormalTok{, }\AttributeTok{dpi =} \DecValTok{300}\NormalTok{)}

\CommentTok{\# 5) Mensagem final}
\FunctionTok{message}\NormalTok{(}\StringTok{"Gráficos gerados: \textquotesingle{}barras\_empilhadas\_contagem.png\textquotesingle{} e \textquotesingle{}barras\_empilhadas\_percentual.png\textquotesingle{} no diretório de trabalho."}\NormalTok{)}
\InformationTok{\textasciigrave{}\textasciigrave{}\textasciigrave{}}
\end{Highlighting}
\end{Shaded}

\pandocbounded{\includegraphics[keepaspectratio]{cap6-moore-tab-dupla-entrada_files/figure-pdf/unnamed-chunk-24-1.pdf}}

\pandocbounded{\includegraphics[keepaspectratio]{cap6-moore-tab-dupla-entrada_files/figure-pdf/unnamed-chunk-24-2.pdf}}

\section{}\label{section}

\bookmarksetup{startatroot}

\chapter{AED - suco uva - Tabelas e
Gráficos}\label{sec-suco-uva-tab-graf}

\section{Objetivos da Aprendizagem}\label{objetivos-da-aprendizagem-4}

\begin{quote}
Após ler este relatório de pesquisa, você deve ser capaz de:

▶ 1 Calcular e interpretar \textbf{\emph{distribuições marginais}} em
tabelas de dupla entrada.

▶ 2 Calcular e interpretar \textbf{\emph{distribuições condicionais}} em
tabelas de dupla entrada.

▶ 3 Reconhecer e explicar o \textbf{\emph{paradoxo de Simpson}}. (MOORE;
NOTZ; FLIGNER, 2023 , p.~130).

▶ 4 Resumir dados em tabelas de dupla entrada e interpretá-las.

▶ 5 Visualizar gráficos adequados para esse tipo de dado e
interpretá-los.

▶ 6 Capturar modelos adequados e interpretá-los.
\end{quote}

Os dados dessa pesquisa empírica em Direito e Políticas Públicas foram
estraídos de (TOMÁS, 2020).

\section{Suco de uva}\label{suco-de-uva}

\subsection{Replicar tabela mês a
mês}\label{replicar-tabela-muxeas-a-muxeas}

Grupo de Controle (GC) -x- Grupo de Taratamento (GT)

\begin{Shaded}
\begin{Highlighting}[numbers=left,,]
\InformationTok{\textasciigrave{}\textasciigrave{}\textasciigrave{}\{r\}}
\CommentTok{\# Gera tabela mensal com contagens por grupo (Controle/Experimental), calcula qui{-}quadrado 2x2 por mês}
\CommentTok{\# e acrescenta notação de significância.}
\CommentTok{\#}
\CommentTok{\# Corrigido: evita avaliação lógica com vetor (\textgreater{}1) usando inherits(..., "htest").}

\CommentTok{\# Pacotes (instala se necessário)}
\ControlFlowTok{if}\NormalTok{ (}\SpecialCharTok{!}\FunctionTok{requireNamespace}\NormalTok{(}\StringTok{"dplyr"}\NormalTok{, }\AttributeTok{quietly =} \ConstantTok{TRUE}\NormalTok{)) }\FunctionTok{install.packages}\NormalTok{(}\StringTok{"dplyr"}\NormalTok{)}
\ControlFlowTok{if}\NormalTok{ (}\SpecialCharTok{!}\FunctionTok{requireNamespace}\NormalTok{(}\StringTok{"knitr"}\NormalTok{, }\AttributeTok{quietly =} \ConstantTok{TRUE}\NormalTok{)) }\FunctionTok{install.packages}\NormalTok{(}\StringTok{"knitr"}\NormalTok{)}
\FunctionTok{library}\NormalTok{(dplyr)}
\FunctionTok{library}\NormalTok{(knitr)}

\CommentTok{\# Dados (criados com data.frame base)}
\NormalTok{dados }\OtherTok{\textless{}{-}} \FunctionTok{data.frame}\NormalTok{(}
  \AttributeTok{Mes =} \FunctionTok{c}\NormalTok{(}\StringTok{"Abril"}\NormalTok{,}\StringTok{"Abril"}\NormalTok{,}\StringTok{"Maio"}\NormalTok{,}\StringTok{"Maio"}\NormalTok{,}\StringTok{"Junho"}\NormalTok{,}\StringTok{"Junho"}\NormalTok{,}\StringTok{"Julho"}\NormalTok{,}\StringTok{"Julho"}\NormalTok{,}
          \StringTok{"Agosto"}\NormalTok{,}\StringTok{"Agosto"}\NormalTok{,}\StringTok{"Setembro"}\NormalTok{,}\StringTok{"Setembro"}\NormalTok{,}\StringTok{"Outubro"}\NormalTok{,}\StringTok{"Outubro"}\NormalTok{,}
          \StringTok{"Novembro"}\NormalTok{,}\StringTok{"Novembro"}\NormalTok{,}\StringTok{"Dezembro"}\NormalTok{,}\StringTok{"Dezembro"}\NormalTok{),}
  \AttributeTok{Grupo =} \FunctionTok{c}\NormalTok{(}\StringTok{"Controle"}\NormalTok{,}\StringTok{"Experimental"}\NormalTok{,}\StringTok{"Controle"}\NormalTok{,}\StringTok{"Experimental"}\NormalTok{,}\StringTok{"Controle"}\NormalTok{,}\StringTok{"Experimental"}\NormalTok{,}
            \StringTok{"Controle"}\NormalTok{,}\StringTok{"Experimental"}\NormalTok{,}\StringTok{"Controle"}\NormalTok{,}\StringTok{"Experimental"}\NormalTok{,}\StringTok{"Controle"}\NormalTok{,}\StringTok{"Experimental"}\NormalTok{,}
            \StringTok{"Controle"}\NormalTok{,}\StringTok{"Experimental"}\NormalTok{,}\StringTok{"Controle"}\NormalTok{,}\StringTok{"Experimental"}\NormalTok{,}\StringTok{"Controle"}\NormalTok{,}\StringTok{"Experimental"}\NormalTok{),}
  \AttributeTok{Nao =} \FunctionTok{c}\NormalTok{(}\DecValTok{35}\NormalTok{,}\DecValTok{20}\NormalTok{,}\DecValTok{17}\NormalTok{, }\DecValTok{7}\NormalTok{,}\DecValTok{18}\NormalTok{,}\DecValTok{17}\NormalTok{,}\DecValTok{20}\NormalTok{, }\DecValTok{4}\NormalTok{,}\DecValTok{19}\NormalTok{, }\DecValTok{8}\NormalTok{,}\DecValTok{24}\NormalTok{,}\DecValTok{13}\NormalTok{,}\DecValTok{8}\NormalTok{, }\DecValTok{3}\NormalTok{,}\DecValTok{16}\NormalTok{,}\DecValTok{10}\NormalTok{,}\DecValTok{10}\NormalTok{, }\DecValTok{2}\NormalTok{),}
  \AttributeTok{Sim =} \FunctionTok{c}\NormalTok{(}\DecValTok{31}\NormalTok{,}\DecValTok{54}\NormalTok{,}\DecValTok{16}\NormalTok{,}\DecValTok{38}\NormalTok{,}\DecValTok{13}\NormalTok{,}\DecValTok{19}\NormalTok{,}\DecValTok{14}\NormalTok{,}\DecValTok{33}\NormalTok{,}\DecValTok{20}\NormalTok{,}\DecValTok{24}\NormalTok{,}\DecValTok{26}\NormalTok{,}\DecValTok{30}\NormalTok{,}\DecValTok{6}\NormalTok{,}\DecValTok{11}\NormalTok{, }\DecValTok{8}\NormalTok{,}\DecValTok{49}\NormalTok{, }\DecValTok{4}\NormalTok{,}\DecValTok{12}\NormalTok{), }\CommentTok{\# 54 em vez 74}
  \AttributeTok{stringsAsFactors =} \ConstantTok{FALSE}                                      \CommentTok{\# pois 54+20=74}
\NormalTok{)}

\CommentTok{\# Função utilitária: formata qui{-}quadrado + estrelas de significância}
\NormalTok{stars\_from\_p }\OtherTok{\textless{}{-}} \ControlFlowTok{function}\NormalTok{(p) \{}
  \ControlFlowTok{if}\NormalTok{ (}\FunctionTok{is.na}\NormalTok{(p))  }\FunctionTok{return}\NormalTok{(}\StringTok{""}\NormalTok{)}
  \CommentTok{\# caractere de ponto em posição elevada (dot above)}
\NormalTok{  dot }\OtherTok{\textless{}{-}} \StringTok{"\textbackslash{}u02D9"}  \CommentTok{\# \textquotesingle{}˙\textquotesingle{} (dot above) — aparece como pequeno ponto superiorizado}
  \ControlFlowTok{if}\NormalTok{ (p }\SpecialCharTok{\textless{}} \FloatTok{0.01}\NormalTok{) }\FunctionTok{return}\NormalTok{(}\StringTok{"**"}\NormalTok{) }\CommentTok{\# teste significativo a 99\% de confiança}
  \ControlFlowTok{if}\NormalTok{ (p }\SpecialCharTok{\textless{}} \FloatTok{0.05}\NormalTok{) }\FunctionTok{return}\NormalTok{(}\StringTok{"*"}\NormalTok{)  }\CommentTok{\# teste significativo a 95\% de confiança}
  \ControlFlowTok{if}\NormalTok{ (p }\SpecialCharTok{\textless{}} \FloatTok{0.10}\NormalTok{) }\FunctionTok{return}\NormalTok{(dot)   }\CommentTok{\# teste significativo a 90\% de confiança}
  \FunctionTok{return}\NormalTok{(}\StringTok{""}\NormalTok{) }
\NormalTok{\}}

\CommentTok{\# Calcular qui{-}quadrado (Pearson) por mês}
\NormalTok{res\_list }\OtherTok{\textless{}{-}}\NormalTok{ dados }\SpecialCharTok{\%\textgreater{}\%}
  \FunctionTok{group\_by}\NormalTok{(Mes) }\SpecialCharTok{\%\textgreater{}\%}
  \FunctionTok{summarise}\NormalTok{(}
    \AttributeTok{c\_nao =}\NormalTok{ Nao[Grupo }\SpecialCharTok{==} \StringTok{"Controle"}\NormalTok{],}
    \AttributeTok{c\_sim =}\NormalTok{ Sim[Grupo }\SpecialCharTok{==} \StringTok{"Controle"}\NormalTok{],}
    \AttributeTok{e\_nao =}\NormalTok{ Nao[Grupo }\SpecialCharTok{==} \StringTok{"Experimental"}\NormalTok{],}
    \AttributeTok{e\_sim =}\NormalTok{ Sim[Grupo }\SpecialCharTok{==} \StringTok{"Experimental"}\NormalTok{],}
    \AttributeTok{.groups =} \StringTok{"drop"}
\NormalTok{  ) }\SpecialCharTok{\%\textgreater{}\%}
  \FunctionTok{rowwise}\NormalTok{() }\SpecialCharTok{\%\textgreater{}\%}
  \FunctionTok{mutate}\NormalTok{(}
    \AttributeTok{chi =}\NormalTok{ \{}
\NormalTok{      m }\OtherTok{\textless{}{-}} \FunctionTok{matrix}\NormalTok{(}\FunctionTok{c}\NormalTok{(c\_nao, c\_sim, e\_nao, e\_sim), }\AttributeTok{nrow =} \DecValTok{2}\NormalTok{, }\AttributeTok{byrow =} \ConstantTok{TRUE}\NormalTok{)}
\NormalTok{      tst }\OtherTok{\textless{}{-}} \FunctionTok{tryCatch}\NormalTok{(}\FunctionTok{chisq.test}\NormalTok{(m, }\AttributeTok{correct =} \ConstantTok{FALSE}\NormalTok{), }\AttributeTok{error =} \ControlFlowTok{function}\NormalTok{(e) }\ConstantTok{NULL}\NormalTok{)}
      \ControlFlowTok{if}\NormalTok{ (}\SpecialCharTok{!}\FunctionTok{inherits}\NormalTok{(tst, }\StringTok{"htest"}\NormalTok{)) \{}
        \StringTok{""}  \CommentTok{\# teste inválido ou erro {-}\textgreater{} string vazia}
\NormalTok{      \} }\ControlFlowTok{else}\NormalTok{ \{}
\NormalTok{        stat }\OtherTok{\textless{}{-}} \FunctionTok{unname}\NormalTok{(tst}\SpecialCharTok{$}\NormalTok{statistic)}
\NormalTok{        pval }\OtherTok{\textless{}{-}}\NormalTok{ tst}\SpecialCharTok{$}\NormalTok{p.value}
\NormalTok{        stat\_rounded }\OtherTok{\textless{}{-}} \FunctionTok{formatC}\NormalTok{(stat, }\AttributeTok{digits =} \DecValTok{2}\NormalTok{, }\AttributeTok{format =} \StringTok{"f"}\NormalTok{)}
        \FunctionTok{paste0}\NormalTok{(stat\_rounded, }\FunctionTok{stars\_from\_p}\NormalTok{(pval))}
\NormalTok{      \}}
\NormalTok{    \},}
    \AttributeTok{pval =}\NormalTok{ \{}
\NormalTok{      m }\OtherTok{\textless{}{-}} \FunctionTok{matrix}\NormalTok{(}\FunctionTok{c}\NormalTok{(c\_nao, c\_sim, e\_nao, e\_sim), }\AttributeTok{nrow =} \DecValTok{2}\NormalTok{, }\AttributeTok{byrow =} \ConstantTok{TRUE}\NormalTok{)}
\NormalTok{      tst }\OtherTok{\textless{}{-}} \FunctionTok{tryCatch}\NormalTok{(}\FunctionTok{chisq.test}\NormalTok{(m, }\AttributeTok{correct =} \ConstantTok{FALSE}\NormalTok{), }\AttributeTok{error =} \ControlFlowTok{function}\NormalTok{(e) }\ConstantTok{NULL}\NormalTok{)}
      \ControlFlowTok{if}\NormalTok{ (}\SpecialCharTok{!}\FunctionTok{inherits}\NormalTok{(tst, }\StringTok{"htest"}\NormalTok{)) \{}
        \StringTok{""}  \CommentTok{\# teste inválido ou erro {-}\textgreater{} string vazia}
\NormalTok{      \} }\ControlFlowTok{else}\NormalTok{ \{}
\NormalTok{        pval }\OtherTok{\textless{}{-}}\NormalTok{ tst}\SpecialCharTok{$}\NormalTok{p.value}
        \FunctionTok{formatC}\NormalTok{(pval, }\AttributeTok{digits =} \DecValTok{4}\NormalTok{, }\AttributeTok{format =} \StringTok{"f"}\NormalTok{)}
\NormalTok{      \}}
\NormalTok{    \}}
\NormalTok{  ) }\SpecialCharTok{\%\textgreater{}\%}
  \FunctionTok{ungroup}\NormalTok{() }\SpecialCharTok{\%\textgreater{}\%}
  \FunctionTok{select}\NormalTok{(Mes, chi, pval)}

\CommentTok{\# Juntar resultado à tabela original (mostrar chi apenas na linha Controle)}
\NormalTok{tabela\_final }\OtherTok{\textless{}{-}}\NormalTok{ dados }\SpecialCharTok{\%\textgreater{}\%}
  \FunctionTok{left\_join}\NormalTok{(res\_list, }\AttributeTok{by =} \StringTok{"Mes"}\NormalTok{) }\SpecialCharTok{\%\textgreater{}\%}
  \FunctionTok{group\_by}\NormalTok{(Mes) }\SpecialCharTok{\%\textgreater{}\%}
  \FunctionTok{mutate}\NormalTok{(}
    \AttributeTok{Valor\_do\_qui\_quadrado =} \FunctionTok{ifelse}\NormalTok{(Grupo }\SpecialCharTok{==} \StringTok{"Controle"}\NormalTok{, chi, }\StringTok{""}\NormalTok{),}
    \AttributeTok{Valor\_p =} \FunctionTok{ifelse}\NormalTok{(Grupo }\SpecialCharTok{==} \StringTok{"Controle"}\NormalTok{, pval, }\StringTok{""}\NormalTok{)}
\NormalTok{  ) }\SpecialCharTok{\%\textgreater{}\%}
  \FunctionTok{ungroup}\NormalTok{() }\SpecialCharTok{\%\textgreater{}\%}
  \FunctionTok{select}\NormalTok{(Mes, Grupo, Nao, Sim, Valor\_do\_qui\_quadrado, Valor\_p)}

\CommentTok{\# Exibir tabela formatada}
\FunctionTok{cat}\NormalTok{(}\StringTok{"}\SpecialCharTok{\textbackslash{}n}\StringTok{Tabela com qui{-}quadrado por mês (aparecendo na linha Controle):}\SpecialCharTok{\textbackslash{}n}\StringTok{"}\NormalTok{)}
\FunctionTok{print}\NormalTok{(}\FunctionTok{kable}\NormalTok{(tabela\_final, }\AttributeTok{align =} \FunctionTok{c}\NormalTok{(}\StringTok{"l"}\NormalTok{, }\StringTok{"l"}\NormalTok{, }\StringTok{"r"}\NormalTok{, }\StringTok{"c"}\NormalTok{, }\StringTok{"l"}\NormalTok{)))}

\CommentTok{\# (Opcional) salvar CSV:}
\CommentTok{\# write.csv(tabela\_final, "tabela\_mensal\_chisq.csv", row.names = FALSE)}
\InformationTok{\textasciigrave{}\textasciigrave{}\textasciigrave{}}
\end{Highlighting}
\end{Shaded}

\begin{verbatim}

Tabela com qui-quadrado por mês (aparecendo na linha Controle):


|Mes      |Grupo        | Nao| Sim |Valor_do_qui_quadrado |Valor_p |
|:--------|:------------|---:|:---:|:---------------------|:-------|
|Abril    |Controle     |  35| 31  |9.89**                |0.0017  |
|Abril    |Experimental |  20| 54  |                      |        |
|Maio     |Controle     |  17| 16  |11.56**               |0.0007  |
|Maio     |Experimental |   7| 38  |                      |        |
|Junho    |Controle     |  18| 13  |0.78                  |0.3757  |
|Junho    |Experimental |  17| 19  |                      |        |
|Julho    |Controle     |  20| 14  |18.25**               |0.0000  |
|Julho    |Experimental |   4| 33  |                      |        |
|Agosto   |Controle     |  19| 20  |4.20*                 |0.0405  |
|Agosto   |Experimental |   8| 24  |                      |        |
|Setembro |Controle     |  24| 26  |3.05˙                 |0.0809  |
|Setembro |Experimental |  13| 30  |                      |        |
|Outubro  |Controle     |   8|  6  |3.74˙                 |0.0530  |
|Outubro  |Experimental |   3| 11  |                      |        |
|Novembro |Controle     |  16|  8  |19.60**               |0.0000  |
|Novembro |Experimental |  10| 49  |                      |        |
|Dezembro |Controle     |  10|  4  |9.33**                |0.0023  |
|Dezembro |Experimental |   2| 12  |                      |        |
\end{verbatim}

\subsection{Qui-quadrado dados
agregados}\label{qui-quadrado-dados-agregados}

Teste qui-quadrado para todo o conjunto de dados não desagregado mês a
mês.

Manter os meses em ordem cronológica.

\begin{Shaded}
\begin{Highlighting}[numbers=left,,]
\InformationTok{\textasciigrave{}\textasciigrave{}\textasciigrave{}\{r\}}
\CommentTok{\# Teste qui{-}quadrado agregando todos os meses em Controle vs Tratamento}
\CommentTok{\# e executando testes por mês com meses ordenados cronologicamente.}
\CommentTok{\#}
\CommentTok{\# Entrada: objeto \textquotesingle{}tabela\_final\textquotesingle{} ou \textquotesingle{}dados\textquotesingle{} no ambiente (colunas Mes, Grupo, Nao, Sim),}
\CommentTok{\# ou arquivo "tabela\_mensal\_chisq.csv".}
\CommentTok{\#}
\CommentTok{\# Saídas: tabela agregada 2x2, resultado do chi{-}squared (ou Fisher), phi e resultados por mês}
\CommentTok{\# com meses exibidos na ordem: Abril, Maio, Junho, Julho, Agosto, Setembro, Outubro, Novembro, Dezembro.}

\ControlFlowTok{if}\NormalTok{ (}\SpecialCharTok{!}\FunctionTok{requireNamespace}\NormalTok{(}\StringTok{"dplyr"}\NormalTok{, }\AttributeTok{quietly =} \ConstantTok{TRUE}\NormalTok{)) }\FunctionTok{install.packages}\NormalTok{(}\StringTok{"dplyr"}\NormalTok{)}
\ControlFlowTok{if}\NormalTok{ (}\SpecialCharTok{!}\FunctionTok{requireNamespace}\NormalTok{(}\StringTok{"tidyr"}\NormalTok{, }\AttributeTok{quietly =} \ConstantTok{TRUE}\NormalTok{)) }\FunctionTok{install.packages}\NormalTok{(}\StringTok{"tidyr"}\NormalTok{)}
\FunctionTok{library}\NormalTok{(dplyr)}
\FunctionTok{library}\NormalTok{(tidyr)}

\CommentTok{\# {-}{-}{-} Obter data.frame (procura objetos no ambiente ou CSV) {-}{-}{-}}
\NormalTok{df }\OtherTok{\textless{}{-}} \ConstantTok{NULL}
\ControlFlowTok{if}\NormalTok{ (}\FunctionTok{exists}\NormalTok{(}\StringTok{"tabela\_final"}\NormalTok{, }\AttributeTok{envir =}\NormalTok{ .GlobalEnv)) \{}
\NormalTok{  df }\OtherTok{\textless{}{-}} \FunctionTok{get}\NormalTok{(}\StringTok{"tabela\_final"}\NormalTok{, }\AttributeTok{envir =}\NormalTok{ .GlobalEnv)}
\NormalTok{\} }\ControlFlowTok{else} \ControlFlowTok{if}\NormalTok{ (}\FunctionTok{exists}\NormalTok{(}\StringTok{"dados"}\NormalTok{, }\AttributeTok{envir =}\NormalTok{ .GlobalEnv)) \{}
\NormalTok{  df }\OtherTok{\textless{}{-}} \FunctionTok{get}\NormalTok{(}\StringTok{"dados"}\NormalTok{, }\AttributeTok{envir =}\NormalTok{ .GlobalEnv)}
\NormalTok{\} }\ControlFlowTok{else} \ControlFlowTok{if}\NormalTok{ (}\FunctionTok{file.exists}\NormalTok{(}\StringTok{"tabela\_mensal\_chisq.csv"}\NormalTok{)) \{}
\NormalTok{  df }\OtherTok{\textless{}{-}} \FunctionTok{read.csv}\NormalTok{(}\StringTok{"tabela\_mensal\_chisq.csv"}\NormalTok{, }\AttributeTok{stringsAsFactors =} \ConstantTok{FALSE}\NormalTok{, }\AttributeTok{fileEncoding =} \StringTok{"UTF{-}8"}\NormalTok{)}
\NormalTok{\} }\ControlFlowTok{else}\NormalTok{ \{}
  \FunctionTok{stop}\NormalTok{(}\StringTok{"Não encontrei \textquotesingle{}tabela\_final\textquotesingle{} nem \textquotesingle{}dados\textquotesingle{} no ambiente, nem o ficheiro \textquotesingle{}tabela\_mensal\_chisq.csv\textquotesingle{}."}\NormalTok{)}
\NormalTok{\}}

\CommentTok{\# {-}{-}{-} Normalizar nomes de colunas (minúsculas, sem caracteres especiais) {-}{-}{-}}
\FunctionTok{names}\NormalTok{(df) }\OtherTok{\textless{}{-}} \FunctionTok{tolower}\NormalTok{(}\FunctionTok{names}\NormalTok{(df))}
\FunctionTok{names}\NormalTok{(df) }\OtherTok{\textless{}{-}} \FunctionTok{gsub}\NormalTok{(}\StringTok{"[\^{}a{-}z0{-}9\_]"}\NormalTok{, }\StringTok{"\_"}\NormalTok{, }\FunctionTok{names}\NormalTok{(df))}

\CommentTok{\# identificar possíveis colunas}
\NormalTok{col\_grupo }\OtherTok{\textless{}{-}} \FunctionTok{intersect}\NormalTok{(}\FunctionTok{names}\NormalTok{(df), }\FunctionTok{c}\NormalTok{(}\StringTok{"grupo"}\NormalTok{,}\StringTok{"group"}\NormalTok{,}\StringTok{"tratamento"}\NormalTok{,}\StringTok{"treatment"}\NormalTok{))}
\NormalTok{col\_nao   }\OtherTok{\textless{}{-}} \FunctionTok{intersect}\NormalTok{(}\FunctionTok{names}\NormalTok{(df), }\FunctionTok{c}\NormalTok{(}\StringTok{"nao"}\NormalTok{,}\StringTok{"nao\_"}\NormalTok{,}\StringTok{"n\_nao"}\NormalTok{,}\StringTok{"n\_nao"}\NormalTok{,}\StringTok{"no"}\NormalTok{,}\StringTok{"n"}\NormalTok{))}
\NormalTok{col\_sim   }\OtherTok{\textless{}{-}} \FunctionTok{intersect}\NormalTok{(}\FunctionTok{names}\NormalTok{(df), }\FunctionTok{c}\NormalTok{(}\StringTok{"sim"}\NormalTok{,}\StringTok{"yes"}\NormalTok{,}\StringTok{"y"}\NormalTok{,}\StringTok{"s"}\NormalTok{))}
\CommentTok{\# detectar coluna de mês por padrão buscando substrings}
\NormalTok{col\_mes }\OtherTok{\textless{}{-}} \FunctionTok{names}\NormalTok{(df)[}\FunctionTok{grepl}\NormalTok{(}\StringTok{"mes|month|m\_"}\NormalTok{, }\FunctionTok{names}\NormalTok{(df), }\AttributeTok{ignore.case =} \ConstantTok{TRUE}\NormalTok{)]}
\ControlFlowTok{if}\NormalTok{ (}\FunctionTok{length}\NormalTok{(col\_grupo) }\SpecialCharTok{==} \DecValTok{0}\NormalTok{) }\FunctionTok{stop}\NormalTok{(}\StringTok{"Coluna de grupo não encontrada (procure por \textquotesingle{}Grupo\textquotesingle{}/\textquotesingle{}group\textquotesingle{})."}\NormalTok{)}
\ControlFlowTok{if}\NormalTok{ (}\FunctionTok{length}\NormalTok{(col\_nao) }\SpecialCharTok{==} \DecValTok{0} \SpecialCharTok{||} \FunctionTok{length}\NormalTok{(col\_sim) }\SpecialCharTok{==} \DecValTok{0}\NormalTok{) }\FunctionTok{stop}\NormalTok{(}\StringTok{"Colunas de contagem \textquotesingle{}Nao\textquotesingle{} e \textquotesingle{}Sim\textquotesingle{} não foram encontradas."}\NormalTok{)}

\NormalTok{col\_grupo }\OtherTok{\textless{}{-}}\NormalTok{ col\_grupo[}\DecValTok{1}\NormalTok{]}
\NormalTok{col\_nao   }\OtherTok{\textless{}{-}}\NormalTok{ col\_nao[}\DecValTok{1}\NormalTok{]}
\NormalTok{col\_sim   }\OtherTok{\textless{}{-}}\NormalTok{ col\_sim[}\DecValTok{1}\NormalTok{]}
\NormalTok{mes\_col   }\OtherTok{\textless{}{-}} \ControlFlowTok{if}\NormalTok{ (}\FunctionTok{length}\NormalTok{(col\_mes) }\SpecialCharTok{\textgreater{}=} \DecValTok{1}\NormalTok{) col\_mes[}\DecValTok{1}\NormalTok{] }\ControlFlowTok{else} \ConstantTok{NA\_character\_}

\CommentTok{\# {-}{-}{-} Padronizar rótulos de grupo: Controle / Tratamento {-}{-}{-}}
\NormalTok{df }\OtherTok{\textless{}{-}}\NormalTok{ df }\SpecialCharTok{\%\textgreater{}\%}
  \FunctionTok{mutate}\NormalTok{(}
    \AttributeTok{grupo\_raw =} \FunctionTok{as.character}\NormalTok{(.data[[col\_grupo]]),}
    \AttributeTok{grupo =} \FunctionTok{ifelse}\NormalTok{(}\FunctionTok{tolower}\NormalTok{(}\FunctionTok{trimws}\NormalTok{(grupo\_raw)) }\SpecialCharTok{\%in\%} \FunctionTok{c}\NormalTok{(}\StringTok{"controle"}\NormalTok{,}\StringTok{"control"}\NormalTok{), }\StringTok{"Controle"}\NormalTok{, }\StringTok{"Tratamento"}\NormalTok{),}
    \AttributeTok{n\_nao =} \FunctionTok{as.numeric}\NormalTok{(.data[[col\_nao]]),}
    \AttributeTok{n\_sim =} \FunctionTok{as.numeric}\NormalTok{(.data[[col\_sim]])}
\NormalTok{  )}

\CommentTok{\# {-}{-}{-} Ordenar meses (se coluna de mês existir) {-}{-}{-}}
\NormalTok{month\_levels }\OtherTok{\textless{}{-}} \FunctionTok{c}\NormalTok{(}\StringTok{"Abril"}\NormalTok{,}\StringTok{"Maio"}\NormalTok{,}\StringTok{"Junho"}\NormalTok{,}\StringTok{"Julho"}\NormalTok{,}\StringTok{"Agosto"}\NormalTok{,}\StringTok{"Setembro"}\NormalTok{,}\StringTok{"Outubro"}\NormalTok{,}\StringTok{"Novembro"}\NormalTok{,}\StringTok{"Dezembro"}\NormalTok{)}

\ControlFlowTok{if}\NormalTok{ (}\SpecialCharTok{!}\FunctionTok{is.na}\NormalTok{(mes\_col)) \{}
  \CommentTok{\# mapear valores existentes para título padrão dos meses quando possível (ignora case)}
\NormalTok{  map\_month\_value }\OtherTok{\textless{}{-}} \ControlFlowTok{function}\NormalTok{(v) \{}
    \ControlFlowTok{if}\NormalTok{ (}\FunctionTok{is.na}\NormalTok{(v)) }\FunctionTok{return}\NormalTok{(}\ConstantTok{NA\_character\_}\NormalTok{)}
\NormalTok{    v\_trim }\OtherTok{\textless{}{-}} \FunctionTok{trimws}\NormalTok{(}\FunctionTok{as.character}\NormalTok{(v))}
    \CommentTok{\# tentar correspondência direta ignorando case}
\NormalTok{    match\_idx }\OtherTok{\textless{}{-}} \FunctionTok{match}\NormalTok{(}\FunctionTok{tolower}\NormalTok{(v\_trim), }\FunctionTok{tolower}\NormalTok{(month\_levels))}
    \ControlFlowTok{if}\NormalTok{ (}\SpecialCharTok{!}\FunctionTok{is.na}\NormalTok{(match\_idx)) }\FunctionTok{return}\NormalTok{(month\_levels[match\_idx])}
    \CommentTok{\# tentar remover acentos e comparar (fallback)}
\NormalTok{    v\_clean }\OtherTok{\textless{}{-}} \FunctionTok{iconv}\NormalTok{(v\_trim, }\AttributeTok{to =} \StringTok{"ASCII//TRANSLIT"}\NormalTok{)}
\NormalTok{    match\_idx }\OtherTok{\textless{}{-}} \FunctionTok{match}\NormalTok{(}\FunctionTok{tolower}\NormalTok{(v\_clean), }\FunctionTok{tolower}\NormalTok{(}\FunctionTok{iconv}\NormalTok{(month\_levels, }\AttributeTok{to =} \StringTok{"ASCII//TRANSLIT"}\NormalTok{)))}
    \ControlFlowTok{if}\NormalTok{ (}\SpecialCharTok{!}\FunctionTok{is.na}\NormalTok{(match\_idx)) }\FunctionTok{return}\NormalTok{(month\_levels[match\_idx])}
    \CommentTok{\# se não reconhecer, retorna o valor original (será colocado após os níveis definidos)}
    \FunctionTok{return}\NormalTok{(v\_trim)}
\NormalTok{  \}}
\NormalTok{  df}\SpecialCharTok{$}\NormalTok{mes\_mapped }\OtherTok{\textless{}{-}} \FunctionTok{vapply}\NormalTok{(df[[mes\_col]], map\_month\_value, }\FunctionTok{character}\NormalTok{(}\DecValTok{1}\NormalTok{))}
  \CommentTok{\# definir factor com níveis na ordem desejada; níveis extras (não reconhecidos) ficarão NA se não estiverem incluídos}
  \CommentTok{\# incluir somente os meses presentes entre os month\_levels para evitar NAs indesejadas}
\NormalTok{  present\_levels }\OtherTok{\textless{}{-}} \FunctionTok{intersect}\NormalTok{(month\_levels, }\FunctionTok{unique}\NormalTok{(df}\SpecialCharTok{$}\NormalTok{mes\_mapped))}
  \ControlFlowTok{if}\NormalTok{ (}\FunctionTok{length}\NormalTok{(present\_levels) }\SpecialCharTok{==} \DecValTok{0}\NormalTok{) \{}
    \CommentTok{\# se nenhum mês reconhecido, criar factor com ordem alfabética dos valores existentes}
\NormalTok{    df}\SpecialCharTok{$}\NormalTok{mes\_ord }\OtherTok{\textless{}{-}} \FunctionTok{factor}\NormalTok{(df}\SpecialCharTok{$}\NormalTok{mes\_mapped, }\AttributeTok{levels =} \FunctionTok{unique}\NormalTok{(df}\SpecialCharTok{$}\NormalTok{mes\_mapped))}
\NormalTok{  \} }\ControlFlowTok{else}\NormalTok{ \{}
\NormalTok{    df}\SpecialCharTok{$}\NormalTok{mes\_ord }\OtherTok{\textless{}{-}} \FunctionTok{factor}\NormalTok{(df}\SpecialCharTok{$}\NormalTok{mes\_mapped, }\AttributeTok{levels =}\NormalTok{ month\_levels)}
\NormalTok{  \}}
\NormalTok{\} }\ControlFlowTok{else}\NormalTok{ \{}
\NormalTok{  df}\SpecialCharTok{$}\NormalTok{mes\_ord }\OtherTok{\textless{}{-}} \ConstantTok{NA}
\NormalTok{\}}

\CommentTok{\# {-}{-}{-} Agregar todos os meses por grupo (somar Nao/Sim) para teste agregado {-}{-}{-}}
\NormalTok{agg }\OtherTok{\textless{}{-}}\NormalTok{ df }\SpecialCharTok{\%\textgreater{}\%}
  \FunctionTok{group\_by}\NormalTok{(grupo) }\SpecialCharTok{\%\textgreater{}\%}
  \FunctionTok{summarise}\NormalTok{(}\AttributeTok{nao =} \FunctionTok{sum}\NormalTok{(n\_nao, }\AttributeTok{na.rm =} \ConstantTok{TRUE}\NormalTok{), }\AttributeTok{sim =} \FunctionTok{sum}\NormalTok{(n\_sim, }\AttributeTok{na.rm =} \ConstantTok{TRUE}\NormalTok{), }\AttributeTok{.groups =} \StringTok{"drop"}\NormalTok{)}

\CommentTok{\# garantir linhas Controle/Tratamento}
\NormalTok{agg }\OtherTok{\textless{}{-}}\NormalTok{ agg }\SpecialCharTok{\%\textgreater{}\%} \FunctionTok{mutate}\NormalTok{(}\AttributeTok{grupo =} \FunctionTok{ifelse}\NormalTok{(grupo }\SpecialCharTok{==} \StringTok{"Controle"}\NormalTok{, }\StringTok{"Controle"}\NormalTok{, }\StringTok{"Tratamento"}\NormalTok{)) }\SpecialCharTok{\%\textgreater{}\%}
  \FunctionTok{group\_by}\NormalTok{(grupo) }\SpecialCharTok{\%\textgreater{}\%} \FunctionTok{summarise}\NormalTok{(}\AttributeTok{nao =} \FunctionTok{sum}\NormalTok{(nao), }\AttributeTok{sim =} \FunctionTok{sum}\NormalTok{(sim), }\AttributeTok{.groups =} \StringTok{"drop"}\NormalTok{) }\SpecialCharTok{\%\textgreater{}\%}
  \FunctionTok{arrange}\NormalTok{(}\FunctionTok{match}\NormalTok{(grupo, }\FunctionTok{c}\NormalTok{(}\StringTok{"Controle"}\NormalTok{,}\StringTok{"Tratamento"}\NormalTok{)))}

\CommentTok{\# construir tabela 2x2}
\NormalTok{tab }\OtherTok{\textless{}{-}} \FunctionTok{matrix}\NormalTok{(}\FunctionTok{c}\NormalTok{(agg}\SpecialCharTok{$}\NormalTok{nao[agg}\SpecialCharTok{$}\NormalTok{grupo}\SpecialCharTok{==}\StringTok{"Controle"}\NormalTok{], agg}\SpecialCharTok{$}\NormalTok{sim[agg}\SpecialCharTok{$}\NormalTok{grupo}\SpecialCharTok{==}\StringTok{"Controle"}\NormalTok{],}
\NormalTok{                agg}\SpecialCharTok{$}\NormalTok{nao[agg}\SpecialCharTok{$}\NormalTok{grupo}\SpecialCharTok{==}\StringTok{"Tratamento"}\NormalTok{], agg}\SpecialCharTok{$}\NormalTok{sim[agg}\SpecialCharTok{$}\NormalTok{grupo}\SpecialCharTok{==}\StringTok{"Tratamento"}\NormalTok{]),}
              \AttributeTok{nrow =} \DecValTok{2}\NormalTok{, }\AttributeTok{byrow =} \ConstantTok{TRUE}\NormalTok{)}
\FunctionTok{rownames}\NormalTok{(tab) }\OtherTok{\textless{}{-}} \FunctionTok{c}\NormalTok{(}\StringTok{"Controle"}\NormalTok{,}\StringTok{"Tratamento"}\NormalTok{)}
\FunctionTok{colnames}\NormalTok{(tab) }\OtherTok{\textless{}{-}} \FunctionTok{c}\NormalTok{(}\StringTok{"Nao"}\NormalTok{,}\StringTok{"Sim"}\NormalTok{)}

\FunctionTok{cat}\NormalTok{(}\StringTok{"Tabela agregada (linhas = Grupo; colunas = Nao/Sim):}\SpecialCharTok{\textbackslash{}n}\StringTok{"}\NormalTok{)}
\FunctionTok{print}\NormalTok{(tab)}

\CommentTok{\# Tabela de contingência (linhas = Grupo, colunas = acordo Nao/Sim)}
\CommentTok{\# Adicionar totais marginais (linha e coluna)}
\NormalTok{tab\_totais }\OtherTok{\textless{}{-}} \FunctionTok{addmargins}\NormalTok{(tab)}

\CommentTok{\# Substituir o rótulo padrão "Sum" por "Total" nas margens (mais legível em pt{-}BR)}
\FunctionTok{rownames}\NormalTok{(tab\_totais)[}\FunctionTok{nrow}\NormalTok{(tab\_totais)] }\OtherTok{\textless{}{-}} \StringTok{"Total"}
\FunctionTok{colnames}\NormalTok{(tab\_totais)[}\FunctionTok{ncol}\NormalTok{(tab\_totais)] }\OtherTok{\textless{}{-}} \StringTok{"Total"}

\CommentTok{\# Exibir tabela com totais marginais}
\FunctionTok{cat}\NormalTok{(}\StringTok{"}\SpecialCharTok{\textbackslash{}n}\StringTok{Tabela com totais marginais:}\SpecialCharTok{\textbackslash{}n}\StringTok{"}\NormalTok{)}
\FunctionTok{print}\NormalTok{(tab\_totais)}
\FunctionTok{cat}\NormalTok{(}\StringTok{"}\SpecialCharTok{\textbackslash{}n}\StringTok{Observar que o total de 659 observações confere}\SpecialCharTok{\textbackslash{}n}\StringTok{com o tamanho da amostra reportado: n = 659}\SpecialCharTok{\textbackslash{}n}\StringTok{"}\NormalTok{)}
\FunctionTok{cat}\NormalTok{(}\StringTok{"}\SpecialCharTok{\textbackslash{}n}\StringTok{"}\NormalTok{)}
\FunctionTok{cat}\NormalTok{(}\StringTok{"}\SpecialCharTok{\textbackslash{}n}\StringTok{O Grupo de Controle confere com o reportado (n\_GC = 305 audiências).}\SpecialCharTok{\textbackslash{}n}\StringTok{"}\NormalTok{)}
\FunctionTok{cat}\NormalTok{(}\StringTok{"}\SpecialCharTok{\textbackslash{}n}\StringTok{"}\NormalTok{)}
\FunctionTok{cat}\NormalTok{(}\StringTok{"}\SpecialCharTok{\textbackslash{}n}\StringTok{E o Grupo de Experimental confere c/o reportado (n\_GT = 354 audiências).}\SpecialCharTok{\textbackslash{}n}\StringTok{"}\NormalTok{)}

\CommentTok{\# {-}{-}{-} Marginais em contagem {-}{-}{-}}
\CommentTok{\# Totais por linha (por grupo: controle x tratamento)}
\NormalTok{margem\_linhas }\OtherTok{\textless{}{-}} \FunctionTok{margin.table}\NormalTok{(tab, }\DecValTok{1}\NormalTok{)}

\CommentTok{\# Totais por coluna (por acordo: sim ou não)}
\NormalTok{margem\_colunas }\OtherTok{\textless{}{-}} \FunctionTok{margin.table}\NormalTok{(tab, }\DecValTok{2}\NormalTok{)}

\FunctionTok{cat}\NormalTok{(}\StringTok{"}\SpecialCharTok{\textbackslash{}n}\StringTok{Marginal — por grupo: controle x tratamento (contagem):}\SpecialCharTok{\textbackslash{}n}\StringTok{"}\NormalTok{)}
\FunctionTok{print}\NormalTok{(margem\_linhas)}
\FunctionTok{cat}\NormalTok{(}\StringTok{"}\SpecialCharTok{\textbackslash{}n}\StringTok{Marginal — por acordo: sim ou não (contagem):}\SpecialCharTok{\textbackslash{}n}\StringTok{"}\NormalTok{)}
\FunctionTok{print}\NormalTok{(margem\_colunas)}

\CommentTok{\# {-}{-}{-} Marginais em proporção (relativas ao total geral) {-}{-}{-}}
\NormalTok{prop\_linhas  }\OtherTok{\textless{}{-}} \FunctionTok{prop.table}\NormalTok{(margem\_linhas)  }\CommentTok{\# soma = 1}
\NormalTok{prop\_colunas }\OtherTok{\textless{}{-}} \FunctionTok{prop.table}\NormalTok{(margem\_colunas) }\CommentTok{\# soma = 1}

\FunctionTok{cat}\NormalTok{(}\StringTok{"}\SpecialCharTok{\textbackslash{}n}\StringTok{Marginal — por grupo: controle x tratamento (proporção \% do total):}\SpecialCharTok{\textbackslash{}n}\StringTok{"}\NormalTok{)}
\FunctionTok{print}\NormalTok{(}\DecValTok{100}\SpecialCharTok{*}\FunctionTok{round}\NormalTok{(prop\_linhas, }\DecValTok{4}\NormalTok{))}
\FunctionTok{cat}\NormalTok{(}\StringTok{"}\SpecialCharTok{\textbackslash{}n}\StringTok{Marginal — por acordo: sim ou não (proporção \% do total):}\SpecialCharTok{\textbackslash{}n}\StringTok{"}\NormalTok{)}
\FunctionTok{print}\NormalTok{(}\DecValTok{100}\SpecialCharTok{*}\FunctionTok{round}\NormalTok{(prop\_colunas, }\DecValTok{4}\NormalTok{))}


\CommentTok{\# {-}{-}{-} Teste qui{-}quadrado agregado (ou Fisher se necessário) {-}{-}{-}}
\NormalTok{chisq\_ok }\OtherTok{\textless{}{-}} \ConstantTok{TRUE}
\NormalTok{tst\_chisq }\OtherTok{\textless{}{-}} \FunctionTok{tryCatch}\NormalTok{(}\FunctionTok{chisq.test}\NormalTok{(tab, }\AttributeTok{correct =} \ConstantTok{FALSE}\NormalTok{), }\AttributeTok{error =} \ControlFlowTok{function}\NormalTok{(e) e)}
\ControlFlowTok{if}\NormalTok{ (}\FunctionTok{inherits}\NormalTok{(tst\_chisq, }\StringTok{"error"}\NormalTok{)) \{}
\NormalTok{  chisq\_ok }\OtherTok{\textless{}{-}} \ConstantTok{FALSE}
  \FunctionTok{message}\NormalTok{(}\StringTok{"chisq.test falhou: "}\NormalTok{, tst\_chisq}\SpecialCharTok{$}\NormalTok{message)}
\NormalTok{\} }\ControlFlowTok{else}\NormalTok{ \{}
\NormalTok{  expc }\OtherTok{\textless{}{-}} \FunctionTok{as.numeric}\NormalTok{(tst\_chisq}\SpecialCharTok{$}\NormalTok{expected)}
  \ControlFlowTok{if}\NormalTok{ (}\FunctionTok{any}\NormalTok{(expc }\SpecialCharTok{\textless{}} \DecValTok{5}\NormalTok{)) \{}
\NormalTok{    chisq\_ok }\OtherTok{\textless{}{-}} \ConstantTok{FALSE}
    \FunctionTok{message}\NormalTok{(}\StringTok{"Algumas frequências esperadas \textless{} 5; usaremos Fisher exact test em vez de qui{-}quadrado."}\NormalTok{)}
\NormalTok{  \}}
\NormalTok{\}}

\ControlFlowTok{if}\NormalTok{ (chisq\_ok) \{}
  \FunctionTok{cat}\NormalTok{(}\StringTok{"}\SpecialCharTok{\textbackslash{}n}\StringTok{Resultado do chi{-}squared test (Pearson):}\SpecialCharTok{\textbackslash{}n}\StringTok{"}\NormalTok{)}
  \FunctionTok{print}\NormalTok{(tst\_chisq)}
\NormalTok{  chi\_stat }\OtherTok{\textless{}{-}} \FunctionTok{as.numeric}\NormalTok{(tst\_chisq}\SpecialCharTok{$}\NormalTok{statistic)[}\DecValTok{1}\NormalTok{]}
\NormalTok{  pval }\OtherTok{\textless{}{-}} \FunctionTok{as.numeric}\NormalTok{(tst\_chisq}\SpecialCharTok{$}\NormalTok{p.value)[}\DecValTok{1}\NormalTok{]}
\NormalTok{\} }\ControlFlowTok{else}\NormalTok{ \{}
\NormalTok{  fisher\_res }\OtherTok{\textless{}{-}} \FunctionTok{fisher.test}\NormalTok{(tab)}
  \FunctionTok{cat}\NormalTok{(}\StringTok{"}\SpecialCharTok{\textbackslash{}n}\StringTok{Resultado do Fisher exact test:}\SpecialCharTok{\textbackslash{}n}\StringTok{"}\NormalTok{)}
  \FunctionTok{print}\NormalTok{(fisher\_res)}
\NormalTok{  chi\_stat }\OtherTok{\textless{}{-}} \ConstantTok{NA}
\NormalTok{  pval }\OtherTok{\textless{}{-}} \FunctionTok{as.numeric}\NormalTok{(fisher\_res}\SpecialCharTok{$}\NormalTok{p.value)[}\DecValTok{1}\NormalTok{]}
\NormalTok{\}}

\CommentTok{\# {-}{-}{-} Medida de efeito phi para 2x2 {-}{-}{-}}
\NormalTok{n\_total }\OtherTok{\textless{}{-}} \FunctionTok{sum}\NormalTok{(tab)}
\ControlFlowTok{if}\NormalTok{ (}\SpecialCharTok{!}\FunctionTok{is.na}\NormalTok{(chi\_stat)) \{}
\NormalTok{  phi }\OtherTok{\textless{}{-}} \FunctionTok{sqrt}\NormalTok{(chi\_stat }\SpecialCharTok{/}\NormalTok{ n\_total)}
  \FunctionTok{cat}\NormalTok{(}\FunctionTok{sprintf}\NormalTok{(}\StringTok{"}\SpecialCharTok{\textbackslash{}n}\StringTok{Phi (efeito para 2x2) = \%.4f}\SpecialCharTok{\textbackslash{}n}\StringTok{"}\NormalTok{, phi))}
\NormalTok{\} }\ControlFlowTok{else}\NormalTok{ \{}
  \CommentTok{\# tentar calcular chi via chisq.test fallback}
\NormalTok{  chi\_manual }\OtherTok{\textless{}{-}} \FunctionTok{tryCatch}\NormalTok{(\{}
\NormalTok{    ch }\OtherTok{\textless{}{-}} \FunctionTok{chisq.test}\NormalTok{(tab, }\AttributeTok{correct =} \ConstantTok{FALSE}\NormalTok{)}
    \FunctionTok{as.numeric}\NormalTok{(ch}\SpecialCharTok{$}\NormalTok{statistic)}
\NormalTok{  \}, }\AttributeTok{error =} \ControlFlowTok{function}\NormalTok{(e) }\ConstantTok{NA}\NormalTok{)}
  \ControlFlowTok{if}\NormalTok{ (}\SpecialCharTok{!}\FunctionTok{is.na}\NormalTok{(chi\_manual)) \{}
\NormalTok{    phi }\OtherTok{\textless{}{-}} \FunctionTok{sqrt}\NormalTok{(chi\_manual }\SpecialCharTok{/}\NormalTok{ n\_total)}
    \FunctionTok{cat}\NormalTok{(}\FunctionTok{sprintf}\NormalTok{(}\StringTok{"}\SpecialCharTok{\textbackslash{}n}\StringTok{Phi (calculado por fallback) = \%.4f}\SpecialCharTok{\textbackslash{}n}\StringTok{"}\NormalTok{, phi))}
\NormalTok{  \} }\ControlFlowTok{else}\NormalTok{ \{}
    \FunctionTok{cat}\NormalTok{(}\StringTok{"}\SpecialCharTok{\textbackslash{}n}\StringTok{Phi não disponível (teste Fisher usado e chi2 não calculável).}\SpecialCharTok{\textbackslash{}n}\StringTok{"}\NormalTok{)}
\NormalTok{  \}}
\NormalTok{\}}

\CommentTok{\# {-}{-}{-} Testes por mês ordenados (se coluna de mês detectada) {-}{-}{-}}
\ControlFlowTok{if}\NormalTok{ (}\SpecialCharTok{!}\FunctionTok{is.na}\NormalTok{(mes\_col)) \{}
  \FunctionTok{cat}\NormalTok{(}\StringTok{"}\SpecialCharTok{\textbackslash{}n}\StringTok{Testes por mês (ordenados):}\SpecialCharTok{\textbackslash{}n}\StringTok{"}\NormalTok{)}
\NormalTok{  per\_month }\OtherTok{\textless{}{-}}\NormalTok{ df }\SpecialCharTok{\%\textgreater{}\%}
    \FunctionTok{group\_by}\NormalTok{(mes\_ord, grupo) }\SpecialCharTok{\%\textgreater{}\%}
    \FunctionTok{summarise}\NormalTok{(}\AttributeTok{nao =} \FunctionTok{sum}\NormalTok{(n\_nao, }\AttributeTok{na.rm =} \ConstantTok{TRUE}\NormalTok{), }\AttributeTok{sim =} \FunctionTok{sum}\NormalTok{(n\_sim, }\AttributeTok{na.rm =} \ConstantTok{TRUE}\NormalTok{), }\AttributeTok{.groups =} \StringTok{"drop"}\NormalTok{) }\SpecialCharTok{\%\textgreater{}\%}
    \CommentTok{\# garantir que todos grupos apareçam para cada mês}
    \FunctionTok{complete}\NormalTok{(mes\_ord, grupo, }\AttributeTok{fill =} \FunctionTok{list}\NormalTok{(}\AttributeTok{nao =} \DecValTok{0}\NormalTok{, }\AttributeTok{sim =} \DecValTok{0}\NormalTok{)) }\SpecialCharTok{\%\textgreater{}\%}
    \FunctionTok{arrange}\NormalTok{(mes\_ord, grupo)}
  \CommentTok{\# iterar mantendo a ordem de mes\_ord (fator)}
\NormalTok{  months\_to\_iter }\OtherTok{\textless{}{-}} \FunctionTok{unique}\NormalTok{(per\_month}\SpecialCharTok{$}\NormalTok{mes\_ord)}
  \ControlFlowTok{for}\NormalTok{ (m }\ControlFlowTok{in}\NormalTok{ months\_to\_iter) \{}
\NormalTok{    sub }\OtherTok{\textless{}{-}} \FunctionTok{filter}\NormalTok{(per\_month, mes\_ord }\SpecialCharTok{==}\NormalTok{ m)}
    \CommentTok{\# montar matriz 2x2 Controle vs Tratamento}
\NormalTok{    row\_ctrl }\OtherTok{\textless{}{-}}\NormalTok{ sub }\SpecialCharTok{\%\textgreater{}\%} \FunctionTok{filter}\NormalTok{(grupo }\SpecialCharTok{==} \StringTok{"Controle"}\NormalTok{)}
\NormalTok{    row\_trt  }\OtherTok{\textless{}{-}}\NormalTok{ sub }\SpecialCharTok{\%\textgreater{}\%} \FunctionTok{filter}\NormalTok{(grupo }\SpecialCharTok{==} \StringTok{"Tratamento"}\NormalTok{)}
\NormalTok{    mtx }\OtherTok{\textless{}{-}} \FunctionTok{matrix}\NormalTok{(}\FunctionTok{c}\NormalTok{(row\_ctrl}\SpecialCharTok{$}\NormalTok{nao, row\_ctrl}\SpecialCharTok{$}\NormalTok{sim, row\_trt}\SpecialCharTok{$}\NormalTok{nao, row\_trt}\SpecialCharTok{$}\NormalTok{sim), }\AttributeTok{nrow =} \DecValTok{2}\NormalTok{, }\AttributeTok{byrow =} \ConstantTok{TRUE}\NormalTok{)}
    \CommentTok{\# escolher teste apropriado}
\NormalTok{    tst }\OtherTok{\textless{}{-}} \FunctionTok{tryCatch}\NormalTok{(}\FunctionTok{chisq.test}\NormalTok{(mtx, }\AttributeTok{correct =} \ConstantTok{FALSE}\NormalTok{), }\AttributeTok{error =} \ControlFlowTok{function}\NormalTok{(e) }\ConstantTok{NULL}\NormalTok{)}
    \ControlFlowTok{if}\NormalTok{ (}\FunctionTok{is.null}\NormalTok{(tst) }\SpecialCharTok{||} \FunctionTok{any}\NormalTok{(}\FunctionTok{suppressWarnings}\NormalTok{(tst}\SpecialCharTok{$}\NormalTok{expected) }\SpecialCharTok{\textless{}} \DecValTok{5}\NormalTok{)) \{}
\NormalTok{      tstf }\OtherTok{\textless{}{-}} \FunctionTok{fisher.test}\NormalTok{(mtx)}
      \FunctionTok{cat}\NormalTok{(}\FunctionTok{sprintf}\NormalTok{(}\StringTok{"\%{-}10s: Fisher p = \%.4g (Controle: \%d/\%d, Tratamento: \%d/\%d)}\SpecialCharTok{\textbackslash{}n}\StringTok{"}\NormalTok{,}
                  \FunctionTok{as.character}\NormalTok{(m), tstf}\SpecialCharTok{$}\NormalTok{p.value,}
\NormalTok{                  row\_ctrl}\SpecialCharTok{$}\NormalTok{nao, row\_ctrl}\SpecialCharTok{$}\NormalTok{nao }\SpecialCharTok{+}\NormalTok{ row\_ctrl}\SpecialCharTok{$}\NormalTok{sim,}
\NormalTok{                  row\_trt}\SpecialCharTok{$}\NormalTok{nao, row\_trt}\SpecialCharTok{$}\NormalTok{nao }\SpecialCharTok{+}\NormalTok{ row\_trt}\SpecialCharTok{$}\NormalTok{sim))}
\NormalTok{    \} }\ControlFlowTok{else}\NormalTok{ \{}
      \FunctionTok{cat}\NormalTok{(}\FunctionTok{sprintf}\NormalTok{(}\StringTok{"\%{-}10s: chi2 = \%.2f, p = \%.4g (Controle: \%d/\%d, Tratamento: \%d/\%d)}\SpecialCharTok{\textbackslash{}n}\StringTok{"}\NormalTok{,}
                  \FunctionTok{as.character}\NormalTok{(m), }\FunctionTok{as.numeric}\NormalTok{(tst}\SpecialCharTok{$}\NormalTok{statistic), }\FunctionTok{as.numeric}\NormalTok{(tst}\SpecialCharTok{$}\NormalTok{p.value),}
\NormalTok{                  row\_ctrl}\SpecialCharTok{$}\NormalTok{nao, row\_ctrl}\SpecialCharTok{$}\NormalTok{nao }\SpecialCharTok{+}\NormalTok{ row\_ctrl}\SpecialCharTok{$}\NormalTok{sim,}
\NormalTok{                  row\_trt}\SpecialCharTok{$}\NormalTok{nao, row\_trt}\SpecialCharTok{$}\NormalTok{nao }\SpecialCharTok{+}\NormalTok{ row\_trt}\SpecialCharTok{$}\NormalTok{sim))}
\NormalTok{    \}}
\NormalTok{  \}}
\NormalTok{\}}

\FunctionTok{cat}\NormalTok{(}\StringTok{"}\SpecialCharTok{\textbackslash{}n}\StringTok{Análise concluída.}\SpecialCharTok{\textbackslash{}n}\StringTok{"}\NormalTok{)}
\InformationTok{\textasciigrave{}\textasciigrave{}\textasciigrave{}}
\end{Highlighting}
\end{Shaded}

\begin{verbatim}
Tabela agregada (linhas = Grupo; colunas = Nao/Sim):
           Nao Sim
Controle   167 138
Tratamento  84 270

Tabela com totais marginais:
           Nao Sim Total
Controle   167 138   305
Tratamento  84 270   354
Total      251 408   659

Observar que o total de 659 observações confere
com o tamanho da amostra reportado: n = 659


O Grupo de Controle confere com o reportado (n_GC = 305 audiências).


E o Grupo de Experimental confere c/o reportado (n_GT = 354 audiências).

Marginal — por grupo: controle x tratamento (contagem):
  Controle Tratamento 
       305        354 

Marginal — por acordo: sim ou não (contagem):
Nao Sim 
251 408 

Marginal — por grupo: controle x tratamento (proporção % do total):
  Controle Tratamento 
     46.28      53.72 

Marginal — por acordo: sim ou não (proporção % do total):
  Nao   Sim 
38.09 61.91 

Resultado do chi-squared test (Pearson):

    Pearson's Chi-squared test

data:  tab
X-squared = 66.878, df = 1, p-value = 2.888e-16


Phi (efeito para 2x2) = 0.3186

Testes por mês (ordenados):
Abril     : chi2 = 9.89, p = 0.001662 (Controle: 35/66, Tratamento: 20/74)
Maio      : chi2 = 11.56, p = 0.0006749 (Controle: 17/33, Tratamento: 7/45)
Junho     : chi2 = 0.78, p = 0.3757 (Controle: 18/31, Tratamento: 17/36)
Julho     : chi2 = 18.25, p = 1.934e-05 (Controle: 20/34, Tratamento: 4/37)
Agosto    : chi2 = 4.20, p = 0.04053 (Controle: 19/39, Tratamento: 8/32)
Setembro  : chi2 = 3.05, p = 0.08092 (Controle: 24/50, Tratamento: 13/43)
Outubro   : chi2 = 3.74, p = 0.05302 (Controle: 8/14, Tratamento: 3/14)
Novembro  : chi2 = 19.60, p = 9.534e-06 (Controle: 16/24, Tratamento: 10/59)
Dezembro  : chi2 = 9.33, p = 0.00225 (Controle: 10/14, Tratamento: 2/14)

Análise concluída.
\end{verbatim}

\subsection{Gráfico barras agregado}\label{gruxe1fico-barras-agregado}

Gráfico de barras empilhadas dos dados agregados com as proporções
indicadas e o valor-p também.

\begin{Shaded}
\begin{Highlighting}[numbers=left,,]
\InformationTok{\textasciigrave{}\textasciigrave{}\textasciigrave{}\{r\}}
\CommentTok{\# Gráfico de barras empilhadas com proporções e valor{-}p agregado}
\CommentTok{\# {-} Detecta/analisa dados agregados no ambiente ou arquivo "tabela\_mensal\_chisq.csv"}
\CommentTok{\# {-} Calcula chi{-}squared (ou Fisher se necessário) e anota p{-}valor no gráfico}
\CommentTok{\# {-} Rótulos percentuais dentro dos segmentos (omitidos se muito pequenos)}

\CommentTok{\# Pacotes necessários}
\ControlFlowTok{if}\NormalTok{ (}\SpecialCharTok{!}\FunctionTok{requireNamespace}\NormalTok{(}\StringTok{"ggplot2"}\NormalTok{, }\AttributeTok{quietly =} \ConstantTok{TRUE}\NormalTok{)) }\FunctionTok{install.packages}\NormalTok{(}\StringTok{"ggplot2"}\NormalTok{)}
\ControlFlowTok{if}\NormalTok{ (}\SpecialCharTok{!}\FunctionTok{requireNamespace}\NormalTok{(}\StringTok{"dplyr"}\NormalTok{, }\AttributeTok{quietly =} \ConstantTok{TRUE}\NormalTok{)) }\FunctionTok{install.packages}\NormalTok{(}\StringTok{"dplyr"}\NormalTok{)}
\ControlFlowTok{if}\NormalTok{ (}\SpecialCharTok{!}\FunctionTok{requireNamespace}\NormalTok{(}\StringTok{"scales"}\NormalTok{, }\AttributeTok{quietly =} \ConstantTok{TRUE}\NormalTok{)) }\FunctionTok{install.packages}\NormalTok{(}\StringTok{"scales"}\NormalTok{)}
\FunctionTok{library}\NormalTok{(ggplot2)}
\FunctionTok{library}\NormalTok{(dplyr)}
\FunctionTok{library}\NormalTok{(scales)}

\CommentTok{\# montar tabela 2x2 (linhas = grupo; colunas = nao/sim)}
\NormalTok{tab\_mat }\OtherTok{\textless{}{-}}\NormalTok{ tab}
\NormalTok{agg\_df  }\OtherTok{\textless{}{-}} \FunctionTok{as.data.frame.matrix}\NormalTok{(tab)}
\NormalTok{agg\_df}\SpecialCharTok{$}\NormalTok{grupo }\OtherTok{\textless{}{-}} \FunctionTok{rownames}\NormalTok{(agg\_df)}
\FunctionTok{names}\NormalTok{(agg\_df) }\OtherTok{\textless{}{-}} \FunctionTok{tolower}\NormalTok{( }\FunctionTok{names}\NormalTok{(agg\_df) )}
\NormalTok{agg\_df }\OtherTok{\textless{}{-}}\NormalTok{ agg\_df }\SpecialCharTok{\%\textgreater{}\%} \FunctionTok{select}\NormalTok{(grupo, }\FunctionTok{everything}\NormalTok{())}


\CommentTok{\# {-}{-}{-} Teste estatístico (chi{-}square ou Fisher se necessário) {-}{-}{-}}
\NormalTok{use\_fisher }\OtherTok{\textless{}{-}} \ConstantTok{FALSE}
\NormalTok{test\_res }\OtherTok{\textless{}{-}} \FunctionTok{tryCatch}\NormalTok{(}\FunctionTok{chisq.test}\NormalTok{(tab\_mat, }\AttributeTok{correct =} \ConstantTok{FALSE}\NormalTok{), }\AttributeTok{error =} \ControlFlowTok{function}\NormalTok{(e) e)}
\ControlFlowTok{if}\NormalTok{ (}\FunctionTok{inherits}\NormalTok{(test\_res, }\StringTok{"error"}\NormalTok{)) \{}
\NormalTok{  use\_fisher }\OtherTok{\textless{}{-}} \ConstantTok{TRUE}
\NormalTok{\} }\ControlFlowTok{else}\NormalTok{ \{}
\NormalTok{  expc }\OtherTok{\textless{}{-}} \FunctionTok{suppressWarnings}\NormalTok{(test\_res}\SpecialCharTok{$}\NormalTok{expected)}
  \ControlFlowTok{if}\NormalTok{ (}\FunctionTok{any}\NormalTok{(expc }\SpecialCharTok{\textless{}} \DecValTok{5}\NormalTok{)) use\_fisher }\OtherTok{\textless{}{-}} \ConstantTok{TRUE}
\NormalTok{\}}

\ControlFlowTok{if}\NormalTok{ (use\_fisher) \{}
\NormalTok{  test\_f }\OtherTok{\textless{}{-}} \FunctionTok{fisher.test}\NormalTok{(tab\_mat)}
\NormalTok{  pval   }\OtherTok{\textless{}{-}} \FunctionTok{as.numeric}\NormalTok{(test\_f}\SpecialCharTok{$}\NormalTok{p.value)}
\NormalTok{  test\_label }\OtherTok{\textless{}{-}} \FunctionTok{sprintf}\NormalTok{(}\StringTok{"Fisher exact (p = \%s)"}\NormalTok{, }\FunctionTok{ifelse}\NormalTok{(pval }\SpecialCharTok{\textless{}} \FloatTok{0.001}\NormalTok{, }\StringTok{"\textless{}0.001"}\NormalTok{, }\FunctionTok{format.pval}\NormalTok{(pval, }\AttributeTok{digits =} \DecValTok{3}\NormalTok{)))}
\NormalTok{\} }\ControlFlowTok{else}\NormalTok{ \{}
\NormalTok{  test\_chi }\OtherTok{\textless{}{-}} \FunctionTok{chisq.test}\NormalTok{(tab\_mat, }\AttributeTok{correct =} \ConstantTok{FALSE}\NormalTok{)}
\NormalTok{  pval     }\OtherTok{\textless{}{-}} \FunctionTok{as.numeric}\NormalTok{(test\_chi}\SpecialCharTok{$}\NormalTok{p.value)}
\NormalTok{  test\_label }\OtherTok{\textless{}{-}} \FunctionTok{sprintf}\NormalTok{(}\StringTok{"Chi{-}square (p = \%s, X² = \%.2f)"}\NormalTok{, }\FunctionTok{ifelse}\NormalTok{(pval }\SpecialCharTok{\textless{}} \FloatTok{0.001}\NormalTok{, }\StringTok{"\textless{}0.001"}\NormalTok{, }\FunctionTok{format.pval}\NormalTok{(pval, }\AttributeTok{digits =} \DecValTok{3}\NormalTok{)), }\FunctionTok{as.numeric}\NormalTok{(test\_chi}\SpecialCharTok{$}\NormalTok{statistic))}
\NormalTok{\}}

\CommentTok{\# {-}{-}{-} Preparar dados para plotagem (formato long) {-}{-}{-}}
\NormalTok{plot\_df }\OtherTok{\textless{}{-}}\NormalTok{ agg\_df }\SpecialCharTok{\%\textgreater{}\%}
  \FunctionTok{pivot\_longer}\NormalTok{(}\AttributeTok{cols =} \FunctionTok{c}\NormalTok{(}\StringTok{"nao"}\NormalTok{,}\StringTok{"sim"}\NormalTok{),}
               \AttributeTok{names\_to =} \StringTok{"resposta"}\NormalTok{,}
               \AttributeTok{values\_to =} \StringTok{"contagem"}\NormalTok{) }\SpecialCharTok{\%\textgreater{}\%}
  \FunctionTok{group\_by}\NormalTok{(grupo) }\SpecialCharTok{\%\textgreater{}\%}
  \FunctionTok{mutate}\NormalTok{(}\AttributeTok{total =} \FunctionTok{sum}\NormalTok{(contagem),}
         \AttributeTok{prop =}\NormalTok{ contagem }\SpecialCharTok{/}\NormalTok{ total,}
         \AttributeTok{pct\_label =} \FunctionTok{ifelse}\NormalTok{(prop }\SpecialCharTok{\textgreater{}=} \FloatTok{0.03}\NormalTok{, }\FunctionTok{paste0}\NormalTok{(}\FunctionTok{round}\NormalTok{(}\DecValTok{100}\SpecialCharTok{*}\NormalTok{prop,}\DecValTok{1}\NormalTok{), }\StringTok{"\%"}\NormalTok{), }\StringTok{""}\NormalTok{)) }\SpecialCharTok{\%\textgreater{}\%}
  \FunctionTok{ungroup}\NormalTok{()}

\CommentTok{\# ordem das categorias na pilha (opcional: sim sobre nao)}
\NormalTok{plot\_df}\SpecialCharTok{$}\NormalTok{resposta }\OtherTok{\textless{}{-}} \FunctionTok{factor}\NormalTok{(plot\_df}\SpecialCharTok{$}\NormalTok{resposta, }\AttributeTok{levels =} \FunctionTok{c}\NormalTok{(}\StringTok{"nao"}\NormalTok{,}\StringTok{"sim"}\NormalTok{))}

\CommentTok{\# {-}{-}{-} Gráfico com título da legenda "Acordo" {-}{-}{-}}
\NormalTok{p }\OtherTok{\textless{}{-}} \FunctionTok{ggplot}\NormalTok{(plot\_df, }\FunctionTok{aes}\NormalTok{(}\AttributeTok{x =}\NormalTok{ grupo, }\AttributeTok{y =}\NormalTok{ prop, }\AttributeTok{fill =}\NormalTok{ resposta)) }\SpecialCharTok{+}
  \FunctionTok{geom\_col}\NormalTok{(}\AttributeTok{colour =} \StringTok{"grey30"}\NormalTok{, }\AttributeTok{width =} \FloatTok{0.6}\NormalTok{) }\SpecialCharTok{+}
  \FunctionTok{geom\_text}\NormalTok{(}\FunctionTok{aes}\NormalTok{(}\AttributeTok{label =}\NormalTok{ pct\_label),}
            \AttributeTok{position =} \FunctionTok{position\_stack}\NormalTok{(}\AttributeTok{vjust =} \FloatTok{0.5}\NormalTok{),}
            \AttributeTok{colour =} \StringTok{"white"}\NormalTok{, }\AttributeTok{size =} \DecValTok{3}\NormalTok{, }\AttributeTok{fontface =} \StringTok{"bold"}\NormalTok{) }\SpecialCharTok{+}
  \CommentTok{\# definir quebras do eixo y de 0 a 1 (0\% a 100\%)}
  \CommentTok{\# de 0.1 em 0.1 =\textgreater{} mostra 0\%,10\%,...,100\%}
  \FunctionTok{scale\_y\_continuous}\NormalTok{(}\AttributeTok{labels =} \FunctionTok{percent\_format}\NormalTok{(}\AttributeTok{accuracy =} \DecValTok{1}\NormalTok{),}
                   \AttributeTok{breaks =} \FunctionTok{seq}\NormalTok{(}\DecValTok{0}\NormalTok{, }\DecValTok{1}\NormalTok{, }\AttributeTok{by =} \FloatTok{0.1}\NormalTok{)) }\SpecialCharTok{+}
  \FunctionTok{scale\_fill\_manual}\NormalTok{(}\AttributeTok{name =} \StringTok{"Acordo"}\NormalTok{, }\CommentTok{\# \textless{}{-}{-} título da legenda definido aqui}
                  \AttributeTok{values =} \FunctionTok{c}\NormalTok{(}\StringTok{"nao"} \OtherTok{=} \StringTok{"\#bdbdbd"}\NormalTok{, }\StringTok{"sim"} \OtherTok{=} \StringTok{"\#4E79A7"}\NormalTok{),}
                  \AttributeTok{labels =} \FunctionTok{c}\NormalTok{(}\StringTok{"Não"}\NormalTok{,}\StringTok{"Sim"}\NormalTok{)) }\SpecialCharTok{+}
  \FunctionTok{labs}\NormalTok{(}\AttributeTok{title =} \StringTok{"Composição por resposta ao acordo (agregado)"}\NormalTok{,}
       \AttributeTok{subtitle =}\NormalTok{ test\_label,}
       \AttributeTok{x =} \StringTok{"Grupo"}\NormalTok{,}
       \AttributeTok{y =} \StringTok{"Proporção dentro do grupo"}\NormalTok{,}
       \AttributeTok{fill =} \StringTok{""}\NormalTok{) }\SpecialCharTok{+}
  \FunctionTok{theme\_minimal}\NormalTok{() }\SpecialCharTok{+}
  \FunctionTok{theme}\NormalTok{(}\AttributeTok{plot.title =} \FunctionTok{element\_text}\NormalTok{(}\AttributeTok{hjust =} \FloatTok{0.5}\NormalTok{),}
        \AttributeTok{plot.subtitle =} \FunctionTok{element\_text}\NormalTok{(}\AttributeTok{hjust =} \FloatTok{0.5}\NormalTok{, }\AttributeTok{face =} \StringTok{"italic"}\NormalTok{))}

\CommentTok{\# expandir espaço superior para anotar p{-}valor num lugar destacado}
\NormalTok{p }\OtherTok{\textless{}{-}}\NormalTok{ p }\SpecialCharTok{+} \FunctionTok{coord\_cartesian}\NormalTok{(}\AttributeTok{ylim =} \FunctionTok{c}\NormalTok{(}\DecValTok{0}\NormalTok{, }\FloatTok{1.09}\NormalTok{))}

\CommentTok{\# adicionar texto com p{-}valor acima das barras (centro)}
\NormalTok{p }\OtherTok{\textless{}{-}}\NormalTok{ p }\SpecialCharTok{+} \FunctionTok{annotate}\NormalTok{(}\StringTok{"text"}\NormalTok{,}
                  \AttributeTok{x =} \FunctionTok{mean}\NormalTok{(}\FunctionTok{seq\_along}\NormalTok{(}\FunctionTok{unique}\NormalTok{(plot\_df}\SpecialCharTok{$}\NormalTok{grupo))),}
                  \AttributeTok{y =} \FloatTok{1.05}\NormalTok{,}
                  \AttributeTok{label =} \FunctionTok{paste0}\NormalTok{(}\StringTok{"Teste: "}\NormalTok{, test\_label),}
                  \AttributeTok{size =} \FloatTok{3.2}\NormalTok{)}

\CommentTok{\# exibir}
\FunctionTok{print}\NormalTok{(p)}

\CommentTok{\# (Opcional) salvar figura}
\CommentTok{\# ggsave("stacked\_bar\_aggregated\_pvalue.png", p, width = 7, height = 5, dpi = 300)}
\InformationTok{\textasciigrave{}\textasciigrave{}\textasciigrave{}}
\end{Highlighting}
\end{Shaded}

\pandocbounded{\includegraphics[keepaspectratio]{replicar-suco-uva-tab-graf_files/figure-pdf/unnamed-chunk-3-1.pdf}}

\subsection{Gráfico barras facetas}\label{gruxe1fico-barras-facetas}

O mesmo gráfico acima, um para cada mês, em facetas na ordem cronológica
dos mêses.

\begin{Shaded}
\begin{Highlighting}[numbers=left,,]
\InformationTok{\textasciigrave{}\textasciigrave{}\textasciigrave{}\{r\}}
\CommentTok{\# Faceted stacked bars por mês com p{-}valor correto por faceta}
\CommentTok{\# {-} calcula p{-}valor mês a mês (chi2 ou Fisher quando apropriado)}
\CommentTok{\# {-} anota cada faceta com o p{-}valor correspondente}
\CommentTok{\# {-} reduz tamanho dos valores do eixo y}
\CommentTok{\#}
\CommentTok{\# Requisitos: ggplot2, dplyr, tidyr, scales}

\ControlFlowTok{if}\NormalTok{ (}\SpecialCharTok{!}\FunctionTok{requireNamespace}\NormalTok{(}\StringTok{"ggplot2"}\NormalTok{, }\AttributeTok{quietly =} \ConstantTok{TRUE}\NormalTok{)) }\FunctionTok{install.packages}\NormalTok{(}\StringTok{"ggplot2"}\NormalTok{)}
\ControlFlowTok{if}\NormalTok{ (}\SpecialCharTok{!}\FunctionTok{requireNamespace}\NormalTok{(}\StringTok{"dplyr"}\NormalTok{, }\AttributeTok{quietly =} \ConstantTok{TRUE}\NormalTok{)) }\FunctionTok{install.packages}\NormalTok{(}\StringTok{"dplyr"}\NormalTok{)}
\ControlFlowTok{if}\NormalTok{ (}\SpecialCharTok{!}\FunctionTok{requireNamespace}\NormalTok{(}\StringTok{"tidyr"}\NormalTok{, }\AttributeTok{quietly =} \ConstantTok{TRUE}\NormalTok{)) }\FunctionTok{install.packages}\NormalTok{(}\StringTok{"tidyr"}\NormalTok{)}
\ControlFlowTok{if}\NormalTok{ (}\SpecialCharTok{!}\FunctionTok{requireNamespace}\NormalTok{(}\StringTok{"scales"}\NormalTok{, }\AttributeTok{quietly =} \ConstantTok{TRUE}\NormalTok{)) }\FunctionTok{install.packages}\NormalTok{(}\StringTok{"scales"}\NormalTok{)}
\FunctionTok{library}\NormalTok{(ggplot2); }\FunctionTok{library}\NormalTok{(dplyr); }\FunctionTok{library}\NormalTok{(tidyr); }\FunctionTok{library}\NormalTok{(scales)}

\CommentTok{\# {-}{-}{-} localizar/ler dados {-}{-}{-}}
\NormalTok{df\_raw }\OtherTok{\textless{}{-}}\NormalTok{ tabela\_final}

\CommentTok{\# {-}{-}{-} normalizar nomes de colunas {-}{-}{-}}
\FunctionTok{names}\NormalTok{(df\_raw) }\OtherTok{\textless{}{-}} \FunctionTok{tolower}\NormalTok{(}\FunctionTok{names}\NormalTok{(df\_raw))}
\FunctionTok{names}\NormalTok{(df\_raw) }\OtherTok{\textless{}{-}} \FunctionTok{gsub}\NormalTok{(}\StringTok{"[\^{}a{-}z0{-}9\_]"}\NormalTok{, }\StringTok{"\_"}\NormalTok{, }\FunctionTok{names}\NormalTok{(df\_raw))}

\CommentTok{\# identificar colunas}
\NormalTok{col\_mes  }\OtherTok{\textless{}{-}} \FunctionTok{intersect}\NormalTok{(}\FunctionTok{names}\NormalTok{(df\_raw), }\FunctionTok{c}\NormalTok{(}\StringTok{"mes"}\NormalTok{,}\StringTok{"month"}\NormalTok{,}\StringTok{"mês"}\NormalTok{,}\StringTok{"m"}\NormalTok{))}
\NormalTok{col\_grp  }\OtherTok{\textless{}{-}} \FunctionTok{intersect}\NormalTok{(}\FunctionTok{names}\NormalTok{(df\_raw), }\FunctionTok{c}\NormalTok{(}\StringTok{"grupo"}\NormalTok{,}\StringTok{"group"}\NormalTok{,}\StringTok{"tratamento"}\NormalTok{,}\StringTok{"treatment"}\NormalTok{))}
\NormalTok{col\_nao  }\OtherTok{\textless{}{-}} \FunctionTok{intersect}\NormalTok{(}\FunctionTok{names}\NormalTok{(df\_raw), }\FunctionTok{c}\NormalTok{(}\StringTok{"nao"}\NormalTok{,}\StringTok{"no"}\NormalTok{,}\StringTok{"n\_nao"}\NormalTok{,}\StringTok{"n\_nao"}\NormalTok{))}
\NormalTok{col\_sim  }\OtherTok{\textless{}{-}} \FunctionTok{intersect}\NormalTok{(}\FunctionTok{names}\NormalTok{(df\_raw), }\FunctionTok{c}\NormalTok{(}\StringTok{"sim"}\NormalTok{,}\StringTok{"yes"}\NormalTok{,}\StringTok{"y"}\NormalTok{))}

\ControlFlowTok{if}\NormalTok{ (}\FunctionTok{length}\NormalTok{(col\_mes)}\SpecialCharTok{==}\DecValTok{0}\NormalTok{) }\FunctionTok{stop}\NormalTok{(}\StringTok{"Coluna de mês não encontrada (procure por \textquotesingle{}Mes\textquotesingle{}/\textquotesingle{}month\textquotesingle{})."}\NormalTok{)}
\ControlFlowTok{if}\NormalTok{ (}\FunctionTok{length}\NormalTok{(col\_grp)}\SpecialCharTok{==}\DecValTok{0}\NormalTok{) }\FunctionTok{stop}\NormalTok{(}\StringTok{"Coluna de grupo não encontrada (procure por \textquotesingle{}Grupo\textquotesingle{})."}\NormalTok{)}
\ControlFlowTok{if}\NormalTok{ (}\FunctionTok{length}\NormalTok{(col\_nao)}\SpecialCharTok{==}\DecValTok{0} \SpecialCharTok{||} \FunctionTok{length}\NormalTok{(col\_sim)}\SpecialCharTok{==}\DecValTok{0}\NormalTok{) }\FunctionTok{stop}\NormalTok{(}\StringTok{"Colunas de contagem \textquotesingle{}Nao\textquotesingle{} e \textquotesingle{}Sim\textquotesingle{} não encontradas."}\NormalTok{)}

\NormalTok{mes\_col }\OtherTok{\textless{}{-}}\NormalTok{ col\_mes[}\DecValTok{1}\NormalTok{]; grp\_col }\OtherTok{\textless{}{-}}\NormalTok{ col\_grp[}\DecValTok{1}\NormalTok{]; nao\_col }\OtherTok{\textless{}{-}}\NormalTok{ col\_nao[}\DecValTok{1}\NormalTok{]; sim\_col }\OtherTok{\textless{}{-}}\NormalTok{ col\_sim[}\DecValTok{1}\NormalTok{]}

\CommentTok{\# {-}{-}{-} padronizar meses e grupos {-}{-}{-}}
\NormalTok{month\_levels }\OtherTok{\textless{}{-}} \FunctionTok{c}\NormalTok{(}\StringTok{"Abril"}\NormalTok{,}\StringTok{"Maio"}\NormalTok{,}\StringTok{"Junho"}\NormalTok{,}\StringTok{"Julho"}\NormalTok{,}\StringTok{"Agosto"}\NormalTok{,}\StringTok{"Setembro"}\NormalTok{,}
                  \StringTok{"Outubro"}\NormalTok{,}\StringTok{"Novembro"}\NormalTok{,}\StringTok{"Dezembro"}\NormalTok{)}
\NormalTok{map\_month }\OtherTok{\textless{}{-}} \ControlFlowTok{function}\NormalTok{(x) \{}
\NormalTok{  x0 }\OtherTok{\textless{}{-}} \FunctionTok{trimws}\NormalTok{(}\FunctionTok{as.character}\NormalTok{(x))}
\NormalTok{  idx }\OtherTok{\textless{}{-}} \FunctionTok{match}\NormalTok{(}\FunctionTok{tolower}\NormalTok{(x0), }\FunctionTok{tolower}\NormalTok{(month\_levels))}
  \ControlFlowTok{if}\NormalTok{ (}\SpecialCharTok{!}\FunctionTok{is.na}\NormalTok{(idx)) }\FunctionTok{return}\NormalTok{(month\_levels[idx])}
\NormalTok{  x\_clean }\OtherTok{\textless{}{-}} \FunctionTok{iconv}\NormalTok{(x0, }\AttributeTok{to =} \StringTok{"ASCII//TRANSLIT"}\NormalTok{)}
\NormalTok{  idx2 }\OtherTok{\textless{}{-}} \FunctionTok{match}\NormalTok{(}\FunctionTok{tolower}\NormalTok{(x\_clean), }\FunctionTok{tolower}\NormalTok{(}\FunctionTok{iconv}\NormalTok{(month\_levels, }\AttributeTok{to =} \StringTok{"ASCII//TRANSLIT"}\NormalTok{)))}
  \ControlFlowTok{if}\NormalTok{ (}\SpecialCharTok{!}\FunctionTok{is.na}\NormalTok{(idx2)) }\FunctionTok{return}\NormalTok{(month\_levels[idx2])}
  \FunctionTok{return}\NormalTok{(x0)}
\NormalTok{\}}

\NormalTok{df }\OtherTok{\textless{}{-}}\NormalTok{ df\_raw }\SpecialCharTok{\%\textgreater{}\%}
  \FunctionTok{mutate}\NormalTok{(}
    \AttributeTok{mes\_raw =}\NormalTok{ .data[[mes\_col]],}
    \AttributeTok{mes\_mapped =} \FunctionTok{vapply}\NormalTok{(mes\_raw, map\_month, }\FunctionTok{character}\NormalTok{(}\DecValTok{1}\NormalTok{)),}
    \AttributeTok{mes\_ord =} \FunctionTok{factor}\NormalTok{(mes\_mapped, }\AttributeTok{levels =}\NormalTok{ month\_levels),}
    \AttributeTok{grupo\_raw =} \FunctionTok{as.character}\NormalTok{(.data[[grp\_col]]),}
    \AttributeTok{grupo =} \FunctionTok{ifelse}\NormalTok{(}\FunctionTok{tolower}\NormalTok{(}\FunctionTok{trimws}\NormalTok{(grupo\_raw)) }\SpecialCharTok{\%in\%} \FunctionTok{c}\NormalTok{(}\StringTok{"controle"}\NormalTok{,}\StringTok{"control"}\NormalTok{), }\StringTok{"Controle"}\NormalTok{, }\StringTok{"Tratamento"}\NormalTok{),}
    \AttributeTok{nao =} \FunctionTok{as.numeric}\NormalTok{(.data[[nao\_col]]),}
    \AttributeTok{sim =} \FunctionTok{as.numeric}\NormalTok{(.data[[sim\_col]])}
\NormalTok{  )}

\CommentTok{\# {-}{-}{-} agregar por mês e grupo e garantir presença de ambos os grupos {-}{-}{-}}
\NormalTok{per\_month }\OtherTok{\textless{}{-}}\NormalTok{ df }\SpecialCharTok{\%\textgreater{}\%}
  \FunctionTok{group\_by}\NormalTok{(mes\_ord, grupo) }\SpecialCharTok{\%\textgreater{}\%}
  \FunctionTok{summarise}\NormalTok{(}\AttributeTok{nao =} \FunctionTok{sum}\NormalTok{(nao, }\AttributeTok{na.rm =} \ConstantTok{TRUE}\NormalTok{),}
            \AttributeTok{sim =} \FunctionTok{sum}\NormalTok{(sim, }\AttributeTok{na.rm =} \ConstantTok{TRUE}\NormalTok{), }\AttributeTok{.groups =} \StringTok{"drop"}\NormalTok{) }\SpecialCharTok{\%\textgreater{}\%}
  \FunctionTok{filter}\NormalTok{(}\SpecialCharTok{!}\FunctionTok{is.na}\NormalTok{(mes\_ord)) }\SpecialCharTok{\%\textgreater{}\%}
  \FunctionTok{ungroup}\NormalTok{()}

\CommentTok{\# garantir linhas para ambos grupos em cada mês (preencher com zeros)}
\NormalTok{per\_month }\OtherTok{\textless{}{-}}\NormalTok{ per\_month }\SpecialCharTok{\%\textgreater{}\%}
  \FunctionTok{complete}\NormalTok{(mes\_ord, }\AttributeTok{grupo =} \FunctionTok{c}\NormalTok{(}\StringTok{"Controle"}\NormalTok{, }\StringTok{"Tratamento"}\NormalTok{),}
           \AttributeTok{fill =} \FunctionTok{list}\NormalTok{(}\AttributeTok{nao =} \DecValTok{0}\NormalTok{, }\AttributeTok{sim =} \DecValTok{0}\NormalTok{)) }\SpecialCharTok{\%\textgreater{}\%}
  \FunctionTok{arrange}\NormalTok{(mes\_ord, grupo)}

\CommentTok{\# {-}{-}{-} calcular p{-}valor corretamente para cada mês {-}{-}{-}}
\NormalTok{months }\OtherTok{\textless{}{-}} \FunctionTok{unique}\NormalTok{(per\_month}\SpecialCharTok{$}\NormalTok{mes\_ord)}
\NormalTok{p\_list }\OtherTok{\textless{}{-}} \FunctionTok{lapply}\NormalTok{(months, }\ControlFlowTok{function}\NormalTok{(m) \{}
\NormalTok{  sub }\OtherTok{\textless{}{-}} \FunctionTok{filter}\NormalTok{(per\_month, mes\_ord }\SpecialCharTok{==}\NormalTok{ m) }\SpecialCharTok{\%\textgreater{}\%} \FunctionTok{arrange}\NormalTok{(}\FunctionTok{match}\NormalTok{(grupo, }\FunctionTok{c}\NormalTok{(}\StringTok{"Controle"}\NormalTok{,}\StringTok{"Tratamento"}\NormalTok{)))}
  \CommentTok{\# montar matriz 2x2}
\NormalTok{  mtx }\OtherTok{\textless{}{-}} \FunctionTok{matrix}\NormalTok{(}\FunctionTok{c}\NormalTok{(sub}\SpecialCharTok{$}\NormalTok{nao[}\DecValTok{1}\NormalTok{], sub}\SpecialCharTok{$}\NormalTok{sim[}\DecValTok{1}\NormalTok{], sub}\SpecialCharTok{$}\NormalTok{nao[}\DecValTok{2}\NormalTok{], sub}\SpecialCharTok{$}\NormalTok{sim[}\DecValTok{2}\NormalTok{]), }\AttributeTok{nrow =} \DecValTok{2}\NormalTok{, }\AttributeTok{byrow =} \ConstantTok{TRUE}\NormalTok{)}
  \CommentTok{\# tentar chi2; se erro ou expected \textless{} 5 usar Fisher}
\NormalTok{  tst }\OtherTok{\textless{}{-}} \FunctionTok{tryCatch}\NormalTok{(}\FunctionTok{chisq.test}\NormalTok{(mtx, }\AttributeTok{correct =} \ConstantTok{FALSE}\NormalTok{), }\AttributeTok{error =} \ControlFlowTok{function}\NormalTok{(e) }\ConstantTok{NULL}\NormalTok{)}
\NormalTok{  pval }\OtherTok{\textless{}{-}} \ConstantTok{NA\_real\_}
\NormalTok{  test\_type }\OtherTok{\textless{}{-}} \ConstantTok{NA\_character\_}
  \ControlFlowTok{if}\NormalTok{ (}\FunctionTok{is.null}\NormalTok{(tst)) \{}
\NormalTok{    f }\OtherTok{\textless{}{-}} \FunctionTok{fisher.test}\NormalTok{(mtx)}
\NormalTok{    pval }\OtherTok{\textless{}{-}}\NormalTok{ f}\SpecialCharTok{$}\NormalTok{p.value; test\_type }\OtherTok{\textless{}{-}} \StringTok{"Fisher"}
\NormalTok{  \} }\ControlFlowTok{else}\NormalTok{ \{}
\NormalTok{    expc }\OtherTok{\textless{}{-}} \FunctionTok{suppressWarnings}\NormalTok{(tst}\SpecialCharTok{$}\NormalTok{expected)}
    \ControlFlowTok{if}\NormalTok{ (}\FunctionTok{any}\NormalTok{(expc }\SpecialCharTok{\textless{}} \DecValTok{5}\NormalTok{)) \{}
\NormalTok{      f }\OtherTok{\textless{}{-}} \FunctionTok{fisher.test}\NormalTok{(mtx)}
\NormalTok{      pval }\OtherTok{\textless{}{-}}\NormalTok{ f}\SpecialCharTok{$}\NormalTok{p.value; test\_type }\OtherTok{\textless{}{-}} \StringTok{"Fisher"}
\NormalTok{    \} }\ControlFlowTok{else}\NormalTok{ \{}
\NormalTok{      pval }\OtherTok{\textless{}{-}}\NormalTok{ tst}\SpecialCharTok{$}\NormalTok{p.value; test\_type }\OtherTok{\textless{}{-}} \StringTok{"Chi{-}square"}
\NormalTok{    \}}
\NormalTok{  \}}
\NormalTok{  tibble}\SpecialCharTok{::}\FunctionTok{tibble}\NormalTok{(}\AttributeTok{mes\_ord =}\NormalTok{ m, }\AttributeTok{p\_value =}\NormalTok{ pval, }\AttributeTok{test =}\NormalTok{ test\_type)}
\NormalTok{\})}
\NormalTok{p\_by\_month }\OtherTok{\textless{}{-}} \FunctionTok{bind\_rows}\NormalTok{(p\_list)}

\CommentTok{\# Função utilitária: formata qui{-}quadrado + estrelas de significância}
\NormalTok{stars\_from\_p }\OtherTok{\textless{}{-}} \ControlFlowTok{function}\NormalTok{(p) \{}
  \ControlFlowTok{if}\NormalTok{ ( }\FunctionTok{is.na}\NormalTok{(p) )  }\FunctionTok{return}\NormalTok{(}\StringTok{""}\NormalTok{)}
  \CommentTok{\# caractere de ponto em posição elevada (dot above)}
\NormalTok{  dot }\OtherTok{\textless{}{-}} \StringTok{"\textbackslash{}u02D9"}  \CommentTok{\# \textquotesingle{}˙\textquotesingle{} (dot above) — aparece como pequeno ponto superiorizado}
  \ControlFlowTok{if}\NormalTok{ (p }\SpecialCharTok{\textless{}} \FloatTok{0.001}\NormalTok{) }\FunctionTok{return}\NormalTok{(}\StringTok{"\textless{}0.001"}\NormalTok{) }\CommentTok{\# teste significativo a 99.9\% de confiança}
  \ControlFlowTok{if}\NormalTok{ (p }\SpecialCharTok{\textless{}} \FloatTok{0.01}\NormalTok{) }\FunctionTok{return}\NormalTok{(}\StringTok{"**"}\NormalTok{)      }\CommentTok{\# teste significativo a 99\% de confiança}
  \ControlFlowTok{if}\NormalTok{ (p }\SpecialCharTok{\textless{}} \FloatTok{0.05}\NormalTok{) }\FunctionTok{return}\NormalTok{(}\StringTok{"*"}\NormalTok{)       }\CommentTok{\# teste significativo a 95\% de confiança}
  \ControlFlowTok{if}\NormalTok{ (p }\SpecialCharTok{\textless{}} \FloatTok{0.10}\NormalTok{) }\FunctionTok{return}\NormalTok{(dot)       }\CommentTok{\# teste significativo a 90\% de confiança}
  \FunctionTok{return}\NormalTok{(}\StringTok{""}\NormalTok{) }
\NormalTok{\}}

\CommentTok{\# label formatado}
\CommentTok{\# sapply(p\_by\_month$p\_value, FUN = stars\_from\_p)}

\NormalTok{p\_by\_month }\OtherTok{\textless{}{-}}\NormalTok{ p\_by\_month }\SpecialCharTok{\%\textgreater{}\%}
  \FunctionTok{mutate}\NormalTok{(}\AttributeTok{p\_label =} \FunctionTok{format.pval}\NormalTok{(p\_value, }\AttributeTok{digits =} \DecValTok{3}\NormalTok{),}
         \AttributeTok{label =} \FunctionTok{paste0}\NormalTok{(test, }\StringTok{": p = "}\NormalTok{, p\_label,}
                        \FunctionTok{sapply}\NormalTok{(p\_by\_month}\SpecialCharTok{$}\NormalTok{p\_value, }\AttributeTok{FUN =}\NormalTok{ stars\_from\_p)))}

\CommentTok{\# {-}{-}{-} preparar dados para plotagem {-}{-}{-}}
\NormalTok{plot\_df }\OtherTok{\textless{}{-}}\NormalTok{ per\_month }\SpecialCharTok{\%\textgreater{}\%}
  \FunctionTok{pivot\_longer}\NormalTok{(}\AttributeTok{cols =} \FunctionTok{c}\NormalTok{(}\StringTok{"nao"}\NormalTok{,}\StringTok{"sim"}\NormalTok{),}
               \AttributeTok{names\_to  =} \StringTok{"resposta"}\NormalTok{,}
               \AttributeTok{values\_to =} \StringTok{"contagem"}\NormalTok{) }\SpecialCharTok{\%\textgreater{}\%}
  \FunctionTok{group\_by}\NormalTok{(mes\_ord, grupo) }\SpecialCharTok{\%\textgreater{}\%}
  \FunctionTok{mutate}\NormalTok{(}\AttributeTok{total =} \FunctionTok{sum}\NormalTok{(contagem), }\AttributeTok{prop =} \FunctionTok{ifelse}\NormalTok{(total}\SpecialCharTok{\textgreater{}}\DecValTok{0}\NormalTok{, contagem }\SpecialCharTok{/}\NormalTok{ total, }\DecValTok{0}\NormalTok{)) }\SpecialCharTok{\%\textgreater{}\%}
  \FunctionTok{ungroup}\NormalTok{()}

\NormalTok{plot\_df}\SpecialCharTok{$}\NormalTok{resposta }\OtherTok{\textless{}{-}} \FunctionTok{factor}\NormalTok{(plot\_df}\SpecialCharTok{$}\NormalTok{resposta, }\AttributeTok{levels =} \FunctionTok{c}\NormalTok{(}\StringTok{"nao"}\NormalTok{,}\StringTok{"sim"}\NormalTok{))}

\CommentTok{\# annotation\_df contém mes\_ord para cada faceta}
\NormalTok{annotation\_df }\OtherTok{\textless{}{-}}\NormalTok{ p\_by\_month }\SpecialCharTok{\%\textgreater{}\%} \FunctionTok{mutate}\NormalTok{(}\AttributeTok{x =} \FloatTok{1.5}\NormalTok{, }\AttributeTok{y =} \FloatTok{1.08}\NormalTok{)}

\CommentTok{\# {-}{-}{-} plot final {-}{-}{-}}
\NormalTok{p }\OtherTok{\textless{}{-}} \FunctionTok{ggplot}\NormalTok{(plot\_df, }\FunctionTok{aes}\NormalTok{(}\AttributeTok{x =}\NormalTok{ grupo, }\AttributeTok{y =}\NormalTok{ prop, }\AttributeTok{fill =}\NormalTok{ resposta)) }\SpecialCharTok{+}
  \FunctionTok{geom\_col}\NormalTok{(}\AttributeTok{colour =} \StringTok{"grey30"}\NormalTok{, }\AttributeTok{width =} \FloatTok{0.7}\NormalTok{) }\SpecialCharTok{+}
  \FunctionTok{geom\_text}\NormalTok{(}\FunctionTok{aes}\NormalTok{(}\AttributeTok{label =} \FunctionTok{ifelse}\NormalTok{(prop }\SpecialCharTok{\textgreater{}=} \FloatTok{0.03}\NormalTok{, }\FunctionTok{paste0}\NormalTok{(}\FunctionTok{round}\NormalTok{(}\DecValTok{100}\SpecialCharTok{*}\NormalTok{prop,}\DecValTok{1}\NormalTok{), }\StringTok{"\%"}\NormalTok{), }\StringTok{""}\NormalTok{)),}
            \AttributeTok{position =} \FunctionTok{position\_stack}\NormalTok{(}\AttributeTok{vjust =} \FloatTok{0.5}\NormalTok{), }\AttributeTok{colour =} \StringTok{"white"}\NormalTok{, }\AttributeTok{size =} \DecValTok{3}\NormalTok{) }\SpecialCharTok{+}
  \FunctionTok{facet\_wrap}\NormalTok{(}\SpecialCharTok{\textasciitilde{}}\NormalTok{ mes\_ord, }\AttributeTok{ncol =} \DecValTok{3}\NormalTok{, }\AttributeTok{drop =} \ConstantTok{TRUE}\NormalTok{) }\SpecialCharTok{+}
  \FunctionTok{scale\_y\_continuous}\NormalTok{(}\AttributeTok{labels =} \FunctionTok{percent\_format}\NormalTok{(}\AttributeTok{accuracy =} \DecValTok{1}\NormalTok{),}
                     \AttributeTok{breaks =} \FunctionTok{seq}\NormalTok{(}\DecValTok{0}\NormalTok{, }\DecValTok{1}\NormalTok{, }\AttributeTok{by =} \FloatTok{0.2}\NormalTok{),}
                     \AttributeTok{limits =} \FunctionTok{c}\NormalTok{(}\DecValTok{0}\NormalTok{,}\FloatTok{1.12}\NormalTok{),}
                     \AttributeTok{expand =} \FunctionTok{c}\NormalTok{(}\DecValTok{0}\NormalTok{,}\DecValTok{0}\NormalTok{)) }\SpecialCharTok{+}
  \FunctionTok{scale\_fill\_manual}\NormalTok{(}\AttributeTok{name =} \StringTok{"Acordo"}\NormalTok{, }\AttributeTok{values =} \FunctionTok{c}\NormalTok{(}\StringTok{"nao"} \OtherTok{=} \StringTok{"\#bdbdbd"}\NormalTok{, }\StringTok{"sim"} \OtherTok{=} \StringTok{"\#4E79A7"}\NormalTok{), }\AttributeTok{labels =} \FunctionTok{c}\NormalTok{(}\StringTok{"Não"}\NormalTok{,}\StringTok{"Sim"}\NormalTok{)) }\SpecialCharTok{+}
  \FunctionTok{labs}\NormalTok{(}\AttributeTok{title =} \StringTok{"Composição por resposta ao acordo — por mês/2018 (n = 648)"}\NormalTok{,}
       \AttributeTok{x =} \StringTok{"Grupo"}\NormalTok{,}
       \AttributeTok{y =} \StringTok{"Proporção dentro grupos: Controle x Tratamento"}\NormalTok{) }\SpecialCharTok{+}
  \FunctionTok{theme\_minimal}\NormalTok{() }\SpecialCharTok{+}
  \FunctionTok{theme}\NormalTok{(}
    \AttributeTok{plot.title =} \FunctionTok{element\_text}\NormalTok{(}\AttributeTok{hjust =} \FloatTok{0.5}\NormalTok{),}
    \AttributeTok{strip.text =} \FunctionTok{element\_text}\NormalTok{(}\AttributeTok{face =} \StringTok{"bold"}\NormalTok{),}
    \AttributeTok{panel.grid.major.x =} \FunctionTok{element\_blank}\NormalTok{(),}
    \AttributeTok{axis.text.y =} \FunctionTok{element\_text}\NormalTok{(}\AttributeTok{size =} \DecValTok{7}\NormalTok{) }\CommentTok{\# reduzir tamanho dos valores no eixo y}
\NormalTok{  ) }\SpecialCharTok{+}
  \FunctionTok{geom\_text}\NormalTok{(}\AttributeTok{data =}\NormalTok{ annotation\_df,}
            \FunctionTok{aes}\NormalTok{(}\AttributeTok{x =}\NormalTok{ x, }\AttributeTok{y =} \FloatTok{1.07}\NormalTok{, }\AttributeTok{label =}\NormalTok{ label),}
            \AttributeTok{inherit.aes =} \ConstantTok{FALSE}\NormalTok{, }\AttributeTok{size =} \DecValTok{2}\NormalTok{)}

\FunctionTok{print}\NormalTok{(p)}
\InformationTok{\textasciigrave{}\textasciigrave{}\textasciigrave{}}
\end{Highlighting}
\end{Shaded}

\pandocbounded{\includegraphics[keepaspectratio]{replicar-suco-uva-tab-graf_files/figure-pdf/unnamed-chunk-4-1.pdf}}

Mês de \textbf{junho} é explicado pela mudança no nudge para um simples
``SIRVA-SE'' colocado sobre a mesa em frente às partes (autor e réu).

Mês seguinte \textbf{julho} retornou-se para uma pessoa que entrava em
cada audiência, servia o suco de uva para partes e seus advogados,
depois dizia: ``Podem se servir do suco pois é uma cortesia do Forum''.
Assim como ocorrera nos meses anteriores de \textbf{abril e maio}. E
assim permaneceu \textbf{até o mês de dezembro}.

Esse \emph{nudge} mostrou-se muito mais persuasivo.

Todavia registrou-se \ul{\textbf{falta de significância}}, no nivel de
confiança adotado de 95\% (erro tipo I = 5\%), para os meses de
\ul{\textbf{setembro}} e \ul{\textbf{outubro}}\textbf{/2018.}

\subsection{Gráfico da Série
temporal}\label{gruxe1fico-da-suxe9rie-temporal}

Gráfico da Série temporal da proporção de Acordos de Conciliação (Grupos
Experimental e de Controle).

\begin{Shaded}
\begin{Highlighting}[numbers=left,,]
\InformationTok{\textasciigrave{}\textasciigrave{}\textasciigrave{}\{r\}}
\CommentTok{\# Série temporal da proporção de Acordos de Conciliação (Controle vs Tratamento)}
\CommentTok{\# Saída: gráfico com proporções por mês, IC95\% e eixo y em percentuais (0\%..100\%, passo 10\%)}
\CommentTok{\#}
\CommentTok{\# Requisitos: ggplot2, dplyr, tidyr, scales}

\ControlFlowTok{if}\NormalTok{ (}\SpecialCharTok{!}\FunctionTok{requireNamespace}\NormalTok{(}\StringTok{"ggplot2"}\NormalTok{, }\AttributeTok{quietly =} \ConstantTok{TRUE}\NormalTok{)) }\FunctionTok{install.packages}\NormalTok{(}\StringTok{"ggplot2"}\NormalTok{)}
\ControlFlowTok{if}\NormalTok{ (}\SpecialCharTok{!}\FunctionTok{requireNamespace}\NormalTok{(}\StringTok{"dplyr"}\NormalTok{, }\AttributeTok{quietly =} \ConstantTok{TRUE}\NormalTok{)) }\FunctionTok{install.packages}\NormalTok{(}\StringTok{"dplyr"}\NormalTok{)}
\ControlFlowTok{if}\NormalTok{ (}\SpecialCharTok{!}\FunctionTok{requireNamespace}\NormalTok{(}\StringTok{"tidyr"}\NormalTok{, }\AttributeTok{quietly =} \ConstantTok{TRUE}\NormalTok{)) }\FunctionTok{install.packages}\NormalTok{(}\StringTok{"tidyr"}\NormalTok{)}
\ControlFlowTok{if}\NormalTok{ (}\SpecialCharTok{!}\FunctionTok{requireNamespace}\NormalTok{(}\StringTok{"scales"}\NormalTok{, }\AttributeTok{quietly =} \ConstantTok{TRUE}\NormalTok{)) }\FunctionTok{install.packages}\NormalTok{(}\StringTok{"scales"}\NormalTok{)}
\FunctionTok{library}\NormalTok{(ggplot2); }\FunctionTok{library}\NormalTok{(dplyr); }\FunctionTok{library}\NormalTok{(tidyr); }\FunctionTok{library}\NormalTok{(scales)}

\CommentTok{\# {-}{-}{-} localizar data.frame mensal {-}{-}{-}}
\NormalTok{df\_raw }\OtherTok{\textless{}{-}}\NormalTok{ tabela\_final}

\CommentTok{\# {-}{-}{-} Normalizar nomes de colunas e identificar colunas relevantes {-}{-}{-}}
\FunctionTok{names}\NormalTok{(df\_raw) }\OtherTok{\textless{}{-}} \FunctionTok{tolower}\NormalTok{(}\FunctionTok{names}\NormalTok{(df\_raw))}
\FunctionTok{names}\NormalTok{(df\_raw) }\OtherTok{\textless{}{-}} \FunctionTok{gsub}\NormalTok{(}\StringTok{"[\^{}a{-}z0{-}9\_]"}\NormalTok{, }\StringTok{"\_"}\NormalTok{, }\FunctionTok{names}\NormalTok{(df\_raw))}
\NormalTok{col\_mes  }\OtherTok{\textless{}{-}} \FunctionTok{intersect}\NormalTok{(}\FunctionTok{names}\NormalTok{(df\_raw), }\FunctionTok{c}\NormalTok{(}\StringTok{"mes"}\NormalTok{,}\StringTok{"month"}\NormalTok{,}\StringTok{"mês"}\NormalTok{,}\StringTok{"m"}\NormalTok{))[}\DecValTok{1}\NormalTok{]}
\NormalTok{col\_grp  }\OtherTok{\textless{}{-}} \FunctionTok{intersect}\NormalTok{(}\FunctionTok{names}\NormalTok{(df\_raw), }\FunctionTok{c}\NormalTok{(}\StringTok{"grupo"}\NormalTok{,}\StringTok{"group"}\NormalTok{,}\StringTok{"tratamento"}\NormalTok{,}\StringTok{"treatment"}\NormalTok{))[}\DecValTok{1}\NormalTok{]}
\NormalTok{col\_nao  }\OtherTok{\textless{}{-}} \FunctionTok{intersect}\NormalTok{(}\FunctionTok{names}\NormalTok{(df\_raw), }\FunctionTok{c}\NormalTok{(}\StringTok{"nao"}\NormalTok{,}\StringTok{"no"}\NormalTok{,}\StringTok{"n\_nao"}\NormalTok{,}\StringTok{"n\_nao"}\NormalTok{))[}\DecValTok{1}\NormalTok{]}
\NormalTok{col\_sim  }\OtherTok{\textless{}{-}} \FunctionTok{intersect}\NormalTok{(}\FunctionTok{names}\NormalTok{(df\_raw), }\FunctionTok{c}\NormalTok{(}\StringTok{"sim"}\NormalTok{,}\StringTok{"yes"}\NormalTok{,}\StringTok{"y"}\NormalTok{))[}\DecValTok{1}\NormalTok{]}
\ControlFlowTok{if}\NormalTok{ (}\FunctionTok{any}\NormalTok{(}\FunctionTok{is.na}\NormalTok{(}\FunctionTok{c}\NormalTok{(col\_mes, col\_grp, col\_nao, col\_sim)))) }\FunctionTok{stop}\NormalTok{(}\StringTok{"Colunas esperadas (mes, grupo, nao, sim) não foram encontradas."}\NormalTok{)}

\CommentTok{\# {-}{-}{-} Padronizar mês e grupo {-}{-}{-}}
\NormalTok{month\_levels }\OtherTok{\textless{}{-}} \FunctionTok{c}\NormalTok{(}\StringTok{"Abril"}\NormalTok{,}\StringTok{"Maio"}\NormalTok{,}\StringTok{"Junho"}\NormalTok{,}\StringTok{"Julho"}\NormalTok{,}\StringTok{"Agosto"}\NormalTok{,}
                  \StringTok{"Setembro"}\NormalTok{,}\StringTok{"Outubro"}\NormalTok{,}\StringTok{"Novembro"}\NormalTok{,}\StringTok{"Dezembro"}\NormalTok{)}
\NormalTok{map\_month }\OtherTok{\textless{}{-}} \ControlFlowTok{function}\NormalTok{(x) \{}
\NormalTok{  x0 }\OtherTok{\textless{}{-}} \FunctionTok{trimws}\NormalTok{(}\FunctionTok{as.character}\NormalTok{(x))}
\NormalTok{  idx }\OtherTok{\textless{}{-}} \FunctionTok{match}\NormalTok{(}\FunctionTok{tolower}\NormalTok{(x0), }\FunctionTok{tolower}\NormalTok{(month\_levels))}
  \ControlFlowTok{if}\NormalTok{ (}\SpecialCharTok{!}\FunctionTok{is.na}\NormalTok{(idx)) }\FunctionTok{return}\NormalTok{(month\_levels[idx])}
\NormalTok{  x\_clean }\OtherTok{\textless{}{-}} \FunctionTok{iconv}\NormalTok{(x0, }\AttributeTok{to =} \StringTok{"ASCII//TRANSLIT"}\NormalTok{)}
\NormalTok{  idx2 }\OtherTok{\textless{}{-}} \FunctionTok{match}\NormalTok{(}\FunctionTok{tolower}\NormalTok{(x\_clean), }\FunctionTok{tolower}\NormalTok{(}\FunctionTok{iconv}\NormalTok{(month\_levels, }\AttributeTok{to =} \StringTok{"ASCII//TRANSLIT"}\NormalTok{)))}
  \ControlFlowTok{if}\NormalTok{ (}\SpecialCharTok{!}\FunctionTok{is.na}\NormalTok{(idx2)) }\FunctionTok{return}\NormalTok{(month\_levels[idx2])}
  \FunctionTok{return}\NormalTok{(x0)}
\NormalTok{\}}

\NormalTok{df }\OtherTok{\textless{}{-}}\NormalTok{ df\_raw }\SpecialCharTok{\%\textgreater{}\%}
  \FunctionTok{mutate}\NormalTok{(}
    \AttributeTok{mes\_raw =}\NormalTok{ .data[[col\_mes]],}
    \AttributeTok{mes =} \FunctionTok{vapply}\NormalTok{(mes\_raw, map\_month, }\FunctionTok{character}\NormalTok{(}\DecValTok{1}\NormalTok{)),}
    \AttributeTok{mes =} \FunctionTok{factor}\NormalTok{(mes, }\AttributeTok{levels =}\NormalTok{ month\_levels),}
    \AttributeTok{grupo\_raw =} \FunctionTok{as.character}\NormalTok{(.data[[col\_grp]]),}
    \AttributeTok{grupo =} \FunctionTok{ifelse}\NormalTok{(}\FunctionTok{tolower}\NormalTok{(}\FunctionTok{trimws}\NormalTok{(grupo\_raw)) }\SpecialCharTok{\%in\%} \FunctionTok{c}\NormalTok{(}\StringTok{"controle"}\NormalTok{,}\StringTok{"control"}\NormalTok{), }\StringTok{"Controle"}\NormalTok{, }\StringTok{"Tratamento"}\NormalTok{),}
    \AttributeTok{nao =} \FunctionTok{as.numeric}\NormalTok{(.data[[col\_nao]]),}
    \AttributeTok{sim =} \FunctionTok{as.numeric}\NormalTok{(.data[[col\_sim]])}
\NormalTok{  ) }\SpecialCharTok{\%\textgreater{}\%}
  \FunctionTok{filter}\NormalTok{(}\SpecialCharTok{!}\FunctionTok{is.na}\NormalTok{(mes))}

\CommentTok{\# {-}{-}{-} Agregar por mês e grupo (caso haja múltiplas linhas por combinação) {-}{-}{-}}
\NormalTok{per\_month }\OtherTok{\textless{}{-}}\NormalTok{ df }\SpecialCharTok{\%\textgreater{}\%}
  \FunctionTok{group\_by}\NormalTok{(mes, grupo) }\SpecialCharTok{\%\textgreater{}\%}
  \FunctionTok{summarise}\NormalTok{(}\AttributeTok{nao =} \FunctionTok{sum}\NormalTok{(nao, }\AttributeTok{na.rm =} \ConstantTok{TRUE}\NormalTok{), }\AttributeTok{sim =} \FunctionTok{sum}\NormalTok{(sim, }\AttributeTok{na.rm =} \ConstantTok{TRUE}\NormalTok{), }\AttributeTok{.groups =} \StringTok{"drop"}\NormalTok{) }\SpecialCharTok{\%\textgreater{}\%}
  \FunctionTok{complete}\NormalTok{(mes, }\AttributeTok{grupo =} \FunctionTok{c}\NormalTok{(}\StringTok{"Controle"}\NormalTok{, }\StringTok{"Tratamento"}\NormalTok{), }\AttributeTok{fill =} \FunctionTok{list}\NormalTok{(}\AttributeTok{nao =} \DecValTok{0}\NormalTok{, }\AttributeTok{sim =} \DecValTok{0}\NormalTok{)) }\SpecialCharTok{\%\textgreater{}\%}
  \FunctionTok{arrange}\NormalTok{(mes, grupo)}

\CommentTok{\# {-}{-}{-} Calcular proporção e IC95\% (prop.test) para cada (mes, grupo) {-}{-}{-}}
\NormalTok{stats }\OtherTok{\textless{}{-}}\NormalTok{ per\_month }\SpecialCharTok{\%\textgreater{}\%}
  \FunctionTok{rowwise}\NormalTok{() }\SpecialCharTok{\%\textgreater{}\%}
  \FunctionTok{mutate}\NormalTok{(}
    \AttributeTok{total =} \FunctionTok{as.integer}\NormalTok{(nao }\SpecialCharTok{+}\NormalTok{ sim),}
    \AttributeTok{successes =} \FunctionTok{as.integer}\NormalTok{(sim),}
    \AttributeTok{prop =} \ControlFlowTok{if}\NormalTok{ (total }\SpecialCharTok{\textgreater{}} \DecValTok{0}\NormalTok{) successes }\SpecialCharTok{/}\NormalTok{ total }\ControlFlowTok{else} \ConstantTok{NA\_real\_}\NormalTok{,}
    \AttributeTok{ci =} \FunctionTok{list}\NormalTok{(}
      \ControlFlowTok{if}\NormalTok{ (total }\SpecialCharTok{\textgreater{}} \DecValTok{0}\NormalTok{) \{}
        \CommentTok{\# prop.test devolve intervalo; tryCatch protege contra erros}
\NormalTok{        res }\OtherTok{\textless{}{-}} \FunctionTok{tryCatch}\NormalTok{(}\FunctionTok{prop.test}\NormalTok{(successes, total, }\AttributeTok{correct =} \ConstantTok{FALSE}\NormalTok{), }\AttributeTok{error =} \ControlFlowTok{function}\NormalTok{(e) }\ConstantTok{NULL}\NormalTok{)}
        \ControlFlowTok{if}\NormalTok{ (}\FunctionTok{is.null}\NormalTok{(res)) }\FunctionTok{c}\NormalTok{(}\ConstantTok{NA\_real\_}\NormalTok{, }\ConstantTok{NA\_real\_}\NormalTok{) }\ControlFlowTok{else} \FunctionTok{as.numeric}\NormalTok{(res}\SpecialCharTok{$}\NormalTok{conf.int)}
\NormalTok{      \} }\ControlFlowTok{else} \FunctionTok{c}\NormalTok{(}\ConstantTok{NA\_real\_}\NormalTok{, }\ConstantTok{NA\_real\_}\NormalTok{)}
\NormalTok{    ),}
    \AttributeTok{ci\_low =}\NormalTok{ ci[[}\DecValTok{1}\NormalTok{]],}
    \AttributeTok{ci\_up  =}\NormalTok{ ci[[}\DecValTok{2}\NormalTok{]]}
\NormalTok{  ) }\SpecialCharTok{\%\textgreater{}\%}
  \FunctionTok{ungroup}\NormalTok{() }\SpecialCharTok{\%\textgreater{}\%}
  \FunctionTok{select}\NormalTok{(mes, grupo, total, successes, prop, ci\_low, ci\_up)}

\CommentTok{\# {-}{-}{-} Preparar para plotagem (long format se necessário) {-}{-}{-}}
\NormalTok{plot\_df }\OtherTok{\textless{}{-}}\NormalTok{ stats }\SpecialCharTok{\%\textgreater{}\%}
  \FunctionTok{mutate}\NormalTok{(}\AttributeTok{prop =} \FunctionTok{as.numeric}\NormalTok{(prop), }\AttributeTok{ci\_low =} \FunctionTok{as.numeric}\NormalTok{(ci\_low), }\AttributeTok{ci\_up =} \FunctionTok{as.numeric}\NormalTok{(ci\_up))}

\CommentTok{\# INVERSÃO DA ORDEM: definir factor com níveis invertidos (Tratamento primeiro, depois Controle)}
\NormalTok{plot\_df}\SpecialCharTok{$}\NormalTok{grupo }\OtherTok{\textless{}{-}} \FunctionTok{factor}\NormalTok{(plot\_df}\SpecialCharTok{$}\NormalTok{grupo, }\AttributeTok{levels =} \FunctionTok{c}\NormalTok{(}\StringTok{"Tratamento"}\NormalTok{, }\StringTok{"Controle"}\NormalTok{))}

\CommentTok{\# {-}{-}{-} Plot: linhas por grupo com pontos e barras de erro; eixo y 0\%..100\% {-}{-}{-}}
\NormalTok{p }\OtherTok{\textless{}{-}} \FunctionTok{ggplot}\NormalTok{(plot\_df, }\FunctionTok{aes}\NormalTok{(}\AttributeTok{x =}\NormalTok{ mes, }\AttributeTok{y =}\NormalTok{ prop, }\AttributeTok{group =}\NormalTok{ grupo, }\AttributeTok{color =}\NormalTok{ grupo)) }\SpecialCharTok{+}
  \FunctionTok{geom\_line}\NormalTok{(}\AttributeTok{size =} \FloatTok{0.9}\NormalTok{, }\AttributeTok{na.rm =} \ConstantTok{TRUE}\NormalTok{) }\SpecialCharTok{+}
  \FunctionTok{geom\_point}\NormalTok{(}\AttributeTok{size =} \FloatTok{2.4}\NormalTok{, }\AttributeTok{na.rm =} \ConstantTok{TRUE}\NormalTok{) }\SpecialCharTok{+}
  \FunctionTok{geom\_errorbar}\NormalTok{(}\FunctionTok{aes}\NormalTok{(}\AttributeTok{ymin =}\NormalTok{ ci\_low, }\AttributeTok{ymax =}\NormalTok{ ci\_up), }\AttributeTok{width =} \FloatTok{0.15}\NormalTok{, }\AttributeTok{size =} \FloatTok{0.7}\NormalTok{, }\AttributeTok{position =} \FunctionTok{position\_dodge}\NormalTok{(}\AttributeTok{width =} \FloatTok{0.2}\NormalTok{), }\AttributeTok{na.rm =} \ConstantTok{TRUE}\NormalTok{) }\SpecialCharTok{+}
  \FunctionTok{scale\_y\_continuous}\NormalTok{(}\AttributeTok{labels =} \FunctionTok{percent\_format}\NormalTok{(}\AttributeTok{accuracy =} \DecValTok{1}\NormalTok{), }\AttributeTok{breaks =} \FunctionTok{seq}\NormalTok{(}\DecValTok{0}\NormalTok{, }\DecValTok{1}\NormalTok{, }\AttributeTok{by =} \FloatTok{0.1}\NormalTok{), }\AttributeTok{limits =} \FunctionTok{c}\NormalTok{(}\DecValTok{0}\NormalTok{, }\DecValTok{1}\NormalTok{)) }\SpecialCharTok{+}
  \CommentTok{\# definir cores e BREAKS na ordem desejada para garantir legenda nessa ordem}
  \FunctionTok{scale\_color\_manual}\NormalTok{(}\AttributeTok{breaks =} \FunctionTok{c}\NormalTok{(}\StringTok{"Tratamento"}\NormalTok{, }\StringTok{"Controle"}\NormalTok{),}
                     \AttributeTok{values =} \FunctionTok{c}\NormalTok{(}\StringTok{"Controle"} \OtherTok{=} \StringTok{"\#1F78B4"}\NormalTok{, }\StringTok{"Tratamento"} \OtherTok{=} \StringTok{"\#E31A1C"}\NormalTok{)) }\SpecialCharTok{+}
  \FunctionTok{labs}\NormalTok{(}
    \AttributeTok{title =} \StringTok{"Série temporal: proporção de Acordos de Conciliação"}\NormalTok{,}
    \AttributeTok{subtitle =} \StringTok{"Comparação entre Grupo Controle e Tratamento por mês/2018 (n=659)"}\NormalTok{,}
    \AttributeTok{x =} \StringTok{"Mês"}\NormalTok{, }\AttributeTok{y =} \StringTok{"Proporção de Acordos (IC95\%)"}\NormalTok{,}
    \AttributeTok{color =} \StringTok{"Grupo"}
\NormalTok{  ) }\SpecialCharTok{+}
  \FunctionTok{theme\_minimal}\NormalTok{() }\SpecialCharTok{+}
  \FunctionTok{theme}\NormalTok{(}
    \AttributeTok{plot.title =} \FunctionTok{element\_text}\NormalTok{(}\AttributeTok{hjust =} \FloatTok{0.5}\NormalTok{),}
    \AttributeTok{plot.subtitle =} \FunctionTok{element\_text}\NormalTok{(}\AttributeTok{hjust =} \FloatTok{0.5}\NormalTok{),}
    \AttributeTok{axis.text.x =} \FunctionTok{element\_text}\NormalTok{(}\AttributeTok{angle =} \DecValTok{45}\NormalTok{, }\AttributeTok{vjust =} \FloatTok{0.5}\NormalTok{),}
    \AttributeTok{axis.text.y =} \FunctionTok{element\_text}\NormalTok{(}\AttributeTok{size =} \DecValTok{8}\NormalTok{)}
\NormalTok{  )}

\FunctionTok{print}\NormalTok{(p)}

\CommentTok{\# Opcional: salvar figura}
\CommentTok{\# ggsave("ts\_prop\_acordos\_by\_month\_inverted\_groups.png", p, width = 10, height = 5)}

\CommentTok{\# Fim do script}
\InformationTok{\textasciigrave{}\textasciigrave{}\textasciigrave{}}
\end{Highlighting}
\end{Shaded}

\pandocbounded{\includegraphics[keepaspectratio]{replicar-suco-uva-tab-graf_files/figure-pdf/unnamed-chunk-5-1.pdf}}

A observação de que ``2 - Apenas 28 audiências de conciliação
aconteceram no mês de \ul{\textbf{\emph{outubro}}}. Motivo: férias da
magistrada.'' (TOMÁS, 2020 , p.~221). Não serve de justificativa para a
variação observada nesse mês. Uma vez que todas as condições do teste
foram mantidas e \textbf{\emph{não era a magistrada}} que conduzia as
ausiências de conciliação juunto ao CEJUSC.

Outras observações registradas foram:

\begin{quote}
``3 A \textbf{Semana Nacional de Conciliação} ocorreu entre os dias 5 e
9 de \ul{\emph{novembro}} de 2018.

4 Apenas 28 audiências de conciliação aconteceram no mês de
\ul{\emph{dezembro}}. Motivo: \textbf{Recesso forense} a partir de 20 de
dezembro. Fonte: Dados da pesquisa.'' (TOMÁS, 2020 , p.~221)
\end{quote}

O que também não justifica a variação observada, nos meses de
\ul{\textbf{setembro}} e \ul{\textbf{outubro}}\textbf{/2018}, na
replicação mensal deste \emph{quase experimento}.

Como o experimento foi replicado por \textbf{\emph{nove meses}}, o fato
de em dois deles (2 / 9 = 22,2\%) \textbf{\emph{não resultar
estatisticamente significativo}} é uma evidência no sentido que ele
precisa ser novamente replicado, por período igual ou superior; pois há
a possibilidade de que o acaso poderia ser uma explicação concorrente
para a variabilidade desse subconjunto de insucessos.

E é preciso que suas \ul{\textbf{circunstâncias}} sejam \ul{\textbf{mais
detidamente controladas}}, tais como:

\begin{enumerate}
\def\labelenumi{\arabic{enumi}.}
\tightlist
\item
  aumentar o período de aplicação para 12 meses;
\item
  manter um único \emph{nudge} do começo ao fim: pessoa que entra, serve
  o suco ou a água fresca e sempre repete os mesmos dizeres;
\item
  essa pessoa que serve não poderá dizer a ninguem o que foi servido em
  cada audiência;
\item
  \textbf{\emph{aleatorizar}} a ordem de chegada dos casos entre os doi
  grupos: de controle e tratamento;
\item
  utilizar \emph{uma mesma sala} de conciliação para os dois grupos,
  desde que atenda à todas as recomendações da Resolução do CNJ;
\item
  \textbf{\emph{aleatorizar}} qual dos 4 conciliadores (2 homens e 2
  mulheres) irá presidir cada audiência de conciliação. Valer-se de pelo
  menos dois conciliadores treinados e com experiência de pelo menos 3
  anos (1 homem e 1 mulher), a serem aleatorizados para acada audiência.
\item
  \textbf{\emph{aleatorizar}} qual técnica de conciliação, entre seis
  encontradas na revisão da literatura, irá ser aplicada a cada
  audiência de conciliação;
\item
  \textbf{\emph{organizar as pautas}} das audiências de conciliação de
  modo a respeitar os itens 4, 5, 6 e 7;
\item
  para buscar atender um pouco mais ao ``\emph{duplo cego}'', servir
  tanto o suco de uva (Grupo de Tratamento) como o placebo água fresca
  (Grupo de Controle) em \emph{copos de papel descartáveis não
  transparentes} e \emph{grandes o suficiente}, de modo a evitar que o
  conciliador possa ver qual tipo de bebida (suco ou água) as partes e
  seus advogados estão ingerindo;
\item
  refletir se vale a pena servir água fresca apenas com corante para
  deixar ela com a mesma cor do suco de uva, mas sem adição de qualquer
  glicose e dizer: que se trata de um ``\emph{suco de uva diet}''
  servido como cortesia na sala do Grupo de Controle;
\item
  cuidar para que o Grupo de Controle mantenha seu \emph{nível base de
  acordos observado no início do experimento} (abril a setembro de 2018,
  cf.~\emph{gráfico da série temporal} acima), \emph{evitando-se} assim
  a \emph{tendência de queda} observada nos meses de setembro a dezembro
  de 2018;
\item
  \textbf{Evitar} que \textbf{\emph{amostras pequenas}}, menores que
  trinta (n \textless{} 30), ocorram em qualquer um dos dois Grupos:
  Controle e Tratamento. Por exemplo agrupando no mesmo CEJUSC,
  processos oriundos de mais de uam vara nos meses de outubro e
  dezembro/2018 (n = 14; cf.~abaixo gráfico da série temporal da
  ingestão de suco \textless sim/não\textgreater{} apenas no Grupo de
  Tratamento).
\end{enumerate}

\subsection{Gráfico da ingestão de
suco}\label{gruxe1fico-da-ingestuxe3o-de-suco}

Gerar dados sobre ingestão ou não de suci de uvo apenas no Grupo de
Tratamento.

\begin{Shaded}
\begin{Highlighting}[numbers=left,,]
\InformationTok{\textasciigrave{}\textasciigrave{}\textasciigrave{}\{r\}}
\CommentTok{\# Replicar tabela: indivíduos do grupo Experimental que ingeriram / não ingeriram suco por mês}
\CommentTok{\# Saída: objeto \textquotesingle{}tabela\_experimental\_summary\textquotesingle{} e impressão formatada (kable HTML quando possível)}
\CommentTok{\#}
\CommentTok{\# Copie/cole e execute no R/RStudio.}

\CommentTok{\# Pacotes}
\ControlFlowTok{if}\NormalTok{ (}\SpecialCharTok{!}\FunctionTok{requireNamespace}\NormalTok{(}\StringTok{"dplyr"}\NormalTok{, }\AttributeTok{quietly =} \ConstantTok{TRUE}\NormalTok{)) }\FunctionTok{install.packages}\NormalTok{(}\StringTok{"dplyr"}\NormalTok{)}
\ControlFlowTok{if}\NormalTok{ (}\SpecialCharTok{!}\FunctionTok{requireNamespace}\NormalTok{(}\StringTok{"knitr"}\NormalTok{, }\AttributeTok{quietly =} \ConstantTok{TRUE}\NormalTok{)) }\FunctionTok{install.packages}\NormalTok{(}\StringTok{"knitr"}\NormalTok{)}
\ControlFlowTok{if}\NormalTok{ (}\SpecialCharTok{!}\FunctionTok{requireNamespace}\NormalTok{(}\StringTok{"kableExtra"}\NormalTok{, }\AttributeTok{quietly =} \ConstantTok{TRUE}\NormalTok{)) \{}
  \FunctionTok{message}\NormalTok{(}\StringTok{"kableExtra não encontrado: a impressão usará knitr::kable simples."}\NormalTok{)}
\NormalTok{\}}
\FunctionTok{library}\NormalTok{(dplyr)}
\FunctionTok{library}\NormalTok{(knitr)}
\ControlFlowTok{if}\NormalTok{ (}\StringTok{"kableExtra"} \SpecialCharTok{\%in\%} \FunctionTok{rownames}\NormalTok{(}\FunctionTok{installed.packages}\NormalTok{())) }\FunctionTok{library}\NormalTok{(kableExtra)}
\FunctionTok{library}\NormalTok{(tinytex)}

\CommentTok{\# {-}{-}{-} Dados (valores extraídos da imagem) {-}{-}{-}}
\NormalTok{tabela\_experimental\_summary }\OtherTok{\textless{}{-}} \FunctionTok{data.frame}\NormalTok{(}
  \AttributeTok{Mes =} \FunctionTok{c}\NormalTok{(}\StringTok{"Abril"}\NormalTok{,}\StringTok{"Maio"}\NormalTok{,}\StringTok{"Junho"}\NormalTok{,}\StringTok{"Julho"}\NormalTok{,}\StringTok{"Agosto"}\NormalTok{,}
          \StringTok{"Setembro"}\NormalTok{,}\StringTok{"Outubro"}\NormalTok{,}\StringTok{"Novembro"}\NormalTok{,}\StringTok{"Dezembro"}\NormalTok{),}
  \AttributeTok{ing\_n  =} \FunctionTok{c}\NormalTok{(}\DecValTok{47}\NormalTok{, }\DecValTok{43}\NormalTok{, }\DecValTok{16}\NormalTok{, }\DecValTok{35}\NormalTok{, }\DecValTok{29}\NormalTok{, }\DecValTok{38}\NormalTok{, }\DecValTok{13}\NormalTok{, }\DecValTok{55}\NormalTok{, }\DecValTok{12}\NormalTok{),  }\CommentTok{\# ingeriram suco}
  \AttributeTok{not\_n  =} \FunctionTok{c}\NormalTok{(}\DecValTok{27}\NormalTok{,  }\DecValTok{2}\NormalTok{, }\DecValTok{20}\NormalTok{,  }\DecValTok{2}\NormalTok{,  }\DecValTok{3}\NormalTok{,  }\DecValTok{5}\NormalTok{,  }\DecValTok{1}\NormalTok{,  }\DecValTok{4}\NormalTok{,  }\DecValTok{2}\NormalTok{),  }\CommentTok{\# não ingeriram suco}
  \AttributeTok{stringsAsFactors =} \ConstantTok{FALSE}
\NormalTok{)}

\CommentTok{\# {-}{-}{-} Cálculos: total e percentuais (vetorizado) {-}{-}{-}}
\NormalTok{tabela\_experimental\_summary }\OtherTok{\textless{}{-}}\NormalTok{ tabela\_experimental\_summary }\SpecialCharTok{\%\textgreater{}\%}
  \FunctionTok{mutate}\NormalTok{(}
    \AttributeTok{total =}\NormalTok{ ing\_n }\SpecialCharTok{+}\NormalTok{ not\_n,}
    \AttributeTok{pct\_ing =} \FunctionTok{ifelse}\NormalTok{(total }\SpecialCharTok{\textgreater{}} \DecValTok{0}\NormalTok{, }\DecValTok{100} \SpecialCharTok{*}\NormalTok{ ing\_n }\SpecialCharTok{/}\NormalTok{ total, }\ConstantTok{NA\_real\_}\NormalTok{),}
    \AttributeTok{pct\_not =} \FunctionTok{ifelse}\NormalTok{(total }\SpecialCharTok{\textgreater{}} \DecValTok{0}\NormalTok{, }\DecValTok{100} \SpecialCharTok{*}\NormalTok{ not\_n }\SpecialCharTok{/}\NormalTok{ total, }\ConstantTok{NA\_real\_}\NormalTok{),}
    \CommentTok{\# colunas formatadas com quebra de linha (HTML \textless{}br/\textgreater{}) para visual tipo tabela da imagem}
    \AttributeTok{Ingere     =} \FunctionTok{ifelse}\NormalTok{(}\SpecialCharTok{!}\FunctionTok{is.na}\NormalTok{(total),}
                        \FunctionTok{paste0}\NormalTok{(ing\_n, }\StringTok{"\textless{}br/\textgreater{}("}\NormalTok{, }\FunctionTok{sprintf}\NormalTok{(}\StringTok{"\%.2f\%\%"}\NormalTok{, pct\_ing), }\StringTok{")"}\NormalTok{), }\ConstantTok{NA\_character\_}\NormalTok{),}
    \AttributeTok{Nao\_ingere =} \FunctionTok{ifelse}\NormalTok{(}\SpecialCharTok{!}\FunctionTok{is.na}\NormalTok{(total),}
                        \FunctionTok{paste0}\NormalTok{(not\_n, }\StringTok{"\textless{}br/\textgreater{}("}\NormalTok{, }\FunctionTok{sprintf}\NormalTok{(}\StringTok{"\%.2f\%\%"}\NormalTok{, pct\_not), }\StringTok{")"}\NormalTok{), }\ConstantTok{NA\_character\_}\NormalTok{),}
    \AttributeTok{Total\_fmt  =} \FunctionTok{ifelse}\NormalTok{(}\SpecialCharTok{!}\FunctionTok{is.na}\NormalTok{(total),}
                        \FunctionTok{paste0}\NormalTok{(total, }\StringTok{"\textless{}br/\textgreater{}(100\%)"}\NormalTok{), }\ConstantTok{NA\_character\_}\NormalTok{)}
\NormalTok{  ) }\SpecialCharTok{\%\textgreater{}\%}
  \FunctionTok{select}\NormalTok{(Mês }\OtherTok{=}\NormalTok{ Mes, Ingere, Nao\_ingere, }\AttributeTok{Total =}\NormalTok{ Total\_fmt)}

\CommentTok{\# {-}{-}{-} Imprimir: tentar kable HTML com kableExtra; fallback para kable texto {-}{-}{-}}
\CommentTok{\# if ("kableExtra" \%in\% rownames(installed.packages())) \{}
\CommentTok{\#   kable(tabela\_experimental\_summary, format = "html", escape = FALSE, align = c("l","c","c","c")) \%\textgreater{}\%}
\CommentTok{\#     kable\_styling(full\_width = FALSE, bootstrap\_options = c("striped", "condensed")) \%\textgreater{}\%}
\CommentTok{\#     add\_header\_above(c(" " = 1, "Indivíduos do grupo experimental que ingeriram suco" = 1,}
\CommentTok{\#                        "Indivíduos do grupo experimental que não ingeriram suco" = 1,}
\CommentTok{\#                        "Total" = 1))}
\CommentTok{\# \} else \{}
\CommentTok{\#   \# versão texto: usar \textbackslash{}n nas células para console}
\CommentTok{\#   tabela\_texto \textless{}{-} tabela\_experimental\_summary \%\textgreater{}\%}
\CommentTok{\#     mutate(}
\CommentTok{\#       Ingere = gsub("\textless{}br/\textgreater{}", "\textbackslash{}n", Ingere),}
\CommentTok{\#       Nao\_ingere = gsub("\textless{}br/\textgreater{}", "\textbackslash{}n", Nao\_ingere),}
\CommentTok{\#       Total = gsub("\textless{}br/\textgreater{}", "\textbackslash{}n", Total)}
\CommentTok{\#     )}
\CommentTok{\#   cat("\textbackslash{}nTabela Experimental — ingestão de suco por mês\textbackslash{}n")}
\CommentTok{\#   print(tabela\_texto, right = FALSE, row.names = FALSE)}
\CommentTok{\# \}}
\CommentTok{\# }
\CommentTok{\# \# objeto disponível para uso posterior}
\FunctionTok{assign}\NormalTok{(}\StringTok{"tabela\_experimental\_summary"}\NormalTok{, tabela\_experimental\_summary, }\AttributeTok{envir =}\NormalTok{ .GlobalEnv)}

\FunctionTok{print}\NormalTok{(}\FunctionTok{kable}\NormalTok{(tabela\_experimental\_summary,}
            \AttributeTok{align =}\FunctionTok{c}\NormalTok{(}\StringTok{"l"}\NormalTok{,}\StringTok{"c"}\NormalTok{,}\StringTok{"c"}\NormalTok{,}\StringTok{"c"}\NormalTok{)))}

\CommentTok{\# Fim do script}
\InformationTok{\textasciigrave{}\textasciigrave{}\textasciigrave{}}
\end{Highlighting}
\end{Shaded}

\begin{verbatim}


|Mês      |     Ingere      |   Nao_ingere    |     Total     |
|:--------|:---------------:|:---------------:|:-------------:|
|Abril    | 47<br/>(63.51%) | 27<br/>(36.49%) | 74<br/>(100%) |
|Maio     | 43<br/>(95.56%) |  2<br/>(4.44%)  | 45<br/>(100%) |
|Junho    | 16<br/>(44.44%) | 20<br/>(55.56%) | 36<br/>(100%) |
|Julho    | 35<br/>(94.59%) |  2<br/>(5.41%)  | 37<br/>(100%) |
|Agosto   | 29<br/>(90.62%) |  3<br/>(9.38%)  | 32<br/>(100%) |
|Setembro | 38<br/>(88.37%) | 5<br/>(11.63%)  | 43<br/>(100%) |
|Outubro  | 13<br/>(92.86%) |  1<br/>(7.14%)  | 14<br/>(100%) |
|Novembro | 55<br/>(93.22%) |  4<br/>(6.78%)  | 59<br/>(100%) |
|Dezembro | 12<br/>(85.71%) | 2<br/>(14.29%)  | 14<br/>(100%) |
\end{verbatim}

Tabela como impressa no artigo (TOMÁS, 2020 , p.~223).

\begin{figure}[H]

{\centering \pandocbounded{\includegraphics[keepaspectratio]{fig/dados-ingestao-suco-uva-GT.png}}

}

\caption{Quantidade e proporção de indivíduos que ingeriram e não
ingeriram suco no grupo experimental}

\end{figure}%

Gráfico da série temporal da ingestão de suco (sim/não) apenas no Grupo
de Tratamento.

Com intervalo de confiança 95\% para ponto da série.

\begin{Shaded}
\begin{Highlighting}[numbers=left,,]
\InformationTok{\textasciigrave{}\textasciigrave{}\textasciigrave{}\{r\}}
\CommentTok{\# Entrada: tabela\_experimental\_summary (criada anteriormente)}
\CommentTok{\# Saída: ts\_suco\_tratamento\_table (tibble) e ts\_suco\_tratamento\_plot (ggplot) no GlobalEnv}
\CommentTok{\#}
\CommentTok{\# O script:}
\CommentTok{\# {-} detecta/obtém colunas numéricas (ing\_n / not\_n) ou extrai números de strings formatadas}
\CommentTok{\# {-} normaliza meses na ordem Abril..Dezembro}
\CommentTok{\# {-} calcula proporção de "Sim" e IC95\% (prop.test) por mês}
\CommentTok{\# {-} plota linha com pontos e barras de erro (IC95\%)}

\ControlFlowTok{if}\NormalTok{ (}\SpecialCharTok{!}\FunctionTok{requireNamespace}\NormalTok{(}\StringTok{"dplyr"}\NormalTok{, }\AttributeTok{quietly =} \ConstantTok{TRUE}\NormalTok{)) }\FunctionTok{install.packages}\NormalTok{(}\StringTok{"dplyr"}\NormalTok{)}
\ControlFlowTok{if}\NormalTok{ (}\SpecialCharTok{!}\FunctionTok{requireNamespace}\NormalTok{(}\StringTok{"tidyr"}\NormalTok{, }\AttributeTok{quietly =} \ConstantTok{TRUE}\NormalTok{)) }\FunctionTok{install.packages}\NormalTok{(}\StringTok{"tidyr"}\NormalTok{)}
\ControlFlowTok{if}\NormalTok{ (}\SpecialCharTok{!}\FunctionTok{requireNamespace}\NormalTok{(}\StringTok{"ggplot2"}\NormalTok{, }\AttributeTok{quietly =} \ConstantTok{TRUE}\NormalTok{)) }\FunctionTok{install.packages}\NormalTok{(}\StringTok{"ggplot2"}\NormalTok{)}
\ControlFlowTok{if}\NormalTok{ (}\SpecialCharTok{!}\FunctionTok{requireNamespace}\NormalTok{(}\StringTok{"scales"}\NormalTok{, }\AttributeTok{quietly =} \ConstantTok{TRUE}\NormalTok{)) }\FunctionTok{install.packages}\NormalTok{(}\StringTok{"scales"}\NormalTok{)}
\FunctionTok{library}\NormalTok{(dplyr); }\FunctionTok{library}\NormalTok{(tidyr); }\FunctionTok{library}\NormalTok{(ggplot2); }\FunctionTok{library}\NormalTok{(scales)}

\CommentTok{\# {-}{-}{-} Verificar existência do objeto {-}{-}{-}}
\ControlFlowTok{if}\NormalTok{ (}\SpecialCharTok{!}\FunctionTok{exists}\NormalTok{(}\StringTok{"tabela\_experimental\_summary"}\NormalTok{, }\AttributeTok{envir =}\NormalTok{ .GlobalEnv)) \{}
  \FunctionTok{stop}\NormalTok{(}\StringTok{"Objeto \textquotesingle{}tabela\_experimental\_summary\textquotesingle{} não encontrado no Global Environment."}\NormalTok{)}
\NormalTok{\}}
\NormalTok{tab\_in }\OtherTok{\textless{}{-}} \FunctionTok{get}\NormalTok{(}\StringTok{"tabela\_experimental\_summary"}\NormalTok{, }\AttributeTok{envir =}\NormalTok{ .GlobalEnv)}

\CommentTok{\# normalizar nomes}
\FunctionTok{names}\NormalTok{(tab\_in) }\OtherTok{\textless{}{-}} \FunctionTok{tolower}\NormalTok{(}\FunctionTok{names}\NormalTok{(tab\_in))}
\FunctionTok{names}\NormalTok{(tab\_in) }\OtherTok{\textless{}{-}} \FunctionTok{gsub}\NormalTok{(}\StringTok{"[\^{}a{-}z0{-}9\_áéíóúâêôãõç]"}\NormalTok{, }\StringTok{"\_"}\NormalTok{, }\FunctionTok{names}\NormalTok{(tab\_in), }\AttributeTok{perl =} \ConstantTok{TRUE}\NormalTok{)}

\CommentTok{\# candidates for month and count columns}
\NormalTok{month\_candidates }\OtherTok{\textless{}{-}} \FunctionTok{c}\NormalTok{(}\StringTok{"mes"}\NormalTok{,}\StringTok{"mês"}\NormalTok{,}\StringTok{"m"}\NormalTok{,}\StringTok{"month"}\NormalTok{)}
\NormalTok{ing\_candidates }\OtherTok{\textless{}{-}} \FunctionTok{c}\NormalTok{(}\StringTok{"ing\_n"}\NormalTok{,}\StringTok{"ingere"}\NormalTok{,}\StringTok{"ingeriram"}\NormalTok{,}\StringTok{"ingeriram\_suco"}\NormalTok{,}\StringTok{"ingested"}\NormalTok{,}\StringTok{"ing"}\NormalTok{)}
\NormalTok{not\_candidates }\OtherTok{\textless{}{-}} \FunctionTok{c}\NormalTok{(}\StringTok{"not\_n"}\NormalTok{,}\StringTok{"nao\_n"}\NormalTok{,}\StringTok{"nao"}\NormalTok{,}\StringTok{"nao\_ingere"}\NormalTok{,}\StringTok{"nao\_ingere"}\NormalTok{,}\StringTok{"not"}\NormalTok{,}\StringTok{"not\_ing"}\NormalTok{)}

\CommentTok{\# localizar coluna de mês}
\NormalTok{find\_col }\OtherTok{\textless{}{-}} \ControlFlowTok{function}\NormalTok{(nms, candidates) \{}
\NormalTok{  nms\_l }\OtherTok{\textless{}{-}} \FunctionTok{tolower}\NormalTok{(nms)}
\NormalTok{  cand }\OtherTok{\textless{}{-}} \FunctionTok{tolower}\NormalTok{(candidates)}
\NormalTok{  m }\OtherTok{\textless{}{-}}\NormalTok{ nms[nms\_l }\SpecialCharTok{\%in\%}\NormalTok{ cand]}
  \ControlFlowTok{if}\NormalTok{ (}\FunctionTok{length}\NormalTok{(m)) }\FunctionTok{return}\NormalTok{(m[}\DecValTok{1}\NormalTok{])}
  \ControlFlowTok{for}\NormalTok{ (pat }\ControlFlowTok{in}\NormalTok{ cand) \{}
\NormalTok{    mm }\OtherTok{\textless{}{-}}\NormalTok{ nms[}\FunctionTok{grepl}\NormalTok{(pat, nms\_l, }\AttributeTok{ignore.case =} \ConstantTok{TRUE}\NormalTok{)]}
    \ControlFlowTok{if}\NormalTok{ (}\FunctionTok{length}\NormalTok{(mm)) }\FunctionTok{return}\NormalTok{(mm[}\DecValTok{1}\NormalTok{])}
\NormalTok{  \}}
  \FunctionTok{return}\NormalTok{(}\ConstantTok{NA\_character\_}\NormalTok{)}
\NormalTok{\}}
\NormalTok{col\_mes }\OtherTok{\textless{}{-}} \FunctionTok{find\_col}\NormalTok{(}\FunctionTok{names}\NormalTok{(tab\_in), month\_candidates)}
\NormalTok{col\_ing }\OtherTok{\textless{}{-}} \FunctionTok{find\_col}\NormalTok{(}\FunctionTok{names}\NormalTok{(tab\_in), ing\_candidates)}
\NormalTok{col\_not }\OtherTok{\textless{}{-}} \FunctionTok{find\_col}\NormalTok{(}\FunctionTok{names}\NormalTok{(tab\_in), not\_candidates)}

\ControlFlowTok{if}\NormalTok{ (}\FunctionTok{is.na}\NormalTok{(col\_mes)) }\FunctionTok{stop}\NormalTok{(}\StringTok{"Não foi possível localizar a coluna de mês em \textquotesingle{}tabela\_experimental\_summary\textquotesingle{}."}\NormalTok{)}

\CommentTok{\# {-}{-}{-} função para extrair inteiro do início de uma string (tratamento de células formatadas) {-}{-}{-}}
\NormalTok{extract\_leading\_int }\OtherTok{\textless{}{-}} \ControlFlowTok{function}\NormalTok{(x) \{}
\NormalTok{  xch }\OtherTok{\textless{}{-}} \FunctionTok{as.character}\NormalTok{(x)}
  \CommentTok{\# remove tags HTML e quebras, guarda primeiro inteiro encontrado}
\NormalTok{  xch }\OtherTok{\textless{}{-}} \FunctionTok{gsub}\NormalTok{(}\StringTok{"\textless{}[\^{}\textgreater{}]+\textgreater{}"}\NormalTok{, }\StringTok{""}\NormalTok{, xch)        }\CommentTok{\# remover HTML tags}
\NormalTok{  xch }\OtherTok{\textless{}{-}} \FunctionTok{gsub}\NormalTok{(}\StringTok{"}\SpecialCharTok{\textbackslash{}\textbackslash{}\textbackslash{}\textbackslash{}}\StringTok{n"}\NormalTok{, }\StringTok{"}\SpecialCharTok{\textbackslash{}n}\StringTok{"}\NormalTok{, xch)        }\CommentTok{\# unescape se necessário}
  \FunctionTok{sapply}\NormalTok{(xch, }\ControlFlowTok{function}\NormalTok{(s) \{}
\NormalTok{    m }\OtherTok{\textless{}{-}} \FunctionTok{regmatches}\NormalTok{(s, }\FunctionTok{regexpr}\NormalTok{(}\StringTok{"}\SpecialCharTok{\textbackslash{}\textbackslash{}}\StringTok{d+"}\NormalTok{, s))}
    \ControlFlowTok{if}\NormalTok{ (}\FunctionTok{length}\NormalTok{(m) }\SpecialCharTok{==} \DecValTok{0} \SpecialCharTok{||}\NormalTok{ m }\SpecialCharTok{==} \StringTok{""}\NormalTok{) }\ConstantTok{NA\_integer\_} \ControlFlowTok{else} \FunctionTok{as.integer}\NormalTok{(m)}
\NormalTok{  \}, }\AttributeTok{USE.NAMES =} \ConstantTok{FALSE}\NormalTok{)}
\NormalTok{\}}

\CommentTok{\# tentar obter contagens numéricas diretamente}
\NormalTok{ing\_vec }\OtherTok{\textless{}{-}} \ControlFlowTok{if}\NormalTok{ (}\SpecialCharTok{!}\FunctionTok{is.na}\NormalTok{(col\_ing) }\SpecialCharTok{\&\&} \FunctionTok{is.numeric}\NormalTok{(tab\_in[[col\_ing]])) }\FunctionTok{as.integer}\NormalTok{(tab\_in[[col\_ing]]) }\ControlFlowTok{else} \ConstantTok{NULL}
\NormalTok{not\_vec }\OtherTok{\textless{}{-}} \ControlFlowTok{if}\NormalTok{ (}\SpecialCharTok{!}\FunctionTok{is.na}\NormalTok{(col\_not) }\SpecialCharTok{\&\&} \FunctionTok{is.numeric}\NormalTok{(tab\_in[[col\_not]])) }\FunctionTok{as.integer}\NormalTok{(tab\_in[[col\_not]]) }\ControlFlowTok{else} \ConstantTok{NULL}

\CommentTok{\# se não numéricas, tentar extrair de strings formatadas (ex.: "47\textless{}br/\textgreater{}(63.51\%)" ou "47\textbackslash{}n(63.51\%)")}
\ControlFlowTok{if}\NormalTok{ (}\FunctionTok{is.null}\NormalTok{(ing\_vec) }\SpecialCharTok{\&\&} \SpecialCharTok{!}\FunctionTok{is.na}\NormalTok{(col\_ing)) ing\_vec }\OtherTok{\textless{}{-}} \FunctionTok{extract\_leading\_int}\NormalTok{(tab\_in[[col\_ing]])}
\ControlFlowTok{if}\NormalTok{ (}\FunctionTok{is.null}\NormalTok{(not\_vec) }\SpecialCharTok{\&\&} \SpecialCharTok{!}\FunctionTok{is.na}\NormalTok{(col\_not)) not\_vec }\OtherTok{\textless{}{-}} \FunctionTok{extract\_leading\_int}\NormalTok{(tab\_in[[col\_not]])}

\CommentTok{\# se ainda NULL, tentar detectar colunas originais ing\_n / not\_n antes de formatação}
\ControlFlowTok{if}\NormalTok{ (}\FunctionTok{is.null}\NormalTok{(ing\_vec)) \{}
  \CommentTok{\# talvez o objeto contenha colunas ing\_n / not\_n sem formatação}
\NormalTok{  alt\_ing }\OtherTok{\textless{}{-}} \FunctionTok{intersect}\NormalTok{(}\FunctionTok{names}\NormalTok{(tab\_in), }\FunctionTok{c}\NormalTok{(}\StringTok{"ing\_n"}\NormalTok{,}\StringTok{"ingeriram"}\NormalTok{,}\StringTok{"ingeriram\_suco"}\NormalTok{))}
  \ControlFlowTok{if}\NormalTok{ (}\FunctionTok{length}\NormalTok{(alt\_ing)) ing\_vec }\OtherTok{\textless{}{-}} \FunctionTok{extract\_leading\_int}\NormalTok{(tab\_in[[alt\_ing[}\DecValTok{1}\NormalTok{]]])}
\NormalTok{\}}
\ControlFlowTok{if}\NormalTok{ (}\FunctionTok{is.null}\NormalTok{(not\_vec)) \{}
\NormalTok{  alt\_not }\OtherTok{\textless{}{-}} \FunctionTok{intersect}\NormalTok{(}\FunctionTok{names}\NormalTok{(tab\_in), }\FunctionTok{c}\NormalTok{(}\StringTok{"not\_n"}\NormalTok{,}\StringTok{"nao\_n"}\NormalTok{,}\StringTok{"nao\_ingere"}\NormalTok{))}
  \ControlFlowTok{if}\NormalTok{ (}\FunctionTok{length}\NormalTok{(alt\_not)) not\_vec }\OtherTok{\textless{}{-}} \FunctionTok{extract\_leading\_int}\NormalTok{(tab\_in[[alt\_not[}\DecValTok{1}\NormalTok{]]])}
\NormalTok{\}}

\CommentTok{\# se não conseguimos obter contagens, aborta com mensagem informativa}
\ControlFlowTok{if}\NormalTok{ (}\FunctionTok{is.null}\NormalTok{(ing\_vec) }\SpecialCharTok{||} \FunctionTok{is.null}\NormalTok{(not\_vec)) \{}
  \FunctionTok{stop}\NormalTok{(}\StringTok{"Não foi possível extrair contagens \textquotesingle{}ingeriram\textquotesingle{} e \textquotesingle{}não ingeriram\textquotesingle{} a partir de \textquotesingle{}tabela\_experimental\_summary\textquotesingle{}. Verifique nomes/formatos das colunas."}\NormalTok{)}
\NormalTok{\}}

\CommentTok{\# montar data.frame padronizado}
\NormalTok{df }\OtherTok{\textless{}{-}} \FunctionTok{data.frame}\NormalTok{(}
  \AttributeTok{mes\_raw =}\NormalTok{ tab\_in[[col\_mes]],}
  \AttributeTok{ing\_n =}\NormalTok{ ing\_vec,}
  \AttributeTok{not\_n =}\NormalTok{ not\_vec,}
  \AttributeTok{stringsAsFactors =} \ConstantTok{FALSE}
\NormalTok{)}

\CommentTok{\# mapear meses e ordenar cronologicamente}
\NormalTok{month\_levels }\OtherTok{\textless{}{-}} \FunctionTok{c}\NormalTok{(}\StringTok{"Abril"}\NormalTok{,}\StringTok{"Maio"}\NormalTok{,}\StringTok{"Junho"}\NormalTok{,}\StringTok{"Julho"}\NormalTok{,}\StringTok{"Agosto"}\NormalTok{,}\StringTok{"Setembro"}\NormalTok{,}\StringTok{"Outubro"}\NormalTok{,}\StringTok{"Novembro"}\NormalTok{,}\StringTok{"Dezembro"}\NormalTok{)}
\NormalTok{map\_month }\OtherTok{\textless{}{-}} \ControlFlowTok{function}\NormalTok{(x) \{}
\NormalTok{  s }\OtherTok{\textless{}{-}} \FunctionTok{trimws}\NormalTok{(}\FunctionTok{as.character}\NormalTok{(x))}
\NormalTok{  idx }\OtherTok{\textless{}{-}} \FunctionTok{match}\NormalTok{(}\FunctionTok{tolower}\NormalTok{(s), }\FunctionTok{tolower}\NormalTok{(month\_levels))}
  \ControlFlowTok{if}\NormalTok{ (}\SpecialCharTok{!}\FunctionTok{is.na}\NormalTok{(idx)) }\FunctionTok{return}\NormalTok{(month\_levels[idx])}
\NormalTok{  s\_clean }\OtherTok{\textless{}{-}} \FunctionTok{iconv}\NormalTok{(s, }\AttributeTok{to =} \StringTok{"ASCII//TRANSLIT"}\NormalTok{)}
\NormalTok{  idx2 }\OtherTok{\textless{}{-}} \FunctionTok{match}\NormalTok{(}\FunctionTok{tolower}\NormalTok{(s\_clean), }\FunctionTok{tolower}\NormalTok{(}\FunctionTok{iconv}\NormalTok{(month\_levels, }\AttributeTok{to =} \StringTok{"ASCII//TRANSLIT"}\NormalTok{)))}
  \ControlFlowTok{if}\NormalTok{ (}\SpecialCharTok{!}\FunctionTok{is.na}\NormalTok{(idx2)) }\FunctionTok{return}\NormalTok{(month\_levels[idx2])}
  \FunctionTok{return}\NormalTok{(}\ConstantTok{NA\_character\_}\NormalTok{)}
\NormalTok{\}}
\NormalTok{df}\SpecialCharTok{$}\NormalTok{mes }\OtherTok{\textless{}{-}} \FunctionTok{vapply}\NormalTok{(df}\SpecialCharTok{$}\NormalTok{mes\_raw, map\_month, }\FunctionTok{character}\NormalTok{(}\DecValTok{1}\NormalTok{))}
\NormalTok{df }\OtherTok{\textless{}{-}}\NormalTok{ df }\SpecialCharTok{\%\textgreater{}\%} \FunctionTok{filter}\NormalTok{(}\SpecialCharTok{!}\FunctionTok{is.na}\NormalTok{(mes))}
\NormalTok{df}\SpecialCharTok{$}\NormalTok{mes }\OtherTok{\textless{}{-}} \FunctionTok{factor}\NormalTok{(df}\SpecialCharTok{$}\NormalTok{mes, }\AttributeTok{levels =}\NormalTok{ month\_levels)}

\CommentTok{\# agregar por mês (caso haja múltiplas linhas)}
\NormalTok{per\_month }\OtherTok{\textless{}{-}}\NormalTok{ df }\SpecialCharTok{\%\textgreater{}\%}
  \FunctionTok{group\_by}\NormalTok{(mes) }\SpecialCharTok{\%\textgreater{}\%}
  \FunctionTok{summarise}\NormalTok{(}
    \AttributeTok{successes =} \FunctionTok{sum}\NormalTok{(}\FunctionTok{as.integer}\NormalTok{(ing\_n), }\AttributeTok{na.rm =} \ConstantTok{TRUE}\NormalTok{),}
    \AttributeTok{total =} \FunctionTok{sum}\NormalTok{(}\FunctionTok{as.integer}\NormalTok{(ing\_n) }\SpecialCharTok{+} \FunctionTok{as.integer}\NormalTok{(not\_n), }\AttributeTok{na.rm =} \ConstantTok{TRUE}\NormalTok{),}
    \AttributeTok{.groups =} \StringTok{"drop"}
\NormalTok{  ) }\SpecialCharTok{\%\textgreater{}\%}
  \CommentTok{\# garantir todos os meses na sequência}
  \FunctionTok{complete}\NormalTok{(}\AttributeTok{mes =} \FunctionTok{factor}\NormalTok{(month\_levels, }\AttributeTok{levels =}\NormalTok{ month\_levels), }\AttributeTok{fill =} \FunctionTok{list}\NormalTok{(}\AttributeTok{successes =} \DecValTok{0}\NormalTok{, }\AttributeTok{total =} \DecValTok{0}\NormalTok{)) }\SpecialCharTok{\%\textgreater{}\%}
  \FunctionTok{arrange}\NormalTok{(mes)}

\CommentTok{\# calcular proporção e IC95\% (prop.test) por mês}
\NormalTok{per\_month }\OtherTok{\textless{}{-}}\NormalTok{ per\_month }\SpecialCharTok{\%\textgreater{}\%}
  \FunctionTok{rowwise}\NormalTok{() }\SpecialCharTok{\%\textgreater{}\%}
  \FunctionTok{mutate}\NormalTok{(}
    \AttributeTok{prop =} \ControlFlowTok{if}\NormalTok{ (total }\SpecialCharTok{\textgreater{}} \DecValTok{0}\NormalTok{) successes }\SpecialCharTok{/}\NormalTok{ total }\ControlFlowTok{else} \ConstantTok{NA\_real\_}\NormalTok{,}
    \AttributeTok{ci =} \FunctionTok{list}\NormalTok{(}\ControlFlowTok{if}\NormalTok{ (total }\SpecialCharTok{\textgreater{}} \DecValTok{0}\NormalTok{) \{}
\NormalTok{      res }\OtherTok{\textless{}{-}} \FunctionTok{tryCatch}\NormalTok{(}\FunctionTok{prop.test}\NormalTok{(successes, total, }\AttributeTok{correct =} \ConstantTok{FALSE}\NormalTok{), }\AttributeTok{error =} \ControlFlowTok{function}\NormalTok{(e) }\ConstantTok{NULL}\NormalTok{)}
      \ControlFlowTok{if}\NormalTok{ (}\FunctionTok{is.null}\NormalTok{(res)) }\FunctionTok{c}\NormalTok{(}\ConstantTok{NA\_real\_}\NormalTok{, }\ConstantTok{NA\_real\_}\NormalTok{) }\ControlFlowTok{else} \FunctionTok{as.numeric}\NormalTok{(res}\SpecialCharTok{$}\NormalTok{conf.int)}
\NormalTok{    \} }\ControlFlowTok{else} \FunctionTok{c}\NormalTok{(}\ConstantTok{NA\_real\_}\NormalTok{, }\ConstantTok{NA\_real\_}\NormalTok{)),}
    \AttributeTok{ci\_low =}\NormalTok{ ci[[}\DecValTok{1}\NormalTok{]],}
    \AttributeTok{ci\_up  =}\NormalTok{ ci[[}\DecValTok{2}\NormalTok{]]}
\NormalTok{  ) }\SpecialCharTok{\%\textgreater{}\%}
  \FunctionTok{ungroup}\NormalTok{()}

\CommentTok{\# salvar tabela de estatísticas}
\FunctionTok{assign}\NormalTok{(}\StringTok{"ts\_suco\_tratamento\_table"}\NormalTok{, per\_month, }\AttributeTok{envir =}\NormalTok{ .GlobalEnv)}

\CommentTok{\# {-}{-}{-} Plot: linha com pontos e IC95\% {-}{-}{-}}
\NormalTok{p }\OtherTok{\textless{}{-}} \FunctionTok{ggplot}\NormalTok{(per\_month, }\FunctionTok{aes}\NormalTok{(}\AttributeTok{x =}\NormalTok{ mes, }\AttributeTok{y =}\NormalTok{ prop, }\AttributeTok{group =} \DecValTok{1}\NormalTok{)) }\SpecialCharTok{+}
  \FunctionTok{geom\_line}\NormalTok{(}\AttributeTok{color =} \StringTok{"\#E31A1C"}\NormalTok{, }\AttributeTok{size =} \FloatTok{0.9}\NormalTok{, }\AttributeTok{na.rm =} \ConstantTok{TRUE}\NormalTok{) }\SpecialCharTok{+}
  \FunctionTok{geom\_point}\NormalTok{(}\AttributeTok{color =} \StringTok{"\#E31A1C"}\NormalTok{, }\AttributeTok{size =} \DecValTok{3}\NormalTok{, }\AttributeTok{na.rm =} \ConstantTok{TRUE}\NormalTok{) }\SpecialCharTok{+}
  \FunctionTok{geom\_errorbar}\NormalTok{(}\FunctionTok{aes}\NormalTok{(}\AttributeTok{ymin =}\NormalTok{ ci\_low, }\AttributeTok{ymax =}\NormalTok{ ci\_up), }\AttributeTok{width =} \FloatTok{0.12}\NormalTok{, }\AttributeTok{size =} \FloatTok{0.7}\NormalTok{, }\AttributeTok{na.rm =} \ConstantTok{TRUE}\NormalTok{) }\SpecialCharTok{+}
  \FunctionTok{geom\_text}\NormalTok{(}\FunctionTok{aes}\NormalTok{(}\AttributeTok{label =} \FunctionTok{ifelse}\NormalTok{(}\SpecialCharTok{!}\FunctionTok{is.na}\NormalTok{(prop), }\FunctionTok{paste0}\NormalTok{(}\FunctionTok{sprintf}\NormalTok{(}\StringTok{"\%.1f\%\%"}\NormalTok{, }\DecValTok{100}\SpecialCharTok{*}\NormalTok{prop), }\StringTok{" (n="}\NormalTok{, total, }\StringTok{")"}\NormalTok{), }\StringTok{""}\NormalTok{)),}
            \AttributeTok{vjust =} \SpecialCharTok{{-}}\FloatTok{1.1}\NormalTok{, }\AttributeTok{size =} \DecValTok{3}\NormalTok{) }\SpecialCharTok{+}
  \FunctionTok{scale\_y\_continuous}\NormalTok{(}\AttributeTok{labels =} \FunctionTok{percent\_format}\NormalTok{(}\AttributeTok{accuracy =} \DecValTok{1}\NormalTok{),}
                     \AttributeTok{breaks =} \FunctionTok{seq}\NormalTok{(}\DecValTok{0}\NormalTok{,}\DecValTok{1}\NormalTok{,}\AttributeTok{by=}\FloatTok{0.1}\NormalTok{),}
                     \AttributeTok{limits =} \FunctionTok{c}\NormalTok{(}\DecValTok{0}\NormalTok{,}\FloatTok{1.1}\NormalTok{), }\AttributeTok{expand =} \FunctionTok{c}\NormalTok{(}\DecValTok{0}\NormalTok{,}\DecValTok{0}\NormalTok{)) }\SpecialCharTok{+}
  \FunctionTok{labs}\NormalTok{(}
    \AttributeTok{title =} \StringTok{"Série temporal — Proporção de ingestão de suco (Grupo Experimental)"}\NormalTok{,}
    \AttributeTok{subtitle =} \StringTok{"Pontos: proporção mensal de \textquotesingle{}Sim\textquotesingle{}; barras: IC95\%{-}prop.test (n=288/354)"}\NormalTok{,}
    \AttributeTok{x =} \StringTok{"Mês"}\NormalTok{,}
    \AttributeTok{y =} \StringTok{"Proporção de ingestão de suco"}\NormalTok{,}
    \AttributeTok{caption =} \StringTok{"Fonte: tabela\_experimental\_summary"}
\NormalTok{  ) }\SpecialCharTok{+}
  \FunctionTok{theme\_minimal}\NormalTok{() }\SpecialCharTok{+}
  \FunctionTok{theme}\NormalTok{(}
    \AttributeTok{plot.title =} \FunctionTok{element\_text}\NormalTok{(}\AttributeTok{hjust =} \FloatTok{0.5}\NormalTok{),}
    \AttributeTok{plot.subtitle =} \FunctionTok{element\_text}\NormalTok{(}\AttributeTok{hjust =} \FloatTok{0.5}\NormalTok{),}
    \AttributeTok{axis.text.y =} \FunctionTok{element\_text}\NormalTok{(}\AttributeTok{size =} \DecValTok{8}\NormalTok{)}
\NormalTok{  )}

\CommentTok{\# salvar objeto de plot}
\FunctionTok{assign}\NormalTok{(}\StringTok{"ts\_suco\_tratamento\_plot"}\NormalTok{, p, }\AttributeTok{envir =}\NormalTok{ .GlobalEnv)}

\CommentTok{\# exibir}
\FunctionTok{print}\NormalTok{(p)}
\InformationTok{\textasciigrave{}\textasciigrave{}\textasciigrave{}}
\end{Highlighting}
\end{Shaded}

\pandocbounded{\includegraphics[keepaspectratio]{replicar-suco-uva-tab-graf_files/figure-pdf/unnamed-chunk-7-1.pdf}}

O gráfico acima evidencia a falha na troca do \emph{nudge} utilizado
apenas no mês de junho/2018.

Ele também evidencia a falta de explicação para a \ul{\textbf{ausência
de significância estatística}} verificada no \ul{\textbf{teste
qui-quadrado}} para os \textbf{meses} de \ul{\textbf{setembro e
outubro/2018}}. O que reforça ainda mais a \textbf{\emph{necessidade de
replicar o quasi-experimento}} a fim de testar se isso poderia ter sido
provocado pela simples alea presente na variabilidade amostral, que
acentua-se no caso de amostras pequenas, como ocorreu em outubro e
dezembro (n = 14); mas não ocorreu em setembro/2018 (n = 43).

\subsection{Gráfico do efeito dos conciliadores no
GT}\label{gruxe1fico-do-efeito-dos-conciliadores-no-gt}

Replicar tabela 5 - Desempenho dos conciliadores -- Tabelas com
qui-quadrado, do artigo (TOMÁS, 2020 , p.~222).

\begin{Shaded}
\begin{Highlighting}[numbers=left,,]
\InformationTok{\textasciigrave{}\textasciigrave{}\textasciigrave{}\{r\}}
\CommentTok{\# Recria tabela Observado + Esperado por conciliador (A, B, C, D)}
\CommentTok{\# e formata expected com 1 casa decimal}
\CommentTok{\# Saída: tabela\_conciliadores\_display, obs\_mat\_conciliadores, expected\_mat\_conciliadores, chisq\_res\_conciliadores}

\ControlFlowTok{if}\NormalTok{ (}\SpecialCharTok{!}\FunctionTok{requireNamespace}\NormalTok{(}\StringTok{"knitr"}\NormalTok{, }\AttributeTok{quietly =} \ConstantTok{TRUE}\NormalTok{)) }\FunctionTok{install.packages}\NormalTok{(}\StringTok{"knitr"}\NormalTok{)}
\FunctionTok{library}\NormalTok{(knitr)}

\CommentTok{\# {-}{-}{-} Dados observados (copiados da imagem) {-}{-}{-}}
\NormalTok{obs\_mat }\OtherTok{\textless{}{-}} \FunctionTok{matrix}\NormalTok{(}
  \FunctionTok{c}\NormalTok{(}
    \DecValTok{35}\NormalTok{, }\DecValTok{28}\NormalTok{, }\DecValTok{21}\NormalTok{, }\DecValTok{21}\NormalTok{,   }\CommentTok{\# Controle}
    \DecValTok{25}\NormalTok{, }\DecValTok{37}\NormalTok{, }\DecValTok{65}\NormalTok{, }\DecValTok{89}    \CommentTok{\# Experimental}
\NormalTok{  ),}
  \AttributeTok{nrow =} \DecValTok{2}\NormalTok{,}
  \AttributeTok{byrow =} \ConstantTok{TRUE}
\NormalTok{)}
\FunctionTok{rownames}\NormalTok{(obs\_mat) }\OtherTok{\textless{}{-}} \FunctionTok{c}\NormalTok{(}\StringTok{"Controle"}\NormalTok{, }\StringTok{"Experimental"}\NormalTok{)}
\FunctionTok{colnames}\NormalTok{(obs\_mat) }\OtherTok{\textless{}{-}} \FunctionTok{c}\NormalTok{(}\StringTok{"Conciliador A"}\NormalTok{, }\StringTok{"Conciliador B"}\NormalTok{, }\StringTok{"Conciliador C"}\NormalTok{, }\StringTok{"Conciliador D"}\NormalTok{)}

\CommentTok{\# {-}{-}{-} Totais e expected (numéricos para cálculo) {-}{-}{-}}
\NormalTok{tot\_col }\OtherTok{\textless{}{-}} \FunctionTok{colSums}\NormalTok{(obs\_mat)}
\NormalTok{tot\_row }\OtherTok{\textless{}{-}} \FunctionTok{rowSums}\NormalTok{(obs\_mat)}
\NormalTok{grand\_total }\OtherTok{\textless{}{-}} \FunctionTok{sum}\NormalTok{(obs\_mat)}

\NormalTok{expected\_mat }\OtherTok{\textless{}{-}} \FunctionTok{outer}\NormalTok{(tot\_row, tot\_col) }\SpecialCharTok{/}\NormalTok{ grand\_total}
\FunctionTok{rownames}\NormalTok{(expected\_mat) }\OtherTok{\textless{}{-}} \FunctionTok{rownames}\NormalTok{(obs\_mat)}
\FunctionTok{colnames}\NormalTok{(expected\_mat) }\OtherTok{\textless{}{-}} \FunctionTok{colnames}\NormalTok{(obs\_mat)}

\CommentTok{\# chi{-}squared test (Pearson)}
\NormalTok{chisq\_res }\OtherTok{\textless{}{-}} \FunctionTok{suppressWarnings}\NormalTok{(}\FunctionTok{chisq.test}\NormalTok{(obs\_mat, }\AttributeTok{correct =} \ConstantTok{FALSE}\NormalTok{))}

\CommentTok{\# {-}{-}{-} Preparar linhas para exibição {-}{-}{-}}
\CommentTok{\# Observados (inteiros)}
\NormalTok{obs\_row1 }\OtherTok{\textless{}{-}} \FunctionTok{as.character}\NormalTok{(obs\_mat[}\DecValTok{1}\NormalTok{, ])}
\NormalTok{obs\_row2 }\OtherTok{\textless{}{-}} \FunctionTok{as.character}\NormalTok{(obs\_mat[}\DecValTok{2}\NormalTok{, ])}

\CommentTok{\# Esperados formatados com 1 casa decimal PARA EXIBIÇÃO}
\NormalTok{exp\_row1\_fmt }\OtherTok{\textless{}{-}} \FunctionTok{formatC}\NormalTok{(expected\_mat[}\DecValTok{1}\NormalTok{, ], }\AttributeTok{format =} \StringTok{"f"}\NormalTok{, }\AttributeTok{digits =} \DecValTok{1}\NormalTok{)}
\NormalTok{exp\_row2\_fmt }\OtherTok{\textless{}{-}} \FunctionTok{formatC}\NormalTok{(expected\_mat[}\DecValTok{2}\NormalTok{, ], }\AttributeTok{format =} \StringTok{"f"}\NormalTok{, }\AttributeTok{digits =} \DecValTok{1}\NormalTok{)}

\CommentTok{\# Totais}
\NormalTok{tot\_col\_fmt }\OtherTok{\textless{}{-}} \FunctionTok{as.character}\NormalTok{(tot\_col)}
\NormalTok{total\_obs\_row1 }\OtherTok{\textless{}{-}} \FunctionTok{as.character}\NormalTok{(tot\_row[}\DecValTok{1}\NormalTok{])}
\NormalTok{total\_obs\_row2 }\OtherTok{\textless{}{-}} \FunctionTok{as.character}\NormalTok{(tot\_row[}\DecValTok{2}\NormalTok{])}
\NormalTok{total\_expected\_row1 }\OtherTok{\textless{}{-}} \FunctionTok{formatC}\NormalTok{(}\FunctionTok{sum}\NormalTok{(expected\_mat[}\DecValTok{1}\NormalTok{, ]), }\AttributeTok{format =} \StringTok{"f"}\NormalTok{, }\AttributeTok{digits =} \DecValTok{1}\NormalTok{)}
\NormalTok{total\_expected\_row2 }\OtherTok{\textless{}{-}} \FunctionTok{formatC}\NormalTok{(}\FunctionTok{sum}\NormalTok{(expected\_mat[}\DecValTok{2}\NormalTok{, ]), }\AttributeTok{format =} \StringTok{"f"}\NormalTok{, }\AttributeTok{digits =} \DecValTok{1}\NormalTok{)}
\NormalTok{grand\_total\_fmt }\OtherTok{\textless{}{-}} \FunctionTok{as.character}\NormalTok{(grand\_total)}

\CommentTok{\# montar data.frame em ordem desejada (Controle, Esperado, Experimental, Esperado, Total)}
\NormalTok{df\_display }\OtherTok{\textless{}{-}} \FunctionTok{data.frame}\NormalTok{(}
  \AttributeTok{Grupo =} \FunctionTok{c}\NormalTok{(}\StringTok{"Controle"}\NormalTok{, }\StringTok{"Esperado"}\NormalTok{, }\StringTok{"Experimental"}\NormalTok{, }\StringTok{"Esperado"}\NormalTok{, }\StringTok{"Total"}\NormalTok{),}
  \StringTok{\textasciigrave{}}\AttributeTok{Conciliador A}\StringTok{\textasciigrave{}} \OtherTok{=} \FunctionTok{c}\NormalTok{(obs\_row1[}\DecValTok{1}\NormalTok{], exp\_row1\_fmt[}\DecValTok{1}\NormalTok{], obs\_row2[}\DecValTok{1}\NormalTok{], exp\_row2\_fmt[}\DecValTok{1}\NormalTok{], tot\_col\_fmt[}\DecValTok{1}\NormalTok{]),}
  \StringTok{\textasciigrave{}}\AttributeTok{Conciliador B}\StringTok{\textasciigrave{}} \OtherTok{=} \FunctionTok{c}\NormalTok{(obs\_row1[}\DecValTok{2}\NormalTok{], exp\_row1\_fmt[}\DecValTok{2}\NormalTok{], obs\_row2[}\DecValTok{2}\NormalTok{], exp\_row2\_fmt[}\DecValTok{2}\NormalTok{], tot\_col\_fmt[}\DecValTok{2}\NormalTok{]),}
  \StringTok{\textasciigrave{}}\AttributeTok{Conciliador C}\StringTok{\textasciigrave{}} \OtherTok{=} \FunctionTok{c}\NormalTok{(obs\_row1[}\DecValTok{3}\NormalTok{], exp\_row1\_fmt[}\DecValTok{3}\NormalTok{], obs\_row2[}\DecValTok{3}\NormalTok{], exp\_row2\_fmt[}\DecValTok{3}\NormalTok{], tot\_col\_fmt[}\DecValTok{3}\NormalTok{]),}
  \StringTok{\textasciigrave{}}\AttributeTok{Conciliador D}\StringTok{\textasciigrave{}} \OtherTok{=} \FunctionTok{c}\NormalTok{(obs\_row1[}\DecValTok{4}\NormalTok{], exp\_row1\_fmt[}\DecValTok{4}\NormalTok{], obs\_row2[}\DecValTok{4}\NormalTok{], exp\_row2\_fmt[}\DecValTok{4}\NormalTok{], tot\_col\_fmt[}\DecValTok{4}\NormalTok{]),}
  \AttributeTok{Total =} \FunctionTok{c}\NormalTok{(total\_obs\_row1, total\_expected\_row1, total\_obs\_row2, total\_expected\_row2, grand\_total\_fmt),}
  \AttributeTok{stringsAsFactors =} \ConstantTok{FALSE}
\NormalTok{)}


\CommentTok{\# {-}{-}{-} Imprimir tabela com formato simples {-}{-}{-}}

\FunctionTok{cat}\NormalTok{(}\StringTok{"}\SpecialCharTok{\textbackslash{}n}\StringTok{Tabela: Desempenho dos conciliadores — Observado / Esperado}\SpecialCharTok{\textbackslash{}n}\StringTok{(ambos com 1 casa decimal) apenas quando houve acordo}\SpecialCharTok{\textbackslash{}n\textbackslash{}n}\StringTok{"}\NormalTok{)}
\FunctionTok{print}\NormalTok{(df\_display)}
\CommentTok{\# print(knitr::kable(df\_display, format = kable\_format, escape = FALSE, align = align\_vec, booktabs = TRUE))}

\CommentTok{\# {-}{-}{-} Imprimir resultado do qui{-}quadrado {-}{-}{-}}
\FunctionTok{cat}\NormalTok{(}\StringTok{"}\SpecialCharTok{\textbackslash{}n}\StringTok{Resultado do teste qui{-}quadrado (Pearson):}\SpecialCharTok{\textbackslash{}n}\StringTok{"}\NormalTok{)}
\FunctionTok{cat}\NormalTok{(}\FunctionTok{sprintf}\NormalTok{(}\StringTok{"X{-}squared = \%.2f, df = \%d, p{-}value = \%g}\SpecialCharTok{\textbackslash{}n\textbackslash{}n}\StringTok{"}\NormalTok{,}
            \FunctionTok{as.numeric}\NormalTok{(chisq\_res}\SpecialCharTok{$}\NormalTok{statistic), chisq\_res}\SpecialCharTok{$}\NormalTok{parameter, chisq\_res}\SpecialCharTok{$}\NormalTok{p.value))}

\CommentTok{\# salvar objetos no Global Environment}
\FunctionTok{assign}\NormalTok{(}\StringTok{"obs\_mat\_conciliadores"}\NormalTok{, obs\_mat, }\AttributeTok{envir =}\NormalTok{ .GlobalEnv)}
\FunctionTok{assign}\NormalTok{(}\StringTok{"expected\_mat\_conciliadores"}\NormalTok{, expected\_mat, }\AttributeTok{envir =}\NormalTok{ .GlobalEnv)}
\FunctionTok{assign}\NormalTok{(}\StringTok{"chisq\_res\_conciliadores"}\NormalTok{, chisq\_res, }\AttributeTok{envir =}\NormalTok{ .GlobalEnv)}
\FunctionTok{assign}\NormalTok{(}\StringTok{"tabela\_conciliadores\_display"}\NormalTok{, df\_display, }\AttributeTok{envir =}\NormalTok{ .GlobalEnv)}
\InformationTok{\textasciigrave{}\textasciigrave{}\textasciigrave{}}
\end{Highlighting}
\end{Shaded}

\begin{verbatim}

Tabela: Desempenho dos conciliadores — Observado / Esperado
(ambos com 1 casa decimal) apenas quando houve acordo

         Grupo Conciliador.A Conciliador.B Conciliador.C Conciliador.D Total
1     Controle            35            28            21            21   105
2     Esperado          19.6          21.3          28.1          36.0 105.0
3 Experimental            25            37            65            89   216
4     Esperado          40.4          43.7          57.9          74.0 216.0
5        Total            60            65            86           110   321

Resultado do teste qui-quadrado (Pearson):
X-squared = 33.03, df = 3, p-value = 3.17907e-07
\end{verbatim}

Observar que o total de audiências presidida pelos quatro conciliadores
foi \textbf{321}.

O que é bem menor que o total geral de audiências do esperimento:
\textbf{659}.

Uma diferença de \textbf{-321} audiências, que foi parcialmente exlicada
no artigo.

\begin{quote}
``Durante o primeiro mês da pesquisa (abril de 2018), nenhum registro
foi feito em relação aos conciliadores. Apenas a partir de maio de 2018
a variável `conciliador' começou a ser mensurada, não havendo informação
sobre a atuação de conciliadores em \emph{\textbf{140} audiências
realizadas em abril}.'' (TOMÁS, 2020 , p.~222)
\end{quote}

Gerar gráfico de barras lado a lado para exibir o desempnho dos
conciliadores em audiências com acordo no Grupo de Controle e
Experimental.

\begin{Shaded}
\begin{Highlighting}[numbers=left,,]
\InformationTok{\textasciigrave{}\textasciigrave{}\textasciigrave{}\{r\}}
\CommentTok{\# Script corrigido: proporções de acordos por conciliador (A,B,C,D)}
\CommentTok{\# Corrige bug que mapeava erroneamente vários rótulos para "C" (ex.: detectava \textquotesingle{}C\textquotesingle{} em "Controle")}
\CommentTok{\#}
\CommentTok{\# Saída: df\_counts\_conciliadores, plot\_long\_conciliadores, p\_conciliadores\_proporcao}

\ControlFlowTok{if}\NormalTok{ (}\SpecialCharTok{!}\FunctionTok{requireNamespace}\NormalTok{(}\StringTok{"ggplot2"}\NormalTok{, }\AttributeTok{quietly =} \ConstantTok{TRUE}\NormalTok{)) }\FunctionTok{install.packages}\NormalTok{(}\StringTok{"ggplot2"}\NormalTok{)}
\ControlFlowTok{if}\NormalTok{ (}\SpecialCharTok{!}\FunctionTok{requireNamespace}\NormalTok{(}\StringTok{"dplyr"}\NormalTok{, }\AttributeTok{quietly =} \ConstantTok{TRUE}\NormalTok{)) }\FunctionTok{install.packages}\NormalTok{(}\StringTok{"dplyr"}\NormalTok{)}
\ControlFlowTok{if}\NormalTok{ (}\SpecialCharTok{!}\FunctionTok{requireNamespace}\NormalTok{(}\StringTok{"tidyr"}\NormalTok{, }\AttributeTok{quietly =} \ConstantTok{TRUE}\NormalTok{)) }\FunctionTok{install.packages}\NormalTok{(}\StringTok{"tidyr"}\NormalTok{)}
\ControlFlowTok{if}\NormalTok{ (}\SpecialCharTok{!}\FunctionTok{requireNamespace}\NormalTok{(}\StringTok{"scales"}\NormalTok{, }\AttributeTok{quietly =} \ConstantTok{TRUE}\NormalTok{)) }\FunctionTok{install.packages}\NormalTok{(}\StringTok{"scales"}\NormalTok{)}

\FunctionTok{library}\NormalTok{(ggplot2); }\FunctionTok{library}\NormalTok{(dplyr); }\FunctionTok{library}\NormalTok{(tidyr); }\FunctionTok{library}\NormalTok{(scales)}

\CommentTok{\# conciliadores esperados (ordem fixa)}
\NormalTok{desired }\OtherTok{\textless{}{-}} \FunctionTok{c}\NormalTok{(}\StringTok{"A"}\NormalTok{,}\StringTok{"B"}\NormalTok{,}\StringTok{"C"}\NormalTok{,}\StringTok{"D"}\NormalTok{)}

\CommentTok{\# helper: extrai inteiro inicial de string (ex: "47\textless{}br/\textgreater{}(63.51\%)" {-}\textgreater{} 47)}
\NormalTok{extract\_leading\_int }\OtherTok{\textless{}{-}} \ControlFlowTok{function}\NormalTok{(x) \{}
\NormalTok{  xch }\OtherTok{\textless{}{-}} \FunctionTok{as.character}\NormalTok{(x)}
  \FunctionTok{sapply}\NormalTok{(xch, }\ControlFlowTok{function}\NormalTok{(s) \{}
\NormalTok{    m }\OtherTok{\textless{}{-}} \FunctionTok{regmatches}\NormalTok{(s, }\FunctionTok{regexpr}\NormalTok{(}\StringTok{"}\SpecialCharTok{\textbackslash{}\textbackslash{}}\StringTok{d+"}\NormalTok{, s))}
    \ControlFlowTok{if}\NormalTok{ (}\FunctionTok{length}\NormalTok{(m) }\SpecialCharTok{==} \DecValTok{0} \SpecialCharTok{||}\NormalTok{ m }\SpecialCharTok{==} \StringTok{""}\NormalTok{) }\ConstantTok{NA\_integer\_} \ControlFlowTok{else} \FunctionTok{as.integer}\NormalTok{(m)}
\NormalTok{  \}, }\AttributeTok{USE.NAMES =} \ConstantTok{FALSE}\NormalTok{)}
\NormalTok{\}}

\CommentTok{\# CORREÇÃO: função robusta para extrair letra conciliador (A{-}D)}
\NormalTok{extract\_letter }\OtherTok{\textless{}{-}} \ControlFlowTok{function}\NormalTok{(name) \{}
  \ControlFlowTok{if}\NormalTok{ (}\FunctionTok{is.na}\NormalTok{(name)) }\FunctionTok{return}\NormalTok{(}\ConstantTok{NA\_character\_}\NormalTok{)}
\NormalTok{  s }\OtherTok{\textless{}{-}} \FunctionTok{as.character}\NormalTok{(name)}
\NormalTok{  s\_trim }\OtherTok{\textless{}{-}} \FunctionTok{trimws}\NormalTok{(s)}
  \CommentTok{\# 1) procurar letra A{-}D como token isolado (word boundary) {-} evita "Controle"}
\NormalTok{  m1 }\OtherTok{\textless{}{-}} \FunctionTok{regmatches}\NormalTok{(s\_trim, }\FunctionTok{regexpr}\NormalTok{(}\StringTok{"}\SpecialCharTok{\textbackslash{}\textbackslash{}}\StringTok{b([A{-}Da{-}d])}\SpecialCharTok{\textbackslash{}\textbackslash{}}\StringTok{b"}\NormalTok{, s\_trim, }\AttributeTok{perl =} \ConstantTok{TRUE}\NormalTok{))}
  \ControlFlowTok{if}\NormalTok{ (}\FunctionTok{length}\NormalTok{(m1) }\SpecialCharTok{\&\&} \FunctionTok{nzchar}\NormalTok{(m1)) }\FunctionTok{return}\NormalTok{(}\FunctionTok{toupper}\NormalTok{(m1))}
  \CommentTok{\# 2) procurar letra A{-}D no final do rótulo (ex.: "Conciliador A", "A")}
\NormalTok{  m2 }\OtherTok{\textless{}{-}} \FunctionTok{regmatches}\NormalTok{(s\_trim, }\FunctionTok{regexpr}\NormalTok{(}\StringTok{"([A{-}Da{-}d])}\SpecialCharTok{\textbackslash{}\textbackslash{}}\StringTok{s*$"}\NormalTok{, s\_trim, }\AttributeTok{perl =} \ConstantTok{TRUE}\NormalTok{))}
  \ControlFlowTok{if}\NormalTok{ (}\FunctionTok{length}\NormalTok{(m2) }\SpecialCharTok{\&\&} \FunctionTok{nzchar}\NormalTok{(m2)) }\FunctionTok{return}\NormalTok{(}\FunctionTok{toupper}\NormalTok{(m2))}
  \CommentTok{\# 3) procurar padrão "conciliador ... A" ou "conciliador\_a"}
\NormalTok{  m3 }\OtherTok{\textless{}{-}} \FunctionTok{regmatches}\NormalTok{(s\_trim, }\FunctionTok{regexpr}\NormalTok{(}\StringTok{"conciliador[\^{}A{-}Za{-}z0{-}9]*([A{-}Da{-}d])"}\NormalTok{, s\_trim, }\AttributeTok{ignore.case =} \ConstantTok{TRUE}\NormalTok{, }\AttributeTok{perl =} \ConstantTok{TRUE}\NormalTok{))}
  \ControlFlowTok{if}\NormalTok{ (}\FunctionTok{length}\NormalTok{(m3) }\SpecialCharTok{\&\&} \FunctionTok{nzchar}\NormalTok{(m3)) \{}
    \CommentTok{\# extract captured group}
\NormalTok{    cap }\OtherTok{\textless{}{-}} \FunctionTok{sub}\NormalTok{(}\StringTok{".*([A{-}Da{-}d]).*$"}\NormalTok{, }\StringTok{"}\SpecialCharTok{\textbackslash{}\textbackslash{}}\StringTok{1"}\NormalTok{, m3)}
    \FunctionTok{return}\NormalTok{(}\FunctionTok{toupper}\NormalTok{(cap))}
\NormalTok{  \}}
  \CommentTok{\# 4) fallback: primeira letter of last token if it matches A{-}D}
\NormalTok{  toks }\OtherTok{\textless{}{-}} \FunctionTok{unlist}\NormalTok{(}\FunctionTok{strsplit}\NormalTok{(}\FunctionTok{gsub}\NormalTok{(}\StringTok{"[\^{}A{-}Za{-}z0{-}9 ]"}\NormalTok{, }\StringTok{" "}\NormalTok{, s\_trim), }\StringTok{"}\SpecialCharTok{\textbackslash{}\textbackslash{}}\StringTok{s+"}\NormalTok{))}
  \ControlFlowTok{if}\NormalTok{ (}\FunctionTok{length}\NormalTok{(toks) }\SpecialCharTok{\textgreater{}} \DecValTok{0}\NormalTok{) \{}
\NormalTok{    last }\OtherTok{\textless{}{-}}\NormalTok{ toks[}\FunctionTok{length}\NormalTok{(toks)]}
\NormalTok{    fl }\OtherTok{\textless{}{-}} \FunctionTok{toupper}\NormalTok{(}\FunctionTok{substr}\NormalTok{(last, }\DecValTok{1}\NormalTok{, }\DecValTok{1}\NormalTok{))}
    \ControlFlowTok{if}\NormalTok{ (fl }\SpecialCharTok{\%in\%}\NormalTok{ desired) }\FunctionTok{return}\NormalTok{(fl)}
\NormalTok{  \}}
  \CommentTok{\# nothing found}
  \FunctionTok{return}\NormalTok{(}\ConstantTok{NA\_character\_}\NormalTok{)}
\NormalTok{\}}

\NormalTok{df\_counts }\OtherTok{\textless{}{-}} \ConstantTok{NULL}

\CommentTok{\# 1) usar obs\_mat\_conciliadores (matrix) se existir}
\ControlFlowTok{if}\NormalTok{ (}\FunctionTok{exists}\NormalTok{(}\StringTok{"obs\_mat\_conciliadores"}\NormalTok{, }\AttributeTok{envir =}\NormalTok{ .GlobalEnv)) \{}
\NormalTok{  obs\_mat }\OtherTok{\textless{}{-}} \FunctionTok{get}\NormalTok{(}\StringTok{"obs\_mat\_conciliadores"}\NormalTok{, }\AttributeTok{envir =}\NormalTok{ .GlobalEnv)}
\NormalTok{  cn }\OtherTok{\textless{}{-}} \FunctionTok{colnames}\NormalTok{(obs\_mat)}
  \ControlFlowTok{if}\NormalTok{ (}\FunctionTok{is.null}\NormalTok{(cn)) cn }\OtherTok{\textless{}{-}} \FunctionTok{paste0}\NormalTok{(}\StringTok{"Col"}\NormalTok{, }\FunctionTok{seq\_len}\NormalTok{(}\FunctionTok{ncol}\NormalTok{(obs\_mat)))}
\NormalTok{  letters }\OtherTok{\textless{}{-}} \FunctionTok{sapply}\NormalTok{(cn, extract\_letter, }\AttributeTok{USE.NAMES =} \ConstantTok{FALSE}\NormalTok{)}
  \CommentTok{\# se não extraiu corretamente para alguma coluna, tentar pegar último caractere}
\NormalTok{  letters[}\FunctionTok{is.na}\NormalTok{(letters)] }\OtherTok{\textless{}{-}} \FunctionTok{toupper}\NormalTok{(}\FunctionTok{substr}\NormalTok{(}\FunctionTok{gsub}\NormalTok{(}\StringTok{"[\^{}A{-}Za{-}z0{-}9]"}\NormalTok{, }\StringTok{""}\NormalTok{, cn[}\FunctionTok{is.na}\NormalTok{(letters)]), }\DecValTok{1}\NormalTok{, }\DecValTok{1}\NormalTok{))}
\NormalTok{  df\_counts }\OtherTok{\textless{}{-}} \FunctionTok{data.frame}\NormalTok{(}
    \AttributeTok{conciliador =}\NormalTok{ letters,}
    \AttributeTok{Controle =} \FunctionTok{as.integer}\NormalTok{(}\ControlFlowTok{if}\NormalTok{ (}\StringTok{"Controle"} \SpecialCharTok{\%in\%} \FunctionTok{rownames}\NormalTok{(obs\_mat)) obs\_mat[}\StringTok{"Controle"}\NormalTok{, , }\AttributeTok{drop =} \ConstantTok{TRUE}\NormalTok{]}
                         \ControlFlowTok{else} \ControlFlowTok{if}\NormalTok{ (}\StringTok{"Control"} \SpecialCharTok{\%in\%} \FunctionTok{rownames}\NormalTok{(obs\_mat)) obs\_mat[}\StringTok{"Control"}\NormalTok{, , }\AttributeTok{drop =} \ConstantTok{TRUE}\NormalTok{]}
                         \ControlFlowTok{else} \FunctionTok{rep}\NormalTok{(}\DecValTok{0}\NormalTok{, }\FunctionTok{ncol}\NormalTok{(obs\_mat))),}
    \AttributeTok{Experimental =} \FunctionTok{as.integer}\NormalTok{(}\ControlFlowTok{if}\NormalTok{ (}\StringTok{"Experimental"} \SpecialCharTok{\%in\%} \FunctionTok{rownames}\NormalTok{(obs\_mat)) obs\_mat[}\StringTok{"Experimental"}\NormalTok{, , }\AttributeTok{drop =} \ConstantTok{TRUE}\NormalTok{]}
                              \ControlFlowTok{else} \FunctionTok{rep}\NormalTok{(}\DecValTok{0}\NormalTok{, }\FunctionTok{ncol}\NormalTok{(obs\_mat))),}
    \AttributeTok{stringsAsFactors =} \ConstantTok{FALSE}
\NormalTok{  )}
\NormalTok{\}}

\CommentTok{\# 2) tentar tabela\_conciliadores\_summary se existir}
\ControlFlowTok{if}\NormalTok{ (}\FunctionTok{is.null}\NormalTok{(df\_counts) }\SpecialCharTok{\&\&} \FunctionTok{exists}\NormalTok{(}\StringTok{"tabela\_conciliadores\_summary"}\NormalTok{, }\AttributeTok{envir =}\NormalTok{ .GlobalEnv)) \{}
\NormalTok{  tab }\OtherTok{\textless{}{-}} \FunctionTok{get}\NormalTok{(}\StringTok{"tabela\_conciliadores\_summary"}\NormalTok{, }\AttributeTok{envir =}\NormalTok{ .GlobalEnv)}
\NormalTok{  conciliador\_col }\OtherTok{\textless{}{-}} \FunctionTok{names}\NormalTok{(tab)[}\DecValTok{1}\NormalTok{]}
  \CommentTok{\# tentar localizar colunas numéricas de ingere/not}
\NormalTok{  num\_cols }\OtherTok{\textless{}{-}} \FunctionTok{names}\NormalTok{(tab)[}\FunctionTok{sapply}\NormalTok{(tab, is.numeric)]}
  \ControlFlowTok{if}\NormalTok{ (}\FunctionTok{length}\NormalTok{(num\_cols) }\SpecialCharTok{\textgreater{}=} \DecValTok{2}\NormalTok{) \{}
    \CommentTok{\# presumir que uma coluna é Experimental e outra Controle (usuário deve verificar)}
\NormalTok{    df\_counts }\OtherTok{\textless{}{-}} \FunctionTok{data.frame}\NormalTok{(}
      \AttributeTok{conciliador =} \FunctionTok{sapply}\NormalTok{(tab[[conciliador\_col]], extract\_letter),}
      \AttributeTok{Experimental =} \FunctionTok{as.integer}\NormalTok{(tab[[num\_cols[}\DecValTok{1}\NormalTok{]]]),}
      \AttributeTok{Controle =} \FunctionTok{as.integer}\NormalTok{(tab[[num\_cols[}\DecValTok{2}\NormalTok{]]]),}
      \AttributeTok{stringsAsFactors =} \ConstantTok{FALSE}
\NormalTok{    )}
\NormalTok{  \} }\ControlFlowTok{else}\NormalTok{ \{}
    \CommentTok{\# extrair de strings formatadas}
\NormalTok{    col\_ing }\OtherTok{\textless{}{-}} \FunctionTok{grep}\NormalTok{(}\StringTok{"ing|ingere|ingeram|sim"}\NormalTok{, }\FunctionTok{names}\NormalTok{(tab), }\AttributeTok{ignore.case =} \ConstantTok{TRUE}\NormalTok{, }\AttributeTok{value =} \ConstantTok{TRUE}\NormalTok{)[}\DecValTok{1}\NormalTok{]}
\NormalTok{    col\_not }\OtherTok{\textless{}{-}} \FunctionTok{grep}\NormalTok{(}\StringTok{"nao|not|controle"}\NormalTok{, }\FunctionTok{names}\NormalTok{(tab), }\AttributeTok{ignore.case =} \ConstantTok{TRUE}\NormalTok{, }\AttributeTok{value =} \ConstantTok{TRUE}\NormalTok{)[}\DecValTok{1}\NormalTok{]}
    \ControlFlowTok{if}\NormalTok{ (}\SpecialCharTok{!}\FunctionTok{is.na}\NormalTok{(col\_ing) }\SpecialCharTok{\&\&} \SpecialCharTok{!}\FunctionTok{is.na}\NormalTok{(col\_not)) \{}
\NormalTok{      df\_counts }\OtherTok{\textless{}{-}} \FunctionTok{data.frame}\NormalTok{(}
        \AttributeTok{conciliador =} \FunctionTok{sapply}\NormalTok{(tab[[conciliador\_col]], extract\_letter),}
        \AttributeTok{Experimental =} \FunctionTok{extract\_leading\_int}\NormalTok{(tab[[col\_ing]]),}
        \AttributeTok{Controle =} \FunctionTok{extract\_leading\_int}\NormalTok{(tab[[col\_not]]),}
        \AttributeTok{stringsAsFactors =} \ConstantTok{FALSE}
\NormalTok{      )}
\NormalTok{    \}}
\NormalTok{  \}}
\NormalTok{\}}

\CommentTok{\# 3) fallback: usar tabela\_final nível individual}
\ControlFlowTok{if}\NormalTok{ (}\FunctionTok{is.null}\NormalTok{(df\_counts) }\SpecialCharTok{\&\&} \FunctionTok{exists}\NormalTok{(}\StringTok{"tabela\_final"}\NormalTok{, }\AttributeTok{envir =}\NormalTok{ .GlobalEnv)) \{}
\NormalTok{  tf }\OtherTok{\textless{}{-}} \FunctionTok{get}\NormalTok{(}\StringTok{"tabela\_final"}\NormalTok{, }\AttributeTok{envir =}\NormalTok{ .GlobalEnv)}
  \FunctionTok{names}\NormalTok{(tf) }\OtherTok{\textless{}{-}} \FunctionTok{tolower}\NormalTok{(}\FunctionTok{names}\NormalTok{(tf))}
\NormalTok{  col\_conc }\OtherTok{\textless{}{-}} \FunctionTok{names}\NormalTok{(tf)[}\FunctionTok{grepl}\NormalTok{(}\StringTok{"conciliador|conc|mediador|operador"}\NormalTok{, }\FunctionTok{names}\NormalTok{(tf), }\AttributeTok{ignore.case =} \ConstantTok{TRUE}\NormalTok{)][}\DecValTok{1}\NormalTok{]}
\NormalTok{  col\_grp  }\OtherTok{\textless{}{-}} \FunctionTok{names}\NormalTok{(tf)[}\FunctionTok{grepl}\NormalTok{(}\StringTok{"grupo|group|tratamento|condition"}\NormalTok{, }\FunctionTok{names}\NormalTok{(tf), }\AttributeTok{ignore.case =} \ConstantTok{TRUE}\NormalTok{)][}\DecValTok{1}\NormalTok{]}
\NormalTok{  col\_acr  }\OtherTok{\textless{}{-}} \FunctionTok{names}\NormalTok{(tf)[}\FunctionTok{grepl}\NormalTok{(}\StringTok{"acordo|agreement|resposta|sim|yes"}\NormalTok{, }\FunctionTok{names}\NormalTok{(tf), }\AttributeTok{ignore.case =} \ConstantTok{TRUE}\NormalTok{)][}\DecValTok{1}\NormalTok{]}
  \ControlFlowTok{if}\NormalTok{ (}\FunctionTok{is.na}\NormalTok{(col\_conc) }\SpecialCharTok{||} \FunctionTok{is.na}\NormalTok{(col\_grp) }\SpecialCharTok{||} \FunctionTok{is.na}\NormalTok{(col\_acr)) \{}
    \FunctionTok{stop}\NormalTok{(}\StringTok{"Não identifiquei colunas conciliador/grupo/acordo em tabela\_final; forneça obs\_mat\_conciliadores ou tabela\_conciliadores\_summary."}\NormalTok{)}
\NormalTok{  \}}
\NormalTok{  tf2 }\OtherTok{\textless{}{-}}\NormalTok{ tf }\SpecialCharTok{\%\textgreater{}\%}
    \FunctionTok{mutate}\NormalTok{(}
      \AttributeTok{conciliador =} \FunctionTok{toupper}\NormalTok{(}\FunctionTok{substr}\NormalTok{(}\FunctionTok{trimws}\NormalTok{(}\FunctionTok{as.character}\NormalTok{(.data[[col\_conc]])), }\DecValTok{1}\NormalTok{, }\DecValTok{1}\NormalTok{)),}
      \AttributeTok{grupo =} \FunctionTok{ifelse}\NormalTok{(}\FunctionTok{tolower}\NormalTok{(}\FunctionTok{trimws}\NormalTok{(}\FunctionTok{as.character}\NormalTok{(.data[[col\_grp]]))) }\SpecialCharTok{\%in\%} \FunctionTok{c}\NormalTok{(}\StringTok{"controle"}\NormalTok{,}\StringTok{"control"}\NormalTok{), }\StringTok{"Controle"}\NormalTok{, }\StringTok{"Experimental"}\NormalTok{),}
      \AttributeTok{acordo =} \FunctionTok{tolower}\NormalTok{(}\FunctionTok{trimws}\NormalTok{(}\FunctionTok{as.character}\NormalTok{(.data[[col\_acr]]))),}
      \AttributeTok{agree =} \FunctionTok{ifelse}\NormalTok{(acordo }\SpecialCharTok{\%in\%} \FunctionTok{c}\NormalTok{(}\StringTok{"s"}\NormalTok{,}\StringTok{"sim"}\NormalTok{,}\StringTok{"yes"}\NormalTok{,}\StringTok{"y"}\NormalTok{,}\StringTok{"1"}\NormalTok{,}\StringTok{"true"}\NormalTok{), }\DecValTok{1}\NormalTok{,}
                     \FunctionTok{ifelse}\NormalTok{(acordo }\SpecialCharTok{\%in\%} \FunctionTok{c}\NormalTok{(}\StringTok{"n"}\NormalTok{,}\StringTok{"nao"}\NormalTok{,}\StringTok{"não"}\NormalTok{,}\StringTok{"no"}\NormalTok{,}\StringTok{"0"}\NormalTok{,}\StringTok{"false"}\NormalTok{), }\DecValTok{0}\NormalTok{, }\ConstantTok{NA}\NormalTok{))}
\NormalTok{    ) }\SpecialCharTok{\%\textgreater{}\%}
    \FunctionTok{filter}\NormalTok{(conciliador }\SpecialCharTok{\%in\%}\NormalTok{ desired)}
\NormalTok{  tab\_counts }\OtherTok{\textless{}{-}}\NormalTok{ tf2 }\SpecialCharTok{\%\textgreater{}\%}
    \FunctionTok{group\_by}\NormalTok{(conciliador, grupo) }\SpecialCharTok{\%\textgreater{}\%}
    \FunctionTok{summarise}\NormalTok{(}\AttributeTok{agreements =} \FunctionTok{sum}\NormalTok{(agree }\SpecialCharTok{==} \DecValTok{1}\NormalTok{, }\AttributeTok{na.rm =} \ConstantTok{TRUE}\NormalTok{), }\AttributeTok{.groups =} \StringTok{"drop"}\NormalTok{)}
\NormalTok{  df\_counts }\OtherTok{\textless{}{-}}\NormalTok{ tab\_counts }\SpecialCharTok{\%\textgreater{}\%}
    \FunctionTok{pivot\_wider}\NormalTok{(}\AttributeTok{names\_from =}\NormalTok{ grupo, }\AttributeTok{values\_from =}\NormalTok{ agreements, }\AttributeTok{values\_fill =} \DecValTok{0}\NormalTok{) }\SpecialCharTok{\%\textgreater{}\%}
    \FunctionTok{mutate}\NormalTok{(}\AttributeTok{Controle =} \FunctionTok{ifelse}\NormalTok{(}\FunctionTok{is.na}\NormalTok{(Controle), }\DecValTok{0}\NormalTok{, Controle),}
           \AttributeTok{Experimental =} \FunctionTok{ifelse}\NormalTok{(}\FunctionTok{is.na}\NormalTok{(Experimental), }\DecValTok{0}\NormalTok{, Experimental),}
           \AttributeTok{conciliador =} \FunctionTok{as.character}\NormalTok{(conciliador))}
\NormalTok{\}}

\ControlFlowTok{if}\NormalTok{ (}\FunctionTok{is.null}\NormalTok{(df\_counts)) }\FunctionTok{stop}\NormalTok{(}\StringTok{"Falha ao construir df\_counts. Forneça dados em formato esperado."}\NormalTok{)}

\CommentTok{\# normalizar conciliador e garantir A{-}D presentes}
\NormalTok{df\_counts }\OtherTok{\textless{}{-}}\NormalTok{ df\_counts }\SpecialCharTok{\%\textgreater{}\%}
  \FunctionTok{mutate}\NormalTok{(}\AttributeTok{conciliador =} \FunctionTok{sapply}\NormalTok{(conciliador, }\ControlFlowTok{function}\NormalTok{(x) \{}
    \ControlFlowTok{if}\NormalTok{ (}\FunctionTok{is.na}\NormalTok{(x)) }\FunctionTok{return}\NormalTok{(}\ConstantTok{NA\_character\_}\NormalTok{)}
    \FunctionTok{extract\_letter}\NormalTok{(x)}
\NormalTok{  \})) }\SpecialCharTok{\%\textgreater{}\%}
  \FunctionTok{filter}\NormalTok{(}\SpecialCharTok{!}\FunctionTok{is.na}\NormalTok{(conciliador)) }\SpecialCharTok{\%\textgreater{}\%}
  \FunctionTok{group\_by}\NormalTok{(conciliador) }\SpecialCharTok{\%\textgreater{}\%}
  \FunctionTok{summarise}\NormalTok{(}\AttributeTok{Controle =} \FunctionTok{sum}\NormalTok{(}\FunctionTok{as.integer}\NormalTok{(Controle), }\AttributeTok{na.rm =} \ConstantTok{TRUE}\NormalTok{),}
            \AttributeTok{Experimental =} \FunctionTok{sum}\NormalTok{(}\FunctionTok{as.integer}\NormalTok{(Experimental), }\AttributeTok{na.rm =} \ConstantTok{TRUE}\NormalTok{),}
            \AttributeTok{.groups =} \StringTok{"drop"}\NormalTok{) }\SpecialCharTok{\%\textgreater{}\%}
  \FunctionTok{complete}\NormalTok{(}\AttributeTok{conciliador =}\NormalTok{ desired, }\AttributeTok{fill =} \FunctionTok{list}\NormalTok{(}\AttributeTok{Controle =} \DecValTok{0}\NormalTok{, }\AttributeTok{Experimental =} \DecValTok{0}\NormalTok{)) }\SpecialCharTok{\%\textgreater{}\%}
  \FunctionTok{arrange}\NormalTok{(}\FunctionTok{match}\NormalTok{(conciliador, desired))}

\FunctionTok{message}\NormalTok{(}\StringTok{"Contagens por conciliador (Controle / Experimental):"}\NormalTok{)}
\FunctionTok{print}\NormalTok{(df\_counts)}

\CommentTok{\# preparar dados long e calcular proporção dentro de cada conciliador}
\NormalTok{plot\_long }\OtherTok{\textless{}{-}}\NormalTok{ df\_counts }\SpecialCharTok{\%\textgreater{}\%}
  \FunctionTok{pivot\_longer}\NormalTok{(}\AttributeTok{cols =} \FunctionTok{c}\NormalTok{(}\StringTok{"Controle"}\NormalTok{,}\StringTok{"Experimental"}\NormalTok{), }\AttributeTok{names\_to =} \StringTok{"grupo"}\NormalTok{, }\AttributeTok{values\_to =} \StringTok{"agreements"}\NormalTok{) }\SpecialCharTok{\%\textgreater{}\%}
  \FunctionTok{group\_by}\NormalTok{(conciliador) }\SpecialCharTok{\%\textgreater{}\%}
  \FunctionTok{mutate}\NormalTok{(}\AttributeTok{total\_by\_conc =} \FunctionTok{sum}\NormalTok{(agreements, }\AttributeTok{na.rm =} \ConstantTok{TRUE}\NormalTok{),}
         \AttributeTok{prop\_within\_conc =} \FunctionTok{ifelse}\NormalTok{(total\_by\_conc }\SpecialCharTok{\textgreater{}} \DecValTok{0}\NormalTok{, agreements }\SpecialCharTok{/}\NormalTok{ total\_by\_conc, }\DecValTok{0}\NormalTok{)) }\SpecialCharTok{\%\textgreater{}\%}
  \FunctionTok{ungroup}\NormalTok{()}

\FunctionTok{message}\NormalTok{(}\StringTok{"Dados long para plotagem (verifique se A{-}D aparecem):"}\NormalTok{)}
\FunctionTok{print}\NormalTok{(plot\_long)}

\CommentTok{\# plot}
\NormalTok{p\_conciliadores\_proporcao }\OtherTok{\textless{}{-}} \FunctionTok{ggplot}\NormalTok{(plot\_long, }\FunctionTok{aes}\NormalTok{(}\AttributeTok{x =}\NormalTok{ conciliador, }\AttributeTok{y =}\NormalTok{ prop\_within\_conc, }\AttributeTok{fill =}\NormalTok{ grupo)) }\SpecialCharTok{+}
  \FunctionTok{geom\_col}\NormalTok{(}\AttributeTok{position =} \FunctionTok{position\_dodge}\NormalTok{(}\AttributeTok{width =} \FloatTok{0.8}\NormalTok{), }\AttributeTok{width =} \FloatTok{0.7}\NormalTok{, }\AttributeTok{colour =} \StringTok{"grey30"}\NormalTok{) }\SpecialCharTok{+}
  \FunctionTok{geom\_text}\NormalTok{(}\FunctionTok{aes}\NormalTok{(}\AttributeTok{label =} \FunctionTok{ifelse}\NormalTok{(}\SpecialCharTok{!}\FunctionTok{is.na}\NormalTok{(prop\_within\_conc), scales}\SpecialCharTok{::}\FunctionTok{percent}\NormalTok{(prop\_within\_conc, }\AttributeTok{accuracy =} \FloatTok{0.1}\NormalTok{), }\StringTok{"0.0\%"}\NormalTok{)),}
            \AttributeTok{position =} \FunctionTok{position\_dodge}\NormalTok{(}\AttributeTok{width =} \FloatTok{0.8}\NormalTok{), }\AttributeTok{vjust =} \SpecialCharTok{{-}}\FloatTok{0.3}\NormalTok{, }\AttributeTok{size =} \DecValTok{3}\NormalTok{) }\SpecialCharTok{+}
  \FunctionTok{scale\_y\_continuous}\NormalTok{(}\AttributeTok{labels =}\NormalTok{ scales}\SpecialCharTok{::}\FunctionTok{percent\_format}\NormalTok{(}\AttributeTok{accuracy =} \DecValTok{1}\NormalTok{), }\AttributeTok{breaks =} \FunctionTok{seq}\NormalTok{(}\DecValTok{0}\NormalTok{,}\DecValTok{1}\NormalTok{,}\AttributeTok{by=}\FloatTok{0.1}\NormalTok{), }\AttributeTok{limits =} \FunctionTok{c}\NormalTok{(}\DecValTok{0}\NormalTok{,}\DecValTok{1}\NormalTok{)) }\SpecialCharTok{+}
  \FunctionTok{scale\_fill\_manual}\NormalTok{(}\AttributeTok{values =} \FunctionTok{c}\NormalTok{(}\StringTok{"Controle"} \OtherTok{=} \StringTok{"\#4E79A7"}\NormalTok{, }\StringTok{"Experimental"} \OtherTok{=} \StringTok{"\#F28E2B"}\NormalTok{)) }\SpecialCharTok{+}
  \FunctionTok{labs}\NormalTok{(}\AttributeTok{title =} \StringTok{"Proporção de acordos por conciliador (por Grupos)"}\NormalTok{,}
       \AttributeTok{subtitle =} \StringTok{"Comparação: Controle vs Experimental apenas quando houve acordo"}\NormalTok{,}
       \AttributeTok{x =} \StringTok{"Conciliador"}\NormalTok{, }\AttributeTok{y =} \StringTok{"Proporção de acordos (por conciliador)"}\NormalTok{,}
       \AttributeTok{fill =} \StringTok{"Grupo"}\NormalTok{) }\SpecialCharTok{+}
  \FunctionTok{theme\_minimal}\NormalTok{() }\SpecialCharTok{+}
  \FunctionTok{theme}\NormalTok{(}\AttributeTok{plot.title =} \FunctionTok{element\_text}\NormalTok{(}\AttributeTok{hjust =} \FloatTok{0.5}\NormalTok{))}

\FunctionTok{assign}\NormalTok{(}\StringTok{"df\_counts\_conciliadores"}\NormalTok{, df\_counts, }\AttributeTok{envir =}\NormalTok{ .GlobalEnv)}
\FunctionTok{assign}\NormalTok{(}\StringTok{"plot\_long\_conciliadores"}\NormalTok{, plot\_long, }\AttributeTok{envir =}\NormalTok{ .GlobalEnv)}
\FunctionTok{assign}\NormalTok{(}\StringTok{"p\_conciliadores\_proporcao"}\NormalTok{, p\_conciliadores\_proporcao, }\AttributeTok{envir =}\NormalTok{ .GlobalEnv)}

\FunctionTok{print}\NormalTok{(p\_conciliadores\_proporcao)}
\InformationTok{\textasciigrave{}\textasciigrave{}\textasciigrave{}}
\end{Highlighting}
\end{Shaded}

\begin{verbatim}
# A tibble: 4 x 3
  conciliador Controle Experimental
  <chr>          <int>        <int>
1 A                 35           25
2 B                 28           37
3 C                 21           65
4 D                 21           89
# A tibble: 8 x 5
  conciliador grupo        agreements total_by_conc prop_within_conc
  <chr>       <chr>             <int>         <int>            <dbl>
1 A           Controle             35            60            0.583
2 A           Experimental         25            60            0.417
3 B           Controle             28            65            0.431
4 B           Experimental         37            65            0.569
5 C           Controle             21            86            0.244
6 C           Experimental         65            86            0.756
7 D           Controle             21           110            0.191
8 D           Experimental         89           110            0.809
\end{verbatim}

\pandocbounded{\includegraphics[keepaspectratio]{replicar-suco-uva-tab-graf_files/figure-pdf/unnamed-chunk-9-1.pdf}}

Há um claro padrão discrepante no desempenho dos conciliadores C e D em
relação aos conciliadores A e B.

Considerando que todos os conciliadores foram orientados para aplicarem
sempre a mesma e única técnica de conciliação, essa discrepância
acentuada é uma evidência de que C e D podem ter atuado com mais
eficácia no Grupo Experimental que no Grupo de Controle.

mmm

\bookmarksetup{startatroot}

\chapter{AEI - cap 16 moore - IC: o Básico}\label{sec-IC-basico}

\section{Objetivos da Aprendizagem}\label{objetivos-da-aprendizagem-5}

\begin{quote}
Após ler este capítulo, você deve ser capaz de:

▶ 16.1 Usar os \ul{\textbf{princípios}} da \ul{\textbf{inferência}} e
\ul{\textbf{estimação}} estatísticas para a \textbf{\emph{interpretação
de intervalos de confiança}}.

▶ 16.2 Articular o \ul{\textbf{significado}} de \emph{afirmativas que
envolvem} \ul{\textbf{níveis de confiança}} e \ul{\textbf{margens de
erro}}.

▶ 16.3 Calcular \ul{\textbf{intervalos de confiança para médias}},
depois de confirmar que as \ul{\textbf{condições necessárias são
satisfeitas}}.

▶ 16.4 Compreender como a \ul{\textbf{margem de erro}} muda com o
\ul{\textbf{tamanho amostral}} e \ul{\textbf{nível de confiança}}.
\end{quote}

\section{Intervalos de Confiança: o
Básico}\label{intervalos-de-confianuxe7a-o-buxe1sico}

\begin{quote}
Os Capítulos 8 e 9 dizem que a maneira pela qual produzimos dados
(amostragem, planejamentos experimentais) afeta a condição de termos, ou
não, uma boa base para a generalização para alguma população mais ampla.

Os Capítulos 12, 13 e 14 discutem probabilidade, a ferramenta matemática
que determina a natureza das inferências que fazemos.

O Capítulo 15 discute distribuições amostrais, que nos dizem como
repetidas AASs se comportam e o que uma estatística (em particular, uma
média amostral), calculada a partir de nossa amostra, pode nos dizer
sobre o parâmetro correspondente da população da qual a amostra foi
selecionada.

Neste capítulo {[}16{]}, discutimos o \textbf{raciocínio básico} da
\ul{\textbf{estimação estatística}}, com ênfase na \emph{estimação
{[}intervalar{]} da média de uma população}.

Após extrairmos uma amostra, sabemos as respostas dos indivíduos na
amostra. O motivo usual da extração de uma amostra não é conhecer os
indivíduos que a compõem, mas \ul{\textbf{inferir}}, \textbf{\emph{a
partir dos dados amostrais}}, alguma \ul{\textbf{conclusão}} sobre a
\ul{\textbf{população}} mais ampla que a \textbf{\emph{amostra
representa}}. (MOORE; NOTZ; FLIGNER, 2023 , cap. 16, p.~296)
\end{quote}

\begin{tcolorbox}[enhanced jigsaw, arc=.35mm, opacitybacktitle=0.6, colframe=quarto-callout-important-color-frame, titlerule=0mm, leftrule=.75mm, left=2mm, colbacktitle=quarto-callout-important-color!10!white, breakable, toprule=.15mm, bottomtitle=1mm, opacityback=0, coltitle=black, title=\textcolor{quarto-callout-important-color}{\faExclamation}\hspace{0.5em}{Inferência estatística}, rightrule=.15mm, bottomrule=.15mm, toptitle=1mm, colback=white]

A inferência estatística fornece \ul{\textbf{métodos}} para a
\textbf{\emph{extração de conclusões sobre uma população a partir de
dados amostrais}}.

\end{tcolorbox}

Como diferentes amostras podem conduzir a conclusões diferentes, não
podemos ter certeza de que nossas conclusões sejam corretas. A
inferência estatística usa a linguagem da probabilidade para expressar o
grau de confiabilidade de nossas conclusões. Este capítulo introduz um
dos dois tipos mais comuns de inferência, intervalos de confiança para
estimar o valor de um parâmetro populacional. O próximo capítulo discute
o outro tipo comum de inferência, testes de significância para avaliar a
evidência de uma afirmativa sobre uma população. Ambos os tipos de
inferência se baseiam nas distribuições amostrais de estatísticas. Ou
seja, ambos utilizam a probabilidade para dizer o que aconteceria se
usássemos o método de inferência muitas vezes.

Este capítulo apresenta a lógica básica da inferência estatística. Para
torná-la mais clara possível, começamos com um contexto que é muito
simples para ser realista. Eis o contexto para nosso trabalho neste
capítulo.

\begin{tcolorbox}[enhanced jigsaw, arc=.35mm, opacitybacktitle=0.6, colframe=quarto-callout-important-color-frame, titlerule=0mm, leftrule=.75mm, left=2mm, colbacktitle=quarto-callout-important-color!10!white, breakable, toprule=.15mm, bottomtitle=1mm, opacityback=0, coltitle=black, title=\textcolor{quarto-callout-important-color}{\faExclamation}\hspace{0.5em}{Condições simples para inferência sobre uma média}, rightrule=.15mm, bottomrule=.15mm, toptitle=1mm, colback=white]

\begin{enumerate}
\def\labelenumi{\arabic{enumi}.}
\tightlist
\item
  Temos \ul{\textbf{uma amostra aleatória simples}} \textbf{(AAS)} da
  população de interesse. Não há não resposta ou qualquer outra
  dificuldade prática. A \ul{\textbf{população é grande}} em comparação
  ao tamanho da amostra {[}N \textgreater{} 20 x n{]}.
\item
  A \textbf{variável} que \emph{medimos} tem \ul{\textbf{uma
  distribuição exatamente normal}} \textbf{N(µ; σ)} na população.
\item
  \textbf{\emph{Não conhecemos}} a \textbf{média da população} µ. Mas
  \ul{\textbf{conhecemos}} o \ul{\textbf{desvio-padrão populacional}} σ.
\end{enumerate}

\end{tcolorbox}

A condição de que a população seja grande em relação ao tamanho da
amostra será adequadamente satisfeita se a população for, digamos, pelo
menos 20 vezes maior.

\begin{tcolorbox}[enhanced jigsaw, arc=.35mm, opacitybacktitle=0.6, colframe=quarto-callout-important-color-frame, titlerule=0mm, leftrule=.75mm, left=2mm, colbacktitle=quarto-callout-important-color!10!white, breakable, toprule=.15mm, bottomtitle=1mm, opacityback=0, coltitle=black, title=\textcolor{quarto-callout-important-color}{\faExclamation}\hspace{0.5em}{Important}, rightrule=.15mm, bottomrule=.15mm, toptitle=1mm, colback=white]

\emph{As condições de que temos uma AAS perfeita, de que a população é
exatamente Normal e de que conhecemos o σ populacional são todas não
realistas}.

O Capítulo 18 inicia o movimento que parte das ``condições simples'' em
direção à realidade da prática estatística. Capítulos posteriores tratam
da inferência em contextos completamente realistas.

Se essas ``condições simples'' não são realistas, por que então
estudá-las? Uma razão é que, sob essas condições simples, podemos
aplicar o que aprendemos nos capítulos anteriores sobre distribuição
Normal e sobre distribuição amostral de uma média amostral, para
desenvolver, passo a passo, métodos para inferência sobre uma média. O
raciocínio usado sob condições simples se aplica a contextos mais
realistas, com matemática mais complicada.

Embora nunca saibamos se uma população é exatamente Normal, e nunca
conheçamos o σ populacional, os \ul{\textbf{métodos}} que discutiremos
neste e nos dois próximos capítulos \ul{\textbf{são aproximadamente
corretos para tamanhos amostrais suficientemente grandes}} {[}TCL{]},
desde que \ul{\textbf{\emph{tratemos o desvio-padrão amostral como se
fosse o σ populacional}}}. Assim, \textbf{\emph{há situações
(admitidamente raras) em que esses métodos podem ser usados na
prática}}.

\end{tcolorbox}

\section{A lógica da estimação
estatística}\label{a-luxf3gica-da-estimauxe7uxe3o-estatuxedstica}

O índice de massa corporal (IMC) é usado para a análise de possíveis
problemas de peso. Seu cálculo é feito dividindo o peso pelo quadrado da
altura, sendo o peso medido em quilogramas, e a altura, em metros.
Muitos programas online para cálculo do IMC permitem que você introduza
o peso em libras e a altura em polegadas. Adultos com IMC menor do que
18,5 kg/m2 são considerados em subpeso, e aqueles com IMC acima de 25
kg/m2 estão em sobrepeso. Para dados sobre IMC, recorremos ao National
Health and Nutrition Examination Survey (\texttt{NHANES}), uma pesquisa
amostral contínua do governo que monitora a saúde da população
norte-americana.

\subsection{EXEMPLO 16.1 Índice de massa corporal de homens
jovens}\label{exemplo-16.1-uxedndice-de-massa-corporal-de-homens-jovens}

Um relatório da NHANES fornece dados para \ul{\textbf{936 homens}} com
\ul{\textbf{idade entre 20 e 29 anos}}.

O IMC médio desses 936 homens foi \(\bar{x} = 27,2\).

Com \textbf{\emph{base nessa amostra}}, desejamos \textbf{\emph{estimar
o IMC médio}} µ na \ul{\textbf{população}} de \emph{todos} os
\ul{\textbf{\emph{23,2 milhões de homens nessa faixa etária}}}.

Para nos adequarmos às ``condições simples'', trataremos a amostra da
NHANES como uma AAS de uma população Normal, e vamos supor que
conheçamos o desvio-padrão σ = 11,6. (O desvio-padrão amostral para
esses 936 homens é 11,63 kg/m2. Para propósitos do exemplo, vamos
arredondá-lo para 11,6 e \ul{\emph{prosseguir como se isso fosse o
desvio-padrão da população}} σ.)

Eis o raciocínio da estimação estatística em poucas palavras:

1.Para estimar o \textbf{\emph{desconhecido}} IMC médio µ da população,
usamos a média \(\bar{x} = 27,2\) da amostra aleatória. Não esperamos
que x seja exatamente igual a µ, de modo que desejamos
\textbf{\emph{dizer quão precisa é essa estimativa {[}pontual{]}}}.

2.Conhecemos a distribuição amostral de \(\bar{x}\). Em
\textbf{\emph{amostras repetidas}}, \(\bar{x}\) tem
\ul{\textbf{distribuição Normal}} com média µ e desvio-padrão
\(\sigma/\sqrt{n}\). Então, o IMC médio \(\bar{x}\) de uma AAS de 936
homens jovens tem desvio-padrão

\[
\text{Erro Padrão das médias amostrais} = \frac{\sigma}{\sqrt{n}} = \frac{11.6}{\sqrt{936}} = 0.4
\]

3.A parte 95 da \ul{\textbf{regra 68-95-99,7}} para distribuições
Normais afirma que x está a \ul{\textbf{até dois desvios-padrão da média
µ em 95\% de todas as amostras}}. O desvio-padrão é 0,4, de modo que
\ul{\textbf{dois desvios-padrão valem 0,8}}. Isto é, \ul{\textbf{para
95\% de todas as amostras de tamanho 936}}, a \emph{distância entre a
média amostral} \(\bar{x}\) e a \emph{média populacional} µ é
\emph{menor do que 0,8}. Logo, \ul{\textbf{se estimarmos}} que µ
\ul{\textbf{\emph{esteja em algum lugar no intervalo}}} de \(\bar{x}\)
-- 0,8 a \(\bar{x}\) + 0,8, \ul{\textbf{\emph{estaremos corretos em 95\%
de todas as possíveis amostras}}}. Para essa amostra particular, esse
intervalo é

\(\bar{x}\) -- 0,8 = 27,2 -- 0,8 = 26,4

a

\(\bar{x}\) + 0,8 = 27,2 + 0,8 = 28,0

4.Como obtivemos o intervalo 26,4 a 28,0 a partir de \ul{\textbf{um
método que captura a média populacional}} \ul{\textbf{em 95\% de todas
as amostras possíveis}}, dizemos que \ul{\textbf{estamos 95\% confiantes
em que o IMC médio µ para todos os homens jovens seja algum valor
naquele intervalo}} -- não menor do que 26,4 e não maior do que 28,0.

Carregar o arquivo \texttt{NHANES} do R.

Para simular, na maior medida possível, esse exemplo 16.1 (MOORE; NOTZ;
FLIGNER, 2023 , p.~297).

Código a seguir adaptado a partir de (Poldrack, 2025 , cap. 5, p.~42).

\subsubsection{Carregar pacotes
necessários}\label{carregar-pacotes-necessuxe1rios}

\begin{Shaded}
\begin{Highlighting}[numbers=left,,]
\InformationTok{\textasciigrave{}\textasciigrave{}\textasciigrave{}\{r\}}
\FunctionTok{library}\NormalTok{(tidyverse)}
\FunctionTok{library}\NormalTok{(NHANES)}
\FunctionTok{library}\NormalTok{(cowplot)}
\FunctionTok{library}\NormalTok{(mapproj)}
\FunctionTok{library}\NormalTok{(pander)}
\FunctionTok{library}\NormalTok{(knitr)}
\FunctionTok{library}\NormalTok{(modelr)}

\FunctionTok{panderOptions}\NormalTok{(}\StringTok{\textquotesingle{}round\textquotesingle{}}\NormalTok{,}\DecValTok{2}\NormalTok{)}
\FunctionTok{panderOptions}\NormalTok{(}\StringTok{\textquotesingle{}digits\textquotesingle{}}\NormalTok{,}\DecValTok{7}\NormalTok{)}
\FunctionTok{theme\_set}\NormalTok{(}\FunctionTok{theme\_minimal}\NormalTok{(}\AttributeTok{base\_size =} \DecValTok{14}\NormalTok{))}

\FunctionTok{options}\NormalTok{(}\AttributeTok{digits =} \DecValTok{2}\NormalTok{)}
\FunctionTok{set.seed}\NormalTok{(}\DecValTok{123456}\NormalTok{) }\CommentTok{\# set random seed to exactly replicate results}
\InformationTok{\textasciigrave{}\textasciigrave{}\textasciigrave{}}
\end{Highlighting}
\end{Shaded}

\subsubsection{Carregar data set:
NHANES}\label{carregar-data-set-nhanes}

E filtrar um subconjunto do data sete para \texttt{sexo} homens e com
\texttt{idade} entre 20 e 29 anos.

\begin{Shaded}
\begin{Highlighting}[numbers=left,,]
\InformationTok{\textasciigrave{}\textasciigrave{}\textasciigrave{}\{r\}}
\NormalTok{code }\OtherTok{\textless{}{-}} \DecValTok{0} \CommentTok{\# somente irá resetar a Job Area se code == 1}
\ControlFlowTok{if}\NormalTok{(code}\SpecialCharTok{==}\DecValTok{1}\NormalTok{) }\FunctionTok{rm}\NormalTok{(}\AttributeTok{list=}\FunctionTok{ls}\NormalTok{()) }\CommentTok{\# Remove toda a list de variáveis da Job Area, i. e., dá um reset na Environment}

\CommentTok{\# drop duplicated IDs within the NHANES dataset}
\NormalTok{NHANES }\OtherTok{\textless{}{-}}
\NormalTok{  NHANES }\SpecialCharTok{\%\textgreater{}\%}
\NormalTok{  dplyr}\SpecialCharTok{::}\FunctionTok{distinct}\NormalTok{(ID, }\AttributeTok{.keep\_all =} \ConstantTok{TRUE}\NormalTok{)}

\CommentTok{\# select the appropriate men with good Gender and Age measurements}
\NormalTok{NHANES\_men }\OtherTok{\textless{}{-}}
\NormalTok{  NHANES }\SpecialCharTok{\%\textgreater{}\%}
  \FunctionTok{drop\_na}\NormalTok{(Gender) }\SpecialCharTok{\%\textgreater{}\%}
  \FunctionTok{drop\_na}\NormalTok{(Age) }\SpecialCharTok{\%\textgreater{}\%}
  \FunctionTok{subset}\NormalTok{(Age }\SpecialCharTok{\textgreater{}=} \DecValTok{20} \SpecialCharTok{\&}\NormalTok{ Age }\SpecialCharTok{\textless{}=} \DecValTok{29}\NormalTok{)}

\NormalTok{NHANES\_men }\SpecialCharTok{\%\textgreater{}\%} \FunctionTok{nrow}\NormalTok{() }\CommentTok{\# 880 homens entre 20 e 29 anos}

\NormalTok{NHANES\_men }\SpecialCharTok{\%\textgreater{}\%}
  \FunctionTok{ggplot}\NormalTok{(}\FunctionTok{aes}\NormalTok{(BMI)) }\SpecialCharTok{+}
  \FunctionTok{geom\_histogram}\NormalTok{(}\AttributeTok{bins =} \DecValTok{100}\NormalTok{) }\SpecialCharTok{+} 
  \FunctionTok{labs}\NormalTok{(}
  \AttributeTok{title =} \StringTok{"Histograma do IMC [BMI]: 880 homens no NHANES"}\NormalTok{,}
  \AttributeTok{subtitle =} \StringTok{"filtro: idade (age) \textgreater{}= 20 e \textless{}= 29 anos"}\NormalTok{,}
  \AttributeTok{caption =} \StringTok{"Fonte: Poldrack, 2025, p. 42; Moore, 2023, p. 297"}\NormalTok{,}
  \AttributeTok{x =} \StringTok{"IMC (kg/m2)"}\NormalTok{,}
  \AttributeTok{y =} \StringTok{"Contagem (Count)"}
\NormalTok{)}
\InformationTok{\textasciigrave{}\textasciigrave{}\textasciigrave{}}
\end{Highlighting}
\end{Shaded}

\begin{verbatim}
[1] 880
\end{verbatim}

\pandocbounded{\includegraphics[keepaspectratio]{cap16-moore-IC-o-basico_files/figure-pdf/unnamed-chunk-2-1.pdf}}

Salvar arquivo .csv

\begin{Shaded}
\begin{Highlighting}[numbers=left,,]
\InformationTok{\textasciigrave{}\textasciigrave{}\textasciigrave{}\{r\}}
\CommentTok{\# Função para salvar um data.frame em CSV com opções comuns e mensagens informativas.}
\CommentTok{\# Uso:}
\CommentTok{\#   save\_df\_to\_csv(df, "output/meu\_arquivo.csv")}
\CommentTok{\# Parâmetros:}
\CommentTok{\#   df               {-} objeto data.frame a ser salvo (obrigatório)}
\CommentTok{\#   file\_path        {-} caminho do arquivo de saída (obrigatório)}
\CommentTok{\#   sep              {-} separador de campos (padrão: ",")}
\CommentTok{\#   na               {-} representação de valores NA no arquivo (padrão: "")}
\CommentTok{\#   row.names        {-} incluir nomes de linha (padrão: FALSE)}
\CommentTok{\#   col.names        {-} incluir cabeçalho (padrão: TRUE)}
\CommentTok{\#   quote            {-} colocar aspas em campos (padrão: TRUE)}
\CommentTok{\#   fileEncoding     {-} codificação do arquivo (padrão: "UTF{-}8")}
\CommentTok{\#   append\_timestamp {-} acrescentar timestamp ao nome do arquivo (padrão: FALSE)}
\NormalTok{save\_df\_to\_csv }\OtherTok{\textless{}{-}} \ControlFlowTok{function}\NormalTok{(df,}
\NormalTok{                           file\_path,}
                           \AttributeTok{sep =} \StringTok{","}\NormalTok{,}
                           \AttributeTok{na =} \StringTok{""}\NormalTok{,}
                           \AttributeTok{row.names =} \ConstantTok{FALSE}\NormalTok{,}
                           \AttributeTok{col.names =} \ConstantTok{TRUE}\NormalTok{,}
                           \AttributeTok{quote =} \ConstantTok{TRUE}\NormalTok{,}
                           \AttributeTok{fileEncoding =} \StringTok{"UTF{-}8"}\NormalTok{,}
                           \AttributeTok{append\_timestamp =} \ConstantTok{FALSE}\NormalTok{) \{}
  \CommentTok{\# Validações básicas}
  \ControlFlowTok{if}\NormalTok{ (}\FunctionTok{missing}\NormalTok{(df) }\SpecialCharTok{||} \SpecialCharTok{!}\FunctionTok{is.data.frame}\NormalTok{(df)) \{}
    \FunctionTok{stop}\NormalTok{(}\StringTok{"Parâmetro \textquotesingle{}df\textquotesingle{} obrigatório e deve ser um data.frame."}\NormalTok{)}
\NormalTok{  \}}
  \ControlFlowTok{if}\NormalTok{ (}\FunctionTok{missing}\NormalTok{(file\_path) }\SpecialCharTok{||} \SpecialCharTok{!}\FunctionTok{is.character}\NormalTok{(file\_path) }\SpecialCharTok{||} \FunctionTok{length}\NormalTok{(file\_path) }\SpecialCharTok{!=} \DecValTok{1}\NormalTok{) \{}
    \FunctionTok{stop}\NormalTok{(}\StringTok{"Parâmetro \textquotesingle{}file\_path\textquotesingle{} obrigatório e deve ser uma string única com o caminho."}\NormalTok{)}
\NormalTok{  \}}

  \CommentTok{\# Se solicitado, anexa timestamp ao nome do arquivo antes da extensão}
  \ControlFlowTok{if}\NormalTok{ (append\_timestamp) \{}
\NormalTok{    ext\_index }\OtherTok{\textless{}{-}} \FunctionTok{regexpr}\NormalTok{(}\StringTok{"}\SpecialCharTok{\textbackslash{}\textbackslash{}}\StringTok{.[\^{}}\SpecialCharTok{\textbackslash{}\textbackslash{}}\StringTok{.]*$"}\NormalTok{, file\_path)}
\NormalTok{    timestamp }\OtherTok{\textless{}{-}} \FunctionTok{format}\NormalTok{(}\FunctionTok{Sys.time}\NormalTok{(), }\StringTok{"\%Y\%m\%d\_\%H\%M\%S"}\NormalTok{)}
    \ControlFlowTok{if}\NormalTok{ (ext\_index[}\DecValTok{1}\NormalTok{] }\SpecialCharTok{\textgreater{}} \DecValTok{0}\NormalTok{) \{}
\NormalTok{      file\_path }\OtherTok{\textless{}{-}} \FunctionTok{paste0}\NormalTok{(}\FunctionTok{substr}\NormalTok{(file\_path, }\DecValTok{1}\NormalTok{, ext\_index[}\DecValTok{1}\NormalTok{] }\SpecialCharTok{{-}} \DecValTok{1}\NormalTok{), }\StringTok{"\_"}\NormalTok{, timestamp, }\FunctionTok{substr}\NormalTok{(file\_path, ext\_index[}\DecValTok{1}\NormalTok{], }\FunctionTok{nchar}\NormalTok{(file\_path)))}
\NormalTok{    \} }\ControlFlowTok{else}\NormalTok{ \{}
\NormalTok{      file\_path }\OtherTok{\textless{}{-}} \FunctionTok{paste0}\NormalTok{(file\_path, }\StringTok{"\_"}\NormalTok{, timestamp, }\StringTok{".csv"}\NormalTok{)}
\NormalTok{    \}}
\NormalTok{  \}}

  \CommentTok{\# Garante que o diretório de destino exista}
\NormalTok{  dir\_path }\OtherTok{\textless{}{-}} \FunctionTok{dirname}\NormalTok{(file\_path)}
  \ControlFlowTok{if}\NormalTok{ (}\SpecialCharTok{!}\FunctionTok{dir.exists}\NormalTok{(dir\_path)) \{}
    \FunctionTok{dir.create}\NormalTok{(dir\_path, }\AttributeTok{recursive =} \ConstantTok{TRUE}\NormalTok{, }\AttributeTok{showWarnings =} \ConstantTok{FALSE}\NormalTok{)}
\NormalTok{  \}}

  \CommentTok{\# Escrita do arquivo com tratamento de erro}
\NormalTok{  result }\OtherTok{\textless{}{-}} \FunctionTok{tryCatch}\NormalTok{(\{}
    \CommentTok{\# write.table usado para permitir controle fino do separador e encoding}
    \FunctionTok{write.table}\NormalTok{(df,}
                \AttributeTok{file =}\NormalTok{ file\_path,}
                \AttributeTok{sep =}\NormalTok{ sep,}
                \AttributeTok{na =}\NormalTok{ na,}
                \AttributeTok{row.names =}\NormalTok{ row.names,}
                \AttributeTok{col.names =}\NormalTok{ col.names,}
                \AttributeTok{quote =}\NormalTok{ quote,}
                \AttributeTok{fileEncoding =}\NormalTok{ fileEncoding)}
    \FunctionTok{message}\NormalTok{(}\FunctionTok{sprintf}\NormalTok{(}\StringTok{"Arquivo salvo com sucesso em: \%s"}\NormalTok{, }\FunctionTok{normalizePath}\NormalTok{(file\_path, }\AttributeTok{winslash =} \StringTok{"/"}\NormalTok{)))}
    \ConstantTok{TRUE}
\NormalTok{  \}, }\AttributeTok{error =} \ControlFlowTok{function}\NormalTok{(e) \{}
    \FunctionTok{message}\NormalTok{(}\FunctionTok{sprintf}\NormalTok{(}\StringTok{"Falha ao salvar arquivo: \%s"}\NormalTok{, e}\SpecialCharTok{$}\NormalTok{message))}
    \ConstantTok{FALSE}
\NormalTok{  \})}

  \FunctionTok{invisible}\NormalTok{(result)}
\NormalTok{\}}

\CommentTok{\# Exemplo de uso:}
\CommentTok{\# save\_df\_to\_csv(df\_exemplo, "out/meu\_df.csv", append\_timestamp = TRUE)  \# salva com timestamp}

\FunctionTok{save\_df\_to\_csv}\NormalTok{(NHANES\_men, }\StringTok{"out/NHANES\_men.csv"}\NormalTok{)  }\CommentTok{\# salva na pasta out}
\InformationTok{\textasciigrave{}\textasciigrave{}\textasciigrave{}}
\end{Highlighting}
\end{Shaded}

Arquivo com n = 880 observações de homens com idade entre 20 e 29 anos
da pesquisa NHANES (2009-2012).

É possível ver o TCL em ação no
\href{https://www.lock5stat.com/StatKey/index.html}{statkey}. Em
especial na aba para gerar uma distribuição amostral das médias
amostrais.

O script abaixo simula uma distribuição amostral de tamanho n para BMI.

\begin{Shaded}
\begin{Highlighting}[numbers=left,,]
\InformationTok{\textasciigrave{}\textasciigrave{}\textasciigrave{}\{r\}}
\CommentTok{\# Gera distribuição amostral das médias (variável BMI) e calcula IC 95\%}
\CommentTok{\# Uso: sample\_means\_bmi(data\_or\_vector, bmi\_col = "BMI", n = 30, sims = 1000, ...)}
\CommentTok{\# Retorno: lista com vetor de médias amostrais e estatísticas (inclui intervalos de confiança)}
\NormalTok{sample\_means\_bmi }\OtherTok{\textless{}{-}} \ControlFlowTok{function}\NormalTok{(data\_or\_vector,}
                             \AttributeTok{bmi\_col =} \StringTok{"BMI"}\NormalTok{,}
                             \AttributeTok{n =} \DecValTok{30}\NormalTok{,}
                             \AttributeTok{sims =} \DecValTok{1000}\NormalTok{,}
                             \AttributeTok{replace =} \ConstantTok{TRUE}\NormalTok{,}
                             \AttributeTok{seed =} \ConstantTok{NULL}\NormalTok{,}
                             \AttributeTok{plot =} \ConstantTok{TRUE}\NormalTok{,}
                             \AttributeTok{save\_csv =} \ConstantTok{FALSE}\NormalTok{,}
                             \AttributeTok{out\_file =} \StringTok{"sampling\_means\_bmi.csv"}\NormalTok{,}
                             \AttributeTok{fileEncoding =} \StringTok{"UTF{-}8"}\NormalTok{,}
                             \AttributeTok{na.rm =} \ConstantTok{TRUE}\NormalTok{) \{}
  \CommentTok{\# Validações básicas}
  \ControlFlowTok{if}\NormalTok{ (}\SpecialCharTok{!}\FunctionTok{is.null}\NormalTok{(seed)) }\FunctionTok{set.seed}\NormalTok{(}\FunctionTok{as.integer}\NormalTok{(seed))}
  \CommentTok{\# Extrai vetor numérico de BMI}
\NormalTok{  bmi\_vec }\OtherTok{\textless{}{-}} \ConstantTok{NULL}
  \ControlFlowTok{if}\NormalTok{ (}\FunctionTok{is.data.frame}\NormalTok{(data\_or\_vector)) \{}
    \ControlFlowTok{if}\NormalTok{ (}\SpecialCharTok{!}\NormalTok{bmi\_col }\SpecialCharTok{\%in\%} \FunctionTok{names}\NormalTok{(data\_or\_vector)) \{}
      \FunctionTok{stop}\NormalTok{(}\FunctionTok{sprintf}\NormalTok{(}\StringTok{"Coluna \textquotesingle{}\%s\textquotesingle{} não encontrada no data.frame."}\NormalTok{, bmi\_col))}
\NormalTok{    \}}
\NormalTok{    bmi\_vec }\OtherTok{\textless{}{-}}\NormalTok{ data\_or\_vector[[bmi\_col]]}
\NormalTok{  \} }\ControlFlowTok{else} \ControlFlowTok{if}\NormalTok{ (}\FunctionTok{is.numeric}\NormalTok{(data\_or\_vector)) \{}
\NormalTok{    bmi\_vec }\OtherTok{\textless{}{-}}\NormalTok{ data\_or\_vector}
\NormalTok{  \} }\ControlFlowTok{else}\NormalTok{ \{}
    \FunctionTok{stop}\NormalTok{(}\StringTok{"data\_or\_vector deve ser um data.frame ou um vetor numérico."}\NormalTok{)}
\NormalTok{  \}}
  \CommentTok{\# Remove NAs se solicitado}
  \ControlFlowTok{if}\NormalTok{ (na.rm) bmi\_vec }\OtherTok{\textless{}{-}}\NormalTok{ bmi\_vec[}\SpecialCharTok{!}\FunctionTok{is.na}\NormalTok{(bmi\_vec)]}
  \ControlFlowTok{if}\NormalTok{ (}\FunctionTok{length}\NormalTok{(bmi\_vec) }\SpecialCharTok{==} \DecValTok{0}\NormalTok{) }\FunctionTok{stop}\NormalTok{(}\StringTok{"Vetor BMI está vazio após remoção de NAs."}\NormalTok{)}
  \ControlFlowTok{if}\NormalTok{ (}\SpecialCharTok{!}\FunctionTok{is.numeric}\NormalTok{(bmi\_vec)) }\FunctionTok{stop}\NormalTok{(}\StringTok{"Valores de BMI devem ser numéricos."}\NormalTok{)}

  \CommentTok{\# Parâmetros populacionais estimados (a partir do vetor fornecido)}
\NormalTok{  pop\_mean }\OtherTok{\textless{}{-}} \FunctionTok{mean}\NormalTok{(bmi\_vec)}
\NormalTok{  pop\_sd   }\OtherTok{\textless{}{-}} \FunctionTok{sd}\NormalTok{(bmi\_vec)}

  \CommentTok{\# Gerar médias amostrais}
\NormalTok{  samp\_means }\OtherTok{\textless{}{-}} \FunctionTok{numeric}\NormalTok{(sims)}
  \ControlFlowTok{for}\NormalTok{ (i }\ControlFlowTok{in} \FunctionTok{seq\_len}\NormalTok{(sims)) \{}
\NormalTok{    samp }\OtherTok{\textless{}{-}} \FunctionTok{sample}\NormalTok{(bmi\_vec, }\AttributeTok{size =}\NormalTok{ n, }\AttributeTok{replace =}\NormalTok{ replace)}
\NormalTok{    samp\_means[i] }\OtherTok{\textless{}{-}} \FunctionTok{mean}\NormalTok{(samp)}
\NormalTok{  \}}

  \CommentTok{\# Estatísticas da distribuição amostral}
\NormalTok{  dist\_mean }\OtherTok{\textless{}{-}} \FunctionTok{mean}\NormalTok{(samp\_means)}
\NormalTok{  dist\_sd   }\OtherTok{\textless{}{-}} \FunctionTok{sd}\NormalTok{(samp\_means)}
  \CommentTok{\# Desvio padrão teórico via TEORIA (se usar pop\_sd como "população")}
\NormalTok{  theoretical\_sd }\OtherTok{\textless{}{-}}\NormalTok{ pop\_sd }\SpecialCharTok{/} \FunctionTok{sqrt}\NormalTok{(n)}

  \CommentTok{\# Intervalos de confiança}
  \CommentTok{\# 1) IC 95\% para a média populacional de BMI a partir dos dados (t{-}test)}
\NormalTok{  t\_res }\OtherTok{\textless{}{-}} \FunctionTok{tryCatch}\NormalTok{(}\FunctionTok{t.test}\NormalTok{(bmi\_vec, }\AttributeTok{conf.level =} \FloatTok{0.95}\NormalTok{), }\AttributeTok{error =} \ControlFlowTok{function}\NormalTok{(e) }\ConstantTok{NULL}\NormalTok{)}
\NormalTok{  ci\_population }\OtherTok{\textless{}{-}} \ControlFlowTok{if}\NormalTok{ (}\SpecialCharTok{!}\FunctionTok{is.null}\NormalTok{(t\_res)) t\_res}\SpecialCharTok{$}\NormalTok{conf.int }\ControlFlowTok{else} \FunctionTok{c}\NormalTok{(}\ConstantTok{NA\_real\_}\NormalTok{, }\ConstantTok{NA\_real\_}\NormalTok{)}

  \CommentTok{\# 2) IC 95\% para a média amostral baseado na distribuição amostral (teórico)}
\NormalTok{  z }\OtherTok{\textless{}{-}} \FunctionTok{qnorm}\NormalTok{(}\FloatTok{0.975}\NormalTok{)}
\NormalTok{  ci\_sampling\_theoretical }\OtherTok{\textless{}{-}} \FunctionTok{c}\NormalTok{(dist\_mean }\SpecialCharTok{{-}}\NormalTok{ z }\SpecialCharTok{*}\NormalTok{ theoretical\_sd, dist\_mean }\SpecialCharTok{+}\NormalTok{ z }\SpecialCharTok{*}\NormalTok{ theoretical\_sd)}

  \CommentTok{\# 3) IC 95\% empírico da própria distribuição simulada (usando desvio empírico)}
\NormalTok{  ci\_sampling\_empirical }\OtherTok{\textless{}{-}} \FunctionTok{c}\NormalTok{(dist\_mean }\SpecialCharTok{{-}}\NormalTok{ z }\SpecialCharTok{*}\NormalTok{ dist\_sd, dist\_mean }\SpecialCharTok{+}\NormalTok{ z }\SpecialCharTok{*}\NormalTok{ dist\_sd)}

\NormalTok{  stats }\OtherTok{\textless{}{-}} \FunctionTok{list}\NormalTok{(}
    \AttributeTok{population\_mean =}\NormalTok{ pop\_mean,}
    \AttributeTok{population\_sd =}\NormalTok{ pop\_sd,}
    \AttributeTok{sampling\_mean =}\NormalTok{ dist\_mean,}
    \AttributeTok{sampling\_sd =}\NormalTok{ dist\_sd,}
    \AttributeTok{theoretical\_sd =}\NormalTok{ theoretical\_sd,}
    \AttributeTok{ci\_population\_95 =}\NormalTok{ ci\_population,}
    \AttributeTok{ci\_sampling\_theoretical\_95 =}\NormalTok{ ci\_sampling\_theoretical,}
    \AttributeTok{ci\_sampling\_empirical\_95 =}\NormalTok{ ci\_sampling\_empirical,}
    \AttributeTok{sims =}\NormalTok{ sims,}
    \AttributeTok{n =}\NormalTok{ n,}
    \AttributeTok{replace =}\NormalTok{ replace}
\NormalTok{  )}

  \CommentTok{\# Mensagem resumida com IC}
  \FunctionTok{message}\NormalTok{(}\FunctionTok{sprintf}\NormalTok{(}\StringTok{"Média (pop): \%.4f | IC95\%\% (pop, t{-}test): [\%.4f, \%.4f]"}\NormalTok{,}
\NormalTok{                  pop\_mean, ci\_population[}\DecValTok{1}\NormalTok{], ci\_population[}\DecValTok{2}\NormalTok{]))}
  \FunctionTok{message}\NormalTok{(}\FunctionTok{sprintf}\NormalTok{(}\StringTok{"Média (amostral): \%.4f | IC95\%\% (teórico): [\%.4f, \%.4f]"}\NormalTok{,}
\NormalTok{                  dist\_mean, ci\_sampling\_theoretical[}\DecValTok{1}\NormalTok{], ci\_sampling\_theoretical[}\DecValTok{2}\NormalTok{]))}

  \CommentTok{\# Salvar em CSV se pedido}
  \ControlFlowTok{if}\NormalTok{ (save\_csv) \{}
\NormalTok{    df\_out }\OtherTok{\textless{}{-}} \FunctionTok{data.frame}\NormalTok{(}\AttributeTok{sample\_mean =}\NormalTok{ samp\_means)}
\NormalTok{    write\_success }\OtherTok{\textless{}{-}} \FunctionTok{tryCatch}\NormalTok{(\{}
      \FunctionTok{write.csv}\NormalTok{(df\_out, }\AttributeTok{file =}\NormalTok{ out\_file, }\AttributeTok{row.names =} \ConstantTok{FALSE}\NormalTok{, }\AttributeTok{fileEncoding =}\NormalTok{ fileEncoding)}
      \ConstantTok{TRUE}
\NormalTok{    \}, }\AttributeTok{error =} \ControlFlowTok{function}\NormalTok{(e) \{}
      \FunctionTok{warning}\NormalTok{(}\FunctionTok{sprintf}\NormalTok{(}\StringTok{"Falha ao salvar CSV: \%s"}\NormalTok{, e}\SpecialCharTok{$}\NormalTok{message))}
      \ConstantTok{FALSE}
\NormalTok{    \})}
    \ControlFlowTok{if}\NormalTok{ (write\_success) }\FunctionTok{message}\NormalTok{(}\FunctionTok{sprintf}\NormalTok{(}\StringTok{"Distribuição salva em: \%s"}\NormalTok{, }\FunctionTok{normalizePath}\NormalTok{(out\_file, }\AttributeTok{mustWork =} \ConstantTok{FALSE}\NormalTok{)))}
\NormalTok{  \}}

  \CommentTok{\# Plotagem (base R) — histograma + densidade empírica + curva normal teórica e QQ{-}plot}
  \ControlFlowTok{if}\NormalTok{ (plot) \{}
\NormalTok{    op }\OtherTok{\textless{}{-}} \FunctionTok{par}\NormalTok{(}\AttributeTok{no.readonly =} \ConstantTok{TRUE}\NormalTok{)}
    \FunctionTok{on.exit}\NormalTok{(}\FunctionTok{par}\NormalTok{(op), }\AttributeTok{add =} \ConstantTok{TRUE}\NormalTok{)}
    \FunctionTok{par}\NormalTok{(}\AttributeTok{mfrow =} \FunctionTok{c}\NormalTok{(}\DecValTok{1}\NormalTok{, }\DecValTok{2}\NormalTok{))}
    \CommentTok{\# Histograma com densidade empírica}
    \FunctionTok{hist}\NormalTok{(samp\_means,}
         \AttributeTok{breaks =} \FunctionTok{max}\NormalTok{(}\DecValTok{10}\NormalTok{, }\FunctionTok{round}\NormalTok{(}\FunctionTok{sqrt}\NormalTok{(sims))),}
         \AttributeTok{prob =} \ConstantTok{TRUE}\NormalTok{,}
         \AttributeTok{col =} \StringTok{"\#cce5ff"}\NormalTok{,}
         \AttributeTok{border =} \StringTok{"\#2b6fa6"}\NormalTok{,}
         \AttributeTok{main =} \FunctionTok{sprintf}\NormalTok{(}\StringTok{"Distribuição amostral das médias}\SpecialCharTok{\textbackslash{}n}\StringTok{(n = \%d, sims = \%d)"}\NormalTok{, n, sims),}
         \AttributeTok{xlab =} \StringTok{"Média amostral (BMI)"}\NormalTok{)}
    \FunctionTok{lines}\NormalTok{(}\FunctionTok{density}\NormalTok{(samp\_means), }\AttributeTok{col =} \StringTok{"\#0055a4"}\NormalTok{, }\AttributeTok{lwd =} \DecValTok{2}\NormalTok{) }\CommentTok{\# densidade empírica}
    \CommentTok{\# Curva normal teórica usando média empírica e desvio teórico}
    \FunctionTok{curve}\NormalTok{(}\FunctionTok{dnorm}\NormalTok{(x, }\AttributeTok{mean =}\NormalTok{ dist\_mean, }\AttributeTok{sd =}\NormalTok{ theoretical\_sd),}
          \AttributeTok{col =} \StringTok{"\#d9534f"}\NormalTok{, }\AttributeTok{lwd =} \DecValTok{2}\NormalTok{, }\AttributeTok{add =} \ConstantTok{TRUE}\NormalTok{)}
    \FunctionTok{legend}\NormalTok{(}\StringTok{"topright"}\NormalTok{,}
           \AttributeTok{legend =} \FunctionTok{c}\NormalTok{(}\StringTok{"Densidade empírica"}\NormalTok{, }\StringTok{"Curva normal (teórica)"}\NormalTok{),}
           \AttributeTok{col =} \FunctionTok{c}\NormalTok{(}\StringTok{"\#0055a4"}\NormalTok{, }\StringTok{"\#d9534f"}\NormalTok{),}
           \AttributeTok{lwd =} \DecValTok{2}\NormalTok{, }\AttributeTok{bty =} \StringTok{"n"}\NormalTok{)}
    \CommentTok{\# Linhas verticais para os ICs}
    \FunctionTok{abline}\NormalTok{(}\AttributeTok{v =}\NormalTok{ ci\_sampling\_theoretical, }\AttributeTok{col =} \StringTok{"\#d9534f"}\NormalTok{, }\AttributeTok{lty =} \DecValTok{2}\NormalTok{, }\AttributeTok{lwd =} \FloatTok{1.5}\NormalTok{)}
    \FunctionTok{abline}\NormalTok{(}\AttributeTok{v =}\NormalTok{ ci\_sampling\_empirical, }\AttributeTok{col =} \StringTok{"\#0055a4"}\NormalTok{, }\AttributeTok{lty =} \DecValTok{3}\NormalTok{, }\AttributeTok{lwd =} \FloatTok{1.2}\NormalTok{)}
    \CommentTok{\# Marca média populacional}
    \FunctionTok{abline}\NormalTok{(}\AttributeTok{v =}\NormalTok{ pop\_mean, }\AttributeTok{col =} \StringTok{"darkgreen"}\NormalTok{, }\AttributeTok{lwd =} \DecValTok{2}\NormalTok{)}

    \CommentTok{\# QQ{-}plot para checar normalidade da distribuição amostral}
    \FunctionTok{qqnorm}\NormalTok{(samp\_means, }\AttributeTok{main =} \StringTok{"QQ{-}plot das médias amostrais"}\NormalTok{)}
    \FunctionTok{qqline}\NormalTok{(samp\_means, }\AttributeTok{col =} \StringTok{"red"}\NormalTok{, }\AttributeTok{lwd =} \DecValTok{2}\NormalTok{)}
\NormalTok{  \}}

  \FunctionTok{invisible}\NormalTok{(}\FunctionTok{list}\NormalTok{(}\AttributeTok{sampling\_means =}\NormalTok{ samp\_means, }\AttributeTok{stats =}\NormalTok{ stats))}
\NormalTok{\}}

\CommentTok{\# Exemplo de uso:}
\CommentTok{\# 1) Usando um vetor:}
\CommentTok{\# bmi\_vector \textless{}{-} c(22.1, 24.7, 30.2, 27.4, 23.5, 26.8, 21.9, 28.0, 24.3, 29.1)}
\CommentTok{\# res \textless{}{-} sample\_means\_bmi(bmi\_vector, n = 5, sims = 1000, replace = TRUE, seed = 123, plot = TRUE)}
\CommentTok{\#}
\CommentTok{\# 2) Usando um data.frame com coluna "BMI":}
\CommentTok{\# df \textless{}{-} data.frame(ID = 1:100, BMI = rnorm(100, mean = 26, sd = 4))}
\NormalTok{res }\OtherTok{\textless{}{-}} \FunctionTok{sample\_means\_bmi}\NormalTok{(NHANES\_men,}
                        \AttributeTok{bmi\_col =} \StringTok{"BMI"}\NormalTok{,}
                        \AttributeTok{n =} \DecValTok{30}\NormalTok{,}
                        \AttributeTok{sims =} \DecValTok{5000}\NormalTok{,}
                        \AttributeTok{replace =} \ConstantTok{TRUE}\NormalTok{,}
                        \AttributeTok{save\_csv =} \ConstantTok{FALSE}\NormalTok{,}
                        \AttributeTok{out\_file =} \StringTok{"means\_bmi.csv"}\NormalTok{)}
\CommentTok{\#}
\CommentTok{\# Resultado:}
\CommentTok{\# {-} res$sampling\_means: vetor com as médias amostrais}
\CommentTok{\# {-} res$stats: estatísticas resumidas (média populacional, sd populacional, média da distribuição, sd empírico, sd teórico)}
\InformationTok{\textasciigrave{}\textasciigrave{}\textasciigrave{}}
\end{Highlighting}
\end{Shaded}

\pandocbounded{\includegraphics[keepaspectratio]{cap16-moore-IC-o-basico_files/figure-pdf/unnamed-chunk-4-1.pdf}}

A seguir 4 simulações com amostras de tamanho n = 10, 20, 30 e 50.

\begin{Shaded}
\begin{Highlighting}[numbers=left,,]
\InformationTok{\textasciigrave{}\textasciigrave{}\textasciigrave{}\{r\}}
\CommentTok{\# Função para gerar médias amostrais (sem plot automático) — versão enxuta da função anterior}
\NormalTok{sample\_means\_bmi }\OtherTok{\textless{}{-}} \ControlFlowTok{function}\NormalTok{(data\_or\_vector,}
                             \AttributeTok{bmi\_col =} \StringTok{"BMI"}\NormalTok{,}
                             \AttributeTok{n =} \DecValTok{30}\NormalTok{,}
                             \AttributeTok{sims =} \DecValTok{1000}\NormalTok{,}
                             \AttributeTok{replace =} \ConstantTok{TRUE}\NormalTok{,}
                             \AttributeTok{seed =} \ConstantTok{NULL}\NormalTok{,}
                             \AttributeTok{na.rm =} \ConstantTok{TRUE}\NormalTok{) \{}
  \ControlFlowTok{if}\NormalTok{ (}\SpecialCharTok{!}\FunctionTok{is.null}\NormalTok{(seed)) }\FunctionTok{set.seed}\NormalTok{(}\FunctionTok{as.integer}\NormalTok{(seed))}
\NormalTok{  bmi\_vec }\OtherTok{\textless{}{-}} \ConstantTok{NULL}
  \ControlFlowTok{if}\NormalTok{ (}\FunctionTok{is.data.frame}\NormalTok{(data\_or\_vector)) \{}
    \ControlFlowTok{if}\NormalTok{ (}\SpecialCharTok{!}\NormalTok{bmi\_col }\SpecialCharTok{\%in\%} \FunctionTok{names}\NormalTok{(data\_or\_vector)) }\FunctionTok{stop}\NormalTok{(}\FunctionTok{sprintf}\NormalTok{(}\StringTok{"Coluna \textquotesingle{}\%s\textquotesingle{} não encontrada."}\NormalTok{, bmi\_col))}
\NormalTok{    bmi\_vec }\OtherTok{\textless{}{-}}\NormalTok{ data\_or\_vector[[bmi\_col]]}
\NormalTok{  \} }\ControlFlowTok{else} \ControlFlowTok{if}\NormalTok{ (}\FunctionTok{is.numeric}\NormalTok{(data\_or\_vector)) \{}
\NormalTok{    bmi\_vec }\OtherTok{\textless{}{-}}\NormalTok{ data\_or\_vector}
\NormalTok{  \} }\ControlFlowTok{else} \FunctionTok{stop}\NormalTok{(}\StringTok{"data\_or\_vector deve ser data.frame ou vetor numérico."}\NormalTok{)}
  \ControlFlowTok{if}\NormalTok{ (na.rm) bmi\_vec }\OtherTok{\textless{}{-}}\NormalTok{ bmi\_vec[}\SpecialCharTok{!}\FunctionTok{is.na}\NormalTok{(bmi\_vec)]}
  \ControlFlowTok{if}\NormalTok{ (}\FunctionTok{length}\NormalTok{(bmi\_vec) }\SpecialCharTok{==} \DecValTok{0}\NormalTok{) }\FunctionTok{stop}\NormalTok{(}\StringTok{"Vetor BMI vazio após remoção de NAs."}\NormalTok{)}
\NormalTok{  pop\_mean }\OtherTok{\textless{}{-}} \FunctionTok{mean}\NormalTok{(bmi\_vec)}
\NormalTok{  pop\_sd   }\OtherTok{\textless{}{-}} \FunctionTok{sd}\NormalTok{(bmi\_vec)}
\NormalTok{  samp\_means }\OtherTok{\textless{}{-}} \FunctionTok{numeric}\NormalTok{(sims)}
  \ControlFlowTok{for}\NormalTok{ (i }\ControlFlowTok{in} \FunctionTok{seq\_len}\NormalTok{(sims)) \{}
\NormalTok{    samp }\OtherTok{\textless{}{-}} \FunctionTok{sample}\NormalTok{(bmi\_vec, }\AttributeTok{size =}\NormalTok{ n, }\AttributeTok{replace =}\NormalTok{ replace)}
\NormalTok{    samp\_means[i] }\OtherTok{\textless{}{-}} \FunctionTok{mean}\NormalTok{(samp)}
\NormalTok{  \}}
\NormalTok{  dist\_mean }\OtherTok{\textless{}{-}} \FunctionTok{mean}\NormalTok{(samp\_means)}
\NormalTok{  dist\_sd   }\OtherTok{\textless{}{-}} \FunctionTok{sd}\NormalTok{(samp\_means)}
\NormalTok{  theoretical\_sd }\OtherTok{\textless{}{-}}\NormalTok{ pop\_sd }\SpecialCharTok{/} \FunctionTok{sqrt}\NormalTok{(n)}
  \FunctionTok{list}\NormalTok{(}\AttributeTok{sampling\_means =}\NormalTok{ samp\_means,}
       \AttributeTok{stats =} \FunctionTok{list}\NormalTok{(}\AttributeTok{population\_mean =}\NormalTok{ pop\_mean,}
                    \AttributeTok{population\_sd =}\NormalTok{ pop\_sd,}
                    \AttributeTok{sampling\_mean =}\NormalTok{ dist\_mean,}
                    \AttributeTok{sampling\_sd =}\NormalTok{ dist\_sd,}
                    \AttributeTok{theoretical\_sd =}\NormalTok{ theoretical\_sd,}
                    \AttributeTok{n =}\NormalTok{ n,}
                    \AttributeTok{sims =}\NormalTok{ sims,}
                    \AttributeTok{replace =}\NormalTok{ replace))}
\NormalTok{\}}

\CommentTok{\# Função para plotar matriz 2x2 com histogramas (n valores padrão: 10,20,30,50)}
\NormalTok{plot\_sampling\_matrix }\OtherTok{\textless{}{-}} \ControlFlowTok{function}\NormalTok{(data\_or\_vector,}
                                 \AttributeTok{bmi\_col =} \StringTok{"BMI"}\NormalTok{,}
                                 \AttributeTok{ns =} \FunctionTok{c}\NormalTok{(}\DecValTok{10}\NormalTok{, }\DecValTok{20}\NormalTok{, }\DecValTok{30}\NormalTok{, }\DecValTok{50}\NormalTok{),}
                                 \AttributeTok{sims =} \DecValTok{2000}\NormalTok{,}
                                 \AttributeTok{replace =} \ConstantTok{TRUE}\NormalTok{,}
                                 \AttributeTok{seed =} \ConstantTok{NULL}\NormalTok{,}
                                 \AttributeTok{conf.level =} \FloatTok{0.95}\NormalTok{,}
                                 \AttributeTok{main\_title =} \StringTok{"Distribuição amostral médias (BMI)"}\NormalTok{,}
                                 \AttributeTok{colors =} \FunctionTok{list}\NormalTok{(}\AttributeTok{bg =} \StringTok{"\#f7fbff"}\NormalTok{, }\AttributeTok{hist =} \StringTok{"\#cce5ff"}\NormalTok{, }\AttributeTok{dens =} \StringTok{"\#0055a4"}\NormalTok{, }\AttributeTok{normal =} \StringTok{"\#d9534f"}\NormalTok{)) \{}
  \ControlFlowTok{if}\NormalTok{ (}\SpecialCharTok{!}\FunctionTok{is.null}\NormalTok{(seed)) }\FunctionTok{set.seed}\NormalTok{(}\FunctionTok{as.integer}\NormalTok{(seed))}
  \ControlFlowTok{if}\NormalTok{ (}\FunctionTok{length}\NormalTok{(ns) }\SpecialCharTok{!=} \DecValTok{4}\NormalTok{) }\FunctionTok{stop}\NormalTok{(}\StringTok{"Parâmetro \textquotesingle{}ns\textquotesingle{} deve conter exatamente 4 tamanhos para a matriz 2x2."}\NormalTok{)}
\NormalTok{  z }\OtherTok{\textless{}{-}} \FunctionTok{qnorm}\NormalTok{((}\DecValTok{1} \SpecialCharTok{+}\NormalTok{ conf.level) }\SpecialCharTok{/} \DecValTok{2}\NormalTok{)}
\NormalTok{  op }\OtherTok{\textless{}{-}} \FunctionTok{par}\NormalTok{(}\AttributeTok{no.readonly =} \ConstantTok{TRUE}\NormalTok{)}
  \FunctionTok{on.exit}\NormalTok{(}\FunctionTok{par}\NormalTok{(op), }\AttributeTok{add =} \ConstantTok{TRUE}\NormalTok{)}
  \FunctionTok{par}\NormalTok{(}\AttributeTok{mfrow =} \FunctionTok{c}\NormalTok{(}\DecValTok{2}\NormalTok{, }\DecValTok{2}\NormalTok{), }\AttributeTok{mar =} \FunctionTok{c}\NormalTok{(}\FloatTok{4.2}\NormalTok{, }\DecValTok{4}\NormalTok{, }\DecValTok{3}\NormalTok{, }\DecValTok{1}\NormalTok{))}
\NormalTok{  results }\OtherTok{\textless{}{-}} \FunctionTok{vector}\NormalTok{(}\StringTok{"list"}\NormalTok{, }\FunctionTok{length}\NormalTok{(ns))}
  \FunctionTok{names}\NormalTok{(results) }\OtherTok{\textless{}{-}} \FunctionTok{paste0}\NormalTok{(}\StringTok{"n="}\NormalTok{, ns)}
  \ControlFlowTok{for}\NormalTok{ (i }\ControlFlowTok{in} \FunctionTok{seq\_along}\NormalTok{(ns)) \{}
\NormalTok{    n }\OtherTok{\textless{}{-}}\NormalTok{ ns[i]}
\NormalTok{    res }\OtherTok{\textless{}{-}} \FunctionTok{sample\_means\_bmi}\NormalTok{(data\_or\_vector, }\AttributeTok{bmi\_col =}\NormalTok{ bmi\_col, }\AttributeTok{n =}\NormalTok{ n, }\AttributeTok{sims =}\NormalTok{ sims, }\AttributeTok{replace =}\NormalTok{ replace)}
\NormalTok{    samp\_means }\OtherTok{\textless{}{-}}\NormalTok{ res}\SpecialCharTok{$}\NormalTok{sampling\_means}
\NormalTok{    stats }\OtherTok{\textless{}{-}}\NormalTok{ res}\SpecialCharTok{$}\NormalTok{stats}
\NormalTok{    pop\_mean }\OtherTok{\textless{}{-}}\NormalTok{ stats}\SpecialCharTok{$}\NormalTok{population\_mean}
\NormalTok{    dist\_mean }\OtherTok{\textless{}{-}}\NormalTok{ stats}\SpecialCharTok{$}\NormalTok{sampling\_mean}
\NormalTok{    dist\_sd }\OtherTok{\textless{}{-}}\NormalTok{ stats}\SpecialCharTok{$}\NormalTok{sampling\_sd}
\NormalTok{    theoretical\_sd }\OtherTok{\textless{}{-}}\NormalTok{ stats}\SpecialCharTok{$}\NormalTok{theoretical\_sd}
\NormalTok{    ci\_theoretical }\OtherTok{\textless{}{-}} \FunctionTok{c}\NormalTok{(dist\_mean }\SpecialCharTok{{-}}\NormalTok{ z }\SpecialCharTok{*}\NormalTok{ theoretical\_sd, dist\_mean }\SpecialCharTok{+}\NormalTok{ z }\SpecialCharTok{*}\NormalTok{ theoretical\_sd)}
\NormalTok{    ci\_empirical  }\OtherTok{\textless{}{-}} \FunctionTok{c}\NormalTok{(dist\_mean }\SpecialCharTok{{-}}\NormalTok{ z }\SpecialCharTok{*}\NormalTok{ dist\_sd, dist\_mean }\SpecialCharTok{+}\NormalTok{ z }\SpecialCharTok{*}\NormalTok{ dist\_sd)}
    \CommentTok{\# Histograma}
    \FunctionTok{hist}\NormalTok{(samp\_means,}
         \AttributeTok{breaks =} \FunctionTok{max}\NormalTok{(}\DecValTok{10}\NormalTok{, }\FunctionTok{round}\NormalTok{(}\FunctionTok{sqrt}\NormalTok{(sims))),}
         \AttributeTok{prob =} \ConstantTok{TRUE}\NormalTok{,}
         \AttributeTok{col =}\NormalTok{ colors}\SpecialCharTok{$}\NormalTok{hist,}
         \AttributeTok{border =} \StringTok{"\#2b6fa6"}\NormalTok{,}
         \AttributeTok{main =} \FunctionTok{sprintf}\NormalTok{(}\StringTok{"\%s (n=\%d)"}\NormalTok{, }\FunctionTok{paste0}\NormalTok{(main\_title, }\StringTok{"}\SpecialCharTok{\textbackslash{}n}\StringTok{"}\NormalTok{) , n),}
         \AttributeTok{xlab =} \StringTok{"Média amostral (BMI)"}\NormalTok{,}
         \AttributeTok{ylab =} \StringTok{"Densidade"}\NormalTok{,}
         \AttributeTok{ylim =} \FunctionTok{c}\NormalTok{(}\DecValTok{0}\NormalTok{, }\FunctionTok{max}\NormalTok{(}\FunctionTok{density}\NormalTok{(samp\_means)}\SpecialCharTok{$}\NormalTok{y, }\FloatTok{0.1}\NormalTok{)))}
    \FunctionTok{lines}\NormalTok{(}\FunctionTok{density}\NormalTok{(samp\_means), }\AttributeTok{col =}\NormalTok{ colors}\SpecialCharTok{$}\NormalTok{dens, }\AttributeTok{lwd =} \DecValTok{2}\NormalTok{)}
    \FunctionTok{curve}\NormalTok{(}\FunctionTok{dnorm}\NormalTok{(x, }\AttributeTok{mean =}\NormalTok{ dist\_mean, }\AttributeTok{sd =}\NormalTok{ theoretical\_sd),}
          \AttributeTok{col =}\NormalTok{ colors}\SpecialCharTok{$}\NormalTok{normal, }\AttributeTok{lwd =} \DecValTok{2}\NormalTok{, }\AttributeTok{add =} \ConstantTok{TRUE}\NormalTok{)}
    \CommentTok{\# Linhas IC e média população}
    \FunctionTok{abline}\NormalTok{(}\AttributeTok{v =}\NormalTok{ ci\_theoretical, }\AttributeTok{col =}\NormalTok{ colors}\SpecialCharTok{$}\NormalTok{normal, }\AttributeTok{lty =} \DecValTok{2}\NormalTok{, }\AttributeTok{lwd =} \FloatTok{1.5}\NormalTok{)}
    \FunctionTok{abline}\NormalTok{(}\AttributeTok{v =}\NormalTok{ ci\_empirical, }\AttributeTok{col =}\NormalTok{ colors}\SpecialCharTok{$}\NormalTok{dens, }\AttributeTok{lty =} \DecValTok{3}\NormalTok{, }\AttributeTok{lwd =} \FloatTok{1.2}\NormalTok{)}
    \FunctionTok{abline}\NormalTok{(}\AttributeTok{v =}\NormalTok{ pop\_mean, }\AttributeTok{col =} \StringTok{"darkgreen"}\NormalTok{, }\AttributeTok{lwd =} \DecValTok{2}\NormalTok{)}
    \CommentTok{\# Legenda compacta com estatísticas principais}
    \FunctionTok{legend}\NormalTok{(}\StringTok{"topright"}\NormalTok{,}
           \AttributeTok{legend =} \FunctionTok{c}\NormalTok{(}\FunctionTok{sprintf}\NormalTok{(}\StringTok{"µ pop = \%.2f"}\NormalTok{, pop\_mean),}
                      \FunctionTok{sprintf}\NormalTok{(}\StringTok{"µ samp = \%.2f"}\NormalTok{, dist\_mean),}
                      \FunctionTok{sprintf}\NormalTok{(}\StringTok{"IC teor. \%.0f\%\%: [\%.2f, \%.2f]"}\NormalTok{, conf.level}\SpecialCharTok{*}\DecValTok{100}\NormalTok{, ci\_theoretical[}\DecValTok{1}\NormalTok{], ci\_theoretical[}\DecValTok{2}\NormalTok{])),}
           \AttributeTok{bg =} \StringTok{"white"}\NormalTok{, }\AttributeTok{cex =} \FloatTok{0.9}\NormalTok{, }\AttributeTok{bty =} \StringTok{"n"}\NormalTok{)}
    \CommentTok{\# Guardar resultado}
\NormalTok{    results[[i]] }\OtherTok{\textless{}{-}} \FunctionTok{list}\NormalTok{(}\AttributeTok{n =}\NormalTok{ n, }\AttributeTok{sampling\_means =}\NormalTok{ samp\_means, }\AttributeTok{stats =}\NormalTok{ stats,}
                         \AttributeTok{ci\_theoretical =}\NormalTok{ ci\_theoretical, }\AttributeTok{ci\_empirical =}\NormalTok{ ci\_empirical)}
\NormalTok{  \}}
  \FunctionTok{invisible}\NormalTok{(results)}
\NormalTok{\}}

\CommentTok{\# Exemplo de uso:}
\CommentTok{\# 1) Com vetor de BMI}
\CommentTok{\# bmi\_vector \textless{}{-} rnorm(500, mean = 26, sd = 4)}
\CommentTok{\# plot\_sampling\_matrix(bmi\_vector, ns = c(10,20,30,50), sims = 2000, seed = 123)}

\CommentTok{\#}
\CommentTok{\# 2) Com data.frame contendo coluna "BMI"}
\CommentTok{\# df \textless{}{-} data.frame(ID = 1:500, BMI = rnorm(500, 26, 4))}
\FunctionTok{plot\_sampling\_matrix}\NormalTok{(NHANES\_men,}
                     \AttributeTok{bmi\_col =} \StringTok{"BMI"}\NormalTok{,}
                     \AttributeTok{ns =} \FunctionTok{c}\NormalTok{(}\DecValTok{10}\NormalTok{,}\DecValTok{20}\NormalTok{,}\DecValTok{30}\NormalTok{,}\DecValTok{50}\NormalTok{),}
                     \AttributeTok{sims =} \DecValTok{2000}\NormalTok{,}
                     \AttributeTok{seed =} \DecValTok{123}\NormalTok{)}
\InformationTok{\textasciigrave{}\textasciigrave{}\textasciigrave{}}
\end{Highlighting}
\end{Shaded}

\pandocbounded{\includegraphics[keepaspectratio]{cap16-moore-IC-o-basico_files/figure-pdf/unnamed-chunk-5-1.pdf}}

Verificar o efeito de aumentar o tamanho da amostra para 100, 200, 300,
500 no IC95\%.

\begin{Shaded}
\begin{Highlighting}[numbers=left,,]
\InformationTok{\textasciigrave{}\textasciigrave{}\textasciigrave{}\{r\}}
\FunctionTok{plot\_sampling\_matrix}\NormalTok{(NHANES\_men,}
                     \AttributeTok{bmi\_col =} \StringTok{"BMI"}\NormalTok{,}
                     \AttributeTok{ns =} \FunctionTok{c}\NormalTok{(}\DecValTok{100}\NormalTok{,}\DecValTok{200}\NormalTok{,}\DecValTok{300}\NormalTok{,}\DecValTok{500}\NormalTok{),}
                     \AttributeTok{sims =} \DecValTok{2000}\NormalTok{,}
                     \AttributeTok{seed =} \DecValTok{123}\NormalTok{)}
\InformationTok{\textasciigrave{}\textasciigrave{}\textasciigrave{}}
\end{Highlighting}
\end{Shaded}

\pandocbounded{\includegraphics[keepaspectratio]{cap16-moore-IC-o-basico_files/figure-pdf/unnamed-chunk-6-1.pdf}}

A ideia principal é que a \textbf{\emph{distribuição amostral}} de
\(\bar{x}\) {[}\textbf{\emph{das médias amostrais}}{]} nos
\textbf{\emph{diz quão próximo de µ \ul{está}}},
\ul{\textbf{provavelmente}}, a \textbf{\emph{média amostral}}
\(\bar{x}\).

A \textbf{\emph{estimação}} estatística apenas \textbf{\emph{inverte}}
essa informação \ul{\textbf{para dizer quão perto de}} \(\bar{x}\) a
\ul{\textbf{média populacional}} µ \ul{\textbf{provavelmente estará}}.

Chamamos o intervalo de números entre os valores \(\bar{x}\) ± 0,8 de
\ul{\textbf{intervalo de confiança de 95\%}} para µ.

\section{Margem de erro e nível de
confiança}\label{margem-de-erro-e-nuxedvel-de-confianuxe7a}

A maioria dos intervalos de confiança tem forma similar a esta:

\[
estimativa \pm \text{margem de erro}
\]

A estimativa (\(\bar{x}\) = 27,2 no nosso exemplo) é a \ul{\textbf{nossa
conjectura}} sobre o \ul{\textbf{valor}} do \ul{\textbf{parâmetro
desconhecido}}.

A margem de erro {[}ME{]} ±0,8 mostra o \ul{\textbf{grau de precisão}}
que \ul{\textbf{acreditamos}} que \ul{\textbf{nossa conjectura tenha}},
\ul{\textbf{com base na variabilidade da estimativa}}.

Temos \ul{\textbf{um intervalo de confiança de 95\%}} porque o
\ul{\textbf{intervalo}} \(\bar{x}\) ± 0,8 \ul{\textbf{\emph{contém o
parâmetro desconhecido em 95\% de todas as amostras possíveis}}}.

Essa \ul{\textbf{forma}} para um intervalo de confiança e
\ul{\textbf{sua interpretação}} se \emph{\textbf{aplicam} à maioria dos
\textbf{parâmetros}} que consideraremos neste livro, incluindo
\ul{\textbf{médias}} e \ul{\textbf{proporções}}.

\begin{tcolorbox}[enhanced jigsaw, arc=.35mm, opacitybacktitle=0.6, colframe=quarto-callout-important-color-frame, titlerule=0mm, leftrule=.75mm, left=2mm, colbacktitle=quarto-callout-important-color!10!white, breakable, toprule=.15mm, bottomtitle=1mm, opacityback=0, coltitle=black, title=\textcolor{quarto-callout-important-color}{\faExclamation}\hspace{0.5em}{Margem de erro}, rightrule=.15mm, bottomrule=.15mm, toptitle=1mm, colback=white]

A margem de erro é um número que é acrescentado a, ou subtraído de uma
estimativa estatística para definir o intervalo de confiança a dado
nível de confiança.

\end{tcolorbox}

Os usuários podem escolher o nível de confiança, quase sempre 90\% ou
mais, por quererem estar bastante seguros de suas conclusões. O nível de
confiança mais comum é 95\%.

\begin{tcolorbox}[enhanced jigsaw, arc=.35mm, opacitybacktitle=0.6, colframe=quarto-callout-important-color-frame, titlerule=0mm, leftrule=.75mm, left=2mm, colbacktitle=quarto-callout-important-color!10!white, breakable, toprule=.15mm, bottomtitle=1mm, opacityback=0, coltitle=black, title=\textcolor{quarto-callout-important-color}{\faExclamation}\hspace{0.5em}{Intervalo de confiança}, rightrule=.15mm, bottomrule=.15mm, toptitle=1mm, colback=white]

Um intervalo de confiança de nível C para um parâmetro tem \textbf{duas
partes}:

•Um intervalo calculado a partir dos dados, usualmente da \textbf{forma}

\[
estimativa \pm \text{margem de erro}
\]

•Um \textbf{nível de confiança C}, que dá a \ul{\textbf{probabilidade}}
de que \ul{\textbf{o intervalo contenha o verdadeiro valor do parâmetro
em amostras repetidas}}. Ou seja, o nível de confiança \ul{\textbf{é a
taxa de sucesso do método}}.

\end{tcolorbox}

NC que tem de ser interpretado do seguinte modo.

\begin{tcolorbox}[enhanced jigsaw, arc=.35mm, opacitybacktitle=0.6, colframe=quarto-callout-important-color-frame, titlerule=0mm, leftrule=.75mm, left=2mm, colbacktitle=quarto-callout-important-color!10!white, breakable, toprule=.15mm, bottomtitle=1mm, opacityback=0, coltitle=black, title=\textcolor{quarto-callout-important-color}{\faExclamation}\hspace{0.5em}{Interpretação de um nível de confiança}, rightrule=.15mm, bottomrule=.15mm, toptitle=1mm, colback=white]

O nível de confiança \ul{\textbf{é a taxa de sucesso do método que
produz o intervalo}}. Não sabemos se o intervalo de confiança de 95\%
obtido a partir de uma amostra particular é um dos 95\% que contêm µ, ou
se é um dos 5\% que não contêm.

Dizer que temos 95\% de confiança em que o parâmetro desconhecido µ
esteja entre 26,4 e 28,0 é uma maneira abreviada de dizer que
\ul{\textbf{``Obtivemos esses números por um método que fornece
resultados corretos em 95\% das vezes''}}.

\end{tcolorbox}

\subsection{EXEMPLO 16.2 Estimação estatística em
figuras}\label{exemplo-16.2-estimauxe7uxe3o-estatuxedstica-em-figuras}

Um srcipt R que simula 20 IC-NC95\% para o \texttt{BMI} do data set
\texttt{NHANES}.

\begin{Shaded}
\begin{Highlighting}[numbers=left,,]
\InformationTok{\textasciigrave{}\textasciigrave{}\textasciigrave{}\{r\}}
\CommentTok{\# Simula 20 IC 95\% para a média do BMI no dataset NHANES e plota os intervalos.}
\CommentTok{\# Requisitos: pacote \textquotesingle{}NHANES\textquotesingle{} (CRAN). Instale com install.packages("NHANES") se necessário.}
\CommentTok{\#}
\CommentTok{\# Uso:}
\CommentTok{\#   simular\_20\_ic\_bmi\_nhanes(nsims = 20, n = 50, seed = 123, replace = TRUE, save\_csv = FALSE, out\_file = "ic20\_bmi.csv")}
\CommentTok{\#}
\NormalTok{simular\_20\_ic\_bmi\_nhanes }\OtherTok{\textless{}{-}} \ControlFlowTok{function}\NormalTok{(}\AttributeTok{nsims =} \DecValTok{20}\NormalTok{,}
                                     \AttributeTok{n =} \DecValTok{50}\NormalTok{,}
                                     \AttributeTok{seed =} \DecValTok{123}\NormalTok{,}
                                     \AttributeTok{replace =} \ConstantTok{TRUE}\NormalTok{,}
                                     \AttributeTok{bmi\_col =} \StringTok{"BMI"}\NormalTok{,}
                                     \AttributeTok{conf.level =} \FloatTok{0.95}\NormalTok{,}
                                     \AttributeTok{save\_csv =} \ConstantTok{FALSE}\NormalTok{,}
                                     \AttributeTok{out\_file =} \StringTok{"ic20\_bmi.csv"}\NormalTok{,}
                                     \AttributeTok{quiet =} \ConstantTok{FALSE}\NormalTok{) \{}
  \ControlFlowTok{if}\NormalTok{ (}\SpecialCharTok{!}\FunctionTok{requireNamespace}\NormalTok{(}\StringTok{"NHANES"}\NormalTok{, }\AttributeTok{quietly =} \ConstantTok{TRUE}\NormalTok{)) \{}
    \FunctionTok{stop}\NormalTok{(}\StringTok{"Pacote \textquotesingle{}NHANES\textquotesingle{} não encontrado. Instale com: install.packages(\textquotesingle{}NHANES\textquotesingle{})"}\NormalTok{)}
\NormalTok{  \}}
  \ControlFlowTok{if}\NormalTok{ (}\SpecialCharTok{!}\FunctionTok{is.numeric}\NormalTok{(nsims) }\SpecialCharTok{||}\NormalTok{ nsims }\SpecialCharTok{\textless{}=} \DecValTok{0}\NormalTok{) }\FunctionTok{stop}\NormalTok{(}\StringTok{"\textquotesingle{}nsims\textquotesingle{} deve ser inteiro positivo."}\NormalTok{)}
  \ControlFlowTok{if}\NormalTok{ (}\SpecialCharTok{!}\FunctionTok{is.numeric}\NormalTok{(n) }\SpecialCharTok{||}\NormalTok{ n }\SpecialCharTok{\textless{}=} \DecValTok{1}\NormalTok{) }\FunctionTok{stop}\NormalTok{(}\StringTok{"\textquotesingle{}n\textquotesingle{} deve ser inteiro maior que 1."}\NormalTok{)}
  \FunctionTok{set.seed}\NormalTok{(}\FunctionTok{as.integer}\NormalTok{(seed))}

\NormalTok{  data }\OtherTok{\textless{}{-}}\NormalTok{ NHANES}\SpecialCharTok{::}\NormalTok{NHANES}
  \ControlFlowTok{if}\NormalTok{ (}\SpecialCharTok{!}\NormalTok{bmi\_col }\SpecialCharTok{\%in\%} \FunctionTok{names}\NormalTok{(data)) }\FunctionTok{stop}\NormalTok{(}\FunctionTok{sprintf}\NormalTok{(}\StringTok{"Coluna \textquotesingle{}\%s\textquotesingle{} não encontrada no dataset NHANES."}\NormalTok{, bmi\_col))}
\NormalTok{  bmi\_all }\OtherTok{\textless{}{-}}\NormalTok{ data[[bmi\_col]]}
\NormalTok{  bmi\_all }\OtherTok{\textless{}{-}}\NormalTok{ bmi\_all[}\SpecialCharTok{!}\FunctionTok{is.na}\NormalTok{(bmi\_all)]}
  \ControlFlowTok{if}\NormalTok{ (}\FunctionTok{length}\NormalTok{(bmi\_all) }\SpecialCharTok{\textless{}} \DecValTok{2}\NormalTok{) }\FunctionTok{stop}\NormalTok{(}\StringTok{"Poucos valores de BMI disponíveis no dataset."}\NormalTok{)}

  \CommentTok{\# "Média populacional" estimada a partir do dataset completo}
\NormalTok{  pop\_mean }\OtherTok{\textless{}{-}} \FunctionTok{mean}\NormalTok{(bmi\_all)}
\NormalTok{  z\_or\_t }\OtherTok{\textless{}{-}} \FunctionTok{qt}\NormalTok{((}\DecValTok{1} \SpecialCharTok{+}\NormalTok{ conf.level) }\SpecialCharTok{/} \DecValTok{2}\NormalTok{, }\AttributeTok{df =}\NormalTok{ n }\SpecialCharTok{{-}} \DecValTok{1}\NormalTok{) }\CommentTok{\# t crítico}

  \CommentTok{\# Armazenar resultados}
\NormalTok{  res }\OtherTok{\textless{}{-}} \FunctionTok{data.frame}\NormalTok{(}\AttributeTok{iter =} \FunctionTok{seq\_len}\NormalTok{(nsims),}
                    \AttributeTok{mean =} \FunctionTok{numeric}\NormalTok{(nsims),}
                    \AttributeTok{sd =} \FunctionTok{numeric}\NormalTok{(nsims),}
                    \AttributeTok{se =} \FunctionTok{numeric}\NormalTok{(nsims),}
                    \AttributeTok{lower =} \FunctionTok{numeric}\NormalTok{(nsims),}
                    \AttributeTok{upper =} \FunctionTok{numeric}\NormalTok{(nsims),}
                    \AttributeTok{contains =} \FunctionTok{logical}\NormalTok{(nsims),}
                    \AttributeTok{stringsAsFactors =} \ConstantTok{FALSE}\NormalTok{)}

  \ControlFlowTok{for}\NormalTok{ (i }\ControlFlowTok{in} \FunctionTok{seq\_len}\NormalTok{(nsims)) \{}
\NormalTok{    samp }\OtherTok{\textless{}{-}} \FunctionTok{sample}\NormalTok{(bmi\_all, }\AttributeTok{size =}\NormalTok{ n, }\AttributeTok{replace =}\NormalTok{ replace)}
\NormalTok{    m }\OtherTok{\textless{}{-}} \FunctionTok{mean}\NormalTok{(samp)}
\NormalTok{    s }\OtherTok{\textless{}{-}} \FunctionTok{sd}\NormalTok{(samp)}
\NormalTok{    se }\OtherTok{\textless{}{-}}\NormalTok{ s }\SpecialCharTok{/} \FunctionTok{sqrt}\NormalTok{(n)}
\NormalTok{    lower }\OtherTok{\textless{}{-}}\NormalTok{ m }\SpecialCharTok{{-}}\NormalTok{ z\_or\_t }\SpecialCharTok{*}\NormalTok{ se}
\NormalTok{    upper }\OtherTok{\textless{}{-}}\NormalTok{ m }\SpecialCharTok{+}\NormalTok{ z\_or\_t }\SpecialCharTok{*}\NormalTok{ se}
\NormalTok{    contains }\OtherTok{\textless{}{-}}\NormalTok{ (lower }\SpecialCharTok{\textless{}=}\NormalTok{ pop\_mean) }\SpecialCharTok{\&\&}\NormalTok{ (pop\_mean }\SpecialCharTok{\textless{}=}\NormalTok{ upper)}

\NormalTok{    res}\SpecialCharTok{$}\NormalTok{mean[i]  }\OtherTok{\textless{}{-}}\NormalTok{ m}
\NormalTok{    res}\SpecialCharTok{$}\NormalTok{sd[i]    }\OtherTok{\textless{}{-}}\NormalTok{ s}
\NormalTok{    res}\SpecialCharTok{$}\NormalTok{se[i]    }\OtherTok{\textless{}{-}}\NormalTok{ se}
\NormalTok{    res}\SpecialCharTok{$}\NormalTok{lower[i] }\OtherTok{\textless{}{-}}\NormalTok{ lower}
\NormalTok{    res}\SpecialCharTok{$}\NormalTok{upper[i] }\OtherTok{\textless{}{-}}\NormalTok{ upper}
\NormalTok{    res}\SpecialCharTok{$}\NormalTok{contains[i] }\OtherTok{\textless{}{-}}\NormalTok{ contains}
\NormalTok{  \}}

\NormalTok{  coverage }\OtherTok{\textless{}{-}} \FunctionTok{mean}\NormalTok{(res}\SpecialCharTok{$}\NormalTok{contains)}
  \ControlFlowTok{if}\NormalTok{ (}\SpecialCharTok{!}\NormalTok{quiet) \{}
    \FunctionTok{message}\NormalTok{(}\FunctionTok{sprintf}\NormalTok{(}\StringTok{"Média estimada (}\SpecialCharTok{\textbackslash{}"}\StringTok{população}\SpecialCharTok{\textbackslash{}"}\StringTok{ NHANES): \%.4f"}\NormalTok{, pop\_mean))}
    \FunctionTok{message}\NormalTok{(}\FunctionTok{sprintf}\NormalTok{(}\StringTok{"Simulações: \%d | n = \%d | IC nível: \%.1f\%\%"}\NormalTok{, nsims, n, conf.level }\SpecialCharTok{*} \DecValTok{100}\NormalTok{))}
    \FunctionTok{message}\NormalTok{(}\FunctionTok{sprintf}\NormalTok{(}\StringTok{"Cobertura observada (proporção de ICs que contêm a média populacional): \%.2f\%\%"}\NormalTok{, coverage }\SpecialCharTok{*} \DecValTok{100}\NormalTok{))}
\NormalTok{  \}}

  \CommentTok{\# Plot básico (base R): cada linha = um IC; cor azul = contém, vermelho = não contém}
\NormalTok{  x\_min }\OtherTok{\textless{}{-}} \FunctionTok{min}\NormalTok{(res}\SpecialCharTok{$}\NormalTok{lower)}
\NormalTok{  x\_max }\OtherTok{\textless{}{-}} \FunctionTok{max}\NormalTok{(res}\SpecialCharTok{$}\NormalTok{upper)}
\NormalTok{  op }\OtherTok{\textless{}{-}} \FunctionTok{par}\NormalTok{(}\AttributeTok{no.readonly =} \ConstantTok{TRUE}\NormalTok{)}
  \FunctionTok{on.exit}\NormalTok{(}\FunctionTok{par}\NormalTok{(op), }\AttributeTok{add =} \ConstantTok{TRUE}\NormalTok{)}
  \FunctionTok{par}\NormalTok{(}\AttributeTok{mar =} \FunctionTok{c}\NormalTok{(}\FloatTok{4.2}\NormalTok{, }\FloatTok{2.5}\NormalTok{, }\FloatTok{3.2}\NormalTok{, }\FloatTok{1.5}\NormalTok{))}
  \FunctionTok{plot}\NormalTok{(}\ConstantTok{NA}\NormalTok{, }\AttributeTok{xlim =} \FunctionTok{c}\NormalTok{(x\_min, x\_max), }\AttributeTok{ylim =} \FunctionTok{c}\NormalTok{(}\FloatTok{0.5}\NormalTok{, nsims }\SpecialCharTok{+} \FloatTok{0.5}\NormalTok{),}
       \AttributeTok{xlab =} \StringTok{"Intervalo de Confiança 95\% da média (BMI)"}\NormalTok{,}
       \AttributeTok{ylab =} \StringTok{""}\NormalTok{, }\AttributeTok{yaxt =} \StringTok{"n"}\NormalTok{,}
       \AttributeTok{main =} \FunctionTok{sprintf}\NormalTok{(}\StringTok{"20 IC95\%\% para média do BMI (NHANES) — cobertura: \%.1f\%\%"}\NormalTok{, coverage }\SpecialCharTok{*} \DecValTok{100}\NormalTok{))}
  \FunctionTok{axis}\NormalTok{(}\DecValTok{2}\NormalTok{, }\AttributeTok{at =} \FunctionTok{seq\_len}\NormalTok{(nsims), }\AttributeTok{labels =} \FunctionTok{seq\_len}\NormalTok{(nsims), }\AttributeTok{las =} \DecValTok{1}\NormalTok{, }\AttributeTok{cex.axis =} \FloatTok{0.8}\NormalTok{)}
  \ControlFlowTok{for}\NormalTok{ (i }\ControlFlowTok{in} \FunctionTok{seq\_len}\NormalTok{(nsims)) \{}
\NormalTok{    col }\OtherTok{\textless{}{-}} \ControlFlowTok{if}\NormalTok{ (res}\SpecialCharTok{$}\NormalTok{contains[i]) }\StringTok{"steelblue"} \ControlFlowTok{else} \StringTok{"tomato"}
    \FunctionTok{segments}\NormalTok{(}\AttributeTok{x0 =}\NormalTok{ res}\SpecialCharTok{$}\NormalTok{lower[i], }\AttributeTok{y0 =}\NormalTok{ i, }\AttributeTok{x1 =}\NormalTok{ res}\SpecialCharTok{$}\NormalTok{upper[i], }\AttributeTok{y1 =}\NormalTok{ i, }\AttributeTok{col =}\NormalTok{ col, }\AttributeTok{lwd =} \DecValTok{2}\NormalTok{)}
    \FunctionTok{points}\NormalTok{(res}\SpecialCharTok{$}\NormalTok{mean[i], i, }\AttributeTok{pch =} \DecValTok{16}\NormalTok{, }\AttributeTok{col =}\NormalTok{ col)}
\NormalTok{  \}}
  \FunctionTok{abline}\NormalTok{(}\AttributeTok{v =}\NormalTok{ pop\_mean, }\AttributeTok{col =} \StringTok{"darkgreen"}\NormalTok{, }\AttributeTok{lwd =} \DecValTok{2}\NormalTok{)}
  \FunctionTok{legend}\NormalTok{(}\StringTok{"topright"}\NormalTok{, }\AttributeTok{legend =} \FunctionTok{c}\NormalTok{(}\StringTok{"Contém média pop."}\NormalTok{, }\StringTok{"Não contém"}\NormalTok{, }\StringTok{"Média pop."}\NormalTok{),}
         \AttributeTok{col =} \FunctionTok{c}\NormalTok{(}\StringTok{"steelblue"}\NormalTok{, }\StringTok{"tomato"}\NormalTok{, }\StringTok{"darkgreen"}\NormalTok{), }\AttributeTok{pch =} \FunctionTok{c}\NormalTok{(}\DecValTok{16}\NormalTok{, }\DecValTok{16}\NormalTok{, }\ConstantTok{NA}\NormalTok{), }\AttributeTok{lty =} \FunctionTok{c}\NormalTok{(}\ConstantTok{NA}\NormalTok{, }\ConstantTok{NA}\NormalTok{, }\DecValTok{1}\NormalTok{), }\AttributeTok{lwd =} \FunctionTok{c}\NormalTok{(}\ConstantTok{NA}\NormalTok{, }\ConstantTok{NA}\NormalTok{, }\DecValTok{2}\NormalTok{),}
         \AttributeTok{bty =} \StringTok{"n"}\NormalTok{)}

  \CommentTok{\# Salvar CSV opcional}
  \ControlFlowTok{if}\NormalTok{ (save\_csv) \{}
    \FunctionTok{tryCatch}\NormalTok{(\{}
      \FunctionTok{write.csv}\NormalTok{(res, }\AttributeTok{file =}\NormalTok{ out\_file, }\AttributeTok{row.names =} \ConstantTok{FALSE}\NormalTok{)}
      \ControlFlowTok{if}\NormalTok{ (}\SpecialCharTok{!}\NormalTok{quiet) }\FunctionTok{message}\NormalTok{(}\FunctionTok{sprintf}\NormalTok{(}\StringTok{"Resultados salvos em: \%s"}\NormalTok{, }\FunctionTok{normalizePath}\NormalTok{(out\_file, }\AttributeTok{mustWork =} \ConstantTok{FALSE}\NormalTok{)))}
\NormalTok{    \}, }\AttributeTok{error =} \ControlFlowTok{function}\NormalTok{(e) }\FunctionTok{warning}\NormalTok{(}\StringTok{"Falha ao salvar CSV: "}\NormalTok{, e}\SpecialCharTok{$}\NormalTok{message))}
\NormalTok{  \}}

  \FunctionTok{invisible}\NormalTok{(}\FunctionTok{list}\NormalTok{(}\AttributeTok{results =}\NormalTok{ res, }\AttributeTok{pop\_mean =}\NormalTok{ pop\_mean, }\AttributeTok{coverage =}\NormalTok{ coverage))}
\NormalTok{\}}

\FunctionTok{simular\_20\_ic\_bmi\_nhanes}\NormalTok{()}

\FunctionTok{simular\_20\_ic\_bmi\_nhanes}\NormalTok{(}\AttributeTok{nsims =} \DecValTok{20}\NormalTok{, }\AttributeTok{n =} \DecValTok{50}\NormalTok{, }\AttributeTok{seed =} \DecValTok{123}\NormalTok{, }\AttributeTok{replace =} \ConstantTok{TRUE}\NormalTok{, }\AttributeTok{save\_csv =} \ConstantTok{FALSE}\NormalTok{, }\AttributeTok{out\_file =} \StringTok{"ic20\_bmi.csv"}\NormalTok{)}
\InformationTok{\textasciigrave{}\textasciigrave{}\textasciigrave{}}
\end{Highlighting}
\end{Shaded}

\pandocbounded{\includegraphics[keepaspectratio]{cap16-moore-IC-o-basico_files/figure-pdf/unnamed-chunk-7-1.pdf}}

mmm

\bookmarksetup{startatroot}

\chapter{AEI - cap 17 moore - TSH0}\label{sec-TSHN-basico}

\section{Objetivos da Aprendizagem}\label{objetivos-da-aprendizagem-6}

\begin{quote}
Após ler este capítulo, você deve ser capaz de:

▶ 17.1 Usar o \textbf{raciocínio} dos \textbf{testes estatísticos} para
estabelecer \textbf{se} os \ul{\textbf{dados amostrais suportam}},
\ul{\textbf{ou não}}, \ul{\textbf{uma afirmativa}} sobre a
\ul{\textbf{população}}.

▶ 17.2 Estabelecer as \ul{\textbf{hipóteses nula e alternativa}} ao
\textbf{\emph{testar uma afirmativa sobre a \ul{média}}} de uma
\textbf{população}.

▶ 17.3 Encontrar e \ul{\textbf{interpretar valores P}} e estabelecer
\textbf{se} um \textbf{resultado de teste é, ou não, estatisticamente
significante} em dado nível.

▶ 17.4 Calcular a \textbf{estatística} de \textbf{teste z} de
\textbf{uma amostra}, para \textbf{testes} tanto \textbf{unilaterais}
quanto \textbf{bilaterais}, de \ul{\textbf{uma média populacional}}, e
tirar \ul{\textbf{conclusões}} a partir dos resultados.

▶ 17.5 Usar uma \textbf{tabela} para procurar \textbf{valores P}
aproximados com \textbf{base} na \textbf{estatística z} e estabelecer
\ul{\textbf{se}} o resultado \ul{\textbf{é estatisticamente
significante}}.
\end{quote}

\section{Testes de Significância: o
Básico}\label{testes-de-significuxe2ncia-o-buxe1sico}

\begin{quote}
\textbf{Intervalos de confiança} são um dos dois tipos mais comuns de
\textbf{inferência estatística}.

Neste capítulo, discutimos \textbf{testes de significância} {[}da
Hipótese Nula - TSHN{]}, o \textbf{segundo tipo} de \textbf{inferência
estatística}.

A matemática da \textbf{probabilidade} -- em particular, as
\textbf{distribuições amostrais} discutidas no \textbf{Capítulo 15} --
fornece a \textbf{base formal} para \ul{\textbf{um teste de
significância}}.

Aqui aplicaremos o \textbf{raciocínio} de \textbf{testes de
significância} para a \textbf{média} de \textbf{uma população} que tem
\ul{\textbf{distribuição Normal}}, em \emph{um contexto simples e
artificial} (em que \textbf{\emph{supomos conhecer}} o
\ul{\textbf{desvio-padrão populacional}}).

Usaremos a mesma lógica em \emph{capítulos futuros} para a
\emph{construção de testes de significância} para \emph{parâmetros
populacionais} em \ul{\emph{contextos mais realistas}}.

Use um \textbf{intervalo de confiança} quando seu \textbf{objetivo} for
\ul{\textbf{estimar um parâmetro da população}}.

Os \ul{\textbf{testes de significância}} têm um \textbf{objetivo
diferente}: \ul{\textbf{avaliar}} a \ul{\textbf{evidência}}
\emph{fornecida pelos \textbf{dados}} sobre \ul{\textbf{\emph{alguma
afirmativa anterior}}} relativa a \ul{\textbf{um parâmetro da
população}}.

A seguir, apresentamos sucintamente a \ul{\textbf{lógica}} de testes
estatísticos {[}TSHN ou \textbf{\emph{NHST}} - \textbf{\emph{Null
Hypothese Significant Test}}{]} (MOORE; NOTZ; FLIGNER, 2023 , cap. 17,
p.~308-323)
\end{quote}

\begin{tcolorbox}[enhanced jigsaw, arc=.35mm, opacitybacktitle=0.6, colframe=quarto-callout-important-color-frame, titlerule=0mm, leftrule=.75mm, left=2mm, colbacktitle=quarto-callout-important-color!10!white, breakable, toprule=.15mm, bottomtitle=1mm, opacityback=0, coltitle=black, title=\textcolor{quarto-callout-important-color}{\faExclamation}\hspace{0.5em}{Inferência estatística}, rightrule=.15mm, bottomrule=.15mm, toptitle=1mm, colback=white]

A inferência estatística fornece \ul{\textbf{métodos}} para a
\textbf{\emph{extração de conclusões sobre uma população a partir de
dados amostrais}}. (MOORE; NOTZ; FLIGNER, 2023 , cap. 16, p.~296)

\end{tcolorbox}

Recapitular as condições simples ou pressupostos para um TSHN de uma
média populacional desconhecida.

\begin{tcolorbox}[enhanced jigsaw, arc=.35mm, opacitybacktitle=0.6, colframe=quarto-callout-important-color-frame, titlerule=0mm, leftrule=.75mm, left=2mm, colbacktitle=quarto-callout-important-color!10!white, breakable, toprule=.15mm, bottomtitle=1mm, opacityback=0, coltitle=black, title=\textcolor{quarto-callout-important-color}{\faExclamation}\hspace{0.5em}{Condições simples para inferência sobre uma média}, rightrule=.15mm, bottomrule=.15mm, toptitle=1mm, colback=white]

\begin{enumerate}
\def\labelenumi{\arabic{enumi}.}
\tightlist
\item
  Temos \ul{\textbf{uma amostra aleatória simples}} \textbf{(AAS)} da
  população de interesse.
\item
  Não há não resposta {[}NA{]} ou qualquer outra dificuldade prática.
\item
  A \ul{\textbf{população é grande}} em comparação ao tamanho da amostra
  {[}N \textgreater{} 20 x n{]}.
\item
  A \textbf{variável} que \emph{medimos} tem \ul{\textbf{uma
  distribuição exatamente normal}} \textbf{N(µ; σ)} na população.
\item
  \textbf{\emph{Não conhecemos}} a \textbf{média da população} µ.
\item
  Mas \ul{\textbf{conhecemos}} o \ul{\textbf{desvio-padrão
  populacional}} σ. (MOORE; NOTZ; FLIGNER, 2023 , cap. 16, p.~297)
\end{enumerate}

\end{tcolorbox}

A condição de que a população seja grande em relação ao tamanho da
amostra será adequadamente satisfeita se a população for, digamos, pelo
menos 20 vezes maior {[}N \textgreater{} 20 x n ou n \textless{} 5\% x
N{]}.

\begin{tcolorbox}[enhanced jigsaw, arc=.35mm, opacitybacktitle=0.6, colframe=quarto-callout-important-color-frame, titlerule=0mm, leftrule=.75mm, left=2mm, colbacktitle=quarto-callout-important-color!10!white, breakable, toprule=.15mm, bottomtitle=1mm, opacityback=0, coltitle=black, title=\textcolor{quarto-callout-important-color}{\faExclamation}\hspace{0.5em}{Important}, rightrule=.15mm, bottomrule=.15mm, toptitle=1mm, colback=white]

{[}\ldots{]} \emph{As \textbf{condições} {[}ou pressupostos,
\textbf{presunções relativas}{]} de que temos uma AAS perfeita, de que a
população é exatamente Normal e de que conhecemos o σ populacional são
todas não realistas}. (MOORE; NOTZ; FLIGNER, 2023 , cap. 16, p.~297)

\end{tcolorbox}

O \textbf{pesquisador} é o \textbf{responsável} pelo \textbf{ônus da
prova} de \ul{\textbf{verificar se}}, na prática e diante de seus dados
válidos, fidedignos e reproduzíveis, essas \ul{\textbf{\emph{presunções
relativas}}} \ul{\textbf{\emph{não}}} \textbf{\emph{são
\ul{satisfeitas}}}.

Ou seja, de que existe evidência de não satisfação das condições para
aplicação de um TSHN.

Por exemplo:

\begin{enumerate}
\def\labelenumi{\arabic{enumi}.}
\tightlist
\item
  a distribuição original na População da variável a ser testada não é
  Normal (padrão geral dos dados).
\item
  a amostra obtida não é uma AAS.
\item
  há considerável viés de NA.
\item
  há viés de subcobertura.
\item
  há viés de autoseleção.
\item
  há viés de resposta por autodeclaração.
\item
  há viés de resposta devido ao fraseado de questões do survey
  (intrumento de coleta).
\item
  pode haver insinceridade na resposta do entrevistado.
\item
  há \emph{outliers} na amostra com forte desvio de assimetria à direita
  que fogem ao padrão geral dos dados.
\item
  uso de \emph{estimador enviesado} para obter uma estimativa do
  \textbf{desvio padrão} da \textbf{população} (\(\sigma\)) a partir do
  \textbf{\emph{desvio padrão}} da \textbf{\emph{amostra}} (s), o que é
  evitado pela divisão da soma dos desvios quadráticos pelo número de
  graus de liberdade da amostra (g.l. = n - 1) (MOORE; NOTZ; FLIGNER,
  2023 , cap. 2, p.~46).
\end{enumerate}

Recapitulando o conceito de desvio padrão amostral (s) como a raiz
quadrada da \textbf{variâcia amostral} (s\textsuperscript{2}):

\[
s^2 = \frac{(x_1-\bar{x})^2 + (x_2-\bar{x})^2 + \cdots + (x_n-\bar{x})^2}{n-1}
\]

Ou seja, a variância amostral também pode ser expressa por:

\[
s^2 = \frac{1}{n-1} \sum_{i=1}^n (x_i-\bar{x})^2
\]

Uma vez calculada a variância amostral acima, o desvio padrão amostral é
obtido extraindo sua raiz quadrada (MOORE; NOTZ; FLIGNER, 2023 , cap. 2,
p.~46):

\[
s^2 = \sqrt{\frac{1}{n-1} \sum_{i=1}^n (x_i-\bar{x})^2}
\]

\subsection{EXEMPLO 17.1 Eu sou um grande atirador de lances
livres}\label{exemplo-17.1-eu-sou-um-grande-atirador-de-lances-livres}

Eu \textbf{afirmo} que acerto 80\% de meus lances livres no jogo de
basquete.

Para \ul{\textbf{testar}} \emph{minha} \textbf{afirmativa}, \emph{você
me pede para} \textbf{fazer 20 lances livres}. {[}coleta de uma amostra
de tamanho 20{]}

Eu \textbf{\emph{acerto apenas oito dos 20}}.

``Ah!'', você diz. ``\emph{Alguém que acerta 80\% de seus lances livres}
\ul{\textbf{quase nunca}} \emph{acertaria \ul{\textbf{apenas oito entre
20}}}.

\ul{\textbf{Logo}}, \ul{\textbf{não acredito}} em \ul{\textbf{sua
afirmativa}}.'' {[}\emph{decisão} baseada em \emph{evidências} após um
\emph{teste} aplicado em \emph{dados} de \emph{uma amostra}
validademente coletada{]}

Seu \ul{\textbf{raciocínio}} se baseia no \textbf{questionamento} do
\ul{\textbf{\emph{que ocorreria se minha afirmativa fosse verdadeira}}}
\ul{\textbf{e}} \ul{\textbf{\emph{repetíssemos a amostra de 20
lançamentos muitas vezes}}}: eu \ul{\textbf{quase nunca acertaria oito
ou menos}}.

{[}Lei dos Grandes Números e Teorema Central do Limite:{]}

Esse \ul{\textbf{resultado de oito em 20 é tão improvável}}, que fornece
uma \ul{forte} {[}recitus razoável{]} \ul{\textbf{evidência}} de que
minha \textbf{afirmativa \ul{não} seja \ul{verdadeira}} {[}sem jamais
afastar a possibilidade de cometer um Erro de decisão do Tipo I, ou
seja, o erro de decisão de \emph{rejeitar a hipotese nula quando ela for
verdadeira}{]}.

Você pode dizer quão \ul{forte} é a evidência contra minha afirmativa,
fornecendo a \ul{\textbf{probabilidade}} de eu acertar oito ou menos
entre 20 lances livres, se eu realmente acertasse 80\% no longo prazo.

Essa probabilidade é 0,0001; como descrito no Capítulo 14, esse cálculo
é feito com o uso da distribuição binomial.

Assim, eu \ul{\textbf{acertaria oito ou menos em 20 lances}} em
\ul{\textbf{apenas}} \ul{\textbf{uma vez em 10 mil tentativas no longo
prazo}} -- onde \textbf{cada ``tentativa'' são 20 lances livres} jogados
-- \ul{\textbf{se}} minha \ul{\textbf{afirmativa de acertar 80\% fosse
verdadeira}}.

O \ul{\textbf{pequeno valor da probabilidade}} o \ul{\textbf{convence}}
de que minha \textbf{afirmativa é \ul{falsa}}.

Corretíssimo, mas esse indicador \textbf{\emph{probabilidade}},
\ul{\textbf{valor P}} ou \textbf{\emph{Área sob a Curva}}
\ul{\textbf{não}} nos fornece a \textbf{força} dessa
\ul{\textbf{evidência}}.

O \ul{\textbf{coeficiente de Bayes}} cumpre esse papel.

\section{\texorpdfstring{A lógica dos Testes de Significância da
H\textsubscript{0}}{A lógica dos Testes de Significância da H0}}\label{a-luxf3gica-dos-testes-de-significuxe2ncia-da-h0}

A \ul{\textbf{lógica}} dos \ul{\textbf{testes estatísticos}}, assim como
a dos \ul{\textbf{intervalos de confiança}}, se \ul{\textbf{baseia}} no
\ul{\textbf{questionamento}} do que \ul{\textbf{ocorreria se
repetíssemos a amostra ou experimento muitas vezes}}.

Essa lógica baseia-se na \ul{\textbf{Lei dos Grandes Números}} (LGN).

Agiremos novamente \ul{\textbf{\emph{como se}}} as
``\ul{\textbf{condições simples}}'' listadas em ``Condições simples para
inferência sobre uma média'', no Capítulo 16, \textbf{\emph{fossem
verdadeiras}}: temos \textbf{\emph{uma AAS perfeita}} de
\textbf{\emph{uma população exatamente Normal}} com
\textbf{\emph{desvio-padrão σ conhecido}} por nós.

Eis um exemplo que analisaremos.

\subsection{EXEMPLO 17.2 Adoçantes de
refrigerantes}\label{exemplo-17.2-adouxe7antes-de-refrigerantes}

Refrigerantes dietéticos usam adoçantes artificiais para evitar o uso de
açúcar.

Esses adoçantes gradualmente perdem sua doçura ao longo do tempo.

Os fabricantes, portanto, testam a perda de doçura dos refrigerantes
novos antes de colocá-los no mercado.

Provadores treinados bebem um pequeno gole de refrigerante, juntamente
com bebidas de doçura padrão, e atribuem ao refrigerante um ``escore de
doçura'' de 1 a 10, com maiores escores correspondendo a maior doçura.

O refrigerante é, então, \textbf{\emph{armazenado por um mês em alta
temperatura para imitar o efeito do armazenamento por 4 meses em
temperatura ambiente}}.

\textbf{\emph{Cada provador atribui um escore ao refrigerante novamente
após o armazenamento}}.

Esse é um experimento de \textbf{\emph{dados emparelhados}}.

Nossos dados são as \ul{\textbf{diferenças}} (escore
\ul{\textbf{\emph{antes}}} do armazenamento \ul{\textbf{menos}} escore
\ul{\textbf{\emph{após}}} o armazenamento) dos escores dos provadores.
{[}Perda doçura = Antes - Após; se \textgreater{} 0 então menos
doçura{]}

Quanto \ul{\textbf{maior a diferença}} (diferença \textgreater{} 0),
\textbf{\emph{maior será a perda de doçura}}.

\textbf{\emph{Suponha sabermos}} que, \textbf{\emph{para qualquer
refrigerante}}, os \textbf{\emph{escores de perda de doçura variem de
provador para provador}} de acordo com uma \ul{\textbf{distribuição
Normal}}, com \ul{\textbf{desvio-padrão}} σ = 1.

A média µ de todos os provadores mede a perda de doçura e é
\textbf{\emph{diferente para diferentes refrigerantes}}.

A seguir, estão as perdas de doçura de um novo refrigerante, medidas por
10 provadores treinados:

1,6   0,4   0,5 --2,0  1,5 --1,1  1,3 --0,1 --0,3   1,2

A \ul{\textbf{perda média de doçura}} é \ul{\textbf{dada}} pela
\textbf{\emph{média amostral}} x = 0,3, de modo que, \textbf{\emph{em
média}}, os \textbf{\emph{10 provadores encontraram uma pequena perda de
doçura}}.

Também, \emph{mais da metade}, (seis) dos \emph{provadores encontraram
uma perda de doçura}.

Esses dados são uma \ul{\textbf{boa evidência}} de que o refrigerante
perdeu doçura com o armazenamento?

O raciocínio é o mesmo do Exemplo 17.1.

Fazemos \ul{\textbf{uma afirmativa}} e perguntamos \ul{\textbf{se}} os
\ul{\textbf{dados fornecem evidência contrária a ela}}.

Procuramos \ul{\textbf{evidência}} de que \ul{\textbf{haja uma perda de
doçura}};

logo, a \ul{\textbf{afirmativa que testamos}} é que \ul{\textbf{não há
perda}}.

Nesse caso, a \textbf{perda média para a população de todos os
provadores treinados} seria µ = 0. {[}H\textsubscript{0}{]}

• \ul{\textbf{Se}} a afirmativa de que \textbf{µ = 0}
{[}H\textsubscript{0}{]} é \textbf{verdadeira}, a
\ul{\textbf{distribuição amostral}} de \(\bar{x}\) dos \ul{\textbf{10
provadores}} é \ul{\textbf{Normal}} com \textbf{média µ = 0} e
desvio-padrão {[}pelo Teorema Central do Limite{]}:

\[
\text{Erro Padrão das médias amostrais} = \frac{\sigma}{\sqrt{n}} = \frac{1.0}{\sqrt{10}} = 0.316
\]

Esses são exatamente os cálculos que fizemos no Capítulo 15 (ver Exemplo
15.5) e no Capítulo 16 (ver Exemplo 16.1).

A Figura 17.1 mostra essa distribuição amostral.

\textbf{\emph{Podemos}} \ul{\textbf{julgar}} \ul{\textbf{se}} qualquer
\(\bar{x}\) \textbf{\emph{observado}} {[}nos dados amostrais{]} é
\ul{\textbf{surpreendente}}, localizando-o nessa distribuição.

• Para esse refrigerante, \textbf{\emph{10 provadores}} acusaram
\textbf{\emph{perda média}} {[}doçura{]} \(\bar{x}\) = 0,3. É claro, a
partir da Figura 17.1, que um \(\bar{x}\) desse tamanho \ul{\textbf{não
é particularmente surpreendente}}. Ele \ul{\textbf{poderia facilmente
ocorrer apenas devido}} ao \ul{\textbf{acaso}}, \textbf{quando a média
da população é µ = 0 {[}supondo H\textsubscript{0} verdadeira{]}}. O
\ul{\textbf{fato}} de \textbf{\emph{obter}} \(\bar{x}\) = 0,3 para
\ul{\textbf{10}} provadores \ul{\textbf{não é}} \ul{forte}
{[}\emph{rectius}{]} \ul{\textbf{boa evidência}} de que esse
refrigerante perca doçura.

Script abaixo gera um gráfico da \textbf{\emph{distribuição amostral das
médias amostrais}} supondo que a Hipótese Nula fosse Verdadeira:
\(H_0: \mu = 0.0\)

Trata-se de um TSHN.

\begin{Shaded}
\begin{Highlighting}[numbers=left,,]
\InformationTok{\textasciigrave{}\textasciigrave{}\textasciigrave{}\{r\}}
\CommentTok{\# Gera gráfico da distribuição amostral das médias (nrep = 5000) para um vetor de observações.}
\CommentTok{\# Ajuste \textquotesingle{}sample\_size\textquotesingle{} conforme desejar; usa amostragem com reposição.}
\CommentTok{\# Requisitos: install.packages(c("ggplot2","dplyr"))}
\FunctionTok{library}\NormalTok{(ggplot2)}
\FunctionTok{library}\NormalTok{(dplyr)}

\CommentTok{\# Dados amostrais fornecidos}
\NormalTok{x }\OtherTok{\textless{}{-}} \FunctionTok{c}\NormalTok{(}\FloatTok{1.6}\NormalTok{, }\FloatTok{0.4}\NormalTok{, }\FloatTok{0.5}\NormalTok{, }\SpecialCharTok{{-}}\FloatTok{2.0}\NormalTok{, }\FloatTok{1.5}\NormalTok{, }\SpecialCharTok{{-}}\FloatTok{1.1}\NormalTok{, }\FloatTok{1.3}\NormalTok{, }\SpecialCharTok{{-}}\FloatTok{0.1}\NormalTok{, }\SpecialCharTok{{-}}\FloatTok{0.3}\NormalTok{, }\FloatTok{1.2}\NormalTok{)}

\CommentTok{\# Parâmetros}
\NormalTok{sample\_size }\OtherTok{\textless{}{-}} \DecValTok{10}   \CommentTok{\# tamanho da amostra n (ajuste conforme necessário)}
\NormalTok{nrep }\OtherTok{\textless{}{-}} \DecValTok{5000}        \CommentTok{\# número de repetições}
\NormalTok{replace }\OtherTok{\textless{}{-}} \ConstantTok{TRUE}     \CommentTok{\# TRUE = com reposição (bootstrap{-}like); FALSE = sem reposição}

\CommentTok{\# Geração das médias amostrais}
\FunctionTok{set.seed}\NormalTok{(}\DecValTok{123}\NormalTok{) }\CommentTok{\# para reprodutibilidade}
\NormalTok{samp\_means }\OtherTok{\textless{}{-}} \FunctionTok{replicate}\NormalTok{(nrep,}
                        \FunctionTok{mean}\NormalTok{(}\FunctionTok{sample}\NormalTok{(x,}
                                    \AttributeTok{size =}\NormalTok{ sample\_size,}
                                    \AttributeTok{replace =}\NormalTok{ replace)}
\NormalTok{                             )}
\NormalTok{                        )}

\CommentTok{\# Estatísticas}
\NormalTok{obs\_mean }\OtherTok{\textless{}{-}} \FunctionTok{mean}\NormalTok{(x)}
\NormalTok{pop\_sd   }\OtherTok{\textless{}{-}} \FunctionTok{sd}\NormalTok{(x)                       }\CommentTok{\# desvio amostral da "população" dada}
\NormalTok{theoretical\_sd }\OtherTok{\textless{}{-}}\NormalTok{ pop\_sd }\SpecialCharTok{/} \FunctionTok{sqrt}\NormalTok{(sample\_size)}
\NormalTok{dist\_mean }\OtherTok{\textless{}{-}} \FunctionTok{mean}\NormalTok{(samp\_means)}
\NormalTok{dist\_sd   }\OtherTok{\textless{}{-}} \FunctionTok{sd}\NormalTok{(samp\_means)}
\NormalTok{ci95 }\OtherTok{\textless{}{-}}\NormalTok{ dist\_mean }\SpecialCharTok{+} \FunctionTok{c}\NormalTok{(}\SpecialCharTok{{-}}\DecValTok{1}\NormalTok{, }\DecValTok{1}\NormalTok{) }\SpecialCharTok{*} \FunctionTok{qnorm}\NormalTok{(}\FloatTok{0.975}\NormalTok{) }\SpecialCharTok{*}\NormalTok{ theoretical\_sd}

\CommentTok{\# Data frame para ggplot}
\NormalTok{df\_plot }\OtherTok{\textless{}{-}} \FunctionTok{tibble}\NormalTok{(}\AttributeTok{mean =}\NormalTok{ samp\_means)}

\CommentTok{\# Plotagem: histograma (densidade), densidade empírica e curva normal teórica}
\NormalTok{p }\OtherTok{\textless{}{-}} \FunctionTok{ggplot}\NormalTok{(df\_plot, }\FunctionTok{aes}\NormalTok{(}\AttributeTok{x =}\NormalTok{ mean)) }\SpecialCharTok{+}
  \FunctionTok{geom\_histogram}\NormalTok{(}\FunctionTok{aes}\NormalTok{(}\AttributeTok{y =}\NormalTok{ ..density..), }\AttributeTok{bins =} \FunctionTok{max}\NormalTok{(}\DecValTok{10}\NormalTok{, }\FunctionTok{round}\NormalTok{(}\FunctionTok{sqrt}\NormalTok{(nrep))),}
                 \AttributeTok{fill =} \StringTok{"\#cce5ff"}\NormalTok{, }\AttributeTok{color =} \StringTok{"\#2b6fa6"}\NormalTok{) }\SpecialCharTok{+}
  \FunctionTok{geom\_density}\NormalTok{(}\AttributeTok{color =} \StringTok{"\#0055a4"}\NormalTok{, }\AttributeTok{size =} \DecValTok{1}\NormalTok{) }\SpecialCharTok{+}
  \FunctionTok{stat\_function}\NormalTok{(}\AttributeTok{fun =} \ControlFlowTok{function}\NormalTok{(x) }\FunctionTok{dnorm}\NormalTok{(x, }\AttributeTok{mean =}\NormalTok{ dist\_mean, }\AttributeTok{sd =}\NormalTok{ theoretical\_sd),}
                \AttributeTok{color =} \StringTok{"\#d9534f"}\NormalTok{, }\AttributeTok{size =} \DecValTok{1}\NormalTok{, }\AttributeTok{linetype =} \StringTok{"dashed"}\NormalTok{) }\SpecialCharTok{+}
  \FunctionTok{geom\_vline}\NormalTok{(}\AttributeTok{xintercept =}\NormalTok{ dist\_mean, }\AttributeTok{color =} \StringTok{"darkgreen"}\NormalTok{, }\AttributeTok{size =} \DecValTok{1}\NormalTok{, }\AttributeTok{linetype =} \StringTok{"solid"}\NormalTok{) }\SpecialCharTok{+}
  \FunctionTok{geom\_vline}\NormalTok{(}\AttributeTok{xintercept =} \FloatTok{0.0}\NormalTok{, }\AttributeTok{color =} \StringTok{"purple"}\NormalTok{, }\AttributeTok{size =} \DecValTok{1}\NormalTok{, }\AttributeTok{linetype =} \StringTok{"dotdash"}\NormalTok{) }\SpecialCharTok{+}
  \FunctionTok{geom\_vline}\NormalTok{(}\AttributeTok{xintercept =}\NormalTok{ ci95, }\AttributeTok{color =} \StringTok{"\#d9534f"}\NormalTok{, }\AttributeTok{linetype =} \StringTok{"dotted"}\NormalTok{, }\AttributeTok{size =} \FloatTok{0.8}\NormalTok{) }\SpecialCharTok{+}
  \FunctionTok{annotate}\NormalTok{(}\StringTok{"text"}\NormalTok{, }\AttributeTok{x =}\NormalTok{ dist\_mean, }\AttributeTok{y =} \FunctionTok{max}\NormalTok{(}\FunctionTok{density}\NormalTok{(samp\_means)}\SpecialCharTok{$}\NormalTok{y)}\SpecialCharTok{*}\FloatTok{0.95}\NormalTok{,}
           \AttributeTok{label =} \FunctionTok{sprintf}\NormalTok{(}\StringTok{"Média das médias = \%.3f"}\NormalTok{, dist\_mean), }\AttributeTok{vjust =} \SpecialCharTok{{-}}\FloatTok{0.5}\NormalTok{, }\AttributeTok{color =} \StringTok{"darkgreen"}\NormalTok{, }\AttributeTok{size =} \FloatTok{3.5}\NormalTok{) }\SpecialCharTok{+}
  \FunctionTok{annotate}\NormalTok{(}\StringTok{"text"}\NormalTok{, }\AttributeTok{x =}\NormalTok{ ci95[}\DecValTok{1}\NormalTok{], }\AttributeTok{y =} \FunctionTok{max}\NormalTok{(}\FunctionTok{density}\NormalTok{(samp\_means)}\SpecialCharTok{$}\NormalTok{y)}\SpecialCharTok{*}\FloatTok{0.75}\NormalTok{,}
           \AttributeTok{label =} \FunctionTok{sprintf}\NormalTok{(}\StringTok{"IC95 teórico:[\%.3f, \%.3f]"}\NormalTok{, ci95[}\DecValTok{1}\NormalTok{], ci95[}\DecValTok{2}\NormalTok{]), }\AttributeTok{color =} \StringTok{"\#d9534f"}\NormalTok{, }\AttributeTok{hjust =} \DecValTok{0}\NormalTok{, }\AttributeTok{size =} \DecValTok{3}\NormalTok{) }\SpecialCharTok{+}
  \FunctionTok{labs}\NormalTok{(}\AttributeTok{title =} \FunctionTok{sprintf}\NormalTok{(}\StringTok{"Distribuição amostral das médias amostrais (n = \%d, nrep = \%d)"}\NormalTok{, sample\_size, nrep),}
       \AttributeTok{x =} \StringTok{"Média amostral"}\NormalTok{, }\AttributeTok{y =} \StringTok{"Densidade"}\NormalTok{) }\SpecialCharTok{+}
  \FunctionTok{theme\_minimal}\NormalTok{(}\AttributeTok{base\_size =} \DecValTok{12}\NormalTok{)}

\CommentTok{\# Exibir}
\FunctionTok{print}\NormalTok{(p)}

\CommentTok{\# Opcional: salvar}
\CommentTok{\# ggsave("dist\_media\_amostral.png", p, width = 8, height = 4.5, dpi = 300)}
\InformationTok{\textasciigrave{}\textasciigrave{}\textasciigrave{}}
\end{Highlighting}
\end{Shaded}

\pandocbounded{\includegraphics[keepaspectratio]{cap17-moore-TSHN-o-basico_files/figure-pdf/unnamed-chunk-1-1.pdf}}

Agora centrando a distribuição amostral das médias amostrais em mu = 0 e
mantendo a reta vertical em xbarra = 0.301

\begin{Shaded}
\begin{Highlighting}[numbers=left,,]
\InformationTok{\textasciigrave{}\textasciigrave{}\textasciigrave{}\{r\}}
\CommentTok{\# Gera distribuição amostral das médias, centra em mu = 0 e mantém reta vertical em xbarra = 0.301}
\CommentTok{\# Requisitos: install.packages(c("ggplot2","dplyr"))}
\FunctionTok{library}\NormalTok{(ggplot2)}
\FunctionTok{library}\NormalTok{(dplyr)}

\CommentTok{\# Dados amostrais fornecidos}
\NormalTok{x }\OtherTok{\textless{}{-}} \FunctionTok{c}\NormalTok{(}\FloatTok{1.6}\NormalTok{, }\FloatTok{0.4}\NormalTok{, }\FloatTok{0.5}\NormalTok{, }\SpecialCharTok{{-}}\FloatTok{2.0}\NormalTok{, }\FloatTok{1.5}\NormalTok{, }\SpecialCharTok{{-}}\FloatTok{1.1}\NormalTok{, }\FloatTok{1.3}\NormalTok{, }\SpecialCharTok{{-}}\FloatTok{0.1}\NormalTok{, }\SpecialCharTok{{-}}\FloatTok{0.3}\NormalTok{, }\FloatTok{1.2}\NormalTok{)}



\CommentTok{\# Parâmetros}
\NormalTok{sample\_size }\OtherTok{\textless{}{-}} \DecValTok{10}   \CommentTok{\# tamanho da amostra n (ajuste se desejar)}
\NormalTok{nrep }\OtherTok{\textless{}{-}} \DecValTok{5000}        \CommentTok{\# número de repetições}
\NormalTok{replace }\OtherTok{\textless{}{-}} \ConstantTok{TRUE}     \CommentTok{\# amostragem com reposição}
\FunctionTok{set.seed}\NormalTok{(}\DecValTok{123}\NormalTok{)}


\CommentTok{\# Gerar médias amostrais}
\NormalTok{samp\_means }\OtherTok{\textless{}{-}} \FunctionTok{replicate}\NormalTok{(nrep,}
                        \FunctionTok{mean}\NormalTok{(}\FunctionTok{sample}\NormalTok{(x,}
                                    \AttributeTok{size =}\NormalTok{ sample\_size,}
                                    \AttributeTok{replace =}\NormalTok{ replace)}
\NormalTok{                             )}
\NormalTok{                        )}

\CommentTok{\# média das médias (xbarra) e valor fixo solicitado para a reta vertical}
\NormalTok{dist\_mean }\OtherTok{\textless{}{-}} \FunctionTok{mean}\NormalTok{(samp\_means)        }\CommentTok{\# média observada das médias (ex.: \textasciitilde{}0.301)}
\NormalTok{xbar\_fixed }\OtherTok{\textless{}{-}} \FloatTok{0.301}                  \CommentTok{\# reta vertical fixada conforme solicitado}

\CommentTok{\# Centralizar a distribuição em mu = 0 (subtrair a média das médias)}
\NormalTok{centered\_means }\OtherTok{\textless{}{-}}\NormalTok{ samp\_means }\SpecialCharTok{{-}}\NormalTok{ dist\_mean}

\CommentTok{\# Estatísticas}
\NormalTok{pop\_sd }\OtherTok{\textless{}{-}} \FunctionTok{sd}\NormalTok{(x)                       }\CommentTok{\# desvio amostral dos dados fornecidos}
\NormalTok{theoretical\_sd }\OtherTok{\textless{}{-}}\NormalTok{ pop\_sd }\SpecialCharTok{/} \FunctionTok{sqrt}\NormalTok{(sample\_size)  }\CommentTok{\# sd teórico da média amostral}
\NormalTok{empirical\_sd }\OtherTok{\textless{}{-}} \FunctionTok{sd}\NormalTok{(centered\_means)    }\CommentTok{\# sd empírico da distribuição centralizada}
\NormalTok{ci95 }\OtherTok{\textless{}{-}}\NormalTok{ dist\_mean }\SpecialCharTok{{-}}\NormalTok{ xbar\_fixed }\SpecialCharTok{+} \FunctionTok{c}\NormalTok{(}\SpecialCharTok{{-}}\DecValTok{1}\NormalTok{, }\DecValTok{1}\NormalTok{) }\SpecialCharTok{*} \FunctionTok{qnorm}\NormalTok{(}\FloatTok{0.975}\NormalTok{) }\SpecialCharTok{*}\NormalTok{ theoretical\_sd}

\CommentTok{\# Data frame para ggplot}
\NormalTok{df\_plot }\OtherTok{\textless{}{-}} \FunctionTok{tibble}\NormalTok{(}\AttributeTok{mean\_centered =}\NormalTok{ centered\_means)}

\CommentTok{\# Plotagem: histograma (densidade), densidade empírica e curva normal teórica centrada em 0}
\NormalTok{p }\OtherTok{\textless{}{-}} \FunctionTok{ggplot}\NormalTok{(df\_plot, }\FunctionTok{aes}\NormalTok{(}\AttributeTok{x =}\NormalTok{ mean\_centered)) }\SpecialCharTok{+}
  \FunctionTok{geom\_histogram}\NormalTok{(}\FunctionTok{aes}\NormalTok{(}\AttributeTok{y =}\NormalTok{ ..density..), }\AttributeTok{bins =} \FunctionTok{max}\NormalTok{(}\DecValTok{10}\NormalTok{, }\FunctionTok{round}\NormalTok{(}\FunctionTok{sqrt}\NormalTok{(nrep))),}
                 \AttributeTok{fill =} \StringTok{"\#cce5ff"}\NormalTok{, }\AttributeTok{color =} \StringTok{"\#2b6fa6"}\NormalTok{) }\SpecialCharTok{+}
  \FunctionTok{geom\_density}\NormalTok{(}\AttributeTok{color =} \StringTok{"\#0055a4"}\NormalTok{, }\AttributeTok{size =} \DecValTok{1}\NormalTok{) }\SpecialCharTok{+}
  \CommentTok{\# curva normal teórica centrada em mu = 0 (usar theoretical\_sd)}
  \FunctionTok{stat\_function}\NormalTok{(}\AttributeTok{fun =} \ControlFlowTok{function}\NormalTok{(x) }\FunctionTok{dnorm}\NormalTok{(x, }\AttributeTok{mean =} \DecValTok{0}\NormalTok{, }\AttributeTok{sd =}\NormalTok{ theoretical\_sd),}
                \AttributeTok{color =} \StringTok{"\#d9534f"}\NormalTok{, }\AttributeTok{size =} \DecValTok{1}\NormalTok{, }\AttributeTok{linetype =} \StringTok{"dashed"}\NormalTok{) }\SpecialCharTok{+}
  \CommentTok{\# linha vertical no zero (mu alvo)}
  \FunctionTok{geom\_vline}\NormalTok{(}\AttributeTok{xintercept =} \DecValTok{0}\NormalTok{, }\AttributeTok{color =} \StringTok{"darkgreen"}\NormalTok{, }\AttributeTok{linetype =} \StringTok{"solid"}\NormalTok{, }\AttributeTok{size =} \DecValTok{1}\NormalTok{) }\SpecialCharTok{+}
  \CommentTok{\# linha vertical fixa em xbar = 0.301 (na escala centrada, posiciona em x = 0.301 {-} dist\_mean)}
  \FunctionTok{geom\_vline}\NormalTok{(}\AttributeTok{xintercept =}\NormalTok{ xbar\_fixed, }\AttributeTok{color =} \StringTok{"purple"}\NormalTok{, }\AttributeTok{linetype =} \StringTok{"dotdash"}\NormalTok{, }\AttributeTok{size =} \DecValTok{1}\NormalTok{) }\SpecialCharTok{+}
  \FunctionTok{geom\_vline}\NormalTok{(}\AttributeTok{xintercept =}\NormalTok{ ci95, }\AttributeTok{color =} \StringTok{"\#d9534f"}\NormalTok{, }\AttributeTok{linetype =} \StringTok{"dotted"}\NormalTok{, }\AttributeTok{size =} \FloatTok{0.8}\NormalTok{) }\SpecialCharTok{+}
  \CommentTok{\# anotações: média das médias (valor real) e legenda para mu=0}
  \FunctionTok{annotate}\NormalTok{(}\StringTok{"text"}\NormalTok{, }\AttributeTok{x =} \DecValTok{0}\NormalTok{, }\AttributeTok{y =} \FunctionTok{max}\NormalTok{(}\FunctionTok{density}\NormalTok{(centered\_means)}\SpecialCharTok{$}\NormalTok{y) }\SpecialCharTok{*} \FloatTok{0.95}\NormalTok{,}
           \AttributeTok{label =} \StringTok{"mu = 0"}\NormalTok{, }\AttributeTok{color =} \StringTok{"darkgreen"}\NormalTok{, }\AttributeTok{vjust =} \SpecialCharTok{{-}}\FloatTok{0.5}\NormalTok{, }\AttributeTok{size =} \FloatTok{3.5}\NormalTok{) }\SpecialCharTok{+}
  \FunctionTok{annotate}\NormalTok{(}\StringTok{"text"}\NormalTok{, }\AttributeTok{x =}\NormalTok{ xbar\_fixed, }\AttributeTok{y =} \FunctionTok{max}\NormalTok{(}\FunctionTok{density}\NormalTok{(centered\_means)}\SpecialCharTok{$}\NormalTok{y) }\SpecialCharTok{*} \FloatTok{0.85}\NormalTok{,}
           \AttributeTok{label =} \FunctionTok{sprintf}\NormalTok{(}\StringTok{"x̄ = \%.3f"}\NormalTok{, xbar\_fixed), }\AttributeTok{color =} \StringTok{"purple"}\NormalTok{, }\AttributeTok{vjust =} \SpecialCharTok{{-}}\FloatTok{0.5}\NormalTok{, }\AttributeTok{size =} \FloatTok{3.5}\NormalTok{) }\SpecialCharTok{+}
  \FunctionTok{annotate}\NormalTok{(}\StringTok{"text"}\NormalTok{, }\AttributeTok{x =}\NormalTok{ ci95[}\DecValTok{1}\NormalTok{], }\AttributeTok{y =} \FunctionTok{max}\NormalTok{(}\FunctionTok{density}\NormalTok{(samp\_means)}\SpecialCharTok{$}\NormalTok{y)}\SpecialCharTok{*}\FloatTok{0.75}\NormalTok{,}
           \AttributeTok{label =} \FunctionTok{sprintf}\NormalTok{(}\StringTok{"IC95 teórico:[\%.3f, \%.3f]"}\NormalTok{, ci95[}\DecValTok{1}\NormalTok{], ci95[}\DecValTok{2}\NormalTok{]), }\AttributeTok{color =} \StringTok{"\#d9534f"}\NormalTok{, }\AttributeTok{hjust =} \SpecialCharTok{+}\FloatTok{0.5}\NormalTok{, }\AttributeTok{size =} \DecValTok{3}\NormalTok{) }\SpecialCharTok{+}
  \FunctionTok{labs}\NormalTok{(}\AttributeTok{title =} \FunctionTok{sprintf}\NormalTok{(}\StringTok{"Distribuição amostral das médias (centralizada em 0)}\SpecialCharTok{\textbackslash{}n}\StringTok{ n = \%d, rep = \%d"}\NormalTok{, sample\_size, nrep),}
       \AttributeTok{subtitle =} \FunctionTok{sprintf}\NormalTok{(}\StringTok{"Média real das médias = \%.3f (subtraída para centralizar)"}\NormalTok{, dist\_mean),}
       \AttributeTok{x =} \StringTok{"Média amostral (centralizada)"}\NormalTok{, }\AttributeTok{y =} \StringTok{"Densidade"}\NormalTok{) }\SpecialCharTok{+}
  \FunctionTok{theme\_minimal}\NormalTok{(}\AttributeTok{base\_size =} \DecValTok{12}\NormalTok{)}

\FunctionTok{print}\NormalTok{(p)}

\CommentTok{\# Opcional: salvar gráfico}
\CommentTok{\# ggsave("dist\_media\_centered.png", p, width = 8, height = 4.5, dpi = 300)}
\InformationTok{\textasciigrave{}\textasciigrave{}\textasciigrave{}}
\end{Highlighting}
\end{Shaded}

\pandocbounded{\includegraphics[keepaspectratio]{cap17-moore-TSHN-o-basico_files/figure-pdf/unnamed-chunk-2-1.pdf}}

mmm

\bookmarksetup{startatroot}

\chapter{Testes de Significância: o
Básico}\label{testes-de-significuxe2ncia-o-buxe1sico-1}

Capítulo 17 - A Estatística Básica e sua prática (9ª ed.) (MOORE; NOTZ;
FLIGNER, 2023 , cap. 17, p.~296-323)

Intervalos de confiança são um dos dois tipos mais comuns de inferência
estatística. Neste capítulo, discutimos testes de significância, o
segundo tipo de inferência estatística.

A matemática da probabilidade - em particular, as distribuições
amostrais discutidas no Capítulo 15 - fornece a base formal para um
teste de significância.

Aqui aplicaremos o raciocínio de testes de significância para a média de
uma população que tem distribuição Normal, em um contexto simples e
artificial (em que supomos conhecer o desvio-padrão populacional
\(\sigma\)). Usaremos a mesma lógica em capítulos futuros para a
construção e testes de significância para parâmetros populacionais em
contextos mais realistas.

Use um intervalo de confiança quando seu objetivo for estimar um
parâmetro da população. Os testes de significância têm um objetivo
diferente: avaliar a evidência fornecida pelos dados sobre alguma
afirmativa anterior relativa a um parâmetro da população.

A seguir, apresentamos sucintamente a lógica de testes estatísticos.

\begin{tcolorbox}[enhanced jigsaw, arc=.35mm, opacitybacktitle=0.6, colframe=quarto-callout-note-color-frame, titlerule=0mm, leftrule=.75mm, left=2mm, colbacktitle=quarto-callout-note-color!10!white, breakable, toprule=.15mm, bottomtitle=1mm, opacityback=0, coltitle=black, title=\textcolor{quarto-callout-note-color}{\faInfo}\hspace{0.5em}{Exemplo 17.1 - Eu sou um grande atirador de lances livres}, rightrule=.15mm, bottomrule=.15mm, toptitle=1mm, colback=white]

Eu afirmo que acerto 80\% de meus lances livres no jogo de basquete.
Para testar minha afirmativa, você me pede para fazer 20 lances livres.
Eu acerto apenas oito dos 20. ``Ah!'', você diz. ``Alguém que acerta
80\% de seus lances livres quase nunca acertaria apenas oito entre 20.
Logo, não acredito em sua afirmativa.''

Seu raciocínio se baseia no questionamento do que ocorreria se minha
afirmativa fosse verdadeira e repetíssemos a amostra de 20 lançamentos
muitas vezes: eu quase nunca acertaria oito ou menos. Esse resultado de
oito em 20 é tão improvável, que fornece uma forte evidência de que
minha afirmativa não seja verdadeira.

Você pode dizer quão forte é a evidência contra minha afirmativa,
fornecendo a probabilidade de eu acertar oito ou menos entre 20 lances
livres, se eu realmente acertasse 80\% no longo prazo. Essa
probabilidade é 0,0001; como descrito no Capítulo 14, esse cálculo é
feito com o uso da distribuição binomial.

Assim, eu acertaria oito ou menos em 20 lances em apenas uma vez em 10
mil tentativas no longo prazo - onde cada ``tentativa'' são 20 lances
livres jogados - se minha afirmativa de acertar 80\% fosse verdadeira. O
pequeno valor da probabilidade o convence de que minha afirmativa é
falsa.

O \emph{applet Reasoning of a Statistical Test} (conteúdo em inglês) faz
uma animação do Exemplo 17.1. Você pode pedir a um jogador que faça
lances livres até que os dados lhe convençam, ou não, de que ele faz
menos do que 80\%.

Testes de significância usam um vocabulário elaborado, mas a ideia
básica é simples: um resultado que raramente ocorreria se uma afirmativa
fosse verdadeira é boa evidência de que a afirmativa não seja
verdadeira.

\end{tcolorbox}

\section{A lógica dos testes de significância}\label{sec-logica}

A lógica dos testes estatísticos, assim como a dos intervalos de
confiança, se baseia no questionamento do que ocorreria se repetíssemos
a amostra ou experimento muitas vezes. Agiremos novamente como se as
``condições simples'' listadas em ``Condições simples para inferência
sobre uma média'', no Capítulo 16, fossem verdadeiras: temos uma AAS
perfeita de uma população exatamente Normal com desvio-padrão \(\sigma\)
conhecido por nós. Eis um exemplo que analisaremos.

\begin{tcolorbox}[enhanced jigsaw, arc=.35mm, opacitybacktitle=0.6, colframe=quarto-callout-note-color-frame, titlerule=0mm, leftrule=.75mm, left=2mm, colbacktitle=quarto-callout-note-color!10!white, breakable, toprule=.15mm, bottomtitle=1mm, opacityback=0, coltitle=black, title=\textcolor{quarto-callout-note-color}{\faInfo}\hspace{0.5em}{Exemplo 17.2 - Adoçantes de refrigerantes}, rightrule=.15mm, bottomrule=.15mm, toptitle=1mm, colback=white]

Refrigerantes dietéticos usam adoçantes artificiais para evitar o uso de
açúcar. Esses adoçantes gradualmente perdem sua doçura ao longo do
tempo. Os fabricantes, portanto, testam a perda de doçura dos
refrigerantes novos antes de colocá-los no mercado.

Provadores treinados bebem um pequeno gole de refrigerante, juntamente
com bebidas de doçura padrão, e atribuem ao refrigerante um ``escore de
doçura'' de 1 a 10, com maiores escores correspondendo a maior doçura. O
refrigerante é, então, armazenado por um mês em alta temperatura para
imitar o efeito do armazenamento por 4 meses em temperatura ambiente.

Cada provador atribui um escore ao refrigerante novamente após o
armazenamento. Esse é um experimento de dados emparelhados. Nossos dados
são as diferenças (escore antes do armazenamento menos escore após o
armazenamento) dos escores dos provadores. Quanto maior a diferença
(diferença \textgreater{} 0), maior será a perda de doçura.

Suponha sabermos que, para qualquer refrigerante, os escores de perda de
doçura variem de provador para provador de acordo com uma distribuição
Normal, com desvio-padrão \(\sigma = 1\). A média \(\mu\) de todos os
provadores mede a perda de doçura e é diferente para diferentes
refrigerantes.

A seguir, estão as perdas de doçura de um novo refrigerante, medidas por
10 provadores treinados:

\textbf{1,6 0,4 0,5 -2,0 1,5 -1,1 1,3 -0,1 -0,3 1,2}

A perda média de doçura é dada pela média amostral \(\bar{x} = 0,3\), de
modo que, em média, os 10 provadores encontraram uma pequena perda de
doçura. Também, mais da metade, (seis) dos provadores encontraram uma
perda de doçura. Esses dados são uma boa evidência de que o refrigerante
perdeu doçura com o armazenamento?

O raciocínio é o mesmo do Exemplo 17.1. Fazemos uma afirmativa e
perguntamos se os dados fornecem evidência contrária a ela. Procuramos
evidência de que haja uma perda de doçura; logo, a afirmativa que
testamos é que não há perda. Nesse caso, a perda média para a população
de todos os provadores treinados seria \(\mu = 0\).

\begin{itemize}
\tightlist
\item
  Se a afirmativa de que \(\mu = 0\) é verdadeira, a distribuição
  amostral de \(\bar{x}\) dos 10 provadores é Normal com média
  \(\mu = 0\) e desvio-padrão:
\end{itemize}

\[\frac{\sigma}{\sqrt{n}} = \frac{1}{\sqrt{10}} = 0.316\]

Para esse refrigerante, 10 provadores acusaram perda média
\(\bar{x} = 0,3\). É claro que um \(\bar{x}\) desse tamanho não é
particularmente surpreendente. Ele poderia facilmente ocorrer apenas
devido ao acaso, quando a média da população é \(\mu = 0\). O fato de
obter \(\bar{x} = 0,3\) para 10 provadores não é forte evidência de que
esse refrigerante perca doçura.

\end{tcolorbox}

\begin{tcolorbox}[enhanced jigsaw, arc=.35mm, opacitybacktitle=0.6, colframe=quarto-callout-note-color-frame, titlerule=0mm, leftrule=.75mm, left=2mm, colbacktitle=quarto-callout-note-color!10!white, breakable, toprule=.15mm, bottomtitle=1mm, opacityback=0, coltitle=black, title=\textcolor{quarto-callout-note-color}{\faInfo}\hspace{0.5em}{Exemplo 17.3 - Adoçantes de refrigerantes, novamente}, rightrule=.15mm, bottomrule=.15mm, toptitle=1mm, colback=white]

A seguir, estão as perdas de doçura de um novo refrigerante, conforme
medidas por 10 provadores treinados:

\textbf{2,0 0,4 0,7 2,0 -0,4 2,2 -1,3 1,2 1,1 2,3}

A perda média de doçura é dada pela média amostral \(\bar{x} = 1,02\). A
maioria dos escores é positiva. Isto é, a maioria dos provadores
encontrou uma perda de doçura. Mas as perdas são pequenas, e dois
provadores (os escores negativos) acharam que o refrigerante ganhou
doçura. Esses dados constituem \textbf{boa evidência} de que o
refrigerante perdeu doçura no armazenamento?

O teste de sabor para o novo refrigerante produziu \(\bar{x} = 1,02\).
Isso está bem longe, na cauda da curva Normal - tão longe que
\textbf{\emph{um valor observado desse tamanho raramente ocorreria por
acaso se o verdadeiro}} \(\mu\) fosse 0. Esse valor observado é boa
evidência de que o verdadeiro \(\mu\) é, de fato, maior do que 0 - isto
é, o refrigerante perdeu doçura. O fabricante deve reformular o novo
refrigerante e tentar novamente.

\begin{Shaded}
\begin{Highlighting}[numbers=left,,]
\InformationTok{\textasciigrave{}\textasciigrave{}\textasciigrave{}\{r\}}
\CommentTok{\# Exemplo dos Adoçantes de Refrigerante {-} Distribuição Amostral da Média}
\CommentTok{\# Baseado no Capítulo 17 {-} Testes de Significância: o Básico}

\CommentTok{\# Parâmetros do problema}
\NormalTok{mu\_0 }\OtherTok{\textless{}{-}} \DecValTok{0}        \CommentTok{\# Hipótese nula: μ = 0 (sem perda de doçura)}
\NormalTok{sigma }\OtherTok{\textless{}{-}} \DecValTok{1}       \CommentTok{\# Desvio{-}padrão populacional conhecido}
\NormalTok{n }\OtherTok{\textless{}{-}} \DecValTok{10}          \CommentTok{\# Tamanho da amostra (10 provadores)}
\NormalTok{alpha }\OtherTok{\textless{}{-}} \FloatTok{0.05}    \CommentTok{\# Nível de significância}

\CommentTok{\# Desvio{-}padrão da distribuição amostral}
\NormalTok{sigma\_x\_bar }\OtherTok{\textless{}{-}}\NormalTok{ sigma }\SpecialCharTok{/} \FunctionTok{sqrt}\NormalTok{(n)}
\FunctionTok{cat}\NormalTok{(}\StringTok{"Desvio{-}padrão da distribuição amostral:"}\NormalTok{, }\FunctionTok{round}\NormalTok{(sigma\_x\_bar, }\DecValTok{4}\NormalTok{), }\StringTok{"}\SpecialCharTok{\textbackslash{}n}\StringTok{"}\NormalTok{)}

\CommentTok{\# Dados dos dois refrigerantes do exemplo}
\NormalTok{x\_bar\_1 }\OtherTok{\textless{}{-}} \FloatTok{0.3}   \CommentTok{\# Primeiro refrigerante}
\NormalTok{x\_bar\_2 }\OtherTok{\textless{}{-}} \FloatTok{1.02}  \CommentTok{\# Segundo refrigerante}

\CommentTok{\# Estatísticas de teste Z}
\NormalTok{z\_1 }\OtherTok{\textless{}{-}}\NormalTok{ (x\_bar\_1 }\SpecialCharTok{{-}}\NormalTok{ mu\_0) }\SpecialCharTok{/}\NormalTok{ sigma\_x\_bar}
\NormalTok{z\_2 }\OtherTok{\textless{}{-}}\NormalTok{ (x\_bar\_2 }\SpecialCharTok{{-}}\NormalTok{ mu\_0) }\SpecialCharTok{/}\NormalTok{ sigma\_x\_bar}

\FunctionTok{cat}\NormalTok{(}\StringTok{"}\SpecialCharTok{\textbackslash{}n}\StringTok{Estatísticas de teste:"}\NormalTok{)}
\FunctionTok{cat}\NormalTok{(}\StringTok{"}\SpecialCharTok{\textbackslash{}n}\StringTok{Refrigerante 1: x̄ ="}\NormalTok{, x\_bar\_1, }\StringTok{", z ="}\NormalTok{, }\FunctionTok{round}\NormalTok{(z\_1, }\DecValTok{3}\NormalTok{))}
\FunctionTok{cat}\NormalTok{(}\StringTok{"}\SpecialCharTok{\textbackslash{}n}\StringTok{Refrigerante 2: x̄ ="}\NormalTok{, x\_bar\_2, }\StringTok{", z ="}\NormalTok{, }\FunctionTok{round}\NormalTok{(z\_2, }\DecValTok{3}\NormalTok{), }\StringTok{"}\SpecialCharTok{\textbackslash{}n}\StringTok{"}\NormalTok{)}

\CommentTok{\# Valores P (teste unilateral: H₁: μ \textgreater{} 0)}
\NormalTok{p\_value\_1 }\OtherTok{\textless{}{-}} \DecValTok{1} \SpecialCharTok{{-}} \FunctionTok{pnorm}\NormalTok{(z\_1)}
\NormalTok{p\_value\_2 }\OtherTok{\textless{}{-}} \DecValTok{1} \SpecialCharTok{{-}} \FunctionTok{pnorm}\NormalTok{(z\_2)}

\FunctionTok{cat}\NormalTok{(}\StringTok{"}\SpecialCharTok{\textbackslash{}n}\StringTok{Valores P (teste unilateral):"}\NormalTok{)}
\FunctionTok{cat}\NormalTok{(}\StringTok{"}\SpecialCharTok{\textbackslash{}n}\StringTok{Refrigerante 1: P ="}\NormalTok{, }\FunctionTok{round}\NormalTok{(p\_value\_1, }\DecValTok{4}\NormalTok{))}
\FunctionTok{cat}\NormalTok{(}\StringTok{"}\SpecialCharTok{\textbackslash{}n}\StringTok{Refrigerante 2: P ="}\NormalTok{, }\FunctionTok{round}\NormalTok{(p\_value\_2, }\DecValTok{4}\NormalTok{), }\StringTok{"}\SpecialCharTok{\textbackslash{}n}\StringTok{"}\NormalTok{)}

\CommentTok{\# Valor crítico para α = 0.05 (teste unilateral)}
\NormalTok{z\_critico }\OtherTok{\textless{}{-}} \FunctionTok{qnorm}\NormalTok{(}\DecValTok{1} \SpecialCharTok{{-}}\NormalTok{ alpha)}
\NormalTok{x\_bar\_critico }\OtherTok{\textless{}{-}}\NormalTok{ mu\_0 }\SpecialCharTok{+}\NormalTok{ z\_critico }\SpecialCharTok{*}\NormalTok{ sigma\_x\_bar}

\FunctionTok{cat}\NormalTok{(}\StringTok{"}\SpecialCharTok{\textbackslash{}n}\StringTok{Região crítica:"}\NormalTok{)}
\FunctionTok{cat}\NormalTok{(}\StringTok{"}\SpecialCharTok{\textbackslash{}n}\StringTok{z crítico ="}\NormalTok{, }\FunctionTok{round}\NormalTok{(z\_critico, }\DecValTok{3}\NormalTok{))}
\FunctionTok{cat}\NormalTok{(}\StringTok{"}\SpecialCharTok{\textbackslash{}n}\StringTok{x̄ crítico ="}\NormalTok{, }\FunctionTok{round}\NormalTok{(x\_bar\_critico, }\DecValTok{3}\NormalTok{), }\StringTok{"}\SpecialCharTok{\textbackslash{}n}\StringTok{"}\NormalTok{)}

\CommentTok{\# Gráfico da distribuição amostral}
\FunctionTok{library}\NormalTok{(ggplot2)}

\CommentTok{\# Criar sequência de valores para x̄}
\NormalTok{x\_seq }\OtherTok{\textless{}{-}} \FunctionTok{seq}\NormalTok{(}\SpecialCharTok{{-}}\FloatTok{1.5}\NormalTok{, }\DecValTok{2}\NormalTok{, }\AttributeTok{length.out =} \DecValTok{1000}\NormalTok{)}
\NormalTok{y\_seq }\OtherTok{\textless{}{-}} \FunctionTok{dnorm}\NormalTok{(x\_seq, }\AttributeTok{mean =}\NormalTok{ mu\_0, }\AttributeTok{sd =}\NormalTok{ sigma\_x\_bar)}

\CommentTok{\# Criar data frame para o gráfico}
\NormalTok{df }\OtherTok{\textless{}{-}} \FunctionTok{data.frame}\NormalTok{(}\AttributeTok{x =}\NormalTok{ x\_seq, }\AttributeTok{y =}\NormalTok{ y\_seq)}

\CommentTok{\# Região de rejeição (α = 0.05)}
\NormalTok{x\_reject }\OtherTok{\textless{}{-}} \FunctionTok{seq}\NormalTok{(x\_bar\_critico, }\DecValTok{2}\NormalTok{, }\AttributeTok{length.out =} \DecValTok{100}\NormalTok{)}
\NormalTok{y\_reject }\OtherTok{\textless{}{-}} \FunctionTok{dnorm}\NormalTok{(x\_reject, }\AttributeTok{mean =}\NormalTok{ mu\_0, }\AttributeTok{sd =}\NormalTok{ sigma\_x\_bar)}
\NormalTok{df\_reject }\OtherTok{\textless{}{-}} \FunctionTok{data.frame}\NormalTok{(}\AttributeTok{x =}\NormalTok{ x\_reject, }\AttributeTok{y =}\NormalTok{ y\_reject)}

\CommentTok{\# Criar o gráfico}
\NormalTok{p }\OtherTok{\textless{}{-}} \FunctionTok{ggplot}\NormalTok{(df, }\FunctionTok{aes}\NormalTok{(}\AttributeTok{x =}\NormalTok{ x, }\AttributeTok{y =}\NormalTok{ y)) }\SpecialCharTok{+}
  \FunctionTok{geom\_line}\NormalTok{(}\AttributeTok{size =} \FloatTok{1.2}\NormalTok{, }\AttributeTok{color =} \StringTok{"blue"}\NormalTok{) }\SpecialCharTok{+}
  
  \CommentTok{\# Região de rejeição}
  \FunctionTok{geom\_area}\NormalTok{(}\AttributeTok{data =}\NormalTok{ df\_reject, }\FunctionTok{aes}\NormalTok{(}\AttributeTok{x =}\NormalTok{ x, }\AttributeTok{y =}\NormalTok{ y), }
            \AttributeTok{fill =} \StringTok{"red"}\NormalTok{, }\AttributeTok{alpha =} \FloatTok{0.3}\NormalTok{) }\SpecialCharTok{+}
  
  \CommentTok{\# Linha vertical para H₀: μ = 0}
  \FunctionTok{geom\_vline}\NormalTok{(}\AttributeTok{xintercept =}\NormalTok{ mu\_0, }\AttributeTok{linetype =} \StringTok{"dashed"}\NormalTok{, }
             \AttributeTok{color =} \StringTok{"black"}\NormalTok{, }\AttributeTok{size =} \DecValTok{1}\NormalTok{) }\SpecialCharTok{+}
  
  \CommentTok{\# Linha vertical para valor crítico}
  \FunctionTok{geom\_vline}\NormalTok{(}\AttributeTok{xintercept =}\NormalTok{ x\_bar\_critico, }\AttributeTok{linetype =} \StringTok{"solid"}\NormalTok{, }
             \AttributeTok{color =} \StringTok{"red"}\NormalTok{, }\AttributeTok{size =} \DecValTok{1}\NormalTok{) }\SpecialCharTok{+}
  
  \CommentTok{\# Pontos das médias amostrais observadas}
  \FunctionTok{geom\_point}\NormalTok{(}\FunctionTok{aes}\NormalTok{(}\AttributeTok{x =}\NormalTok{ x\_bar\_1, }\AttributeTok{y =} \FunctionTok{dnorm}\NormalTok{(x\_bar\_1, mu\_0, sigma\_x\_bar)), }
             \AttributeTok{color =} \StringTok{"green"}\NormalTok{, }\AttributeTok{size =} \DecValTok{4}\NormalTok{, }\AttributeTok{shape =} \DecValTok{16}\NormalTok{) }\SpecialCharTok{+}
  \FunctionTok{geom\_point}\NormalTok{(}\FunctionTok{aes}\NormalTok{(}\AttributeTok{x =}\NormalTok{ x\_bar\_2, }\AttributeTok{y =} \FunctionTok{dnorm}\NormalTok{(x\_bar\_2, mu\_0, sigma\_x\_bar)), }
             \AttributeTok{color =} \StringTok{"orange"}\NormalTok{, }\AttributeTok{size =} \DecValTok{4}\NormalTok{, }\AttributeTok{shape =} \DecValTok{16}\NormalTok{) }\SpecialCharTok{+}
  
  \CommentTok{\# Rótulos e títulos}
  \FunctionTok{labs}\NormalTok{(}
    \AttributeTok{title =} \StringTok{"Distribuição Amostral da Média {-} Exemplo dos Adoçantes"}\NormalTok{,}
    \AttributeTok{subtitle =} \FunctionTok{paste}\NormalTok{(}\StringTok{"n ="}\NormalTok{, n, }\StringTok{", σ ="}\NormalTok{, sigma, }\StringTok{", σx̄ ="}\NormalTok{, }\FunctionTok{round}\NormalTok{(sigma\_x\_bar, }\DecValTok{3}\NormalTok{)),}
    \AttributeTok{x =} \StringTok{"Média Amostral (x̄)"}\NormalTok{,}
    \AttributeTok{y =} \StringTok{"Densidade"}\NormalTok{,}
    \AttributeTok{caption =} \StringTok{"Região vermelha: α = 0.05 (região de rejeição para H₁: μ \textgreater{} 0)"}
\NormalTok{  ) }\SpecialCharTok{+}
  
  \CommentTok{\# Anotações}
  \FunctionTok{annotate}\NormalTok{(}\StringTok{"text"}\NormalTok{, }\AttributeTok{x =}\NormalTok{ mu\_0, }\AttributeTok{y =} \FloatTok{0.9}\NormalTok{, }
           \AttributeTok{label =} \StringTok{"H₀: μ = 0"}\NormalTok{, }\AttributeTok{vjust =} \SpecialCharTok{{-}}\FloatTok{0.5}\NormalTok{, }\AttributeTok{hjust =} \FloatTok{0.5}\NormalTok{) }\SpecialCharTok{+}
  \FunctionTok{annotate}\NormalTok{(}\StringTok{"text"}\NormalTok{, }\AttributeTok{x =}\NormalTok{ x\_bar\_critico, }\AttributeTok{y =} \FloatTok{0.7}\NormalTok{, }
           \AttributeTok{label =} \FunctionTok{paste}\NormalTok{(}\StringTok{"x̄ crítico ="}\NormalTok{, }\FunctionTok{round}\NormalTok{(x\_bar\_critico, }\DecValTok{3}\NormalTok{)), }
           \AttributeTok{vjust =} \SpecialCharTok{{-}}\FloatTok{0.5}\NormalTok{, }\AttributeTok{hjust =} \FloatTok{1.1}\NormalTok{, }\AttributeTok{color =} \StringTok{"red"}\NormalTok{) }\SpecialCharTok{+}
  \FunctionTok{annotate}\NormalTok{(}\StringTok{"text"}\NormalTok{, }\AttributeTok{x =}\NormalTok{ x\_bar\_1, }\AttributeTok{y =} \FloatTok{0.3}\NormalTok{, }
           \AttributeTok{label =} \FunctionTok{paste}\NormalTok{(}\StringTok{"Refrig. 1}\SpecialCharTok{\textbackslash{}n}\StringTok{x̄ ="}\NormalTok{, x\_bar\_1, }\StringTok{"}\SpecialCharTok{\textbackslash{}n}\StringTok{z ="}\NormalTok{, }\FunctionTok{round}\NormalTok{(z\_1, }\DecValTok{3}\NormalTok{), }
                        \StringTok{"}\SpecialCharTok{\textbackslash{}n}\StringTok{P ="}\NormalTok{, }\FunctionTok{round}\NormalTok{(p\_value\_1, }\DecValTok{4}\NormalTok{)), }
           \AttributeTok{vjust =} \DecValTok{1}\NormalTok{, }\AttributeTok{hjust =} \FloatTok{0.5}\NormalTok{, }\AttributeTok{color =} \StringTok{"green"}\NormalTok{, }\AttributeTok{size =} \DecValTok{3}\NormalTok{) }\SpecialCharTok{+}
  \FunctionTok{annotate}\NormalTok{(}\StringTok{"text"}\NormalTok{, }\AttributeTok{x =}\NormalTok{ x\_bar\_2, }\AttributeTok{y =} \FloatTok{0.15}\NormalTok{, }
           \AttributeTok{label =} \FunctionTok{paste}\NormalTok{(}\StringTok{"Refrig. 2}\SpecialCharTok{\textbackslash{}n}\StringTok{x̄ ="}\NormalTok{, x\_bar\_2, }\StringTok{"}\SpecialCharTok{\textbackslash{}n}\StringTok{z ="}\NormalTok{, }\FunctionTok{round}\NormalTok{(z\_2, }\DecValTok{3}\NormalTok{), }
                        \StringTok{"}\SpecialCharTok{\textbackslash{}n}\StringTok{P ="}\NormalTok{, }\FunctionTok{round}\NormalTok{(p\_value\_2, }\DecValTok{4}\NormalTok{)), }
           \AttributeTok{vjust =} \DecValTok{1}\NormalTok{, }\AttributeTok{hjust =} \FloatTok{0.5}\NormalTok{, }\AttributeTok{color =} \StringTok{"orange"}\NormalTok{, }\AttributeTok{size =} \DecValTok{3}\NormalTok{) }\SpecialCharTok{+}
  
  \CommentTok{\# Tema}
  \FunctionTok{theme\_minimal}\NormalTok{() }\SpecialCharTok{+}
  \FunctionTok{theme}\NormalTok{(}
    \AttributeTok{plot.title =} \FunctionTok{element\_text}\NormalTok{(}\AttributeTok{hjust =} \FloatTok{0.5}\NormalTok{, }\AttributeTok{size =} \DecValTok{14}\NormalTok{, }\AttributeTok{face =} \StringTok{"bold"}\NormalTok{),}
    \AttributeTok{plot.subtitle =} \FunctionTok{element\_text}\NormalTok{(}\AttributeTok{hjust =} \FloatTok{0.5}\NormalTok{, }\AttributeTok{size =} \DecValTok{12}\NormalTok{),}
    \AttributeTok{axis.title =} \FunctionTok{element\_text}\NormalTok{(}\AttributeTok{size =} \DecValTok{12}\NormalTok{),}
    \AttributeTok{legend.position =} \StringTok{"none"}
\NormalTok{  ) }\SpecialCharTok{+}
  
  \CommentTok{\# Escalas}
  \FunctionTok{scale\_x\_continuous}\NormalTok{(}\AttributeTok{breaks =} \FunctionTok{seq}\NormalTok{(}\SpecialCharTok{{-}}\FloatTok{1.5}\NormalTok{, }\DecValTok{2}\NormalTok{, }\FloatTok{0.5}\NormalTok{)) }\SpecialCharTok{+}
  \FunctionTok{ylim}\NormalTok{(}\DecValTok{0}\NormalTok{, }\FloatTok{1.3}\NormalTok{)}

\CommentTok{\# Exibir o gráfico}
\FunctionTok{print}\NormalTok{(p)}

\CommentTok{\# Gráfico adicional: Distribuição Normal Padrão (escala Z)}
\NormalTok{z\_seq }\OtherTok{\textless{}{-}} \FunctionTok{seq}\NormalTok{(}\SpecialCharTok{{-}}\DecValTok{4}\NormalTok{, }\DecValTok{4}\NormalTok{, }\AttributeTok{length.out =} \DecValTok{1000}\NormalTok{)}
\NormalTok{y\_z\_seq }\OtherTok{\textless{}{-}} \FunctionTok{dnorm}\NormalTok{(z\_seq)}
\NormalTok{df\_z }\OtherTok{\textless{}{-}} \FunctionTok{data.frame}\NormalTok{(}\AttributeTok{z =}\NormalTok{ z\_seq, }\AttributeTok{y =}\NormalTok{ y\_z\_seq)}

\CommentTok{\# Região de rejeição na escala Z}
\NormalTok{z\_reject\_seq }\OtherTok{\textless{}{-}} \FunctionTok{seq}\NormalTok{(z\_critico, }\DecValTok{4}\NormalTok{, }\AttributeTok{length.out =} \DecValTok{100}\NormalTok{)}
\NormalTok{y\_z\_reject }\OtherTok{\textless{}{-}} \FunctionTok{dnorm}\NormalTok{(z\_reject\_seq)}
\NormalTok{df\_z\_reject }\OtherTok{\textless{}{-}} \FunctionTok{data.frame}\NormalTok{(}\AttributeTok{z =}\NormalTok{ z\_reject\_seq, }\AttributeTok{y =}\NormalTok{ y\_z\_reject)}

\NormalTok{p\_z }\OtherTok{\textless{}{-}} \FunctionTok{ggplot}\NormalTok{(df\_z, }\FunctionTok{aes}\NormalTok{(}\AttributeTok{x =}\NormalTok{ z, }\AttributeTok{y =}\NormalTok{ y)) }\SpecialCharTok{+}
  \FunctionTok{geom\_line}\NormalTok{(}\AttributeTok{size =} \FloatTok{1.2}\NormalTok{, }\AttributeTok{color =} \StringTok{"blue"}\NormalTok{) }\SpecialCharTok{+}
  
  \CommentTok{\# Região de rejeição}
  \FunctionTok{geom\_area}\NormalTok{(}\AttributeTok{data =}\NormalTok{ df\_z\_reject, }\FunctionTok{aes}\NormalTok{(}\AttributeTok{x =}\NormalTok{ z, }\AttributeTok{y =}\NormalTok{ y), }
            \AttributeTok{fill =} \StringTok{"red"}\NormalTok{, }\AttributeTok{alpha =} \FloatTok{0.3}\NormalTok{) }\SpecialCharTok{+}
  
  \CommentTok{\# Linha vertical para z = 0}
  \FunctionTok{geom\_vline}\NormalTok{(}\AttributeTok{xintercept =} \DecValTok{0}\NormalTok{, }\AttributeTok{linetype =} \StringTok{"dashed"}\NormalTok{, }
             \AttributeTok{color =} \StringTok{"black"}\NormalTok{, }\AttributeTok{size =} \DecValTok{1}\NormalTok{) }\SpecialCharTok{+}
  
  \CommentTok{\# Linha vertical para z crítico}
  \FunctionTok{geom\_vline}\NormalTok{(}\AttributeTok{xintercept =}\NormalTok{ z\_critico, }\AttributeTok{linetype =} \StringTok{"solid"}\NormalTok{, }
             \AttributeTok{color =} \StringTok{"red"}\NormalTok{, }\AttributeTok{size =} \DecValTok{1}\NormalTok{) }\SpecialCharTok{+}
  
  \CommentTok{\# Pontos das estatísticas Z observadas}
  \FunctionTok{geom\_point}\NormalTok{(}\FunctionTok{aes}\NormalTok{(}\AttributeTok{x =}\NormalTok{ z\_1, }\AttributeTok{y =} \FunctionTok{dnorm}\NormalTok{(z\_1)), }
             \AttributeTok{color =} \StringTok{"green"}\NormalTok{, }\AttributeTok{size =} \DecValTok{4}\NormalTok{, }\AttributeTok{shape =} \DecValTok{16}\NormalTok{) }\SpecialCharTok{+}
  \FunctionTok{geom\_point}\NormalTok{(}\FunctionTok{aes}\NormalTok{(}\AttributeTok{x =}\NormalTok{ z\_2, }\AttributeTok{y =} \FunctionTok{dnorm}\NormalTok{(z\_2)), }
             \AttributeTok{color =} \StringTok{"orange"}\NormalTok{, }\AttributeTok{size =} \DecValTok{4}\NormalTok{, }\AttributeTok{shape =} \DecValTok{16}\NormalTok{) }\SpecialCharTok{+}
  
  \CommentTok{\# Rótulos e títulos}
  \FunctionTok{labs}\NormalTok{(}
    \AttributeTok{title =} \StringTok{"Distribuição Normal Padrão {-} Estatística de Teste Z"}\NormalTok{,}
    \AttributeTok{subtitle =} \StringTok{"Teste Unilateral: H₀: μ = 0 vs H₁: μ \textgreater{} 0"}\NormalTok{,}
    \AttributeTok{x =} \StringTok{"Estatística Z"}\NormalTok{,}
    \AttributeTok{y =} \StringTok{"Densidade"}\NormalTok{,}
    \AttributeTok{caption =} \StringTok{"Região vermelha: α = 0.05 (região de rejeição)"}
\NormalTok{  ) }\SpecialCharTok{+}
  
  \CommentTok{\# Anotações}
  \FunctionTok{annotate}\NormalTok{(}\StringTok{"text"}\NormalTok{, }\AttributeTok{x =} \DecValTok{0}\NormalTok{, }\AttributeTok{y =} \FloatTok{0.3}\NormalTok{, }
           \AttributeTok{label =} \StringTok{"H₀: μ = 0}\SpecialCharTok{\textbackslash{}n}\StringTok{(z = 0)"}\NormalTok{, }\AttributeTok{vjust =} \SpecialCharTok{{-}}\FloatTok{0.5}\NormalTok{, }\AttributeTok{hjust =} \FloatTok{0.5}\NormalTok{) }\SpecialCharTok{+}
  \FunctionTok{annotate}\NormalTok{(}\StringTok{"text"}\NormalTok{, }\AttributeTok{x =}\NormalTok{ z\_critico, }\AttributeTok{y =} \FloatTok{0.25}\NormalTok{, }
           \AttributeTok{label =} \FunctionTok{paste}\NormalTok{(}\StringTok{"z crítico ="}\NormalTok{, }\FunctionTok{round}\NormalTok{(z\_critico, }\DecValTok{3}\NormalTok{)), }
           \AttributeTok{vjust =} \SpecialCharTok{{-}}\FloatTok{0.5}\NormalTok{, }\AttributeTok{hjust =} \FloatTok{1.1}\NormalTok{, }\AttributeTok{color =} \StringTok{"red"}\NormalTok{) }\SpecialCharTok{+}
  \FunctionTok{annotate}\NormalTok{(}\StringTok{"text"}\NormalTok{, }\AttributeTok{x =}\NormalTok{ z\_1, }\AttributeTok{y =} \FloatTok{0.2}\NormalTok{, }
           \AttributeTok{label =} \FunctionTok{paste}\NormalTok{(}\StringTok{"z₁ ="}\NormalTok{, }\FunctionTok{round}\NormalTok{(z\_1, }\DecValTok{3}\NormalTok{)), }
           \AttributeTok{vjust =} \DecValTok{1}\NormalTok{, }\AttributeTok{hjust =} \FloatTok{0.5}\NormalTok{, }\AttributeTok{color =} \StringTok{"green"}\NormalTok{, }\AttributeTok{size =} \DecValTok{3}\NormalTok{) }\SpecialCharTok{+}
  \FunctionTok{annotate}\NormalTok{(}\StringTok{"text"}\NormalTok{, }\AttributeTok{x =}\NormalTok{ z\_2, }\AttributeTok{y =} \FloatTok{0.15}\NormalTok{, }
           \AttributeTok{label =} \FunctionTok{paste}\NormalTok{(}\StringTok{"z₂ ="}\NormalTok{, }\FunctionTok{round}\NormalTok{(z\_2, }\DecValTok{3}\NormalTok{)), }
           \AttributeTok{vjust =} \DecValTok{1}\NormalTok{, }\AttributeTok{hjust =} \FloatTok{0.5}\NormalTok{, }\AttributeTok{color =} \StringTok{"orange"}\NormalTok{, }\AttributeTok{size =} \DecValTok{3}\NormalTok{) }\SpecialCharTok{+}
  
  \CommentTok{\# Tema}
  \FunctionTok{theme\_minimal}\NormalTok{() }\SpecialCharTok{+}
  \FunctionTok{theme}\NormalTok{(}
    \AttributeTok{plot.title =} \FunctionTok{element\_text}\NormalTok{(}\AttributeTok{hjust =} \FloatTok{0.5}\NormalTok{, }\AttributeTok{size =} \DecValTok{14}\NormalTok{, }\AttributeTok{face =} \StringTok{"bold"}\NormalTok{),}
    \AttributeTok{plot.subtitle =} \FunctionTok{element\_text}\NormalTok{(}\AttributeTok{hjust =} \FloatTok{0.5}\NormalTok{, }\AttributeTok{size =} \DecValTok{12}\NormalTok{),}
    \AttributeTok{axis.title =} \FunctionTok{element\_text}\NormalTok{(}\AttributeTok{size =} \DecValTok{12}\NormalTok{)}
\NormalTok{  ) }\SpecialCharTok{+}
  
  \CommentTok{\# Escalas}
  \FunctionTok{scale\_x\_continuous}\NormalTok{(}\AttributeTok{breaks =} \FunctionTok{seq}\NormalTok{(}\SpecialCharTok{{-}}\DecValTok{4}\NormalTok{, }\DecValTok{4}\NormalTok{, }\DecValTok{1}\NormalTok{)) }\SpecialCharTok{+}
  \FunctionTok{ylim}\NormalTok{(}\DecValTok{0}\NormalTok{, }\FloatTok{0.45}\NormalTok{)}

\CommentTok{\# Exibir o segundo gráfico}
\FunctionTok{print}\NormalTok{(p\_z)}

\CommentTok{\# Resumo dos resultados}
\FunctionTok{cat}\NormalTok{(}\StringTok{"}\SpecialCharTok{\textbackslash{}n}\StringTok{"}\NormalTok{ , }\FunctionTok{rep}\NormalTok{(}\StringTok{"="}\NormalTok{, }\DecValTok{50}\NormalTok{), }\StringTok{"}\SpecialCharTok{\textbackslash{}n}\StringTok{"}\NormalTok{)}
\FunctionTok{cat}\NormalTok{(}\StringTok{"RESUMO DOS RESULTADOS}\SpecialCharTok{\textbackslash{}n}\StringTok{"}\NormalTok{)}
\FunctionTok{cat}\NormalTok{(}\FunctionTok{rep}\NormalTok{(}\StringTok{"="}\NormalTok{, }\DecValTok{50}\NormalTok{), }\StringTok{"}\SpecialCharTok{\textbackslash{}n}\StringTok{"}\NormalTok{)}
\FunctionTok{cat}\NormalTok{(}\StringTok{"Teste: H₀: μ = 0 vs H₁: μ \textgreater{} 0 (unilateral)}\SpecialCharTok{\textbackslash{}n}\StringTok{"}\NormalTok{)}
\FunctionTok{cat}\NormalTok{(}\StringTok{"Nível de significância: α ="}\NormalTok{, alpha, }\StringTok{"}\SpecialCharTok{\textbackslash{}n}\StringTok{"}\NormalTok{)}
\FunctionTok{cat}\NormalTok{(}\StringTok{"Valor crítico: z ="}\NormalTok{, }\FunctionTok{round}\NormalTok{(z\_critico, }\DecValTok{3}\NormalTok{), }\StringTok{", x̄ ="}\NormalTok{, }\FunctionTok{round}\NormalTok{(x\_bar\_critico, }\DecValTok{3}\NormalTok{), }\StringTok{"}\SpecialCharTok{\textbackslash{}n\textbackslash{}n}\StringTok{"}\NormalTok{)}

\FunctionTok{cat}\NormalTok{(}\StringTok{"REFRIGERANTE 1:}\SpecialCharTok{\textbackslash{}n}\StringTok{"}\NormalTok{)}
\FunctionTok{cat}\NormalTok{(}\StringTok{"  Média amostral: x̄ ="}\NormalTok{, x\_bar\_1, }\StringTok{"}\SpecialCharTok{\textbackslash{}n}\StringTok{"}\NormalTok{)}
\FunctionTok{cat}\NormalTok{(}\StringTok{"  Estatística Z: z ="}\NormalTok{, }\FunctionTok{round}\NormalTok{(z\_1, }\DecValTok{3}\NormalTok{), }\StringTok{"}\SpecialCharTok{\textbackslash{}n}\StringTok{"}\NormalTok{)}
\FunctionTok{cat}\NormalTok{(}\StringTok{"  Valor P:"}\NormalTok{, }\FunctionTok{round}\NormalTok{(p\_value\_1, }\DecValTok{4}\NormalTok{), }\StringTok{"}\SpecialCharTok{\textbackslash{}n}\StringTok{"}\NormalTok{)}
\FunctionTok{cat}\NormalTok{(}\StringTok{"  Conclusão:"}\NormalTok{, }\FunctionTok{ifelse}\NormalTok{(p\_value\_1 }\SpecialCharTok{\textless{}}\NormalTok{ alpha, }\StringTok{"Rejeita H₀"}\NormalTok{, }\StringTok{"Não rejeita H₀"}\NormalTok{), }\StringTok{"}\SpecialCharTok{\textbackslash{}n}\StringTok{"}\NormalTok{)}
\FunctionTok{cat}\NormalTok{(}\StringTok{"  Interpretação:"}\NormalTok{, }\FunctionTok{ifelse}\NormalTok{(p\_value\_1 }\SpecialCharTok{\textless{}}\NormalTok{ alpha, }
                              \StringTok{"Evidência significativa de perda de doçura"}\NormalTok{, }
                              \StringTok{"Não há evidência significativa de perda de doçura"}\NormalTok{), }\StringTok{"}\SpecialCharTok{\textbackslash{}n\textbackslash{}n}\StringTok{"}\NormalTok{)}

\FunctionTok{cat}\NormalTok{(}\StringTok{"REFRIGERANTE 2:}\SpecialCharTok{\textbackslash{}n}\StringTok{"}\NormalTok{)}
\FunctionTok{cat}\NormalTok{(}\StringTok{"  Média amostral: x̄ ="}\NormalTok{, x\_bar\_2, }\StringTok{"}\SpecialCharTok{\textbackslash{}n}\StringTok{"}\NormalTok{)}
\FunctionTok{cat}\NormalTok{(}\StringTok{"  Estatística Z: z ="}\NormalTok{, }\FunctionTok{round}\NormalTok{(z\_2, }\DecValTok{3}\NormalTok{), }\StringTok{"}\SpecialCharTok{\textbackslash{}n}\StringTok{"}\NormalTok{)}
\FunctionTok{cat}\NormalTok{(}\StringTok{"  Valor P:"}\NormalTok{, }\FunctionTok{round}\NormalTok{(p\_value\_2, }\DecValTok{4}\NormalTok{), }\StringTok{"}\SpecialCharTok{\textbackslash{}n}\StringTok{"}\NormalTok{)}
\FunctionTok{cat}\NormalTok{(}\StringTok{"  Conclusão:"}\NormalTok{, }\FunctionTok{ifelse}\NormalTok{(p\_value\_2 }\SpecialCharTok{\textless{}}\NormalTok{ alpha, }\StringTok{"Rejeita H₀"}\NormalTok{, }\StringTok{"Não rejeita H₀"}\NormalTok{), }\StringTok{"}\SpecialCharTok{\textbackslash{}n}\StringTok{"}\NormalTok{)}
\FunctionTok{cat}\NormalTok{(}\StringTok{"  Interpretação:"}\NormalTok{, }\FunctionTok{ifelse}\NormalTok{(p\_value\_2 }\SpecialCharTok{\textless{}}\NormalTok{ alpha, }
                              \StringTok{"Evidência significativa de perda de doçura"}\NormalTok{, }
                              \StringTok{"Não há evidência significativa de perda de doçura"}\NormalTok{), }\StringTok{"}\SpecialCharTok{\textbackslash{}n}\StringTok{"}\NormalTok{)}
\InformationTok{\textasciigrave{}\textasciigrave{}\textasciigrave{}}
\end{Highlighting}
\end{Shaded}

\begin{verbatim}
Desvio-padrão da distribuição amostral: 0.3162 

Estatísticas de teste:
Refrigerante 1: x̄ = 0.3 , z = 0.949
Refrigerante 2: x̄ = 1.02 , z = 3.226 

Valores P (teste unilateral):
Refrigerante 1: P = 0.1714
Refrigerante 2: P = 6e-04 

Região crítica:
z crítico = 1.645
x̄ crítico = 0.52 

 = = = = = = = = = = = = = = = = = = = = = = = = = = = = = = = = = = = = = = = = = = = = = = = = = = 
RESUMO DOS RESULTADOS
= = = = = = = = = = = = = = = = = = = = = = = = = = = = = = = = = = = = = = = = = = = = = = = = = = 
Teste: H₀: μ = 0 vs H₁: μ > 0 (unilateral)
Nível de significância: α = 0.05 
Valor crítico: z = 1.645 , x̄ = 0.52 

REFRIGERANTE 1:
  Média amostral: x̄ = 0.3 
  Estatística Z: z = 0.949 
  Valor P: 0.1714 
  Conclusão: Não rejeita H₀ 
  Interpretação: Não há evidência significativa de perda de doçura 

REFRIGERANTE 2:
  Média amostral: x̄ = 1.02 
  Estatística Z: z = 3.226 
  Valor P: 6e-04 
  Conclusão: Rejeita H₀ 
  Interpretação: Evidência significativa de perda de doçura 
\end{verbatim}

\pandocbounded{\includegraphics[keepaspectratio]{cap17-LO-testes-significancia-basico_files/figure-pdf/unnamed-chunk-1-1.pdf}}

\pandocbounded{\includegraphics[keepaspectratio]{cap17-LO-testes-significancia-basico_files/figure-pdf/unnamed-chunk-1-2.pdf}}

\end{tcolorbox}

\subsection{Aplique seu conhecimento}\label{aplique-seu-conhecimento-2}

\begin{tcolorbox}[enhanced jigsaw, arc=.35mm, opacitybacktitle=0.6, colframe=quarto-callout-tip-color-frame, titlerule=0mm, leftrule=.75mm, left=2mm, colbacktitle=quarto-callout-tip-color!10!white, breakable, toprule=.15mm, bottomtitle=1mm, opacityback=0, coltitle=black, title=\textcolor{quarto-callout-tip-color}{\faLightbulb}\hspace{0.5em}{Exercício 17.1 - GMAT}, rightrule=.15mm, bottomrule=.15mm, toptitle=1mm, colback=white]

O \emph{Graduate Management Admission Test} (GMAT) é feito por
indivíduos interessados em seguir a educação na graduação em
administração. Os escores do GMAT são utilizados como parte do processo
de admissão para mais de 6 mil programas de graduação em administração
em todo o mundo. O escore médio para todos os que fazem o teste é 563,
com um desvio-padrão \(\sigma\) de 118.

Uma pesquisadora nas Filipinas está preocupada com o desempenho no GMAT
de não graduados nas Filipinas. Ela acredita que, na época, o escore
médio para os alunos de último ano de faculdades nas Filipinas, que
estão interessados em seguir a educação na graduação em administração,
será menor do que 563. Ela tem uma amostra aleatória de 250 alunos de
último ano de faculdades nas Filipinas interessados em seguir a educação
na graduação em administração que vão fazer o GMAT. Suponha que saibamos
que os escores GMAT são distribuídos Normalmente, com desvio-padrão
\(\sigma = 118\).

\begin{enumerate}
\def\labelenumi{(\alph{enumi})}
\item
  Procuramos evidência contra a afirmativa de que \(\mu = 563\). Qual é
  a distribuição amostral do escore médio \(\bar{x}\) de uma amostra de
  250 estudantes, se a afirmativa é verdadeira? Esboce a curva de
  densidade dessa distribuição.
\item
  Suponha que os dados amostrais resultem em \(\bar{x} = 555\). Marque
  esse ponto no eixo de seu esboço.
\item
  Suponha que os dados amostrais resultem em \(\bar{x} = 540\). Marque
  esse ponto em seu esboço. Usando seu esboço, explique, em linguagem
  simples, por que um resultado é boa evidência de que o escore médio de
  todos os estudantes de último ano de faculdades nas Filipinas,
  interessados em fazer a graduação em administração e que planejam
  fazer o GMAT, seria menor do que 563, e o outro resultado não é.
\end{enumerate}

\end{tcolorbox}

\section{Estabelecimento de hipóteses}\label{sec-hipoteses}

Um teste estatístico de significância começa com um enunciado cuidadoso
das afirmativas que queremos comparar. No Exemplo 17.3, vimos que os
dados de teste de sabor não são plausíveis se, de fato, o novo
refrigerante não perde doçura. Como a lógica dos testes procura por
evidência contrária à afirmativa, começamos com a afirmativa contra a
qual buscamos evidência, tal qual ``nenhuma perda de doçura''.

\subsection{Hipóteses nula e
alternativa}\label{hipuxf3teses-nula-e-alternativa}

\begin{tcolorbox}[enhanced jigsaw, arc=.35mm, opacitybacktitle=0.6, colframe=quarto-callout-important-color-frame, titlerule=0mm, leftrule=.75mm, left=2mm, colbacktitle=quarto-callout-important-color!10!white, breakable, toprule=.15mm, bottomtitle=1mm, opacityback=0, coltitle=black, title=\textcolor{quarto-callout-important-color}{\faExclamation}\hspace{0.5em}{Definições importantes}, rightrule=.15mm, bottomrule=.15mm, toptitle=1mm, colback=white]

A afirmativa testada por um teste estatístico de significância é chamada
de \textbf{hipótese nula}. O teste é planejado para avaliar a força da
evidência contra a hipótese nula. Usualmente, a hipótese nula é uma
afirmativa de ``nenhum efeito'' ou ``nenhuma diferença''.

A afirmativa sobre a população para a qual estamos tentando encontrar
evidência a favor é a \textbf{hipótese alternativa}. A hipótese
alternativa é \textbf{unilateral} se afirmar que um parâmetro é maior do
que ou menor do que o valor da hipótese nula. Ela é \textbf{bilateral}
se afirmar que o parâmetro é diferente do valor nulo.

Abrevia-se a hipótese nula como \(H_0\) e a hipótese alternativa como
\(H_a\). \textbf{\emph{As hipóteses sempre se referem a um parâmetro
populacional}}, \textbf{\emph{não a um resultado amostral particular}}.
Certifique-se de \textbf{\emph{estabelecer}} \(H_0\) e \(H_a\) em termos
de parâmetros da população.

\end{tcolorbox}

Como \(H_a\) expressa o efeito a favor do qual esperamos encontrar
evidência, é frequentemente mais fácil começar pelo enunciado de \(H_a\)
e, então, enunciar \(H_0\) como uma afirmativa de que o efeito esperado
não esteja presente. \(H_0\), em geral, inclui ``igual''.

Nos Exemplos 17.2 e 17.3, estamos buscando evidência a favor de perda na
doçura. A hipótese nula diz ``nenhuma perda'' em média em uma grande
população de provadores. A hipótese alternativa diz ``há uma perda''.
Logo, as hipóteses são:

\[H_0: \mu = 0\] \[H_a: \mu > 0\]

A \textbf{\emph{hipótese alternativa}} é \textbf{\emph{unilateral}}
porque estamos interessados apenas em saber se o refrigerante perdeu
doçura.

\begin{tcolorbox}[enhanced jigsaw, arc=.35mm, opacitybacktitle=0.6, colframe=quarto-callout-note-color-frame, titlerule=0mm, leftrule=.75mm, left=2mm, colbacktitle=quarto-callout-note-color!10!white, breakable, toprule=.15mm, bottomtitle=1mm, opacityback=0, coltitle=black, title=\textcolor{quarto-callout-note-color}{\faInfo}\hspace{0.5em}{Exemplo 17.4 - Estudo da satisfação no emprego}, rightrule=.15mm, bottomrule=.15mm, toptitle=1mm, colback=white]

A satisfação no emprego de operários de montadoras difere quando seu
trabalho é ritmado pelas máquinas em vez de autorritmado? Aloque
trabalhadores a uma linha de montagem que se move em um ritmo fixo ou a
uma condição autorritmada. Todos os sujeitos trabalham em ambas as
condições, em ordem aleatória. Esse é um planejamento de dados
emparelhados.

Após 2 semanas em cada condição de trabalho, os trabalhadores são
submetidos a um teste de satisfação com o emprego. A variável de
resposta é a diferença entre os escores de satisfação, autorritmado
menos ritmado pela máquina.

O parâmetro de interesse é a média \(\mu\) das diferenças dos escores na
população de todos os operários da montadora. A hipótese nula diz que
não há diferença entre trabalho autorritmado e ritmado pela máquina:

\[H_0: \mu = 0\]

Os autores do estudo queriam saber se as duas condições de trabalho
geravam níveis diferentes de satisfação no emprego. Eles não
especificaram a direção da diferença. A hipótese alternativa é,
portanto, bilateral:

\[H_a: \mu \neq 0\]

\end{tcolorbox}

\begin{tcolorbox}[enhanced jigsaw, arc=.35mm, opacitybacktitle=0.6, colframe=quarto-callout-warning-color-frame, titlerule=0mm, leftrule=.75mm, left=2mm, colbacktitle=quarto-callout-warning-color!10!white, breakable, toprule=.15mm, bottomtitle=1mm, opacityback=0, coltitle=black, title=\textcolor{quarto-callout-warning-color}{\faExclamationTriangle}\hspace{0.5em}{Estatística no mundo real - Hipóteses honestas?}, rightrule=.15mm, bottomrule=.15mm, toptitle=1mm, colback=white]

Pessoas chinesas e japonesas, para as quais o número 4 é de má sorte,
morrem mais frequentemente no quarto dia do mês do que em outros dias.
Os autores de um estudo fizeram um teste estatístico da afirmativa de
que o quarto dia tem mais mortes do que os outros dias, e encontraram
boa evidência a favor dessa afirmativa.

Você acredita nisso? Não, se os autores examinaram todos os dias,
tomaram o que tinha mais mortes e, então, fizeram a afirmativa a ser
testada ``esse dia é diferente''. Um crítico levantou esse problema, e
os autores replicaram, ``Não, nós tínhamos o dia 4 em mente antes, de
modo que nosso teste é legítimo''.

\textbf{As hipóteses devem expressar as expectativas ou suspeitas que
temos antes de vermos os dados.} É \textbf{trapaça} olhar primeiro os
dados e então estabelecer hipóteses que se ajustem ao que os dados
mostram.

\end{tcolorbox}

\section{Valor P e significância estatística}\label{sec-valor-p}

A ideia do estabelecimento de uma hipótese nula, contra a qual desejamos
encontrar evidência, parece estranha no início. Pode ser útil pensar em
um julgamento criminal. O acusado é ``inocente até que se prove o
contrário''. É exatamente assim que funcionam os testes estatísticos de
significância, embora, em estatística, lidemos com evidência fornecida
por dados e usemos a probabilidade para dizer quão forte é a evidência.

A probabilidade que mede a força da evidência contra a hipótese nula é
chamada de \textbf{valor P}.

\begin{tcolorbox}[enhanced jigsaw, arc=.35mm, opacitybacktitle=0.6, colframe=quarto-callout-important-color-frame, titlerule=0mm, leftrule=.75mm, left=2mm, colbacktitle=quarto-callout-important-color!10!white, breakable, toprule=.15mm, bottomtitle=1mm, opacityback=0, coltitle=black, title=\textcolor{quarto-callout-important-color}{\faExclamation}\hspace{0.5em}{Estatística de teste e valor P}, rightrule=.15mm, bottomrule=.15mm, toptitle=1mm, colback=white]

Uma \textbf{\emph{estatística de teste}} calculada a partir de dados
amostrais \textbf{\emph{mede quanto os dados divergem do que
esperaríamos}}, \textbf{\emph{se a hipótese}} \textbf{\emph{nula}}
\(H_0\) fosse verdadeira.

Valores não usualmente grandes da estatística mostram que os dados não
são consistentes com \(H_0\).

A probabilidade, calculada supondo \(H_0\) verdadeira, de que a
estatística de teste assuma um valor tão ou mais extremo do que o valor
realmente observado é chamada de \textbf{valor P} do teste.

\textbf{Quanto menor o valor P, mais forte é a evidência contra} \(H_0\)
fornecida pelos dados.

\textbf{Valores P pequenos são evidência contra} \(H_0\), pois
\textbf{\emph{afirmam}} \textbf{\emph{que o resultado observado tem
ocorrência improvável se}} \(H_0\) for \textbf{\emph{verdadeira}}.

\textbf{Valores P grandes não fornecem evidência contra} \(H_0\).

\end{tcolorbox}

Quão pequeno deve ser o valor P para ser evidência convincente contra
\(H_0\)? Muitos usuários de estatística consideram valores menores do
que 0,05 ou 0,01 como convincentes.

\begin{tcolorbox}[enhanced jigsaw, arc=.35mm, opacitybacktitle=0.6, colframe=quarto-callout-note-color-frame, titlerule=0mm, leftrule=.75mm, left=2mm, colbacktitle=quarto-callout-note-color!10!white, breakable, toprule=.15mm, bottomtitle=1mm, opacityback=0, coltitle=black, title=\textcolor{quarto-callout-note-color}{\faInfo}\hspace{0.5em}{Exemplo 17.5 - Adoçante de refrigerantes: valor P unilateral}, rightrule=.15mm, bottomrule=.15mm, toptitle=1mm, colback=white]

O estudo da perda de doçura nos Exemplos 17.2 e 17.3 testa as seguintes
hipóteses:

\[H_0: \mu = 0\] \[H_a: \mu > 0\]

Como a hipótese alternativa diz que \(\mu > 0\), valores de \(\bar{x}\)
maiores do que 0 favorecem \(H_a\) em detrimento de \(H_0\). A
estatística de teste compara o \(\bar{x}\) observado com o valor da
hipótese \(\mu = 0\). Por enquanto, vamos nos concentrar no valor P.

O experimento apresentado nos Exemplos 17.2 e 17.3 realmente comparava
dois refrigerantes. Para o primeiro refrigerante, os 10 provadores
encontraram uma perda média de doçura de \(\bar{x} = 0,3\). Para o
segundo, os dados forneceram \(\bar{x} = 1,02\).

O valor P para cada teste é a probabilidade de obter um \(\bar{x}\)
desse tamanho quando a perda média de doçura é realmente \(\mu = 0\).

A curva Normal é a distribuição amostral de \(\bar{x}\) quando a
hipótese nula \(H_0: \mu = 0\) é verdadeira, usando o desvio-padrão
populacional \(\sigma = 1\).

Um cálculo de probabilidade Normal mostra que o valor P é
\(P(\bar{x} \geq 0,3) = 0,1714\).

Um valor tão grande quanto \(\bar{x} = 0,3\) apareceria por acaso em
17\% de todas as amostras, quando \(H_0: \mu = 0\) fosse verdadeira.
Assim, a observação de \(\bar{x} = 0,3\) \textbf{não é evidência} forte
contra \(H_0\).

Por outro lado, pode-se verificar que a probabilidade de que \(\bar{x}\)
seja 1,02 ou maior, quando de fato \(\mu = 0\), é de apenas 0,0006. Ou
seja, raramente observaríamos uma perda média de doçura de 1,02 ou maior
se \(H_0\) fosse verdadeira. Esse \textbf{valor P pequeno} fornece forte
\textbf{evidência contra} \(H_0\) e a favor da alternativa
\(H_a: \mu > 0\).

\end{tcolorbox}

A hipótese alternativa estabelece a direção que conta como evidência
contra \(H_0\). No Exemplo 17.5, apenas valores grandes, positivos,
contam, porque a alternativa é unilateral do lado mais alto. Se a
alternativa for bilateral, ambas as direções contam.

\begin{tcolorbox}[enhanced jigsaw, arc=.35mm, opacitybacktitle=0.6, colframe=quarto-callout-note-color-frame, titlerule=0mm, leftrule=.75mm, left=2mm, colbacktitle=quarto-callout-note-color!10!white, breakable, toprule=.15mm, bottomtitle=1mm, opacityback=0, coltitle=black, title=\textcolor{quarto-callout-note-color}{\faInfo}\hspace{0.5em}{Exemplo 17.6 - Satisfação no emprego: valor P bilateral}, rightrule=.15mm, bottomrule=.15mm, toptitle=1mm, colback=white]

O estudo sobre satisfação no emprego no Exemplo 17.4 requer que
testemos:

\[H_0: \mu = 0\] \[H_a: \mu \neq 0\]

Suponha que saibamos que as diferenças nos escores de satisfação
(autorritmado menos ritmado pela máquina) na população de todos os
trabalhadores sigam uma distribuição Normal, com desvio-padrão
\(\sigma = 60\).

Dados de 18 trabalhadores fornecem \(\bar{x} = 17\). Isto é, esses
trabalhadores preferem, na média, o ambiente autorritmado. Como a
\textbf{alternativa é bilateral}, o valor P é a probabilidade de obter
\(\bar{x}\) pelo menos tão distante de \(\mu = 0\), em ambas as
direções, quanto o valor observado \(\bar{x} = 17\).

O valor P é \(P = 0,2293\). Valores tão distantes de 0 quanto
\(\bar{x} = 17\) (em qualquer direção) aconteceriam 23\% das vezes,
quando a verdadeira média populacional é \(\mu = 0\).

Um resultado que ocorreria tão frequentemente quando \(H_0\) é
verdadeira \textbf{não é} \textbf{boa evidência contra} \(H_0\).

\end{tcolorbox}

\begin{tcolorbox}[enhanced jigsaw, arc=.35mm, opacitybacktitle=0.6, colframe=quarto-callout-warning-color-frame, titlerule=0mm, leftrule=.75mm, left=2mm, colbacktitle=quarto-callout-warning-color!10!white, breakable, toprule=.15mm, bottomtitle=1mm, opacityback=0, coltitle=black, title=\textcolor{quarto-callout-warning-color}{\faExclamationTriangle}\hspace{0.5em}{Importante sobre interpretação}, rightrule=.15mm, bottomrule=.15mm, toptitle=1mm, colback=white]

\textbf{A conclusão do Exemplo 17.6 não é que} \(H_0\) seja verdadeira.

O estudo procurou evidência contrária a \(H_0: \mu = 0\) e não conseguiu
encontrar uma forte evidência. É tudo o que podemos dizer. Sem dúvida, a
média \(\mu\) para a população de todos os trabalhadores da montadora
não é exatamente igual a 0. Uma amostra suficientemente grande
forneceria evidência da diferença, mesmo que fosse muito pequena.

Testes de significância avaliam a evidência contra \(H_0\). \textbf{Se}
a \textbf{evidência} é forte, \textbf{podemos confiantemente rejeitar}
\(H_0\) em \textbf{favor da alternativa}.

O fato de não conseguir encontrar evidência contra \(H_0\) significa
apenas que os \textbf{\emph{dados não são}} \textbf{\emph{inconsistentes
com}} \(H_0\), e \textbf{\emph{não que tenhamos uma evidência clara de
que}} \(H_0\) seja verdadeira.

\textbf{Apenas dados que são inconsistentes com} \(H_0\) nos permitem
fazer uma afirmativa positiva de que temos forte \textbf{evidência
contra} \(H_0\).

\end{tcolorbox}

\subsection{Significância
estatística}\label{significuxe2ncia-estatuxedstica}

Nos Exemplos 17.5 e 17.6, decidimos que o valor P \textbf{P = 0,0006}
era \textbf{evidência} forte contra a hipótese nula e que os
\textbf{valores P = 0,1714 e P = 0,2293 não eram evidência convincente}.

Não há uma regra sobre quão pequeno um valor P deva ser para que
rejeitemos \(H_0\); é \textbf{\emph{uma questão de julgamento}} e
depende das circunstâncias específicas.

No entanto, podemos comparar um valor P com alguns valores fixos que
comumente são utilizados como \textbf{padrões para evidência} contra
\(H_0\).

\begin{tcolorbox}[enhanced jigsaw, arc=.35mm, opacitybacktitle=0.6, colframe=quarto-callout-important-color-frame, titlerule=0mm, leftrule=.75mm, left=2mm, colbacktitle=quarto-callout-important-color!10!white, breakable, toprule=.15mm, bottomtitle=1mm, opacityback=0, coltitle=black, title=\textcolor{quarto-callout-important-color}{\faExclamation}\hspace{0.5em}{Significância estatística}, rightrule=.15mm, bottomrule=.15mm, toptitle=1mm, colback=white]

Se o valor P é tão pequeno quanto \(\alpha\), ou menor do que
\(\alpha\), dizemos que os dados são \textbf{estatisticamente
significantes} no nível \(\alpha\). A quantidade \(\alpha\) é chamada de
\textbf{nível de significância}.

``Significante'', em linguagem estatística, não tem o sentido de
``importante''. \textbf{Significa} simplesmente \textbf{``improvável de
acontecer apenas por acaso''}. O \textbf{nível de significância}
\(\alpha\) torna ``improvável'' mais exato.

\end{tcolorbox}

Os valores fixos mais comuns são 0,05 e 0,01. Se P = 0,05, não há mais
do que uma chance em 20 de que uma amostra dê evidência tão forte apenas
por acaso, quando \(H_0\) é realmente verdadeira. Se P = 0,01, temos um
resultado que, no longo prazo, aconteceria não mais do que uma vez em
100 amostras, se \(H_0\) fosse verdadeira.

Para evitar confusão, usaremos \textbf{``estatisticamente
significante''} em vez de ``significante'' neste capítulo. No entanto,
em artigos de pesquisa e publicações da mídia, você geralmente verá a
palavra ``significante'' em vez da expressão ``estatisticamente
significante''.

É \textbf{boa prática interpretar as descobertas de significância}
\textbf{estatística no contexto do problema} para o qual os dados foram
coletados.

\begin{tcolorbox}[enhanced jigsaw, arc=.35mm, opacitybacktitle=0.6, colframe=quarto-callout-warning-color-frame, titlerule=0mm, leftrule=.75mm, left=2mm, colbacktitle=quarto-callout-warning-color!10!white, breakable, toprule=.15mm, bottomtitle=1mm, opacityback=0, coltitle=black, title=\textcolor{quarto-callout-warning-color}{\faExclamationTriangle}\hspace{0.5em}{Estatística no mundo real - Significância derruba um novo medicamento}, rightrule=.15mm, bottomrule=.15mm, toptitle=1mm, colback=white]

A companhia farmacêutica Pfizer gastou US\$ 1 bilhão no desenvolvimento
de uma nova droga contra o colesterol. A \textbf{verificação final} de
sua \textbf{eficácia} foi um teste clínico com 15 mil sujeitos. Para
reforçar o estudo duplo-cego, apenas um grupo independente de
especialistas viu os dados durante o teste. Após 3 anos de testes, os
\textbf{monitores declararam que houve um número excessivo,}
\textbf{estatisticamente significante, de mortes e de problemas
cardíacos no grupo} \textbf{alocado à nova droga}. A Pfizer encerrou o
teste.

\end{tcolorbox}

\section{Testes para uma média populacional}\label{sec-testes-media}

Usamos testes para hipóteses sobre a média \(\mu\) de uma população, sob
as ``condições simples'', para introduzir os testes de significância. O
importante é a lógica de um teste: dados amostrais que ocorreriam
raramente se a hipótese nula \(H_0\) fosse verdadeira fornecem evidência
de que \(H_0\) não é verdadeira.

O valor P nos dá uma probabilidade para medir ``ocorreriam raramente''.

Na prática, os passos para a realização de um teste de significância
refletem o processo geral de \textbf{quatro passos} para a
\textbf{organização} \textbf{de problemas estatísticos realistas}.

\begin{tcolorbox}[enhanced jigsaw, arc=.35mm, opacitybacktitle=0.6, colframe=quarto-callout-important-color-frame, titlerule=0mm, leftrule=.75mm, left=2mm, colbacktitle=quarto-callout-important-color!10!white, breakable, toprule=.15mm, bottomtitle=1mm, opacityback=0, coltitle=black, title=\textcolor{quarto-callout-important-color}{\faExclamation}\hspace{0.5em}{Testes de significância: o processo de quatro passos}, rightrule=.15mm, bottomrule=.15mm, toptitle=1mm, colback=white]

\textbf{ESTABELEÇA}: qual é a questão prática que requer um teste
estatístico?

\textbf{PLANEJE}: identifique o parâmetro, estabeleça as hipóteses nula
e alternativa e escolha o tipo de teste que seja adequado à sua
situação.

\textbf{RESOLVA}: realize o \textbf{teste} em \textbf{três fases}: 1.
\textbf{\emph{Verifique as condições}} para o teste que você planeja
usar. 2. Calcule a \textbf{\emph{estatística de teste}}. 3. Encontre o
\textbf{\emph{valor P}}.

\textbf{CONCLUA}: volte à questão prática para \textbf{\emph{descrever
seus resultados}} \textbf{\emph{nesse contexto}}.

\end{tcolorbox}

Após estabelecer o problema, enunciar as hipóteses e verificar as
condições para seu teste, você ou um programa de computador podem
encontrar a estatística de teste e o valor P seguindo um roteiro. Esse é
o roteiro para o teste que usamos em nossos exemplos.

\begin{tcolorbox}[enhanced jigsaw, arc=.35mm, opacitybacktitle=0.6, colframe=quarto-callout-important-color-frame, titlerule=0mm, leftrule=.75mm, left=2mm, colbacktitle=quarto-callout-important-color!10!white, breakable, toprule=.15mm, bottomtitle=1mm, opacityback=0, coltitle=black, title=\textcolor{quarto-callout-important-color}{\faExclamation}\hspace{0.5em}{Teste z de uma amostra para uma média populacional}, rightrule=.15mm, bottomrule=.15mm, toptitle=1mm, colback=white]

Extraia uma AAS de tamanho n de uma população Normal que tenha média
\(\mu\) desconhecida e desvio-padrão \(\sigma\) conhecido. Para testar a
hipótese nula de que \(\mu\) tenha um valor especificado:

\[H_0: \mu = \mu_0\]

calcule a \textbf{estatística de teste z de uma amostra}:

\[z = \frac{\bar{x} - \mu_0}{\sigma / \sqrt{n}}\]

\textbf{Em termos de uma variável Z com distribuição Normal padrão},
\textbf{o valor P para um teste de} \(H_0\) contra:

\begin{itemize}
\tightlist
\item
  \(H_a: \mu > \mu_0\) é \(P(Z \geq z)\)
\item
  \(H_a: \mu < \mu_0\) é \(P(Z \leq z)\)\\
\item
  \(H_a: \mu \neq \mu_0\) é \(2P(Z \geq |z|)\)
\end{itemize}

\end{tcolorbox}

\begin{tcolorbox}[enhanced jigsaw, arc=.35mm, opacitybacktitle=0.6, colframe=quarto-callout-note-color-frame, titlerule=0mm, leftrule=.75mm, left=2mm, colbacktitle=quarto-callout-note-color!10!white, breakable, toprule=.15mm, bottomtitle=1mm, opacityback=0, coltitle=black, title=\textcolor{quarto-callout-note-color}{\faInfo}\hspace{0.5em}{Exemplo 17.7 - Colesterol de executivos}, rightrule=.15mm, bottomrule=.15mm, toptitle=1mm, colback=white]

\textbf{ESTABELEÇA}: o \emph{National Center for Health Statistics}
relata que o colesterol LDL para adultos tem média 130 e desvio-padrão
\(\sigma = 40\). O diretor médico de uma grande companhia farmacêutica
observa os registros médicos de 72 executivos e vê que o LDL médio nessa
amostra é \(\bar{x} = 124,86\). \textbf{\emph{Isso}} \textbf{\emph{é
evidência}} de que os executivos da companhia tenham um
\textbf{\emph{LDL médio diferente}} do da \textbf{\emph{população}}
geral?

\textbf{PLANEJE}: a hipótese nula é ``nenhuma diferença'' da média
nacional \(\mu_0 = 130\). A \textbf{alternativa é bilateral}, porque o
diretor médico não tinha em mente uma direção particular antes de
examinar os dados. Assim, as hipóteses acerca da média desconhecida
\(\mu\) da população de executivos são:

\[H_0: \mu = 130\] \[H_a: \mu \neq 130\]

Sabemos que o teste z de uma amostra é apropriado para essas hipóteses
sob as ``condições simples''.

\textbf{RESOLVA}: como parte das ``condições simples'', suponha que
estejamos desejosos em assumir que o LDL de executivos siga uma
distribuição Normal, com desvio-padrão \(\sigma = 40\). A estatística de
teste é:

\[z = \frac{\bar{x} - \mu_0}{\sigma/\sqrt{n}} = \frac{124.86 - 130}{40/\sqrt{72}} = -1.09\]

Para ajudar a determinar o \textbf{valor P}, \textbf{esboce} a
\textbf{curva Normal} \textbf{padrão} e marque nela o \textbf{valor
observado} de \textbf{z}.

O \textbf{valor P é a probabilidade} de que \textbf{uma variável}
\textbf{Normal padrão Z assuma um valor distante de zero} em,
\textbf{pelo} \textbf{menos, 1,09}.

\[P = 2P(Z > 1.09) = 2(0.1379) = 0.2758\]

\textbf{CONCLUA}: mais de 27\% das vezes, uma AAS de tamanho 72 da
população adulta em geral teria um LDL médio pelo menos tão longe de 130
quanto o da amostra de executivos. \textbf{O} \(\bar{x} = 124,86\)
observado não é, portanto, \textbf{boa evidência} de \textbf{que os
executivos sejam diferentes dos outros adultos}.

\end{tcolorbox}

Neste capítulo, estamos agindo como se as ``condições simples''
estabelecidas em ``Condições simples para inferência sobre uma média'',
no Capítulo 16, fossem verdadeiras. Na \textbf{prática}, \textbf{você
deve verificar essas condições}.

\begin{enumerate}
\def\labelenumi{\arabic{enumi}.}
\item
  \textbf{AAS}: a condição mais importante é que os 72 executivos na
  \textbf{amostra} sejam \textbf{uma AAS} da \textbf{população de todos
  os executivos} \textbf{na empresa}. Devemos conferir essa exigência
  questionando como os dados foram produzidos.
\item
  \textbf{Distribuição Normal}: devemos examinar, também, a
  \textbf{distribuição} \textbf{das 72 observações} à \textbf{procura}
  de \textbf{sinais} de que a \textbf{distribuição populacional não seja
  Normal}.
\item
  \(\sigma\) conhecido: é, de fato, \textbf{não realista} supor que
  saibamos que \(\sigma = 40\). Veremos, no Capítulo 20, que é
  \textbf{fácil} nos \textbf{livrarmos} da \textbf{necessidade} de
  \textbf{conhecer} \(\sigma\).
\end{enumerate}

\section{Resumo}\label{sec-resumo}

\begin{itemize}
\item
  \textbf{Um teste de significância avalia a evidência fornecida pelos
  dados contra uma hipótese nula} \(H_0\) em favor de uma hipótese
  alternativa \(H_a\).
\item
  As \textbf{hipóteses são sempre enunciadas em termos de parâmetros
  populacionais}. Em geral, \(H_0\) é uma afirmativa de que não há
  qualquer efeito presente, e \(H_a\) afirma que um parâmetro diverge de
  seu valor nulo em uma direção específica (alternativa
  \textbf{unilateral}) \textbf{ou em qualquer direção} (alternativa
  \textbf{bilateral}).
\item
  O \textbf{fundamento essencial} de um teste de significância é como
  segue. \textbf{Suponha}, para raciocinar, que a \textbf{hipótese nula
  seja} \textbf{verdadeira}. \textbf{Se repetíssemos nossa produção de
  dados muitas} \textbf{vezes, obteríamos frequentemente dados tão
  inconsistentes} \textbf{com} \(H_0\) como os dados que realmente
  temos?
\item
  Um teste se baseia em \textbf{uma estatística de teste}, que
  \textbf{mede quão} \textbf{distante o resultado amostral está do valor
  estabelecido por} \(H_0\).
\item
  O \textbf{valor P} de um teste é a \textbf{probabilidade}, calculada
  \textbf{supondo} \(H_0\) verdadeira, de \textbf{que a estatística de
  teste assuma um valor} \textbf{pelo menos tão extremo quanto o de fato
  observado}.
\item
  \textbf{Se} o \textbf{valor P} for \textbf{tão pequeno quanto, ou
  menor que um} \textbf{valor especificado} \(\alpha\), \textbf{os dados
  são estatisticamente significantes} \textbf{no nível de significância}
  \(\alpha\).
\item
  Testes de significância para a hipótese nula \(H_0: \mu = \mu_0\)
  relativos à média desconhecida \(\mu\) de uma população se baseiam
  \textbf{na estatística de teste z de uma amostra}:
\end{itemize}

\[z = \frac{\bar{x} - \mu_0}{\sigma / \sqrt{n}}\]

O teste z \textbf{pressupõe} uma \textbf{AAS} de tamanho \textbf{n} de
uma \textbf{população Normal} com desvio-padrão populacional \(\sigma\)
conhecido.

\section{Exercícios selecionados}\label{sec-exercicios}

\begin{tcolorbox}[enhanced jigsaw, arc=.35mm, opacitybacktitle=0.6, colframe=quarto-callout-tip-color-frame, titlerule=0mm, leftrule=.75mm, left=2mm, colbacktitle=quarto-callout-tip-color!10!white, breakable, toprule=.15mm, bottomtitle=1mm, opacityback=0, coltitle=black, title=\textcolor{quarto-callout-tip-color}{\faLightbulb}\hspace{0.5em}{Para praticar}, rightrule=.15mm, bottomrule=.15mm, toptitle=1mm, colback=white]

\begin{enumerate}
\def\labelenumi{\arabic{enumi}.}
\item
  \textbf{GMAT}: Complete os exercícios sobre o estudo de desempenho no
  GMAT.
\item
  \textbf{Inspeção de pesos}: Trabalhe os exercícios sobre pesos de
  caixas de cookies.
\item
  \textbf{Teste de significância completo}: Realize o processo de quatro
  passos para exercícios sobre gorjetas em restaurantes.
\end{enumerate}

\end{tcolorbox}

\begin{center}\rule{0.5\linewidth}{0.5pt}\end{center}

\textbf{Referências}: Este material é baseado em Moore, D. S., Notz, W.
I., \& Fligner, M. A. \emph{A Estatística Básica e sua prática} (9ª
ed.). Tradução e adaptação para formato Quarto.

\bookmarksetup{startatroot}

\chapter{Testes de Significância:
Oltramari}\label{testes-de-significuxe2ncia-oltramari}

Tema: Controle Externo da Atividade Policial (MP-GO).

Testes de Significância: Qui-Quadrado.

Cleuler apud Oltramari (2024).

A seguir, exemplificamos sucintamente a lógica de testes estatísticos.

\begin{tcolorbox}[enhanced jigsaw, arc=.35mm, opacitybacktitle=0.6, colframe=quarto-callout-note-color-frame, titlerule=0mm, leftrule=.75mm, left=2mm, colbacktitle=quarto-callout-note-color!10!white, breakable, toprule=.15mm, bottomtitle=1mm, opacityback=0, coltitle=black, title=\textcolor{quarto-callout-note-color}{\faInfo}\hspace{0.5em}{Exemplo - Felipe Oltramari}, rightrule=.15mm, bottomrule=.15mm, toptitle=1mm, colback=white]

Trata-se de pesquisa consistente na verificação da (in)existência de
impacto da Resolução n.º 4, de 28 de março de 2022, do Colégio de
Procuradores de Justiça do Ministério Público do Estado de Goiás -- que
redistribuiu a atribuição para atuação na fase investigativa nos crimes
militares -- na tutela e garantia da cidadania e dos direitos dos
cidadãos.

Procurou-se observar, em uma pesquisa empírica, se a norma
administrativa possui a eficiência, a eficácia e a efetividade
esperadas, daí se extraindo conclusões generalizantes.

Em busca de tal desiderato, foram analisados, dentre outros documentos,
todos os \textbf{1.978 Registros de Atendimento Integrado} que
registraram \textbf{mortes decorrentes de intervenção policial} no
estado de Goiás entre os anos de \textbf{2019 e 2023}.

Também foram pesquisados e acessados \textbf{4.995 autos judiciais que
apuram a prática de crime de competência da Justiça Militar Estadual}
autuados entre \textbf{janeiro de 2017 e outubro de 2023}.

Com os dados coletados e tratados, é feita a exposição dos resultados
finais da pesquisa.

Analisamos a variação do número de crimes militares investigados no
âmbito do próprio Ministério Público; \ul{\textbf{comparamos a atuação
dos Promotores de Justiça da capital}}, que antes possuíam atuação
exclusiva tanto da investigação quanto do exercício da \emph{opinio
delicti} em matéria afeta à competência da Justiça Militar Estadual,
\ul{\textbf{com}} a \textbf{atuação dos Promotores de Justiça do
interior}; verificamos eventual existência de interferência dessa forma
de atuação, comprovando sua efetividade (ou não) \textbf{\emph{sobre o
número de mortes decorrentes de ação policial}}; analisamos o número de
agentes que se envolveram em ações com resultado morte (e também em
fatos apurados pela Justiça Militar Estadual) e a
\ul{\textbf{proporção}} entre estes e o número total de agentes ativos
lotados em funções operacionais; fizemos levantamento da
\ul{\textbf{atuação processual}} do Ministério Público em matéria de
competência da Justiça Militar e da coincidência (ou não) entre sua
análise jurídica e a do Juízo; e relacionamos, em Comarca específica, a
atuação do Ministério Público em inquéritos policiais que apuram mortes
violentas intencionais com a atuação do órgão em inquéritos policiais
que apuram morte decorrente de intervenção policial.

Após, considerando-se a data de início da vigência da \textbf{Resolução
n.º 4, de 28 de março de 2022, do Colégio de Procuradores de Justiça},
um marco temporal divisório, \textbf{confrontamos} os resultados finais
da pesquisa referente ao \textbf{período anterior} com os resultados da
pesquisa relativa ao \textbf{período posterior}, explorando, finalmente,
eventual \textbf{\emph{existência ou não de elementos que confirmam a
hipótese}}.

A contribuição desta pesquisa consiste na revelação de dados de
interesse do Ministério Público e de entrega de produto à instituição
(aplicativo Métis), aptos a contribuir no desenvolvimento de ações para
\textbf{\emph{aprimorar}} o \ul{\textbf{controle externo da atividade
policial}} e no \ul{\textbf{\emph{fomentar políticas públicas de
segurança}}} com enfoque na \ul{\textbf{\emph{redução}}} dos
\ul{\textbf{\emph{índices de violência policial no estado de Goiás}}}.
(OLTRAMARI, 2024 , p.~7)

\end{tcolorbox}

\section{\texorpdfstring{Aplicar um teste de significância da
H\textsubscript{0}}{Aplicar um teste de significância da H0}}\label{aplicar-um-teste-de-significuxe2ncia-da-h0}

\subsection{Teste Qui-Quadrado}\label{teste-qui-quadrado}

Replicar o Teste qui-quadrado da figura abaixo.

\begin{center}
\pandocbounded{\includegraphics[keepaspectratio]{fig/AJCM-QuiQuad-nAJCM-Gyn&Int-x-ano2022&2023-TesteH-Rejeita-Ho-NC=99.9.JPG}}
\end{center}

Através do seguinte código R.

\begin{Shaded}
\begin{Highlighting}[numbers=left,,]
\InformationTok{\textasciigrave{}\textasciigrave{}\textasciigrave{}\{r\}}
\CommentTok{\# TESTE QUI{-}QUADRADO: num. Autos Judiciais Crime Militar {-} AJCM x Promotoria (Gyn e Int.)}
\CommentTok{\# Vara Auditoria Militar {-} VAM}
\CommentTok{\# Replicação do gráfico e tabelas da análise estatística}
\CommentTok{\# Arquivo: AJCM{-}QuiQuad{-}nAJCM{-}Gyn\&Int{-}x{-}ano2022\&2023{-}TesteH{-}Rejeita{-}Ho{-}NC=99.9\%.JPG}

\CommentTok{\# Carregando bibliotecas necessárias}
\FunctionTok{library}\NormalTok{(ggplot2)}
\FunctionTok{library}\NormalTok{(dplyr)}
\FunctionTok{library}\NormalTok{(gridExtra)}
\FunctionTok{library}\NormalTok{(knitr)}
\FunctionTok{library}\NormalTok{(kableExtra)}

\CommentTok{\# ==============================================================================}
\CommentTok{\# DADOS OBSERVADOS {-} AJCM Gynecology \& Internal Medicine por Ano: 2022 e 2023}
\CommentTok{\# ==============================================================================}

\CommentTok{\# Baseado na imagem, criando os dados da tabela de contingência}
\CommentTok{\# (Ajuste os valores conforme a imagem específica)}
\NormalTok{dados\_observados }\OtherTok{\textless{}{-}} \FunctionTok{matrix}\NormalTok{(}\FunctionTok{c}\NormalTok{(}
  \DecValTok{304}\NormalTok{, }\DecValTok{220}\NormalTok{,   }\CommentTok{\# 2022: Gyn}
  \DecValTok{761}\NormalTok{, }\DecValTok{814}    \CommentTok{\# 2023: Inter}
\NormalTok{), }\AttributeTok{nrow =} \DecValTok{2}\NormalTok{, }\AttributeTok{byrow =} \ConstantTok{TRUE}\NormalTok{,}
\AttributeTok{dimnames =} \FunctionTok{list}\NormalTok{(}
  \AttributeTok{Ano =} \FunctionTok{c}\NormalTok{(}\StringTok{"2022"}\NormalTok{, }\StringTok{"2023"}\NormalTok{),}
  \AttributeTok{Local =} \FunctionTok{c}\NormalTok{(}\StringTok{"Gyn"}\NormalTok{, }\StringTok{"Inter"}\NormalTok{)}
\NormalTok{))}

\FunctionTok{cat}\NormalTok{(}\StringTok{"TABELA DE CONTINGÊNCIA {-} DADOS OBSERVADOS}\SpecialCharTok{\textbackslash{}n}\StringTok{"}\NormalTok{)}
\FunctionTok{print}\NormalTok{(dados\_observados)}

\CommentTok{\# Totais marginais}
\NormalTok{totais\_linha }\OtherTok{\textless{}{-}} \FunctionTok{rowSums}\NormalTok{(dados\_observados)}
\NormalTok{totais\_coluna }\OtherTok{\textless{}{-}} \FunctionTok{colSums}\NormalTok{(dados\_observados)}
\NormalTok{total\_geral }\OtherTok{\textless{}{-}} \FunctionTok{sum}\NormalTok{(dados\_observados)}

\FunctionTok{cat}\NormalTok{(}\StringTok{"}\SpecialCharTok{\textbackslash{}n}\StringTok{Totais por Ano:}\SpecialCharTok{\textbackslash{}n}\StringTok{"}\NormalTok{)}
\FunctionTok{print}\NormalTok{(totais\_linha)}
\FunctionTok{cat}\NormalTok{(}\StringTok{"}\SpecialCharTok{\textbackslash{}n}\StringTok{Totais por Local:}\SpecialCharTok{\textbackslash{}n}\StringTok{"}\NormalTok{)}
\FunctionTok{print}\NormalTok{(totais\_coluna)}
\FunctionTok{cat}\NormalTok{(}\StringTok{"}\SpecialCharTok{\textbackslash{}n}\StringTok{Total Geral:"}\NormalTok{, total\_geral, }\StringTok{"}\SpecialCharTok{\textbackslash{}n\textbackslash{}n}\StringTok{"}\NormalTok{)}

\CommentTok{\# ==============================================================================}
\CommentTok{\# CÁLCULO DAS FREQUÊNCIAS ESPERADAS}
\CommentTok{\# ==============================================================================}

\CommentTok{\# Frequências esperadas sob H₀ (independência)}
\NormalTok{freq\_esperadas }\OtherTok{\textless{}{-}} \FunctionTok{outer}\NormalTok{(totais\_linha, totais\_coluna) }\SpecialCharTok{/}\NormalTok{ total\_geral}

\FunctionTok{cat}\NormalTok{(}\StringTok{"FREQUÊNCIAS ESPERADAS (sob H₀):}\SpecialCharTok{\textbackslash{}n}\StringTok{"}\NormalTok{)}
\FunctionTok{print}\NormalTok{(}\FunctionTok{round}\NormalTok{(freq\_esperadas, }\DecValTok{2}\NormalTok{))}

\CommentTok{\# ==============================================================================}
\CommentTok{\# TESTE QUI{-}QUADRADO DE INDEPENDÊNCIA}
\CommentTok{\# ==============================================================================}

\CommentTok{\# Cálculo manual da estatística qui{-}quadrado}
\NormalTok{chi\_squared\_calc }\OtherTok{\textless{}{-}} \FunctionTok{sum}\NormalTok{((dados\_observados }\SpecialCharTok{{-}}\NormalTok{ freq\_esperadas)}\SpecialCharTok{\^{}}\DecValTok{2} \SpecialCharTok{/}\NormalTok{ freq\_esperadas)}

\CommentTok{\# Graus de liberdade}
\NormalTok{gl }\OtherTok{\textless{}{-}}\NormalTok{ (}\FunctionTok{nrow}\NormalTok{(dados\_observados) }\SpecialCharTok{{-}} \DecValTok{1}\NormalTok{) }\SpecialCharTok{*}\NormalTok{ (}\FunctionTok{ncol}\NormalTok{(dados\_observados) }\SpecialCharTok{{-}} \DecValTok{1}\NormalTok{)}

\CommentTok{\# Valor crítico para NC = 99.9\% (α = 0.001)}
\NormalTok{alpha }\OtherTok{\textless{}{-}} \FloatTok{0.001}
\NormalTok{chi\_critico }\OtherTok{\textless{}{-}} \FunctionTok{qchisq}\NormalTok{(}\DecValTok{1} \SpecialCharTok{{-}}\NormalTok{ alpha, gl)}

\CommentTok{\# Valor P}
\NormalTok{p\_value }\OtherTok{\textless{}{-}} \DecValTok{1} \SpecialCharTok{{-}} \FunctionTok{pchisq}\NormalTok{(chi\_squared\_calc, gl)}

\CommentTok{\# Teste usando função do R para verificação}
\NormalTok{teste\_chi }\OtherTok{\textless{}{-}} \FunctionTok{chisq.test}\NormalTok{(dados\_observados)}

\FunctionTok{cat}\NormalTok{(}\StringTok{"TESTE QUI{-}QUADRADO DE INDEPENDÊNCIA}\SpecialCharTok{\textbackslash{}n}\StringTok{"}\NormalTok{)}
\FunctionTok{cat}\NormalTok{(}\StringTok{"H₀: Não há associação entre Ano e Local}\SpecialCharTok{\textbackslash{}n}\StringTok{"}\NormalTok{)}
\FunctionTok{cat}\NormalTok{(}\StringTok{"H₁: Há associação entre Ano e Local}\SpecialCharTok{\textbackslash{}n}\StringTok{"}\NormalTok{)}
\FunctionTok{cat}\NormalTok{(}\StringTok{"Nível de Confiança: 99.9\% (α = 0.001)}\SpecialCharTok{\textbackslash{}n\textbackslash{}n}\StringTok{"}\NormalTok{)}

\FunctionTok{cat}\NormalTok{(}\StringTok{"Estatística qui{-}quadrado calculada:"}\NormalTok{, }\FunctionTok{round}\NormalTok{(chi\_squared\_calc, }\DecValTok{4}\NormalTok{), }\StringTok{"}\SpecialCharTok{\textbackslash{}n}\StringTok{"}\NormalTok{)}
\FunctionTok{cat}\NormalTok{(}\StringTok{"Graus de liberdade:"}\NormalTok{, gl, }\StringTok{"}\SpecialCharTok{\textbackslash{}n}\StringTok{"}\NormalTok{)}
\FunctionTok{cat}\NormalTok{(}\StringTok{"Valor crítico (α = 0.001):"}\NormalTok{, }\FunctionTok{round}\NormalTok{(chi\_critico, }\DecValTok{4}\NormalTok{), }\StringTok{"}\SpecialCharTok{\textbackslash{}n}\StringTok{"}\NormalTok{)}
\FunctionTok{cat}\NormalTok{(}\StringTok{"Valor P:"}\NormalTok{, }\FunctionTok{ifelse}\NormalTok{(p\_value }\SpecialCharTok{\textless{}} \FloatTok{0.0001}\NormalTok{, }\StringTok{"\textless{} 0.0001"}\NormalTok{, }\FunctionTok{round}\NormalTok{(p\_value, }\DecValTok{6}\NormalTok{)), }\StringTok{"}\SpecialCharTok{\textbackslash{}n}\StringTok{"}\NormalTok{)}
\FunctionTok{cat}\NormalTok{(}\StringTok{"Decisão:"}\NormalTok{, }\FunctionTok{ifelse}\NormalTok{(chi\_squared\_calc }\SpecialCharTok{\textgreater{}}\NormalTok{ chi\_critico, }\StringTok{"REJEITA H₀"}\NormalTok{, }\StringTok{"NÃO REJEITA H₀"}\NormalTok{), }\StringTok{"}\SpecialCharTok{\textbackslash{}n}\StringTok{"}\NormalTok{)}
\FunctionTok{cat}\NormalTok{(}\StringTok{"Conclusão:"}\NormalTok{, }\FunctionTok{ifelse}\NormalTok{(chi\_squared\_calc }\SpecialCharTok{\textgreater{}}\NormalTok{ chi\_critico, }
                        \StringTok{"Há evidência significativa de associação"}\NormalTok{, }
                        \StringTok{"Não há evidência significativa de associação"}\NormalTok{), }\StringTok{"}\SpecialCharTok{\textbackslash{}n\textbackslash{}n}\StringTok{"}\NormalTok{)}

\CommentTok{\# ==============================================================================}
\CommentTok{\# TABELA DETALHADA DOS CÁLCULOS}
\CommentTok{\# ==============================================================================}

\CommentTok{\# Criando tabela detalhada dos cálculos}
\NormalTok{calc\_detalhado }\OtherTok{\textless{}{-}} \FunctionTok{data.frame}\NormalTok{(}
\NormalTok{  Célula }\OtherTok{=} \FunctionTok{c}\NormalTok{(}\StringTok{"2022{-}Gyn"}\NormalTok{, }\StringTok{"2022{-}Int"}\NormalTok{, }\StringTok{"2023{-}Gyn"}\NormalTok{, }\StringTok{"2023{-}Int"}\NormalTok{),}
  \AttributeTok{Observado =} \FunctionTok{as.vector}\NormalTok{(dados\_observados),}
  \AttributeTok{Esperado =} \FunctionTok{round}\NormalTok{(}\FunctionTok{as.vector}\NormalTok{(freq\_esperadas), }\DecValTok{2}\NormalTok{),}
\NormalTok{  Diferença }\OtherTok{=} \FunctionTok{round}\NormalTok{(}\FunctionTok{as.vector}\NormalTok{(dados\_observados }\SpecialCharTok{{-}}\NormalTok{ freq\_esperadas), }\DecValTok{2}\NormalTok{),}
  \AttributeTok{Qui\_Quadrado =} \FunctionTok{round}\NormalTok{(}\FunctionTok{as.vector}\NormalTok{((dados\_observados }\SpecialCharTok{{-}}\NormalTok{ freq\_esperadas)}\SpecialCharTok{\^{}}\DecValTok{2} \SpecialCharTok{/}\NormalTok{ freq\_esperadas), }\DecValTok{4}\NormalTok{)}
\NormalTok{)}

\NormalTok{calc\_detalhado}\SpecialCharTok{$}\NormalTok{Contribuição\_Perc }\OtherTok{\textless{}{-}} \FunctionTok{round}\NormalTok{(calc\_detalhado}\SpecialCharTok{$}\NormalTok{Qui\_Quadrado }\SpecialCharTok{/}\NormalTok{ chi\_squared\_calc }\SpecialCharTok{*} \DecValTok{100}\NormalTok{, }\DecValTok{1}\NormalTok{)}

\FunctionTok{cat}\NormalTok{(}\StringTok{"CÁLCULOS DETALHADOS POR CÉLULA:}\SpecialCharTok{\textbackslash{}n}\StringTok{"}\NormalTok{)}
\FunctionTok{print}\NormalTok{(calc\_detalhado)}

\CommentTok{\# ==============================================================================}
\CommentTok{\# GRÁFICO DA DISTRIBUIÇÃO QUI{-}QUADRADO}
\CommentTok{\# ==============================================================================}

\CommentTok{\# Sequência de valores para o gráfico}
\NormalTok{x\_seq }\OtherTok{\textless{}{-}} \FunctionTok{seq}\NormalTok{(}\DecValTok{0}\NormalTok{, }\FunctionTok{max}\NormalTok{(chi\_squared\_calc }\SpecialCharTok{+} \DecValTok{2}\NormalTok{, chi\_critico }\SpecialCharTok{+} \DecValTok{2}\NormalTok{), }\AttributeTok{length.out =} \DecValTok{1000}\NormalTok{)}
\NormalTok{y\_seq }\OtherTok{\textless{}{-}} \FunctionTok{dchisq}\NormalTok{(x\_seq, gl)}

\NormalTok{df\_chi }\OtherTok{\textless{}{-}} \FunctionTok{data.frame}\NormalTok{(}\AttributeTok{x =}\NormalTok{ x\_seq, }\AttributeTok{y =}\NormalTok{ y\_seq)}

\CommentTok{\# Região de rejeição}
\NormalTok{x\_reject }\OtherTok{\textless{}{-}} \FunctionTok{seq}\NormalTok{(chi\_critico, }\FunctionTok{max}\NormalTok{(x\_seq), }\AttributeTok{length.out =} \DecValTok{100}\NormalTok{)}
\NormalTok{y\_reject }\OtherTok{\textless{}{-}} \FunctionTok{dchisq}\NormalTok{(x\_reject, gl)}
\NormalTok{df\_reject }\OtherTok{\textless{}{-}} \FunctionTok{data.frame}\NormalTok{(}\AttributeTok{x =}\NormalTok{ x\_reject, }\AttributeTok{y =}\NormalTok{ y\_reject)}

\CommentTok{\# Gráfico principal}
\NormalTok{p\_chi }\OtherTok{\textless{}{-}} \FunctionTok{ggplot}\NormalTok{(df\_chi, }\FunctionTok{aes}\NormalTok{(}\AttributeTok{x =}\NormalTok{ x, }\AttributeTok{y =}\NormalTok{ y)) }\SpecialCharTok{+}
  \FunctionTok{geom\_line}\NormalTok{(}\AttributeTok{size =} \FloatTok{1.2}\NormalTok{, }\AttributeTok{color =} \StringTok{"blue"}\NormalTok{) }\SpecialCharTok{+}
  \FunctionTok{geom\_area}\NormalTok{(}\AttributeTok{data =}\NormalTok{ df\_reject, }\FunctionTok{aes}\NormalTok{(}\AttributeTok{x =}\NormalTok{ x, }\AttributeTok{y =}\NormalTok{ y), }
            \AttributeTok{fill =} \StringTok{"red"}\NormalTok{, }\AttributeTok{alpha =} \FloatTok{0.3}\NormalTok{) }\SpecialCharTok{+}
  \FunctionTok{geom\_vline}\NormalTok{(}\AttributeTok{xintercept =}\NormalTok{ chi\_critico, }\AttributeTok{color =} \StringTok{"red"}\NormalTok{, }
             \AttributeTok{linetype =} \StringTok{"solid"}\NormalTok{, }\AttributeTok{size =} \FloatTok{1.2}\NormalTok{) }\SpecialCharTok{+}
  \FunctionTok{geom\_vline}\NormalTok{(}\AttributeTok{xintercept =}\NormalTok{ chi\_squared\_calc, }\AttributeTok{color =} \StringTok{"darkgreen"}\NormalTok{, }
             \AttributeTok{linetype =} \StringTok{"dashed"}\NormalTok{, }\AttributeTok{size =} \FloatTok{1.5}\NormalTok{) }\SpecialCharTok{+}
  \FunctionTok{geom\_point}\NormalTok{(}\FunctionTok{aes}\NormalTok{(}\AttributeTok{x =}\NormalTok{ chi\_squared\_calc, }\AttributeTok{y =} \FunctionTok{dchisq}\NormalTok{(chi\_squared\_calc, gl)), }
             \AttributeTok{color =} \StringTok{"darkgreen"}\NormalTok{, }\AttributeTok{size =} \DecValTok{4}\NormalTok{) }\SpecialCharTok{+}
  \FunctionTok{labs}\NormalTok{(}
    \AttributeTok{title =} \StringTok{"Teste Qui{-}Quadrado: Ano (2022, 2023) x Local (Gyn, Inter.)"}\NormalTok{,}
    \AttributeTok{subtitle =} \FunctionTok{paste}\NormalTok{(}\StringTok{"H₀: Independência entre Ano e Local | NC = 99.9\% | gl ="}\NormalTok{, gl),}
    \AttributeTok{x =} \StringTok{"Estatística Qui{-}Quadrado (χ²)"}\NormalTok{,}
    \AttributeTok{y =} \StringTok{"Densidade"}\NormalTok{,}
    \AttributeTok{caption =} \FunctionTok{paste}\NormalTok{(}\StringTok{"χ² ="}\NormalTok{, }\FunctionTok{round}\NormalTok{(chi\_squared\_calc, }\DecValTok{3}\NormalTok{), }
                   \StringTok{"| χ² crítico ="}\NormalTok{, }\FunctionTok{round}\NormalTok{(chi\_critico, }\DecValTok{3}\NormalTok{),}
                   \StringTok{"| Decisão:"}\NormalTok{, }\FunctionTok{ifelse}\NormalTok{(chi\_squared\_calc }\SpecialCharTok{\textgreater{}}\NormalTok{ chi\_critico, }\StringTok{"REJEITA H₀"}\NormalTok{, }\StringTok{"NÃO REJEITA H₀"}\NormalTok{))}
\NormalTok{  ) }\SpecialCharTok{+}
  \FunctionTok{annotate}\NormalTok{(}\StringTok{"text"}\NormalTok{, }\AttributeTok{x =}\NormalTok{ chi\_critico, }\AttributeTok{y =} \FunctionTok{max}\NormalTok{(y\_seq) }\SpecialCharTok{*} \FloatTok{0.8}\NormalTok{, }
           \AttributeTok{label =} \FunctionTok{paste}\NormalTok{(}\StringTok{"χ² crítico ="}\NormalTok{, }\FunctionTok{round}\NormalTok{(chi\_critico, }\DecValTok{3}\NormalTok{)), }
           \AttributeTok{angle =} \DecValTok{90}\NormalTok{, }\AttributeTok{vjust =} \SpecialCharTok{{-}}\FloatTok{0.5}\NormalTok{, }\AttributeTok{color =} \StringTok{"red"}\NormalTok{, }\AttributeTok{size =} \DecValTok{3}\NormalTok{) }\SpecialCharTok{+}
  \FunctionTok{annotate}\NormalTok{(}\StringTok{"text"}\NormalTok{, }\AttributeTok{x =}\NormalTok{ chi\_squared\_calc, }\AttributeTok{y =} \FunctionTok{max}\NormalTok{(y\_seq) }\SpecialCharTok{*} \FloatTok{0.6}\NormalTok{, }
           \AttributeTok{label =} \FunctionTok{paste}\NormalTok{(}\StringTok{"χ² observado ="}\NormalTok{, }\FunctionTok{round}\NormalTok{(chi\_squared\_calc, }\DecValTok{3}\NormalTok{)), }
           \AttributeTok{angle =} \DecValTok{90}\NormalTok{, }\AttributeTok{vjust =} \FloatTok{1.2}\NormalTok{, }\AttributeTok{color =} \StringTok{"darkgreen"}\NormalTok{, }\AttributeTok{size =} \DecValTok{3}\NormalTok{) }\SpecialCharTok{+}
  \FunctionTok{annotate}\NormalTok{(}\StringTok{"text"}\NormalTok{, }\AttributeTok{x =}\NormalTok{ (chi\_critico }\SpecialCharTok{+} \FunctionTok{max}\NormalTok{(x\_seq))}\SpecialCharTok{/}\DecValTok{2}\NormalTok{, }\AttributeTok{y =} \FunctionTok{max}\NormalTok{(y\_seq) }\SpecialCharTok{*} \FloatTok{0.4}\NormalTok{, }
           \AttributeTok{label =} \FunctionTok{paste}\NormalTok{(}\StringTok{"Região de}\SpecialCharTok{\textbackslash{}n}\StringTok{Rejeição}\SpecialCharTok{\textbackslash{}n}\StringTok{α ="}\NormalTok{, alpha), }
           \AttributeTok{color =} \StringTok{"red"}\NormalTok{, }\AttributeTok{size =} \DecValTok{3}\NormalTok{, }\AttributeTok{hjust =} \FloatTok{0.5}\NormalTok{) }\SpecialCharTok{+}
  \FunctionTok{theme\_minimal}\NormalTok{() }\SpecialCharTok{+}
  \FunctionTok{theme}\NormalTok{(}
    \AttributeTok{plot.title =} \FunctionTok{element\_text}\NormalTok{(}\AttributeTok{hjust =} \FloatTok{0.5}\NormalTok{, }\AttributeTok{size =} \DecValTok{14}\NormalTok{, }\AttributeTok{face =} \StringTok{"bold"}\NormalTok{),}
    \AttributeTok{plot.subtitle =} \FunctionTok{element\_text}\NormalTok{(}\AttributeTok{hjust =} \FloatTok{0.5}\NormalTok{, }\AttributeTok{size =} \DecValTok{12}\NormalTok{),}
    \AttributeTok{plot.caption =} \FunctionTok{element\_text}\NormalTok{(}\AttributeTok{hjust =} \FloatTok{0.5}\NormalTok{, }\AttributeTok{size =} \DecValTok{10}\NormalTok{, }\AttributeTok{face =} \StringTok{"bold"}\NormalTok{)}
\NormalTok{  )}

\FunctionTok{print}\NormalTok{(p\_chi)}

\CommentTok{\# ==============================================================================}
\CommentTok{\# GRÁFICO DE BARRAS COMPARATIVO}
\CommentTok{\# ==============================================================================}

\CommentTok{\# Preparando dados para gráfico de barras}
\NormalTok{dados\_long }\OtherTok{\textless{}{-}} \FunctionTok{data.frame}\NormalTok{(}
  \AttributeTok{Ano =} \FunctionTok{rep}\NormalTok{(}\FunctionTok{c}\NormalTok{(}\StringTok{"2022"}\NormalTok{, }\StringTok{"2023"}\NormalTok{), }\AttributeTok{each =} \DecValTok{2}\NormalTok{),}
  \AttributeTok{Local =} \FunctionTok{rep}\NormalTok{(}\FunctionTok{c}\NormalTok{(}\StringTok{"Gyn"}\NormalTok{, }\StringTok{"Inter"}\NormalTok{), }\DecValTok{2}\NormalTok{),}
  \AttributeTok{Frequencia =} \FunctionTok{as.vector}\NormalTok{(dados\_observados),}
  \AttributeTok{Tipo =} \StringTok{"Observado"}
\NormalTok{)}

\NormalTok{dados\_esp\_long }\OtherTok{\textless{}{-}} \FunctionTok{data.frame}\NormalTok{(}
  \AttributeTok{Ano =} \FunctionTok{rep}\NormalTok{(}\FunctionTok{c}\NormalTok{(}\StringTok{"2022"}\NormalTok{, }\StringTok{"2023"}\NormalTok{), }\AttributeTok{each =} \DecValTok{2}\NormalTok{),}
  \AttributeTok{Local =} \FunctionTok{rep}\NormalTok{(}\FunctionTok{c}\NormalTok{(}\StringTok{"Gyn"}\NormalTok{, }\StringTok{"Inter"}\NormalTok{), }\DecValTok{2}\NormalTok{),}
  \AttributeTok{Frequencia =} \FunctionTok{as.vector}\NormalTok{(freq\_esperadas),}
  \AttributeTok{Tipo =} \StringTok{"Esperado"}
\NormalTok{)}

\NormalTok{dados\_comparacao }\OtherTok{\textless{}{-}} \FunctionTok{rbind}\NormalTok{(dados\_long, dados\_esp\_long)}

\NormalTok{p\_barras }\OtherTok{\textless{}{-}} \FunctionTok{ggplot}\NormalTok{(dados\_comparacao, }\FunctionTok{aes}\NormalTok{(}\AttributeTok{x =}\NormalTok{ Ano,}
                                         \AttributeTok{y =}\NormalTok{ Frequencia,}
                                         \AttributeTok{fill =}\NormalTok{ Local,}
                                         \AttributeTok{alpha =}\NormalTok{ Tipo)) }\SpecialCharTok{+}
  \FunctionTok{geom\_bar}\NormalTok{(}\AttributeTok{stat =} \StringTok{"identity"}\NormalTok{, }\AttributeTok{position =} \StringTok{"dodge"}\NormalTok{) }\SpecialCharTok{+}
  \FunctionTok{scale\_alpha\_manual}\NormalTok{(}\AttributeTok{values =} \FunctionTok{c}\NormalTok{(}\StringTok{"Observado"} \OtherTok{=} \FloatTok{0.8}\NormalTok{, }\StringTok{"Esperado"} \OtherTok{=} \FloatTok{0.4}\NormalTok{)) }\SpecialCharTok{+}
  \FunctionTok{scale\_fill\_manual}\NormalTok{(}\AttributeTok{values =} \FunctionTok{c}\NormalTok{(}\StringTok{"Gyn"} \OtherTok{=} \StringTok{"\#E69F00"}\NormalTok{, }\StringTok{"Inter"} \OtherTok{=} \StringTok{"\#56B4E9"}\NormalTok{)) }\SpecialCharTok{+}
  \FunctionTok{labs}\NormalTok{(}
    \AttributeTok{title =} \StringTok{"Frequências Observadas vs Esperadas"}\NormalTok{,}
    \AttributeTok{subtitle =} \StringTok{"num. AJCM: Local por Ano"}\NormalTok{,}
    \AttributeTok{x =} \StringTok{"Ano"}\NormalTok{,}
    \AttributeTok{y =} \StringTok{"Frequência"}\NormalTok{,}
    \AttributeTok{fill =} \StringTok{"Local"}\NormalTok{,}
    \AttributeTok{alpha =} \StringTok{"Tipo"}
\NormalTok{  ) }\SpecialCharTok{+}
  \FunctionTok{theme\_minimal}\NormalTok{() }\SpecialCharTok{+}
  \FunctionTok{theme}\NormalTok{(}
    \AttributeTok{plot.title =} \FunctionTok{element\_text}\NormalTok{(}\AttributeTok{hjust =} \FloatTok{0.5}\NormalTok{, }\AttributeTok{size =} \DecValTok{14}\NormalTok{, }\AttributeTok{face =} \StringTok{"bold"}\NormalTok{),}
    \AttributeTok{plot.subtitle =} \FunctionTok{element\_text}\NormalTok{(}\AttributeTok{hjust =} \FloatTok{0.5}\NormalTok{, }\AttributeTok{size =} \DecValTok{12}\NormalTok{),}
    \AttributeTok{legend.position =} \StringTok{"bottom"}
\NormalTok{  )}

\FunctionTok{print}\NormalTok{(p\_barras)}

\CommentTok{\# ==============================================================================}
\CommentTok{\# TABELA RESUMO FINAL}
\CommentTok{\# ==============================================================================}

\CommentTok{\# Criando tabela resumo para apresentação}
\NormalTok{tabela\_resumo }\OtherTok{\textless{}{-}} \FunctionTok{data.frame}\NormalTok{(}
\NormalTok{  Parâmetro }\OtherTok{=} \FunctionTok{c}\NormalTok{(}\StringTok{"Estatística χ²"}\NormalTok{, }\StringTok{"Graus de Liberdade"}\NormalTok{, }\StringTok{"Valor Crítico"}\NormalTok{, }
                \StringTok{"Valor P"}\NormalTok{, }\StringTok{"Nível de Significância"}\NormalTok{, }\StringTok{"Decisão"}\NormalTok{, }\StringTok{"Conclusão"}\NormalTok{),}
  \AttributeTok{Valor =} \FunctionTok{c}\NormalTok{(}
    \FunctionTok{round}\NormalTok{(chi\_squared\_calc, }\DecValTok{4}\NormalTok{),}
\NormalTok{    gl,}
    \FunctionTok{round}\NormalTok{(chi\_critico, }\DecValTok{4}\NormalTok{),}
    \FunctionTok{ifelse}\NormalTok{(p\_value }\SpecialCharTok{\textless{}} \FloatTok{0.0001}\NormalTok{, }\StringTok{"\textless{} 0.0001"}\NormalTok{, }\FunctionTok{round}\NormalTok{(p\_value, }\DecValTok{6}\NormalTok{)),}
    \FunctionTok{paste0}\NormalTok{(alpha }\SpecialCharTok{*} \DecValTok{100}\NormalTok{, }\StringTok{"\%"}\NormalTok{),}
    \FunctionTok{ifelse}\NormalTok{(chi\_squared\_calc }\SpecialCharTok{\textgreater{}}\NormalTok{ chi\_critico, }\StringTok{"REJEITA H₀"}\NormalTok{, }\StringTok{"NÃO REJEITA H₀"}\NormalTok{),}
    \FunctionTok{ifelse}\NormalTok{(chi\_squared\_calc }\SpecialCharTok{\textgreater{}}\NormalTok{ chi\_critico, }
           \StringTok{"Associação significativa"}\NormalTok{, }\StringTok{"Sem associação significativa"}\NormalTok{)}
\NormalTok{  )}
\NormalTok{)}

\FunctionTok{cat}\NormalTok{(}\StringTok{"}\SpecialCharTok{\textbackslash{}n}\StringTok{RESUMO DO TESTE QUI{-}QUADRADO:}\SpecialCharTok{\textbackslash{}n}\StringTok{"}\NormalTok{)}
\FunctionTok{print}\NormalTok{(tabela\_resumo, }\AttributeTok{row.names =} \ConstantTok{FALSE}\NormalTok{)}

\CommentTok{\# ==============================================================================}
\CommentTok{\# MEDIDAS DE ASSOCIAÇÃO}
\CommentTok{\# ==============================================================================}

\CommentTok{\# Coeficiente de contingência}
\NormalTok{C }\OtherTok{\textless{}{-}} \FunctionTok{sqrt}\NormalTok{(chi\_squared\_calc }\SpecialCharTok{/}\NormalTok{ (chi\_squared\_calc }\SpecialCharTok{+}\NormalTok{ total\_geral))}

\CommentTok{\# V de Cramér}
\NormalTok{V }\OtherTok{\textless{}{-}} \FunctionTok{sqrt}\NormalTok{(chi\_squared\_calc }\SpecialCharTok{/}\NormalTok{ (total\_geral }\SpecialCharTok{*} \FunctionTok{min}\NormalTok{(}\FunctionTok{nrow}\NormalTok{(dados\_observados) }\SpecialCharTok{{-}} \DecValTok{1}\NormalTok{, }
                                               \FunctionTok{ncol}\NormalTok{(dados\_observados) }\SpecialCharTok{{-}} \DecValTok{1}\NormalTok{)))}

\FunctionTok{cat}\NormalTok{(}\StringTok{"}\SpecialCharTok{\textbackslash{}n}\StringTok{MEDIDAS DE ASSOCIAÇÃO:}\SpecialCharTok{\textbackslash{}n}\StringTok{"}\NormalTok{)}
\FunctionTok{cat}\NormalTok{(}\StringTok{"Coeficiente de Contingência (C):"}\NormalTok{, }\FunctionTok{round}\NormalTok{(C, }\DecValTok{4}\NormalTok{), }\StringTok{"}\SpecialCharTok{\textbackslash{}n}\StringTok{"}\NormalTok{)}
\FunctionTok{cat}\NormalTok{(}\StringTok{"V de Cramér:"}\NormalTok{, }\FunctionTok{round}\NormalTok{(V, }\DecValTok{4}\NormalTok{), }\StringTok{"}\SpecialCharTok{\textbackslash{}n}\StringTok{"}\NormalTok{)}
\FunctionTok{cat}\NormalTok{(}\StringTok{"Interpretação:"}\NormalTok{, }\FunctionTok{ifelse}\NormalTok{(V }\SpecialCharTok{\textless{}} \FloatTok{0.1}\NormalTok{, }\StringTok{"Associação fraca"}\NormalTok{, }
                            \FunctionTok{ifelse}\NormalTok{(V }\SpecialCharTok{\textless{}} \FloatTok{0.3}\NormalTok{, }\StringTok{"Associação moderada"}\NormalTok{, }\StringTok{"Associação forte"}\NormalTok{)), }\StringTok{"}\SpecialCharTok{\textbackslash{}n}\StringTok{"}\NormalTok{)}

\CommentTok{\# Salvando resultados em arquivo (opcional)}
\CommentTok{\# write.csv(dados\_observados, "ajcm\_dados\_observados.csv")}
\CommentTok{\# ggsave("ajcm\_teste\_qui\_quadrado.png", p\_chi, width = 12, height = 8, dpi = 300)}
\InformationTok{\textasciigrave{}\textasciigrave{}\textasciigrave{}}
\end{Highlighting}
\end{Shaded}

\begin{verbatim}
TABELA DE CONTINGÊNCIA - DADOS OBSERVADOS
      Local
Ano    Gyn Inter
  2022 304   220
  2023 761   814

Totais por Ano:
2022 2023 
 524 1575 

Totais por Local:
  Gyn Inter 
 1065  1034 

Total Geral: 2099 

FREQUÊNCIAS ESPERADAS (sob H₀):
        Gyn  Inter
2022 265.87 258.13
2023 799.13 775.87
TESTE QUI-QUADRADO DE INDEPENDÊNCIA
H₀: Não há associação entre Ano e Local
H₁: Há associação entre Ano e Local
Nível de Confiança: 99.9% (α = 0.001)

Estatística qui-quadrado calculada: 14.7945 
Graus de liberdade: 1 
Valor crítico (α = 0.001): 10.8276 
Valor P: 0.00012 
Decisão: REJEITA H₀ 
Conclusão: Há evidência significativa de associação 

CÁLCULOS DETALHADOS POR CÉLULA:
    Célula Observado Esperado Diferença Qui_Quadrado Contribuição_Perc
1 2022-Gyn       304   265.87     38.13       5.4686              37.0
2 2022-Int       761   799.13    -38.13       1.8194              12.3
3 2023-Gyn       220   258.13    -38.13       5.6326              38.1
4 2023-Int       814   775.87     38.13       1.8739              12.7

RESUMO DO TESTE QUI-QUADRADO:
              Parâmetro                    Valor
         Estatística χ²                  14.7945
     Graus de Liberdade                        1
          Valor Crítico                  10.8276
                Valor P                  0.00012
 Nível de Significância                     0.1%
                Decisão               REJEITA H₀
              Conclusão Associação significativa

MEDIDAS DE ASSOCIAÇÃO:
Coeficiente de Contingência (C): 0.0837 
V de Cramér: 0.084 
Interpretação: Associação fraca 
\end{verbatim}

\pandocbounded{\includegraphics[keepaspectratio]{cap17-TSHo-Oltramari_files/figure-pdf/unnamed-chunk-1-1.pdf}}

\pandocbounded{\includegraphics[keepaspectratio]{cap17-TSHo-Oltramari_files/figure-pdf/unnamed-chunk-1-2.pdf}}

Os resultados acima constituem \ul{\textbf{evidência}} \emph{decorrente}
de uma \textbf{\emph{significância estatística}} (valor P = 0,00012
\textless{} 0,001 = 0,1\%; tamanho da amostra = 2099) pela
\ul{\textbf{refutação}} da \emph{Hipótese Nula de nenhuma associação},
para um Nível de Confiaça de 99,9\%, e \ul{\textbf{em favor da Hipótese
Alternativa}} de que \ul{\textbf{há uma associação significativa}},
todavia \emph{fraca} (V de Cramér = 0,084), para a
\ul{\textbf{\emph{distribuição}}} do \ul{\textbf{\emph{número de Autos
Judiciais de Crimes Militares}}} (AJCM) no \textbf{Ano} de \textbf{2022}
- \ul{\textbf{após}} e de 2023 - logicamente \ul{\textbf{após}} a
\ul{\textbf{vigência}} da \ul{\textbf{Resolução CPJ n.~04/2022 do
MP-GO}}, de \ul{\textbf{28 de março de 2022}}, e o \textbf{Local da
Promotoria}: em \textbf{Gyn} (Promotoria de Goiânia especializada com 2
Promotores que atuam junto à VAM - Vara de Auditoria Militar) \emph{ou}
no \textbf{Interior}, este em matéria afeta à competência da Justiça
Militar Estadual, após a entrada em vigor da Resolução CPJ n.º 4/2022.

Período observado foi: de 30 de março de 2022 a 31 de outubro de 2023.
Para verificar se com a nova atribuição de competência aos Promotores do
interior, houve um acréscimo na \ul{\textbf{taxa}} de
\ul{\textbf{\emph{Autos Judiciais de Crimes Militares}}} (AJCM)
promovidos em 2022 e em 2023 pelas Promotorias de Gyn e do Interior.
(OLTRAMARI, 2024 , p.~161)

\subsection{Teste da diferença entre duas
médias}\label{teste-da-diferenuxe7a-entre-duas-muxe9dias}

No período objeto da pesquisa, foram 249 denúncias em AJCMs antes da
vigência da Resolução CPJ n.º 4/2022 (63 meses) e outras 164 sob a égide
desta, em um período de 19 meses coberto pelo estudo (entre 30 de março
de 2022 e 31 de outubro de 2023). Apresenta-se, portanto, uma
\textbf{média} de 3,95 denúncias/mês \ul{\textbf{antes}} (249/63) da
norma, para uma \textbf{média} de 8,63 \ul{\textbf{após}} (164/19), o
que representa um aumento de 118,5\%.

\[
\text{Taxa aumento} = \frac{8.63-3.95}{3.95}=\frac{4.68}{3.95}=1.1848 \sim 118.5\%
\]

O script a seguir realiza um teste de significância randomizado para
diferença entre duas médias, que, supostamente, são oriundas de duas
amostras aleatórias independentes de tamanhos diferentes:
n\textsubscript{antes} = 64; n\textsubscript{depois} = 21 e
n\textsubscript{total} = 64+21 = 85 pontos amostrais ao longo do tempo
(número de denúncias ofertadas pelo MP em cada mês).

Considerando a data de vigência da Resolução, temos que, no ano de 2022,
foram 8 casos de denúncias oferecidas em AJCMs autuados em data anterior
àquela, sendo o restante (62 casos) hipótese de denúncia de AJCM autuado
após. Daqueles 8 casos, em apenas 3 foi oferecida denúncia por uma das
Promotorias de Justiça de Goiânia (ou seja, as outras 5 já foram
oferecidas por Promotorias de Justiça do interior por força da alteração
normativa). Há, ainda, outras 6 denúncias relativas aos AJCMs de 2021
oferecidas após o marco temporal de 30 de março de 2022 por Promotoria
de Justiça do interior do Estado.

São casos em que, apesar de a autuação judicial ter ocorrido antes da
vigência da Resolução, a \textbf{\emph{denúncia foi oferecida após sua
vigência}}, com a remessa dos autos às Promotorias de Justiça criminais
do local em que os fatos ocorreram, em cumprimento ao determinado pelo
\textbf{artigo 5º} da Resolução CPJ n.º 4, de \ul{\textbf{28 de março de
2022}}: ``Art. 5º No prazo de 90 (noventa) dias contados da publicação
da presente Resolução, as Promotorias de Justiça da comarca de Goiânia
com atuação perante a Vara da Auditoria Militar deverão encaminhar à
Superintendência Judiciária da Instituição os inquéritos policiais
militares e os autos administrativos similares que estiverem em sua
posse e forem relacionados aos \textbf{\emph{crimes militares}} ou
\textbf{\emph{praticados por militares}} a que se referem o artigo 3º, a
fim de que sejam \ul{\textbf{remetidos}} para as \ul{\textbf{Promotorias
de Justiça criminais do local}} em que os \ul{\textbf{fatos}} tiverem
sido praticados.'' (MINISTÉRIO PÚBLICO DO ESTADO DE GOIÁS. Resolução n.º
4, de 28 de março de 2022, do Colégio de Procuradores de Justiça.
Disponível em: \url{https://www.mpgo.mp.br/portal/atos_normas/.} Acesso
em: 15 nov. 2023). (OLTRAMARI, 2024 , p.~157)

\begin{Shaded}
\begin{Highlighting}[numbers=left,,]
\InformationTok{\textasciigrave{}\textasciigrave{}\textasciigrave{}\{r\}}
\CommentTok{\# TESTE BOOTSTRAP PARA DIFERENÇA ENTRE DUAS MÉDIAS}
\CommentTok{\# Baseado em: Oltramari {-} Número de Denúncias Antes vs Depois}
\CommentTok{\# Dados: x1\_bar = 3.95, n1 = 63 vs x2\_bar = 8.63, n2 = 19}
\CommentTok{\# NÍVEL DE CONFIANÇA: 99.9\% (α = 0.001)}

\CommentTok{\# Carregando bibliotecas necessárias}
\FunctionTok{library}\NormalTok{(ggplot2)}
\FunctionTok{library}\NormalTok{(dplyr)}
\FunctionTok{library}\NormalTok{(boot)}
\FunctionTok{library}\NormalTok{(gridExtra)}

\CommentTok{\# ==============================================================================}
\CommentTok{\# PARÂMETROS DO ESTUDO}
\CommentTok{\# ==============================================================================}

\CommentTok{\# Dados das amostras}
\NormalTok{x1\_bar }\OtherTok{\textless{}{-}} \FloatTok{3.95}    \CommentTok{\# Média grupo 1 (antes): 3.95 denúncias/mês}
\NormalTok{x2\_bar }\OtherTok{\textless{}{-}} \FloatTok{8.63}    \CommentTok{\# Média grupo 2 (depois): 8.63 denúncias/mês}
\NormalTok{n1 }\OtherTok{\textless{}{-}} \DecValTok{63}          \CommentTok{\# Tamanho amostra 1}
\NormalTok{n2 }\OtherTok{\textless{}{-}} \DecValTok{19}          \CommentTok{\# Tamanho amostra 2}

\CommentTok{\# Nível de confiança}
\NormalTok{conf\_level }\OtherTok{\textless{}{-}} \FloatTok{0.999}  \CommentTok{\# 99.9\%}
\NormalTok{alpha }\OtherTok{\textless{}{-}} \DecValTok{1} \SpecialCharTok{{-}}\NormalTok{ conf\_level}

\CommentTok{\# Diferença observada}
\NormalTok{diff\_observada }\OtherTok{\textless{}{-}}\NormalTok{ x2\_bar }\SpecialCharTok{{-}}\NormalTok{ x1\_bar}

\FunctionTok{cat}\NormalTok{(}\StringTok{"TESTE BOOTSTRAP PARA DIFERENÇA ENTRE DUAS MÉDIAS}\SpecialCharTok{\textbackslash{}n}\StringTok{"}\NormalTok{)}
\FunctionTok{cat}\NormalTok{(}\StringTok{"===============================================}\SpecialCharTok{\textbackslash{}n}\StringTok{"}\NormalTok{)}
\FunctionTok{cat}\NormalTok{(}\StringTok{"NÍVEL DE CONFIANÇA: 99.9\% (α = 0.001)}\SpecialCharTok{\textbackslash{}n}\StringTok{"}\NormalTok{)}
\FunctionTok{cat}\NormalTok{(}\StringTok{"Grupo 1 (Antes): x̄₁ ="}\NormalTok{, x1\_bar, }\StringTok{"denúncias/mês, n₁ ="}\NormalTok{, n1, }\StringTok{"}\SpecialCharTok{\textbackslash{}n}\StringTok{"}\NormalTok{)}
\FunctionTok{cat}\NormalTok{(}\StringTok{"Grupo 2 (Depois): x̄₂ ="}\NormalTok{, x2\_bar, }\StringTok{"denúncias/mês, n₂ ="}\NormalTok{, n2, }\StringTok{"}\SpecialCharTok{\textbackslash{}n}\StringTok{"}\NormalTok{)}
\FunctionTok{cat}\NormalTok{(}\StringTok{"Diferença observada (x̄₂ {-} x̄₁) ="}\NormalTok{, }\FunctionTok{round}\NormalTok{(diff\_observada, }\DecValTok{3}\NormalTok{), }\StringTok{"}\SpecialCharTok{\textbackslash{}n\textbackslash{}n}\StringTok{"}\NormalTok{)}

\CommentTok{\# ==============================================================================}
\CommentTok{\# SIMULAÇÃO DOS DADOS ORIGINAIS (MELHORADA)}
\CommentTok{\# ==============================================================================}

\FunctionTok{set.seed}\NormalTok{(}\DecValTok{123}\NormalTok{)  }\CommentTok{\# Para reprodutibilidade}

\CommentTok{\# Estimativa mais robusta de desvios{-}padrão baseada nos tamanhos amostrais}
\CommentTok{\# e nas médias observadas, considerando a natureza dos dados de denúncias}
\NormalTok{s1\_est }\OtherTok{\textless{}{-}} \FunctionTok{sqrt}\NormalTok{(x1\_bar }\SpecialCharTok{*} \FloatTok{0.8}\NormalTok{)  }\CommentTok{\# Aproximação baseada em distribuição de contagens}
\NormalTok{s2\_est }\OtherTok{\textless{}{-}} \FunctionTok{sqrt}\NormalTok{(x2\_bar }\SpecialCharTok{*} \FloatTok{0.9}\NormalTok{)  }\CommentTok{\# Aproximação baseada em distribuição de contagens}

\CommentTok{\# Ajustando para garantir variabilidade realística}
\NormalTok{s1\_est }\OtherTok{\textless{}{-}} \FunctionTok{max}\NormalTok{(s1\_est, }\FloatTok{1.5}\NormalTok{)  }\CommentTok{\# Mínimo de variabilidade}
\NormalTok{s2\_est }\OtherTok{\textless{}{-}} \FunctionTok{max}\NormalTok{(s2\_est, }\FloatTok{2.0}\NormalTok{)  }\CommentTok{\# Mínimo de variabilidade}

\CommentTok{\# Simulando os dados que resultariam nas médias observadas}
\CommentTok{\# Usando distribuição normal truncada para evitar valores negativos}
\NormalTok{amostra1 }\OtherTok{\textless{}{-}} \FunctionTok{pmax}\NormalTok{(}\DecValTok{0}\NormalTok{, }\FunctionTok{rnorm}\NormalTok{(n1, }\AttributeTok{mean =}\NormalTok{ x1\_bar, }\AttributeTok{sd =}\NormalTok{ s1\_est))}
\NormalTok{amostra2 }\OtherTok{\textless{}{-}} \FunctionTok{pmax}\NormalTok{(}\DecValTok{0}\NormalTok{, }\FunctionTok{rnorm}\NormalTok{(n2, }\AttributeTok{mean =}\NormalTok{ x2\_bar, }\AttributeTok{sd =}\NormalTok{ s2\_est))}

\CommentTok{\# Ajustando para que as médias sejam exatamente as observadas}
\NormalTok{amostra1 }\OtherTok{\textless{}{-}}\NormalTok{ amostra1 }\SpecialCharTok{{-}} \FunctionTok{mean}\NormalTok{(amostra1) }\SpecialCharTok{+}\NormalTok{ x1\_bar}
\NormalTok{amostra2 }\OtherTok{\textless{}{-}}\NormalTok{ amostra2 }\SpecialCharTok{{-}} \FunctionTok{mean}\NormalTok{(amostra2) }\SpecialCharTok{+}\NormalTok{ x2\_bar}

\CommentTok{\# Garantindo valores não{-}negativos (denúncias não podem ser negativas)}
\NormalTok{amostra1 }\OtherTok{\textless{}{-}} \FunctionTok{pmax}\NormalTok{(}\DecValTok{0}\NormalTok{, amostra1)}
\NormalTok{amostra2 }\OtherTok{\textless{}{-}} \FunctionTok{pmax}\NormalTok{(}\DecValTok{0}\NormalTok{, amostra2)}

\CommentTok{\# Verificando as médias simuladas}
\FunctionTok{cat}\NormalTok{(}\StringTok{"Verificação das médias simuladas:}\SpecialCharTok{\textbackslash{}n}\StringTok{"}\NormalTok{)}
\FunctionTok{cat}\NormalTok{(}\StringTok{"Média simulada grupo 1:"}\NormalTok{, }\FunctionTok{round}\NormalTok{(}\FunctionTok{mean}\NormalTok{(amostra1), }\DecValTok{3}\NormalTok{), }
    \StringTok{"(alvo:"}\NormalTok{, x1\_bar, }\StringTok{")}\SpecialCharTok{\textbackslash{}n}\StringTok{"}\NormalTok{)}
\FunctionTok{cat}\NormalTok{(}\StringTok{"Média simulada grupo 2:"}\NormalTok{, }\FunctionTok{round}\NormalTok{(}\FunctionTok{mean}\NormalTok{(amostra2), }\DecValTok{3}\NormalTok{), }
    \StringTok{"(alvo:"}\NormalTok{, x2\_bar, }\StringTok{")}\SpecialCharTok{\textbackslash{}n}\StringTok{"}\NormalTok{)}
\FunctionTok{cat}\NormalTok{(}\StringTok{"DP grupo 1:"}\NormalTok{, }\FunctionTok{round}\NormalTok{(}\FunctionTok{sd}\NormalTok{(amostra1), }\DecValTok{3}\NormalTok{), }\StringTok{"}\SpecialCharTok{\textbackslash{}n}\StringTok{"}\NormalTok{)}
\FunctionTok{cat}\NormalTok{(}\StringTok{"DP grupo 2:"}\NormalTok{, }\FunctionTok{round}\NormalTok{(}\FunctionTok{sd}\NormalTok{(amostra2), }\DecValTok{3}\NormalTok{), }\StringTok{"}\SpecialCharTok{\textbackslash{}n}\StringTok{"}\NormalTok{)}
\FunctionTok{cat}\NormalTok{(}\StringTok{"Diferença simulada:"}\NormalTok{, }\FunctionTok{round}\NormalTok{(}\FunctionTok{mean}\NormalTok{(amostra2) }\SpecialCharTok{{-}} \FunctionTok{mean}\NormalTok{(amostra1), }\DecValTok{3}\NormalTok{), }
    \StringTok{"(alvo:"}\NormalTok{, }\FunctionTok{round}\NormalTok{(diff\_observada, }\DecValTok{3}\NormalTok{), }\StringTok{")}\SpecialCharTok{\textbackslash{}n\textbackslash{}n}\StringTok{"}\NormalTok{)}

\CommentTok{\# ==============================================================================}
\CommentTok{\# TESTE BOOTSTRAP PARA DIFERENÇA ENTRE MÉDIAS (CORRIGIDO)}
\CommentTok{\# ==============================================================================}

\CommentTok{\# Função melhorada para calcular a diferença entre médias}
\NormalTok{diff\_medias }\OtherTok{\textless{}{-}} \ControlFlowTok{function}\NormalTok{(data, indices) \{}
  \CommentTok{\# Separando os grupos baseado nos índices originais}
\NormalTok{  grupo1\_indices }\OtherTok{\textless{}{-}}\NormalTok{ indices[indices }\SpecialCharTok{\textless{}=}\NormalTok{ n1]}
\NormalTok{  grupo2\_indices }\OtherTok{\textless{}{-}}\NormalTok{ indices[indices }\SpecialCharTok{\textgreater{}}\NormalTok{ n1] }\SpecialCharTok{{-}}\NormalTok{ n1}
  
  \CommentTok{\# Se não há índices suficientes, usar reamostragem com reposição}
  \ControlFlowTok{if}\NormalTok{(}\FunctionTok{length}\NormalTok{(grupo1\_indices) }\SpecialCharTok{==} \DecValTok{0}\NormalTok{) grupo1\_indices }\OtherTok{\textless{}{-}} \FunctionTok{sample}\NormalTok{(}\DecValTok{1}\SpecialCharTok{:}\NormalTok{n1, n1, }\AttributeTok{replace =} \ConstantTok{TRUE}\NormalTok{)}
  \ControlFlowTok{if}\NormalTok{(}\FunctionTok{length}\NormalTok{(grupo2\_indices) }\SpecialCharTok{==} \DecValTok{0}\NormalTok{) grupo2\_indices }\OtherTok{\textless{}{-}} \FunctionTok{sample}\NormalTok{(}\DecValTok{1}\SpecialCharTok{:}\NormalTok{n2, n2, }\AttributeTok{replace =} \ConstantTok{TRUE}\NormalTok{)}
  
  \CommentTok{\# Calculando médias das amostras bootstrap}
\NormalTok{  media1\_boot }\OtherTok{\textless{}{-}} \FunctionTok{mean}\NormalTok{(amostra1[grupo1\_indices])}
\NormalTok{  media2\_boot }\OtherTok{\textless{}{-}} \FunctionTok{mean}\NormalTok{(amostra2[grupo2\_indices])}
  
  \FunctionTok{return}\NormalTok{(media2\_boot }\SpecialCharTok{{-}}\NormalTok{ media1\_boot)}
\NormalTok{\}}

\CommentTok{\# Função alternativa mais robusta para bootstrap}
\NormalTok{bootstrap\_diff }\OtherTok{\textless{}{-}} \ControlFlowTok{function}\NormalTok{() \{}
  \CommentTok{\# Reamostragem independente de cada grupo}
\NormalTok{  boot1 }\OtherTok{\textless{}{-}} \FunctionTok{sample}\NormalTok{(amostra1, n1, }\AttributeTok{replace =} \ConstantTok{TRUE}\NormalTok{)}
\NormalTok{  boot2 }\OtherTok{\textless{}{-}} \FunctionTok{sample}\NormalTok{(amostra2, n2, }\AttributeTok{replace =} \ConstantTok{TRUE}\NormalTok{)}
  \FunctionTok{return}\NormalTok{(}\FunctionTok{mean}\NormalTok{(boot2) }\SpecialCharTok{{-}} \FunctionTok{mean}\NormalTok{(boot1))}
\NormalTok{\}}

\CommentTok{\# Realizando o bootstrap manualmente para maior controle}
\NormalTok{n\_bootstrap }\OtherTok{\textless{}{-}} \DecValTok{20000}  \CommentTok{\# Aumentando para maior precisão com NC = 99.9\%}
\NormalTok{boot\_diffs }\OtherTok{\textless{}{-}} \FunctionTok{replicate}\NormalTok{(n\_bootstrap, }\FunctionTok{bootstrap\_diff}\NormalTok{())}

\CommentTok{\# Estatísticas do bootstrap}
\NormalTok{media\_boot }\OtherTok{\textless{}{-}} \FunctionTok{mean}\NormalTok{(boot\_diffs)}
\NormalTok{sd\_boot }\OtherTok{\textless{}{-}} \FunctionTok{sd}\NormalTok{(boot\_diffs)}

\FunctionTok{cat}\NormalTok{(}\StringTok{"RESULTADOS DO BOOTSTRAP:}\SpecialCharTok{\textbackslash{}n}\StringTok{"}\NormalTok{)}
\FunctionTok{cat}\NormalTok{(}\StringTok{"Número de reamostragens:"}\NormalTok{, n\_bootstrap, }\StringTok{"}\SpecialCharTok{\textbackslash{}n}\StringTok{"}\NormalTok{)}
\FunctionTok{cat}\NormalTok{(}\StringTok{"Diferença original:"}\NormalTok{, }\FunctionTok{round}\NormalTok{(diff\_observada, }\DecValTok{3}\NormalTok{), }\StringTok{"}\SpecialCharTok{\textbackslash{}n}\StringTok{"}\NormalTok{)}
\FunctionTok{cat}\NormalTok{(}\StringTok{"Média das diferenças bootstrap:"}\NormalTok{, }\FunctionTok{round}\NormalTok{(media\_boot, }\DecValTok{3}\NormalTok{), }\StringTok{"}\SpecialCharTok{\textbackslash{}n}\StringTok{"}\NormalTok{)}
\FunctionTok{cat}\NormalTok{(}\StringTok{"Desvio{-}padrão das diferenças bootstrap:"}\NormalTok{, }\FunctionTok{round}\NormalTok{(sd\_boot, }\DecValTok{3}\NormalTok{), }\StringTok{"}\SpecialCharTok{\textbackslash{}n\textbackslash{}n}\StringTok{"}\NormalTok{)}

\CommentTok{\# ==============================================================================}
\CommentTok{\# TESTE DE HIPÓTESE BOOTSTRAP}
\CommentTok{\# ==============================================================================}

\CommentTok{\# H₀: μ₂ {-} μ₁ = 0 (não há diferença entre as médias)}
\CommentTok{\# H₁: μ₂ {-} μ₁ ≠ 0 (há diferença entre as médias)}

\CommentTok{\# Bootstrap sob H₀: combinando os grupos e reamostrando}
\NormalTok{dados\_combinados\_h0 }\OtherTok{\textless{}{-}} \FunctionTok{c}\NormalTok{(amostra1, amostra2)}
\NormalTok{media\_geral }\OtherTok{\textless{}{-}} \FunctionTok{mean}\NormalTok{(dados\_combinados\_h0)}

\CommentTok{\# Centralizando os dados na média geral para simular H₀}
\NormalTok{bootstrap\_h0 }\OtherTok{\textless{}{-}} \ControlFlowTok{function}\NormalTok{() \{}
  \CommentTok{\# Reamostragem do conjunto combinado}
\NormalTok{  boot\_combined }\OtherTok{\textless{}{-}} \FunctionTok{sample}\NormalTok{(dados\_combinados\_h0, n1 }\SpecialCharTok{+}\NormalTok{ n2, }\AttributeTok{replace =} \ConstantTok{TRUE}\NormalTok{)}
  
  \CommentTok{\# Separando em dois grupos do tamanho original}
\NormalTok{  boot1\_h0 }\OtherTok{\textless{}{-}}\NormalTok{ boot\_combined[}\DecValTok{1}\SpecialCharTok{:}\NormalTok{n1]}
\NormalTok{  boot2\_h0 }\OtherTok{\textless{}{-}}\NormalTok{ boot\_combined[(n1}\SpecialCharTok{+}\DecValTok{1}\NormalTok{)}\SpecialCharTok{:}\NormalTok{(n1}\SpecialCharTok{+}\NormalTok{n2)]}
  
  \FunctionTok{return}\NormalTok{(}\FunctionTok{mean}\NormalTok{(boot2\_h0) }\SpecialCharTok{{-}} \FunctionTok{mean}\NormalTok{(boot1\_h0))}
\NormalTok{\}}

\CommentTok{\# Bootstrap sob H₀}
\NormalTok{boot\_diffs\_h0 }\OtherTok{\textless{}{-}} \FunctionTok{replicate}\NormalTok{(n\_bootstrap, }\FunctionTok{bootstrap\_h0}\NormalTok{())}

\CommentTok{\# Valores P}
\NormalTok{p\_value\_bilateral }\OtherTok{\textless{}{-}} \FunctionTok{mean}\NormalTok{(}\FunctionTok{abs}\NormalTok{(boot\_diffs\_h0) }\SpecialCharTok{\textgreater{}=} \FunctionTok{abs}\NormalTok{(diff\_observada))}
\NormalTok{p\_value\_unilateral }\OtherTok{\textless{}{-}} \FunctionTok{mean}\NormalTok{(boot\_diffs\_h0 }\SpecialCharTok{\textgreater{}=}\NormalTok{ diff\_observada)}

\FunctionTok{cat}\NormalTok{(}\StringTok{"TESTE DE HIPÓTESE BOOTSTRAP:}\SpecialCharTok{\textbackslash{}n}\StringTok{"}\NormalTok{)}
\FunctionTok{cat}\NormalTok{(}\StringTok{"H₀: μ₂ {-} μ₁ = 0 vs H₁: μ₂ {-} μ₁ ≠ 0}\SpecialCharTok{\textbackslash{}n}\StringTok{"}\NormalTok{)}
\FunctionTok{cat}\NormalTok{(}\StringTok{"Nível de significância: α ="}\NormalTok{, alpha, }\StringTok{"(NC = 99.9\%)}\SpecialCharTok{\textbackslash{}n}\StringTok{"}\NormalTok{)}
\FunctionTok{cat}\NormalTok{(}\StringTok{"Valor P (bilateral):"}\NormalTok{, }\FunctionTok{round}\NormalTok{(p\_value\_bilateral, }\DecValTok{6}\NormalTok{), }\StringTok{"}\SpecialCharTok{\textbackslash{}n}\StringTok{"}\NormalTok{)}
\FunctionTok{cat}\NormalTok{(}\StringTok{"Valor P (unilateral):"}\NormalTok{, }\FunctionTok{round}\NormalTok{(p\_value\_unilateral, }\DecValTok{6}\NormalTok{), }\StringTok{"}\SpecialCharTok{\textbackslash{}n}\StringTok{"}\NormalTok{)}
\FunctionTok{cat}\NormalTok{(}\StringTok{"Conclusão (α = 0.001):"}\NormalTok{, }
    \FunctionTok{ifelse}\NormalTok{(p\_value\_bilateral }\SpecialCharTok{\textless{}}\NormalTok{ alpha, }\StringTok{"REJEITA H₀"}\NormalTok{, }\StringTok{"NÃO REJEITA H₀"}\NormalTok{), }\StringTok{"}\SpecialCharTok{\textbackslash{}n\textbackslash{}n}\StringTok{"}\NormalTok{)}

\CommentTok{\# ==============================================================================}
\CommentTok{\# INTERVALOS DE CONFIANÇA BOOTSTRAP (CORRIGIDOS)}
\CommentTok{\# ==============================================================================}

\CommentTok{\# Intervalo de confiança percentil para NC = 99.9\%}
\NormalTok{alpha\_ic }\OtherTok{\textless{}{-}} \DecValTok{1} \SpecialCharTok{{-}}\NormalTok{ conf\_level}
\NormalTok{ic\_999\_percentil }\OtherTok{\textless{}{-}} \FunctionTok{quantile}\NormalTok{(boot\_diffs, }\FunctionTok{c}\NormalTok{(alpha\_ic}\SpecialCharTok{/}\DecValTok{2}\NormalTok{, }\DecValTok{1} \SpecialCharTok{{-}}\NormalTok{ alpha\_ic}\SpecialCharTok{/}\DecValTok{2}\NormalTok{))}

\CommentTok{\# Intervalos adicionais para comparação}
\NormalTok{ic\_95\_percentil }\OtherTok{\textless{}{-}} \FunctionTok{quantile}\NormalTok{(boot\_diffs, }\FunctionTok{c}\NormalTok{(}\FloatTok{0.025}\NormalTok{, }\FloatTok{0.975}\NormalTok{))}
\NormalTok{ic\_99\_percentil }\OtherTok{\textless{}{-}} \FunctionTok{quantile}\NormalTok{(boot\_diffs, }\FunctionTok{c}\NormalTok{(}\FloatTok{0.005}\NormalTok{, }\FloatTok{0.995}\NormalTok{))}

\FunctionTok{cat}\NormalTok{(}\StringTok{"INTERVALOS DE CONFIANÇA BOOTSTRAP:}\SpecialCharTok{\textbackslash{}n}\StringTok{"}\NormalTok{)}
\FunctionTok{cat}\NormalTok{(}\StringTok{"IC 95\% (Percentil):"}\NormalTok{, }\FunctionTok{round}\NormalTok{(ic\_95\_percentil[}\DecValTok{1}\NormalTok{], }\DecValTok{3}\NormalTok{), }\StringTok{"a"}\NormalTok{, }
    \FunctionTok{round}\NormalTok{(ic\_95\_percentil[}\DecValTok{2}\NormalTok{], }\DecValTok{3}\NormalTok{), }\StringTok{"}\SpecialCharTok{\textbackslash{}n}\StringTok{"}\NormalTok{)}
\FunctionTok{cat}\NormalTok{(}\StringTok{"IC 99\% (Percentil):"}\NormalTok{, }\FunctionTok{round}\NormalTok{(ic\_99\_percentil[}\DecValTok{1}\NormalTok{], }\DecValTok{3}\NormalTok{), }\StringTok{"a"}\NormalTok{, }
    \FunctionTok{round}\NormalTok{(ic\_99\_percentil[}\DecValTok{2}\NormalTok{], }\DecValTok{3}\NormalTok{), }\StringTok{"}\SpecialCharTok{\textbackslash{}n}\StringTok{"}\NormalTok{)}
\FunctionTok{cat}\NormalTok{(}\StringTok{"IC 99.9\% (Percentil):"}\NormalTok{, }\FunctionTok{round}\NormalTok{(ic\_999\_percentil[}\DecValTok{1}\NormalTok{], }\DecValTok{3}\NormalTok{), }\StringTok{"a"}\NormalTok{, }
    \FunctionTok{round}\NormalTok{(ic\_999\_percentil[}\DecValTok{2}\NormalTok{], }\DecValTok{3}\NormalTok{), }\StringTok{"}\SpecialCharTok{\textbackslash{}n}\StringTok{"}\NormalTok{)}

\CommentTok{\# Método básico (mais robusto que BCa)}
\FunctionTok{tryCatch}\NormalTok{(\{}
  \CommentTok{\# Usando boot() apenas para ICs básicos e percentil}
\NormalTok{  boot\_obj }\OtherTok{\textless{}{-}} \FunctionTok{boot}\NormalTok{(}\AttributeTok{data =} \FunctionTok{c}\NormalTok{(amostra1, amostra2), }
                  \AttributeTok{statistic =} \ControlFlowTok{function}\NormalTok{(data, i) \{}
\NormalTok{                    boot1 }\OtherTok{\textless{}{-}}\NormalTok{ data[i[}\DecValTok{1}\SpecialCharTok{:}\NormalTok{n1]]}
\NormalTok{                    boot2 }\OtherTok{\textless{}{-}}\NormalTok{ data[i[(n1}\SpecialCharTok{+}\DecValTok{1}\NormalTok{)}\SpecialCharTok{:}\NormalTok{(n1}\SpecialCharTok{+}\NormalTok{n2)]]}
                    \FunctionTok{return}\NormalTok{(}\FunctionTok{mean}\NormalTok{(boot2) }\SpecialCharTok{{-}} \FunctionTok{mean}\NormalTok{(boot1))}
\NormalTok{                  \}, }
                  \AttributeTok{R =} \DecValTok{5000}\NormalTok{)  }\CommentTok{\# Menor número para evitar problemas}
  
  \CommentTok{\# Tentando diferentes tipos de IC}
\NormalTok{  ic\_basico }\OtherTok{\textless{}{-}} \FunctionTok{boot.ci}\NormalTok{(boot\_obj, }\AttributeTok{type =} \StringTok{"basic"}\NormalTok{, }\AttributeTok{conf =}\NormalTok{ conf\_level)}
\NormalTok{  ic\_percentil\_boot }\OtherTok{\textless{}{-}} \FunctionTok{boot.ci}\NormalTok{(boot\_obj, }\AttributeTok{type =} \StringTok{"perc"}\NormalTok{, }\AttributeTok{conf =}\NormalTok{ conf\_level)}
  
  \ControlFlowTok{if}\NormalTok{(}\SpecialCharTok{!}\FunctionTok{is.null}\NormalTok{(ic\_basico}\SpecialCharTok{$}\NormalTok{basic)) \{}
    \FunctionTok{cat}\NormalTok{(}\StringTok{"IC 99.9\% (Básico):"}\NormalTok{, }\FunctionTok{round}\NormalTok{(ic\_basico}\SpecialCharTok{$}\NormalTok{basic[}\DecValTok{4}\NormalTok{], }\DecValTok{3}\NormalTok{), }\StringTok{"a"}\NormalTok{, }
        \FunctionTok{round}\NormalTok{(ic\_basico}\SpecialCharTok{$}\NormalTok{basic[}\DecValTok{5}\NormalTok{], }\DecValTok{3}\NormalTok{), }\StringTok{"}\SpecialCharTok{\textbackslash{}n}\StringTok{"}\NormalTok{)}
\NormalTok{  \}}
  
  \ControlFlowTok{if}\NormalTok{(}\SpecialCharTok{!}\FunctionTok{is.null}\NormalTok{(ic\_percentil\_boot}\SpecialCharTok{$}\NormalTok{percent)) \{}
    \FunctionTok{cat}\NormalTok{(}\StringTok{"IC 99.9\% (Percentil boot.ci):"}\NormalTok{, }\FunctionTok{round}\NormalTok{(ic\_percentil\_boot}\SpecialCharTok{$}\NormalTok{percent[}\DecValTok{4}\NormalTok{], }\DecValTok{3}\NormalTok{), }\StringTok{"a"}\NormalTok{, }
        \FunctionTok{round}\NormalTok{(ic\_percentil\_boot}\SpecialCharTok{$}\NormalTok{percent[}\DecValTok{5}\NormalTok{], }\DecValTok{3}\NormalTok{), }\StringTok{"}\SpecialCharTok{\textbackslash{}n}\StringTok{"}\NormalTok{)}
\NormalTok{  \}}
  
\NormalTok{\}, }\AttributeTok{error =} \ControlFlowTok{function}\NormalTok{(e) \{}
  \FunctionTok{cat}\NormalTok{(}\StringTok{"Aviso: Métodos boot.ci indisponíveis, usando método percentil manual}\SpecialCharTok{\textbackslash{}n}\StringTok{"}\NormalTok{)}
\NormalTok{\})}

\FunctionTok{cat}\NormalTok{(}\StringTok{"}\SpecialCharTok{\textbackslash{}n}\StringTok{"}\NormalTok{)}

\CommentTok{\# ==============================================================================}
\CommentTok{\# GRÁFICOS ATUALIZADOS}
\CommentTok{\# ==============================================================================}

\CommentTok{\# 1. Histograma das diferenças bootstrap}
\NormalTok{df\_boot }\OtherTok{\textless{}{-}} \FunctionTok{data.frame}\NormalTok{(}\AttributeTok{diferenca =}\NormalTok{ boot\_diffs)}

\NormalTok{p1 }\OtherTok{\textless{}{-}} \FunctionTok{ggplot}\NormalTok{(df\_boot, }\FunctionTok{aes}\NormalTok{(}\AttributeTok{x =}\NormalTok{ diferenca)) }\SpecialCharTok{+}
  \FunctionTok{geom\_histogram}\NormalTok{(}\FunctionTok{aes}\NormalTok{(}\AttributeTok{y =}\NormalTok{ ..density..), }\AttributeTok{bins =} \DecValTok{60}\NormalTok{, }
                 \AttributeTok{fill =} \StringTok{"lightblue"}\NormalTok{, }\AttributeTok{alpha =} \FloatTok{0.7}\NormalTok{, }\AttributeTok{color =} \StringTok{"black"}\NormalTok{) }\SpecialCharTok{+}
  \FunctionTok{geom\_density}\NormalTok{(}\AttributeTok{color =} \StringTok{"red"}\NormalTok{, }\AttributeTok{size =} \FloatTok{1.2}\NormalTok{) }\SpecialCharTok{+}
  \FunctionTok{geom\_vline}\NormalTok{(}\AttributeTok{xintercept =}\NormalTok{ diff\_observada, }\AttributeTok{color =} \StringTok{"darkgreen"}\NormalTok{, }
             \AttributeTok{size =} \DecValTok{2}\NormalTok{, }\AttributeTok{linetype =} \StringTok{"dashed"}\NormalTok{) }\SpecialCharTok{+}
  \FunctionTok{geom\_vline}\NormalTok{(}\AttributeTok{xintercept =}\NormalTok{ ic\_999\_percentil, }\AttributeTok{color =} \StringTok{"orange"}\NormalTok{, }
             \AttributeTok{size =} \FloatTok{1.2}\NormalTok{, }\AttributeTok{linetype =} \StringTok{"dotted"}\NormalTok{) }\SpecialCharTok{+}
  \FunctionTok{labs}\NormalTok{(}
    \AttributeTok{title =} \StringTok{"Distribuição Bootstrap das Diferenças entre Médias"}\NormalTok{,}
    \AttributeTok{subtitle =} \FunctionTok{paste}\NormalTok{(}\StringTok{"n₁ ="}\NormalTok{, n1, }\StringTok{", n₂ ="}\NormalTok{, n2, }
                    \StringTok{", Bootstrap samples ="}\NormalTok{, }\FunctionTok{format}\NormalTok{(n\_bootstrap, }\AttributeTok{big.mark =} \StringTok{","}\NormalTok{)),}
    \AttributeTok{x =} \StringTok{"Diferença (x̄₂ {-} x̄₁) denúncias/mês"}\NormalTok{,}
    \AttributeTok{y =} \StringTok{"Densidade"}\NormalTok{,}
    \AttributeTok{caption =} \FunctionTok{paste}\NormalTok{(}\StringTok{"Diferença observada ="}\NormalTok{, }\FunctionTok{round}\NormalTok{(diff\_observada, }\DecValTok{3}\NormalTok{),}
                   \StringTok{"| IC 99.9\%: ["}\NormalTok{, }\FunctionTok{round}\NormalTok{(ic\_999\_percentil[}\DecValTok{1}\NormalTok{], }\DecValTok{2}\NormalTok{), }
                   \StringTok{","}\NormalTok{, }\FunctionTok{round}\NormalTok{(ic\_999\_percentil[}\DecValTok{2}\NormalTok{], }\DecValTok{2}\NormalTok{), }\StringTok{"]"}\NormalTok{)}
\NormalTok{  ) }\SpecialCharTok{+}
  \FunctionTok{annotate}\NormalTok{(}\StringTok{"text"}\NormalTok{, }\AttributeTok{x =}\NormalTok{ diff\_observada, }\AttributeTok{y =} \FunctionTok{max}\NormalTok{(}\FunctionTok{density}\NormalTok{(boot\_diffs)}\SpecialCharTok{$}\NormalTok{y) }\SpecialCharTok{*} \FloatTok{0.9}\NormalTok{, }
           \AttributeTok{label =} \FunctionTok{paste}\NormalTok{(}\StringTok{"Observado}\SpecialCharTok{\textbackslash{}n}\StringTok{"}\NormalTok{, }\FunctionTok{round}\NormalTok{(diff\_observada, }\DecValTok{2}\NormalTok{)), }
           \AttributeTok{vjust =} \FloatTok{0.5}\NormalTok{, }\AttributeTok{color =} \StringTok{"darkgreen"}\NormalTok{, }\AttributeTok{size =} \DecValTok{4}\NormalTok{, }\AttributeTok{fontface =} \StringTok{"bold"}\NormalTok{) }\SpecialCharTok{+}
  \FunctionTok{annotate}\NormalTok{(}\StringTok{"text"}\NormalTok{, }\AttributeTok{x =} \FunctionTok{mean}\NormalTok{(ic\_999\_percentil), }\AttributeTok{y =} \FunctionTok{max}\NormalTok{(}\FunctionTok{density}\NormalTok{(boot\_diffs)}\SpecialCharTok{$}\NormalTok{y) }\SpecialCharTok{*} \FloatTok{0.1}\NormalTok{, }
           \AttributeTok{label =} \StringTok{"IC 99.9\%"}\NormalTok{, }\AttributeTok{color =} \StringTok{"orange"}\NormalTok{, }\AttributeTok{size =} \DecValTok{3}\NormalTok{, }\AttributeTok{fontface =} \StringTok{"bold"}\NormalTok{) }\SpecialCharTok{+}
  \FunctionTok{theme\_minimal}\NormalTok{() }\SpecialCharTok{+}
  \FunctionTok{theme}\NormalTok{(}
    \AttributeTok{plot.title =} \FunctionTok{element\_text}\NormalTok{(}\AttributeTok{hjust =} \FloatTok{0.5}\NormalTok{, }\AttributeTok{size =} \DecValTok{14}\NormalTok{, }\AttributeTok{face =} \StringTok{"bold"}\NormalTok{),}
    \AttributeTok{plot.subtitle =} \FunctionTok{element\_text}\NormalTok{(}\AttributeTok{hjust =} \FloatTok{0.5}\NormalTok{, }\AttributeTok{size =} \DecValTok{12}\NormalTok{),}
    \AttributeTok{plot.caption =} \FunctionTok{element\_text}\NormalTok{(}\AttributeTok{hjust =} \FloatTok{0.5}\NormalTok{, }\AttributeTok{size =} \DecValTok{10}\NormalTok{)}
\NormalTok{  )}

\FunctionTok{print}\NormalTok{(p1)}

\CommentTok{\# 2. Distribuição bootstrap sob H₀ com NC = 99.9\%}
\NormalTok{df\_boot\_h0 }\OtherTok{\textless{}{-}} \FunctionTok{data.frame}\NormalTok{(}\AttributeTok{diferenca =}\NormalTok{ boot\_diffs\_h0)}

\CommentTok{\# Região crítica para α = 0.001}
\NormalTok{z\_critico\_999 }\OtherTok{\textless{}{-}} \FunctionTok{qnorm}\NormalTok{(}\DecValTok{1} \SpecialCharTok{{-}}\NormalTok{ alpha}\SpecialCharTok{/}\DecValTok{2}\NormalTok{)  }\CommentTok{\# Valor Z para 99.9\%}
\NormalTok{limite\_critico }\OtherTok{\textless{}{-}} \FunctionTok{quantile}\NormalTok{(boot\_diffs\_h0, }\FunctionTok{c}\NormalTok{(alpha}\SpecialCharTok{/}\DecValTok{2}\NormalTok{, }\DecValTok{1}\SpecialCharTok{{-}}\NormalTok{alpha}\SpecialCharTok{/}\DecValTok{2}\NormalTok{))}

\NormalTok{p2 }\OtherTok{\textless{}{-}} \FunctionTok{ggplot}\NormalTok{(df\_boot\_h0, }\FunctionTok{aes}\NormalTok{(}\AttributeTok{x =}\NormalTok{ diferenca)) }\SpecialCharTok{+}
  \FunctionTok{geom\_histogram}\NormalTok{(}\FunctionTok{aes}\NormalTok{(}\AttributeTok{y =}\NormalTok{ ..density..), }\AttributeTok{bins =} \DecValTok{60}\NormalTok{, }
                 \AttributeTok{fill =} \StringTok{"lightcoral"}\NormalTok{, }\AttributeTok{alpha =} \FloatTok{0.7}\NormalTok{, }\AttributeTok{color =} \StringTok{"black"}\NormalTok{) }\SpecialCharTok{+}
  \FunctionTok{geom\_density}\NormalTok{(}\AttributeTok{color =} \StringTok{"blue"}\NormalTok{, }\AttributeTok{size =} \FloatTok{1.2}\NormalTok{) }\SpecialCharTok{+}
  \FunctionTok{geom\_vline}\NormalTok{(}\AttributeTok{xintercept =} \DecValTok{0}\NormalTok{, }\AttributeTok{color =} \StringTok{"black"}\NormalTok{, }
             \AttributeTok{size =} \FloatTok{1.5}\NormalTok{, }\AttributeTok{linetype =} \StringTok{"solid"}\NormalTok{) }\SpecialCharTok{+}
  \FunctionTok{geom\_vline}\NormalTok{(}\AttributeTok{xintercept =}\NormalTok{ diff\_observada, }\AttributeTok{color =} \StringTok{"darkgreen"}\NormalTok{, }
             \AttributeTok{size =} \DecValTok{2}\NormalTok{, }\AttributeTok{linetype =} \StringTok{"dashed"}\NormalTok{) }\SpecialCharTok{+}
  \FunctionTok{geom\_vline}\NormalTok{(}\AttributeTok{xintercept =} \SpecialCharTok{{-}}\NormalTok{diff\_observada, }\AttributeTok{color =} \StringTok{"darkgreen"}\NormalTok{, }
             \AttributeTok{size =} \DecValTok{2}\NormalTok{, }\AttributeTok{linetype =} \StringTok{"dashed"}\NormalTok{) }\SpecialCharTok{+}
  \FunctionTok{geom\_vline}\NormalTok{(}\AttributeTok{xintercept =}\NormalTok{ limite\_critico, }\AttributeTok{color =} \StringTok{"red"}\NormalTok{, }
             \AttributeTok{size =} \FloatTok{1.2}\NormalTok{, }\AttributeTok{linetype =} \StringTok{"dotted"}\NormalTok{) }\SpecialCharTok{+}
  \CommentTok{\# Sombreando região de rejeição}
  \FunctionTok{geom\_area}\NormalTok{(}\AttributeTok{data =} \FunctionTok{subset}\NormalTok{(df\_boot\_h0, diferenca }\SpecialCharTok{\textless{}=}\NormalTok{ limite\_critico[}\DecValTok{1}\NormalTok{]), }
            \FunctionTok{aes}\NormalTok{(}\AttributeTok{x =}\NormalTok{ diferenca, }\AttributeTok{y =}\NormalTok{ ..density..), }
            \AttributeTok{stat =} \StringTok{"density"}\NormalTok{, }\AttributeTok{fill =} \StringTok{"red"}\NormalTok{, }\AttributeTok{alpha =} \FloatTok{0.3}\NormalTok{) }\SpecialCharTok{+}
  \FunctionTok{geom\_area}\NormalTok{(}\AttributeTok{data =} \FunctionTok{subset}\NormalTok{(df\_boot\_h0, diferenca }\SpecialCharTok{\textgreater{}=}\NormalTok{ limite\_critico[}\DecValTok{2}\NormalTok{]), }
            \FunctionTok{aes}\NormalTok{(}\AttributeTok{x =}\NormalTok{ diferenca, }\AttributeTok{y =}\NormalTok{ ..density..), }
            \AttributeTok{stat =} \StringTok{"density"}\NormalTok{, }\AttributeTok{fill =} \StringTok{"red"}\NormalTok{, }\AttributeTok{alpha =} \FloatTok{0.3}\NormalTok{) }\SpecialCharTok{+}
  \FunctionTok{labs}\NormalTok{(}
    \AttributeTok{title =} \StringTok{"Distribuição Bootstrap sob H₀ (μ₂ {-} μ₁ = 0)"}\NormalTok{,}
    \AttributeTok{subtitle =} \FunctionTok{paste}\NormalTok{(}\StringTok{"Teste bilateral | NC = 99.9\% | Valor P ="}\NormalTok{, }
                    \FunctionTok{ifelse}\NormalTok{(p\_value\_bilateral }\SpecialCharTok{\textless{}} \FloatTok{0.001}\NormalTok{, }\StringTok{"\textless{} 0.001"}\NormalTok{, }
                          \FunctionTok{round}\NormalTok{(p\_value\_bilateral, }\DecValTok{4}\NormalTok{))),}
    \AttributeTok{x =} \StringTok{"Diferença (x̄₂ {-} x̄₁) sob H₀"}\NormalTok{,}
    \AttributeTok{y =} \StringTok{"Densidade"}\NormalTok{,}
    \AttributeTok{caption =} \FunctionTok{paste}\NormalTok{(}\StringTok{"Região crítica (α = 0.001): |diferença| ≥"}\NormalTok{, }
                   \FunctionTok{round}\NormalTok{(}\FunctionTok{abs}\NormalTok{(diff\_observada), }\DecValTok{3}\NormalTok{))}
\NormalTok{  ) }\SpecialCharTok{+}
  \FunctionTok{annotate}\NormalTok{(}\StringTok{"text"}\NormalTok{, }\AttributeTok{x =}\NormalTok{ diff\_observada, }\AttributeTok{y =} \FunctionTok{max}\NormalTok{(}\FunctionTok{density}\NormalTok{(boot\_diffs\_h0)}\SpecialCharTok{$}\NormalTok{y) }\SpecialCharTok{*} \FloatTok{0.8}\NormalTok{, }
           \AttributeTok{label =} \FunctionTok{paste}\NormalTok{(}\StringTok{"±"}\NormalTok{, }\FunctionTok{round}\NormalTok{(}\FunctionTok{abs}\NormalTok{(diff\_observada), }\DecValTok{2}\NormalTok{)), }
           \AttributeTok{vjust =} \FloatTok{0.5}\NormalTok{, }\AttributeTok{color =} \StringTok{"darkgreen"}\NormalTok{, }\AttributeTok{size =} \DecValTok{4}\NormalTok{, }\AttributeTok{fontface =} \StringTok{"bold"}\NormalTok{) }\SpecialCharTok{+}
  \FunctionTok{theme\_minimal}\NormalTok{() }\SpecialCharTok{+}
  \FunctionTok{theme}\NormalTok{(}
    \AttributeTok{plot.title =} \FunctionTok{element\_text}\NormalTok{(}\AttributeTok{hjust =} \FloatTok{0.5}\NormalTok{, }\AttributeTok{size =} \DecValTok{14}\NormalTok{, }\AttributeTok{face =} \StringTok{"bold"}\NormalTok{),}
    \AttributeTok{plot.subtitle =} \FunctionTok{element\_text}\NormalTok{(}\AttributeTok{hjust =} \FloatTok{0.5}\NormalTok{, }\AttributeTok{size =} \DecValTok{12}\NormalTok{),}
    \AttributeTok{plot.caption =} \FunctionTok{element\_text}\NormalTok{(}\AttributeTok{hjust =} \FloatTok{0.5}\NormalTok{, }\AttributeTok{size =} \DecValTok{10}\NormalTok{)}
\NormalTok{  )}

\FunctionTok{print}\NormalTok{(p2)}

\CommentTok{\# 3. Boxplot comparativo das amostras originais}
\NormalTok{df\_amostras }\OtherTok{\textless{}{-}} \FunctionTok{data.frame}\NormalTok{(}
  \AttributeTok{valores =} \FunctionTok{c}\NormalTok{(amostra1, amostra2),}
  \AttributeTok{grupo =} \FunctionTok{factor}\NormalTok{(}\FunctionTok{rep}\NormalTok{(}\FunctionTok{c}\NormalTok{(}\StringTok{"Antes"}\NormalTok{, }\StringTok{"Depois"}\NormalTok{), }\FunctionTok{c}\NormalTok{(n1, n2)),}
                \AttributeTok{levels =} \FunctionTok{c}\NormalTok{(}\StringTok{"Antes"}\NormalTok{, }\StringTok{"Depois"}\NormalTok{))}
\NormalTok{)}

\NormalTok{p3 }\OtherTok{\textless{}{-}} \FunctionTok{ggplot}\NormalTok{(df\_amostras, }\FunctionTok{aes}\NormalTok{(}\AttributeTok{x =}\NormalTok{ grupo, }\AttributeTok{y =}\NormalTok{ valores, }\AttributeTok{fill =}\NormalTok{ grupo)) }\SpecialCharTok{+}
  \FunctionTok{geom\_boxplot}\NormalTok{(}\AttributeTok{alpha =} \FloatTok{0.7}\NormalTok{, }\AttributeTok{outlier.alpha =} \FloatTok{0.6}\NormalTok{) }\SpecialCharTok{+}
  \FunctionTok{geom\_point}\NormalTok{(}\AttributeTok{position =} \FunctionTok{position\_jitter}\NormalTok{(}\AttributeTok{width =} \FloatTok{0.2}\NormalTok{), }\AttributeTok{alpha =} \FloatTok{0.4}\NormalTok{, }\AttributeTok{size =} \DecValTok{1}\NormalTok{) }\SpecialCharTok{+}
  \FunctionTok{stat\_summary}\NormalTok{(}\AttributeTok{fun =}\NormalTok{ mean, }\AttributeTok{geom =} \StringTok{"point"}\NormalTok{, }\AttributeTok{shape =} \DecValTok{23}\NormalTok{, }\AttributeTok{size =} \DecValTok{5}\NormalTok{, }
               \AttributeTok{fill =} \StringTok{"red"}\NormalTok{, }\AttributeTok{color =} \StringTok{"black"}\NormalTok{) }\SpecialCharTok{+}
  \FunctionTok{scale\_fill\_manual}\NormalTok{(}\AttributeTok{values =} \FunctionTok{c}\NormalTok{(}\StringTok{"Antes"} \OtherTok{=} \StringTok{"lightblue"}\NormalTok{, }\StringTok{"Depois"} \OtherTok{=} \StringTok{"lightcoral"}\NormalTok{)) }\SpecialCharTok{+}
  \FunctionTok{labs}\NormalTok{(}
    \AttributeTok{title =} \StringTok{"Comparação: Número de Denúncias Antes vs Depois"}\NormalTok{,}
    \AttributeTok{subtitle =} \FunctionTok{paste}\NormalTok{(}\StringTok{"Antes: x̄₁ ="}\NormalTok{, }\FunctionTok{round}\NormalTok{(}\FunctionTok{mean}\NormalTok{(amostra1), }\DecValTok{2}\NormalTok{), }\StringTok{"(n₁ ="}\NormalTok{, n1, }\StringTok{")"}\NormalTok{,}
                    \StringTok{"| Depois: x̄₂ ="}\NormalTok{, }\FunctionTok{round}\NormalTok{(}\FunctionTok{mean}\NormalTok{(amostra2), }\DecValTok{2}\NormalTok{), }\StringTok{"(n₂ ="}\NormalTok{, n2, }\StringTok{")"}\NormalTok{,}
                    \StringTok{"| Diferença ="}\NormalTok{, }\FunctionTok{round}\NormalTok{(diff\_observada, }\DecValTok{2}\NormalTok{)),}
    \AttributeTok{x =} \StringTok{"Período"}\NormalTok{,}
    \AttributeTok{y =} \StringTok{"Número de Denúncias/mês"}\NormalTok{,}
    \AttributeTok{fill =} \StringTok{"Período"}\NormalTok{,}
    \AttributeTok{caption =} \FunctionTok{paste}\NormalTok{(}\StringTok{"Losango vermelho = média | NC = 99.9\% | α = 0.001"}\NormalTok{)}
\NormalTok{  ) }\SpecialCharTok{+}
  \FunctionTok{theme\_minimal}\NormalTok{() }\SpecialCharTok{+}
  \FunctionTok{theme}\NormalTok{(}
    \AttributeTok{plot.title =} \FunctionTok{element\_text}\NormalTok{(}\AttributeTok{hjust =} \FloatTok{0.5}\NormalTok{, }\AttributeTok{size =} \DecValTok{14}\NormalTok{, }\AttributeTok{face =} \StringTok{"bold"}\NormalTok{),}
    \AttributeTok{plot.subtitle =} \FunctionTok{element\_text}\NormalTok{(}\AttributeTok{hjust =} \FloatTok{0.5}\NormalTok{, }\AttributeTok{size =} \DecValTok{12}\NormalTok{),}
    \AttributeTok{plot.caption =} \FunctionTok{element\_text}\NormalTok{(}\AttributeTok{hjust =} \FloatTok{0.5}\NormalTok{, }\AttributeTok{size =} \DecValTok{10}\NormalTok{),}
    \AttributeTok{legend.position =} \StringTok{"none"}
\NormalTok{  )}

\FunctionTok{print}\NormalTok{(p3)}

\CommentTok{\# ==============================================================================}
\CommentTok{\# RELATÓRIO FINAL COM NC = 99.9\%}
\CommentTok{\# ==============================================================================}

\FunctionTok{cat}\NormalTok{(}\StringTok{"RELATÓRIO FINAL {-} TESTE BOOTSTRAP (NC = 99.9\%)}\SpecialCharTok{\textbackslash{}n}\StringTok{"}\NormalTok{)}
\FunctionTok{cat}\NormalTok{(}\StringTok{"=============================================}\SpecialCharTok{\textbackslash{}n}\StringTok{"}\NormalTok{)}
\FunctionTok{cat}\NormalTok{(}\StringTok{"Estudo: Oltramari {-} Número de Denúncias Antes vs Depois}\SpecialCharTok{\textbackslash{}n}\StringTok{"}\NormalTok{)}
\FunctionTok{cat}\NormalTok{(}\StringTok{"Método: Bootstrap para diferença entre duas médias}\SpecialCharTok{\textbackslash{}n}\StringTok{"}\NormalTok{)}
\FunctionTok{cat}\NormalTok{(}\StringTok{"Nível de Confiança: 99.9\% (α = 0.001)}\SpecialCharTok{\textbackslash{}n\textbackslash{}n}\StringTok{"}\NormalTok{)}

\FunctionTok{cat}\NormalTok{(}\StringTok{"DADOS:}\SpecialCharTok{\textbackslash{}n}\StringTok{"}\NormalTok{)}
\FunctionTok{cat}\NormalTok{(}\StringTok{"Grupo Antes: n₁ ="}\NormalTok{, n1, }\StringTok{", x̄₁ ="}\NormalTok{, x1\_bar, }\StringTok{"denúncias/mês}\SpecialCharTok{\textbackslash{}n}\StringTok{"}\NormalTok{)}
\FunctionTok{cat}\NormalTok{(}\StringTok{"Grupo Depois: n₂ ="}\NormalTok{, n2, }\StringTok{", x̄₂ ="}\NormalTok{, x2\_bar, }\StringTok{"denúncias/mês}\SpecialCharTok{\textbackslash{}n}\StringTok{"}\NormalTok{)}
\FunctionTok{cat}\NormalTok{(}\StringTok{"Diferença observada:"}\NormalTok{, }\FunctionTok{round}\NormalTok{(diff\_observada, }\DecValTok{3}\NormalTok{), }\StringTok{"denúncias/mês}\SpecialCharTok{\textbackslash{}n}\StringTok{"}\NormalTok{)}
\FunctionTok{cat}\NormalTok{(}\StringTok{"Reamostragens bootstrap:"}\NormalTok{, }\FunctionTok{format}\NormalTok{(n\_bootstrap, }\AttributeTok{big.mark =} \StringTok{","}\NormalTok{), }\StringTok{"}\SpecialCharTok{\textbackslash{}n\textbackslash{}n}\StringTok{"}\NormalTok{)}

\FunctionTok{cat}\NormalTok{(}\StringTok{"TESTE DE HIPÓTESE:}\SpecialCharTok{\textbackslash{}n}\StringTok{"}\NormalTok{)}
\FunctionTok{cat}\NormalTok{(}\StringTok{"H₀: μ₂ {-} μ₁ = 0 (não há diferença)}\SpecialCharTok{\textbackslash{}n}\StringTok{"}\NormalTok{)}
\FunctionTok{cat}\NormalTok{(}\StringTok{"H₁: μ₂ {-} μ₁ ≠ 0 (há diferença)}\SpecialCharTok{\textbackslash{}n}\StringTok{"}\NormalTok{)}
\FunctionTok{cat}\NormalTok{(}\StringTok{"Valor P (bilateral):"}\NormalTok{, }\FunctionTok{ifelse}\NormalTok{(p\_value\_bilateral }\SpecialCharTok{\textless{}} \FloatTok{0.001}\NormalTok{, }\StringTok{"\textless{} 0.001"}\NormalTok{, }
                                  \FunctionTok{round}\NormalTok{(p\_value\_bilateral, }\DecValTok{6}\NormalTok{)), }\StringTok{"}\SpecialCharTok{\textbackslash{}n}\StringTok{"}\NormalTok{)}
\FunctionTok{cat}\NormalTok{(}\StringTok{"Nível de significância:"}\NormalTok{, alpha, }\StringTok{"}\SpecialCharTok{\textbackslash{}n}\StringTok{"}\NormalTok{)}
\FunctionTok{cat}\NormalTok{(}\StringTok{"Decisão:"}\NormalTok{, }\FunctionTok{ifelse}\NormalTok{(p\_value\_bilateral }\SpecialCharTok{\textless{}}\NormalTok{ alpha, }\StringTok{"REJEITA H₀"}\NormalTok{, }\StringTok{"NÃO REJEITA H₀"}\NormalTok{), }\StringTok{"}\SpecialCharTok{\textbackslash{}n\textbackslash{}n}\StringTok{"}\NormalTok{)}

\FunctionTok{cat}\NormalTok{(}\StringTok{"INTERVALOS DE CONFIANÇA:}\SpecialCharTok{\textbackslash{}n}\StringTok{"}\NormalTok{)}
\FunctionTok{cat}\NormalTok{(}\StringTok{"IC 95\%:  ["}\NormalTok{, }\FunctionTok{round}\NormalTok{(ic\_95\_percentil[}\DecValTok{1}\NormalTok{], }\DecValTok{3}\NormalTok{), }\StringTok{","}\NormalTok{, }
    \FunctionTok{round}\NormalTok{(ic\_95\_percentil[}\DecValTok{2}\NormalTok{], }\DecValTok{3}\NormalTok{), }\StringTok{"] denúncias/mês}\SpecialCharTok{\textbackslash{}n}\StringTok{"}\NormalTok{)}
\FunctionTok{cat}\NormalTok{(}\StringTok{"IC 99\%:  ["}\NormalTok{, }\FunctionTok{round}\NormalTok{(ic\_99\_percentil[}\DecValTok{1}\NormalTok{], }\DecValTok{3}\NormalTok{), }\StringTok{","}\NormalTok{, }
    \FunctionTok{round}\NormalTok{(ic\_99\_percentil[}\DecValTok{2}\NormalTok{], }\DecValTok{3}\NormalTok{), }\StringTok{"] denúncias/mês}\SpecialCharTok{\textbackslash{}n}\StringTok{"}\NormalTok{)}
\FunctionTok{cat}\NormalTok{(}\StringTok{"IC 99.9\%:["}\NormalTok{, }\FunctionTok{round}\NormalTok{(ic\_999\_percentil[}\DecValTok{1}\NormalTok{], }\DecValTok{3}\NormalTok{), }\StringTok{","}\NormalTok{, }
    \FunctionTok{round}\NormalTok{(ic\_999\_percentil[}\DecValTok{2}\NormalTok{], }\DecValTok{3}\NormalTok{), }\StringTok{"] denúncias/mês}\SpecialCharTok{\textbackslash{}n\textbackslash{}n}\StringTok{"}\NormalTok{)}

\FunctionTok{cat}\NormalTok{(}\StringTok{"INTERPRETAÇÃO (NC = 99.9\%):}\SpecialCharTok{\textbackslash{}n}\StringTok{"}\NormalTok{)}
\ControlFlowTok{if}\NormalTok{(p\_value\_bilateral }\SpecialCharTok{\textless{}}\NormalTok{ alpha) \{}
  \FunctionTok{cat}\NormalTok{(}\StringTok{"Com 99.9\% de confiança, há evidência estatística MUITO FORTE}\SpecialCharTok{\textbackslash{}n}\StringTok{"}\NormalTok{)}
  \FunctionTok{cat}\NormalTok{(}\StringTok{"de que o número de denúncias após a intervenção é diferente}\SpecialCharTok{\textbackslash{}n}\StringTok{"}\NormalTok{)}
  \FunctionTok{cat}\NormalTok{(}\StringTok{"do período anterior.}\SpecialCharTok{\textbackslash{}n}\StringTok{"}\NormalTok{)}
  \ControlFlowTok{if}\NormalTok{(diff\_observada }\SpecialCharTok{\textgreater{}} \DecValTok{0}\NormalTok{) \{}
    \FunctionTok{cat}\NormalTok{(}\StringTok{"}\SpecialCharTok{\textbackslash{}n}\StringTok{Especificamente, houve um AUMENTO SIGNIFICATIVO de aproximadamente}\SpecialCharTok{\textbackslash{}n}\StringTok{"}\NormalTok{)}
    \FunctionTok{cat}\NormalTok{(}\FunctionTok{round}\NormalTok{(diff\_observada, }\DecValTok{2}\NormalTok{), }\StringTok{"denúncias/mês após a intervenção.}\SpecialCharTok{\textbackslash{}n}\StringTok{"}\NormalTok{)}
    \FunctionTok{cat}\NormalTok{(}\StringTok{"Este aumento é estatisticamente significativo mesmo ao nível}\SpecialCharTok{\textbackslash{}n}\StringTok{"}\NormalTok{)}
    \FunctionTok{cat}\NormalTok{(}\StringTok{"de confiança extremamente rigoroso de 99.9\%.}\SpecialCharTok{\textbackslash{}n}\StringTok{"}\NormalTok{)}
\NormalTok{  \}}
\NormalTok{\} }\ControlFlowTok{else}\NormalTok{ \{}
  \FunctionTok{cat}\NormalTok{(}\StringTok{"Mesmo com o nível de confiança rigoroso de 99.9\%, não há}\SpecialCharTok{\textbackslash{}n}\StringTok{"}\NormalTok{)}
  \FunctionTok{cat}\NormalTok{(}\StringTok{"evidência estatística suficiente para concluir que houve}\SpecialCharTok{\textbackslash{}n}\StringTok{"}\NormalTok{)}
  \FunctionTok{cat}\NormalTok{(}\StringTok{"mudança significativa no número de denúncias.}\SpecialCharTok{\textbackslash{}n}\StringTok{"}\NormalTok{)}
\NormalTok{\}}

\CommentTok{\# Estatísticas descritivas adicionais}
\FunctionTok{cat}\NormalTok{(}\StringTok{"}\SpecialCharTok{\textbackslash{}n}\StringTok{ESTATÍSTICAS DESCRITIVAS:}\SpecialCharTok{\textbackslash{}n}\StringTok{"}\NormalTok{)}
\FunctionTok{cat}\NormalTok{(}\StringTok{"Desvio{-}padrão bootstrap:"}\NormalTok{, }\FunctionTok{round}\NormalTok{(sd\_boot, }\DecValTok{3}\NormalTok{), }\StringTok{"}\SpecialCharTok{\textbackslash{}n}\StringTok{"}\NormalTok{)}
\FunctionTok{cat}\NormalTok{(}\StringTok{"Erro padrão da diferença:"}\NormalTok{, }\FunctionTok{round}\NormalTok{(sd\_boot, }\DecValTok{3}\NormalTok{), }\StringTok{"}\SpecialCharTok{\textbackslash{}n}\StringTok{"}\NormalTok{)}
\FunctionTok{cat}\NormalTok{(}\StringTok{"Coeficiente de variação:"}\NormalTok{, }\FunctionTok{round}\NormalTok{(sd\_boot}\SpecialCharTok{/}\FunctionTok{abs}\NormalTok{(media\_boot) }\SpecialCharTok{*} \DecValTok{100}\NormalTok{, }\DecValTok{1}\NormalTok{), }\StringTok{"\%}\SpecialCharTok{\textbackslash{}n}\StringTok{"}\NormalTok{)}

\CommentTok{\# Salvando resultados (opcional)}
\NormalTok{resultados\_bootstrap }\OtherTok{\textless{}{-}} \FunctionTok{data.frame}\NormalTok{(}
  \AttributeTok{diferenca\_bootstrap =}\NormalTok{ boot\_diffs,}
  \AttributeTok{diferenca\_h0 =}\NormalTok{ boot\_diffs\_h0[}\DecValTok{1}\SpecialCharTok{:}\FunctionTok{length}\NormalTok{(boot\_diffs)]}
\NormalTok{)}

\CommentTok{\# write.csv(resultados\_bootstrap, "oltramari\_bootstrap\_NC999.csv", row.names = FALSE)}
\FunctionTok{cat}\NormalTok{(}\StringTok{"}\SpecialCharTok{\textbackslash{}n}\StringTok{Análise concluída com sucesso!}\SpecialCharTok{\textbackslash{}n}\StringTok{"}\NormalTok{)}
\InformationTok{\textasciigrave{}\textasciigrave{}\textasciigrave{}}
\end{Highlighting}
\end{Shaded}

\begin{verbatim}
TESTE BOOTSTRAP PARA DIFERENÇA ENTRE DUAS MÉDIAS
===============================================
NÍVEL DE CONFIANÇA: 99.9% (α = 0.001)
Grupo 1 (Antes): x̄₁ = 3.95 denúncias/mês, n₁ = 63 
Grupo 2 (Depois): x̄₂ = 8.63 denúncias/mês, n₂ = 19 
Diferença observada (x̄₂ - x̄₁) = 4.68 

Verificação das médias simuladas:
Média simulada grupo 1: 3.95 (alvo: 3.95 )
Média simulada grupo 2: 8.63 (alvo: 8.63 )
DP grupo 1: 1.588 
DP grupo 2: 2.771 
Diferença simulada: 4.68 (alvo: 4.68 )

RESULTADOS DO BOOTSTRAP:
Número de reamostragens: 20000 
Diferença original: 4.68 
Média das diferenças bootstrap: 4.674 
Desvio-padrão das diferenças bootstrap: 0.652 

TESTE DE HIPÓTESE BOOTSTRAP:
H₀: μ₂ - μ₁ = 0 vs H₁: μ₂ - μ₁ ≠ 0
Nível de significância: α = 0.001 (NC = 99.9%)
Valor P (bilateral): 0 
Valor P (unilateral): 0 
Conclusão (α = 0.001): REJEITA H₀ 

INTERVALOS DE CONFIANÇA BOOTSTRAP:
IC 95% (Percentil): 3.395 a 5.946 
IC 99% (Percentil): 3 a 6.358 
IC 99.9% (Percentil): 2.611 a 6.784 
IC 99.9% (Básico): 6.91 a 11.47 
IC 99.9% (Percentil boot.ci): -2.11 a 2.45 

RELATÓRIO FINAL - TESTE BOOTSTRAP (NC = 99.9%)
=============================================
Estudo: Oltramari - Número de Denúncias Antes vs Depois
Método: Bootstrap para diferença entre duas médias
Nível de Confiança: 99.9% (α = 0.001)

DADOS:
Grupo Antes: n₁ = 63 , x̄₁ = 3.95 denúncias/mês
Grupo Depois: n₂ = 19 , x̄₂ = 8.63 denúncias/mês
Diferença observada: 4.68 denúncias/mês
Reamostragens bootstrap: 20,000 

TESTE DE HIPÓTESE:
H₀: μ₂ - μ₁ = 0 (não há diferença)
H₁: μ₂ - μ₁ ≠ 0 (há diferença)
Valor P (bilateral): < 0.001 
Nível de significância: 0.001 
Decisão: REJEITA H₀ 

INTERVALOS DE CONFIANÇA:
IC 95%:  [ 3.395 , 5.946 ] denúncias/mês
IC 99%:  [ 3 , 6.358 ] denúncias/mês
IC 99.9%:[ 2.611 , 6.784 ] denúncias/mês

INTERPRETAÇÃO (NC = 99.9%):
Com 99.9% de confiança, há evidência estatística MUITO FORTE
de que o número de denúncias após a intervenção é diferente
do período anterior.

Especificamente, houve um AUMENTO SIGNIFICATIVO de aproximadamente
4.68 denúncias/mês após a intervenção.
Este aumento é estatisticamente significativo mesmo ao nível
de confiança extremamente rigoroso de 99.9%.

ESTATÍSTICAS DESCRITIVAS:
Desvio-padrão bootstrap: 0.652 
Erro padrão da diferença: 0.652 
Coeficiente de variação: 13.9 %

Análise concluída com sucesso!
\end{verbatim}

\pandocbounded{\includegraphics[keepaspectratio]{cap17-TSHo-Oltramari_files/figure-pdf/unnamed-chunk-2-1.pdf}}

\pandocbounded{\includegraphics[keepaspectratio]{cap17-TSHo-Oltramari_files/figure-pdf/unnamed-chunk-2-2.pdf}}

\pandocbounded{\includegraphics[keepaspectratio]{cap17-TSHo-Oltramari_files/figure-pdf/unnamed-chunk-2-3.pdf}}

Este script R fornece uma análise completa do teste bootstrap para
diferença entre duas médias, incluindo:

\begin{enumerate}
\def\labelenumi{\arabic{enumi}.}
\item
  \textbf{Simulação dos dados} baseada nas médias observadas
\item
  \textbf{Bootstrap das diferenças} entre médias
\item
  \textbf{Teste de hipótese} usando bootstrap sob H₀
\item
  \textbf{Intervalos de confiança} (percentil e BCa)
\item
  \textbf{Gráficos informativos} (histogramas, boxplots)
\item
  \textbf{Relatório interpretativo} dos resultados
\end{enumerate}

O script está configurado para os dados específicos mencionados (x₁̄ =
3.95, n₁ = 63, x₂̄ = 8.63, n₂ = 19) e fornece uma análise estatística
robusta usando métodos não-paramétricos bootstrap.

\begin{quote}
O teste de diferença entre essas duas médias \textbf{corroborou}, pela
\textbf{significância estatística} alcançada, a decisão no sentido de
que a hipótese nula (de que a Res. CPJ nº 04/2022 \emph{não} produziria
qualquer efeito sobre o número de denúncias oferecidas pelo MPGO em
AJCMs) restou \textbf{rejeitada}. (OLTRAMARI, 2024 , 157)
\end{quote}

Ou seja, o \ul{\textbf{Tratamento}} (Res. CPJ nº 04/2022) foi
\textbf{\emph{efetivo}} quanto ao aumento da média do \ul{\textbf{número
de denúncias por mês}} \ul{\textbf{antes}} (249 em 63 meses) e número de
denúncias por mês \ul{\textbf{após}} (164 em 19 meses) em relação ao
total de denúncias oferecidas em AJCMs (413) no período de jan. 2017 até
out. 2023. Isso para um Nível de Confiança de 99\%, um Erro Tipo I =
1\%, com valor P = 0,000046 \textless{} 0.01 = 1\%.

\subsection{Teste tab. 37: Denuncia e Arquivamento em 2022 e
2023}\label{teste-tab.-37-denuncia-e-arquivamento-em-2022-e-2023}

Replicar a tabela 37 (Oltramari, 2024, p.~161). (OLTRAMARI, 2024 ,
p.~161)

Acrescentar análise descritiva com tabelas resumo e gráficos.

Realizar teste de significância estatística para as relações observadas.

\begin{Shaded}
\begin{Highlighting}[numbers=left,,]
\InformationTok{\textasciigrave{}\textasciigrave{}\textasciigrave{}\{r\}}
\CommentTok{\# SCRIPT R PARA REPLICAR TABELA: AJCMs Promotorias Capital vs Interior}
\CommentTok{\# Com categorias: Providência (Denúncia/Arquivamento) e Ano (2022/2023)}
\CommentTok{\# Baseado em: Oltramari{-}AJCMs{-}Promotorias{-}capital{-}interior{-}30mar2022{-}31out2023.png}

\CommentTok{\# Carregando bibliotecas necessárias}
\FunctionTok{library}\NormalTok{(dplyr)}
\FunctionTok{library}\NormalTok{(tidyr)}
\FunctionTok{library}\NormalTok{(knitr)}
\FunctionTok{library}\NormalTok{(kableExtra)}
\FunctionTok{library}\NormalTok{(ggplot2)}
\FunctionTok{library}\NormalTok{(gridExtra)}
\FunctionTok{library}\NormalTok{(scales)}
\FunctionTok{library}\NormalTok{(tibble)  }\CommentTok{\# Adicionado para column\_to\_rownames}

\CommentTok{\# ==============================================================================}
\CommentTok{\# DADOS DA TABELA COMPLETA (AJUSTAR VALORES CONFORME A IMAGEM)}
\CommentTok{\# ==============================================================================}

\CommentTok{\# Criando a tabela com todas as dimensões: Localização x Providência x Ano}
\NormalTok{dados\_ajcm\_completo }\OtherTok{\textless{}{-}} \FunctionTok{data.frame}\NormalTok{(}
\NormalTok{  Localização }\OtherTok{=} \FunctionTok{rep}\NormalTok{(}\FunctionTok{c}\NormalTok{(}\StringTok{"Capital"}\NormalTok{, }\StringTok{"Interior"}\NormalTok{), }\AttributeTok{each =} \DecValTok{4}\NormalTok{),}
  \AttributeTok{Ano =} \FunctionTok{rep}\NormalTok{(}\FunctionTok{c}\NormalTok{(}\StringTok{"2022"}\NormalTok{, }\StringTok{"2023"}\NormalTok{), }\AttributeTok{each =} \DecValTok{2}\NormalTok{, }\AttributeTok{times =} \DecValTok{2}\NormalTok{),}
\NormalTok{  Providência }\OtherTok{=} \FunctionTok{rep}\NormalTok{(}\FunctionTok{c}\NormalTok{(}\StringTok{"Denúncia"}\NormalTok{, }\StringTok{"Arquivamento"}\NormalTok{), }\AttributeTok{times =} \DecValTok{4}\NormalTok{),}
  
  \CommentTok{\# AJUSTAR ESTES VALORES CONFORME A IMAGEM ESPECÍFICA}
  \AttributeTok{Quantidade =} \FunctionTok{c}\NormalTok{(}
    \CommentTok{\# Capital 2022}
    \DecValTok{20}\NormalTok{, }\DecValTok{191}\NormalTok{,  }\CommentTok{\# Denúncia, Arquivamento}
    \CommentTok{\# Capital 2023  }
    \DecValTok{50}\NormalTok{, }\DecValTok{595}\NormalTok{,  }\CommentTok{\# Denúncia, Arquivamento}
    \CommentTok{\# Interior 2022}
    \DecValTok{42}\NormalTok{, }\DecValTok{207}\NormalTok{,  }\CommentTok{\# Denúncia, Arquivamento  }
    \CommentTok{\# Interior 2023}
    \DecValTok{41}\NormalTok{, }\DecValTok{479}   \CommentTok{\# Denúncia, Arquivamento}
\NormalTok{  ),}
  
  \AttributeTok{stringsAsFactors =} \ConstantTok{FALSE}
\NormalTok{)}

\FunctionTok{cat}\NormalTok{(}\StringTok{"TABELA: AJCMs nas Promotorias por Localização, Providência e Ano}\SpecialCharTok{\textbackslash{}n}\StringTok{"}\NormalTok{)}
\FunctionTok{cat}\NormalTok{(}\StringTok{"Período: 30/mar/2022 a 31/out/2023}\SpecialCharTok{\textbackslash{}n}\StringTok{"}\NormalTok{)}
\FunctionTok{cat}\NormalTok{(}\StringTok{"==============================================================}\SpecialCharTok{\textbackslash{}n\textbackslash{}n}\StringTok{"}\NormalTok{)}

\CommentTok{\# ==============================================================================}
\CommentTok{\# EXIBINDO OS DADOS BRUTOS}
\CommentTok{\# ==============================================================================}

\FunctionTok{print}\NormalTok{(}\StringTok{"DADOS COMPLETOS (AJUSTAR CONFORME IMAGEM):"}\NormalTok{)}
\FunctionTok{print}\NormalTok{(dados\_ajcm\_completo)}
\FunctionTok{cat}\NormalTok{(}\StringTok{"}\SpecialCharTok{\textbackslash{}n}\StringTok{"}\NormalTok{)}

\CommentTok{\# ==============================================================================}
\CommentTok{\# CRIANDO TABELA AGREGADA POR LOCALIZAÇÃO E PROVIDÊNCIA}
\CommentTok{\# ==============================================================================}

\NormalTok{tabela\_loc\_prov }\OtherTok{\textless{}{-}}\NormalTok{ dados\_ajcm\_completo }\SpecialCharTok{\%\textgreater{}\%}
  \FunctionTok{group\_by}\NormalTok{(Localização, Providência) }\SpecialCharTok{\%\textgreater{}\%}
  \FunctionTok{summarise}\NormalTok{(}\AttributeTok{Total =} \FunctionTok{sum}\NormalTok{(Quantidade), }\AttributeTok{.groups =} \StringTok{"drop"}\NormalTok{) }\SpecialCharTok{\%\textgreater{}\%}
  \FunctionTok{pivot\_wider}\NormalTok{(}\AttributeTok{names\_from =}\NormalTok{ Providência, }\AttributeTok{values\_from =}\NormalTok{ Total) }\SpecialCharTok{\%\textgreater{}\%}
  \FunctionTok{mutate}\NormalTok{(}
    \AttributeTok{Total\_Geral =}\NormalTok{ Denúncia }\SpecialCharTok{+}\NormalTok{ Arquivamento,}
\NormalTok{    Perc\_Denúncia }\OtherTok{=} \FunctionTok{round}\NormalTok{(Denúncia }\SpecialCharTok{/}\NormalTok{ Total\_Geral }\SpecialCharTok{*} \DecValTok{100}\NormalTok{, }\DecValTok{1}\NormalTok{),}
    \AttributeTok{Perc\_Arquivamento =} \FunctionTok{round}\NormalTok{(Arquivamento }\SpecialCharTok{/}\NormalTok{ Total\_Geral }\SpecialCharTok{*} \DecValTok{100}\NormalTok{, }\DecValTok{1}\NormalTok{)}
\NormalTok{  )}

\FunctionTok{cat}\NormalTok{(}\StringTok{"TABELA POR LOCALIZAÇÃO E PROVIDÊNCIA:}\SpecialCharTok{\textbackslash{}n}\StringTok{"}\NormalTok{)}
\FunctionTok{print}\NormalTok{(tabela\_loc\_prov)}
\FunctionTok{cat}\NormalTok{(}\StringTok{"}\SpecialCharTok{\textbackslash{}n}\StringTok{"}\NormalTok{)}

\CommentTok{\# ==============================================================================}
\CommentTok{\# CRIANDO TABELA AGREGADA POR LOCALIZAÇÃO E ANO}
\CommentTok{\# ==============================================================================}

\NormalTok{tabela\_loc\_ano }\OtherTok{\textless{}{-}}\NormalTok{ dados\_ajcm\_completo }\SpecialCharTok{\%\textgreater{}\%}
  \FunctionTok{group\_by}\NormalTok{(Localização, Ano) }\SpecialCharTok{\%\textgreater{}\%}
  \FunctionTok{summarise}\NormalTok{(}\AttributeTok{Total =} \FunctionTok{sum}\NormalTok{(Quantidade), }\AttributeTok{.groups =} \StringTok{"drop"}\NormalTok{) }\SpecialCharTok{\%\textgreater{}\%}
  \FunctionTok{pivot\_wider}\NormalTok{(}\AttributeTok{names\_from =}\NormalTok{ Ano, }\AttributeTok{values\_from =}\NormalTok{ Total) }\SpecialCharTok{\%\textgreater{}\%}
  \FunctionTok{mutate}\NormalTok{(}
    \AttributeTok{Total\_Geral =} \StringTok{\textasciigrave{}}\AttributeTok{2022}\StringTok{\textasciigrave{}} \SpecialCharTok{+} \StringTok{\textasciigrave{}}\AttributeTok{2023}\StringTok{\textasciigrave{}}\NormalTok{,}
\NormalTok{    Variação }\OtherTok{=} \StringTok{\textasciigrave{}}\AttributeTok{2023}\StringTok{\textasciigrave{}} \SpecialCharTok{{-}} \StringTok{\textasciigrave{}}\AttributeTok{2022}\StringTok{\textasciigrave{}}\NormalTok{,}
\NormalTok{    Perc\_Variação }\OtherTok{=} \FunctionTok{round}\NormalTok{((}\StringTok{\textasciigrave{}}\AttributeTok{2023}\StringTok{\textasciigrave{}} \SpecialCharTok{{-}} \StringTok{\textasciigrave{}}\AttributeTok{2022}\StringTok{\textasciigrave{}}\NormalTok{) }\SpecialCharTok{/} \StringTok{\textasciigrave{}}\AttributeTok{2022}\StringTok{\textasciigrave{}} \SpecialCharTok{*} \DecValTok{100}\NormalTok{, }\DecValTok{1}\NormalTok{)}
\NormalTok{  )}

\FunctionTok{cat}\NormalTok{(}\StringTok{"TABELA POR LOCALIZAÇÃO E ANO:}\SpecialCharTok{\textbackslash{}n}\StringTok{"}\NormalTok{)}
\FunctionTok{print}\NormalTok{(tabela\_loc\_ano)}
\FunctionTok{cat}\NormalTok{(}\StringTok{"}\SpecialCharTok{\textbackslash{}n}\StringTok{"}\NormalTok{)}

\CommentTok{\# ==============================================================================}
\CommentTok{\# CRIANDO TABELA AGREGADA POR ANO E PROVIDÊNCIA}
\CommentTok{\# ==============================================================================}

\NormalTok{tabela\_ano\_prov }\OtherTok{\textless{}{-}}\NormalTok{ dados\_ajcm\_completo }\SpecialCharTok{\%\textgreater{}\%}
  \FunctionTok{group\_by}\NormalTok{(Ano, Providência) }\SpecialCharTok{\%\textgreater{}\%}
  \FunctionTok{summarise}\NormalTok{(}\AttributeTok{Total =} \FunctionTok{sum}\NormalTok{(Quantidade), }\AttributeTok{.groups =} \StringTok{"drop"}\NormalTok{) }\SpecialCharTok{\%\textgreater{}\%}
  \FunctionTok{pivot\_wider}\NormalTok{(}\AttributeTok{names\_from =}\NormalTok{ Providência, }\AttributeTok{values\_from =}\NormalTok{ Total) }\SpecialCharTok{\%\textgreater{}\%}
  \FunctionTok{mutate}\NormalTok{(}
    \AttributeTok{Total\_Geral =}\NormalTok{ Denúncia }\SpecialCharTok{+}\NormalTok{ Arquivamento,}
\NormalTok{    Perc\_Denúncia }\OtherTok{=} \FunctionTok{round}\NormalTok{(Denúncia }\SpecialCharTok{/}\NormalTok{ Total\_Geral }\SpecialCharTok{*} \DecValTok{100}\NormalTok{, }\DecValTok{1}\NormalTok{),}
    \AttributeTok{Perc\_Arquivamento =} \FunctionTok{round}\NormalTok{(Arquivamento }\SpecialCharTok{/}\NormalTok{ Total\_Geral }\SpecialCharTok{*} \DecValTok{100}\NormalTok{, }\DecValTok{1}\NormalTok{)}
\NormalTok{  )}

\FunctionTok{cat}\NormalTok{(}\StringTok{"TABELA POR ANO E PROVIDÊNCIA:}\SpecialCharTok{\textbackslash{}n}\StringTok{"}\NormalTok{)}
\FunctionTok{print}\NormalTok{(tabela\_ano\_prov)}
\FunctionTok{cat}\NormalTok{(}\StringTok{"}\SpecialCharTok{\textbackslash{}n}\StringTok{"}\NormalTok{)}

\CommentTok{\# ==============================================================================}
\CommentTok{\# TABELA CRUZADA COMPLETA (FORMATO MATRIZ)}
\CommentTok{\# ==============================================================================}

\CommentTok{\# Criando tabela no formato da imagem original}
\NormalTok{tabela\_cruzada }\OtherTok{\textless{}{-}}\NormalTok{ dados\_ajcm\_completo }\SpecialCharTok{\%\textgreater{}\%}
  \FunctionTok{unite}\NormalTok{(}\StringTok{"Ano\_Providência"}\NormalTok{, Ano, Providência, }\AttributeTok{sep =} \StringTok{"\_"}\NormalTok{) }\SpecialCharTok{\%\textgreater{}\%}
  \FunctionTok{pivot\_wider}\NormalTok{(}\AttributeTok{names\_from =}\NormalTok{ Ano\_Providência, }\AttributeTok{values\_from =}\NormalTok{ Quantidade) }\SpecialCharTok{\%\textgreater{}\%}
  \FunctionTok{mutate}\NormalTok{(}
    \AttributeTok{Total\_2022 =} \StringTok{\textasciigrave{}}\AttributeTok{2022\_Denúncia}\StringTok{\textasciigrave{}} \SpecialCharTok{+} \StringTok{\textasciigrave{}}\AttributeTok{2022\_Arquivamento}\StringTok{\textasciigrave{}}\NormalTok{,}
    \AttributeTok{Total\_2023 =} \StringTok{\textasciigrave{}}\AttributeTok{2023\_Denúncia}\StringTok{\textasciigrave{}} \SpecialCharTok{+} \StringTok{\textasciigrave{}}\AttributeTok{2023\_Arquivamento}\StringTok{\textasciigrave{}}\NormalTok{,}
\NormalTok{    Total\_Denúncia }\OtherTok{=} \StringTok{\textasciigrave{}}\AttributeTok{2022\_Denúncia}\StringTok{\textasciigrave{}} \SpecialCharTok{+} \StringTok{\textasciigrave{}}\AttributeTok{2023\_Denúncia}\StringTok{\textasciigrave{}}\NormalTok{,}
    \AttributeTok{Total\_Arquivamento =} \StringTok{\textasciigrave{}}\AttributeTok{2022\_Arquivamento}\StringTok{\textasciigrave{}} \SpecialCharTok{+} \StringTok{\textasciigrave{}}\AttributeTok{2023\_Arquivamento}\StringTok{\textasciigrave{}}\NormalTok{,}
    \AttributeTok{Total\_Geral =}\NormalTok{ Total\_2022 }\SpecialCharTok{+}\NormalTok{ Total\_2023}
\NormalTok{  )}

\FunctionTok{cat}\NormalTok{(}\StringTok{"TABELA CRUZADA COMPLETA:}\SpecialCharTok{\textbackslash{}n}\StringTok{"}\NormalTok{)}
\FunctionTok{print}\NormalTok{(tabela\_cruzada)}
\FunctionTok{cat}\NormalTok{(}\StringTok{"}\SpecialCharTok{\textbackslash{}n}\StringTok{"}\NormalTok{)}

\CommentTok{\# ==============================================================================}
\CommentTok{\# ANÁLISE ESTATÍSTICA DESCRITIVA}
\CommentTok{\# ==============================================================================}

\CommentTok{\# Totais gerais}
\NormalTok{total\_geral }\OtherTok{\textless{}{-}} \FunctionTok{sum}\NormalTok{(dados\_ajcm\_completo}\SpecialCharTok{$}\NormalTok{Quantidade)}
\NormalTok{total\_capital }\OtherTok{\textless{}{-}} \FunctionTok{sum}\NormalTok{(dados\_ajcm\_completo}\SpecialCharTok{$}\NormalTok{Quantidade[dados\_ajcm\_completo}\SpecialCharTok{$}\NormalTok{Localização }\SpecialCharTok{==} \StringTok{"Capital"}\NormalTok{])}
\NormalTok{total\_interior }\OtherTok{\textless{}{-}} \FunctionTok{sum}\NormalTok{(dados\_ajcm\_completo}\SpecialCharTok{$}\NormalTok{Quantidade[dados\_ajcm\_completo}\SpecialCharTok{$}\NormalTok{Localização }\SpecialCharTok{==} \StringTok{"Interior"}\NormalTok{])}
\NormalTok{total\_2022 }\OtherTok{\textless{}{-}} \FunctionTok{sum}\NormalTok{(dados\_ajcm\_completo}\SpecialCharTok{$}\NormalTok{Quantidade[dados\_ajcm\_completo}\SpecialCharTok{$}\NormalTok{Ano }\SpecialCharTok{==} \StringTok{"2022"}\NormalTok{])}
\NormalTok{total\_2023 }\OtherTok{\textless{}{-}} \FunctionTok{sum}\NormalTok{(dados\_ajcm\_completo}\SpecialCharTok{$}\NormalTok{Quantidade[dados\_ajcm\_completo}\SpecialCharTok{$}\NormalTok{Ano }\SpecialCharTok{==} \StringTok{"2023"}\NormalTok{])}
\NormalTok{total\_denuncia }\OtherTok{\textless{}{-}} \FunctionTok{sum}\NormalTok{(dados\_ajcm\_completo}\SpecialCharTok{$}\NormalTok{Quantidade[dados\_ajcm\_completo}\SpecialCharTok{$}\NormalTok{Providência }\SpecialCharTok{==} \StringTok{"Denúncia"}\NormalTok{])}
\NormalTok{total\_arquivamento }\OtherTok{\textless{}{-}} \FunctionTok{sum}\NormalTok{(dados\_ajcm\_completo}\SpecialCharTok{$}\NormalTok{Quantidade[dados\_ajcm\_completo}\SpecialCharTok{$}\NormalTok{Providência }\SpecialCharTok{==} \StringTok{"Arquivamento"}\NormalTok{])}

\FunctionTok{cat}\NormalTok{(}\StringTok{"ESTATÍSTICAS DESCRITIVAS:}\SpecialCharTok{\textbackslash{}n}\StringTok{"}\NormalTok{)}
\FunctionTok{cat}\NormalTok{(}\StringTok{"=========================}\SpecialCharTok{\textbackslash{}n}\StringTok{"}\NormalTok{)}
\FunctionTok{cat}\NormalTok{(}\StringTok{"Total Geral de AJCMs:"}\NormalTok{, }\FunctionTok{format}\NormalTok{(total\_geral, }\AttributeTok{big.mark =} \StringTok{"."}\NormalTok{), }\StringTok{"}\SpecialCharTok{\textbackslash{}n\textbackslash{}n}\StringTok{"}\NormalTok{)}

\FunctionTok{cat}\NormalTok{(}\StringTok{"Por Localização:}\SpecialCharTok{\textbackslash{}n}\StringTok{"}\NormalTok{)}
\FunctionTok{cat}\NormalTok{(}\StringTok{"  Capital:"}\NormalTok{, }\FunctionTok{format}\NormalTok{(total\_capital, }\AttributeTok{big.mark =} \StringTok{"."}\NormalTok{), }
    \StringTok{"("}\NormalTok{, }\FunctionTok{round}\NormalTok{(total\_capital}\SpecialCharTok{/}\NormalTok{total\_geral}\SpecialCharTok{*}\DecValTok{100}\NormalTok{, }\DecValTok{1}\NormalTok{), }\StringTok{"\%)}\SpecialCharTok{\textbackslash{}n}\StringTok{"}\NormalTok{)}
\FunctionTok{cat}\NormalTok{(}\StringTok{"  Interior:"}\NormalTok{, }\FunctionTok{format}\NormalTok{(total\_interior, }\AttributeTok{big.mark =} \StringTok{"."}\NormalTok{), }
    \StringTok{"("}\NormalTok{, }\FunctionTok{round}\NormalTok{(total\_interior}\SpecialCharTok{/}\NormalTok{total\_geral}\SpecialCharTok{*}\DecValTok{100}\NormalTok{, }\DecValTok{1}\NormalTok{), }\StringTok{"\%)}\SpecialCharTok{\textbackslash{}n\textbackslash{}n}\StringTok{"}\NormalTok{)}

\FunctionTok{cat}\NormalTok{(}\StringTok{"Por Ano:}\SpecialCharTok{\textbackslash{}n}\StringTok{"}\NormalTok{)}
\FunctionTok{cat}\NormalTok{(}\StringTok{"  2022:"}\NormalTok{, }\FunctionTok{format}\NormalTok{(total\_2022, }\AttributeTok{big.mark =} \StringTok{"."}\NormalTok{), }
    \StringTok{"("}\NormalTok{, }\FunctionTok{round}\NormalTok{(total\_2022}\SpecialCharTok{/}\NormalTok{total\_geral}\SpecialCharTok{*}\DecValTok{100}\NormalTok{, }\DecValTok{1}\NormalTok{), }\StringTok{"\%)}\SpecialCharTok{\textbackslash{}n}\StringTok{"}\NormalTok{)}
\FunctionTok{cat}\NormalTok{(}\StringTok{"  2023:"}\NormalTok{, }\FunctionTok{format}\NormalTok{(total\_2023, }\AttributeTok{big.mark =} \StringTok{"."}\NormalTok{), }
    \StringTok{"("}\NormalTok{, }\FunctionTok{round}\NormalTok{(total\_2023}\SpecialCharTok{/}\NormalTok{total\_geral}\SpecialCharTok{*}\DecValTok{100}\NormalTok{, }\DecValTok{1}\NormalTok{), }\StringTok{"\%)}\SpecialCharTok{\textbackslash{}n}\StringTok{"}\NormalTok{)}
\FunctionTok{cat}\NormalTok{(}\StringTok{"  Variação 2022→2023:"}\NormalTok{, }\FunctionTok{ifelse}\NormalTok{(total\_2023 }\SpecialCharTok{\textgreater{}}\NormalTok{ total\_2022, }\StringTok{"+"}\NormalTok{, }\StringTok{""}\NormalTok{), }
    \FunctionTok{round}\NormalTok{((total\_2023}\SpecialCharTok{{-}}\NormalTok{total\_2022)}\SpecialCharTok{/}\NormalTok{total\_2022}\SpecialCharTok{*}\DecValTok{100}\NormalTok{, }\DecValTok{1}\NormalTok{), }\StringTok{"\%}\SpecialCharTok{\textbackslash{}n\textbackslash{}n}\StringTok{"}\NormalTok{)}

\FunctionTok{cat}\NormalTok{(}\StringTok{"Por Providência:}\SpecialCharTok{\textbackslash{}n}\StringTok{"}\NormalTok{)}
\FunctionTok{cat}\NormalTok{(}\StringTok{"  Denúncia:"}\NormalTok{, }\FunctionTok{format}\NormalTok{(total\_denuncia, }\AttributeTok{big.mark =} \StringTok{"."}\NormalTok{), }
    \StringTok{"("}\NormalTok{, }\FunctionTok{round}\NormalTok{(total\_denuncia}\SpecialCharTok{/}\NormalTok{total\_geral}\SpecialCharTok{*}\DecValTok{100}\NormalTok{, }\DecValTok{1}\NormalTok{), }\StringTok{"\%)}\SpecialCharTok{\textbackslash{}n}\StringTok{"}\NormalTok{)}
\FunctionTok{cat}\NormalTok{(}\StringTok{"  Arquivamento:"}\NormalTok{, }\FunctionTok{format}\NormalTok{(total\_arquivamento, }\AttributeTok{big.mark =} \StringTok{"."}\NormalTok{), }
    \StringTok{"("}\NormalTok{, }\FunctionTok{round}\NormalTok{(total\_arquivamento}\SpecialCharTok{/}\NormalTok{total\_geral}\SpecialCharTok{*}\DecValTok{100}\NormalTok{, }\DecValTok{1}\NormalTok{), }\StringTok{"\%)}\SpecialCharTok{\textbackslash{}n\textbackslash{}n}\StringTok{"}\NormalTok{)}

\CommentTok{\# ==============================================================================}
\CommentTok{\# GRÁFICOS ATUALIZADOS}
\CommentTok{\# ==============================================================================}

\CommentTok{\# 1. Gráfico de barras agrupadas por todas as categorias}
\NormalTok{p1 }\OtherTok{\textless{}{-}} \FunctionTok{ggplot}\NormalTok{(dados\_ajcm\_completo, }\FunctionTok{aes}\NormalTok{(}\AttributeTok{x =}\NormalTok{ Localização, }\AttributeTok{y =}\NormalTok{ Quantidade, }
                                     \AttributeTok{fill =}\NormalTok{ Providência)) }\SpecialCharTok{+}
  \FunctionTok{geom\_bar}\NormalTok{(}\AttributeTok{stat =} \StringTok{"identity"}\NormalTok{, }\AttributeTok{position =} \StringTok{"dodge"}\NormalTok{, }\AttributeTok{alpha =} \FloatTok{0.8}\NormalTok{) }\SpecialCharTok{+}
  \FunctionTok{facet\_wrap}\NormalTok{(}\SpecialCharTok{\textasciitilde{}}\NormalTok{ Ano, }\AttributeTok{scales =} \StringTok{"free\_y"}\NormalTok{) }\SpecialCharTok{+}
  \FunctionTok{geom\_text}\NormalTok{(}\FunctionTok{aes}\NormalTok{(}\AttributeTok{label =}\NormalTok{ Quantidade), }
            \AttributeTok{position =} \FunctionTok{position\_dodge}\NormalTok{(}\AttributeTok{width =} \FloatTok{0.9}\NormalTok{), }\AttributeTok{vjust =} \SpecialCharTok{{-}}\FloatTok{0.5}\NormalTok{, }\AttributeTok{size =} \DecValTok{3}\NormalTok{) }\SpecialCharTok{+}
  \FunctionTok{scale\_fill\_manual}\NormalTok{(}\AttributeTok{values =} \FunctionTok{c}\NormalTok{(}\StringTok{"Denúncia"} \OtherTok{=} \StringTok{"\#e74c3c"}\NormalTok{, }\StringTok{"Arquivamento"} \OtherTok{=} \StringTok{"\#3498db"}\NormalTok{)) }\SpecialCharTok{+}
  \FunctionTok{scale\_y\_continuous}\NormalTok{(}\AttributeTok{labels =} \FunctionTok{comma\_format}\NormalTok{(}\AttributeTok{big.mark =} \StringTok{"."}\NormalTok{, }\AttributeTok{decimal.mark =} \StringTok{","}\NormalTok{)) }\SpecialCharTok{+}
  \FunctionTok{labs}\NormalTok{(}
    \AttributeTok{title =} \StringTok{"AJCMs por Localização, Providência e Ano"}\NormalTok{,}
    \AttributeTok{subtitle =} \StringTok{"Período: 30/mar/2022 a 31/out/2023"}\NormalTok{,}
    \AttributeTok{x =} \StringTok{"Localização"}\NormalTok{,}
    \AttributeTok{y =} \StringTok{"Número de AJCMs"}\NormalTok{,}
    \AttributeTok{fill =} \StringTok{"Providência"}\NormalTok{,}
    \AttributeTok{caption =} \StringTok{"Fonte: Dados Oltramari"}
\NormalTok{  ) }\SpecialCharTok{+}
  \FunctionTok{theme\_minimal}\NormalTok{() }\SpecialCharTok{+}
  \FunctionTok{theme}\NormalTok{(}
    \AttributeTok{plot.title =} \FunctionTok{element\_text}\NormalTok{(}\AttributeTok{hjust =} \FloatTok{0.5}\NormalTok{, }\AttributeTok{size =} \DecValTok{14}\NormalTok{, }\AttributeTok{face =} \StringTok{"bold"}\NormalTok{),}
    \AttributeTok{plot.subtitle =} \FunctionTok{element\_text}\NormalTok{(}\AttributeTok{hjust =} \FloatTok{0.5}\NormalTok{, }\AttributeTok{size =} \DecValTok{12}\NormalTok{),}
    \AttributeTok{strip.text =} \FunctionTok{element\_text}\NormalTok{(}\AttributeTok{size =} \DecValTok{12}\NormalTok{, }\AttributeTok{face =} \StringTok{"bold"}\NormalTok{)}
\NormalTok{  )}

\FunctionTok{print}\NormalTok{(p1)}

\CommentTok{\# 2. Gráfico de evolução temporal}
\NormalTok{dados\_evolucao }\OtherTok{\textless{}{-}}\NormalTok{ dados\_ajcm\_completo }\SpecialCharTok{\%\textgreater{}\%}
  \FunctionTok{group\_by}\NormalTok{(Localização, Ano) }\SpecialCharTok{\%\textgreater{}\%}
  \FunctionTok{summarise}\NormalTok{(}\AttributeTok{Total =} \FunctionTok{sum}\NormalTok{(Quantidade), }\AttributeTok{.groups =} \StringTok{"drop"}\NormalTok{)}

\NormalTok{p2 }\OtherTok{\textless{}{-}} \FunctionTok{ggplot}\NormalTok{(dados\_evolucao, }\FunctionTok{aes}\NormalTok{(}\AttributeTok{x =}\NormalTok{ Ano, }\AttributeTok{y =}\NormalTok{ Total, }\AttributeTok{color =}\NormalTok{ Localização, }\AttributeTok{group =}\NormalTok{ Localização)) }\SpecialCharTok{+}
  \FunctionTok{geom\_line}\NormalTok{(}\AttributeTok{size =} \FloatTok{1.5}\NormalTok{) }\SpecialCharTok{+}
  \FunctionTok{geom\_point}\NormalTok{(}\AttributeTok{size =} \DecValTok{4}\NormalTok{) }\SpecialCharTok{+}
  \FunctionTok{geom\_text}\NormalTok{(}\FunctionTok{aes}\NormalTok{(}\AttributeTok{label =}\NormalTok{ Total), }\AttributeTok{vjust =} \SpecialCharTok{{-}}\FloatTok{0.8}\NormalTok{, }\AttributeTok{size =} \DecValTok{4}\NormalTok{, }\AttributeTok{fontface =} \StringTok{"bold"}\NormalTok{) }\SpecialCharTok{+}
  \FunctionTok{scale\_color\_manual}\NormalTok{(}\AttributeTok{values =} \FunctionTok{c}\NormalTok{(}\StringTok{"Capital"} \OtherTok{=} \StringTok{"\#e74c3c"}\NormalTok{, }\StringTok{"Interior"} \OtherTok{=} \StringTok{"\#2ecc71"}\NormalTok{)) }\SpecialCharTok{+}
  \FunctionTok{scale\_y\_continuous}\NormalTok{(}\AttributeTok{labels =} \FunctionTok{comma\_format}\NormalTok{(}\AttributeTok{big.mark =} \StringTok{"."}\NormalTok{, }\AttributeTok{decimal.mark =} \StringTok{","}\NormalTok{)) }\SpecialCharTok{+}
  \FunctionTok{labs}\NormalTok{(}
    \AttributeTok{title =} \StringTok{"Evolução das AJCMs: 2022 vs 2023"}\NormalTok{,}
    \AttributeTok{subtitle =} \StringTok{"Comparação Capital vs Interior"}\NormalTok{,}
    \AttributeTok{x =} \StringTok{"Ano"}\NormalTok{,}
    \AttributeTok{y =} \StringTok{"Total de AJCMs"}\NormalTok{,}
    \AttributeTok{color =} \StringTok{"Localização"}\NormalTok{,}
    \AttributeTok{caption =} \StringTok{"Fonte: Dados Oltramari"}
\NormalTok{  ) }\SpecialCharTok{+}
  \FunctionTok{theme\_minimal}\NormalTok{() }\SpecialCharTok{+}
  \FunctionTok{theme}\NormalTok{(}
    \AttributeTok{plot.title =} \FunctionTok{element\_text}\NormalTok{(}\AttributeTok{hjust =} \FloatTok{0.5}\NormalTok{, }\AttributeTok{size =} \DecValTok{14}\NormalTok{, }\AttributeTok{face =} \StringTok{"bold"}\NormalTok{),}
    \AttributeTok{plot.subtitle =} \FunctionTok{element\_text}\NormalTok{(}\AttributeTok{hjust =} \FloatTok{0.5}\NormalTok{, }\AttributeTok{size =} \DecValTok{12}\NormalTok{),}
    \AttributeTok{legend.position =} \StringTok{"bottom"}
\NormalTok{  )}

\FunctionTok{print}\NormalTok{(p2)}

\CommentTok{\# 3. Gráfico de proporções por providência}
\NormalTok{dados\_prop }\OtherTok{\textless{}{-}}\NormalTok{ dados\_ajcm\_completo }\SpecialCharTok{\%\textgreater{}\%}
  \FunctionTok{group\_by}\NormalTok{(Localização, Providência) }\SpecialCharTok{\%\textgreater{}\%}
  \FunctionTok{summarise}\NormalTok{(}\AttributeTok{Total =} \FunctionTok{sum}\NormalTok{(Quantidade), }\AttributeTok{.groups =} \StringTok{"drop"}\NormalTok{) }\SpecialCharTok{\%\textgreater{}\%}
  \FunctionTok{group\_by}\NormalTok{(Localização) }\SpecialCharTok{\%\textgreater{}\%}
  \FunctionTok{mutate}\NormalTok{(Proporção }\OtherTok{=} \FunctionTok{round}\NormalTok{(Total }\SpecialCharTok{/} \FunctionTok{sum}\NormalTok{(Total) }\SpecialCharTok{*} \DecValTok{100}\NormalTok{, }\DecValTok{1}\NormalTok{))}

\NormalTok{p3 }\OtherTok{\textless{}{-}} \FunctionTok{ggplot}\NormalTok{(dados\_prop, }\FunctionTok{aes}\NormalTok{(}\AttributeTok{x =}\NormalTok{ Localização, }\AttributeTok{y =}\NormalTok{ Proporção, }\AttributeTok{fill =}\NormalTok{ Providência)) }\SpecialCharTok{+}
  \FunctionTok{geom\_bar}\NormalTok{(}\AttributeTok{stat =} \StringTok{"identity"}\NormalTok{, }\AttributeTok{position =} \StringTok{"stack"}\NormalTok{, }\AttributeTok{alpha =} \FloatTok{0.8}\NormalTok{) }\SpecialCharTok{+}
  \FunctionTok{geom\_text}\NormalTok{(}\FunctionTok{aes}\NormalTok{(}\AttributeTok{label =} \FunctionTok{paste0}\NormalTok{(Proporção, }\StringTok{"\%"}\NormalTok{)), }
            \AttributeTok{position =} \FunctionTok{position\_stack}\NormalTok{(}\AttributeTok{vjust =} \FloatTok{0.5}\NormalTok{), }\AttributeTok{size =} \DecValTok{4}\NormalTok{, }\AttributeTok{fontface =} \StringTok{"bold"}\NormalTok{) }\SpecialCharTok{+}
  \FunctionTok{scale\_fill\_manual}\NormalTok{(}\AttributeTok{values =} \FunctionTok{c}\NormalTok{(}\StringTok{"Denúncia"} \OtherTok{=} \StringTok{"\#e74c3c"}\NormalTok{, }\StringTok{"Arquivamento"} \OtherTok{=} \StringTok{"\#3498db"}\NormalTok{)) }\SpecialCharTok{+}
  \FunctionTok{scale\_y\_continuous}\NormalTok{(}\AttributeTok{breaks =} \FunctionTok{seq}\NormalTok{(}\DecValTok{0}\NormalTok{, }\DecValTok{100}\NormalTok{, }\DecValTok{25}\NormalTok{), }\AttributeTok{labels =} \FunctionTok{paste0}\NormalTok{(}\FunctionTok{seq}\NormalTok{(}\DecValTok{0}\NormalTok{, }\DecValTok{100}\NormalTok{, }\DecValTok{25}\NormalTok{), }\StringTok{"\%"}\NormalTok{)) }\SpecialCharTok{+}
  \FunctionTok{labs}\NormalTok{(}
    \AttributeTok{title =} \StringTok{"Proporção de Providências por Localização"}\NormalTok{,}
    \AttributeTok{subtitle =} \StringTok{"Distribuição: Denúncia vs Arquivamento"}\NormalTok{,}
    \AttributeTok{x =} \StringTok{"Localização"}\NormalTok{,}
    \AttributeTok{y =} \StringTok{"Proporção (\%)"}\NormalTok{,}
    \AttributeTok{fill =} \StringTok{"Providência"}\NormalTok{,}
    \AttributeTok{caption =} \StringTok{"Fonte: Dados Oltramari"}
\NormalTok{  ) }\SpecialCharTok{+}
  \FunctionTok{theme\_minimal}\NormalTok{() }\SpecialCharTok{+}
  \FunctionTok{theme}\NormalTok{(}
    \AttributeTok{plot.title =} \FunctionTok{element\_text}\NormalTok{(}\AttributeTok{hjust =} \FloatTok{0.5}\NormalTok{, }\AttributeTok{size =} \DecValTok{14}\NormalTok{, }\AttributeTok{face =} \StringTok{"bold"}\NormalTok{),}
    \AttributeTok{plot.subtitle =} \FunctionTok{element\_text}\NormalTok{(}\AttributeTok{hjust =} \FloatTok{0.5}\NormalTok{, }\AttributeTok{size =} \DecValTok{12}\NormalTok{),}
    \AttributeTok{legend.position =} \StringTok{"bottom"}
\NormalTok{  )}

\FunctionTok{print}\NormalTok{(p3)}

\CommentTok{\# ==============================================================================}
\CommentTok{\# TESTES ESTATÍSTICOS (CORRIGIDOS)}
\CommentTok{\# ==============================================================================}

\CommentTok{\# 1. Teste Qui{-}quadrado: Localização x Providência}
\CommentTok{\# Método alternativo sem column\_to\_rownames}
\NormalTok{dados\_loc\_prov }\OtherTok{\textless{}{-}}\NormalTok{ dados\_ajcm\_completo }\SpecialCharTok{\%\textgreater{}\%}
  \FunctionTok{group\_by}\NormalTok{(Localização, Providência) }\SpecialCharTok{\%\textgreater{}\%}
  \FunctionTok{summarise}\NormalTok{(}\AttributeTok{Total =} \FunctionTok{sum}\NormalTok{(Quantidade), }\AttributeTok{.groups =} \StringTok{"drop"}\NormalTok{) }\SpecialCharTok{\%\textgreater{}\%}
  \FunctionTok{pivot\_wider}\NormalTok{(}\AttributeTok{names\_from =}\NormalTok{ Providência, }\AttributeTok{values\_from =}\NormalTok{ Total)}

\CommentTok{\# Criando matriz manualmente}
\NormalTok{matriz\_loc\_prov }\OtherTok{\textless{}{-}} \FunctionTok{as.matrix}\NormalTok{(dados\_loc\_prov[, }\SpecialCharTok{{-}}\DecValTok{1}\NormalTok{])  }\CommentTok{\# Remove coluna Localização}
\FunctionTok{rownames}\NormalTok{(matriz\_loc\_prov) }\OtherTok{\textless{}{-}}\NormalTok{ dados\_loc\_prov}\SpecialCharTok{$}\NormalTok{Localização}

\NormalTok{teste\_qui\_loc\_prov }\OtherTok{\textless{}{-}} \FunctionTok{chisq.test}\NormalTok{(matriz\_loc\_prov)}

\FunctionTok{cat}\NormalTok{(}\StringTok{"TESTE QUI{-}QUADRADO: LOCALIZAÇÃO x PROVIDÊNCIA}\SpecialCharTok{\textbackslash{}n}\StringTok{"}\NormalTok{)}
\FunctionTok{cat}\NormalTok{(}\StringTok{"============================================}\SpecialCharTok{\textbackslash{}n}\StringTok{"}\NormalTok{)}
\FunctionTok{print}\NormalTok{(matriz\_loc\_prov)}
\FunctionTok{cat}\NormalTok{(}\StringTok{"}\SpecialCharTok{\textbackslash{}n}\StringTok{Qui{-}quadrado:"}\NormalTok{, }\FunctionTok{round}\NormalTok{(teste\_qui\_loc\_prov}\SpecialCharTok{$}\NormalTok{statistic, }\DecValTok{4}\NormalTok{), }\StringTok{"}\SpecialCharTok{\textbackslash{}n}\StringTok{"}\NormalTok{)}
\FunctionTok{cat}\NormalTok{(}\StringTok{"Valor P:"}\NormalTok{, }\FunctionTok{round}\NormalTok{(teste\_qui\_loc\_prov}\SpecialCharTok{$}\NormalTok{p.value, }\DecValTok{6}\NormalTok{), }\StringTok{"}\SpecialCharTok{\textbackslash{}n}\StringTok{"}\NormalTok{)}
\FunctionTok{cat}\NormalTok{(}\StringTok{"Conclusão:"}\NormalTok{, }\FunctionTok{ifelse}\NormalTok{(teste\_qui\_loc\_prov}\SpecialCharTok{$}\NormalTok{p.value }\SpecialCharTok{\textless{}} \FloatTok{0.05}\NormalTok{, }
                        \StringTok{"Há associação significativa"}\NormalTok{, }
                        \StringTok{"Não há associação significativa"}\NormalTok{), }\StringTok{"}\SpecialCharTok{\textbackslash{}n\textbackslash{}n}\StringTok{"}\NormalTok{)}

\CommentTok{\# 2. Teste Qui{-}quadrado: Ano x Providência}
\NormalTok{dados\_ano\_prov }\OtherTok{\textless{}{-}}\NormalTok{ dados\_ajcm\_completo }\SpecialCharTok{\%\textgreater{}\%}
  \FunctionTok{group\_by}\NormalTok{(Ano, Providência) }\SpecialCharTok{\%\textgreater{}\%}
  \FunctionTok{summarise}\NormalTok{(}\AttributeTok{Total =} \FunctionTok{sum}\NormalTok{(Quantidade), }\AttributeTok{.groups =} \StringTok{"drop"}\NormalTok{) }\SpecialCharTok{\%\textgreater{}\%}
  \FunctionTok{pivot\_wider}\NormalTok{(}\AttributeTok{names\_from =}\NormalTok{ Providência, }\AttributeTok{values\_from =}\NormalTok{ Total)}

\CommentTok{\# Criando matriz manualmente}
\NormalTok{matriz\_ano\_prov }\OtherTok{\textless{}{-}} \FunctionTok{as.matrix}\NormalTok{(dados\_ano\_prov[, }\SpecialCharTok{{-}}\DecValTok{1}\NormalTok{])  }\CommentTok{\# Remove coluna Ano}
\FunctionTok{rownames}\NormalTok{(matriz\_ano\_prov) }\OtherTok{\textless{}{-}}\NormalTok{ dados\_ano\_prov}\SpecialCharTok{$}\NormalTok{Ano}

\NormalTok{teste\_qui\_ano\_prov }\OtherTok{\textless{}{-}} \FunctionTok{chisq.test}\NormalTok{(matriz\_ano\_prov)}

\FunctionTok{cat}\NormalTok{(}\StringTok{"TESTE QUI{-}QUADRADO: ANO x PROVIDÊNCIA}\SpecialCharTok{\textbackslash{}n}\StringTok{"}\NormalTok{)}
\FunctionTok{cat}\NormalTok{(}\StringTok{"====================================}\SpecialCharTok{\textbackslash{}n}\StringTok{"}\NormalTok{)}
\FunctionTok{print}\NormalTok{(matriz\_ano\_prov)}
\FunctionTok{cat}\NormalTok{(}\StringTok{"}\SpecialCharTok{\textbackslash{}n}\StringTok{Qui{-}quadrado:"}\NormalTok{, }\FunctionTok{round}\NormalTok{(teste\_qui\_ano\_prov}\SpecialCharTok{$}\NormalTok{statistic, }\DecValTok{4}\NormalTok{), }\StringTok{"}\SpecialCharTok{\textbackslash{}n}\StringTok{"}\NormalTok{)}
\FunctionTok{cat}\NormalTok{(}\StringTok{"Valor P:"}\NormalTok{, }\FunctionTok{round}\NormalTok{(teste\_qui\_ano\_prov}\SpecialCharTok{$}\NormalTok{p.value, }\DecValTok{6}\NormalTok{), }\StringTok{"}\SpecialCharTok{\textbackslash{}n}\StringTok{"}\NormalTok{)}
\FunctionTok{cat}\NormalTok{(}\StringTok{"Conclusão:"}\NormalTok{, }\FunctionTok{ifelse}\NormalTok{(teste\_qui\_ano\_prov}\SpecialCharTok{$}\NormalTok{p.value }\SpecialCharTok{\textless{}} \FloatTok{0.05}\NormalTok{, }
                        \StringTok{"Há associação significativa"}\NormalTok{, }
                        \StringTok{"Não há associação significativa"}\NormalTok{), }\StringTok{"}\SpecialCharTok{\textbackslash{}n\textbackslash{}n}\StringTok{"}\NormalTok{)}

\CommentTok{\# 3. Teste Qui{-}quadrado: Localização x Ano}
\NormalTok{dados\_loc\_ano }\OtherTok{\textless{}{-}}\NormalTok{ dados\_ajcm\_completo }\SpecialCharTok{\%\textgreater{}\%}
  \FunctionTok{group\_by}\NormalTok{(Localização, Ano) }\SpecialCharTok{\%\textgreater{}\%}
  \FunctionTok{summarise}\NormalTok{(}\AttributeTok{Total =} \FunctionTok{sum}\NormalTok{(Quantidade), }\AttributeTok{.groups =} \StringTok{"drop"}\NormalTok{) }\SpecialCharTok{\%\textgreater{}\%}
  \FunctionTok{pivot\_wider}\NormalTok{(}\AttributeTok{names\_from =}\NormalTok{ Ano, }\AttributeTok{values\_from =}\NormalTok{ Total)}

\CommentTok{\# Criando matriz manualmente}
\NormalTok{matriz\_loc\_ano }\OtherTok{\textless{}{-}} \FunctionTok{as.matrix}\NormalTok{(dados\_loc\_ano[, }\SpecialCharTok{{-}}\DecValTok{1}\NormalTok{])  }\CommentTok{\# Remove coluna Localização}
\FunctionTok{rownames}\NormalTok{(matriz\_loc\_ano) }\OtherTok{\textless{}{-}}\NormalTok{ dados\_loc\_ano}\SpecialCharTok{$}\NormalTok{Localização}

\NormalTok{teste\_qui\_loc\_ano }\OtherTok{\textless{}{-}} \FunctionTok{chisq.test}\NormalTok{(matriz\_loc\_ano)}

\FunctionTok{cat}\NormalTok{(}\StringTok{"TESTE QUI{-}QUADRADO: LOCALIZAÇÃO x ANO}\SpecialCharTok{\textbackslash{}n}\StringTok{"}\NormalTok{)}
\FunctionTok{cat}\NormalTok{(}\StringTok{"====================================}\SpecialCharTok{\textbackslash{}n}\StringTok{"}\NormalTok{)}
\FunctionTok{print}\NormalTok{(matriz\_loc\_ano)}
\FunctionTok{cat}\NormalTok{(}\StringTok{"}\SpecialCharTok{\textbackslash{}n}\StringTok{Qui{-}quadrado:"}\NormalTok{, }\FunctionTok{round}\NormalTok{(teste\_qui\_loc\_ano}\SpecialCharTok{$}\NormalTok{statistic, }\DecValTok{4}\NormalTok{), }\StringTok{"}\SpecialCharTok{\textbackslash{}n}\StringTok{"}\NormalTok{)}
\FunctionTok{cat}\NormalTok{(}\StringTok{"Valor P:"}\NormalTok{, }\FunctionTok{round}\NormalTok{(teste\_qui\_loc\_ano}\SpecialCharTok{$}\NormalTok{p.value, }\DecValTok{6}\NormalTok{), }\StringTok{"}\SpecialCharTok{\textbackslash{}n}\StringTok{"}\NormalTok{)}
\FunctionTok{cat}\NormalTok{(}\StringTok{"Conclusão:"}\NormalTok{, }\FunctionTok{ifelse}\NormalTok{(teste\_qui\_loc\_ano}\SpecialCharTok{$}\NormalTok{p.value }\SpecialCharTok{\textless{}} \FloatTok{0.05}\NormalTok{, }
                        \StringTok{"Há associação significativa"}\NormalTok{, }
                        \StringTok{"Não há associação significativa"}\NormalTok{), }\StringTok{"}\SpecialCharTok{\textbackslash{}n\textbackslash{}n}\StringTok{"}\NormalTok{)}

\CommentTok{\# ==============================================================================}
\CommentTok{\# TABELA FINAL FORMATADA (REPLICANDO A IMAGEM)}
\CommentTok{\# ==============================================================================}

\CommentTok{\# Criando tabela no formato da imagem original}
\NormalTok{tabela\_final }\OtherTok{\textless{}{-}}\NormalTok{ dados\_ajcm\_completo }\SpecialCharTok{\%\textgreater{}\%}
  \FunctionTok{pivot\_wider}\NormalTok{(}\AttributeTok{names\_from =} \FunctionTok{c}\NormalTok{(Ano, Providência), }
              \AttributeTok{values\_from =}\NormalTok{ Quantidade,}
              \AttributeTok{names\_sep =} \StringTok{"\_"}\NormalTok{) }\SpecialCharTok{\%\textgreater{}\%}
  \FunctionTok{mutate}\NormalTok{(}
    \AttributeTok{Total\_2022 =} \StringTok{\textasciigrave{}}\AttributeTok{2022\_Denúncia}\StringTok{\textasciigrave{}} \SpecialCharTok{+} \StringTok{\textasciigrave{}}\AttributeTok{2022\_Arquivamento}\StringTok{\textasciigrave{}}\NormalTok{,}
    \AttributeTok{Total\_2023 =} \StringTok{\textasciigrave{}}\AttributeTok{2023\_Denúncia}\StringTok{\textasciigrave{}} \SpecialCharTok{+} \StringTok{\textasciigrave{}}\AttributeTok{2023\_Arquivamento}\StringTok{\textasciigrave{}}\NormalTok{,}
    \AttributeTok{Total\_Geral =}\NormalTok{ Total\_2022 }\SpecialCharTok{+}\NormalTok{ Total\_2023}
\NormalTok{  ) }\SpecialCharTok{\%\textgreater{}\%}
  \FunctionTok{select}\NormalTok{(Localização, }\StringTok{\textasciigrave{}}\AttributeTok{2022\_Denúncia}\StringTok{\textasciigrave{}}\NormalTok{, }\StringTok{\textasciigrave{}}\AttributeTok{2022\_Arquivamento}\StringTok{\textasciigrave{}}\NormalTok{, Total\_2022,}
         \StringTok{\textasciigrave{}}\AttributeTok{2023\_Denúncia}\StringTok{\textasciigrave{}}\NormalTok{, }\StringTok{\textasciigrave{}}\AttributeTok{2023\_Arquivamento}\StringTok{\textasciigrave{}}\NormalTok{, Total\_2023, Total\_Geral)}

\CommentTok{\# Adicionando linha de totais}
\NormalTok{linha\_totais }\OtherTok{\textless{}{-}}\NormalTok{ tabela\_final }\SpecialCharTok{\%\textgreater{}\%}
  \FunctionTok{summarise}\NormalTok{(}
\NormalTok{    Localização }\OtherTok{=} \StringTok{"TOTAL"}\NormalTok{,}
    \FunctionTok{across}\NormalTok{(}\FunctionTok{where}\NormalTok{(is.numeric), sum)}
\NormalTok{  )}

\NormalTok{tabela\_final\_com\_totais }\OtherTok{\textless{}{-}} \FunctionTok{bind\_rows}\NormalTok{(tabela\_final, linha\_totais)}

\FunctionTok{cat}\NormalTok{(}\StringTok{"TABELA FINAL REPLICADA (FORMATO DA IMAGEM):}\SpecialCharTok{\textbackslash{}n}\StringTok{"}\NormalTok{)}
\FunctionTok{cat}\NormalTok{(}\StringTok{"==========================================}\SpecialCharTok{\textbackslash{}n}\StringTok{"}\NormalTok{)}
\FunctionTok{print}\NormalTok{(tabela\_final\_com\_totais)}

\CommentTok{\# ==============================================================================}
\CommentTok{\# TABELA FORMATADA PARA APRESENTAÇÃO}
\CommentTok{\# ==============================================================================}

\CommentTok{\# Criando tabela formatada com nomes de colunas mais legíveis}
\NormalTok{tabela\_apresentacao }\OtherTok{\textless{}{-}}\NormalTok{ tabela\_final\_com\_totais }\SpecialCharTok{\%\textgreater{}\%}
  \FunctionTok{rename}\NormalTok{(}
    \StringTok{"Localização"} \OtherTok{=}\NormalTok{ Localização,}
    \StringTok{"2022 Denúncia"} \OtherTok{=} \StringTok{\textasciigrave{}}\AttributeTok{2022\_Denúncia}\StringTok{\textasciigrave{}}\NormalTok{,}
    \StringTok{"2022 Arquivamento"} \OtherTok{=} \StringTok{\textasciigrave{}}\AttributeTok{2022\_Arquivamento}\StringTok{\textasciigrave{}}\NormalTok{, }
    \StringTok{"Total 2022"} \OtherTok{=}\NormalTok{ Total\_2022,}
    \StringTok{"2023 Denúncia"} \OtherTok{=} \StringTok{\textasciigrave{}}\AttributeTok{2023\_Denúncia}\StringTok{\textasciigrave{}}\NormalTok{,}
    \StringTok{"2023 Arquivamento"} \OtherTok{=} \StringTok{\textasciigrave{}}\AttributeTok{2023\_Arquivamento}\StringTok{\textasciigrave{}}\NormalTok{,}
    \StringTok{"Total 2023"} \OtherTok{=}\NormalTok{ Total\_2023,}
    \StringTok{"Total Geral"} \OtherTok{=}\NormalTok{ Total\_Geral}
\NormalTok{  )}

\CommentTok{\# Aplicando formatação com separadores de milhares}
\NormalTok{tabela\_apresentacao\_formatada }\OtherTok{\textless{}{-}}\NormalTok{ tabela\_apresentacao }\SpecialCharTok{\%\textgreater{}\%}
  \FunctionTok{mutate}\NormalTok{(}\FunctionTok{across}\NormalTok{(}\FunctionTok{where}\NormalTok{(is.numeric), }\SpecialCharTok{\textasciitilde{}} \FunctionTok{format}\NormalTok{(.x, }\AttributeTok{big.mark =} \StringTok{"."}\NormalTok{, }\AttributeTok{decimal.mark =} \StringTok{","}\NormalTok{)))}

\FunctionTok{cat}\NormalTok{(}\StringTok{"}\SpecialCharTok{\textbackslash{}n}\StringTok{TABELA FORMATADA PARA APRESENTAÇÃO:}\SpecialCharTok{\textbackslash{}n}\StringTok{"}\NormalTok{)}
\FunctionTok{cat}\NormalTok{(}\StringTok{"==================================}\SpecialCharTok{\textbackslash{}n}\StringTok{"}\NormalTok{)}
\FunctionTok{print}\NormalTok{(tabela\_apresentacao\_formatada)}

\CommentTok{\# Salvando estrutura para preenchimento}
\FunctionTok{tryCatch}\NormalTok{(\{}
  \FunctionTok{write.csv}\NormalTok{(dados\_ajcm\_completo, }\StringTok{"estrutura\_dados\_oltramari.csv"}\NormalTok{, }\AttributeTok{row.names =} \ConstantTok{FALSE}\NormalTok{)}
  \FunctionTok{cat}\NormalTok{(}\StringTok{"}\SpecialCharTok{\textbackslash{}n}\StringTok{Arquivo \textquotesingle{}estrutura\_dados\_oltramari.csv\textquotesingle{} criado para referência.}\SpecialCharTok{\textbackslash{}n}\StringTok{"}\NormalTok{)}
\NormalTok{\}, }\AttributeTok{error =} \ControlFlowTok{function}\NormalTok{(e) \{}
  \FunctionTok{cat}\NormalTok{(}\StringTok{"}\SpecialCharTok{\textbackslash{}n}\StringTok{Não foi possível criar o arquivo CSV.}\SpecialCharTok{\textbackslash{}n}\StringTok{"}\NormalTok{)}
\NormalTok{\})}

\FunctionTok{cat}\NormalTok{(}\StringTok{"Script executado com sucesso!}\SpecialCharTok{\textbackslash{}n}\StringTok{"}\NormalTok{)}
\InformationTok{\textasciigrave{}\textasciigrave{}\textasciigrave{}}
\end{Highlighting}
\end{Shaded}

\begin{verbatim}
TABELA: AJCMs nas Promotorias por Localização, Providência e Ano
Período: 30/mar/2022 a 31/out/2023
==============================================================

[1] "DADOS COMPLETOS (AJUSTAR CONFORME IMAGEM):"
  Localização  Ano  Providência Quantidade
1     Capital 2022     Denúncia         20
2     Capital 2022 Arquivamento        191
3     Capital 2023     Denúncia         50
4     Capital 2023 Arquivamento        595
5    Interior 2022     Denúncia         42
6    Interior 2022 Arquivamento        207
7    Interior 2023     Denúncia         41
8    Interior 2023 Arquivamento        479

TABELA POR LOCALIZAÇÃO E PROVIDÊNCIA:
# A tibble: 2 x 6
  Localização Arquivamento Denúncia Total_Geral Perc_Denúncia Perc_Arquivamento
  <chr>              <dbl>    <dbl>       <dbl>         <dbl>             <dbl>
1 Capital              786       70         856           8.2              91.8
2 Interior             686       83         769          10.8              89.2

TABELA POR LOCALIZAÇÃO E ANO:
# A tibble: 2 x 6
  Localização `2022` `2023` Total_Geral Variação Perc_Variação
  <chr>        <dbl>  <dbl>       <dbl>    <dbl>         <dbl>
1 Capital        211    645         856      434          206.
2 Interior       249    520         769      271          109.

TABELA POR ANO E PROVIDÊNCIA:
# A tibble: 2 x 6
  Ano   Arquivamento Denúncia Total_Geral Perc_Denúncia Perc_Arquivamento
  <chr>        <dbl>    <dbl>       <dbl>         <dbl>             <dbl>
1 2022           398       62         460          13.5              86.5
2 2023          1074       91        1165           7.8              92.2

TABELA CRUZADA COMPLETA:
# A tibble: 2 x 10
  Localização `2022_Denúncia` `2022_Arquivamento` `2023_Denúncia`
  <chr>                 <dbl>               <dbl>           <dbl>
1 Capital                  20                 191              50
2 Interior                 42                 207              41
# i 6 more variables: `2023_Arquivamento` <dbl>, Total_2022 <dbl>,
#   Total_2023 <dbl>, Total_Denúncia <dbl>, Total_Arquivamento <dbl>,
#   Total_Geral <dbl>

ESTATÍSTICAS DESCRITIVAS:
=========================
Total Geral de AJCMs: 1.625 

Por Localização:
  Capital: 856 ( 52.7 %)
  Interior: 769 ( 47.3 %)

Por Ano:
  2022: 460 ( 28.3 %)
  2023: 1.165 ( 71.7 %)
  Variação 2022→2023: + 153.3 %

Por Providência:
  Denúncia: 153 ( 9.4 %)
  Arquivamento: 1.472 ( 90.6 %)

TESTE QUI-QUADRADO: LOCALIZAÇÃO x PROVIDÊNCIA
============================================
         Arquivamento Denúncia
Capital           786       70
Interior          686       83

Qui-quadrado: 2.9501 
Valor P: 0.085874 
Conclusão: Não há associação significativa 

TESTE QUI-QUADRADO: ANO x PROVIDÊNCIA
====================================
     Arquivamento Denúncia
2022          398       62
2023         1074       91

Qui-quadrado: 11.7627 
Valor P: 0.000604 
Conclusão: Há associação significativa 

TESTE QUI-QUADRADO: LOCALIZAÇÃO x ANO
====================================
         2022 2023
Capital   211  645
Interior  249  520

Qui-quadrado: 11.5496 
Valor P: 0.000678 
Conclusão: Há associação significativa 

TABELA FINAL REPLICADA (FORMATO DA IMAGEM):
==========================================
# A tibble: 3 x 8
  Localização `2022_Denúncia` `2022_Arquivamento` Total_2022 `2023_Denúncia`
  <chr>                 <dbl>               <dbl>      <dbl>           <dbl>
1 Capital                  20                 191        211              50
2 Interior                 42                 207        249              41
3 TOTAL                    62                 398        460              91
# i 3 more variables: `2023_Arquivamento` <dbl>, Total_2023 <dbl>,
#   Total_Geral <dbl>

TABELA FORMATADA PARA APRESENTAÇÃO:
==================================
# A tibble: 3 x 8
  Localização `2022 Denúncia` `2022 Arquivamento` `Total 2022` `2023 Denúncia`
  <chr>       <chr>           <chr>               <chr>        <chr>          
1 Capital     20              191                 211          50             
2 Interior    42              207                 249          41             
3 TOTAL       62              398                 460          91             
# i 3 more variables: `2023 Arquivamento` <chr>, `Total 2023` <chr>,
#   `Total Geral` <chr>

Arquivo 'estrutura_dados_oltramari.csv' criado para referência.
Script executado com sucesso!
\end{verbatim}

\pandocbounded{\includegraphics[keepaspectratio]{cap17-TSHo-Oltramari_files/figure-pdf/unnamed-chunk-3-1.pdf}}

\pandocbounded{\includegraphics[keepaspectratio]{cap17-TSHo-Oltramari_files/figure-pdf/unnamed-chunk-3-2.pdf}}

\pandocbounded{\includegraphics[keepaspectratio]{cap17-TSHo-Oltramari_files/figure-pdf/unnamed-chunk-3-3.pdf}}

\subsection{Gráficos de facetas para 2022 e
2023}\label{gruxe1ficos-de-facetas-para-2022-e-2023}

Acrescentar um gráfico de facetas, uma para cada ano (2022 e 2023), de
barras empilhadas com as proporções (\%) de Providência (Arquivamento ou
Denúncia), uma barra para cada Localização no eixo x (Capital ou
Interior)

\begin{Shaded}
\begin{Highlighting}[numbers=left,,]
\InformationTok{\textasciigrave{}\textasciigrave{}\textasciigrave{}\{r\}}
\CommentTok{\# GRÁFICO DE FACETAS: PROPORÇÕES DE PROVIDÊNCIA POR LOCALIZAÇÃO E ANO}
\CommentTok{\# Aproveitando dados já criados na global environment}

\CommentTok{\# Calculando proporções por Localização e Ano}
\NormalTok{dados\_prop\_facetas }\OtherTok{\textless{}{-}}\NormalTok{ dados\_ajcm\_completo }\SpecialCharTok{\%\textgreater{}\%}
  \FunctionTok{group\_by}\NormalTok{(Localização, Ano) }\SpecialCharTok{\%\textgreater{}\%}
  \FunctionTok{mutate}\NormalTok{(}
    \AttributeTok{Total\_Ano\_Local =} \FunctionTok{sum}\NormalTok{(Quantidade),}
\NormalTok{    Proporção }\OtherTok{=} \FunctionTok{round}\NormalTok{(Quantidade }\SpecialCharTok{/}\NormalTok{ Total\_Ano\_Local }\SpecialCharTok{*} \DecValTok{100}\NormalTok{, }\DecValTok{1}\NormalTok{)}
\NormalTok{  ) }\SpecialCharTok{\%\textgreater{}\%}
  \FunctionTok{ungroup}\NormalTok{()}

\CommentTok{\# Criando o gráfico de facetas com barras empilhadas}
\NormalTok{p\_facetas }\OtherTok{\textless{}{-}} \FunctionTok{ggplot}\NormalTok{(dados\_prop\_facetas, }\FunctionTok{aes}\NormalTok{(}\AttributeTok{x =}\NormalTok{ Localização, }\AttributeTok{y =}\NormalTok{ Proporção, }\AttributeTok{fill =}\NormalTok{ Providência)) }\SpecialCharTok{+}
  \FunctionTok{geom\_bar}\NormalTok{(}\AttributeTok{stat =} \StringTok{"identity"}\NormalTok{, }\AttributeTok{position =} \StringTok{"stack"}\NormalTok{, }\AttributeTok{alpha =} \FloatTok{0.8}\NormalTok{, }\AttributeTok{width =} \FloatTok{0.6}\NormalTok{) }\SpecialCharTok{+}
  
  \CommentTok{\# Adicionando rótulos nas barras}
  \FunctionTok{geom\_text}\NormalTok{(}\FunctionTok{aes}\NormalTok{(}\AttributeTok{label =} \FunctionTok{paste0}\NormalTok{(Proporção, }\StringTok{"\%"}\NormalTok{)), }
            \AttributeTok{position =} \FunctionTok{position\_stack}\NormalTok{(}\AttributeTok{vjust =} \FloatTok{0.5}\NormalTok{), }
            \AttributeTok{size =} \DecValTok{4}\NormalTok{, }\AttributeTok{fontface =} \StringTok{"bold"}\NormalTok{, }\AttributeTok{color =} \StringTok{"white"}\NormalTok{) }\SpecialCharTok{+}
  
  \CommentTok{\# Facetas por ano}
  \FunctionTok{facet\_wrap}\NormalTok{(}\SpecialCharTok{\textasciitilde{}}\NormalTok{ Ano, }\AttributeTok{scales =} \StringTok{"fixed"}\NormalTok{, }\AttributeTok{labeller =}\NormalTok{ label\_both) }\SpecialCharTok{+}
  
  \CommentTok{\# Configurações de cores}
  \FunctionTok{scale\_fill\_manual}\NormalTok{(}
    \AttributeTok{values =} \FunctionTok{c}\NormalTok{(}\StringTok{"Denúncia"} \OtherTok{=} \StringTok{"\#e74c3c"}\NormalTok{, }\StringTok{"Arquivamento"} \OtherTok{=} \StringTok{"\#3498db"}\NormalTok{),}
    \AttributeTok{name =} \StringTok{"Providência"}
\NormalTok{  ) }\SpecialCharTok{+}
  
  \CommentTok{\# Configurações dos eixos}
  \FunctionTok{scale\_y\_continuous}\NormalTok{(}
    \AttributeTok{breaks =} \FunctionTok{seq}\NormalTok{(}\DecValTok{0}\NormalTok{, }\DecValTok{100}\NormalTok{, }\DecValTok{25}\NormalTok{), }
    \AttributeTok{labels =} \FunctionTok{paste0}\NormalTok{(}\FunctionTok{seq}\NormalTok{(}\DecValTok{0}\NormalTok{, }\DecValTok{100}\NormalTok{, }\DecValTok{25}\NormalTok{), }\StringTok{"\%"}\NormalTok{),}
    \AttributeTok{limits =} \FunctionTok{c}\NormalTok{(}\DecValTok{0}\NormalTok{, }\DecValTok{100}\NormalTok{)}
\NormalTok{  ) }\SpecialCharTok{+}
  
  \CommentTok{\# Rótulos e títulos}
  \FunctionTok{labs}\NormalTok{(}
    \AttributeTok{title =} \StringTok{"Proporção de Providências por Localização e Ano"}\NormalTok{,}
    \AttributeTok{subtitle =} \StringTok{"Distribuição percentual: Denúncia vs Arquivamento"}\NormalTok{,}
    \AttributeTok{x =} \StringTok{"Localização das Promotorias"}\NormalTok{,}
    \AttributeTok{y =} \StringTok{"Proporção (\%)"}\NormalTok{,}
    \AttributeTok{fill =} \StringTok{"Tipo de Providência"}\NormalTok{,}
    \AttributeTok{caption =} \StringTok{"Fonte: Dados Oltramari | Período: 30/mar/2022 a 31/out/2023"}
\NormalTok{  ) }\SpecialCharTok{+}
  
  \CommentTok{\# Tema e customizações}
  \FunctionTok{theme\_minimal}\NormalTok{() }\SpecialCharTok{+}
  \FunctionTok{theme}\NormalTok{(}
    \AttributeTok{plot.title =} \FunctionTok{element\_text}\NormalTok{(}\AttributeTok{hjust =} \FloatTok{0.5}\NormalTok{, }\AttributeTok{size =} \DecValTok{16}\NormalTok{, }\AttributeTok{face =} \StringTok{"bold"}\NormalTok{),}
    \AttributeTok{plot.subtitle =} \FunctionTok{element\_text}\NormalTok{(}\AttributeTok{hjust =} \FloatTok{0.5}\NormalTok{, }\AttributeTok{size =} \DecValTok{13}\NormalTok{, }\AttributeTok{color =} \StringTok{"gray30"}\NormalTok{),}
    \AttributeTok{plot.caption =} \FunctionTok{element\_text}\NormalTok{(}\AttributeTok{hjust =} \FloatTok{0.5}\NormalTok{, }\AttributeTok{size =} \DecValTok{10}\NormalTok{, }\AttributeTok{color =} \StringTok{"gray50"}\NormalTok{),}
    
    \CommentTok{\# Configurações das facetas}
    \AttributeTok{strip.text =} \FunctionTok{element\_text}\NormalTok{(}\AttributeTok{size =} \DecValTok{14}\NormalTok{, }\AttributeTok{face =} \StringTok{"bold"}\NormalTok{, }
                             \AttributeTok{color =} \StringTok{"white"}\NormalTok{, }\AttributeTok{margin =} \FunctionTok{margin}\NormalTok{(}\DecValTok{5}\NormalTok{,}\DecValTok{5}\NormalTok{,}\DecValTok{5}\NormalTok{,}\DecValTok{5}\NormalTok{)),}
    \AttributeTok{strip.background =} \FunctionTok{element\_rect}\NormalTok{(}\AttributeTok{fill =} \StringTok{"gray20"}\NormalTok{, }\AttributeTok{color =} \ConstantTok{NA}\NormalTok{),}
    
    \CommentTok{\# Configurações da legenda}
    \AttributeTok{legend.position =} \StringTok{"bottom"}\NormalTok{,}
    \AttributeTok{legend.title =} \FunctionTok{element\_text}\NormalTok{(}\AttributeTok{size =} \DecValTok{12}\NormalTok{, }\AttributeTok{face =} \StringTok{"bold"}\NormalTok{),}
    \AttributeTok{legend.text =} \FunctionTok{element\_text}\NormalTok{(}\AttributeTok{size =} \DecValTok{11}\NormalTok{),}
    \AttributeTok{legend.key.size =} \FunctionTok{unit}\NormalTok{(}\DecValTok{1}\NormalTok{, }\StringTok{"cm"}\NormalTok{),}
    
    \CommentTok{\# Configurações dos eixos}
    \AttributeTok{axis.title.x =} \FunctionTok{element\_text}\NormalTok{(}\AttributeTok{size =} \DecValTok{12}\NormalTok{, }\AttributeTok{face =} \StringTok{"bold"}\NormalTok{, }\AttributeTok{margin =} \FunctionTok{margin}\NormalTok{(}\AttributeTok{t =} \DecValTok{15}\NormalTok{)),}
    \AttributeTok{axis.title.y =} \FunctionTok{element\_text}\NormalTok{(}\AttributeTok{size =} \DecValTok{12}\NormalTok{, }\AttributeTok{face =} \StringTok{"bold"}\NormalTok{, }\AttributeTok{margin =} \FunctionTok{margin}\NormalTok{(}\AttributeTok{r =} \DecValTok{15}\NormalTok{)),}
    \AttributeTok{axis.text =} \FunctionTok{element\_text}\NormalTok{(}\AttributeTok{size =} \DecValTok{11}\NormalTok{),}
    
    \CommentTok{\# Configurações do painel}
    \AttributeTok{panel.grid.major.x =} \FunctionTok{element\_blank}\NormalTok{(),}
    \AttributeTok{panel.grid.minor =} \FunctionTok{element\_blank}\NormalTok{(),}
    \AttributeTok{panel.border =} \FunctionTok{element\_rect}\NormalTok{(}\AttributeTok{color =} \StringTok{"gray30"}\NormalTok{, }\AttributeTok{fill =} \ConstantTok{NA}\NormalTok{, }\AttributeTok{size =} \FloatTok{0.5}\NormalTok{)}
\NormalTok{  )}

\CommentTok{\# Exibindo o gráfico}
\FunctionTok{print}\NormalTok{(p\_facetas)}

\CommentTok{\# Salvando o gráfico (opcional)}
\CommentTok{\# ggsave("grafico\_facetas\_proporcoes\_oltramari.png", p\_facetas, }
\CommentTok{\#        width = 12, height = 8, dpi = 300, bg = "white")}
\InformationTok{\textasciigrave{}\textasciigrave{}\textasciigrave{}}
\end{Highlighting}
\end{Shaded}

\pandocbounded{\includegraphics[keepaspectratio]{cap17-TSHo-Oltramari_files/figure-pdf/unnamed-chunk-4-1.pdf}}

Percebe-se que, em 2022, há uma significativa diferença, deixando claro
descompasso entre a atuação das Promotorias de Justiça da capital em
relação aos Promotores de Justiça do interior, com maior proporção de
oferecimento de denúncias por parte destes enquanto há, em proporções
para o anos de 2023, um semelhante comportamento entre os dois grupos,
quanto à arquivamentos e denúncias.

\subsection{\texorpdfstring{Teste X\textsuperscript{2} para cada
faceta.}{Teste X2 para cada faceta.}}\label{teste-x2-para-cada-faceta.}

Agora Script R para um chunk subsequente apenas para Acrescentar ao
mesmo gráfico de facetas, uma para cada ano (2022 e 2023), de barras
empilhadas com as proporções (\%) de Providência (Arquivamento ou
Denúncia), uma barra para cada Localização no eixo x (Capital ou
Interior), o resultado de testes quiquadrado aplicado a cada um dos
anos. Não é necessário gerar novamente os dados e sim aproveitar as
variáveis já criadas na global environment.

\begin{Shaded}
\begin{Highlighting}[numbers=left,,]
\InformationTok{\textasciigrave{}\textasciigrave{}\textasciigrave{}\{r\}}
\CommentTok{\# GRÁFICO DE FACETAS COM TESTES QUI{-}QUADRADO POR ANO}
\CommentTok{\# Aproveitando dados já criados na global environment}

\CommentTok{\# Realizando testes qui{-}quadrado para cada ano separadamente}
\NormalTok{resultados\_testes\_ano }\OtherTok{\textless{}{-}} \FunctionTok{list}\NormalTok{()}

\CommentTok{\# Teste qui{-}quadrado para 2022}
\NormalTok{dados\_2022 }\OtherTok{\textless{}{-}}\NormalTok{ dados\_ajcm\_completo }\SpecialCharTok{\%\textgreater{}\%} 
  \FunctionTok{filter}\NormalTok{(Ano }\SpecialCharTok{==} \StringTok{"2022"}\NormalTok{) }\SpecialCharTok{\%\textgreater{}\%}
  \FunctionTok{group\_by}\NormalTok{(Localização, Providência) }\SpecialCharTok{\%\textgreater{}\%}
  \FunctionTok{summarise}\NormalTok{(}\AttributeTok{Total =} \FunctionTok{sum}\NormalTok{(Quantidade), }\AttributeTok{.groups =} \StringTok{"drop"}\NormalTok{) }\SpecialCharTok{\%\textgreater{}\%}
  \FunctionTok{pivot\_wider}\NormalTok{(}\AttributeTok{names\_from =}\NormalTok{ Providência, }\AttributeTok{values\_from =}\NormalTok{ Total)}

\NormalTok{matriz\_2022 }\OtherTok{\textless{}{-}} \FunctionTok{as.matrix}\NormalTok{(dados\_2022[, }\SpecialCharTok{{-}}\DecValTok{1}\NormalTok{])}
\FunctionTok{rownames}\NormalTok{(matriz\_2022) }\OtherTok{\textless{}{-}}\NormalTok{ dados\_2022}\SpecialCharTok{$}\NormalTok{Localização}
\NormalTok{teste\_2022 }\OtherTok{\textless{}{-}} \FunctionTok{chisq.test}\NormalTok{(matriz\_2022)}

\NormalTok{resultados\_testes\_ano[[}\StringTok{"2022"}\NormalTok{]] }\OtherTok{\textless{}{-}} \FunctionTok{list}\NormalTok{(}
  \AttributeTok{chi\_squared =}\NormalTok{ teste\_2022}\SpecialCharTok{$}\NormalTok{statistic,}
  \AttributeTok{p\_value =}\NormalTok{ teste\_2022}\SpecialCharTok{$}\NormalTok{p.value,}
  \AttributeTok{significativo =}\NormalTok{ teste\_2022}\SpecialCharTok{$}\NormalTok{p.value }\SpecialCharTok{\textless{}} \FloatTok{0.05}
\NormalTok{)}

\CommentTok{\# Teste qui{-}quadrado para 2023}
\NormalTok{dados\_2023 }\OtherTok{\textless{}{-}}\NormalTok{ dados\_ajcm\_completo }\SpecialCharTok{\%\textgreater{}\%} 
  \FunctionTok{filter}\NormalTok{(Ano }\SpecialCharTok{==} \StringTok{"2023"}\NormalTok{) }\SpecialCharTok{\%\textgreater{}\%}
  \FunctionTok{group\_by}\NormalTok{(Localização, Providência) }\SpecialCharTok{\%\textgreater{}\%}
  \FunctionTok{summarise}\NormalTok{(}\AttributeTok{Total =} \FunctionTok{sum}\NormalTok{(Quantidade), }\AttributeTok{.groups =} \StringTok{"drop"}\NormalTok{) }\SpecialCharTok{\%\textgreater{}\%}
  \FunctionTok{pivot\_wider}\NormalTok{(}\AttributeTok{names\_from =}\NormalTok{ Providência, }\AttributeTok{values\_from =}\NormalTok{ Total)}

\NormalTok{matriz\_2023 }\OtherTok{\textless{}{-}} \FunctionTok{as.matrix}\NormalTok{(dados\_2023[, }\SpecialCharTok{{-}}\DecValTok{1}\NormalTok{])}
\FunctionTok{rownames}\NormalTok{(matriz\_2023) }\OtherTok{\textless{}{-}}\NormalTok{ dados\_2023}\SpecialCharTok{$}\NormalTok{Localização}
\NormalTok{teste\_2023 }\OtherTok{\textless{}{-}} \FunctionTok{chisq.test}\NormalTok{(matriz\_2023)}

\NormalTok{resultados\_testes\_ano[[}\StringTok{"2023"}\NormalTok{]] }\OtherTok{\textless{}{-}} \FunctionTok{list}\NormalTok{(}
  \AttributeTok{chi\_squared =}\NormalTok{ teste\_2023}\SpecialCharTok{$}\NormalTok{statistic,}
  \AttributeTok{p\_value =}\NormalTok{ teste\_2023}\SpecialCharTok{$}\NormalTok{p.value,}
  \AttributeTok{significativo =}\NormalTok{ teste\_2023}\SpecialCharTok{$}\NormalTok{p.value }\SpecialCharTok{\textless{}} \FloatTok{0.05}
\NormalTok{)}

\CommentTok{\# Criando labels com os resultados dos testes para cada faceta}
\NormalTok{criar\_label\_teste }\OtherTok{\textless{}{-}} \ControlFlowTok{function}\NormalTok{(ano) \{}
\NormalTok{  resultado }\OtherTok{\textless{}{-}}\NormalTok{ resultados\_testes\_ano[[ano]]}
\NormalTok{  chi2 }\OtherTok{=} \FunctionTok{round}\NormalTok{(resultado}\SpecialCharTok{$}\NormalTok{chi\_squared, }\DecValTok{3}\NormalTok{)}
\NormalTok{  p\_val }\OtherTok{=} \FunctionTok{ifelse}\NormalTok{(resultado}\SpecialCharTok{$}\NormalTok{p\_value }\SpecialCharTok{\textless{}} \FloatTok{0.001}\NormalTok{, }\StringTok{"\textless{} 0.001"}\NormalTok{, }\FunctionTok{round}\NormalTok{(resultado}\SpecialCharTok{$}\NormalTok{p\_value, }\DecValTok{3}\NormalTok{))}
\NormalTok{  significancia }\OtherTok{=} \FunctionTok{ifelse}\NormalTok{(resultado}\SpecialCharTok{$}\NormalTok{significativo, }\StringTok{"Significativo"}\NormalTok{, }\StringTok{"Não Significativo"}\NormalTok{)}
  
  \FunctionTok{paste0}\NormalTok{(}
    \StringTok{"χ² = "}\NormalTok{, chi2, }\StringTok{"}\SpecialCharTok{\textbackslash{}n}\StringTok{"}\NormalTok{,}
    \StringTok{"p = "}\NormalTok{, p\_val, }\StringTok{"}\SpecialCharTok{\textbackslash{}n}\StringTok{"}\NormalTok{,}
\NormalTok{    significancia}
\NormalTok{  )}
\NormalTok{\}}

\CommentTok{\# Criando data frame com labels dos testes}
\NormalTok{labels\_testes }\OtherTok{\textless{}{-}} \FunctionTok{data.frame}\NormalTok{(}
  \AttributeTok{Ano =} \FunctionTok{c}\NormalTok{(}\StringTok{"2022"}\NormalTok{, }\StringTok{"2023"}\NormalTok{),}
  \AttributeTok{label =} \FunctionTok{sapply}\NormalTok{(}\FunctionTok{c}\NormalTok{(}\StringTok{"2022"}\NormalTok{, }\StringTok{"2023"}\NormalTok{), criar\_label\_teste),}
  \AttributeTok{x =} \FloatTok{1.5}\NormalTok{,  }\CommentTok{\# Posição central entre Capital e Interior}
  \AttributeTok{y =} \DecValTok{85}    \CommentTok{\# Posição no topo do gráfico}
\NormalTok{)}

\CommentTok{\# Recriando o gráfico com os testes qui{-}quadrado}
\NormalTok{p\_facetas\_com\_testes }\OtherTok{\textless{}{-}} \FunctionTok{ggplot}\NormalTok{(dados\_prop\_facetas, }\FunctionTok{aes}\NormalTok{(}\AttributeTok{x =}\NormalTok{ Localização, }\AttributeTok{y =}\NormalTok{ Proporção, }\AttributeTok{fill =}\NormalTok{ Providência)) }\SpecialCharTok{+}
  \FunctionTok{geom\_bar}\NormalTok{(}\AttributeTok{stat =} \StringTok{"identity"}\NormalTok{, }\AttributeTok{position =} \StringTok{"stack"}\NormalTok{, }\AttributeTok{alpha =} \FloatTok{0.8}\NormalTok{, }\AttributeTok{width =} \FloatTok{0.6}\NormalTok{) }\SpecialCharTok{+}
  
  \CommentTok{\# Adicionando rótulos nas barras}
  \FunctionTok{geom\_text}\NormalTok{(}\FunctionTok{aes}\NormalTok{(}\AttributeTok{label =} \FunctionTok{paste0}\NormalTok{(Proporção, }\StringTok{"\%"}\NormalTok{)), }
            \AttributeTok{position =} \FunctionTok{position\_stack}\NormalTok{(}\AttributeTok{vjust =} \FloatTok{0.5}\NormalTok{), }
            \AttributeTok{size =} \DecValTok{4}\NormalTok{, }\AttributeTok{fontface =} \StringTok{"bold"}\NormalTok{, }\AttributeTok{color =} \StringTok{"white"}\NormalTok{) }\SpecialCharTok{+}
  
  \CommentTok{\# Adicionando resultados dos testes qui{-}quadrado}
  \FunctionTok{geom\_text}\NormalTok{(}\AttributeTok{data =}\NormalTok{ labels\_testes, }
            \FunctionTok{aes}\NormalTok{(}\AttributeTok{x =}\NormalTok{ x, }\AttributeTok{y =}\NormalTok{ y, }\AttributeTok{label =}\NormalTok{ label), }
            \AttributeTok{inherit.aes =} \ConstantTok{FALSE}\NormalTok{,}
            \AttributeTok{size =} \FloatTok{3.5}\NormalTok{, }
            \AttributeTok{fontface =} \StringTok{"bold"}\NormalTok{,}
            \AttributeTok{color =} \StringTok{"black"}\NormalTok{,}
            \AttributeTok{fill =} \StringTok{"white"}\NormalTok{,}
            \AttributeTok{alpha =} \FloatTok{0.9}\NormalTok{,}
            \AttributeTok{hjust =} \FloatTok{0.5}\NormalTok{,}
            \AttributeTok{vjust =} \DecValTok{1}\NormalTok{,}
            \CommentTok{\# Criando caixa de texto}
            \AttributeTok{geom =} \StringTok{"label"}\NormalTok{,}
            \AttributeTok{label.padding =} \FunctionTok{unit}\NormalTok{(}\FloatTok{0.3}\NormalTok{, }\StringTok{"lines"}\NormalTok{),}
            \AttributeTok{label.size =} \FloatTok{0.5}\NormalTok{) }\SpecialCharTok{+}
  
  \CommentTok{\# Facetas por ano}
  \FunctionTok{facet\_wrap}\NormalTok{(}\SpecialCharTok{\textasciitilde{}}\NormalTok{ Ano, }\AttributeTok{scales =} \StringTok{"fixed"}\NormalTok{, }\AttributeTok{labeller =}\NormalTok{ label\_both) }\SpecialCharTok{+}
  
  \CommentTok{\# Configurações de cores}
  \FunctionTok{scale\_fill\_manual}\NormalTok{(}
    \AttributeTok{values =} \FunctionTok{c}\NormalTok{(}\StringTok{"Denúncia"} \OtherTok{=} \StringTok{"\#e74c3c"}\NormalTok{, }\StringTok{"Arquivamento"} \OtherTok{=} \StringTok{"\#3498db"}\NormalTok{),}
    \AttributeTok{name =} \StringTok{"Providência"}
\NormalTok{  ) }\SpecialCharTok{+}
  
  \CommentTok{\# Configurações dos eixos}
  \FunctionTok{scale\_y\_continuous}\NormalTok{(}
    \AttributeTok{breaks =} \FunctionTok{seq}\NormalTok{(}\DecValTok{0}\NormalTok{, }\DecValTok{100}\NormalTok{, }\DecValTok{25}\NormalTok{), }
    \AttributeTok{labels =} \FunctionTok{paste0}\NormalTok{(}\FunctionTok{seq}\NormalTok{(}\DecValTok{0}\NormalTok{, }\DecValTok{100}\NormalTok{, }\DecValTok{25}\NormalTok{), }\StringTok{"\%"}\NormalTok{),}
    \AttributeTok{limits =} \FunctionTok{c}\NormalTok{(}\DecValTok{0}\NormalTok{, }\DecValTok{100}\NormalTok{)}
\NormalTok{  ) }\SpecialCharTok{+}
  
  \CommentTok{\# Rótulos e títulos}
  \FunctionTok{labs}\NormalTok{(}
    \AttributeTok{title =} \StringTok{"Proporção de Providências por Localização e Ano"}\NormalTok{,}
    \AttributeTok{subtitle =} \StringTok{"Distribuição percentual com Testes Qui{-}Quadrado de Independência"}\NormalTok{,}
    \AttributeTok{x =} \StringTok{"Localização das Promotorias"}\NormalTok{,}
    \AttributeTok{y =} \StringTok{"Proporção (\%)"}\NormalTok{,}
    \AttributeTok{fill =} \StringTok{"Tipo de Providência"}\NormalTok{,}
    \AttributeTok{caption =} \StringTok{"Fonte: Dados Oltramari | Período: 30/mar/2022 a 31/out/2023}\SpecialCharTok{\textbackslash{}n}\StringTok{Teste: H₀ = Independência entre Localização e Providência"}
\NormalTok{  ) }\SpecialCharTok{+}
  
  \CommentTok{\# Tema e customizações}
  \FunctionTok{theme\_minimal}\NormalTok{() }\SpecialCharTok{+}
  \FunctionTok{theme}\NormalTok{(}
    \AttributeTok{plot.title =} \FunctionTok{element\_text}\NormalTok{(}\AttributeTok{hjust =} \FloatTok{0.5}\NormalTok{, }\AttributeTok{size =} \DecValTok{16}\NormalTok{, }\AttributeTok{face =} \StringTok{"bold"}\NormalTok{),}
    \AttributeTok{plot.subtitle =} \FunctionTok{element\_text}\NormalTok{(}\AttributeTok{hjust =} \FloatTok{0.5}\NormalTok{, }\AttributeTok{size =} \DecValTok{13}\NormalTok{, }\AttributeTok{color =} \StringTok{"gray30"}\NormalTok{),}
    \AttributeTok{plot.caption =} \FunctionTok{element\_text}\NormalTok{(}\AttributeTok{hjust =} \FloatTok{0.5}\NormalTok{, }\AttributeTok{size =} \DecValTok{10}\NormalTok{, }\AttributeTok{color =} \StringTok{"gray50"}\NormalTok{),}
    
    \CommentTok{\# Configurações das facetas}
    \AttributeTok{strip.text =} \FunctionTok{element\_text}\NormalTok{(}\AttributeTok{size =} \DecValTok{14}\NormalTok{, }\AttributeTok{face =} \StringTok{"bold"}\NormalTok{, }
                             \AttributeTok{color =} \StringTok{"white"}\NormalTok{, }\AttributeTok{margin =} \FunctionTok{margin}\NormalTok{(}\DecValTok{5}\NormalTok{,}\DecValTok{5}\NormalTok{,}\DecValTok{5}\NormalTok{,}\DecValTok{5}\NormalTok{)),}
    \AttributeTok{strip.background =} \FunctionTok{element\_rect}\NormalTok{(}\AttributeTok{fill =} \StringTok{"gray20"}\NormalTok{, }\AttributeTok{color =} \ConstantTok{NA}\NormalTok{),}
    
    \CommentTok{\# Configurações da legenda}
    \AttributeTok{legend.position =} \StringTok{"bottom"}\NormalTok{,}
    \AttributeTok{legend.title =} \FunctionTok{element\_text}\NormalTok{(}\AttributeTok{size =} \DecValTok{12}\NormalTok{, }\AttributeTok{face =} \StringTok{"bold"}\NormalTok{),}
    \AttributeTok{legend.text =} \FunctionTok{element\_text}\NormalTok{(}\AttributeTok{size =} \DecValTok{11}\NormalTok{),}
    \AttributeTok{legend.key.size =} \FunctionTok{unit}\NormalTok{(}\DecValTok{1}\NormalTok{, }\StringTok{"cm"}\NormalTok{),}
    
    \CommentTok{\# Configurações dos eixos}
    \AttributeTok{axis.title.x =} \FunctionTok{element\_text}\NormalTok{(}\AttributeTok{size =} \DecValTok{12}\NormalTok{, }\AttributeTok{face =} \StringTok{"bold"}\NormalTok{, }\AttributeTok{margin =} \FunctionTok{margin}\NormalTok{(}\AttributeTok{t =} \DecValTok{15}\NormalTok{)),}
    \AttributeTok{axis.title.y =} \FunctionTok{element\_text}\NormalTok{(}\AttributeTok{size =} \DecValTok{12}\NormalTok{, }\AttributeTok{face =} \StringTok{"bold"}\NormalTok{, }\AttributeTok{margin =} \FunctionTok{margin}\NormalTok{(}\AttributeTok{r =} \DecValTok{15}\NormalTok{)),}
    \AttributeTok{axis.text =} \FunctionTok{element\_text}\NormalTok{(}\AttributeTok{size =} \DecValTok{11}\NormalTok{),}
    
    \CommentTok{\# Configurações do painel}
    \AttributeTok{panel.grid.major.x =} \FunctionTok{element\_blank}\NormalTok{(),}
    \AttributeTok{panel.grid.minor =} \FunctionTok{element\_blank}\NormalTok{(),}
    \AttributeTok{panel.border =} \FunctionTok{element\_rect}\NormalTok{(}\AttributeTok{color =} \StringTok{"gray30"}\NormalTok{, }\AttributeTok{fill =} \ConstantTok{NA}\NormalTok{, }\AttributeTok{size =} \FloatTok{0.5}\NormalTok{)}
\NormalTok{  )}

\CommentTok{\# Exibindo o gráfico}
\FunctionTok{print}\NormalTok{(p\_facetas\_com\_testes)}

\CommentTok{\# Exibindo resumo dos testes no console}
\FunctionTok{cat}\NormalTok{(}\StringTok{"RESUMO DOS TESTES QUI{-}QUADRADO POR ANO:}\SpecialCharTok{\textbackslash{}n}\StringTok{"}\NormalTok{)}
\FunctionTok{cat}\NormalTok{(}\StringTok{"======================================}\SpecialCharTok{\textbackslash{}n\textbackslash{}n}\StringTok{"}\NormalTok{)}

\ControlFlowTok{for}\NormalTok{(ano }\ControlFlowTok{in} \FunctionTok{c}\NormalTok{(}\StringTok{"2022"}\NormalTok{, }\StringTok{"2023"}\NormalTok{)) \{}
\NormalTok{  resultado }\OtherTok{\textless{}{-}}\NormalTok{ resultados\_testes\_ano[[ano]]}
  \FunctionTok{cat}\NormalTok{(}\StringTok{"ANO"}\NormalTok{, ano, }\StringTok{":}\SpecialCharTok{\textbackslash{}n}\StringTok{"}\NormalTok{)}
  \FunctionTok{cat}\NormalTok{(}\StringTok{"  χ² ="}\NormalTok{, }\FunctionTok{round}\NormalTok{(resultado}\SpecialCharTok{$}\NormalTok{chi\_squared, }\DecValTok{4}\NormalTok{), }\StringTok{"}\SpecialCharTok{\textbackslash{}n}\StringTok{"}\NormalTok{)}
  \FunctionTok{cat}\NormalTok{(}\StringTok{"  p{-}valor ="}\NormalTok{, }\FunctionTok{ifelse}\NormalTok{(resultado}\SpecialCharTok{$}\NormalTok{p\_value }\SpecialCharTok{\textless{}} \FloatTok{0.001}\NormalTok{, }\StringTok{"\textless{} 0.001"}\NormalTok{, }\FunctionTok{round}\NormalTok{(resultado}\SpecialCharTok{$}\NormalTok{p\_value, }\DecValTok{6}\NormalTok{)), }\StringTok{"}\SpecialCharTok{\textbackslash{}n}\StringTok{"}\NormalTok{)}
  \FunctionTok{cat}\NormalTok{(}\StringTok{"  Resultado:"}\NormalTok{, }\FunctionTok{ifelse}\NormalTok{(resultado}\SpecialCharTok{$}\NormalTok{significativo, }\StringTok{"SIGNIFICATIVO (p \textless{} 0.05)"}\NormalTok{, }\StringTok{"NÃO SIGNIFICATIVO (p ≥ 0.05)"}\NormalTok{), }\StringTok{"}\SpecialCharTok{\textbackslash{}n}\StringTok{"}\NormalTok{)}
  \FunctionTok{cat}\NormalTok{(}\StringTok{"  Interpretação:"}\NormalTok{, }\FunctionTok{ifelse}\NormalTok{(resultado}\SpecialCharTok{$}\NormalTok{significativo, }
                                \StringTok{"Há associação entre Localização e Providência"}\NormalTok{, }
                                \StringTok{"Não há associação entre Localização e Providência"}\NormalTok{), }\StringTok{"}\SpecialCharTok{\textbackslash{}n\textbackslash{}n}\StringTok{"}\NormalTok{)}
\NormalTok{\}}

\CommentTok{\# Salvando o gráfico atualizado (opcional)}
\CommentTok{\# ggsave("grafico\_facetas\_proporcoes\_com\_testes\_oltramari.png", p\_facetas\_com\_testes, }
\CommentTok{\#        width = 14, height = 9, dpi = 300, bg = "white")}
\InformationTok{\textasciigrave{}\textasciigrave{}\textasciigrave{}}
\end{Highlighting}
\end{Shaded}

\begin{verbatim}
RESUMO DOS TESTES QUI-QUADRADO POR ANO:
======================================

ANO 2022 :
  χ² = 4.7322 
  p-valor = 0.029603 
  Resultado: SIGNIFICATIVO (p < 0.05) 
  Interpretação: Há associação entre Localização e Providência 

ANO 2023 :
  χ² = 0 
  p-valor = 1 
  Resultado: NÃO SIGNIFICATIVO (p ≥ 0.05) 
  Interpretação: Não há associação entre Localização e Providência 
\end{verbatim}

\pandocbounded{\includegraphics[keepaspectratio]{cap17-TSHo-Oltramari_files/figure-pdf/unnamed-chunk-5-1.pdf}}

O teste de significância para a Hipótese de Nula de nenhuma associação
entre Providência (Arquivamento ou Denúncia) e Localização (Capital ou
interior) para o ano de \ul{\textbf{2022}}, após a entrada em vigor do
tratamento, foi \ul{\textbf{significativo}} para um Nível de Confiança
de 95\% (erro tipo I de 5\%).

Para 2022 pode-se \ul{\textbf{rejeitar}} a Hipótese Nula e apoiar a
Hipótese Alternativa de que há uma associação significativa entre
Providência e Localização, sendo que o gráfico de barras empilhadas
ilustra que o interior denunciou proporcionalmente mais e arquivou
proporcionalmente signifcativamente menos que a capital (promororia
especializada junto à VAM).

Todavia o mesmo teste para o ano de \ul{\textbf{2023 não foi
significativo}}. E o mesmo gráfico de barras empilhadas denota uma
atuação similar entre interior e Capital.

\subsection{Variável Oculta}\label{variuxe1vel-oculta}

Nssa análise acima, permaneceu como variável oculta o crescimento no
volume de AJCM.

Uma possível explicação é observar o efeito que o crescimento no volume
de AJCM de um ano (2022) para o outro (2023), que foi de:

Por Ano: 2022: 460 ( 28.3 \%) 2023: 1.165 ( 71.7 \%) Variação 2022→2023:
+ 153.3 \%

\begin{quote}
``Após 2019, os números de AJCM permanecem relativamente estáveis até o
ano de 2022 quando, em 2023, há um aumento abrupto de 145\% em relação
ao ano anterior.'' (OLTRAMARI, 2024 , p.~158)
\end{quote}

Esse cescimento poderia explicar a mudança de comportamento dos
promotores do interior, que passaram a adotar uma postura que refeltiu
em uma proporção anual de arquivamentos em 2023 (92,1\%) muito próxima à
que foi observada na Capital no mesmo ano de 2023 (92,2\%).

Refazer um gráfico de barras lado a lado para duas facetas dos anos 2022
e 2023 com mudança na escala do eixo y para evidenciar esse ocultamento.

\begin{Shaded}
\begin{Highlighting}[numbers=left,,]
\InformationTok{\textasciigrave{}\textasciigrave{}\textasciigrave{}\{r\}}
\CommentTok{\# =====================================================}
\CommentTok{\#                 GRÁFICOS ATUALIZADOS}
\CommentTok{\# =====================================================}

\CommentTok{\# 1. Gráfico de barras agrupadas por todas as categorias com proporções}
\NormalTok{p1 }\OtherTok{\textless{}{-}}\NormalTok{ dados\_ajcm\_completo }\SpecialCharTok{\%\textgreater{}\%}
  \CommentTok{\# Calculando proporções por ano e localização}
  \FunctionTok{group\_by}\NormalTok{(Ano, Localização) }\SpecialCharTok{\%\textgreater{}\%}
  \FunctionTok{mutate}\NormalTok{(}
    \AttributeTok{Total\_Ano\_Local =} \FunctionTok{sum}\NormalTok{(Quantidade),}
\NormalTok{    Proporção }\OtherTok{=} \FunctionTok{round}\NormalTok{(Quantidade }\SpecialCharTok{/}\NormalTok{ Total\_Ano\_Local }\SpecialCharTok{*} \DecValTok{100}\NormalTok{, }\DecValTok{1}\NormalTok{)}
\NormalTok{  ) }\SpecialCharTok{\%\textgreater{}\%}
  \FunctionTok{ungroup}\NormalTok{() }\SpecialCharTok{\%\textgreater{}\%}
  
  \FunctionTok{ggplot}\NormalTok{(}\FunctionTok{aes}\NormalTok{(}\AttributeTok{x =}\NormalTok{ Localização, }\AttributeTok{y =}\NormalTok{ Quantidade, }\AttributeTok{fill =}\NormalTok{ Providência)) }\SpecialCharTok{+}
  \FunctionTok{geom\_bar}\NormalTok{(}\AttributeTok{stat =} \StringTok{"identity"}\NormalTok{, }\AttributeTok{position =} \StringTok{"dodge"}\NormalTok{, }\AttributeTok{alpha =} \FloatTok{0.8}\NormalTok{) }\SpecialCharTok{+}
  \FunctionTok{facet\_wrap}\NormalTok{(}\SpecialCharTok{\textasciitilde{}}\NormalTok{ Ano, }\AttributeTok{scales =} \StringTok{"fixed"}\NormalTok{) }\SpecialCharTok{+}
  
  \CommentTok{\# Rótulos com quantidade e proporção}
  \FunctionTok{geom\_text}\NormalTok{(}\FunctionTok{aes}\NormalTok{(}\AttributeTok{label =} \FunctionTok{paste0}\NormalTok{(Quantidade, }\StringTok{"}\SpecialCharTok{\textbackslash{}n}\StringTok{("}\NormalTok{, Proporção, }\StringTok{"\%)"}\NormalTok{)), }
            \AttributeTok{position =} \FunctionTok{position\_dodge}\NormalTok{(}\AttributeTok{width =} \FloatTok{0.9}\NormalTok{), }
            \AttributeTok{vjust =} \SpecialCharTok{{-}}\FloatTok{0.2}\NormalTok{, }
            \AttributeTok{size =} \DecValTok{3}\NormalTok{,}
            \AttributeTok{fontface =} \StringTok{"bold"}\NormalTok{) }\SpecialCharTok{+}
  
  \FunctionTok{scale\_fill\_manual}\NormalTok{(}\AttributeTok{values =} \FunctionTok{c}\NormalTok{(}\StringTok{"Denúncia"} \OtherTok{=} \StringTok{"\#e74c3c"}\NormalTok{, }\StringTok{"Arquivamento"} \OtherTok{=} \StringTok{"\#3498db"}\NormalTok{)) }\SpecialCharTok{+}
  \FunctionTok{scale\_y\_continuous}\NormalTok{(}
    \AttributeTok{labels =} \FunctionTok{comma\_format}\NormalTok{(}\AttributeTok{big.mark =} \StringTok{"."}\NormalTok{, }\AttributeTok{decimal.mark =} \StringTok{","}\NormalTok{),}
    \CommentTok{\# Expandindo os limites para acomodar os rótulos}
    \AttributeTok{expand =} \FunctionTok{expansion}\NormalTok{(}\AttributeTok{mult =} \FunctionTok{c}\NormalTok{(}\DecValTok{0}\NormalTok{, }\FloatTok{0.15}\NormalTok{))}
\NormalTok{  ) }\SpecialCharTok{+}
  
  \FunctionTok{labs}\NormalTok{(}
    \AttributeTok{title =} \StringTok{"AJCMs por Localização, Providência e Ano"}\NormalTok{,}
    \AttributeTok{subtitle =} \StringTok{"Valores absolutos e proporções (\%) por localização"}\NormalTok{,}
    \AttributeTok{x =} \StringTok{"Localização"}\NormalTok{,}
    \AttributeTok{y =} \StringTok{"Número de AJCMs"}\NormalTok{,}
    \AttributeTok{fill =} \StringTok{"Providência"}\NormalTok{,}
    \AttributeTok{caption =} \StringTok{"Fonte: Dados Oltramari | Proporções calculadas por ano e localização"}
\NormalTok{  ) }\SpecialCharTok{+}
  
  \FunctionTok{theme\_minimal}\NormalTok{() }\SpecialCharTok{+}
  \FunctionTok{theme}\NormalTok{(}
    \AttributeTok{plot.title =} \FunctionTok{element\_text}\NormalTok{(}\AttributeTok{hjust =} \FloatTok{0.5}\NormalTok{, }\AttributeTok{size =} \DecValTok{14}\NormalTok{, }\AttributeTok{face =} \StringTok{"bold"}\NormalTok{),}
    \AttributeTok{plot.subtitle =} \FunctionTok{element\_text}\NormalTok{(}\AttributeTok{hjust =} \FloatTok{0.5}\NormalTok{, }\AttributeTok{size =} \DecValTok{12}\NormalTok{),}
    \AttributeTok{plot.caption =} \FunctionTok{element\_text}\NormalTok{(}\AttributeTok{hjust =} \FloatTok{0.5}\NormalTok{, }\AttributeTok{size =} \DecValTok{10}\NormalTok{),}
    \AttributeTok{strip.text =} \FunctionTok{element\_text}\NormalTok{(}\AttributeTok{size =} \DecValTok{12}\NormalTok{, }\AttributeTok{face =} \StringTok{"bold"}\NormalTok{)}
\NormalTok{  )}

\FunctionTok{print}\NormalTok{(p1)}
\InformationTok{\textasciigrave{}\textasciigrave{}\textasciigrave{}}
\end{Highlighting}
\end{Shaded}

\pandocbounded{\includegraphics[keepaspectratio]{cap17-TSHo-Oltramari_files/figure-pdf/unnamed-chunk-6-1.pdf}}

Essa \textbf{mudança de escala no eixo y} é fundamental para percepção
de que o crescimento do volume de AJCM's pode alterar o padrão da
proporção de denúncias propostas pelos Promotores de Justiça da Capital
(VAM) e do Interior, para os anos de 2022 e 2023, após a vigência da
Resolução CPJ n.~04/2022 do MP-GO, de 28 de março de 2022.

Período observado foi: de 30 de março de 2022 a 31 de outubro de 2023.

Aqui está o script R comentado para teste de significância adequado aos
dados do gráfico anterior:

\begin{Shaded}
\begin{Highlighting}[numbers=left,,]
\InformationTok{\textasciigrave{}\textasciigrave{}\textasciigrave{}\{r\}}
\CommentTok{\# ==============================================================================}
\CommentTok{\# TESTE DE SIGNIFICÂNCIA PARA ASSOCIAÇÃO ENTRE VARIÁVEIS CATEGÓRICAS}
\CommentTok{\# Análise das AJCMs por Localização, Providência e Ano}
\CommentTok{\# ==============================================================================}

\CommentTok{\# Carregando bibliotecas necessárias}
\FunctionTok{library}\NormalTok{(dplyr)}
\FunctionTok{library}\NormalTok{(vcd)        }\CommentTok{\# Para testes de associação avançados}
\FunctionTok{library}\NormalTok{(DescTools)  }\CommentTok{\# Para estatísticas descritivas adicionais}

\CommentTok{\# ==============================================================================}
\CommentTok{\# 1. TESTE QUI{-}QUADRADO DE INDEPENDÊNCIA (ANÁLISE GERAL)}
\CommentTok{\# ==============================================================================}

\CommentTok{\# H₀: Não há associação entre Localização e Providência}
\CommentTok{\# H₁: Há associação entre Localização e Providência}

\FunctionTok{cat}\NormalTok{(}\StringTok{"TESTE QUI{-}QUADRADO DE INDEPENDÊNCIA}\SpecialCharTok{\textbackslash{}n}\StringTok{"}\NormalTok{)}
\FunctionTok{cat}\NormalTok{(}\StringTok{"===================================}\SpecialCharTok{\textbackslash{}n\textbackslash{}n}\StringTok{"}\NormalTok{)}

\CommentTok{\# Criando tabela de contingência agregada (todos os anos)}
\NormalTok{tabela\_geral }\OtherTok{\textless{}{-}}\NormalTok{ dados\_ajcm\_completo }\SpecialCharTok{\%\textgreater{}\%}
  \FunctionTok{group\_by}\NormalTok{(Localização, Providência) }\SpecialCharTok{\%\textgreater{}\%}
  \FunctionTok{summarise}\NormalTok{(}\AttributeTok{Total =} \FunctionTok{sum}\NormalTok{(Quantidade), }\AttributeTok{.groups =} \StringTok{"drop"}\NormalTok{) }\SpecialCharTok{\%\textgreater{}\%}
  \FunctionTok{pivot\_wider}\NormalTok{(}\AttributeTok{names\_from =}\NormalTok{ Providência, }\AttributeTok{values\_from =}\NormalTok{ Total) }\SpecialCharTok{\%\textgreater{}\%}
  \FunctionTok{column\_to\_rownames}\NormalTok{(}\StringTok{"Localização"}\NormalTok{) }\SpecialCharTok{\%\textgreater{}\%}
  \FunctionTok{as.matrix}\NormalTok{()}

\CommentTok{\# Exibindo a tabela}
\FunctionTok{cat}\NormalTok{(}\StringTok{"TABELA DE CONTINGÊNCIA (AGREGADA):}\SpecialCharTok{\textbackslash{}n}\StringTok{"}\NormalTok{)}
\FunctionTok{print}\NormalTok{(tabela\_geral)}
\FunctionTok{cat}\NormalTok{(}\StringTok{"}\SpecialCharTok{\textbackslash{}n}\StringTok{"}\NormalTok{)}

\CommentTok{\# Realizando o teste qui{-}quadrado}
\NormalTok{teste\_qui\_geral }\OtherTok{\textless{}{-}} \FunctionTok{chisq.test}\NormalTok{(tabela\_geral)}

\CommentTok{\# Resultados do teste}
\FunctionTok{cat}\NormalTok{(}\StringTok{"RESULTADOS DO TESTE QUI{-}QUADRADO:}\SpecialCharTok{\textbackslash{}n}\StringTok{"}\NormalTok{)}
\FunctionTok{cat}\NormalTok{(}\StringTok{"χ² ="}\NormalTok{, }\FunctionTok{round}\NormalTok{(teste\_qui\_geral}\SpecialCharTok{$}\NormalTok{statistic, }\DecValTok{4}\NormalTok{), }\StringTok{"}\SpecialCharTok{\textbackslash{}n}\StringTok{"}\NormalTok{)}
\FunctionTok{cat}\NormalTok{(}\StringTok{"gl ="}\NormalTok{, teste\_qui\_geral}\SpecialCharTok{$}\NormalTok{parameter, }\StringTok{"}\SpecialCharTok{\textbackslash{}n}\StringTok{"}\NormalTok{)}
\FunctionTok{cat}\NormalTok{(}\StringTok{"p{-}valor ="}\NormalTok{, }\FunctionTok{ifelse}\NormalTok{(teste\_qui\_geral}\SpecialCharTok{$}\NormalTok{p.value }\SpecialCharTok{\textless{}} \FloatTok{0.001}\NormalTok{, }\StringTok{"\textless{} 0.001"}\NormalTok{, }
                       \FunctionTok{round}\NormalTok{(teste\_qui\_geral}\SpecialCharTok{$}\NormalTok{p.value, }\DecValTok{6}\NormalTok{)), }\StringTok{"}\SpecialCharTok{\textbackslash{}n}\StringTok{"}\NormalTok{)}
\FunctionTok{cat}\NormalTok{(}\StringTok{"Conclusão:"}\NormalTok{, }\FunctionTok{ifelse}\NormalTok{(teste\_qui\_geral}\SpecialCharTok{$}\NormalTok{p.value }\SpecialCharTok{\textless{}} \FloatTok{0.05}\NormalTok{, }
                        \StringTok{"REJEITA H₀ (há associação significativa)"}\NormalTok{, }
                        \StringTok{"NÃO REJEITA H₀ (não há associação significativa)"}\NormalTok{), }\StringTok{"}\SpecialCharTok{\textbackslash{}n\textbackslash{}n}\StringTok{"}\NormalTok{)}

\CommentTok{\# Verificando pressupostos do teste}
\FunctionTok{cat}\NormalTok{(}\StringTok{"VERIFICAÇÃO DE PRESSUPOSTOS:}\SpecialCharTok{\textbackslash{}n}\StringTok{"}\NormalTok{)}
\NormalTok{frequencias\_esperadas }\OtherTok{\textless{}{-}}\NormalTok{ teste\_qui\_geral}\SpecialCharTok{$}\NormalTok{expected}
\FunctionTok{cat}\NormalTok{(}\StringTok{"Frequências esperadas mínimas:"}\NormalTok{, }\FunctionTok{round}\NormalTok{(}\FunctionTok{min}\NormalTok{(frequencias\_esperadas), }\DecValTok{2}\NormalTok{), }\StringTok{"}\SpecialCharTok{\textbackslash{}n}\StringTok{"}\NormalTok{)}
\FunctionTok{cat}\NormalTok{(}\StringTok{"Pressuposto atendido:"}\NormalTok{, }\FunctionTok{ifelse}\NormalTok{(}\FunctionTok{min}\NormalTok{(frequencias\_esperadas) }\SpecialCharTok{\textgreater{}=} \DecValTok{5}\NormalTok{, }\StringTok{"SIM"}\NormalTok{, }\StringTok{"NÃO"}\NormalTok{), }
    \StringTok{"(todas as células ≥ 5)}\SpecialCharTok{\textbackslash{}n\textbackslash{}n}\StringTok{"}\NormalTok{)}

\CommentTok{\# ==============================================================================}
\CommentTok{\# 2. TESTE DE HOMOGENEIDADE ENTRE ANOS (ANÁLISE TEMPORAL)}
\CommentTok{\# ==============================================================================}

\FunctionTok{cat}\NormalTok{(}\StringTok{"TESTE DE HOMOGENEIDADE ENTRE ANOS}\SpecialCharTok{\textbackslash{}n}\StringTok{"}\NormalTok{)}
\FunctionTok{cat}\NormalTok{(}\StringTok{"=================================}\SpecialCharTok{\textbackslash{}n\textbackslash{}n}\StringTok{"}\NormalTok{)}

\CommentTok{\# H₀: A distribuição de providências é homogênea entre os anos}
\CommentTok{\# H₁: A distribuição de providências difere entre os anos}

\CommentTok{\# Teste qui{-}quadrado para cada localização separadamente}
\ControlFlowTok{for}\NormalTok{(local }\ControlFlowTok{in} \FunctionTok{c}\NormalTok{(}\StringTok{"Capital"}\NormalTok{, }\StringTok{"Interior"}\NormalTok{)) \{}
  
  \FunctionTok{cat}\NormalTok{(}\StringTok{"LOCALIZAÇÃO:"}\NormalTok{, }\FunctionTok{toupper}\NormalTok{(local), }\StringTok{"}\SpecialCharTok{\textbackslash{}n}\StringTok{"}\NormalTok{)}
  \FunctionTok{cat}\NormalTok{(}\FunctionTok{rep}\NormalTok{(}\StringTok{"{-}"}\NormalTok{, }\FunctionTok{nchar}\NormalTok{(local) }\SpecialCharTok{+} \DecValTok{13}\NormalTok{), }\StringTok{"}\SpecialCharTok{\textbackslash{}n}\StringTok{"}\NormalTok{, }\AttributeTok{sep =} \StringTok{""}\NormalTok{)}
  
  \CommentTok{\# Criando tabela para a localização específica}
\NormalTok{  dados\_local }\OtherTok{\textless{}{-}}\NormalTok{ dados\_ajcm\_completo }\SpecialCharTok{\%\textgreater{}\%}
    \FunctionTok{filter}\NormalTok{(Localização }\SpecialCharTok{==}\NormalTok{ local) }\SpecialCharTok{\%\textgreater{}\%}
    \FunctionTok{select}\NormalTok{(Ano, Providência, Quantidade) }\SpecialCharTok{\%\textgreater{}\%}
    \FunctionTok{pivot\_wider}\NormalTok{(}\AttributeTok{names\_from =}\NormalTok{ Providência, }\AttributeTok{values\_from =}\NormalTok{ Quantidade) }\SpecialCharTok{\%\textgreater{}\%}
    \FunctionTok{column\_to\_rownames}\NormalTok{(}\StringTok{"Ano"}\NormalTok{) }\SpecialCharTok{\%\textgreater{}\%}
    \FunctionTok{as.matrix}\NormalTok{()}
  
  \FunctionTok{print}\NormalTok{(dados\_local)}
  
  \CommentTok{\# Teste qui{-}quadrado}
\NormalTok{  teste\_local }\OtherTok{\textless{}{-}} \FunctionTok{chisq.test}\NormalTok{(dados\_local)}
  
  \FunctionTok{cat}\NormalTok{(}\StringTok{"χ² ="}\NormalTok{, }\FunctionTok{round}\NormalTok{(teste\_local}\SpecialCharTok{$}\NormalTok{statistic, }\DecValTok{4}\NormalTok{), }\StringTok{"}\SpecialCharTok{\textbackslash{}n}\StringTok{"}\NormalTok{)}
  \FunctionTok{cat}\NormalTok{(}\StringTok{"gl ="}\NormalTok{, teste\_local}\SpecialCharTok{$}\NormalTok{parameter, }\StringTok{"}\SpecialCharTok{\textbackslash{}n}\StringTok{"}\NormalTok{)}
  \FunctionTok{cat}\NormalTok{(}\StringTok{"p{-}valor ="}\NormalTok{, }\FunctionTok{ifelse}\NormalTok{(teste\_local}\SpecialCharTok{$}\NormalTok{p.value }\SpecialCharTok{\textless{}} \FloatTok{0.001}\NormalTok{, }\StringTok{"\textless{} 0.001"}\NormalTok{, }
                         \FunctionTok{round}\NormalTok{(teste\_local}\SpecialCharTok{$}\NormalTok{p.value, }\DecValTok{6}\NormalTok{)), }\StringTok{"}\SpecialCharTok{\textbackslash{}n}\StringTok{"}\NormalTok{)}
  \FunctionTok{cat}\NormalTok{(}\StringTok{"Conclusão:"}\NormalTok{, }\FunctionTok{ifelse}\NormalTok{(teste\_local}\SpecialCharTok{$}\NormalTok{p.value }\SpecialCharTok{\textless{}} \FloatTok{0.05}\NormalTok{, }
                          \StringTok{"Há diferença significativa entre anos"}\NormalTok{, }
                          \StringTok{"Não há diferença significativa entre anos"}\NormalTok{), }\StringTok{"}\SpecialCharTok{\textbackslash{}n\textbackslash{}n}\StringTok{"}\NormalTok{)}
\NormalTok{\}}

\CommentTok{\# ==============================================================================}
\CommentTok{\# 3. TESTE DE MCNEMAR (SE DADOS FOREM PAREADOS)}
\CommentTok{\# ==============================================================================}

\CommentTok{\# Aplicável se os mesmos locais/unidades foram medidos em ambos os anos}
\CommentTok{\# Este teste é mais apropriado para dados pareados temporais}

\FunctionTok{cat}\NormalTok{(}\StringTok{"TESTE DE MCNEMAR (DADOS PAREADOS)}\SpecialCharTok{\textbackslash{}n}\StringTok{"}\NormalTok{)}
\FunctionTok{cat}\NormalTok{(}\StringTok{"=================================}\SpecialCharTok{\textbackslash{}n\textbackslash{}n}\StringTok{"}\NormalTok{)}

\CommentTok{\# Criando tabela 2x2 para cada localização (2022 vs 2023)}
\CommentTok{\# Comparando proporções de denúncias entre anos}

\ControlFlowTok{for}\NormalTok{(local }\ControlFlowTok{in} \FunctionTok{c}\NormalTok{(}\StringTok{"Capital"}\NormalTok{, }\StringTok{"Interior"}\NormalTok{)) \{}
  
  \FunctionTok{cat}\NormalTok{(}\StringTok{"TESTE DE MCNEMAR {-}"}\NormalTok{, }\FunctionTok{toupper}\NormalTok{(local), }\StringTok{"}\SpecialCharTok{\textbackslash{}n}\StringTok{"}\NormalTok{)}
  \FunctionTok{cat}\NormalTok{(}\FunctionTok{rep}\NormalTok{(}\StringTok{"{-}"}\NormalTok{, }\FunctionTok{nchar}\NormalTok{(local) }\SpecialCharTok{+} \DecValTok{18}\NormalTok{), }\StringTok{"}\SpecialCharTok{\textbackslash{}n}\StringTok{"}\NormalTok{, }\AttributeTok{sep =} \StringTok{""}\NormalTok{)}
  
  \CommentTok{\# Extraindo dados para a localização}
\NormalTok{  dados\_mcnemar }\OtherTok{\textless{}{-}}\NormalTok{ dados\_ajcm\_completo }\SpecialCharTok{\%\textgreater{}\%}
    \FunctionTok{filter}\NormalTok{(Localização }\SpecialCharTok{==}\NormalTok{ local) }\SpecialCharTok{\%\textgreater{}\%}
    \FunctionTok{select}\NormalTok{(Ano, Providência, Quantidade) }\SpecialCharTok{\%\textgreater{}\%}
    \FunctionTok{pivot\_wider}\NormalTok{(}\AttributeTok{names\_from =}\NormalTok{ Ano, }\AttributeTok{values\_from =}\NormalTok{ Quantidade) }\SpecialCharTok{\%\textgreater{}\%}
    \FunctionTok{column\_to\_rownames}\NormalTok{(}\StringTok{"Providência"}\NormalTok{) }\SpecialCharTok{\%\textgreater{}\%}
    \FunctionTok{as.matrix}\NormalTok{()}
  
  \FunctionTok{print}\NormalTok{(dados\_mcnemar)}
  
  \CommentTok{\# Aplicando teste de McNemar}
  \CommentTok{\# Nota: Este teste requer dados em formato específico}
  \CommentTok{\# Para proporções, usamos qui{-}quadrado de homogeneidade}
  
\NormalTok{  teste\_mcnemar }\OtherTok{\textless{}{-}} \FunctionTok{chisq.test}\NormalTok{(dados\_mcnemar)}
  
  \FunctionTok{cat}\NormalTok{(}\StringTok{"χ² ="}\NormalTok{, }\FunctionTok{round}\NormalTok{(teste\_mcnemar}\SpecialCharTok{$}\NormalTok{statistic, }\DecValTok{4}\NormalTok{), }\StringTok{"}\SpecialCharTok{\textbackslash{}n}\StringTok{"}\NormalTok{)}
  \FunctionTok{cat}\NormalTok{(}\StringTok{"p{-}valor ="}\NormalTok{, }\FunctionTok{ifelse}\NormalTok{(teste\_mcnemar}\SpecialCharTok{$}\NormalTok{p.value }\SpecialCharTok{\textless{}} \FloatTok{0.001}\NormalTok{, }\StringTok{"\textless{} 0.001"}\NormalTok{, }
                         \FunctionTok{round}\NormalTok{(teste\_mcnemar}\SpecialCharTok{$}\NormalTok{p.value, }\DecValTok{6}\NormalTok{)), }\StringTok{"}\SpecialCharTok{\textbackslash{}n}\StringTok{"}\NormalTok{)}
  \FunctionTok{cat}\NormalTok{(}\StringTok{"Interpretação:"}\NormalTok{, }\FunctionTok{ifelse}\NormalTok{(teste\_mcnemar}\SpecialCharTok{$}\NormalTok{p.value }\SpecialCharTok{\textless{}} \FloatTok{0.05}\NormalTok{, }
                             \StringTok{"Mudança significativa nas proporções entre 2022 e 2023"}\NormalTok{, }
                             \StringTok{"Não há mudança significativa nas proporções"}\NormalTok{), }\StringTok{"}\SpecialCharTok{\textbackslash{}n\textbackslash{}n}\StringTok{"}\NormalTok{)}
\NormalTok{\}}

\CommentTok{\# ==============================================================================}
\CommentTok{\# 4. ANÁLISE DE RESÍDUOS PADRONIZADOS}
\CommentTok{\# ==============================================================================}

\FunctionTok{cat}\NormalTok{(}\StringTok{"ANÁLISE DE RESÍDUOS PADRONIZADOS}\SpecialCharTok{\textbackslash{}n}\StringTok{"}\NormalTok{)}
\FunctionTok{cat}\NormalTok{(}\StringTok{"================================}\SpecialCharTok{\textbackslash{}n\textbackslash{}n}\StringTok{"}\NormalTok{)}

\CommentTok{\# Calculando resíduos padronizados para identificar células que mais contribuem}
\CommentTok{\# para a associação (se houver)}

\NormalTok{residuos }\OtherTok{\textless{}{-}}\NormalTok{ teste\_qui\_geral}\SpecialCharTok{$}\NormalTok{residuals}
\NormalTok{residuos\_padronizados }\OtherTok{\textless{}{-}}\NormalTok{ teste\_qui\_geral}\SpecialCharTok{$}\NormalTok{stdres}

\FunctionTok{cat}\NormalTok{(}\StringTok{"RESÍDUOS PADRONIZADOS:}\SpecialCharTok{\textbackslash{}n}\StringTok{"}\NormalTok{)}
\FunctionTok{print}\NormalTok{(}\FunctionTok{round}\NormalTok{(residuos\_padronizados, }\DecValTok{3}\NormalTok{))}
\FunctionTok{cat}\NormalTok{(}\StringTok{"}\SpecialCharTok{\textbackslash{}n}\StringTok{"}\NormalTok{)}

\FunctionTok{cat}\NormalTok{(}\StringTok{"INTERPRETAÇÃO DOS RESÍDUOS:}\SpecialCharTok{\textbackslash{}n}\StringTok{"}\NormalTok{)}
\FunctionTok{cat}\NormalTok{(}\StringTok{"Valores \textgreater{} |2|: contribuição significativa para associação}\SpecialCharTok{\textbackslash{}n}\StringTok{"}\NormalTok{)}
\FunctionTok{cat}\NormalTok{(}\StringTok{"Valores \textgreater{} |3|: contribuição muito significativa}\SpecialCharTok{\textbackslash{}n\textbackslash{}n}\StringTok{"}\NormalTok{)}

\CommentTok{\# Identificando células com maior contribuição}
\NormalTok{células\_significativas }\OtherTok{\textless{}{-}} \FunctionTok{which}\NormalTok{(}\FunctionTok{abs}\NormalTok{(residuos\_padronizados) }\SpecialCharTok{\textgreater{}} \DecValTok{2}\NormalTok{, }\AttributeTok{arr.ind =} \ConstantTok{TRUE}\NormalTok{)}
\ControlFlowTok{if}\NormalTok{(}\FunctionTok{nrow}\NormalTok{(células\_significativas) }\SpecialCharTok{\textgreater{}} \DecValTok{0}\NormalTok{) \{}
  \FunctionTok{cat}\NormalTok{(}\StringTok{"CÉLULAS COM CONTRIBUIÇÃO SIGNIFICATIVA:}\SpecialCharTok{\textbackslash{}n}\StringTok{"}\NormalTok{)}
  \ControlFlowTok{for}\NormalTok{(i }\ControlFlowTok{in} \DecValTok{1}\SpecialCharTok{:}\FunctionTok{nrow}\NormalTok{(células\_significativas)) \{}
\NormalTok{    linha }\OtherTok{\textless{}{-}} \FunctionTok{rownames}\NormalTok{(residuos\_padronizados)[células\_significativas[i,}\DecValTok{1}\NormalTok{]]}
\NormalTok{    coluna }\OtherTok{\textless{}{-}} \FunctionTok{colnames}\NormalTok{(residuos\_padronizados)[células\_significativas[i,}\DecValTok{2}\NormalTok{]]}
\NormalTok{    valor }\OtherTok{\textless{}{-}}\NormalTok{ residuos\_padronizados[células\_significativas[i,}\DecValTok{1}\NormalTok{], células\_significativas[i,}\DecValTok{2}\NormalTok{]]}
    \FunctionTok{cat}\NormalTok{(}\StringTok{"{-}"}\NormalTok{, linha, }\StringTok{"×"}\NormalTok{, coluna, }\StringTok{": resíduo ="}\NormalTok{, }\FunctionTok{round}\NormalTok{(valor, }\DecValTok{3}\NormalTok{), }\StringTok{"}\SpecialCharTok{\textbackslash{}n}\StringTok{"}\NormalTok{)}
\NormalTok{  \}}
\NormalTok{\} }\ControlFlowTok{else}\NormalTok{ \{}
  \FunctionTok{cat}\NormalTok{(}\StringTok{"Nenhuma célula com contribuição significativa identificada.}\SpecialCharTok{\textbackslash{}n}\StringTok{"}\NormalTok{)}
\NormalTok{\}}

\CommentTok{\# ==============================================================================}
\CommentTok{\# 5. COEFICIENTES DE ASSOCIAÇÃO}
\CommentTok{\# ==============================================================================}

\FunctionTok{cat}\NormalTok{(}\StringTok{"}\SpecialCharTok{\textbackslash{}n}\StringTok{COEFICIENTES DE ASSOCIAÇÃO}\SpecialCharTok{\textbackslash{}n}\StringTok{"}\NormalTok{)}
\FunctionTok{cat}\NormalTok{(}\StringTok{"=========================}\SpecialCharTok{\textbackslash{}n}\StringTok{"}\NormalTok{)}

\CommentTok{\# V de Cramér (intensidade da associação)}
\NormalTok{v\_cramer }\OtherTok{\textless{}{-}} \FunctionTok{sqrt}\NormalTok{(teste\_qui\_geral}\SpecialCharTok{$}\NormalTok{statistic }\SpecialCharTok{/}\NormalTok{ (}\FunctionTok{sum}\NormalTok{(tabela\_geral) }\SpecialCharTok{*}\NormalTok{ (}\FunctionTok{min}\NormalTok{(}\FunctionTok{dim}\NormalTok{(tabela\_geral)) }\SpecialCharTok{{-}} \DecValTok{1}\NormalTok{)))}
\FunctionTok{cat}\NormalTok{(}\StringTok{"V de Cramér:"}\NormalTok{, }\FunctionTok{round}\NormalTok{(v\_cramer, }\DecValTok{4}\NormalTok{), }\StringTok{"}\SpecialCharTok{\textbackslash{}n}\StringTok{"}\NormalTok{)}
\FunctionTok{cat}\NormalTok{(}\StringTok{"Interpretação:"}\NormalTok{, }
    \FunctionTok{ifelse}\NormalTok{(v\_cramer }\SpecialCharTok{\textless{}} \FloatTok{0.1}\NormalTok{, }\StringTok{"Associação fraca"}\NormalTok{,}
           \FunctionTok{ifelse}\NormalTok{(v\_cramer }\SpecialCharTok{\textless{}} \FloatTok{0.3}\NormalTok{, }\StringTok{"Associação moderada"}\NormalTok{, }\StringTok{"Associação forte"}\NormalTok{)), }\StringTok{"}\SpecialCharTok{\textbackslash{}n}\StringTok{"}\NormalTok{)}

\CommentTok{\# Coeficiente Phi (para tabelas 2x2)}
\NormalTok{phi }\OtherTok{\textless{}{-}} \FunctionTok{sqrt}\NormalTok{(teste\_qui\_geral}\SpecialCharTok{$}\NormalTok{statistic }\SpecialCharTok{/} \FunctionTok{sum}\NormalTok{(tabela\_geral))}
\FunctionTok{cat}\NormalTok{(}\StringTok{"Coeficiente Phi:"}\NormalTok{, }\FunctionTok{round}\NormalTok{(phi, }\DecValTok{4}\NormalTok{), }\StringTok{"}\SpecialCharTok{\textbackslash{}n\textbackslash{}n}\StringTok{"}\NormalTok{)}

\CommentTok{\# ==============================================================================}
\CommentTok{\# 6. TESTE DE TENDÊNCIA LINEAR (SE APLICÁVEL)}
\CommentTok{\# ==============================================================================}

\FunctionTok{cat}\NormalTok{(}\StringTok{"TESTE DE TENDÊNCIA TEMPORAL}\SpecialCharTok{\textbackslash{}n}\StringTok{"}\NormalTok{)}
\FunctionTok{cat}\NormalTok{(}\StringTok{"===========================}\SpecialCharTok{\textbackslash{}n\textbackslash{}n}\StringTok{"}\NormalTok{)}

\CommentTok{\# Testando se há tendência linear nas proporções ao longo do tempo}
\CommentTok{\# Usando teste de Cochran{-}Armitage para tendência}

\CommentTok{\# Para cada localização, testamos se a proporção de denúncias muda linearmente}
\ControlFlowTok{for}\NormalTok{(local }\ControlFlowTok{in} \FunctionTok{c}\NormalTok{(}\StringTok{"Capital"}\NormalTok{, }\StringTok{"Interior"}\NormalTok{)) \{}
  
  \FunctionTok{cat}\NormalTok{(}\StringTok{"TENDÊNCIA {-}"}\NormalTok{, }\FunctionTok{toupper}\NormalTok{(local), }\StringTok{"}\SpecialCharTok{\textbackslash{}n}\StringTok{"}\NormalTok{)}
  
\NormalTok{  dados\_tendencia }\OtherTok{\textless{}{-}}\NormalTok{ dados\_ajcm\_completo }\SpecialCharTok{\%\textgreater{}\%}
    \FunctionTok{filter}\NormalTok{(Localização }\SpecialCharTok{==}\NormalTok{ local) }\SpecialCharTok{\%\textgreater{}\%}
    \FunctionTok{mutate}\NormalTok{(}\AttributeTok{Ano\_num =} \FunctionTok{as.numeric}\NormalTok{(Ano)) }\SpecialCharTok{\%\textgreater{}\%}
    \FunctionTok{group\_by}\NormalTok{(Ano\_num) }\SpecialCharTok{\%\textgreater{}\%}
    \FunctionTok{mutate}\NormalTok{(}\AttributeTok{Total\_ano =} \FunctionTok{sum}\NormalTok{(Quantidade)) }\SpecialCharTok{\%\textgreater{}\%}
    \FunctionTok{filter}\NormalTok{(Providência }\SpecialCharTok{==} \StringTok{"Denúncia"}\NormalTok{) }\SpecialCharTok{\%\textgreater{}\%}
    \FunctionTok{mutate}\NormalTok{(}\AttributeTok{Prop\_denuncia =}\NormalTok{ Quantidade }\SpecialCharTok{/}\NormalTok{ Total\_ano)}
  
  \CommentTok{\# Correlação simples como aproximação}
  \ControlFlowTok{if}\NormalTok{(}\FunctionTok{nrow}\NormalTok{(dados\_tendencia) }\SpecialCharTok{\textgreater{}} \DecValTok{1}\NormalTok{) \{}
\NormalTok{    correlacao }\OtherTok{\textless{}{-}} \FunctionTok{cor}\NormalTok{(dados\_tendencia}\SpecialCharTok{$}\NormalTok{Ano\_num, dados\_tendencia}\SpecialCharTok{$}\NormalTok{Prop\_denuncia)}
    \FunctionTok{cat}\NormalTok{(}\StringTok{"Correlação Ano × Proporção Denúncia:"}\NormalTok{, }\FunctionTok{round}\NormalTok{(correlacao, }\DecValTok{4}\NormalTok{), }\StringTok{"}\SpecialCharTok{\textbackslash{}n}\StringTok{"}\NormalTok{)}
    
    \CommentTok{\# Teste t para correlação}
\NormalTok{    n }\OtherTok{\textless{}{-}} \FunctionTok{nrow}\NormalTok{(dados\_tendencia)}
\NormalTok{    t\_stat }\OtherTok{\textless{}{-}}\NormalTok{ correlacao }\SpecialCharTok{*} \FunctionTok{sqrt}\NormalTok{((n}\DecValTok{{-}2}\NormalTok{)}\SpecialCharTok{/}\NormalTok{(}\DecValTok{1}\SpecialCharTok{{-}}\NormalTok{correlacao}\SpecialCharTok{\^{}}\DecValTok{2}\NormalTok{))}
\NormalTok{    p\_valor }\OtherTok{\textless{}{-}} \DecValTok{2} \SpecialCharTok{*}\NormalTok{ (}\DecValTok{1} \SpecialCharTok{{-}} \FunctionTok{pt}\NormalTok{(}\FunctionTok{abs}\NormalTok{(t\_stat), n}\DecValTok{{-}2}\NormalTok{))}
    
    \FunctionTok{cat}\NormalTok{(}\StringTok{"p{-}valor da correlação:"}\NormalTok{, }\FunctionTok{ifelse}\NormalTok{(p\_valor }\SpecialCharTok{\textless{}} \FloatTok{0.001}\NormalTok{, }\StringTok{"\textless{} 0.001"}\NormalTok{, }
                                        \FunctionTok{round}\NormalTok{(p\_valor, }\DecValTok{6}\NormalTok{)), }\StringTok{"}\SpecialCharTok{\textbackslash{}n}\StringTok{"}\NormalTok{)}
    \FunctionTok{cat}\NormalTok{(}\StringTok{"Tendência:"}\NormalTok{, }\FunctionTok{ifelse}\NormalTok{(p\_valor }\SpecialCharTok{\textless{}} \FloatTok{0.05}\NormalTok{, }
                            \FunctionTok{ifelse}\NormalTok{(correlacao }\SpecialCharTok{\textgreater{}} \DecValTok{0}\NormalTok{, }\StringTok{"Crescente significativa"}\NormalTok{, }
                                   \StringTok{"Decrescente significativa"}\NormalTok{), }
                            \StringTok{"Não significativa"}\NormalTok{), }\StringTok{"}\SpecialCharTok{\textbackslash{}n\textbackslash{}n}\StringTok{"}\NormalTok{)}
\NormalTok{  \}}
\NormalTok{\}}

\CommentTok{\# ==============================================================================}
\CommentTok{\# 7. RESUMO EXECUTIVO DOS TESTES}
\CommentTok{\# ==============================================================================}

\FunctionTok{cat}\NormalTok{(}\StringTok{"RESUMO EXECUTIVO DOS TESTES ESTATÍSTICOS}\SpecialCharTok{\textbackslash{}n}\StringTok{"}\NormalTok{)}
\FunctionTok{cat}\NormalTok{(}\StringTok{"========================================}\SpecialCharTok{\textbackslash{}n\textbackslash{}n}\StringTok{"}\NormalTok{)}

\FunctionTok{cat}\NormalTok{(}\StringTok{"1. ASSOCIAÇÃO GERAL (Localização × Providência):}\SpecialCharTok{\textbackslash{}n}\StringTok{"}\NormalTok{)}
\FunctionTok{cat}\NormalTok{(}\StringTok{"   {-} Teste: Qui{-}quadrado de independência}\SpecialCharTok{\textbackslash{}n}\StringTok{"}\NormalTok{)}
\FunctionTok{cat}\NormalTok{(}\StringTok{"   {-} Resultado:"}\NormalTok{, }\FunctionTok{ifelse}\NormalTok{(teste\_qui\_geral}\SpecialCharTok{$}\NormalTok{p.value }\SpecialCharTok{\textless{}} \FloatTok{0.05}\NormalTok{, }\StringTok{"SIGNIFICATIVO"}\NormalTok{, }\StringTok{"NÃO SIGNIFICATIVO"}\NormalTok{), }\StringTok{"}\SpecialCharTok{\textbackslash{}n}\StringTok{"}\NormalTok{)}
\FunctionTok{cat}\NormalTok{(}\StringTok{"   {-} Intensidade: V de Cramér ="}\NormalTok{, }\FunctionTok{round}\NormalTok{(v\_cramer, }\DecValTok{3}\NormalTok{), }
    \StringTok{"("}\NormalTok{, }\FunctionTok{ifelse}\NormalTok{(v\_cramer }\SpecialCharTok{\textless{}} \FloatTok{0.1}\NormalTok{, }\StringTok{"fraca"}\NormalTok{, }\FunctionTok{ifelse}\NormalTok{(v\_cramer }\SpecialCharTok{\textless{}} \FloatTok{0.3}\NormalTok{, }\StringTok{"moderada"}\NormalTok{, }\StringTok{"forte"}\NormalTok{)), }\StringTok{")}\SpecialCharTok{\textbackslash{}n\textbackslash{}n}\StringTok{"}\NormalTok{)}

\FunctionTok{cat}\NormalTok{(}\StringTok{"2. HOMOGENEIDADE TEMPORAL:}\SpecialCharTok{\textbackslash{}n}\StringTok{"}\NormalTok{)}
\FunctionTok{cat}\NormalTok{(}\StringTok{"   {-} Teste: Qui{-}quadrado por localização}\SpecialCharTok{\textbackslash{}n}\StringTok{"}\NormalTok{)}
\FunctionTok{cat}\NormalTok{(}\StringTok{"   {-} Avalia se as proporções mudaram entre 2022 e 2023}\SpecialCharTok{\textbackslash{}n\textbackslash{}n}\StringTok{"}\NormalTok{)}

\FunctionTok{cat}\NormalTok{(}\StringTok{"3. IMPLICAÇÕES PRÁTICAS:}\SpecialCharTok{\textbackslash{}n}\StringTok{"}\NormalTok{)}
\ControlFlowTok{if}\NormalTok{(teste\_qui\_geral}\SpecialCharTok{$}\NormalTok{p.value }\SpecialCharTok{\textless{}} \FloatTok{0.05}\NormalTok{) \{}
  \FunctionTok{cat}\NormalTok{(}\StringTok{"   {-} Há evidência de que o padrão de providências difere}\SpecialCharTok{\textbackslash{}n}\StringTok{"}\NormalTok{)}
  \FunctionTok{cat}\NormalTok{(}\StringTok{"     entre Capital e Interior}\SpecialCharTok{\textbackslash{}n}\StringTok{"}\NormalTok{)}
  \FunctionTok{cat}\NormalTok{(}\StringTok{"   {-} Recomenda{-}se investigar as causas dessa diferença}\SpecialCharTok{\textbackslash{}n}\StringTok{"}\NormalTok{)}
\NormalTok{\} }\ControlFlowTok{else}\NormalTok{ \{}
  \FunctionTok{cat}\NormalTok{(}\StringTok{"   {-} Não há evidência estatística de diferença no padrão}\SpecialCharTok{\textbackslash{}n}\StringTok{"}\NormalTok{)}
  \FunctionTok{cat}\NormalTok{(}\StringTok{"     de providências entre Capital e Interior}\SpecialCharTok{\textbackslash{}n}\StringTok{"}\NormalTok{)}
\NormalTok{\}}

\FunctionTok{cat}\NormalTok{(}\StringTok{"}\SpecialCharTok{\textbackslash{}n}\StringTok{NOTA: Todos os testes assumem α = 0.05}\SpecialCharTok{\textbackslash{}n}\StringTok{"}\NormalTok{)}
\InformationTok{\textasciigrave{}\textasciigrave{}\textasciigrave{}}
\end{Highlighting}
\end{Shaded}

\begin{verbatim}
TESTE QUI-QUADRADO DE INDEPENDÊNCIA
===================================

TABELA DE CONTINGÊNCIA (AGREGADA):
         Arquivamento Denúncia
Capital           786       70
Interior          686       83

RESULTADOS DO TESTE QUI-QUADRADO:
χ² = 2.9501 
gl = 1 
p-valor = 0.085874 
Conclusão: NÃO REJEITA H₀ (não há associação significativa) 

VERIFICAÇÃO DE PRESSUPOSTOS:
Frequências esperadas mínimas: 72.4 
Pressuposto atendido: SIM (todas as células ≥ 5)

TESTE DE HOMOGENEIDADE ENTRE ANOS
=================================

LOCALIZAÇÃO: CAPITAL 
--------------------
     Denúncia Arquivamento
2022       20          191
2023       50          595
χ² = 0.4223 
gl = 1 
p-valor = 0.515793 
Conclusão: Não há diferença significativa entre anos 

LOCALIZAÇÃO: INTERIOR 
---------------------
     Denúncia Arquivamento
2022       42          207
2023       41          479
χ² = 13.1934 
gl = 1 
p-valor = < 0.001 
Conclusão: Há diferença significativa entre anos 

TESTE DE MCNEMAR (DADOS PAREADOS)
=================================

TESTE DE MCNEMAR - CAPITAL 
-------------------------
             2022 2023
Denúncia       20   50
Arquivamento  191  595
χ² = 0.4223 
p-valor = 0.515793 
Interpretação: Não há mudança significativa nas proporções 

TESTE DE MCNEMAR - INTERIOR 
--------------------------
             2022 2023
Denúncia       42   41
Arquivamento  207  479
χ² = 13.1934 
p-valor = < 0.001 
Interpretação: Mudança significativa nas proporções entre 2022 e 2023 

ANÁLISE DE RESÍDUOS PADRONIZADOS
================================

RESÍDUOS PADRONIZADOS:
         Arquivamento Denúncia
Capital         1.803   -1.803
Interior       -1.803    1.803

INTERPRETAÇÃO DOS RESÍDUOS:
Valores > |2|: contribuição significativa para associação
Valores > |3|: contribuição muito significativa

Nenhuma célula com contribuição significativa identificada.

COEFICIENTES DE ASSOCIAÇÃO
=========================
V de Cramér: 0.0426 
Interpretação: Associação fraca 
Coeficiente Phi: 0.0426 

TESTE DE TENDÊNCIA TEMPORAL
===========================

TENDÊNCIA - CAPITAL 
Correlação Ano × Proporção Denúncia: -1 
p-valor da correlação: NA 
Tendência: NA 

TENDÊNCIA - INTERIOR 
Correlação Ano × Proporção Denúncia: -1 
p-valor da correlação: NA 
Tendência: NA 

RESUMO EXECUTIVO DOS TESTES ESTATÍSTICOS
========================================

1. ASSOCIAÇÃO GERAL (Localização × Providência):
   - Teste: Qui-quadrado de independência
   - Resultado: NÃO SIGNIFICATIVO 
   - Intensidade: V de Cramér = 0.043 ( fraca )

2. HOMOGENEIDADE TEMPORAL:
   - Teste: Qui-quadrado por localização
   - Avalia se as proporções mudaram entre 2022 e 2023

3. IMPLICAÇÕES PRÁTICAS:
   - Não há evidência estatística de diferença no padrão
     de providências entre Capital e Interior

NOTA: Todos os testes assumem α = 0.05
\end{verbatim}

mmm

\begin{center}\rule{0.5\linewidth}{0.5pt}\end{center}

\textbf{Referências}: Este material é baseado em Oltramari, Felipe.
\emph{Controle Externo da Atividade Policial como questão de Política
Pública}: análise do arranjo institucional da função
persecutória-investigativa no âmbito do Ministério Público. Goiânia:
2024, PPGDP (linha 2). Relatório final de pesquisa. (OLTRAMARI, 2024)

\bookmarksetup{startatroot}

\chapter{Inferência na Prática}\label{sec-cap18-inferencia-pratica}

\begin{tcolorbox}[enhanced jigsaw, arc=.35mm, opacitybacktitle=0.6, colframe=quarto-callout-note-color-frame, titlerule=0mm, leftrule=.75mm, left=2mm, colbacktitle=quarto-callout-note-color!10!white, breakable, toprule=.15mm, bottomtitle=1mm, opacityback=0, coltitle=black, title=\textcolor{quarto-callout-note-color}{\faInfo}\hspace{0.5em}{Capítulo 18 - Livro ``A Estatística Básica e sua prática'' (9ª ed.)}, rightrule=.15mm, bottomrule=.15mm, toptitle=1mm, colback=white]

Este capítulo apresenta os conceitos fundamentais para aplicação prática
da inferência estatística, abordando as condições necessárias para
procedimentos válidos e confiáveis.

\end{tcolorbox}

\section{Introdução}\label{introduuxe7uxe3o}

Até agora, nos foram apresentados somente dois procedimentos de
inferência estatística. Ambos dizem respeito à inferência sobre a média
μ de uma população, quando as \textbf{``condições simples''} são
verdadeiras: os dados são uma AAS, a população tem distribuição Normal,
e conhecemos o desvio-padrão σ da população.

Sob essas condições, um intervalo de confiança para a média μ é:

\[\bar{x} \pm z^* \frac{\sigma}{\sqrt{n}}\]

em que \(z^*\) é o valor crítico exigido para determinado nível de
confiança. Para testarmos uma hipótese \(H_0: \mu = \mu_0\), usamos a
estatística z de uma amostra:

\[z = \frac{\bar{x} - \mu_0}{\sigma/\sqrt{n}}\]

Chamamos esses procedimentos de \textbf{procedimentos z} porque ambos
começam com a estatística z de uma amostra e usam a distribuição Normal
padrão.

Em capítulos posteriores, modificaremos esses procedimentos para
inferência sobre uma média populacional para torná-los mais úteis na
prática. Introduziremos, também, procedimentos para intervalos de
confiança e testes voltados para a maioria das situações que encontramos
quando aprendemos a explorar dados.

\begin{tcolorbox}[enhanced jigsaw, arc=.35mm, opacitybacktitle=0.6, colframe=quarto-callout-important-color-frame, titlerule=0mm, leftrule=.75mm, left=2mm, colbacktitle=quarto-callout-important-color!10!white, breakable, toprule=.15mm, bottomtitle=1mm, opacityback=0, coltitle=black, title=\textcolor{quarto-callout-important-color}{\faExclamation}\hspace{0.5em}{Ditado Estatístico}, rightrule=.15mm, bottomrule=.15mm, toptitle=1mm, colback=white]

Existe um ditado entre os estatísticos que diz que \textbf{``teoremas
matemáticos são verdadeiros; métodos estatísticos são eficazes quando
usados com discernimento''}. O fato de a estatística z de uma amostra
ter distribuição Normal padrão quando a hipótese nula é verdadeira
corresponde a um teorema matemático. O uso eficaz de métodos
estatísticos requer mais do que o conhecimento de tais fatos.

\end{tcolorbox}

\section{18.1 Condições para Inferência na
Prática}\label{sec-condicoes-inferencia}

\begin{tcolorbox}[enhanced jigsaw, arc=.35mm, opacitybacktitle=0.6, colframe=quarto-callout-warning-color-frame, titlerule=0mm, leftrule=.75mm, left=2mm, colbacktitle=quarto-callout-warning-color!10!white, breakable, toprule=.15mm, bottomtitle=1mm, opacityback=0, coltitle=black, title=\textcolor{quarto-callout-warning-color}{\faExclamationTriangle}\hspace{0.5em}{Warning}, rightrule=.15mm, bottomrule=.15mm, toptitle=1mm, colback=white]

Qualquer intervalo de confiança ou teste de significância só é confiável
sob condições específicas. Cabe a você entender essas condições e julgar
se elas se ajustam ao seu problema.

\end{tcolorbox}

Com isso em mente, vamos examinar novamente as \textbf{``condições
simples''} para os procedimentos z para inferência sobre uma média.

\subsection{Condições Simples para Inferência sobre uma
Média}\label{condiuxe7uxf5es-simples-para-inferuxeancia-sobre-uma-muxe9dia-2}

\begin{enumerate}
\def\labelenumi{\arabic{enumi}.}
\item
  \textbf{Amostragem}: Temos uma amostra aleatória simples (AAS) da
  população de interesse. Não há não resposta ou outra dificuldade
  prática. A população é grande em comparação com o tamanho da amostra.
\item
  \textbf{Normalidade}: A variável que medimos tem uma distribuição
  exatamente Normal N(μ, σ) na população.
\item
  \textbf{Desvio-padrão conhecido}: Não conhecemos a média populacional
  μ. Mas conhecemos o desvio-padrão populacional σ.
\end{enumerate}

A última das ``condições simples'' - conhecemos o desvio-padrão σ da
população - raramente é satisfeita na prática. Os procedimentos z,
portanto, são de pouco uso prático. Felizmente, é fácil remover a
condição ``σ conhecido''. O Capítulo 20 mostra como fazê-lo.

\subsection{A Procedência dos Dados
Importa}\label{a-proceduxeancia-dos-dados-importa}

\begin{tcolorbox}[enhanced jigsaw, arc=.35mm, opacitybacktitle=0.6, colframe=quarto-callout-tip-color-frame, titlerule=0mm, leftrule=.75mm, left=2mm, colbacktitle=quarto-callout-tip-color!10!white, breakable, toprule=.15mm, bottomtitle=1mm, opacityback=0, coltitle=black, title=\textcolor{quarto-callout-tip-color}{\faLightbulb}\hspace{0.5em}{Pergunta Fundamental}, rightrule=.15mm, bottomrule=.15mm, toptitle=1mm, colback=white]

Ao planejar inferência, você deve sempre perguntar-se ``\textbf{De onde
vieram os dados?}'' e também deve responder a outra questão,
``\textbf{Qual é a forma da distribuição populacional?}''

\end{tcolorbox}

\subsubsection{EXEMPLO 18.1: O psicólogo e o
sociólogo}\label{exemplo-18.1-o-psicuxf3logo-e-o-sociuxf3logo}

\textbf{Psicóloga}: Uma psicóloga está interessada em saber como nossa
percepção visual pode ser enganada por ilusões óticas. Seus sujeitos são
alunos da disciplina de Psicologia 101 de sua universidade. A maioria
dos psicólogos concordaria que é seguro tratar os alunos como uma AAS de
todas as pessoas com visão normal. Não há nada incomum em ser estudante
que mude a percepção visual.

\textbf{Sociólogo}: Um sociólogo da mesma universidade usa os alunos da
disciplina de Sociologia 101 para examinar as atitudes com relação a
pessoas pobres e programas de combate à pobreza. Os alunos, como um
grupo, são mais jovens do que a população adulta como um todo. Mesmo
entre pessoas jovens, os alunos como um grupo provêm de lares mais
educados e mais prósperos. O sociólogo não pode, razoavelmente, agir
como se esses alunos fossem uma amostra aleatória de qualquer população
de interesse.

\begin{tcolorbox}[enhanced jigsaw, arc=.35mm, opacitybacktitle=0.6, colframe=quarto-callout-note-color-frame, titlerule=0mm, leftrule=.75mm, left=2mm, colbacktitle=quarto-callout-note-color!10!white, breakable, toprule=.15mm, bottomtitle=1mm, opacityback=0, coltitle=black, title=\textcolor{quarto-callout-note-color}{\faInfo}\hspace{0.5em}{De onde vieram os dados?}, rightrule=.15mm, bottomrule=.15mm, toptitle=1mm, colback=white]

O requisito mais importante de qualquer procedimento de inferência é que
os dados provenham de um processo ao qual se apliquem as leis de
probabilidade. A inferência é mais confiável quando os dados resultam de
uma \textbf{amostra aleatória} ou de um \textbf{experimento comparativo
aleatorizado}.

\end{tcolorbox}

\subsection{A Procedência dos Dados
Importa}\label{a-proceduxeancia-dos-dados-importa-1}

\begin{tcolorbox}[enhanced jigsaw, arc=.35mm, opacitybacktitle=0.6, colframe=quarto-callout-warning-color-frame, titlerule=0mm, leftrule=.75mm, left=2mm, colbacktitle=quarto-callout-warning-color!10!white, breakable, toprule=.15mm, bottomtitle=1mm, opacityback=0, coltitle=black, title=\textcolor{quarto-callout-warning-color}{\faExclamationTriangle}\hspace{0.5em}{Princípio Fundamental}, rightrule=.15mm, bottomrule=.15mm, toptitle=1mm, colback=white]

\textbf{Ao usar inferência estatística, você age como se seus dados
fossem uma amostra aleatória ou provenientes de um experimento
comparativo aleatorizado.}

Se seus dados não provêm de uma amostra aleatória ou de um experimento
comparativo aleatorizado, suas conclusões podem ser questionadas.

\end{tcolorbox}

\subsubsection{EXEMPLO 18.2: É realmente uma
AAS?}\label{exemplo-18.2-uxe9-realmente-uma-aas}

\textbf{Pesquisa NHANES}: A pesquisa NHANES, que produziu os dados de
IMC para o Exemplo 16.1, usou um planejamento amostral complexo de
estágios múltiplos, de modo que é bastante simplista considerar os dados
de IMC como provenientes de uma AAS da população de homens jovens.
Embora o efeito geral da amostra NHANES seja próximo de uma AAS,
estatísticos profissionais usariam procedimentos de inferência mais
complexos para melhor adequação ao planejamento mais complexo da
amostra.

\textbf{Estudo da gorjeta}: Os 20 clientes no estudo da gorjeta
apresentado no Exemplo 16.3 foram escolhidos, entre aqueles que comiam
em um restaurante particular, para receber um de vários tratamentos em
comparação com um experimento comparativo aleatorizado.

\subsection{Cuidados Importantes}\label{cuidados-importantes}

\begin{tcolorbox}[enhanced jigsaw, arc=.35mm, opacitybacktitle=0.6, colframe=quarto-callout-caution-color-frame, titlerule=0mm, leftrule=.75mm, left=2mm, colbacktitle=quarto-callout-caution-color!10!white, breakable, toprule=.15mm, bottomtitle=1mm, opacityback=0, coltitle=black, title=\textcolor{quarto-callout-caution-color}{\faFire}\hspace{0.5em}{Problemas Práticos}, rightrule=.15mm, bottomrule=.15mm, toptitle=1mm, colback=white]

\begin{itemize}
\item
  \textbf{Problemas práticos}, como não resposta em amostras ou
  desistências em um experimento, podem prejudicar a inferência, mesmo
  em um estudo bem planejado.
\item
  \textbf{Métodos diferentes} são necessários para planejamentos
  diferentes. Os procedimentos z não são corretos para planejamentos
  amostrais aleatórios mais complexos do que uma AAS.
\item
  \textbf{Não há remédio} para falhas fundamentais, como resposta
  voluntária ou experimentos não controlados.
\end{itemize}

\end{tcolorbox}

\subsection{Qual é a Forma da Distribuição
Populacional?}\label{qual-uxe9-a-forma-da-distribuiuxe7uxe3o-populacional}

A maioria dos procedimentos de inferência estatística exige algumas
condições sobre a forma da distribuição populacional. Muitos dos métodos
mais básicos de inferência são planejados para populações Normais.

\begin{tcolorbox}[enhanced jigsaw, arc=.35mm, opacitybacktitle=0.6, colframe=quarto-callout-tip-color-frame, titlerule=0mm, leftrule=.75mm, left=2mm, colbacktitle=quarto-callout-tip-color!10!white, breakable, toprule=.15mm, bottomtitle=1mm, opacityback=0, coltitle=black, title=\textcolor{quarto-callout-tip-color}{\faLightbulb}\hspace{0.5em}{Teorema Limite Central na Prática}, rightrule=.15mm, bottomrule=.15mm, toptitle=1mm, colback=white]

Os procedimentos z e muitos outros procedimentos planejados para
distribuições Normais se baseiam na \textbf{Normalidade da distribuição
da média amostral} \(\bar{x}\), e não na Normalidade das observações
individuais.

O teorema limite central nos diz que: - A distribuição amostral de
\(\bar{x}\) é mais Normal do que as observações individuais - A
distribuição amostral de \(\bar{x}\) se torna mais Normal à medida que o
tamanho da amostra aumenta

\end{tcolorbox}

\subsection{Valores Atípicos e
Robustez}\label{valores-atuxedpicos-e-robustez}

\begin{tcolorbox}[enhanced jigsaw, arc=.35mm, opacitybacktitle=0.6, colframe=quarto-callout-warning-color-frame, titlerule=0mm, leftrule=.75mm, left=2mm, colbacktitle=quarto-callout-warning-color!10!white, breakable, toprule=.15mm, bottomtitle=1mm, opacityback=0, coltitle=black, title=\textcolor{quarto-callout-warning-color}{\faExclamationTriangle}\hspace{0.5em}{Exceção Importante}, rightrule=.15mm, bottomrule=.15mm, toptitle=1mm, colback=white]

Há uma importante exceção ao princípio de que a forma da população seja
menos crítica do que a procedência dos dados. \textbf{Valores atípicos
podem distorcer os resultados da inferência.}

\end{tcolorbox}

Qualquer procedimento de inferência que se baseie em estatísticas
amostrais, como a média amostral \(\bar{x}\), que não são resistentes a
valores atípicos, pode ser fortemente influenciado por algumas poucas
observações extremas.

\subsection{Diretrizes Práticas}\label{diretrizes-pruxe1ticas}

\begin{enumerate}
\def\labelenumi{\arabic{enumi}.}
\tightlist
\item
  \textbf{Explore seus dados} antes de fazer inferência
\item
  \textbf{Faça um diagrama} de ramo e folhas ou um histograma de seus
  dados
\item
  \textbf{Verifique} se a forma é razoavelmente Normal
\item
  \textbf{Procure sempre por valores atípicos} e tente corrigi-los ou
  justificar sua remoção
\item
  \textbf{Considere métodos alternativos} que não exijam a Normalidade
  quando apropriado
\end{enumerate}

\section{18.2 Cuidados com os Intervalos de
Confiança}\label{sec-cuidados-intervalos}

\subsection{A Margem de Erro Não Cobre Todos os
Erros}\label{a-margem-de-erro-nuxe3o-cobre-todos-os-erros}

\begin{tcolorbox}[enhanced jigsaw, arc=.35mm, opacitybacktitle=0.6, colframe=quarto-callout-important-color-frame, titlerule=0mm, leftrule=.75mm, left=2mm, colbacktitle=quarto-callout-important-color!10!white, breakable, toprule=.15mm, bottomtitle=1mm, opacityback=0, coltitle=black, title=\textcolor{quarto-callout-important-color}{\faExclamation}\hspace{0.5em}{Limitação da Margem de Erro}, rightrule=.15mm, bottomrule=.15mm, toptitle=1mm, colback=white]

A margem de erro em um intervalo de confiança cobre apenas \textbf{erros
de amostragem aleatória}. Dificuldades práticas, como subcobertura e não
resposta, são, muitas vezes, mais sérias do que os erros de amostragem
aleatória. \textbf{A margem de erro não leva em consideração essas
dificuldades.}

\end{tcolorbox}

A precaução mais importante acerca de intervalos de confiança, em geral,
é uma consequência do uso de uma distribuição amostral. Uma distribuição
amostral revela como uma estatística, como \(\bar{x}\), varia em
amostras repetidas. Essa variação gera erro amostral aleatório, porque a
estatística erra o verdadeiro parâmetro por uma quantidade aleatória.

\subsection{Exemplos de Aplicação}\label{exemplos-de-aplicauxe7uxe3o}

\subsubsection{EXEMPLO 18.4: Qual é o seu
peso?}\label{exemplo-18.4-qual-uxe9-o-seu-peso}

Uma pesquisa Gallup de 2019 pediu a uma amostra aleatória nacional de
507 homens adultos que eles fornecessem seus pesos atuais. O peso médio
na amostra foi \(\bar{x} = 196\). Vamos considerar esses dados como uma
AAS proveniente de uma população Normalmente distribuída, com
desvio-padrão \(\sigma = 35\).

\textbf{Problema}: Você confia no intervalo de confiança calculado como
sendo de 95\% um intervalo de confiança para o peso médio de todos os
homens adultos americanos?

\textbf{Resposta}: Provavelmente não, porque as pessoas frequentemente
mentem sobre seu peso, especialmente em pesquisas por telefone.

\section{18.3 Cuidados com os Testes de
Significância}\label{sec-cuidados-testes}

Testes de significância são amplamente utilizados na maioria das áreas
do trabalho estatístico. Novos produtos farmacêuticos exigem evidência
significante de eficácia e segurança. Tribunais inquirem sobre a
significância estatística nas audiências de casos de discriminação em
ações de classe.

\subsection{Quão Pequeno Deve Ser P para Ser
Convincente?}\label{quuxe3o-pequeno-deve-ser-p-para-ser-convincente}

O propósito de um teste de significância é descrever o grau de evidência
contra a hipótese nula fornecida pela amostra. O valor P faz isso. Mas
quão pequeno deve ser um valor P para ser uma evidência convincente
contra a hipótese nula?

Isso depende principalmente de duas circunstâncias:

\begin{tcolorbox}[enhanced jigsaw, arc=.35mm, opacitybacktitle=0.6, colframe=quarto-callout-note-color-frame, titlerule=0mm, leftrule=.75mm, left=2mm, colbacktitle=quarto-callout-note-color!10!white, breakable, toprule=.15mm, bottomtitle=1mm, opacityback=0, coltitle=black, title=\textcolor{quarto-callout-note-color}{\faInfo}\hspace{0.5em}{Critérios para Avaliação do Valor P}, rightrule=.15mm, bottomrule=.15mm, toptitle=1mm, colback=white]

\begin{enumerate}
\def\labelenumi{\arabic{enumi}.}
\item
  \textbf{Quão plausível é H₀?} Se H₀ for uma suposição na qual as
  pessoas a serem convencidas acreditam há anos, será necessária uma
  forte evidência (P pequeno) para persuadi-las.
\item
  \textbf{Quais são as consequências de rejeitar H₀?} Se a rejeição de
  H₀ em favor de Hₐ significa fazer uma troca dispendiosa de um tipo de
  embalagem de produto por outro, você precisa de uma forte evidência de
  que a nova embalagem impulsionará as vendas.
\end{enumerate}

\end{tcolorbox}

\subsection{Significância Depende da Hipótese
Alternativa}\label{significuxe2ncia-depende-da-hipuxf3tese-alternativa}

Você deve ter notado que o valor P para o teste \textbf{unilateral} é
\textbf{metade} do valor P para o teste \textbf{bilateral} da mesma
hipótese nula e com base nos mesmos dados.

\begin{itemize}
\tightlist
\item
  O valor P bilateral combina duas áreas iguais, uma em cada cauda da
  curva Normal
\item
  O valor P unilateral é apenas uma dessas áreas, na direção
  especificada pela hipótese alternativa
\end{itemize}

\subsection{Significância Depende do Tamanho
Amostral}\label{significuxe2ncia-depende-do-tamanho-amostral}

\begin{tcolorbox}[enhanced jigsaw, arc=.35mm, opacitybacktitle=0.6, colframe=quarto-callout-warning-color-frame, titlerule=0mm, leftrule=.75mm, left=2mm, colbacktitle=quarto-callout-warning-color!10!white, breakable, toprule=.15mm, bottomtitle=1mm, opacityback=0, coltitle=black, title=\textcolor{quarto-callout-warning-color}{\faExclamationTriangle}\hspace{0.5em}{Relação entre Significância e Tamanho da Amostra}, rightrule=.15mm, bottomrule=.15mm, toptitle=1mm, colback=white]

\textbf{Como grandes amostras aleatórias têm pequena variação do acaso,
os efeitos populacionais muito pequenos podem ser altamente
significantes se a amostra for grande.}

\textbf{Como amostras aleatórias pequenas têm muita variação do acaso,
mesmo efeitos populacionais grandes podem deixar de ser significantes se
a amostra for pequena.}

\end{tcolorbox}

A estatística de teste z é:

\[z = \frac{\bar{x} - \mu_0}{\sigma/\sqrt{n}}\]

\begin{itemize}
\tightlist
\item
  O \textbf{numerador} mede quanto a média amostral se afasta da média
  da hipótese μ₀
\item
  O \textbf{denominador} é o desvio-padrão de \(\bar{x}\). Há menos
  variação quando o número de observações n é grande
\item
  Assim, z se torna maior (mais significante) quando o efeito estimado
  \(\bar{x} - \mu_0\) se torna maior ou quando o número de observações n
  aumenta
\end{itemize}

\begin{tcolorbox}[enhanced jigsaw, arc=.35mm, opacitybacktitle=0.6, colframe=quarto-callout-important-color-frame, titlerule=0mm, leftrule=.75mm, left=2mm, colbacktitle=quarto-callout-important-color!10!white, breakable, toprule=.15mm, bottomtitle=1mm, opacityback=0, coltitle=black, title=\textcolor{quarto-callout-important-color}{\faExclamation}\hspace{0.5em}{Significância ≠ Importância Prática}, rightrule=.15mm, bottomrule=.15mm, toptitle=1mm, colback=white]

\textbf{Significância estatística não nos diz se um efeito é grande o
bastante para ser importante. Isto é, significância estatística não é a
mesma coisa que significância prática.}

``Significância estatística'' quer dizer que ``a amostra exibiu um
efeito maior do que em geral ocorreria apenas por acaso''.

\end{tcolorbox}

\subsubsection{EXEMPLO 18.3: É significante. Ou não. E
daí?}\label{exemplo-18.3-uxe9-significante.-ou-nuxe3o.-e-dauxed}

Estamos testando a hipótese de nenhuma correlação entre duas variáveis.
Com mil observações, uma correlação observada de apenas r = 0,08 é uma
evidência significante no nível 1\% de que a correlação na população não
é zero e sim positiva.

O valor P pequeno não significa que haja uma forte associação, apenas
que há forte evidência de alguma associação. Seria possível, então,
concluir que, para fins práticos, podemos ignorar a associação entre
essas variáveis, mesmo estando confiantes (no nível 1\%) de que a
correlação é positiva.

\subsection{Cuidado com as Análises
Múltiplas}\label{cuidado-com-as-anuxe1lises-muxfaltiplas}

\begin{tcolorbox}[enhanced jigsaw, arc=.35mm, opacitybacktitle=0.6, colframe=quarto-callout-caution-color-frame, titlerule=0mm, leftrule=.75mm, left=2mm, colbacktitle=quarto-callout-caution-color!10!white, breakable, toprule=.15mm, bottomtitle=1mm, opacityback=0, coltitle=black, title=\textcolor{quarto-callout-caution-color}{\faFire}\hspace{0.5em}{O Problema das Múltiplas Comparações}, rightrule=.15mm, bottomrule=.15mm, toptitle=1mm, colback=white]

Significância estatística deve indicar que você encontrou um efeito que
estava procurando. O raciocínio que fundamenta a significância
estatística funciona bem se você decide qual efeito está procurando,
planeja um estudo para procurá-lo e usa um teste de significância para
ponderar a evidência obtida.

\end{tcolorbox}

Suponha que as 20 hipóteses nulas (nenhuma associação) para 20 testes de
significância sejam todas verdadeiras. Então, cada teste tem uma chance
de 5\% de ser significante no nível 5\%. Como 5\% são 1/20, esperamos
que cerca de 1, entre 20 testes, forneça, apenas devido ao acaso, um
resultado significante.

\subsubsection{EXEMPLO 18.4: Telefones celulares e câncer no
cérebro}\label{exemplo-18.4-telefones-celulares-e-cuxe2ncer-no-cuxe9rebro}

Um estudo hospitalar, que comparou pacientes com câncer no cérebro e um
grupo similar sem câncer no cérebro, não encontrou associação
estatisticamente significante entre o uso de telefones celulares e um
tipo de câncer no cérebro conhecido como glioma. Porém, quando 20 tipos
de glioma foram estudados separadamente, foi encontrada uma associação
entre o uso de celular e uma forma rara da doença.

\textbf{Conduzir um teste e alcançar o nível de significância de 5\% é
uma evidência razoavelmente boa de que você encontrou algo. Conduzir 20
testes e alcançar esse nível apenas uma vez não corresponde a uma boa
evidência.}

\subsubsection{EXEMPLO 18.5: Viés de
publicação}\label{exemplo-18.5-viuxe9s-de-publicauxe7uxe3o}

Um exemplo sutil de análises múltiplas é o \textbf{viés de publicação}.
Suponha que 20 pesquisadores estejam, independentemente, estudando a
eficácia de uma nova terapia para o tratamento de uma doença. Para
publicar suas descobertas, os pesquisadores devem demonstrar que a nova
terapia é eficaz no nível de significância de 0,05.

Um dos pesquisadores obtém resultados estatisticamente significantes,
mas os outros 19 não. O único pesquisador que obteve resultados
estatisticamente significantes publica suas descobertas. Nada ficamos
sabendo sobre os 19 pesquisadores que deixaram de encontrar
significância estatística.

\section{18.4 Planejamento de Estudos: Tamanho Amostral para Intervalos
de Confiança}\label{sec-tamanho-amostral}

Um usuário experiente de estatística nunca planeja uma amostra ou um
experimento sem, ao mesmo tempo, planejar a inferência. O número de
observações é uma parte crítica do planejamento de um estudo.

\subsection{Fórmula para o Tamanho
Amostral}\label{fuxf3rmula-para-o-tamanho-amostral}

A margem de erro do intervalo de confiança z para a média de uma
população Normalmente distribuída é:

\[m = z^* \frac{\sigma}{\sqrt{n}}\]

Para obter uma margem de erro desejada, m, substitua o valor de z* para
seu nível de confiança desejado e resolva em relação ao tamanho amostral
n:

\begin{tcolorbox}[enhanced jigsaw, arc=.35mm, opacitybacktitle=0.6, colframe=quarto-callout-tip-color-frame, titlerule=0mm, leftrule=.75mm, left=2mm, colbacktitle=quarto-callout-tip-color!10!white, breakable, toprule=.15mm, bottomtitle=1mm, opacityback=0, coltitle=black, title=\textcolor{quarto-callout-tip-color}{\faLightbulb}\hspace{0.5em}{Tamanho da Amostra para uma Margem de Erro Desejada}, rightrule=.15mm, bottomrule=.15mm, toptitle=1mm, colback=white]

Para estimar a média de uma população Normal usando um intervalo de
confiança z com dada margem de erro m e um nível de confiança
especificado, o tamanho da amostra n deve ser:

\[n = \left(\frac{z^* \sigma}{m}\right)^2\]

em que z* é o valor crítico para o nível de confiança desejado.
\textbf{Sempre arredonde n para o próximo inteiro acima} quando usar
essa fórmula.

\end{tcolorbox}

\subsubsection{EXEMPLO 18.6: Quantas
observações?}\label{exemplo-18.6-quantas-observauxe7uxf5es}

No Exemplo 16.3, psicólogos registraram o tamanho das gorjetas de 20
clientes em um restaurante, quando se escrevia, em sua conta, uma
mensagem anunciando tempo bom para o dia seguinte. Sabemos que o
desvio-padrão populacional é σ = 2. Desejamos estimar a gorjeta
percentual média μ para clientes desse restaurante que recebem essa
mensagem em suas contas, dentro de ±0,5, com 90\% de confiança. Quantos
clientes devem ser observados?

\textbf{Solução}: A margem de erro desejada é m = 0,5. Para 90\% de
confiança, a Tabela C fornece z* = 1,645. Portanto:

\[n = \left(\frac{1,645 \times 2}{0,5}\right)^2 = (6,58)^2 = 43,3\]

Como 43 clientes dão uma margem de erro ligeiramente maior do que a
desejada, e 44 clientes, uma margem de erro ligeiramente menor, devemos
observar \textbf{44 clientes}.

\section{18.5 Planejamento de Estudos: O Poder de um Teste Estatístico
de Significância*}\label{sec-poder-teste}

\begin{tcolorbox}[enhanced jigsaw, arc=.35mm, opacitybacktitle=0.6, colframe=quarto-callout-note-color-frame, titlerule=0mm, leftrule=.75mm, left=2mm, colbacktitle=quarto-callout-note-color!10!white, breakable, toprule=.15mm, bottomtitle=1mm, opacityback=0, coltitle=black, title=\textcolor{quarto-callout-note-color}{\faInfo}\hspace{0.5em}{Note}, rightrule=.15mm, bottomrule=.15mm, toptitle=1mm, colback=white]

*Cálculos de poder são importantes no planejamento de estudos, mas este
material mais avançado não é necessário para a leitura do restante do
livro.

\end{tcolorbox}

Qual o tamanho da amostra que devemos extrair quando planejamos realizar
um teste de significância? Sabemos que, se nossa amostra for muito
pequena, mesmo grandes efeitos na população, em geral, deixarão de dar
resultados estatisticamente significantes.

\subsection{Questões para Decidir o Tamanho
Amostral}\label{questuxf5es-para-decidir-o-tamanho-amostral}

\begin{enumerate}
\def\labelenumi{\arabic{enumi}.}
\item
  \textbf{Nível de significância}: Quanta proteção desejamos contra a
  obtenção de um resultado significante a partir de nossa amostra
  quando, na realidade, não há qualquer efeito na população?
\item
  \textbf{Tamanho do efeito}: Qual o tamanho de um efeito na população
  para ser importante na prática?
\item
  \textbf{Poder}: Quão confiantes queremos estar de que nosso estudo
  detectará um efeito do tamanho que consideramos importante?
\end{enumerate}

\begin{tcolorbox}[enhanced jigsaw, arc=.35mm, opacitybacktitle=0.6, colframe=quarto-callout-tip-color-frame, titlerule=0mm, leftrule=.75mm, left=2mm, colbacktitle=quarto-callout-tip-color!10!white, breakable, toprule=.15mm, bottomtitle=1mm, opacityback=0, coltitle=black, title=\textcolor{quarto-callout-tip-color}{\faLightbulb}\hspace{0.5em}{Definições Importantes}, rightrule=.15mm, bottomrule=.15mm, toptitle=1mm, colback=white]

\textbf{Tamanho do efeito}: A magnitude do efeito na população.

\textbf{Poder}: O poder de um teste contra uma alternativa específica é
a probabilidade de o teste rejeitar H₀ em determinado nível de
significância α, quando o valor alternativo especificado do parâmetro é
verdadeiro.

\end{tcolorbox}

\subsection{Erros Tipo I e Tipo II}\label{erros-tipo-i-e-tipo-ii}

\subsubsection{EXEMPLO 18.7: Adoçante de refrigerantes: planejamento de
um
estudo}\label{exemplo-18.7-adouxe7ante-de-refrigerantes-planejamento-de-um-estudo}

Vamos ilustrar respostas típicas às questões que acabamos de colocar,
olhando novamente o exemplo do teste de um novo refrigerante em relação
à perda de doçura na armazenagem. Dez provadores treinados classificam a
doçura em uma escala de 10 pontos, antes e depois do armazenamento.

Para verificar se o teste do sabor fornece razão para pensar que o
refrigerante realmente perde doçura, testaremos:

\[H_0: \mu = 0\] \[H_a: \mu > 0\]

\textbf{Decisões do estudo}:

\begin{enumerate}
\def\labelenumi{\arabic{enumi}.}
\item
  \textbf{Nível de significância}: A exigência de significância no nível
  de 5\% é proteção suficiente contra a afirmativa de que há uma perda
  de doçura quando, de fato, não há qualquer mudança.
\item
  \textbf{Tamanho do efeito}: Uma perda média de doçura de 0,8 ponto na
  escala de 10 pontos será notada pelos consumidores e, assim, é
  importante na prática.
\item
  \textbf{Poder}: Desejamos estar 90\% confiantes de que nosso teste
  detectará uma perda média de 0,8 ponto na população de todos os
  provadores.
\end{enumerate}

\begin{tcolorbox}[enhanced jigsaw, arc=.35mm, opacitybacktitle=0.6, colframe=quarto-callout-important-color-frame, titlerule=0mm, leftrule=.75mm, left=2mm, colbacktitle=quarto-callout-important-color!10!white, breakable, toprule=.15mm, bottomtitle=1mm, opacityback=0, coltitle=black, title=\textcolor{quarto-callout-important-color}{\faExclamation}\hspace{0.5em}{Erros Tipo I e Tipo II}, rightrule=.15mm, bottomrule=.15mm, toptitle=1mm, colback=white]

\textbf{Erro Tipo I}: Se rejeitamos H₀ quando, de fato, H₀ é verdadeira,
esse é um erro Tipo I.

\textbf{Erro Tipo II}: Se deixamos de rejeitar H₀ quando, de fato, Hₐ é
verdadeira, esse é um erro Tipo II.

\begin{itemize}
\tightlist
\item
  O nível de significância α de qualquer teste de nível fixo é a
  probabilidade de um erro Tipo I
\item
  O poder de um teste contra qualquer alternativa é a probabilidade de
  rejeitarmos corretamente a hipótese nula para aquela alternativa. Ele
  pode ser calculado como 1 menos a probabilidade de um erro Tipo II
  para aquela alternativa
\end{itemize}

Uma figura ilustra e resume melhor esses dois conceitos.

\begin{figure}[H]

{\centering \pandocbounded{\includegraphics[keepaspectratio]{fig/Erro-tipo-I-e-II.png}}

}

\caption{Os dois tipos de Erro em Testes de Significância da Hipótese
Nula (Moore, 2023, cap. 18, p.~335)}

\end{figure}%

Nas linhas o que a amostragem, com seu erro amostral, nos mostra.

Nas colunas a vida da População como ela é.

\end{tcolorbox}

\subsection{Fatores que Influenciam o Tamanho da
Amostra}\label{fatores-que-influenciam-o-tamanho-da-amostra}

\begin{tcolorbox}[enhanced jigsaw, arc=.35mm, opacitybacktitle=0.6, colframe=quarto-callout-tip-color-frame, titlerule=0mm, leftrule=.75mm, left=2mm, colbacktitle=quarto-callout-tip-color!10!white, breakable, toprule=.15mm, bottomtitle=1mm, opacityback=0, coltitle=black, title=\textcolor{quarto-callout-tip-color}{\faLightbulb}\hspace{0.5em}{Influências sobre ``Qual o tamanho da amostra de que preciso?''}, rightrule=.15mm, bottomrule=.15mm, toptitle=1mm, colback=white]

\begin{itemize}
\item
  Se você insiste em um \textbf{nível de significância menor} (como 1\%
  em vez de 5\%), você precisará de uma \textbf{amostra maior}
\item
  Se você insiste em \textbf{poder mais alto} (como 99\% em vez de
  90\%), você precisará de uma \textbf{amostra maior}
\item
  Para qualquer nível de significância e poder desejados, uma
  \textbf{alternativa bilateral} requer uma \textbf{amostra maior} do
  que uma alternativa unilateral
\item
  Para qualquer nível de significância e poder desejados, a
  \textbf{detecção de um pequeno efeito} requer uma \textbf{amostra
  maior} do que a detecção de um grande efeito
\end{itemize}

\end{tcolorbox}

\section{Exercícios de Aplicação}\label{sec-exercicios}

\subsection{APLIQUE SEU CONHECIMENTO}\label{aplique-seu-conhecimento-3}

\textbf{18.1 Classifique esse produto}: Um site de compras online pede
que os clientes classifiquem os produtos que compram em uma escala de 1
(não gosto fortemente) a 5 (gosto fortemente). O convite para a
classificação de uma compra recente é enviado aos clientes uma semana
depois da compra, e os clientes podem escolher ignorar o convite. Qual
das seguintes é a razão mais importante para que um intervalo de
confiança, com base nos dados de tais classificações, seja de pouca
utilidade para a classificação média de todos os clientes que compraram
um produto particular?

\begin{enumerate}
\def\labelenumi{(\alph{enumi})}
\tightlist
\item
  Para alguns produtos, o número de clientes que os compram é pequeno,
  de modo que a margem de erro será grande.
\item
  Muitos dos clientes podem não ler seus e-mails ou podem ter um filtro
  de spam que identifica incorretamente o e-mail que pede a
  classificação como spam.
\item
  \textbf{Os clientes que fornecem classificações não podem ser
  considerados uma amostra aleatória da população de todos os clientes
  que compram um produto particular.}
\end{enumerate}

\textbf{Resposta}: (c) - A resposta voluntária é o maior problema, pois
cria viés de seleção.

\subsection{Problemas Adicionais}\label{problemas-adicionais}

\textbf{18.2 Ultrapassando o sinal vermelho}: Uma pesquisa com
motoristas habilitados fez perguntas acerca da ultrapassagem do sinal
vermelho. Uma das perguntas era ``De cada 10 motoristas que ultrapassam
o sinal vermelho, cerca de quantos você acha que serão flagrados?'' O
resultado médio para 880 respondentes foi \(\bar{x} = 1,92\) e o
desvio-padrão s = 1,83.

\begin{enumerate}
\def\labelenumi{(\alph{enumi})}
\tightlist
\item
  Forneça um intervalo de confiança de 95\% para a opinião média da
  população de todos os motoristas habilitados.
\item
  A distribuição das respostas é assimétrica à direita, em vez de
  Normal. Isso não afetará intensamente o intervalo de confiança z para
  essa amostra. Por que não?
\item
  Os 880 respondentes são uma AAS das ligações completadas entre 45.956
  ligações para telefones residenciais selecionados aleatoriamente no
  catálogo telefônico. Apenas 5.029 das chamadas foram completadas. Essa
  informação fornece duas razões para suspeitar que a amostra, talvez,
  não represente todos os motoristas habilitados. Quais são essas
  razões?
\end{enumerate}

\section{Resumo do Capítulo}\label{sec-resumo}

\begin{tcolorbox}[enhanced jigsaw, arc=.35mm, opacitybacktitle=0.6, colframe=quarto-callout-note-color-frame, titlerule=0mm, leftrule=.75mm, left=2mm, colbacktitle=quarto-callout-note-color!10!white, breakable, toprule=.15mm, bottomtitle=1mm, opacityback=0, coltitle=black, title=\textcolor{quarto-callout-note-color}{\faInfo}\hspace{0.5em}{Pontos Principais}, rightrule=.15mm, bottomrule=.15mm, toptitle=1mm, colback=white]

\begin{itemize}
\item
  \textbf{Condições específicas}: Um intervalo de confiança ou teste
  específico são corretos apenas sob condições específicas. As condições
  mais importantes são relativas ao método usado para a produção dos
  dados.
\item
  \textbf{Procedência dos dados}: Sempre que você usar inferência
  estatística, estará agindo como se seus dados fossem uma amostra
  aleatória ou fossem provenientes de um experimento comparativo
  aleatorizado.
\item
  \textbf{Análise exploratória}: Antes da inferência, sempre faça uma
  análise dos dados para detectar valores atípicos ou outros problemas
  que tornariam a inferência não confiável.
\item
  \textbf{Limitações da margem de erro}: A margem de erro em um
  intervalo de confiança considera apenas a variação casual devida à
  amostragem aleatória. Na prática, erros causados pela não resposta ou
  subcobertura são frequentemente mais sérios.
\item
  \textbf{Significância vs.~importância}: Não há uma regra universal que
  determine quão pequeno deva ser um valor P em um teste de
  significância para considerá-lo evidência convincente contra a
  hipótese nula.
\item
  \textbf{Efeito do tamanho amostral}: Efeitos muito pequenos podem ser
  altamente significantes (valor P pequeno) quando um teste se baseia em
  uma amostra grande. Sempre considere se o tamanho do efeito é
  importante na prática.
\item
  \textbf{Planejamento de estudos}: Quando planejar um estudo
  estatístico, planeje também a inferência. Em particular, determine
  qual o tamanho amostral de que você necessita para uma inferência
  bem-sucedida.
\end{itemize}

\end{tcolorbox}

\subsection{Fórmulas Importantes}\label{fuxf3rmulas-importantes}

\textbf{Tamanho amostral para margem de erro desejada}:
\[n = \left(\frac{z^* \sigma}{m}\right)^2\]

\textbf{Poder de um teste}: Probabilidade de rejeitar corretamente H₀
quando Hₐ é verdadeira.

\textbf{Erros Tipo I e II}: - Tipo I: Rejeitar H₀ quando H₀ é verdadeira
(probabilidade = α) - Tipo II: Não rejeitar H₀ quando Hₐ é verdadeira
(probabilidade = β) - Poder = 1 - β

\begin{tcolorbox}[enhanced jigsaw, arc=.35mm, opacitybacktitle=0.6, colframe=quarto-callout-important-color-frame, titlerule=0mm, leftrule=.75mm, left=2mm, colbacktitle=quarto-callout-important-color!10!white, breakable, toprule=.15mm, bottomtitle=1mm, opacityback=0, coltitle=black, title=\textcolor{quarto-callout-important-color}{\faExclamation}\hspace{0.5em}{Lição Principal}, rightrule=.15mm, bottomrule=.15mm, toptitle=1mm, colback=white]

A \textbf{qualidade dos dados} é mais importante do que a
\textbf{sofisticação dos métodos}. Métodos estatísticos são ferramentas
poderosas, mas só são eficazes quando aplicados a dados de boa
qualidade, coletados de forma apropriada.

\end{tcolorbox}

\bookmarksetup{startatroot}

\chapter{Conclusão}\label{sec-conclusao}

\begin{tcolorbox}[enhanced jigsaw, arc=.35mm, opacitybacktitle=0.6, colframe=quarto-callout-tip-color-frame, titlerule=0mm, leftrule=.75mm, left=2mm, colbacktitle=quarto-callout-tip-color!10!white, breakable, toprule=.15mm, bottomtitle=1mm, opacityback=0, coltitle=black, title=\textcolor{quarto-callout-tip-color}{\faLightbulb}\hspace{0.5em}{O poder dos dados}, rightrule=.15mm, bottomrule=.15mm, toptitle=1mm, colback=white]

Iniciar AED com a análise de uma variável de cada vez.

Capturar padrões por meio de gráficos adequados para cada tipo de
variável.

Resumir os dados por meio de um único número; que pode ser uma medida de
tendência central (média, mediana ou moda) ou uma medida da
variabilidade presente no conjunto de dados daquela variável (Amplitude,
desvio, variância, desvio padrão; quartis e AIQ - Amplitude
Interquartil).

Somente depois dessa análise univariada que se deve proceder uma análise
bivariada. Novamente por meio de gráficos e resumos numéricos adequados.

E somente depois de uma AED - Análise Exploratória de Dados, primeiro
descritiva (AED) e depois inferencial (AEI), que se busca modelar os
dados, ou seja, propor, testar e escolher modelos que melhor se ajustem
aos dados empíricos colhidos.

\end{tcolorbox}

\bookmarksetup{startatroot}

\chapter*{Apêndice A - EMENTA}\label{Apend-A}
\addcontentsline{toc}{chapter}{Apêndice A - EMENTA}

\markboth{Apêndice A - EMENTA}{Apêndice A - EMENTA}

\textbf{1.} Ciência de Dados, Estatística e Direito: noções
introdutórias e contexto atual {[}\emph{conceito de Ciclo} da Ciência e
de \ldots{]}.

\textbf{2.} Estatística Básica e Direito: potencial do uso correto
{[}adequado{]} dos \emph{números} para subsidiar análises
\emph{multidimensionais} {[}ℝ\textsuperscript{n}→ℝ\textsuperscript{m}{]}
de questões jurídicas complexas {[}\emph{alea} e sua modelagem{]}.

\textbf{3.} \textbf{Diálogo} entre {[}DS\textsubscript{\^{}}{]}
Estatística,{[}::{]} Políticas Públicas e {[}\&{]} Direito
{[}\emph{conceito de Ciclo} \emph{de} Políticas Públicas e \emph{de}
Direito{]}.

\textbf{4.} \emph{Incerteza} e sua \emph{mensuração}: modelagem
informacional {[}fenômenos determinísticos e estocásticos{]}.

\textbf{5.} Formulação de hipóteses, partição de um espaço amostral e
\emph{confiança estatística} em análises de Políticas Públicas
{[}formulação de problemas e de ℋipóteses de pesquisa como um conjunto
de \emph{proposições teóricas ortogonais} (independentes) empiricamente
\emph{testáveis}, \emph{falseáveis}, mutuamente \emph{exclusivas},
\emph{exaustivas} e \emph{concorrentes}: uma \textbf{\emph{partição}} de
um espaço amostral (\emph{Ω}){]}.

\textbf{6.} Tipos de \emph{erros} e \emph{vieses} numa pesquisa empírica
e o auxílio da Estatística {[}desenho da pesquisa, coleta
\textbf{\emph{válida}} e \textbf{\emph{fidedigna}} de \emph{dados},
organização, transformação, criação, apresentação de tabelas e gráficos,
resumos, exploração e captura de padrões, \textbf{\emph{Μ}odelagem}
{[}\textbf{f}:D→CD{]}, testes de adequação e de associação,
\emph{simulações}, \textbf{\emph{predições}} e medidas de acurácia{]}.

\textbf{7.} Tamanho amostral em amostra aleatória simples {[}AAS, AAE
c/TPP e outros tipos de amostras probabilísticas{]}.

\textbf{8.} \emph{Probabilidade}: definição clássica,
\emph{frequentista} e axiomática. \emph{Conceitos}: População,
indivíduos, amostra representativa {[}\emph{rectius}, probabilística{]},
\emph{inferência}, experimento, \textbf{\emph{espaço amostral}}, evento,
\emph{independência}, estudo observacional, experimento aleatorizado,
\textbf{\emph{variável aleatória}}, observação, \textbf{\emph{medidas}},
\emph{contagem}.

\textbf{9.} \textbf{\emph{Medidas resumos}}, média, média ponderada,
média aparada, mediana, moda.

\textbf{10.} \textbf{\emph{Variabilidade}}, variância, desvio, desvio
padrão, amplitude interquartis {[}AIQ{]}, \emph{outliers},

\textbf{11.} \emph{Frequência} absoluta e \emph{relativa},
\textbf{\emph{probabilidade condicional}},
\textbf{\emph{verossimilhança}}, probabilidade \emph{a priori}, fator de
normalização, probabilidade \emph{a posteriori} {[}Teoria
Bayesiana{]}\emph{.}

\textbf{12.} Densidade de probabilidade, densidade acumulada,
\textbf{\emph{padronização}} (escore-Z).

\textbf{13.} \textbf{\emph{Modelos}} determinísticos e probabilísticos
(estocásticos) e \textbf{\emph{testes estatísticos}} {[}erros de
decisão{]}.

\textbf{14.} Ciência de Dados e conceito de \textbf{\emph{simulação}}
{[}\emph{conceito de Ciclo} da Ciência de Dados{]}.

\textbf{15.} Ciência de Dados, Direito e Políticas Públicas: dados
\emph{transversais} (quadro de dados), \emph{longitudinais} (séries
temporais), em \emph{painel} e outras \emph{estruturas} (grafos,
\emph{DAG}'s).

\textbf{16.} \emph{Jurimetria} e suas interfaces com as Políticas
Públicas {[}operacionalização de conceitos, \textbf{\emph{indicadores}},
\textbf{\emph{índices}} e interfaces com outros ramos da Ciência;
Ciência Forense, Epidemiologia, estudos \emph{clínicos,} análise de
sobrevivência, regressão logística, modelagem por séries temporais,
análise de fatores de risco etc.{]}.

\textbf{17.} Análises quantitativas no Direito e Políticas Públicas
{[}\emph{estudo dirigido} de 9 a 15 casos com dados primários e
secundários \emph{coletados} por pesquisas junto ao PPGDP{]}.

(cf.~Ementa na Plataforma Sucupira da Capes -- APCN n.~61/2023, p.~90;
art. 43, III, § 3º, `d', novo Reg. PPGDP -- RPPGDP, aprovado pela Resol.
CEPEC n.~1941, de 31 de março de 2025).

\bookmarksetup{startatroot}

\chapter*{Apêndice B - Quadro Var}\label{Apend-B}
\addcontentsline{toc}{chapter}{Apêndice B - Quadro Var}

\markboth{Apêndice B - Quadro Var}{Apêndice B - Quadro Var}

\section*{Quadro de Variáveis}\label{quadro-de-variuxe1veis}
\addcontentsline{toc}{section}{Quadro de Variáveis}

\markright{Quadro de Variáveis}

O script a seguir armezena uma tabela do \ul{\textbf{\emph{Quadro de
Variáveis}}} da pesquisa sobre adolescentes em conflito com a lei em
Goiânia (2016-2022), num \emph{data frame} na memória RAM do computador,
a fim de permitir uma sua vizualição mais adequada.

\begin{Shaded}
\begin{Highlighting}[numbers=left,,]
\InformationTok{\textasciigrave{}\textasciigrave{}\textasciigrave{}\{r\}}
\CommentTok{\# Armazanar tab. do Quadro de Variáveis na memória RAM}

\CommentTok{\# carregar o Quadro de Variáveis \textless{}quadrovar.csv\textgreater{}}
\CommentTok{\# Espaço: Comarca [ou Município?] de Goiânia}
\CommentTok{\# Tempo: 2016{-}2022}
\CommentTok{\# Pessoal : jovens investigados (crianças e adolescentes)}
\CommentTok{\# Material: que, posteriormente, vieram a obito (por causas diversas)}
\CommentTok{\# Planilha com todas as variáveis observadas}
\CommentTok{\# Exemplo: sexo, dom \textless{}domicílio\textgreater{} e usudrog \textless{}S, N\textgreater{};}
\CommentTok{\# A variável que foi excluída: renda \textless{}7 categorias\textgreater{}, teve sua descrição mantida}
\CommentTok{\# As variáveis receberam nomes abreviados, cuja explicação encontra{-}se na 2ª linha, ex.:}
\CommentTok{\# subst \textless{}substância\textgreater{}  [esta já existia, nome foi alterado de usudrog para subst]}
\CommentTok{\# sitdiv \textless{}situações diversas, ex: TDAH, pai preso etc.; novo nome\textgreater{}}
\CommentTok{\# obsobt \textless{}observações óbito, ex.: passou mal na cela, fogo colchão morreu asfixiado; novo nome\textgreater{}}
\CommentTok{\# etc.}
\NormalTok{quadrovar }\OtherTok{\textless{}{-}} \FunctionTok{read.csv}\NormalTok{(}\AttributeTok{file  =} \StringTok{"dat/csv/quadrovar.csv"}\NormalTok{,}
                  \AttributeTok{header =} \ConstantTok{TRUE}\NormalTok{,}
                  \AttributeTok{sep    =} \StringTok{","}\NormalTok{,}
                  \AttributeTok{quote  =} \StringTok{"}\SpecialCharTok{\textbackslash{}"}\StringTok{"}\NormalTok{,}
                  \AttributeTok{dec    =} \StringTok{"."}\NormalTok{,}
                  \AttributeTok{stringsAsFactors =} \ConstantTok{FALSE}\NormalTok{, }\CommentTok{\# para ler todas as colunas como \textless{}char\textgreater{}}
                  \AttributeTok{fill   =} \ConstantTok{TRUE}
\NormalTok{                 )}
\FunctionTok{cat}\NormalTok{(}\StringTok{"Abreviaturas dos nomes, descrição, tipo e categorias de todas as 25 variáveis coletadas:}\SpecialCharTok{\textbackslash{}n}\StringTok{"}\NormalTok{)}
\FunctionTok{names}\NormalTok{(quadrovar)}
\InformationTok{\textasciigrave{}\textasciigrave{}\textasciigrave{}}
\end{Highlighting}
\end{Shaded}

\begin{verbatim}
Abreviaturas dos nomes, descrição, tipo e categorias de todas as 25 variáveis coletadas:
 [1] "sent"     "medidase" "tipo"     "n"        "nome"     "mae"     
 [7] "nasc"     "sexo"     "cpf"      "cor"      "renda"    "dataesc1"
[13] "esc1"     "esc2"     "compfam"  "relpai"   "usudrog"  "subst"   
[19] "orgcrim"  "sitdiv"   "dataobt"  "morte"    "paf"      "circobt" 
[25] "obsobt"  
\end{verbatim}

Com o conjunto das \textbf{\emph{variáveis levantadas}} nas
\textbf{\emph{n = 449 observações}} colhidas pelo pesquisador (Queops).

\textbf{Organizar} o Quadro de Variáveis num \emph{dataframe} mais
adequado.

\begin{Shaded}
\begin{Highlighting}[numbers=left,,]
\InformationTok{\textasciigrave{}\textasciigrave{}\textasciigrave{}\{r\}}
\CommentTok{\#{-}{-}{-}Função extrair tipo da variável do quadro de variáveis{-}{-}{-}{-}{-}{-}{-}{-}{-}{-}{-}{-}{-}{-}{-}{-}{-}{-}{-}}
\NormalTok{tipo }\OtherTok{\textless{}{-}} \ControlFlowTok{function}\NormalTok{(x) \{}
  \CommentTok{\# filtro para descartar campos vazios \textless{}blanck\textgreater{} do argumento x}
\NormalTok{  x }\OtherTok{\textless{}{-}}\NormalTok{ x[x }\SpecialCharTok{!=} \StringTok{""}\NormalTok{]}
\NormalTok{  l }\OtherTok{\textless{}{-}} \FunctionTok{length}\NormalTok{(x) }\CommentTok{\# retorna o comprimento de uma coluna x}
  \CommentTok{\# Conforme o padrão observado nos metadados de quadrovar,}
  \CommentTok{\# conforme seu comprimento:}
  \CommentTok{\# 1: tipo caracter \textless{}char\textgreater{} ou string}
  \CommentTok{\# 2: tipo data no formato aaaa{-}mm{-}dd}
  \CommentTok{\# 3 ou mais: tipo categórica binomial ou multinomial (pode ser ordinal)}
  \ControlFlowTok{if}\NormalTok{ (l }\SpecialCharTok{==} \DecValTok{1}\NormalTok{) }\FunctionTok{return}\NormalTok{(}\StringTok{"caracter"}\NormalTok{)   }\CommentTok{\# retorna tipo caracter (string)}
  \ControlFlowTok{if}\NormalTok{ (l }\SpecialCharTok{==} \DecValTok{2}\NormalTok{) }\FunctionTok{return}\NormalTok{(}\StringTok{"data"}\NormalTok{)       }\CommentTok{\# retorna tipo data formato \textless{}aaaa{-}mm{-}dd\textgreater{}}
  \ControlFlowTok{if}\NormalTok{ (l }\SpecialCharTok{\textgreater{}=} \DecValTok{3}\NormalTok{) }\FunctionTok{return}\NormalTok{(}\StringTok{"categórica"}\NormalTok{) }\CommentTok{\# retorna tipo categórica}
\NormalTok{\} }\CommentTok{\#{-}{-}{-}{-}{-}{-}{-}{-}{-}{-}{-}{-}{-}{-}{-}{-}{-}{-}{-}{-}{-}{-}{-}{-}{-}{-}{-}{-}{-}{-}{-}{-}{-}{-}{-}{-}{-}{-}{-}{-}{-}{-}{-}{-}{-}{-}{-}{-}{-}{-}{-}{-}{-}{-}{-}{-}{-}{-}{-}{-}{-}{-}{-}{-}{-}{-}{-}{-}{-}{-}{-}{-}{-}{-}}

\CommentTok{\#{-}{-}{-}Função para extrair as categorias/formatos de todas variáveis}
\CommentTok{\#   Será aplicada sobre todas as colunas do data set quadrovar para:}
\CommentTok{\#   extrair da 1ª linha que se trata de variável armazenada como \textless{}char\textgreater{},}
\CommentTok{\#   extrair da 2ª linha que se trata de variável tipo data \textless{}aaaa{-}mm{-}dd\textgreater{},}
\CommentTok{\#   extrair da 3ª linha em diante de cada coluna, todas as categorias \textless{}char\textgreater{}}
\CommentTok{\#   fazer um paste0 desse vetor para reunir todas elas, sepradas por ", " e}
\CommentTok{\#   retornar esse resultado.}
\NormalTok{categ }\OtherTok{\textless{}{-}} \ControlFlowTok{function}\NormalTok{(x) \{}
  \CommentTok{\# filtro para descartar campos vazios \textless{}blanck\textgreater{} do argumento x}
\NormalTok{  x }\OtherTok{\textless{}{-}}\NormalTok{ x[x }\SpecialCharTok{!=} \StringTok{""}\NormalTok{]}
\NormalTok{  l }\OtherTok{\textless{}{-}} \FunctionTok{length}\NormalTok{(x) }\CommentTok{\# retorna o comprimento de uma coluna x}
  \CommentTok{\# retorna o formato ou o conjunto de categorias de cada coluna (variável).}
  \FunctionTok{ifelse}\NormalTok{(l }\SpecialCharTok{==} \DecValTok{1}\NormalTok{, x[}\DecValTok{1}\NormalTok{], }\FunctionTok{paste0}\NormalTok{(x[}\DecValTok{2}\SpecialCharTok{:}\NormalTok{l], }\AttributeTok{collapse =} \StringTok{", "}\NormalTok{))}
\NormalTok{\} }\CommentTok{\#{-}{-}{-}{-}{-}{-}{-}{-}{-}{-}{-}{-}{-}{-}{-}{-}{-}{-}{-}{-}{-}{-}{-}{-}{-}{-}{-}{-}{-}{-}{-}{-}{-}{-}{-}{-}{-}{-}{-}{-}{-}{-}{-}{-}{-}{-}{-}{-}{-}{-}{-}{-}{-}{-}{-}{-}{-}{-}{-}{-}{-}{-}{-}{-}{-}{-}{-}{-}{-}{-}{-}{-}{-}{-}}

\CommentTok{\# Criar um Dicionário de Dados do Quadro de Variáveis da Pesquisa}
\CommentTok{\# que ainda encontra{-}se incompleto e precisa ser aprimorado.}
\CommentTok{\# Aplicar as duas funções anteriores em todas as colunas de quadrovar}
\CommentTok{\# para extrair o Tipo e as Categorias (metadados) das Variáveis observadas.}
\NormalTok{quadrovar.df }\OtherTok{\textless{}{-}} \FunctionTok{data.frame}\NormalTok{( }\CommentTok{\# extrair os metadados de quadrovar}
  \AttributeTok{n           =} \DecValTok{1}\SpecialCharTok{:}\FunctionTok{length}\NormalTok{(quadrovar), }\CommentTok{\# número de variáveis: 25}
  \AttributeTok{Abreviatura =} \FunctionTok{names}\NormalTok{(quadrovar),    }\CommentTok{\# retorna nomes das variáveis}
  \CommentTok{\# retorna a 1ª linha de quadrovar: contém a descrição de cada Var.}
  \StringTok{"Descrição"} \OtherTok{=}\NormalTok{ quadrovar[}\DecValTok{1}\NormalTok{, ] }\SpecialCharTok{|\textgreater{}} \FunctionTok{unlist}\NormalTok{(), }\CommentTok{\# nome de variável não usual}
  \CommentTok{\# aplicar a função tipo() a cada coluna de quadrovar:}
  \CommentTok{\# para retorna o tipo (metadado) de cada uma das suas 25 variáveis}
  \AttributeTok{Tipo        =} \FunctionTok{apply}\NormalTok{(}\AttributeTok{X =}\NormalTok{ quadrovar, }\AttributeTok{MARGIN =} \DecValTok{2}\NormalTok{, }\AttributeTok{FUN =}\NormalTok{ tipo ),}
  \CommentTok{\# aplicar a função categ() a  cada coluna de quadrovar:}
  \CommentTok{\# para retornar as categorias (metadado) das variáveis categóricas}
  \CommentTok{\# ou se ela é do tipo caracter (string) ou seu formato se ela é tipo data}
  \AttributeTok{Categorias  =} \FunctionTok{apply}\NormalTok{(}\AttributeTok{X =}\NormalTok{ quadrovar, }\AttributeTok{MARGIN =} \DecValTok{2}\NormalTok{, }\AttributeTok{FUN =}\NormalTok{ categ)}
\NormalTok{)}

\CommentTok{\# Apagar os indevidos nomes das linhas do dataframe recem criado}
\CommentTok{\# que armazenou os 25 nomes de colunas de quadrovar}
\CommentTok{\# atribuir o vakor NULL ao seu atributo: rownames (nomes das linhas)}
\FunctionTok{attributes}\NormalTok{(quadrovar.df)}
\FunctionTok{rownames}\NormalTok{(quadrovar.df) }\OtherTok{\textless{}{-}} \ConstantTok{NULL}
\InformationTok{\textasciigrave{}\textasciigrave{}\textasciigrave{}}
\end{Highlighting}
\end{Shaded}

\begin{verbatim}
$names
[1] "n"           "Abreviatura" "Descrição"   "Tipo"        "Categorias" 

$class
[1] "data.frame"

$row.names
 [1] "sent"     "medidase" "tipo"     "n"        "nome"     "mae"     
 [7] "nasc"     "sexo"     "cpf"      "cor"      "renda"    "dataesc1"
[13] "esc1"     "esc2"     "compfam"  "relpai"   "usudrog"  "subst"   
[19] "orgcrim"  "sitdiv"   "dataobt"  "morte"    "paf"      "circobt" 
[25] "obsobt"  
\end{verbatim}

\subsection*{Exibição dos Metadados dos Dados
brutos}\label{exibiuxe7uxe3o-dos-metadados-dos-dados-brutos}
\addcontentsline{toc}{subsection}{Exibição dos Metadados dos Dados
brutos}

\textbf{Exibir} uma tabela com um primeiro \textbf{Dicionário de Dados}
do \textbf{Quadro de Variáveis} inicial da Pesquisa:

\begin{Shaded}
\begin{Highlighting}[numbers=left,,]
\InformationTok{\textasciigrave{}\textasciigrave{}\textasciigrave{}\{r\}}
\CommentTok{\# Carregar e anexar o pacote complementar denominado: gt {-} get table}
\CommentTok{\# na Global Environment:}
\FunctionTok{library}\NormalTok{(gt)}

\NormalTok{quadrovar.df }\SpecialCharTok{|\textgreater{}} \CommentTok{\# utilizar o comando pipe do R base}
  \FunctionTok{gt}\NormalTok{(}
    \AttributeTok{caption =} \StringTok{"Dicionário de Dados do Quadro de Variáveis inicial da Pesquisa: Jovens em conflito com a lei com passagem pela DePAI {-} Goiânia: 2016{-}2023"}
\NormalTok{  )}
\InformationTok{\textasciigrave{}\textasciigrave{}\textasciigrave{}}
\end{Highlighting}
\end{Shaded}

\begin{table}
\fontsize{12.0pt}{14.4pt}\selectfont
\begin{tabular*}{\linewidth}{@{\extracolsep{\fill}}rllll}
\toprule
n & Abreviatura & Descrição & Tipo & Categorias \\ 
\midrule\addlinespace[2.5pt]
1 & sent & sentença & categórica & remissão própria, remissão c/ advertência, remissão c/ LA, remissão c/ PSC, remissão c/  LA + PSC, LA, PSC, Internação, arquivamento infracional, arquivamento - óbito, condenação, absolvição, transação penal -  TCO, arquivamento crime, NC \\ 
2 & medidase & medida sócio educativa & categórica & MS cumpriu, MS não cumpriu, fechado, semi-aberto, aberto, transação penal -  TCO, Arquivamento - óbito, Arquivamento - outros, NC \\ 
3 & tipo & tipificação do ato infracional & categórica & adulteração de sinais identificadores, ameaça, apologia - incitação ao crime, dano, desobediência, resistência ou desacato, drogas - tráfico, drogas - uso, estelionato, estupro, falsidades (documento - moeda), furto, homicídio, injuria, difamação ou calúnia, Lei Geral do Esporte (estatuto do torcedor), lesão corporal, pornografia infantil, posse ou porte de arma de fogo, receptação, roubo/extorsão, trânsito - dirigir sem habilitação, Organização criminosa, NC \\ 
4 & n & número & caracter & número \\ 
5 & nome & nome do jovem & caracter & nome do jovem \\ 
6 & mae & nome da mãe do jovem (não foi coletado o nome do pai) & caracter & nome da mãe do jovem (não foi coletado o nome do pai) \\ 
7 & nasc & data do nascimento & data & aaaa-mm-dd \\ 
8 & sexo & sexo & categórica & fem, masc \\ 
9 & cpf & CPF do jovem (muitas vezes contém o RG da identidade civil) & caracter & CPF do jovem (muitas vezes contém o RG da identidade civil) \\ 
10 & cor & cor & categórica & branco, pardo, preto, indígina, amarelo \\ 
11 & renda & renda & categórica & até 0,5 SM, mais 0,5 - até 1 SM, mais 1 - até 2 SM, mais 2 - até 5 SM, mais 5 - até 10 SM, mais 10 - até 20 SM, mais 20 SM, NC \\ 
12 & dataesc1 & data da escolaridade n. 1 & data & aaaa-mm-dd \\ 
13 & esc1 & escolaridade n. 1 & categórica & Educação infantil, 1 ano, 2 ano, 3 ano, 4 ano, 5 ano, 6 ano, 7 ano, 8 ano, 9 ano, 1 série EM, 2 série EM, 3 série EM \\ 
14 & esc2 & escolaridade n. 2 & categórica & Educação infantil, 1 ano, 2 ano, 3 ano, 4 ano, 5 ano, 6 ano, 7 ano, 8 ano, 9 ano, 1 série EM, 2 série EM, 3 série EM \\ 
15 & compfam & composição familiar & categórica & pai + mãe, pai, mãe, parentes, pai + madrasta, mãe + padrasto \\ 
16 & relpai & relação com o pai & categórica & mesma residência, auxílio, ausente \\ 
17 & usudrog & susuário de drogra & categórica & N, S \\ 
18 & subst & uso de substâncias & categórica & lícitas, maconha, cocaína / crack, lsd, ecstasy, outras \\ 
19 & orgcrim & organização criminosa & categórica & sim, não, NC \\ 
20 & sitdiv & situações diversas & categórica & TDAH, Pais falecidos, Pai preso, pai usuário crack, morador de rua, … \\ 
21 & dataobt & data do óbito & data & aaaa-mm-dd \\ 
22 & morte & tipo de morte & categórica & violenta, natural \\ 
23 & paf & Erfuração por arma de fogo & categórica & sim, não, NC \\ 
24 & circobt & circunstâncias do óbito & categórica & conflitos entre criminalidade, intervenção policial, trânsito, conflito familiar / afetivo, Outros \\ 
25 & obsobt & observaçoes quanto ao óbito & categórica & passou mal na cela, arma branca, morreu asfixiado pela fumaça do fogo, os adolescentes colocaram fogo no colchão da cela e morreram, suicídio no case, Sentiu mal estar na cela e veio a óbito,  câncer, afogamento, suicídio, RAI 30423361 - Delegacia Piracanjuba, colocaram fogo no colchão na cela e morreram asfixiados \\ 
\bottomrule
\end{tabular*}
\end{table}

Então, somente após superadas essas fases de
\textbf{\emph{pré-processamento, checagem e organização}} dos
\textbf{\emph{metadados}} dos \textbf{\emph{dados primários}} levantados
pelo autor da pesquisa, salva-se no disco rígido esses metadados como
\textbf{\emph{dados tratados}} para formar um \textbf{Dicionário de
Dados} inicial do \textbf{Quadro de variáveis} na pasta \texttt{out}.

Que poderá ser lido de qualquer ponto de um \emph{script} qualquer deste
\texttt{EBR-a-DPP.Rproj} e encontra-se pronto para ser objeto de
análises descritivas e exploratória de dados.

\begin{Shaded}
\begin{Highlighting}[numbers=left,,]
\InformationTok{\textasciigrave{}\textasciigrave{}\textasciigrave{}\{r\}}
\CommentTok{\# salvar os metadados dos dados brutos primários: lido, tratado, checado (eventuais testes de consistência), modificados, explorados e exibidos}
\FunctionTok{saveRDS}\NormalTok{(}\AttributeTok{object =}\NormalTok{ quadrovar.df,}
        \AttributeTok{file   =} \StringTok{"out/quadrovar.df.rds"}\NormalTok{)}

\NormalTok{dic.dados }\OtherTok{\textless{}{-}} \FunctionTok{readRDS}\NormalTok{(}\AttributeTok{file =} \StringTok{"out/quadrovar.df.rds"}\NormalTok{)}

\NormalTok{dic.dados }\SpecialCharTok{|\textgreater{}} 
  \FunctionTok{gt}\NormalTok{(}
    \AttributeTok{caption =} \StringTok{"Dicionário de Dados do Quadro de Variáveis inicial da Pesquisa: Jovens em conflito com a lei com passagem pela DePAI {-} Goiânia: 2016{-}2023"}
\NormalTok{  )}
\InformationTok{\textasciigrave{}\textasciigrave{}\textasciigrave{}}
\end{Highlighting}
\end{Shaded}

\begin{table}
\fontsize{12.0pt}{14.4pt}\selectfont
\begin{tabular*}{\linewidth}{@{\extracolsep{\fill}}rllll}
\toprule
n & Abreviatura & Descrição & Tipo & Categorias \\ 
\midrule\addlinespace[2.5pt]
1 & sent & sentença & categórica & remissão própria, remissão c/ advertência, remissão c/ LA, remissão c/ PSC, remissão c/  LA + PSC, LA, PSC, Internação, arquivamento infracional, arquivamento - óbito, condenação, absolvição, transação penal -  TCO, arquivamento crime, NC \\ 
2 & medidase & medida sócio educativa & categórica & MS cumpriu, MS não cumpriu, fechado, semi-aberto, aberto, transação penal -  TCO, Arquivamento - óbito, Arquivamento - outros, NC \\ 
3 & tipo & tipificação do ato infracional & categórica & adulteração de sinais identificadores, ameaça, apologia - incitação ao crime, dano, desobediência, resistência ou desacato, drogas - tráfico, drogas - uso, estelionato, estupro, falsidades (documento - moeda), furto, homicídio, injuria, difamação ou calúnia, Lei Geral do Esporte (estatuto do torcedor), lesão corporal, pornografia infantil, posse ou porte de arma de fogo, receptação, roubo/extorsão, trânsito - dirigir sem habilitação, Organização criminosa, NC \\ 
4 & n & número & caracter & número \\ 
5 & nome & nome do jovem & caracter & nome do jovem \\ 
6 & mae & nome da mãe do jovem (não foi coletado o nome do pai) & caracter & nome da mãe do jovem (não foi coletado o nome do pai) \\ 
7 & nasc & data do nascimento & data & aaaa-mm-dd \\ 
8 & sexo & sexo & categórica & fem, masc \\ 
9 & cpf & CPF do jovem (muitas vezes contém o RG da identidade civil) & caracter & CPF do jovem (muitas vezes contém o RG da identidade civil) \\ 
10 & cor & cor & categórica & branco, pardo, preto, indígina, amarelo \\ 
11 & renda & renda & categórica & até 0,5 SM, mais 0,5 - até 1 SM, mais 1 - até 2 SM, mais 2 - até 5 SM, mais 5 - até 10 SM, mais 10 - até 20 SM, mais 20 SM, NC \\ 
12 & dataesc1 & data da escolaridade n. 1 & data & aaaa-mm-dd \\ 
13 & esc1 & escolaridade n. 1 & categórica & Educação infantil, 1 ano, 2 ano, 3 ano, 4 ano, 5 ano, 6 ano, 7 ano, 8 ano, 9 ano, 1 série EM, 2 série EM, 3 série EM \\ 
14 & esc2 & escolaridade n. 2 & categórica & Educação infantil, 1 ano, 2 ano, 3 ano, 4 ano, 5 ano, 6 ano, 7 ano, 8 ano, 9 ano, 1 série EM, 2 série EM, 3 série EM \\ 
15 & compfam & composição familiar & categórica & pai + mãe, pai, mãe, parentes, pai + madrasta, mãe + padrasto \\ 
16 & relpai & relação com o pai & categórica & mesma residência, auxílio, ausente \\ 
17 & usudrog & susuário de drogra & categórica & N, S \\ 
18 & subst & uso de substâncias & categórica & lícitas, maconha, cocaína / crack, lsd, ecstasy, outras \\ 
19 & orgcrim & organização criminosa & categórica & sim, não, NC \\ 
20 & sitdiv & situações diversas & categórica & TDAH, Pais falecidos, Pai preso, pai usuário crack, morador de rua, … \\ 
21 & dataobt & data do óbito & data & aaaa-mm-dd \\ 
22 & morte & tipo de morte & categórica & violenta, natural \\ 
23 & paf & Erfuração por arma de fogo & categórica & sim, não, NC \\ 
24 & circobt & circunstâncias do óbito & categórica & conflitos entre criminalidade, intervenção policial, trânsito, conflito familiar / afetivo, Outros \\ 
25 & obsobt & observaçoes quanto ao óbito & categórica & passou mal na cela, arma branca, morreu asfixiado pela fumaça do fogo, os adolescentes colocaram fogo no colchão da cela e morreram, suicídio no case, Sentiu mal estar na cela e veio a óbito,  câncer, afogamento, suicídio, RAI 30423361 - Delegacia Piracanjuba, colocaram fogo no colchão na cela e morreram asfixiados \\ 
\bottomrule
\end{tabular*}
\end{table}

\textbf{Exibir} as categorias da variável: \texttt{medidase} por meio de
uma \textbf{tabela} ou de um \textbf{gráfico de barras}.

\begin{Shaded}
\begin{Highlighting}[numbers=left,,]
\InformationTok{\textasciigrave{}\textasciigrave{}\textasciigrave{}\{r\}}
\FunctionTok{library}\NormalTok{(dplyr)}
\FunctionTok{library}\NormalTok{(ggplot2)}

\CommentTok{\# Por meio de uma tabela}
\NormalTok{quadrovar }\SpecialCharTok{|\textgreater{}} 
  \FunctionTok{count}\NormalTok{(medidase) }\CommentTok{\# Há 13 linhas em branco}

\CommentTok{\# Por meio de um gráfico de barras}
\CommentTok{\# filtrar para não incluir as linhas em branco,}
\CommentTok{\# mas sem alterar o dataset original.}
\NormalTok{quadrovar }\SpecialCharTok{|\textgreater{}} 
  \FunctionTok{select}\NormalTok{(medidase) }\SpecialCharTok{|\textgreater{}}
  \FunctionTok{filter}\NormalTok{(medidase }\SpecialCharTok{!=} \StringTok{""}\NormalTok{) }\SpecialCharTok{|\textgreater{}} 
  \FunctionTok{ggplot}\NormalTok{( }\FunctionTok{aes}\NormalTok{(}\AttributeTok{y =}\NormalTok{ medidase) ) }\SpecialCharTok{+}
  \FunctionTok{geom\_bar}\NormalTok{()}
\InformationTok{\textasciigrave{}\textasciigrave{}\textasciigrave{}}
\end{Highlighting}
\end{Shaded}

\begin{verbatim}
                 medidase  n
1                         13
2   Arquivamento - outros  1
3    Arquivamento - óbito  1
4              MS cumpriu  1
5          MS não cumpriu  1
6                      NC  1
7                  aberto  1
8                 fechado  1
9  medida sócio educativa  1
10            semi-aberto  1
11 transação penal -  TCO  1
\end{verbatim}

\pandocbounded{\includegraphics[keepaspectratio]{apendiceB-quadroVar_files/figure-pdf/unnamed-chunk-6-1.pdf}}

\bookmarksetup{startatroot}

\chapter*{Referências}\label{referuxeancias}
\addcontentsline{toc}{chapter}{Referências}

\markboth{Referências}{Referências}

\begin{tcolorbox}[enhanced jigsaw, arc=.35mm, opacitybacktitle=0.6, colframe=quarto-callout-tip-color-frame, titlerule=0mm, leftrule=.75mm, left=2mm, colbacktitle=quarto-callout-tip-color!10!white, breakable, toprule=.15mm, bottomtitle=1mm, opacityback=0, coltitle=black, title=\textcolor{quarto-callout-tip-color}{\faLightbulb}\hspace{0.5em}{Oh! Bendito\ldots{}}, rightrule=.15mm, bottomrule=.15mm, toptitle=1mm, colback=white]

\textbf{\ldots{} \emph{o que semeia}}

Livros à mão cheia

E manda o povo pensar!

O livro, caindo n'alma

É germe --- que faz a palma,

É chuva --- que faz o mar!

(Castro Alves,
\href{https://www.pensador.com/frase/MzcxODgw/}{\emph{Espumas
Flutuantes}}, 1870)

\end{tcolorbox}

\phantomsection\label{refs}
\begin{CSLReferences}{0}{1}
\bibitem[\citeproctext]{ref-Adeodato2014-UTRNJDO}
ADEODATO, João Mauricio. \textbf{Uma Teoria Ret{ó}rica da Norma Jurídica
e do Direito Subjetivo}. 2. ed. São Paulo: Noeses, 2014.

\bibitem[\citeproctext]{ref-Alexy2008-TDF}
ALEXY, Robert. \textbf{Teoria dos Direitos Fundamentais}. Tradução:
Virgílio Afonso da Silva. 1. ed. São Paulo: Malheiros, 2008.

\bibitem[\citeproctext]{ref-Barzellay2022-ppfmpe}
BARZELLAY, Larissa Sampaio; DAS NEVES, Cleuler Barbosa.
\textbf{Pol{ı́}tica p{ú}blica de fomento {à}s micro e pequenas empresas
pelo poder das compras p{ú}blicas no estado de Goi{á}s: controle externo
pelo {TCE/GO} (2006-2019)}. São Paulo: Editora Dial{é}tica, 2022.

\bibitem[\citeproctext]{ref-Becker2015}
BECKER, João Luiz. \textbf{Estatística Básica: transformando dados em
informação}. Porto Alegre: Bookman, 2015.

\bibitem[\citeproctext]{ref-Bobbio1995-TOJ}
BOBBIO, Norberto. \textbf{Teoria do ordenamento jur{ı́}dico}. Tradução:
Maria Celeste Cordeiro Leite dos Santos. 10. ed. Brasília: Ed. UnB,
1999.

\bibitem[\citeproctext]{ref-bolfarineBussab-2005elemAm}
BOLFARINE, Heleno; BUSSAB, Wilton de Oliveira.
\textbf{\href{https://books.google.com.br/books/about/Elementos_de_Amostragem.html?id=iA9QcAAACAAJ&redir_esc=y}{Elementos
de amostragem}}. 1. ed. São Paulo: Blucher, 2005.

\bibitem[\citeproctext]{ref-Borges2011-delimAPPrelevo}
BORGES, Raphael de Oliveira; NEVES, Cleuler Barbosa das; CASTRO, Selma
Simões de. \href{https://doi.org/10.20502/rbg.v12i0.263}{{Delimitação de
Áreas de Preservação Permanente determinadas pelo relevo: aplicação da
legislação ambiental em duas Microbacias Hidrográficas no Estado de
Goiás}}. \textbf{Revista Brasileira de Geomorfologia}, v. 12, n. 3, p.
109--114, 2011.

\bibitem[\citeproctext]{ref-livroMTC2018}
DAS NEVES, Cleuler Barbosa. \textbf{Trabalhos Científicos em Direito: da
elaboração à defesa, no campo acadêmico e profissional}. Rio de Janeiro:
Lumen Juris, "ainda no prelo".

\bibitem[\citeproctext]{ref-dissert2001}
DAS NEVES, Cleuler Barbosa. \textbf{Apropriação das àguas Doces no
Brasil: a Concessão Onerosa de Direito Real Resolúvel de Uso da
Derivação de Corpo de Água}. Dissertação de Mestrado em Direito Agrário,
PPGDA-UFG---Goiânia: Universidade Federal de Goiás, 2001.

\bibitem[\citeproctext]{ref-tese2006}
DAS NEVES, Cleuler Barbosa. \textbf{O ato administrativo na tutela
ambiental do solo rural: uma análise da erosão laminar e do uso do solo
na Bacia do Ribeirão João Leite}. Tese de Doutorado em Ciências
Ambientais, Ciamb-UFG---Goiânia: Universidade Federal de Goiás, 2006.

\bibitem[\citeproctext]{ref-livro2011}
DAS NEVES, Cleuler Barbosa. \textbf{Águas Doces no Brasil}. Rio de
Janeiro: Deescubra, 2011. p. 392

\bibitem[\citeproctext]{ref-DasNeves2022}
DAS NEVES, Cleuler Barbosa. Abordagem Policial -- PMGO (2016-2018):
sexo, idade e luz do dia num baculejo à cor da pele. \emph{In}: REED -
Rede de Estudos Empíricos em Direito; UP - Universidade Positiva, a2022.

\bibitem[\citeproctext]{ref-Neves2022-TGPP-By}
DAS NEVES, Cleuler Barbosa. Para uma Teoria Geral do Processo Penal
Byroniana -- TGPP-By. \textbf{ainda não publicado, aguardando EJUG,
edital n. 05/2021, projeto Coleção Bico de Pena}, b2022.

\bibitem[\citeproctext]{ref-neves2020glicobiologia}
DAS NEVES, Cleuler Barbosa; BEZERRA, Leonardo Naciff.
\href{https://divulgacao.editoraforum.com.br/revista-fa\#rd-text-jsc2coku}{A
Glicobiologia e a Psicologia Comportamental como Elementos Exógenos
Estimuladores da Conciliação Judicial}. \textbf{Revista Fórum
Administrativo}, v. 20, n. 229, p. 26--34, 2020.

\bibitem[\citeproctext]{ref-neves2018deverConsensuAdm}
DAS NEVES, Cleuler Barbosa; FERREIRA FILHO, Marcílio da Silva.
\href{https://www12.senado.leg.br/ril/edicoes/55/218/ril_v55_n218_p63}{Dever
de consensualidade na atuação administrativa}. \textbf{Revista de
Informação Legislativa: RIL}, v. 55, n. 218, p. 63--84, a2018.

\bibitem[\citeproctext]{ref-neves2018escolhaArbitro}
DAS NEVES, Cleuler Barbosa; FERREIRA FILHO, Marcílio da Silva.
\href{https://doi.org/10.21056/aec.v18i71.587}{Escolha do árbitro na
terminação de conflitos administrativos: limites e possibilidades da
atuação de um advogado público}. \textbf{A\&C -- Revista de Direito
Administrativo \& Constitucional}, v. 18, n. 71, p. 167--195, b2018.

\bibitem[\citeproctext]{ref-neves2021colabPremiada}
DAS NEVES, Cleuler Barbosa; FIRMINO, Adriano Godoy.
\href{https://revista.unitins.br/index.php/humanidadeseinovacao/article/download/5115/3204}{Lei
Anticrime e Colaboração Premiada: os limites da sanção premial}.
\textbf{Humanidades \& Inovação: Novas teses jurídicas}, v. 8, n. 51, p.
147--160, 2021.

\bibitem[\citeproctext]{ref-Neves2022-ecommDP-LGPD}
DAS NEVES, Cleuler Barbosa; MATOS, Gisele Gomes. E-commerce dos dados
pessoais e a LGPD: abordagem de uma lacuna à luz da Teoria do
Ordenamento Jurídico de Bobbio. \textbf{ainda não publicado, aguardando
ABDConst-A3}, c2022.

\bibitem[\citeproctext]{ref-Neves2022-varCorMorte}
DAS NEVES, Cleuler Barbosa; MATOS, Gisele Gomes. Variância da cor da
morte violenta no seio da República Federativa do Brasil: um estudo da
variabilidade do perfil da morte violenta por sexo e por cor nas 27
Unidades Federativas (1997-2019). \textbf{ainda não publicado,
aguardando Dossiê 3º milênio - B1}, a2022.

\bibitem[\citeproctext]{ref-Neves2022-avSUSP}
DAS NEVES, Cleuler Barbosa; MATOS, Gisele Gomes. Avaliação do Sistema
Único de Segurança Pública -- SUSP (PNSPDS 2018-2028): um modelo para
capturar os níveis, a tendência e a variabilidade da taxa de homicídios
em cada um dos 26 Estados e no DF (Ipea/IBGE 1998-2019 e MJSP jan.
2018-abr. 2021). \textbf{ainda não publicado, aguardando Dossiê 3º
milênio - B1}, b2022.

\bibitem[\citeproctext]{ref-neves2019controleLicObras}
DAS NEVES, Cleuler Barbosa; NAVES, Fernanda de Moura Ribeiro.
\href{https://divulgacao.editoraforum.com.br/revista-fa\#rd-text-jsc2coku}{Controle
concomitante de editais de licitação de obras como política pública de
prevenção à corrupção}. \textbf{Forum Administrativo}, v. 19, n. 220, p.
20--32, 2019.

\bibitem[\citeproctext]{ref-Neves2021-colPrincPP}
DAS NEVES, Cleuler Barbosa; ROCHA LIMA, Rafael Carvalho da.
\href{https://doi.org/10.21056/aec.v21i84.1210}{Uma hermen{ê}utica para
antinomias de princ{ı́}pios: limites para seu controle constitucional e
pol{ı́}ticas p{ú}blicas}. \textbf{A\&C - Rev. Direito Adm. Const.}, v.
21, n. 84, p. 227--252, jun. 2021.

\bibitem[\citeproctext]{ref-silva2020Fomentar-DPP}
DAS NEVES, Cleuler Barbosa; SILVA, Sérvio Túlio Teixeira.
\href{https://doi.org/10.17058/rdunisc.v3i50.14648}{Avaliação de
Políticas Públicas: uma abordagem DPP aplicada ao programa de incentivo
fiscal PRODUZIR no Estado de Goiás (2000-2017)}. \textbf{Revista do
Direito (Santa Cruz do Sul on line)}, v. 3, p. 104--123, 2020.

\bibitem[\citeproctext]{ref-neves2019custoSucoUva}
DAS NEVES, Cleuler Barbosa; TOMÁS, Aline Vieira.
\href{https://divulgacao.editoraforum.com.br/revista-fa\#rd-text-jsc2coku}{Previsão
orçamentária de custo para alimentação em sessões de conciliação do
Tribunal de Justiça de Goiás, com fundamento em pesquisa empírica}.
\textbf{Forum Administrativo}, v. 19, n. 219, p. 18--25, 2019.

\bibitem[\citeproctext]{ref-DietzKalof2015}
DIETZ, Thomaz; KALOF, LINDA. \textbf{Introdução à Estatística Social}.
Tradução: Ana Maria Lima de Farias;Tradução: Vera Regina Lima de Faria e
Flores. Rio de Janeiro: LTC, 2015.

\bibitem[\citeproctext]{ref-Donovan2019-ug}
DONOVAN, Therese M.; MICKEY, Ruth M. \textbf{Bayesian statistics for
beginners}. London, England: Oxford University Press, 2019.

\bibitem[\citeproctext]{ref-eisenberg2019}
EISENBERG, Ian W. \emph{et al.}
\href{https://doi.org/10.1038/s41467-019-10301-1}{Uncovering the
structure of self-regulation through data-driven ontology discovery}.
\textbf{Nature Communications}, v. 10, n. 1, 24 maio 2019.

\bibitem[\citeproctext]{ref-escovedo_introducao_EpCD_2024}
ESCOVEDO, Tatiana. \textbf{Introdução à {Estatística} para {Ciência} de
{Dados}: {Da} exploração dos dados à experimentação contínua com
exemplos de código em {Python} e {R}}. São Paulo, SP: Aovs Sistemas De
Informatica Ltda., 2024.

\bibitem[\citeproctext]{ref-FerrazJr1980CiDir}
FERRAZ JÚNIOR, Tercio Sampaio Ferraz. \textbf{A Ci{ê}ncia do Direito}.
2. ed. São Paulo: Atlas, 1980.

\bibitem[\citeproctext]{ref-grolemund2014hands}
GROLEMUND, Garrett.
\textbf{\href{https://rstudio-education.github.io/hopr/}{Hands-On
Programming with R: Write Your Own Functions and Simulations}}.
\emph{{[}S.l.{]}}: O'Reilly Media, 2014.

\bibitem[\citeproctext]{ref-harari2018-21licoes}
HARARI, Yuval Noah.
\textbf{\href{https://books.google.com.br/books?id=if5QDwAAQBAJ}{21
li{ç}{õ}es para o s{é}culo 21}}. Tradução: Paulo Geiger. 1. ed.
\emph{{[}S.l.{]}}: Companhia das Letras, 2018.

\bibitem[\citeproctext]{ref-Kahneman2021}
KAHNEMAN, DANIEL; SIBONY, OLIVER; SUNSTEIN, CASS R. \textbf{Ruído: uma
falha no julgamento humano}. Tradução: Cassio de Arantes Leite. 1. ed.
Rio de Janeiro: Objetiva, 2021.

\bibitem[\citeproctext]{ref-Kahneman1982}
KAHNEMAN, DANIEL; SLOVIC, PAUL; TVERSKY, AMOS. \textbf{Judgment under
incertaint: heuristics and biases}. Cambridge, England: Cambridge
University Press, 1982.

\bibitem[\citeproctext]{ref-Lock2017-EstRevPoderDados}
LOCK, Patti Frazer \emph{et al.} \textbf{Estatística: revelando o poder
dos dados}. Tradução: Ana Maria Lima de Farias;Tradução: Vera Regina
Lima de Farias e Flores. 1. ed. Rio de Janeiro: LTC, 2017.

\bibitem[\citeproctext]{ref-Moore2023}
MOORE, David S.; NOTZ, William I.; FLIGNER, Michael A.
\textbf{\href{https://www.grupogen.com.br/livro-a-estatistica-basica-e-sua-pratica-david-s-moore-william-i-notz-e-michael-a-fligner-editora-ltc-9788521638605}{Estatística
Básica e sua prática}}. 9. ed. Rio de Janeiro: LTC, 2023.

\bibitem[\citeproctext]{ref-Oltramari-dissert2024}
OLTRAMARI, Felipe. \textbf{Controle Externo da Atividade Policial como
questão de Política Pública: análise do arranjo institucional da função
persecutória-investigativa no âmbito do Ministério Público}. Relatório
Final de Pesquisa de Mestrado em Direito e Políticas Públicas,
PPGDP-UFG---Goiânia: Universidade Federal de Goiás, 2024.

\bibitem[\citeproctext]{ref-poldrack_pensamento_est_2025}
POLDRACK, Russell. \textbf{Pensamento {Estatístico}: {Analisando}
{Dados} em um {Mundo} de {Incertezas}}. Tradução: Cibelle Ravaglia. Rio
de Janeiro, RJ: Alta Books, 2025.

\bibitem[\citeproctext]{ref-Reale1994-uz}
REALE, Miguel. \textbf{Teoria Tridimensional do Direito}. 5. ed.
\emph{{[}S.l.{]}}: Saraiva, 2017.

\bibitem[\citeproctext]{ref-Silva2022}
SILVA, Sérvio Túlio Teixeira e.; DAS NEVES, Cleuler Barbosa.
\textbf{Inteligência artificial e Jurisprudência: delimitação
jurisprudencial nas decisões do TCU do conceito aberto de cláusula
restritiva ao caráter competitivo em Editais de Licitação}. 1. ed. São
Paulo: Dialética, 2022.

\bibitem[\citeproctext]{ref-silva2021tsFomentar}
SILVA, Sérvio Túlio Teixeira; DAS NEVES, Cleuler Barbosa.
\href{https://doi.org/10.19092/reed.v8i.506}{Avaliação de Políticas
Públicas: análises de quebras estruturais em séries temporais de
indicadores para aferir os resultados do programa de incentivo fiscal
Produzir no estado de Goiás (2000 - 2017)}. \textbf{Revista De Estudos
Empíricos Em Direito}, v. 8, p. 1--51, 2021.

\bibitem[\citeproctext]{ref-Aline-2020-sucouva}
TOMÁS, Aline Vieira.
\href{https://www.cnj.jus.br/ojs/revista-cnj/article/view/170/67}{Resultados
alcançados pelo projeto Adoce: acordos após ingestão de glicose
observados em conciliações judiciais (processuais) e extrajudiciais
(pré-processuais)}. \textbf{Revista Eletrônica CNJ}, v. 4, n. 2, p.
212--232, 2020.

\bibitem[\citeproctext]{ref-Tugendhat1997-le}
TUGENDHAT, Ernst. \textbf{Li{ç}{õ}es sobre {é}tica}. Tradução: Róbson
Ramos Reis \emph{et al.} 5. ed. Petrópolis, RJ: Vozes, 2003.

\bibitem[\citeproctext]{ref-Wickham2017R}
WICKHAM, Hadley; GROLEMUND, Garrett.
\textbf{\href{http://r4ds.had.co.nz/}{R for Data Science: Import, Tidy,
Transform, Visualize, and Model Data}}. 1. ed. \emph{{[}S.l.{]}}:
Paperback; O'Reilly Media, 2017.

\end{CSLReferences}




\end{document}
